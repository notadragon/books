\setminted{texcomments}

%------------------------------------------------------------------------------
\section{}
%------------------------------------------------------------------------------
\begin{frame}[fragile]
  \frametitle{Why standardize contracts?}
\begin{itemize}

\item Make the world a better place
\item Tooling support - having third party static analysis tools able to
  verify contract checks for us
\item Consistent and controllable third-party library behaviors
\item Compiler support for optimizations based on contracts, and
  potentially to AVOID optimizatinos interfering with contracts.
\end{itemize}
\end{frame}

\begin{frame}[fragile]
  \frametitle{Current Phase - MVP}

  \begin{itemize}
  \item Ability to declare \cc{[[pre]]}, \cc{[[post]]}, \cc{[[assert]]} annotations
  \item Central violation handler invoked when ``on'' and violated
  \item No other semantics
  \end{itemize}

  \begin{cppcodebox}
  int foo(int x)
    [[ pre : x >= 0 ]]
    [[ post r : r >= 0 ]]
  {
    [[ assert : x * x > x ]];
    return x;
  }
  \end{cppcodebox}

\end{frame}

\begin{frame}[fragile]
  \frametitle{Current Phase - MVP}

  \begin{cppcodebox}
    [[ assert : x > 0 ]];

    if (x > 0) {
      invoke_violation_handler();
    }
  \end{cppcodebox}

  \begin{itemize}
  \item Partial solution
  \item No ``levels'' - contracts on might break complexity guarantees, or
    those checks have to not be written
  \item No differentiating between ``new'' and ``preexiting'' contracts
  \item No extensibility
  \item No language-based ability to take more advantage of contracts
  \item Some usefulness can be embedded in a smart enough violation handler,
    but many of our use cases will not yet be met
  \end{itemize}
\end{frame}

\begin{frame}[fragile]
  \frametitle{Fallen Heroes - C++20 Contracts}

  \begin{itemize}
  \item P0542 - contracts that were merged - MVP + \cc{audit}, \cc{axiom}
  \item P1332 - introduced semantics
  \item P1429 - full semantic-supporting proposal on top of P0542
  \item P1607 - minimized, JUST semantics, accepted by EWG
  \end{itemize} \pause

  \begin{itemize}
  \item ignore - An ``OFF'' contract that has no impact on execution
  \item observe - An ``ON'' contract that lets the violation handler observe a violation.
  \item enforce - An ``ON'' contract that aborts if the violation handler returns.
  \item assume - A contract where violation is hard UB
  \end{itemize}

\end{frame}

\begin{frame}
  \frametitle{What is needed}

  \begin{itemize}
  \item SG21 has 196 use cases for contracts defined (P1995)
  \item A plan to enabled satisfying most/all of them needs to be developed
  \item Our desire to migrate \cc{BSLS_ASSERT} and \cc{BSLS_REVIEW} to language-based contracts
    should be met (it is a subset of the SG21 use cases)
  \item Contract semantics should be selectable based on a combination of locally specified
    meta-attributes about a contract and build-time specified choices made by the person assembling
    an application.
  \end{itemize}
\end{frame}

\begin{frame}[fragile]
  \frametitle{Next Phase - Levels, Roles, Labels?}

  \begin{cppcodebox}
    BSLS_ASSERT_SAFE(X);
    [[ assert audit : X ]];

    BSLS_REVIEW(X)
    [[ assert review : X ]];

    BSLS_REVIEW_SAFE(X)
    [[ assert audit review : X ]];
  \end{cppcodebox}

  \begin{itemize}
  \item We need to independentaly control the semantics of each of these
  \item \cc{review}'s logic could be captured in code
  \item lots of other things we have identified uses for labelling contracts
  \end{itemize}
\end{frame}

\begin{frame}
  \frametitle{Next Phase - Plan }

  \begin{itemize}
  \item Working with lock3 we have a prototype that supports everything in P0542+P1429+P1607
  \item That work can be leveraged with \cc{BSLS_ASSERT} and \cc{BSLS_REVIEW} today
  \item We are developing a similar proposal and prototype that subsumes P1429/P1607 and
    meets more use cases from P1995.  That will hopefully begin review rounds and prototype implementation
    internally and with SG21 members over the next few months.    
  \end{itemize}

\end{frame}

\end{document}
