% \iffalse
%% Package `float' to use with LaTeX 2e
%% Copyright (C) 1991-2000 Anselm Lingnau, all rights reserved
%%
% This program can be redistributed and/or modified under the terms
% of the LaTeX Project Public License distributed from CTAN archives
% in the directory macros/latex/base/lppl.txt; either version 1 of
% the License, or (at your option) any later version.
%
% $Id: float.dtx,v 1.8 2001/11/07 23:46:25 anselm Exp $
%
%<package>\NeedsTeXFormat{LaTeX2e}[1999/12/01]
%<package>\ProvidesPackage{float}
%<package>         [2001/11/08 v1.3d Float enhancements (AL)]
%
%<*driver>
\documentclass{ltxdoc}
\usepackage{float}
\GetFileInfo{float.sty}
\EnableCrossrefs         
%\DisableCrossrefs   % Say \DisableCrossrefs if index is ready
\RecordChanges                  % Gather update information
\CodelineIndex                  % Index code by line number
\title{An Improved Environment for Floats\thanks{This file
       has version number \fileversion.
       Part of this style option is based on the \textsf{here} option
       by David P. Carlisle (\texttt{carlisle@cs.man.ac.uk}),
       who also provided helpful criticism.}}
\author{Anselm Lingnau\\\texttt{anselm@strathspey.org}}%
\date{\filedate}
\floatstyle{ruled}
\newfloat{Program}{tbp}{lop}[section]
\floatstyle{plain}
\newfloat{Example}{t}{lox}[section]
\floatstyle{boxed}
\restylefloat{table}
\begin{document}
\maketitle
\DocInput{float.dtx}
\end{document}
%</driver>
% \fi
%
% \CheckSum{495}
%
% \changes{v1.3}{18 Jul 2001}{Changed some names to reduce namespace pollution}
% \changes{v1.2d}{29 May 99}{License changed to LPPL}
% \changes{v1.2}{20 Jun 94}{Adapted for LaTeX2e}
% \changes{v1.1d}{17 Dec 92}{Fixed documentation}
%
% \DoNotIndex{\@Mii,\@Miv,\@cons,\@currlist,\@dblarg,\@dbldeferlist}
% \DoNotIndex{\@dblfloat,\@dottedtocline,\@eha,\@ehd,\@Esphack,\@float}
% \DoNotIndex{\@floatpenalty,\@gobble,\@ifnextchar,\@ifundefined,\@ifstar}
% \DoNotIndex{\@latexerr,\@latex@error,\@minipagefalse,\@mkboth,\@namedef}
% \DoNotIndex{\@nameuse,\@parboxrestore,\@setminipage,\@setnobreak,\@spaces}
% \DoNotIndex{\@starttoc,\@tempa,\@tempboxa,\@tempdima,\@tempf,\@undefined}
% \DoNotIndex{\@warning}
% \DoNotIndex{\addcontentsline,\addtocontents,\addtocounter,\advance,\arabic}
% \DoNotIndex{\begingroup,\bfseries,\bgroup,\begin,\box,\chapter}
% \DoNotIndex{\color@endbox,\color@vbox,\columnwidth,\csname,\def,\dimen}
% \DoNotIndex{\docdate,\edef,\egroup,\else,\end,\end@dblfloat,\end@float}
% \DoNotIndex{\endcsname,\endgroup,\endinput,\expandafter,\fi}
% \DoNotIndex{\filedate,\fileversion,\global,\hbadness,\hbox,\hfil,\hrule}
% \DoNotIndex{\hsize,\ht,\if@twocolumn,\ifdim,\iffalse,\ifnum,\iftrue,\ifvoid}
% \DoNotIndex{\ifx,\ignorespaces,\intextsep,\kern,\let,\long}
% \DoNotIndex{\MakeUppercase,\moveleft,\newbox,\newcommand,\newcounter,\newif}
% \DoNotIndex{\newsavebox,\newtoks,\noexpand,\normalcolor,\normalsize}
% \DoNotIndex{\numberline,\outer@nobreak,\PackageError,\PackageWarning}
% \DoNotIndex{\par,\parindent,\parskip,\penalty,\prevdepth,\protect}
% \DoNotIndex{\providecommand,\refstepcounter,\relax,\renewcommand,\reset@font}
% \DoNotIndex{\rmfamily,\section,\setbox,\setcounter,\setlength,\space}
% \DoNotIndex{\textheight,\textwidth,\the,\typeout,\unvbox,\uppercase}
% \DoNotIndex{\vadjust,\value,\vbox,\vrule,\vskip,\vspace,\wd,\xdef}
% \DoNotIndex{\z@skip,\z@}
%
% \begin{abstract}
%   This style option improves the interface for defining floating
%   objects such as figures and tables in \LaTeX{}.
%   It adds the notion of a `float style' that governs appearance
%   of floats.
%   New kinds of floats may be defined using a |\newfloat| command
%   analogous to |\newtheorem|.
%   This style option also incorporates the functionality
%   of David Carlisle's style option \textsf{here},
%   giving floating environments a \texttt{[H]} option
%   which means `PUT IT HERE'
%   (as opposed to the standard \texttt{[h]} option which means
%   `You may put it here if you like').
% \end{abstract}
%
% \section{Introduction}
%
% Among the features of \LaTeX\ are `floating' figures and tables
% that drift from where they appear in the input text to, say, the top
% of a page. The contents and formatting of floats is pretty much up
% to the user, except that there is a |\caption| command that
% governs formatting of the caption --- it is centered if it is short,
% and formatted as a paragraph if it is longer than a single line of text.
% Sometimes other types of floating objects, e.g., algorithms or programs,
% are desirable, but they must be defined by analogy to the existing
% floats since there is no simple command for doing this. This goes
% beyond the knowledge or inclination of the average \LaTeX\ user.
%
% In this style option, I present an interface to floating objects
% that attempts to fix some of these shortcomings.
% First of all, I introduce the notion of a `float style'.
% A float style governs the appearance of a class of floats like a page style
% governs the appearance of a page (\LaTeX\ has page styles \texttt{plain},
% \texttt{empty} and \texttt{headings}, among others). This style option
% provides some exemplary float styles:
% \begin{description}
% \item[plain] This is the float style that \LaTeX\ normally applies to
%   its floats, i.e., nothing in particular. The only difference is
%   that the caption comes out \emph{below} the body of the float,
%   regardless of where it is given in the text.
% \item[plaintop] This is similar to `plain' but the caption always
%   comes out \emph{above} the body of the float.
% \item[boxed] The body of the float is printed inside a box. The caption
%   goes below that box.
% \item[ruled] This float style is patterned on the table style of
%   {\sl Concrete Mathematics}. The caption is printed at the top of the
%   float, surrounded by rules; another rule finishes off the float.
% \end{description}
%
% \begin{Program}
% \begin{verbatim}
% #include <stdio.h>
%
% int main(int argc, char **argv)
% {
%        int i;
%        for (i = 0; i < argc; ++i)
%                printf("argv[%d] = %s\n", i, argv[i]);
%        return 0;
% }
% \end{verbatim}
% \caption{The first program. This hasn't got anything to do with the package
%    but is included as an example. Note the \texttt{ruled} float style.%
%    \label{prog1.1}}
% \end{Program}
% To facilitate the definition of new floating objects, \textsf{float}
% supports the |\newfloat| command. This command is comparable to
% |\newtheorem| in that it allows the user to add a new class of floats
% at the document level. No style option hacking is necessary.
% There's also a |\listof| command that prints a listing of all the
% floats of a given type, like |\listoffigures| and |\listoftables|
% in vanilla \LaTeX.
%
% \section{The User Interface --- New Floats}
%
% \DescribeMacro{\newfloat}
% The most important command in \textsf{float} is the |\newfloat|
% command. As mentioned above, it is patterned on |\newtheorem|.
% The |\newfloat| command takes three required and one optional
% argument; it is of the form\\[2pt]
% \hspace*{\MacroIndent}|\newfloat{|\meta{type}|}{|^^A
%   \meta{placement}|}{|\meta{ext}|}[|\meta{within}{\tt]}\\[2pt]
% \meta{type} is the `type' of the new class of floats, like \texttt{program}
% or \texttt{algorithm}. After the appropriate |\newfloat|, commands
% like |\begin{program}| or |\end{algorithm*}| will be available.
% \meta{placement} gives the default placement parameters for this class
% of floats. The placement parameters are the same as in standard \LaTeX,
% i.e., \texttt{t}, \texttt{b}, \texttt{p} and \texttt{h} for `top', `bottom',
% `page' and `here', respectively.
% When \LaTeX\ writes the captions to an auxiliary file for the list
% of figures (or whatever), it'll use the job name followed by \meta{ext}
% as a file name.
% Finally, the optional argument \meta{within} determines whether floats
% of this class will be numbered within some sectional unit of the document.
% For example, if \meta{within}$=$\texttt{chapter}, the floats will be numbered
% within chapters. (In standard \LaTeX, this happens with figures and tables
% in the \textsf{report} and \textsf{book} document styles.)
% As an example, Program~\ref{prog1.1} above was created by a command
% sequence similar to that shown in the following Example.
% \begin{Example}[H]
% \begin{verse}
% |\floatstyle{ruled}|\\
% |\newfloat{Program}{tbp}{lop}[section]|\\
% \dots\ loads o' stuff \dots\\
% |\begin{Program}|\\
% |\begin{verbatim}|\\
% \dots\ program text \dots\\
% |\end{verbatim}|\\
% |\caption{|\dots\ caption \dots|}|\\
% |\end{Program}|
% \end{verse}
% \caption{This is another silly floating Example. Except that this one
%   doesn't actually float because it uses the {\tt[H]} optional parameter
%   to appear \textbf{Here}. (Gotcha.)}
% \end{Example}
%
% \DescribeMacro{\floatstyle}
% The |\floatstyle| command sets a default float style. This float style
% will be used for all the floats that are subsequently defined using
% |\newfloat|, until another |\floatstyle| command appears.
% The |\floatstyle| command takes one argument, the name of a float
% style. For instance, |\floatstyle{ruled}|. Specifying a string that
% does not name a valid float style is an error.
%
% \DescribeMacro{\floatname}
% The |\floatname| command lets you define the \emph{float name} that
% \LaTeX\ uses in the caption of a float, i.e., `Figure' for a figure
% and so on. For example, |\floatname{program}{Program}|.
% The |\newfloat| command sets the float name to its argument
% \meta{type} if no other name has been specified before.
%
% \DescribeMacro{\floatplacement}
% The |\floatplacement| command resets the default placement specifier
% of a class of floats. E.g., |\floatplacement{figure}{tp}|.
%
% \DescribeMacro{\restylefloat}
% The |\restylefloat| command is necessary to change styles for the
% standard float types \texttt{figure} and \texttt{table}. Since these aren't
% usually defined via |\newfloat|, they don't have a style associated
% with them. Thus you have to say, for example,\\[2pt]
% \hspace*{\MacroIndent}|\floatstyle{ruled}|\\
% \hspace*{\MacroIndent}|\restylefloat{table}|\\[2pt]
% to have tables come out \texttt{ruled}. The command also lets you change
% style for floats that you define via |\newfloat|, although this
% is, typographically speaking, not a good idea.
% See table~\ref{table1} for an example.
% There is a |\restylefloat*| command which will restyle an existing
% float type but will keep the new float style from taking over the
% |\caption| command. In this case the user is responsible for
% handling their own captions.
% \DeleteShortVerb{\|}
% \begin{table} \def\B#1{$\displaystyle{n\choose#1}$}
% \begin{center} \begin{tabular}{c|cccccccc}
% $n$&\B0&\B1&\B2&\B3&\B4&\B5&\B6&\B7\\ \hline
%  0 & 1\\
%  1 & 1&1\\
%  2 & 1&2&1\\
%  3 & 1&3&3&1\\
%  4 & 1&4&6&4&1\\
%  5 & 1&5&10&10&5&1\\
%  6 & 1&6&15&20&15&6&1\\
%  7 & 1&7&21&35&35&21&7&1
% \end{tabular} \end{center}
% \caption{Pascal's triangle. This is a re-styled \LaTeX\ \texttt{table}.%
%   \label{table1}}
% \end{table}
% \MakeShortVerb{\|}
%
% \DescribeMacro{\listof}
% The |\listof| command produces a list of all the floats of a given
% class. Its syntax is\\[2pt]
% \hspace*{\MacroIndent}|\listof{|\meta{type}|}{|\meta{title}^^A
%   |}|\\[2pt]
% \meta{type} is the float type given in the |\newfloat| command.
% \meta{title} is used for the title of the list as well as the headings
% if the current page style includes them. Otherwise, the |\listof|
% command is analogous to the built-in \LaTeX\ commands
% |\listoffigures| and |\listoftables|.
%
% \section{The User Interface --- \texttt{[H]} Placement Specifier}
%
% Many people find \LaTeX's float placement specifiers too restrictive.
% A Commonly Uttered Complaint (CUC) calls for a way to place a float
% exactly at the spot where it occurs in the input file, i.e., to \emph{not}
% have it float at all.
% It seems that the \texttt{[h]} specifier should do that, but in fact it
% only suggests to \LaTeX\ something along the lines of ``put the float
% here if it's OK with you''. As it turns out, \LaTeX\ hardly ever feels
% inclined to actually do that. This situation can be improved by
% judicious manipulation of float style parameters.
%
% The same effect can be achieved by changing the actual method of
% placing floats. David Carlisle's \textsf{here} option introduces a new float
% placement specifier, namely \texttt{[H]}, which, when added to a float,
% tells \LaTeX\ to ``put it HERE, period''. If there isn't enough space
% left on the page, the float is carried over to the next page together
% with whatever follows, even though there might still be room left for
% some of that.
% This style option provides the \texttt{[H]} specifier for newly defined
% classes of floats as well as the predefined \texttt{figure}s and
% \texttt{table}s, thereby superseding \textsf{here}. David suggests
% that the \textsf{here} option be withdrawn from the archives in due course.
%
% The {\tt[H]} specifier may simply be added to the float as an optional
% argument, like all the other specifiers.
% It may \emph{not} be used in conjunction with any other placement
% specifiers, so {\tt[Hhtbp]} is illegal.
% Neither may it be used as the default placement specifier for a whole
% class of floats.
% The following table is
% defined like this:
% \begin{verse}
% |\begin{table}[H]|\\
% |\begin{tabular}{cl}|\\
% |\tt t & Top of the page\\|\\
% \dots\ more stuff \dots\\
% \end{verse}
% (It seems that I have to add some extraneous chatter here just so that
% the float actually comes out right in the middle of a printed page.
% When I \LaTeX ed the documentation just now it turned out that there
% was a page break that fell exactly between the ``So now'' line and the
% float. This wouldn't Prove Anything. Bother.)
% So now we have the following float placement specifiers:
% \begin{table}[H]
% \begin{tabular}{cl}
% \tt t & Top of the page\\
% \tt b & Bottom of the page\\
% \tt p & Page of floats\\
% \tt h & Here, if possible\\
% \tt H & Here, definitely
% \end{tabular}
% ^^A\caption{Could it be that this just needs a caption?}
% \end{table}
%
% \StopEventually{\PrintIndex}
%
% \section{Implementation}
%
% \subsection{Basics}
%
%    \begin{macrocode}
%<*package>
%    \end{macrocode}
%
% In \LaTeX, floats are assigned `type numbers' that are powers of~$2$.
% Since there are only two classes of floats, their type numbers are
% hardwired into the document styles. We need to be somewhat more flexible,
% and thus we initialize a counter to hold the next type number to be
% assigned. This counter will be incremented appropriately later.
%    \begin{macrocode}
\newcounter{float@type}
\@ifundefined{c@figure}%
  {\setcounter{float@type}{1}}%
  {\setcounter{float@type}{4}}
%    \end{macrocode}
% To warm up, we'll look at some of the simpler commands first.
% \begin{macro}{\floatstyle}
% The |\floatstyle| command puts its argument into the
% |\float@style| macro as the name of the new float style.
% But if the argument doesn't denote a float style, an error message
% is output instead: Each float style \meta{style} has a corresponding command
% |\fs@|\meta{style} that contains the appropriate declarations.
% If the control sequence |\fs@|\meta{arg} (which goes with the
% argument \meta{arg} to |\floatstyle|) is undefined, i.e.,
% equals |\relax| under |\ifx|, then the float style \meta{arg}
% is unknown, and we call |\float@error{|\meta{arg}|}| for the
% error message.
%    \begin{macrocode}
\newcommand\floatstyle[1]{\@ifundefined{fs@#1}%
  {\float@error{#1}}%
  {\def\float@style{#1}}}
%    \end{macrocode}
% \end{macro}
% \begin{macro}{\float@error}
% Here's the error message. |\@eha| is the help message that says
% `Your command was ignored.'
%    \begin{macrocode}
\newcommand\float@error[1]{\PackageError{float}{%
  Unknown float style `#1'
}{\@eha}}
%    \end{macrocode}
% \end{macro}
%
% The next two commands are even simpler. \LaTeX\ says that
% |\fps@|\meta{float} contains the default placement specifier for
% the class of floats \meta{float}. |\fname@|\meta{float} expands
% to the name that appears in \meta{float} captions, e.g., `Figure'.
% (This is our own definition.)
%    \begin{macrocode}
\newcommand\floatname[2]{\@namedef{fname@#1}{#2}}
\newcommand\floatplacement[2]{\@namedef{fps@#1}{#2}}
%    \end{macrocode}
%
%    \begin{macrocode}
\newcommand\floatevery[2]{\csname @float@every@#1\endcsname={#2}}
%    \end{macrocode}
% \begin{macro}{\restylefloat}
% The |\restylefloat| command sets up everything so that subsequent
% commands like |\begin{|\meta{float}|}| use the appropriate
% float style. It defines |\fst@|\meta{float} to expand to a command
% that sets up the currently selected float style (|\fs@|\meta{style}).
% Then it defines the commands |\begin{|\meta{float}|}|,
% |\end{|\meta{float}|}|, |\begin{|\meta{float}|*}| and
% |\end{|\meta{float}|*}|. The |\restylefloat*| command is like
% |\restylefloat| but will leave the captions of that float type
% alone.
% \changes{v1.3c}{17 Aug 2001}{Remove token register leak}
% \changes{v1.3a}{15 Aug 2001}{Introduce star form of `restylefloat}
%    \begin{macrocode}
\newcommand\restylefloat{\@ifstar\float@restyle@\float@restyle}
\newcommand\float@restyle@[1]{\float@restyle{#1}%
  \expandafter\let\csname @float@c@#1\endcsname=\@caption}
\newcommand\float@restyle[1]{\expandafter\edef\csname
  fst@#1\endcsname{\expandafter\noexpand\csname
    fs@\float@style\endcsname}%
   \@namedef{#1}{\@nameuse{fst@#1}%
      \@float@setevery{#1}\@float{#1}}%
   \@namedef{#1*}{\@nameuse{fst@#1}%
      \@float@setevery{#1}\@dblfloat{#1}}%
   \expandafter\let\csname end#1\endcsname\float@end
   \expandafter\let\csname end#1*\endcsname\float@dblend
   \expandafter\let\csname @float@c@#1\endcsname=\float@caption
   \@ifundefined{@float@every@#1}{%
     \expandafter\newtoks\csname @float@every@#1\endcsname}{}%
   \@nameuse{@float@every@#1}={}}
%    \end{macrocode}
% \end{macro}
%
% \begin{macro}{\newfloat}
% Now we can explain how to define a new class of floats. Recall that
% the three required arguments to |\newfloat| are \meta{type},
% \meta{placement} and \meta{ext}, respectively. First we save the
% latter two; we also maintain a list of active \meta{ext}s so we can
% later iterate over all currently-open lists of floats.
%    \begin{macrocode}
\newtoks\float@exts
\newcommand\newfloat[3]{\@namedef{ext@#1}{#3}
  \let\float@do=\relax
  \xdef\@tempa{\noexpand\float@exts{\the\float@exts \float@do{#3}}}%
  \@tempa
  \floatplacement{#1}{#2}%
%    \end{macrocode}
% Then we figure out a default value for the `caption name' of this
% class of floats. If the |\fname@|\meta{type} isn't already defined,
% we tentatively use \meta{type} as the name. This is convenient if
% \meta{type} is, say, \texttt{Program}, since no |\floatname| command
% is necessary at all.
%    \begin{macrocode}
  \@ifundefined{fname@#1}{\floatname{#1}{#1}}{}
%    \end{macrocode}
% Then we set up the type number for \LaTeX\ in |\ftype@|\meta{type}.
% Afterwards we have to set the |float@type| to the next greater power
% of two, so that it is ready for the next |\newfloat|. Fortunately,
% we just have to double it by adding. We don't bother checking for overflow
% since it is pretty unlikely that somebody will define 25 different classes
% of floats. Finally, we call |\restylefloat| to define the style and
% commands for this class of floats.
%    \begin{macrocode}
  \expandafter\edef\csname ftype@#1\endcsname{\value{float@type}}%
  \addtocounter{float@type}{\value{float@type}}
  \restylefloat{#1}%
%    \end{macrocode}
% Now all that's left is to assemble the |\fnum@|\meta{type} macro
% that \LaTeX\ wants to use in its captions. Basically it is just
% `caption name'${}+{}$`counter value', disguised so that the command
% \emph{names} appear in the definition instead of their expansions.
%    \begin{macrocode}
  \expandafter\edef\csname fnum@#1\endcsname%
    {\expandafter\noexpand\csname fname@#1\endcsname{}
       \expandafter\noexpand\csname the#1\endcsname}
%    \end{macrocode}
% Finally, we have to take care of the optional argument, \meta{within}.
% If the optional argument is present, we pass control to |\float@newx|.
% Otherwise, we just define the counter for this class of floats.
% By default, the numbers come out |\arabic|.
%    \begin{macrocode}
  \@ifnextchar[{\float@newx{#1}}%
    {\@ifundefined{c@#1}{\newcounter{#1}\@namedef{the#1}{\arabic{#1}}}%
      {}}}
%    \end{macrocode}
% \end{macro}
%
% \begin{macro}{\float@newx}
% Here we deal with the optional argument to |\newfloat|. We have
% to create a new counter as per |\newcounter{|\meta{type}|}|
% and add that counter to the list of counters to be reset whenever
% counter \meta{within} is stepped. The standard command
% |\newcounter{|\meta{type}|}[|\meta{within}{\tt]} takes care of that.
% However, we can't define the \meta{type} counter if it's already defined.
% While this case is simply ignored when \meta{within} is not present,
% we issue a warning here since what comes out is probably not what the user
% expects.
%    \begin{macrocode}
\def\float@newx#1[#2]{\@ifundefined{c@#1}{\newcounter{#1}[#2]%
    \expandafter\edef\csname the#1\endcsname{%
      \expandafter\noexpand\csname
        the#2\endcsname.\noexpand\arabic{#1}}}%
    {\PackageWarning{float}{Can't redefine counter variable for #1.}}}
%    \end{macrocode}
% \end{macro}
% 
% \subsection{Adapting \LaTeX\ internals}
%
% We have to adapt some of \LaTeX's internal macros to our needs.
% There are several things that have to be changed around
% in order to provide the functionality of David Carlisle's \textsf{here}.
% The following is thus lifted from \textsf{here}, with changes and with
% David's permission:
%
% \begin{macro}{\@float@Hx}
% We save the original version of |\@xfloat|. (This macro is called from
% |\@float|, which we used above to define the environment commands for a
% new class of floats.)
%    \begin{macrocode}
\let\@float@Hx\@xfloat
%    \end{macrocode}
% \end{macro}
% \begin{macro}{\@xfloat}
% The new version of |\@xfloat| looks for a |[H]| argument.
% If it is present |\@float@HH| is called, otherwise the original macro
% (renamed to |\@float@Hx|) is called.
%    \begin{macrocode}
\def\@xfloat#1[{\@ifnextchar{H}{\@float@HH{#1}[}{\@float@Hx{#1}[}}
%    \end{macrocode}
% \end{macro}
% Later on we'll need a box to save a |[H]| float.
%    \begin{macrocode}
\newsavebox\float@box
\newif\if@flstyle
%    \end{macrocode}
% \begin{macro}{\@float@HH}
% First gobble the |[H]|. Note that \texttt{H} should not be used in
% conjunction with the other placement options, nor as the value of the
% default placement, as set in |\fps@|{\it type}.
%    \begin{macrocode}
\def\@float@HH#1[H]{%
%    \end{macrocode}
% Locally redefine the end of the environment.
%    \begin{macrocode}
  \expandafter\let\csname end#1\endcsname\float@endH
%    \end{macrocode}
% We don't get a |\@currbox| if we don't actually use the float mechanism.
% Therefore we fake one using the |\float@box| defined above.
%    \begin{macrocode}
  \let\@currbox\float@box
%    \end{macrocode}
% Now we save the current float class name for use in constructing the
% |\caption|. The caption box (defined below) is initialised to an empty
% box to avoid trouble with floats not having a caption. Then we start the
% box that'll hold the float itself.
% |\parindent| is set to zero for compatibility with the standard \texttt{[h]}
% option.
% \changes{v1.2c}{29 Mar 95}{Bring H floats more in line with normal floats}
%    \begin{macrocode}
  \def\@captype{#1}\setbox\@floatcapt=\vbox{}%
  \expandafter\ifx\csname fst@#1\endcsname\relax
    \@flstylefalse\else\@flstyletrue\fi
  \setbox\@currbox\color@vbox\normalcolor
    \vbox\bgroup \hsize\columnwidth \@parboxrestore
      \@floatboxreset \@setnobreak
%    \end{macrocode}
% The final |\ignorespaces| is needed to gobble any spaces or newlines
% after the {\tt[H]} tokens.
%    \begin{macrocode}
  \ignorespaces}
%    \end{macrocode}
% \end{macro}
%
%    \begin{macrocode}
\newtoks\@float@everytoks
\let\@float@boxreset=\@floatboxreset
\def\@floatboxreset{\@float@boxreset\the\@float@everytoks}
\def\@float@setevery#1{\@float@everytoks=\@nameuse{@float@every@#1}}
%    \end{macrocode}
%
% \begin{macro}{\float@makebox}
% Basically, we must arrange for `style commands' to be executed
% at certain points during the generation of the float.
% \LaTeX\ puts a float into a vertical box |\@currbox| which it takes
% off a list of empty boxes for insertions. When the |\float@makebox| macro
% is called, |\@currbox| contains the complete float, minus the caption
% --- we'll see later that we use our own |\caption| command to
% put the caption into a |\vbox| of its own. This is the only way
% we can control the position of the caption by the float style,
% regardless of where the caption appears in the float's input text itself.
% So the `style commands' are |\@fs@pre|, which is inserted at the
% very beginning of the float, |\@fs@mid|, which comes between the
% float and the caption (or the caption and the float, if captions are put
% at the top), and |\@fs@post|, which finishes off the float.
% The caption may appear at the top or at the bottom of the float,
% as defined by |\@fs@iftopcapt|. Therefore, before we hand the
% float to \LaTeX\ for positioning, it is taken apart and reassembled
% using the style commands.
% \changes{v1.3a}{15 Aug 2001}{Changed `ifvbox to `ifvoid}
%    \begin{macrocode}
\newcommand\float@makebox[1]{%
  \vbox{\hsize=#1 \@parboxrestore
    \@fs@pre\@fs@iftopcapt
      \ifvoid\@floatcapt\else\unvbox\@floatcapt\par\@fs@mid\fi
      \unvbox\@currbox
    \else\unvbox\@currbox
      \ifvoid\@floatcapt\else\par\@fs@mid\unvbox\@floatcapt\fi
    \fi\par\@fs@post\vskip\z@}}
%    \end{macrocode}
% \end{macro}
% \begin{macro}{\float@end}
% The internal macro |\end@float| appears here under the name of |\float@end|.
% The main thing which is changed is that we call |\float@makebox| to
% reconstruct the float according to the float style.
% We want to do exactly what the \LaTeX\ kernel does without copying
% actual kernel code if we can help it; therefore we finish off the
% float using the kernel |\@endfloatbox|, then replace \LaTeX's
% contents of the |\@currbox| with our own processed version, and then
% hand the thing off to \LaTeX again. Of course we have already done
% |\@endfloatbox|, which comes at the beginning of |\end@float|, ourselves;
% therefore we neutralize it before calling |\end@float|. This doesn't
% matter since we're in a group anyway (we wanted to keep the style
% commands local), so everything is undone at the end of the environment.
% \changes{v1.3d}{08 Nov 2001}{Removed an extraneous level of grouping}
% \changes{v1.3b}{17 Aug 2001}{Still another new method}
% \changes{v1.3}{18 Jul 2001}{Simplified according to H. A. Sommerfeldt}
% \changes{v1.2a}{29 Mar 95}{Fixed float end handling for LaTeX2e of Dec94}
%    \begin{macrocode}
\newcommand\float@end{\@endfloatbox
  \global\setbox\@currbox\float@makebox\columnwidth
  \let\@endfloatbox\relax\end@float}
%    \end{macrocode}
% \end{macro}
%
% \begin{macro}{\float@endH}
% The |\float@endH| command is, again, derived from \textsf{here}. It'll
% deal correctly with a non-floating float, inserting the proper amounts
% of white space above and below.
% \changes{v1.3b}{18 Aug 2001}{Move more processing to LaTeX kernel}
% \changes{v1.2c}{29 Mar 95}{Bring H floats more in line with normal floats}
%    \begin{macrocode}
\newcommand\float@endH{\@endfloatbox\vskip\intextsep
  \if@flstyle\setbox\@currbox\float@makebox\columnwidth\fi
  \box\@currbox\vskip\intextsep\relax}
%    \end{macrocode}
% \end{macro}
%
% \begin{macro}{\float@dblend}
% The |\float@dblend| command finishes up double-column floats. This
% uses the same approach as |\float@end| above. It seems to work.
% \changes{v1.3b}{17 Aug 2001}{Still another new method}
% \changes{v1.3}{18 Jul 2001}{Rewritten after H. A. Sommerfeldt}
% \changes{v1.2e}{09 Mar 2000}{Fixed bug in double-column floats}
% \changes{v1.2a}{29 Mar 95}{Fixed float end handling for LaTeX2e of Dec94}
%    \begin{macrocode}
\newcommand\float@dblend{\@endfloatbox
  \global\setbox\@currbox\float@makebox\textwidth
  \let\@endfloatbox\relax\end@dblfloat}
%    \end{macrocode}
% \end{macro}
%
% \subsection{Captions and lists of floats}
%
% Now for the caption routines.
% We use a box, |\@floatcapt|, to hold the caption while the float
% is assembled.
%    \begin{macrocode}
\newsavebox\@floatcapt
%    \end{macrocode}
%
% \begin{macro}{\caption}
% This is the only \LaTeX\ macro that this document style supersedes.
% Our |\caption| command checks whether there is a custom style
% defined for the current float (whose type can be found in |\@captype|).
% If so, the caption routines from \textsf{float} are used, otherwise
% (or when the float style has been applied via |\restylefloat*| so
% that captions aren't handled by this package)
% we call the vanilla \LaTeX\ routines.
%    \begin{macrocode}
\renewcommand\caption{%
  \ifx\@captype\@undefined
    \@latex@error{\noexpand\caption outside float}\@ehd
    \expandafter\@gobble
  \else
    \refstepcounter\@captype
    \let\@tempf\@caption
    \expandafter\ifx\csname @float@c@\@captype\endcsname\relax\else
      \expandafter\expandafter\let
        \expandafter\@tempf\csname @float@c@\@captype\endcsname\fi\fi
  \@dblarg{\@tempf\@captype}}
%    \end{macrocode}
% \end{macro}
%
% \begin{macro}{\float@caption}
% The |\float@caption| macro takes care of entering the caption into
% the appropriate listing. It also controls the typesetting of the caption
% itself, although a style-dependent macro |\@fs@capt| is called
% to handle the specifics. Note that because the caption is saved in a box
% instead of being output to the float right away, you cannot simply put
% a legend after the caption proper; it has to follow the |\caption|
% command in an optional argument.
%
% First of all, we call |\addcontentsline| to update the list of
% floats of this class. Note that |\float@caption| is |\long|
% to allow for paragraph breaks in its arguments.
%    \begin{macrocode}
\long\def\float@caption#1[#2]#3{\addcontentsline{\@nameuse{ext@#1}}{#1}%
   {\protect\numberline{\@nameuse{the#1}}{\ignorespaces #2}}
%    \end{macrocode}
% Now we collect the caption proper. The caption name and number are
% taken from |\fnum@|\meta{float}, where \meta{float} is the class
% of float we're currently handling.
%    \begin{macrocode}
  \global\setbox\@floatcapt\vbox\bgroup\@parboxrestore
    \normalsize\@fs@capt{\@nameuse{fnum@#1}}{\ignorespaces #3}%
%    \end{macrocode}
% Finally we check for the presence of the optional argument. If there is
% one, we call |\float@ccon| to pick it up; otherwise, the |\egroup|
% finishes off the box.
%    \begin{macrocode}
    \@ifnextchar[{\float@ccon}{\egroup}}
%    \end{macrocode}
% \end{macro}
%
% \begin{macro}{\float@ccon}
% The |\float@ccon| macro expands to the optional argument of a
% |\caption| command, followed by |\par\egroup|. Note that this
% precludes using |\verb| \& Co.\ in the optional argument; the
% interested reader is urged to fix this problem as an exercise.
%    \begin{macrocode}
\long\def\float@ccon[#1]{#1\par\egroup}
%    \end{macrocode}
% \end{macro}
%
% \begin{macro}{\listof}
% The |\listof| command reads the desired list of floats from the
% appropriate auxiliary file. The file is then restarted.
% First of all, we check whether the float style that's supposed to be
% listed is actually defined. If not, we output a |\float@error|.
%    \begin{macrocode}
\newcommand*{\listof}[2]{%
  \@ifundefined{ext@#1}{\float@error{#1}}{%
%    \end{macrocode}
% All's well until now. We define the |\l@|\meta{float} command
% that \LaTeX\ needs for formatting the list, and then typeset the
% appropriate list header.
%    \begin{macrocode}
    \@namedef{l@#1}{\@dottedtocline{1}{1.5em}{2.3em}}%
    \float@listhead{#2}%
%    \end{macrocode}
% Next we call |\@starttoc| with the correct file extension
% to do the actual work.
% If |\parskip| is non-zero, vertical space would be added between
% the individual list entries. To avoid this, we zero |\parskip|
% locally. This should be done after the |\float@listhead| above since
% |\parskip| also influences the spacing of headings, and the listings
% would look different from other chapters otherwise. (Suggested by
% Markus Kohm.)
% \changes{v1.3c}{17 Aug 2001}{Set parskip to zero for listing}
%    \begin{macrocode}
    \begingroup\setlength{\parskip}{\z@}%
      \@starttoc{\@nameuse{ext@#1}}%
    \endgroup}}
%    \end{macrocode}
% \end{macro}
%
% \begin{macro}{\float@listhead}
% This command generates the beginning of a list of floats.
% Currently the list appears at the chapter or the section level, depending
% on whether chapters are supported in the document class. According to
% a suggestion from Markus Kohm, this is now in a separate command so it
% can be overridden by other packages. We also use |\MakeUppercase| instead
% of |\uppercase|; when this piece of code was first written |\MakeUppercase|
% hadn't been invented yet, and for some reason this never got updated.
% \changes{v1.3}{18 Jul 2001}{Factored out list-of-float starting code}
%    \begin{macrocode}
\providecommand*{\float@listhead}[1]{%
  \@ifundefined{chapter}{\def\@tempa{\section*}}%
    {\def\@tempa{\chapter*}}%
  \@tempa{#1\@mkboth{\MakeUppercase{#1}}{\MakeUppercase{#1}}}}%
%    \end{macrocode}
% \end{macro}
%
% \begin{macro}{\float@addtolists}
% This command allows \LaTeX\ programmers to add something to all
% currently-defined lists of floats, such as some extra vertical
% space at the beginning of a new chapter in the main text
% (|\float@addtolists{\protect\addvspace{10pt}}|), without knowing
% exactly which lists of floats are currently being constructed.
% This command currently does \emph{not} operate on the |lot| and |lof|
% lists.
% \changes{v1.3}{18 Jul 2001}{Added command}
%    \begin{macrocode}
\newcommand\float@addtolists[1]{%
  \def\float@do##1{\addtocontents{##1}{#1}} \the\float@exts}
%    \end{macrocode}
% \end{macro}
%
% \section{The Float Styles}
%
% Finally, we define the standard float styles that were outlined
% in the Introduction. Every float style \meta{style} corresponds to a
% command |\fs@|\meta{style} which contains the definitions of the
% style commands, namely
% \begin{center}\begin{tabular}{rl}
% |\@fs@pre|&top of the float\\
% |\@fs@mid|&between float and caption\\
% |\@fs@post|&bottom of the float\\
% |\@fs@capt|&formatting routine for caption\\
% |\@fs@cfont|&font for caption name \& number
% \end{tabular}\end{center}
%
% \begin{macro}{\floatc@plain}
% The |\floatc@plain| macro formats a caption the way \LaTeX\ does
% it: if the caption is fairly short, it is centered, otherwise it is
% formatted as a paragraph. The only difference is that the portion
% containing the caption name and number uses the |\@fs@captfont|.
%    \begin{macrocode}
\newcommand\floatc@plain[2]{\setbox\@tempboxa\hbox{{\@fs@cfont #1:} #2}%
  \ifdim\wd\@tempboxa>\hsize {\@fs@cfont #1:} #2\par
    \else\hbox to\hsize{\hfil\box\@tempboxa\hfil}\fi}
%    \end{macrocode}
% \end{macro}
% \begin{macro}{\fs@plain}
% The \texttt{plain} float style is similar to what \LaTeX\ does of its
% own accord; the only difference is that the caption is guaranteed
% to come out at the bottom of the float.
% \changes{v1.3a}{15 Aug 2001}{Use `abovecaptionskip rather than
%    hard-coded value}
%    \begin{macrocode}
\newcommand\fs@plain{\def\@fs@cfont{\rmfamily}\let\@fs@capt\floatc@plain
  \def\@fs@pre{}\def\@fs@post{}%
  \def\@fs@mid{\vspace\abovecaptionskip\relax}%
  \let\@fs@iftopcapt\iffalse}
%    \end{macrocode}
% \end{macro}
%
% \begin{macro}{\fs@plaintop}
% The \texttt{plaintop} float style is similar to the \texttt{plain}
% float style except that the caption comes out at the top.
%   \changes{v1.3}{18 Jul 2001}{Added plaintop float style}
%   \changes{v1.3a}{15 Aug 2001}{Use `belowcaptionskip rather than
%      hard-coded value}
%    \begin{macrocode}
\newcommand\fs@plaintop{\fs@plain
  \def\@fs@mid{\vspace\belowcaptionskip\relax}%
  \let\@fs@iftopcapt\iftrue}
\let\floatc@plaintop=\floatc@plain
%    \end{macrocode}
% \end{macro}
%
% \begin{macro}{\floatc@ruled}
% The |\floatc@ruled| command is even simpler than the
% |\floatc@plain| macro. The caption is simply printed `as is'.
%    \begin{macrocode}
\newcommand\floatc@ruled[2]{{\@fs@cfont #1} #2\par}
%    \end{macrocode}
% \end{macro}
%
% \begin{macro}{\fs@ruled}
% In the \texttt{ruled} float style, the caption appears at the top of the
% float, preceded and followed by horizontal rules.
% Another rule followes the whole of the float.
%    \begin{macrocode}
\newcommand\fs@ruled{\def\@fs@cfont{\bfseries}\let\@fs@capt\floatc@ruled
  \def\@fs@pre{\hrule height.8pt depth0pt \kern2pt}%
  \def\@fs@post{\kern2pt\hrule\relax}%
  \def\@fs@mid{\kern2pt\hrule\kern2pt}%
  \let\@fs@iftopcapt\iftrue}
%    \end{macrocode}
% \end{macro}
%
% \begin{macro}{\fs@boxed}
% The \texttt{boxed} float style puts the float into a box (which is slightly
% larger than the usual |\textwidth|). The caption appears below
% the box.
% \changes{v1.2b}{29 Mar 95}{Fixed bug in boxed floatstyle}
%    \begin{macrocode}
\newcommand\fs@boxed{\def\@fs@cfont{\bfseries}\let\@fs@capt\floatc@plain
  \def\@fs@pre{\setbox\@currbox\vbox{\hbadness10000
    \moveleft3.4pt\vbox{\advance\hsize by6.8pt
      \hrule \hbox to\hsize{\vrule\kern3pt
        \vbox{\kern3pt\box\@currbox\kern3pt}\kern3pt\vrule}\hrule}}}%
  \def\@fs@mid{\kern2pt}%
  \def\@fs@post{}\let\@fs@iftopcapt\iffalse}
%    \end{macrocode}
% \end{macro}
%
% Before we finish, we set the float style to \texttt{plain}.
%    \begin{macrocode}
\floatstyle{plain}
%    \end{macrocode}
% Other float styles can be added without much ado.
% If there are many more float styles, it should be worthwhile
% to invent a scheme similar to that used in Frank Mittelbach's
% \textsf{theorem} option in order to conserve space, i.e., put the
% float styles into individual files that can be loaded on demand.
% I would like to hear from people who define interesting
% float styles.
% \PrintChanges
% \Finale
% \end{document}
\endinput
