\emph{NOTE: This white paper (i.e., this is not a proposal) is intended to motivate continued investment
in developing and maturing better memory allocators in the C++ Standard as well as to counter
misinformation about allocators, their costs and benefits, and whether they should have a
continuing role in the C++ library and language.}

\begin{abstract}
The \emph{performance benefits} of employing local memory allocators are well known and
substantial. Still, the real-world costs associated with integrating allocators
throughout a code base, including related training, tools, interface and contract
complexity, and increased potential for inadvertent misuse, cannot be ignored. A
fully \emph{allocator-aware} (AA) software infrastructure (SI) offers a convincing value
proposition despite substantial upfront costs. The \emph{collateral benefits} for clients, such
as object-based instrumentation and effective means of testing allocations, make
investing in AASI even more compelling. Yet many other unwarranted concerns —
based on hearsay or specious conjecture — remain.

In this paper, we discuss all three currently available AA software models, C++11,
BDE, and PMR (C++17)\footnote{The BDE (Bloomberg Development Environment) and PMR (polymorphic memory resource) models
are very similar (the latter being derived from the former), and we often refer to them together as the
BDE/PMR model.}— each of which provides basically the same essential
benefits but requires widely varying development and maintenance effort. We then
separate real from imagined costs, presenting some of the many collateral benefits of
AASI along the way. After all aspects are considered, we continue to advocate for the
adoption of AA software today for all libraries that potentially have performancesensitive clients and specifically for the BDE/PMR model, even as we continue to
research a language-based solution that might someday all but eliminate the costs
while amplifying the benefit.
\end{abstract}
