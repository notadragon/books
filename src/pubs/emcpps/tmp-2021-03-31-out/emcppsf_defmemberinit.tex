% 31 March 2021, cleaned up and commands updated; off to Josh for code check. 
% 31 March 2021 JMB, code cleanup


\emcppsFeature{
    short={Default Member Init},
    long={Default {\SecCode class}/{\SecCode union} Member Initializers},
    toclong={Default \lstinline!class!/\lstinline!union! Member Initializers},
}{Default-Member-Initializers}
\label{default-member-initializers}
\setcounter{table}{0}
\setcounter{footnote}{0}
\setcounter{lstlisting}{0}
%\section[Default Member Init]{Default {\SecCode class}/{\SecCode union} Member Initializers}\label{Default-Member-Initializers}
%\subsection[Default Member Initializers]{Default Member Initializers}\label{default-member-initializers}

Non\lstinline!static! class data members may specify a default initializer,
which is used for constructors that don't initialize the member
explicitly.

\subsection[Description]{Description}\label{description}

The traditional means for initializing non\lstinline!static! data members
and base class objects within a class is a constructor's \emcppsgloss{member
initializer list}:

\begin{emcppslisting}
struct B
{
    int d_i;

    B(int i) : d_i(i) { }     // Initialize (ù{\codeincomments{d\_i}}ù) with (ù{\codeincomments{i}}ù).
};

struct D : B
{
    char d_c;

    D() : B(2), d_c('3') { }  // Initialize base (ù{\codeincomments{B}}ù) with (ù{\codeincomments{2}}ù) and(ù{\codeincomments{d\_c}}ù) with (ù{\codeincomments{'3'}}ù).
};
\end{emcppslisting}
    

Starting with C++11, non\lstinline!static! data members --- except for bit
fields --- can also be initialized using a \emcppsgloss{default member
initializer}, either by using \emcppsgloss{copy initialization} or
\emcppsgloss{direct list initialization}; see
\featureref{\locationc}{bracedinit}:

\begin{emcppslisting}
struct S0
{
    int   d_i = 10;     // OK, uses copy initialization
    char  d_c = {'a'};  // OK, uses copy list initialization
    float d_f{2.f};     // OK, uses direct list initialization
};
\end{emcppslisting}
    

Note that although braced initialization is supported, \emcppsgloss{direct
initialization} with a parenthesized list is not:

\begin{emcppslisting}
struct S1
{
    char d_c('a');  // Error, invalid syntax
};
\end{emcppslisting}
    

For any member, \lstinline!m!, that has a default member initializer,
constructors that don't initialize \lstinline!m! in their \emcppsgloss{member
initializer list} will implicitly initialize \lstinline!m! by using the
default member initializer value:

\begin{emcppslisting}
#include <cassert>  // standard C (ù{\codeincomments{assert}}ù) macro

struct S2
{
    int d_i = 1;
    int d_j = 1;

    S2() { }                           // Initialize (ù{\codeincomments{d\_i}}ù) with (ù{\codeincomments{1}}ù), (ù{\codeincomments{d\_j}}ù) with (ù{\codeincomments{1}}ù).
    S2(int) : d_i(2) { }               // Initialize (ù{\codeincomments{d\_i}}ù) with (ù{\codeincomments{2}}ù), (ù{\codeincomments{d\_j}}ù) with (ù{\codeincomments{1}}ù).
    S2(int, int) : d_i(2), d_j(3) { }  // Initialize (ù{\codeincomments{d\_i}}ù) with (ù{\codeincomments{2}}ù), (ù{\codeincomments{d\_j}}ù) with (ù{\codeincomments{3}}ù).
};

void f()
{
    S2 s2;
    assert(s2.d_i == 1);
    assert(s2.d_j == 1);
}
\end{emcppslisting}
    

Previously initialized non\lstinline!static! data members can be used in
subsequent initializer expressions:

\begin{emcppslisting}[emcppsstandards={c++14}]
struct S4
{
    const char* d_s{"hello"};
    int         d_i{2};
    char        d_c{d_s[1]};  // OK, (ù{\codeincomments{d\_c}}ù) initialized to (ù{\codeincomments{'e'}}ù).

    S4(const char *s) : d_s(s) { }
};

S4 s4("goodbye");          // OK, (ù{\codeincomments{s4.d\_c}}ù) initialized to (ù{\codeincomments{'g'}}ù).
\end{emcppslisting}
    

The default member initializer, just like member function bodies,
executes in a \emcppsgloss[complete class context]{complete-class context}. This means it sees its
enclosing class as a \emcppsgloss{complete type}; therefore, the initializer
can reference the size of the enclosing type and invoke member functions
that have not yet been seen:

\begin{emcppslisting}
struct S5
{
    int d_a[4];
    int d_i = sizeof(S5) + seenBelow();  // OK
    int seenBelow();
};
\end{emcppslisting}
    

Name lookup in default member initializers will find members of the
enclosing class and its bases before looking up at namespace level:

\begin{emcppslisting}
int i = 4;

struct S6
{
    int i = 5;
    int j = i;  // refers to (ù{\codeincomments{S6::i}}ù), not (ù{\codeincomments{::i}}ù)
};

S6 s6;  // OK, (ù{\codeincomments{s6.j}}ù) initialized to (ù{\codeincomments{5}}ù).
\end{emcppslisting}
    

The \lstinline!this! pointer can also be safely used as part of a default
member initializer. As with any other uses of \lstinline!this! inside a
constructor, care must be exercised because the object referred to by
\lstinline!this! will be in a partially constructed state:

\begin{emcppslisting}
int getSomeRuntimeValue();

struct S7
{
    S7* d_selfPtr = this;                   // OK
    int d_bad = this->d_later;              // Bug, (ù{\codeincomments{d\_later}}ù) not yet initialized
    int d_later = getInitialDLaterValue();  // OK
    static int getInitialDLaterValue();
};
\end{emcppslisting}
    

Unlike variables at function or global scope and unlike static data
members, a default member initializer for a member that is an array of
unknown bound will not determine the array bound:

\begin{emcppslisting}
struct S8
{
    static int d_s[];        // OK, (ù{\codeincomments{d\_s}}ù) has unknown bounds.
    int d_a[] = {1, 2, 3};   // Error, (ù{\codeincomments{d\_a}}ù) is an array of unknown bound.
    int d_b[3] = {1, 2, 3};  // OK, bound explicitly specified
};

int a[] = {1, 2, 3};         // OK, the length of (ù{\codeincomments{a}}ù) is deduced to (ù{\codeincomments{3}}ù).
int S8::d_s[] = {4, 5, 6};   // OK, the length of (ù{\codeincomments{S8::d\_s}}ù) is deduced to (ù{\codeincomments{3}}ù).
\end{emcppslisting}
    

Non\lstinline!volatile! \lstinline!const! static data members of integral or
enumeration type and also \lstinline!constexpr! static data members may
have an in-class default member initializer that must be a constant
expression (see \featureref{\locationc}{constexprvar}):

\begin{emcppslisting}
struct MyLiteralType { int d_i; };

int nonConstexprFunction() { return 4; }

struct S9
{
    static int error = 1;
        // Error, non(ù{\codeincomments{const}}ù) static member cannot be initialized here

    static constexpr MyLiteralType mlt = {4};
        // OK, (ù{\codeincomments{constexpr}}ù) literal type initialized with a constant expression

    static const int i = 3;
        // OK, (ù{\codeincomments{const}}ù) integral type initialized with a constant expression

    static const int j = nonConstexprFunction();
        // Error, initializer is not a constant expression
};
\end{emcppslisting}
    

Static member variable declarations that use a placeholder type (such as
\lstinline!auto!) must declare the member as \lstinline!const! and have a
default member initializer:

\begin{emcppslisting}
struct S10
{
    static const auto x;     // Error, no initializer provided
    static const auto y = 6; // OK
};
\end{emcppslisting}
    

\subsubsection[Interactions with unions]{Interactions with unions}\label{interactions-with-unions}

Default member initializer can also be used with union members. However,
only one variant member of a union can have a default member
initializer, since that will determine the default initialization of the
entire union:

\begin{emcppslisting}
union U0
{
    char d_c = 'a';
    int  d_d = 1;
        // Error, only one member of (ù{\codeincomments{U0}}ù) can have a default initializer.
};
\end{emcppslisting}
    

Note that members of an anonymous union are considered to be variant
members of the parent union:

\begin{emcppslisting}
union U1
{
    union { int d_x = 1; };

    union
    {
        int d_y;
        int d_z = 1;
            // Error, only one member of (ù{\codeincomments{U1}}ù) can have a default initializer.
    };

    char d_c = 'a';
        // Error, only one member of (ù{\codeincomments{U1}}ù) can have a default initializer.
};
\end{emcppslisting}
    

As with classes, the default member initializer of a variant member will
be used if there is no explicit initializer for a variant member in the
parent union; in that case, the designated variant member is the active
member of the union. If a union is initialized with a constructor that
has a member initializer list, the default member initializer is
ignored. Similarly, in C++14 if a union is initialized with a nonempty braced
initializer list, the default member initializer is ignored ; see {bracedinit}[.xref]:

\begin{emcppslisting}[emcppsstandards={c++14}]
union U2
{
    char d_c;
    int  d_i = 10;
};

U2 x;       // initializes (ù{\codeincomments{d\_i}}ù) with (ù{\codeincomments{10}}ù)
U2 y{};     // initializes (ù{\codeincomments{d\_i}}ù) with (ù{\codeincomments{10}}ù)

U2 z{'a'};  // initializes (ù{\codeincomments{d\_c}}ù) with (ù{\codeincomments{'a'}}ù), default initializer ignored

union U3
{
   char d_c;
   int  d_i = 10;

   U3() { }                 // default member initializer used for (ù{\codeincomments{d\_i}}ù)

   U3(char c) : d_c{c} { }  // default member initializer ignored
};
\end{emcppslisting}
    

\subsection[Use Cases]{Use Cases}\label{use-cases}

\subsubsection[Concise initialization of simple \lstinline!struct!s]{Concise initialization of simple {\SubsubsecCode struct}s}\label{concise-initialization-of-simple-structs}

Default member initializers provide a concise and effective way of
initializing all the members of a simple \lstinline!struct!. Consider, for
instance, a \lstinline!struct! used to configure a thread pool:

\begin{emcppslisting}
struct ThreadPoolConfiguration
{
    int  d_numThreads         = 8;     // number of worker threads
    bool d_enableWorkStealing = true;  // enable work stealing
    int  d_taskSize           = 64;    // buffer size for an enqueued task
};
\end{emcppslisting}
    
\newpage  % to keep fn  and its code on its page so reviewers don't freak out.     
Compared to the use of a constructor, the above definition of
\lstinline!ThreadPoolConfiguration! provides sensible default values with
minimal \emcppsgloss[boilerplate code]{boilerplate}.{\cprotect\footnote{In C++20, designated
initializers can be used to tweak one or more default settings in a
configuration struct like \lstinline!ThreadPoolConfiguration! in a clear
and concise manner:

\begin{emcppshiddenlisting}[emcppsbatch=e1]
#include <cassert>  // standard C (ù{\codeincomments{assert}}ù) macro
struct ThreadPoolConfiguration {
    int d_numThreads = 8;
    bool d_enableWorkStealing = true;
    int d_taskSize = 17;
};
\end{emcppshiddenlisting}
\begin{emcppslisting}[emcppsbatch=e1,emcppsstandards={c++20},style=footcode]
void testDesignatedInitializer()
{
    ThreadPoolConfiguration tpc = {.d_taskSize = 128};
    assert(tpc.d_numThreads == 8);
    assert(tpc.d_enableWorkStealing);
    assert(tpc.d_taskSize == 128);
}
\end{emcppslisting}
      }}

\subsubsection[Ensuring initialization of a non\lstinline!static! data member]{Ensuring initialization of a non{\SubsubsecCode static} data member}\label{ensuring-initialization-of-a-nonstatic-data-member}

Non\lstinline!static! data members that do not have a default member
initializer nor appear in any constructor member initialization list are
\emcppsgloss[default initialized]{default-initialized}. For \emcppsgloss[user defined type (UDT)]{user-defined types}, default
initialization is equivalent to the default constructor being invoked.
For built-in types, default initialization results in an indeterminate
value.

As an example, consider a \lstinline!struct! tracking the number of times
an user accesses a website:

\begin{emcppslisting}[emcppsbatch=e2]
#include <string>  // (ù{\codeincomments{std::string}}ù)

struct UsageTracker
{
    std::string d_token;
    std::string d_websiteURL;
    int         d_clicks;
};
\end{emcppslisting}
    

The programmer intended \lstinline!UsageTracker! to be used as a simple
aggregate. Forgetting to explicitly initialize \lstinline!d_clicks! can
result in a defect:

\begin{emcppslisting}[emcppsbatch=e2]
#include <map>     // (ù{\codeincomments{std::map}}ù)
#include <vector>  // (ù{\codeincomments{std::vector}}ù)

std::map<std::string, std::vector<UsageTracker>> usageTrackers;
// ...

void onVisitWebsite(const std::string& username, const std::string& token)
{
    UsageTracker ut = {token, "https://emcpps.com"};
    usageTrackers[username].push_back(ut);
        // Bug, (ù{\codeincomments{ut.d\_clicks}}ù) has indeterminate value.
}
\end{emcppslisting}
    

Consistent use of default member initializers for built-in types can
avoid such defects:

\begin{emcppslisting}
#include <string>  // (ù{\codeincomments{std::string}}ù)

struct UsageTracker
{
    std::string d_token;
    std::string d_websiteURL;
    int         d_clicks = 0;  // OK, will not have an indeterminate value
};
\end{emcppslisting}
    

\subsubsection[Avoiding boilerplate repetition across multiple constructors]{Avoiding boilerplate repetition across multiple constructors}\label{avoiding-boilerplate-repetition-across-multiple-constructors}

Certain member variables of a type may be used to track the state of the
object during its lifetime, independently of the initial state of the
object. This means all constructors must set such variables to the same
value, regardless of constructor arguments. Consider a state machine
that controls execution of a background process:

\begin{emcppshiddenlisting}[emcppsbatch={e3,e4}]
#include <utility>
struct MachineProgram { /*...*/ };
MachineProgram getDefaultProgram();
MachineProgram loadProgram(const char*);
\end{emcppshiddenlisting}
\begin{emcppslisting}[emcppsbatch=e3]
class StateMachine
{
private:
    enum State { e_INIT = 1, e_RUNNING, e_DONE, e_FAIL };
    State          d_state;
    MachineProgram d_program;  // instructions to execute

public:
    StateMachine()  // Create a machine to run the default program.
    : d_state(e_INIT)
    , d_program(getDefaultProgram())
    { }

    StateMachine(const MachineProgram &program)  // Run the specified (ù{\codeincomments{program}}ù).
    : d_state(e_INIT)
    , d_program(program)
    { }

    StateMachine(MachineProgram &&program)  // Run the specified (ù{\codeincomments{program}}ù).
    : d_state(e_INIT)
    , d_program(std::move(program))
    { }

    StateMachine(const char *filename)  // read program to run from (ù{\codeincomments{filename}}ù)
    : d_state(e_INIT)
    , d_program(loadProgram(filename))
    { }
};
\end{emcppslisting}
    

For member variables such as \lstinline!d_state!, which always start at
the same value and then are updated as the object is used, using a
default initializer reduces boilerplate and reduces the risk of
accidentally initializing the value wrongly:

\begin{emcppslisting}[emcppsbatch=e4]
class StateMachine
{
    enum State { e_INIT = 1, e_RUNNING, e_DONE, e_FAIL };
    State          d_state = e_INIT;  // default member initializer
    MachineProgram d_program;         // instructions to execute

public:
    StateMachine() : d_program(getDefaultProgram()) { }
    StateMachine(const MachineProgram &program) : d_program(program) { }
    StateMachine(MachineProgram &&program) : d_program(std::move(program)) { }
    StateMachine(const char *filename) : d_program(loadProgram(filename)) { }
};
\end{emcppslisting}
    

Other members that are updated during an object's lifetime, such as
mutable members for caching expensive calculations, can also benefit
from being initialized once as opposed to providing individual
initializers in each constructor. Suppose we define a class,
\lstinline!NetSession!, that caches a resolved IP address (v4 and v6), so
it doesn't need to perform a DNS lookup every time the IP is needed. In
all constructors, the IP is not resolved yet, meaning the cached IP is
invalid. A simple convention is to set both IPs to \lstinline!0! because no
valid IP has that value:

\begin{emcppslisting}
#include <string>   // (ù{\codeincomments{std::string}}ù)
#include <cstdint>  // (ù{\codeincomments{std::uint32\_t}}ù), (ù{\codeincomments{std::uint64\_t}}ù)

class NetSession
{
    std::string d_address;                     // such as "example.com"
    mutable std::uint32_t d_ip      = 0;       // cache of resolved IP address
    mutable std::uint64_t d_ipv6[2] = {0, 0};  // cache of resolved IPv6 address

public:
    NetSession() { }
    NetSession(const std::string& address) : d_address(address) { }

    // ...
};
\end{emcppslisting}
    

\subsubsection[Making default values obvious for documentation purposes]{Making default values obvious for documentation purposes}\label{making-default-values-obvious-for-documentation-purposes}

Configuration objects that bundle together a large number of different
properties are a popular artifact in large systems. Though the values
may be loaded from, e.g., an appropriate configuration file, they often
have meaningful default values. In C++03, the default values for these
properties would typically be documented in the header file
(\lstinline!.h!) but actually effected in the implementation file
(\lstinline!.cpp!), which opens the opportunity for the documentation to go
out of sync with the implementation:

\begin{emcppslisting}
// my_config.h:

#include <string>  // (ù{\codeincomments{std::string}}ù)

struct Config
{
   std::string d_organization;  // default value "ACME"
   long long   d_maxTries;      // default value (ù{\codeincomments{1}}ù)
   double      d_costRatio;     // default value (ù{\codeincomments{13.2}}ù)

   Config();
};

// my_config.cpp:

#include <my_config.h>

Config::Config()
    : d_organization("Acme")    // Bug, doesn't match documentation
    , d_maxTries(3)             // Bug, went out of sync in maintenance
{ }                             // Bug, (ù{\codeincomments{d\_costRatio}}ù) not initialized
\end{emcppslisting}
    

Default initializers can be used in such cases to work as active
documentation:

\begin{emcppslisting}
#include <string>  // (ù{\codeincomments{std::string}}ù)

struct Config
{
   std::string d_organization = "Acme";
   long long   d_maxTries     = 3;
   double      d_costRatio    = 13.2;

   Config() = default;  // no user-provided definition needed any longer
};
\end{emcppslisting}
    

\subsection[Potential Pitfalls]{Potential Pitfalls}\label{potential-pitfalls}

\subsubsection[Loss of insulation]{Loss of insulation}\label{loss-of-insulation}

Although convenient, placing default values in a header file --- and
thus potentially also using a \lstinline!default! default constructor ---
can result in a loss of insulation that can, especially at scale, have
severe consequences. For instance, consider a hash table with a
non\lstinline!static! data member representing the growth factor:

\begin{emcppslisting}[emcppsbatch=e5]
// hashtable.h:

class HashTable
{
private:
    float d_growthFactor;
    // ...

public:
    HashTable();
    // ...
};
\end{emcppslisting}
    

Without using default member initializers, the default growth factor is
provided as part of the member initialization list of the default
constructor:

\begin{emcppslisting}[emcppsbatch=e5]
// hashtable.cpp:
#include <hashtable.h>  // `HashTable`

HashTable::HashTable() : d_growthFactor(2.0f) { }
\end{emcppslisting}
    

In the eventuality that the default growth factor is too large and
results in uncontrolled memory consumption in production, it suffices to
relink the affected applications with a new version of the
library-provided \lstinline!HashTable!, rather than recompiling them.
Depending on the compilation and deployment infrastructure of a company,
relinking can be significantly less expensive than recompiling.

If default member initializers had been used, the otherwise trivial
default constructor might be defined in the header with \lstinline!=!
\lstinline!default!, effectively removing any insulation of these values
that might allow speedy relinking in lieu of expensive recompilation
should these values need to change in a crisis.{\cprotect\footnote{For a
complete description of this real-world example, see \cite{lakos20},
  section~3.10.5, pp.~783--789.}}

\subsubsection[Inconsistent subobject initialization]{Inconsistent subobject initialization}\label{inconsistent-subobject-initialization}

An approach occasionally taken to avoid keeping shared state in global
locations is to have objects keep a handle to a \lstinline!Context! object
holding data that would otherwise be application global:

\begin{emcppslisting}[emcppsbatch={e6,e7,e8}]
struct Context
{
    bool d_isProduction;
    long d_userId;
    int  d_datacenterId;
    // ... other information about the context being run in

    static Context* defaultContext();
};
\end{emcppslisting}
    

Each type that needs \lstinline!Context! information would take an optional
argument to specify a local context; otherwise, it would use the default
context:

\begin{emcppslisting}[emcppsbatch={e6,e7,e8}]
#include <string>  // (ù{\codeincomments{std::string}}ù)

struct ContextualObject
{
// ...
     ContextualObject(const std::string &name,
                      Context *context = Context::defaultContext());
// ...
};
\end{emcppslisting}
    

When combining many objects, all of which might need to access the same
context for configuration, it becomes important to pass the context
specified at construction to each subobject:

\begin{emcppslisting}[emcppsbatch=e6]
struct CompoundObject
{
// ...
    ContextualObject d_o1;
    ContextualObject d_o2;
// ...
    CompoundObject(Context *context = Context::defaultContext())
    : d_o1("First", context)
    , d_o2("Second", context)
    { }
// ...
};
\end{emcppslisting}
    

This might seem like a situation well suited for using default member
initializers, but the naive approach would have a serious flaw:

\begin{emcppslisting}[emcppsbatch=e7]
struct CompoundObject
{
// ...
    ContextualObject d_o1{"first"};   // Bug, does not use (ù{\codeincomments{context}}ù) passed to
    ContextualObject d_o2{"second"};  // (ù{\codeincomments{CompoundObject}}ù) constructor
// ...
    CompoundObject(Context *context = Context::defaultContext());
// ...
};
\end{emcppslisting}
    

This bug, frustratingly, will impact only those applications that use
multiple contexts, which might be a small subset of applications and
libraries that use contextual objects. The only nonintrusive approach
that avoids this bug is to forgo default member initializers for
subobjects that can take a \lstinline!context! parameter. The other, still
error-prone alternative is to store a context pointer as the
\emph{first} member, initialize it in all constructors, and use that in
the default member initializer of the subobjects:

\begin{emcppslisting}[emcppsbatch=e8]
struct CompoundObject
{
// ...
    Context          *d_context;
    ContextualObject  d_o1{"first", d_context};   // OK
    ContextualObject  d_o2{"second", d_context};  // OK
// ...
    CompoundObject(Context *context = Context::defaultContext())
    : d_context(context)
    { }
// ...
};
\end{emcppslisting}
    

This has the downside of requiring an extra copy of the
\lstinline!Context*! member to transmit the passed-in value from the
constructor to the subobject initializers. Additionally, now the
subobjects of \lstinline!CompoundObject! strongly depend on being
initialized after the \lstinline!d_context! member variable, so the order
of members is subtly constrained. Consequently, innocuous changes in
member order during maintenance are liable to introduce
bugs.{\cprotect\footnote{C++17 introduces the \emcppsgloss{polymorphic memory
resource} allocator model, which is a \emcppsgloss{scoped allocator model}
  with issues similar to \lstinline!ContextualObject!.}}

\subsection[Annoyances]{Annoyances}\label{annoyances}

\subsubsection[Parenthesized direct-initialization syntax cannot be used]{Parenthesized direct-initialization syntax cannot be used}\label{parenthesized-direct-initialization-syntax-cannot-be-used}

Default member initializers make member variables tantalizingly close to
allowing all of the same syntax usable for local and global variables.
However, the direct initialization syntax is not allowed in default
member initialization, which makes it tedious to copy code from
automatic variables into class members. For example, suppose we set out
to transform a function into an equivalent \emcppsgloss{function object}
(useful for applications needing callbacks). This transformation entails
migrating a function's automatic variables to member variables of the
corresponding function object:

\begin{emcppslisting}
void function()  // before
{
    int i1 = 17;
    int i2(18);
    int i3{42};

    // ... do stuff
    // ... do more stuff
}

class Functor  // after
{
    int i1 = 17;  // OK
    int i2(18);   // Error, invalid syntax
    int i3{42};   // OK

    void operator()(int step)
    {
       switch (step)
       {
       case 0:  // ... do stuff
           break;
       case 1:  // ... do more stuff
           break;
           // ...
       }
    }
};
\end{emcppslisting}
    

Cutting and pasting the definitions of the locals \lstinline!i1!,
\lstinline!i2!, and \lstinline!i3! will result in compile-time errors. The
code initializing \lstinline!i2! needs to be fixed to use either copy
initialization (like \lstinline!i1!) or braced initialization (like
\lstinline!i3!).

\subsubsection[Limitations of applicability]{Limitations of applicability}\label{limitations-of-applicability}

Default member initializers can be used to replace the initialization of
almost all members that can be placed in a constructor's \emcppsgloss{member
initializer list}, except for two cases:

\begin{itemize}
\item{Bit-field data members\cprotect\footnote{In C++20, bit fields can now be initialized through a default member initializer:
\begin{emcppslisting}[style=footcode]
struct S
{
    int d_i : 4 = 8;  // Error, before C++20; OK, in C++20
    int d_j : 4 {8};  // Error, before C++20; OK, in C++20
};
\end{emcppslisting}
    }}
\item{Base classes and \lstinline!virtual! base classes}
\end{itemize}

Both of these situations continue to require initialization in member
initializer lists, which translates to boilerplate that may be
inconsistent with the rest of the codebase.

\subsubsection[Loss of triviality]{Loss of triviality}\label{loss-of-triviality}

Having member initializers makes the default constructor nontrivial,
which in turn makes the class not a \emcppsgloss[POD]{POD (plain old data)} type.
The presence of a nontrivial constructor can prevent some optimizations
that might otherwise be possible; see \featureref{\locationc}{gpods}:

\begin{emcppslisting}
#include <type_traits>  // (ù{\codeincomments{std::is\_trivial}}ù)

struct S0 { int d_i;     };
struct S1 { int d_i = 0; };
struct S2 { int d_i; S2() : d_i(0) { } };

static_assert(std::is_trivial<S0>::value, "");
static_assert(!std::is_trivial<S1>::value, "");
static_assert(!std::is_trivial<S2>::value, "");
\end{emcppslisting}
    

\subsubsection[Loss of aggregate status]{Loss of aggregate status}\label{loss-of-aggregate-status}

In C++11, classes using default member initializers are not considered
\emcppsgloss[aggregate]{aggregates}, and that \emcppsgloss{aggregate initialization} can't be
used is probably the biggest annoyance. Fortunately, the restriction has
been lifted in C++14; see \featureref{\locationb}{aggregate-member-initialization-relaxation}:

\begin{emcppslisting}
struct ThreadPoolConfiguration
{
    int  d_numThreads         = 8;     // number of worker threads
    bool d_enableWorkStealing = true;  // enable work stealing
    int  d_taskSize           = 64;    // buffer size for an enqueued task
};

void f()
{
    ThreadPoolConfiguration tpc0;               // OK in C++11
    ThreadPoolConfiguration tpc2{16, true, 64}; // Error, in C++11; OK in C++14
}
\end{emcppslisting}
    

\subsection[See Also]{See Also}\label{see-also}

\begin{itemize}
\item{\seealsoref{delegating-constructors}{\seealsolocationa}can reduce code repetition in initialization of non\lstinline!static! data members by chaining constructors together.}
\item{\seealsoref{aggregate-member-initialization-relaxation}{\seealsolocationb}can be applied to classes with and will use the initializers from default member initializers.}
\item{\seealsoref{bracedinit}{\seealsolocationc}default member initializers support \emcppsgloss{list initialization}, which is part of the uniform initialization effort.}
\item{\seealsoref{enumopaque}{\seealsolocationc}provides another means of insulating clients from unnecessary implementation details.}
\end{itemize}

\subsection[Further Reading]{Further Reading}

TBD
