%%% 15 Jan: copyedits in and proofed 


The \lstinline![[noreturn]]! attribute promises that the function to which
it pertains never \nobreak{returns}.

\subsection[Description]{Description}\label{description}

The presence of the standard \lstinline![[noreturn]]! attribute as part of
a function declaration informs both the compiler and human readers that
such a function never returns control flow to the caller:

%\begin{lstlisting}[language=C++]
%[[noreturn]] void f()
%{
%    throw;
%}
%\end{lstlisting}
\begin{lstlisting}[language=C++]
[[noreturn]] void f()
{
    throw 1;
}
\end{lstlisting}

    
\noindent The \lstinline![[noreturn]]! attribute is not part of a function's type
and is also, therefore, not part of the type of a function pointer. Applying \lstinline![[noreturn]]! to a function
pointer is not an error, though doing so has no actual effect in standard C++; see 
%{\it\titleref{noreturn-potential-pitfalls}: \titleref{misuse-of-[[noreturn]]-on-function-pointers}} on page~\pageref{misuse-of-[[noreturn]]-on-function-pointers}
\intraref{noreturn-potential-pitfalls}{misuse-of-[[noreturn]]-on-function-pointers}. Using it on a pointer might have
benefits for external tooling, code expressiveness, and future language
evolution:

\begin{lstlisting}[language=C++]
void (*fp [[noreturn]])() = f;
\end{lstlisting}
    

\subsection[Use Cases]{Use Cases}\label{use-cases}

\subsubsection[Better compiler diagnostics]{Better compiler diagnostics}\label{better-compiler-diagnostics}

Consider the task of creating an assertion handler that, when invoked,
always aborts execution of the program after printing some useful
information about the source of the assertion. Since this specific
handler will never return because it unconditionally invokes a \linebreak[4]%%%%%
\mbox{\lstinline![[noreturn]]std::abort!} function, it is a viable candidate for
\lstinline![[noreturn]]!:

\begin{lstlisting}[language=C++]
[[noreturn]] void abortingAssertionHandler(const char* filename, int line)
{
    LOG_ERROR << "Assertion fired at " << filename << ':' << line;
    std::abort();
}
\end{lstlisting}
    

\noindent The additional information provided by the attribute will allow a
compiler to warn if it determines that a code path in the
function would allow it to return normally:

%\begin{lstlisting}[language=C++]
%[[noreturn]] void abortingAssertionHandler(const char* filename, int line)
%{
%    if (filename)  // just being safe, but see "Further Reading," below
%    {
%        LOG_ERROR << "Assertion fired at " << filename << ':' << line;
%        std::abort();
%    }
%}  // compile-time warning made possible
%\end{lstlisting}
\begin{lstlisting}[language=C++]
[[noreturn]] void abortingAssertionHandler(const char* filename, int line)
{
    if (filename)  
    {
        LOG_ERROR << "Assertion fired at " << filename << ':' << line;
        std::abort();
    }
}  // compile-time warning made possible
\end{lstlisting}

    

\noindent This information can also be used to warn in case unreachable code is
present after\linebreak[4] \lstinline!abortingAssertionHandler! is invoked:

\begin{lstlisting}[language=C++]
int main()
{
    // ...
    abortingAssertionHandler("main.cpp", __LINE__);
    std::cout << "We got here.\n";  // compile-time warning made possible
    // ...
}
\end{lstlisting}
    

\noindent Note that this warning is made possible by decorating just the
declaration of the handler function --- i.e., even if the definition of
the function is not visible in the current translation unit.

\subsubsection[Improved runtime performance]{Improved runtime performance}\label{improved-runtime-performance}

If the compiler knows that it is going to invoke a function
that is guaranteed not to return, the compiler is within its rights to optimize
that function by removing what it can now determine to be dead code. As
an example, consider a utility component, \lstinline!util!, that defines a
function, \lstinline!throwBadAlloc!, that is used to \romeogloss{insulate} the
throwing of an \lstinline!std::bad\_alloc! exception in what would
otherwise be template code fully exposed to clients:

%\begin{lstlisting}[language=C++]
%// util.h:
%[[noreturn]] void throwBadAlloc();
%
%// util.cpp:
%#include <util.h>  // (ù{\codeincomments{[[noreturn]] void throwBadAlloc()}}ù)
%
%void throwBadAlloc()  // This redeclaration is also (ù{\codeincomments{[[noreturn]]}}ù).
%{
%    throw std::bad_alloc();
%}
%\end{lstlisting}
\begin{lstlisting}[language=C++]
// util.h:
[[noreturn]] void throwBadAlloc();

// util.cpp:
#include <util.h>  // (ù{\codeincomments{[[noreturn]] void throwBadAlloc()}}ù)
                                                                                
#include <new>    // (ù{\codeincomments{std::bad\_alloc}}ù)

void throwBadAlloc()  // This redeclaration is also (ù{\codeincomments{[[noreturn]]}}ù).
{
    throw std::bad_alloc();
}
\end{lstlisting}
    

\noindent The compiler is within its
rights to elide code that is rendered unreachable by the call to the
\lstinline!throwBadAlloc! function due to the function being decorated with the
\lstinline![[noreturn]]! attribute on its declaration:

\begin{lstlisting}[language=C++]
#include <util.h>  // [[noreturn]] void throwBadAlloc()

void client()
{
    // ...
    throwBadAlloc();
    // ... (Everything below this line can be optimized away.)
}
\end{lstlisting}
    

\noindent Notice that even though \lstinline![[noreturn]]! appeared only on the first
declaration --- that in the \lstinline!util.h! header --- the \lstinline![[noreturn]]!
attribute carries over to the redeclaration used in the\linebreak[4] \lstinline!throwBadAlloc!
function's definition because the header was included in the
corresponding \lstinline!.cpp! file.

\subsection[Potential Pitfalls]{Potential Pitfalls}\label{noreturn-potential-pitfalls}

\subsubsection[\tt{[[noreturn]]} can inadvertently break an otherwise working program]{{\SubsubsecCode [[noreturn]]} can inadvertently break an otherwise working program}\label{[[noreturn]]-can-inadvertently-break-an-otherwise-working-program}

Unlike many attributes, using \lstinline![[noreturn]]! \emph{can} alter
the semantics of a well-formed program, potentially introducing a
runtime defect and/or making the program ill-formed. If a function that
can potentially return is decorated with \lstinline![[noreturn]]! and then,
in the course of executing a program, it ever does return, that behavior
is \romeogloss{undefined}.

Consider a \lstinline!printAndExit! function whose role is to print a fatal
error message before aborting the program:

\begin{lstlisting}[language=C++]
[[noreturn]] void printAndExit()
{
    std::cout << "Fatal error. Exiting the program.\n";
    assert(false);
}
\end{lstlisting}
    
\noindent The programmer chose to (sloppily) implement termination by using an
assertion, which would not be incorporated into a program compiled with
the preprocessor definition \lstinline!NDEBUG! active, and thus
\lstinline!printAndExit! would return normally in such a build mode. If the
compiler of the client is informed that function will not return, the compiler is
free to optimize accordingly. If the function then does return, any
number of hard-to-diagnose defects (e.g., due to incorrectly elided
code) might materialize as a consequence of the ensuing
\romeogloss{undefined behavior}. Furthermore, if a function is declared \lstinline![[noreturn]]! in some
translation units within a program but not in others, that program is 
\romeogloss{ill-formed, no diagnostic required (IFNDR)}.

\subsubsection[Misuse of {\tt[[noreturn]]} on function pointers]{Misuse of {\SubsubsecCode [[noreturn]]} on function pointers}\label{misuse-of-[[noreturn]]-on-function-pointers}

Although the \lstinline![[noreturn]]! attribute is permitted to syntactically appertain
to a function pointer for the benefit of external tools,
it has no effect in standard C++; fortunately, most compilers will issue a warning:

%\begin{lstlisting}[language=C++]
%void (*fp [[noreturn]])();  // not supported by standard C++ (will likely warn)
%\end{lstlisting}
\begin{lstlisting}[language=C++]
void (*fp [[noreturn]])();  // no effect in standard C++ (will likely warn)
\end{lstlisting}
    
\noindent What's more, assigning the address of a function
that is not decorated with \lstinline![[noreturn]]! to an otherwise
suitable function pointer that is so decorated is perfectly fine:

\begin{lstlisting}[language=C++]
void f() { return; };  // function that always returns

void g()
{
    fp = f;  // (ù{\codeincomments{[[noreturn]]}}ù) on (ù{\codeincomments{fp}}ù) is silently ignored.
}
\end{lstlisting}  
% code line changed from
% fp = f;  // "OK" -- that (ù{\codeincomments{fp}}ù) is (ù{\codeincomments{[[noreturn]]}}ù) is silently ignored
% to
% fp = f;  // (ù{\codeincomments{[[noreturn]]}}ù) on (ù{\codeincomments{fp}}ù) is silently ignored.
% per JD Garcia and Slava
    
\noindent Any reliance on \lstinline![[noreturn]]! to have any effect in standard C++
when applied to other than a function's declaration is misguided.

\subsection[Annoyances]{Annoyances}\label{annoyances}

\hspace*{\fill}

% per email, Nina would omit. 
%The contradictory declaration:
%\begin{lstlisting}[language=C++]
%[[noreturn]] int f();
%\end{lstlisting}
%compiles without error or even a warning.

\subsection[See Also]{See Also}\label{see-also}

\begin{itemize}
\item{%``\titleref{attributes}" on page~\pageref{attributes} — 
\seealsoref{attributes}{\locationa}\lstinline![[noreturn]]! is a built-in attribute that follows the general syntax and placement rules of C++ attributes.}
\end{itemize}

\subsection[Further Reading]{Further Reading}\label{further-reading}

\begin{itemize}
\item{\cite{svoboda10}}
\item{\cite{sutter12}}
\end{itemize}




