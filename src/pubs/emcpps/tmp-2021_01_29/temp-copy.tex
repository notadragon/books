% Options for packages loaded elsewhere
\PassOptionsToPackage{unicode}{hyperref}
\PassOptionsToPackage{hyphens}{url}
%

% \documentclass[
% % % % % % ]{report}

% PEARSON
\documentclass[twoside,10pt,letterpaper,usenames]{newstyle-PearsonGeneric-7-38}
\usepackage[twoside]{geometry} % to set the dimensions of the page

% PEARSON
\geometry{                                  % setting all the necessary dimensions...
paperwidth=7.375in,                  % Do not change these settings.
paperheight=9.125in,                 % These set up the trim size for the book.
lmargin=1in,
rmargin=.875in,
bmargin=.650in,
tmargin=.95in,
width=5.5in,
height=7.4in,
marginparwidth=0in,
marginparsep=0in,
headheight=0.2in,
headsep=.25in,
footskip=.025in}

% PEARSON
%%%%%%%% Skip over this part.  These lines should remain unaltered.
%
%%    These lines input packages that are used for this template.
\usepackage[letter,cam,center]{crop-AlteredCamMarks}  % to add trim marks; updated Nov 2015 to allow more space between marks and trim edge
\usepackage{graphicx} % for graphics layout

%%%% CHANGED FOR ROMEO
% Changing code font to Cousine
\usepackage{mathspec}
\setmonofont[Ligatures=NoCommon,Scale=.9,SlantedFont={Cousine Italic},BoldSlantedFont={Cousine Bold Italic}]{Cousine Regular} % to set our code font

\usepackage{amsfonts} % to ensure attractive mathematics
\usepackage{amssymb} % to ensure attractive mathematics
\usepackage{float}      % to allow things to float even if they normally don't
\usepackage{wrapfig} % to allow text to wrap around small figures
\usepackage{outline} % to allow outline creation
\usepackage{framed-PearsonGeneric} % to create the leftside vertical bar for the example environment

\usepackage{listings}  % for setting attractive computer code
%%%%% CHANGED FOR ROMEO
% changed font to Cousine, removed 5ex indent
\lstset{basicstyle=\ttfamily\small,framerule=0.5pt} % setting the code in small sans serif font and
% setting	the margin for the code.  Nov 2015: setting the rules to be 0.5pt
%%%%%%%%%%%%%
\usepackage[stable,bottom]{footmisc} % to stabilize the footnote environment and to ensure footnotes
							  % appear at the bottom of the page, even if a figure or table
							  % appears at the bottom
\usepackage[flushleft]{threeparttable} % to set attractive footnotes within table environments
\usepackage{titleref} % to allow cross-referencing by title
\usepackage{xr} % to allow cross-referencing between files
% \usepackage{colortbl} % to allow the use of color/shading in tables (TODO)
\usepackage{caption} % to set the appearance of captions
\captionsetup[table]{singlelinecheck=off,labelfont={sf,bf,small},font={sf,bf,small}}
\captionsetup[figure]{labelfont={sf,bf,small},font={sf,bf,small}}
\captionsetup[lstlisting]{labelfont={sf,bf,small},font={sf,bf,small}}
\usepackage{tocloft-hacked-PearsonGeneric} % to allow manipulation of the TOC

%%%%%%%% Determine your indexing plan.
%
% Most of the time, the publisher will hire an independent indexing expert to create an
%    index for your book.  Your production editor can let you know the plan for your book.
% Use the line below only if your production plan includes the creation of a LaTeX index.
% Note that this situation is uncommon.
%\makeindex           % if your book will include a LaTeX-generated index
%\usepackage{See-makeidx}

%%%%%%%%%% These lines set up the Preamble of the document. You can skip over this part.
%% Preamble:
% New commands and/or command redefinitions
%
% You can also place such commands in a macro file (e.g. mydefs.tex)
% and load them in the preamble with "input mydefs"
% Here are some examples:

\newcommand{\be}{\begin{equation}}           %---- begin numbered equation
\newcommand{\ee}{\end{equation}}             %---- end numbered equation
\newcommand{\dst}{\everymath{\displaystyle}} %---- use displaystle in eqs.
\newcommand{\Nuclide}[2]{${}^{#2}$#1}        %---- Nuclide macro
\renewcommand{\SS}[1]{${}^{#1}$}             %---- superscript in text
\newcommand{\STRUT}{\rule{0in}{3ex}}         %--- small strut
\newcommand{\dotprod}{{\scriptscriptstyle \stackrel{\bullet}{{}}}}
\renewcommand{\tilde}{\widetilde}
\renewcommand{\hat}{\widehat}
\renewcommand{\theequation}{\arabic{chapter}.\arabic{equation}}
\renewcommand{\thefigure}{\arabic{chapter}--\arabic{figure}}
\renewcommand{\thetable}{\arabic{chapter}--\arabic{table}}
\newcommand{\ul}{\underline}
\newcommand{\dul}{\underline{\underline}}
\newcommand{\ol}{\overline}
\newcommand{\beq}{\begin{equation}}
\newcommand{\eeq}{\end{equation}}
\newcommand{\bea}{\begin{eqnarray}}
\newcommand{\eea}{\end{eqnarray}}
\newcommand{\beas}{\begin{eqnarray*}}
\newcommand{\eeas}{\end{eqnarray*}}
\newcommand{\ba}{\begin{array}}
\newcommand{\ea}{\end{array}}

%%%%%%%% set chapter author commands
\newcommand{\chapauthor}[1]{\hspace*{\fill}{\textsf{\large 
 #1}}\\[1.5ex]}
%%%%%%%%%%%%%%

%%%%%%% set epigraph commands
\newcommand{\epigraph}[1]{\vspace*{2ex}{\raggedleft{\textit{#1}}\\[6ex]}}
%%%%%%%%%%%%

%%%%%% set definition commands
\newcounter{definitionctr}[chapter]
\def\thedefinitionctr{\arabic{chapter}--\arabic{definitionctr}}
\newcommand{\definition}[1]{\refstepcounter{definitionctr}\par\vskip2ex \indent\parbox[]{.96\textwidth}{\indent{\textbf{Definition~\thedefinitionctr.}~~#1}\par\vskip2ex}}
%%%%%%%%%%%%%%%%%

%%%%%%%%%%%%%%% create a thick line for tables
\newcommand{\thickhline}{\noalign{\hrule height 1.5pt}} 
%%%%%%%%%%%%%

%%%%%%%%%%%%%% define the unnumbered list
\newenvironment{unnumlist}{\begin{list}{}% empty for no label
{}} % empty since we don't need a counter and the standard lengths/indentations are fine
{\end{list}} 
%%%%%%%%%%%%%%%%%%%

%%%%%%%%%%% define a simple command for the quote environment
\newenvironment{italquote}{\begin{quote}{}% empty for no label
{}\itshape} % empty since we don't need a counter and the standard lengths/indentations are fine
{\end{quote}} 
%%%%%%%%%%%

%%%%%%%%%%% define a simple command for the quote environment
\newenvironment{dialogue}[2]{\begin{quote}{\bfseries #1:~}% empty for no label
{} #2} % empty since we don't need a counter & the standard lengths/indentations and text style are fine
{\end{quote}} 
%%%%%%%%%%%

%%%%%%%%%%%%% define the multicolumn list environment
\newcommand{\mclhead}{\bfseries} % define a multi-column list (mcl) heading
\newlength{\mcltopsep}% define a spacing command
\setlength{\mcltopsep}{\topsep} % 
\newenvironment{mcl}{\par\vspace*{\mcltopsep}
\begin{tabular}{ll}}{\end{tabular}\linebreak[4] \par\vspace*{.5\mcltopsep}}
%%%%%%%%%%%%%%%%

%%%%%%%%%%%% define some sidebar commands
% the sidebar is created via the framed.sty package
\definecolor{shadecolor}{gray}{0.9} % define the color to be used for the sidebar
% define the sidebar head  ---  corrected bug 4/19/2011
\newcommand{\sidebarhead}[1]{\noindent{\cmssbxsection #1}\hspace*{\fill}\linebreak\nopagebreak} 
%%%%%%%%%%%%%%

%%%%%%%%%%% define note, tip, and warning environments
\newcommand{\note}[2]{\pagebreak[3]\par\vspace*{2ex} % define a new note command 
\noindent{\sffamily\bfseries\small NOTE}\linebreak\vspace*{-1.5ex}\nopagebreak
\rule[1.5ex]{\textwidth}{.75pt}\linebreak\nopagebreak
{\sffamily\small  {\bfseries{\raggedright #1}\\}\nopagebreak
\noindent{#2}}\par
\noindent\rule[1ex]{\textwidth}{.75pt}\par\vspace*{1ex}\pagebreak[3] 
}

\newcommand{\tip}[2]{\pagebreak[3]\par\vspace*{2ex} % define a new tip command 
\noindent{\sffamily\bfseries\small TIP}\linebreak\vspace*{-1.5ex}\nopagebreak
\rule[1.5ex]{\textwidth}{.75pt}\linebreak\nopagebreak
{\sffamily\small  {\bfseries{\raggedright #1}\\}\nopagebreak
\nopagebreak
\noindent{#2}}\par
\noindent\rule[1ex]{\textwidth}{.75pt}\par\vspace*{1ex}\pagebreak[3]
}

\newcommand{\warning}[2]{\pagebreak[3]\par\vspace*{2ex} % define a new warning command 
\noindent{\sffamily\bfseries\small WARNING}\linebreak\vspace*{-1.5ex}\nopagebreak
\rule[1.5ex]{\textwidth}{.75pt}\linebreak\nopagebreak
{\sffamily\small  {\bfseries{\raggedright #1}\\}\nopagebreak
\noindent{#2}}\par
\noindent\rule[1ex]{\textwidth}{.75pt}\par\vspace*{1ex}\pagebreak[3]
}
%%%%%%%%%%%%%%

%%%%%%%%%%% define glossary command
\newcommand{\gloss}[2]{\pagebreak[3]\par
\noindent{\sffamily\bfseries #1}\\
\noindent{#2}\hspace*{\fill}\\ 
\par\pagebreak[3]}
%%%%%%%%%%%%%%%%
  % Loading the definitions needed for this book:


%%%%%%%%%%CHANGED FOR ROMEO
\setcounter{secnumdepth}{2}

% Required for PANDOC conversion
\newcommand{\passthrough}[1]{\lstset{mathescape=false}#1\lstset{mathescape=true}}

% Hack to allow PANDOC to parse Markdown in inline LaTeX environments
\let\Begin\begin
\let\End\end

% PANDOC stuff
\IfFileExists{upquote.sty}{\usepackage{upquote}}{}
\IfFileExists{microtype.sty}{% use microtype if available
  \usepackage[]{microtype}
  \UseMicrotypeSet[protrusion]{basicmath} % disable protrusion for tt fonts
}{}
\makeatletter
\@ifundefined{KOMAClassName}{% if non-KOMA class
  \IfFileExists{parskip.sty}{%
    \usepackage{parskip}
  }{% else
    \setlength{\parindent}{0pt}
    \setlength{\parskip}{6pt plus 2pt minus 1pt}}
}{% if KOMA class
  \KOMAoptions{parskip=half}}
\makeatother
\usepackage{fancyvrb}
\usepackage{xcolor}
\IfFileExists{xurl.sty}{\usepackage{xurl}}{} % add URL line breaks if available
\IfFileExists{bookmark.sty}{\usepackage{bookmark}}{\usepackage{hyperref}}
\hypersetup{
  hidelinks,
  pdfcreator={LaTeX via pandoc}}
\urlstyle{same} % disable monospaced font for URLs
\VerbatimFootnotes % allow verbatim text in footnotes


\usepackage{longtable,booktabs}
% Correct order of tables after \paragraph or \subparagraph
\usepackage{etoolbox}
\makeatletter
\patchcmd\longtable{\par}{\if@noskipsec\mbox{}\fi\par}{}{}
\makeatother
% Allow footnotes in longtable head/foot
\IfFileExists{footnotehyper.sty}{\usepackage{footnotehyper}}{\usepackage{footnote}}
\makesavenoteenv{longtable}

% 


\setlength{\emergencystretch}{3em} % prevent overfull lines
\providecommand{\tightlist}{%
\setlength{\itemsep}{0pt}\setlength{\parskip}{0pt}}

% 




\author{}
\date{}

\begin{document}

% PEARSON
%%%%%% commands to fine-tune to table of contents
%\cftnodots
%\renewcommand{\@dotsep}{10000}


\renewcommand{\cfttoctitlefont}{\cmsschapname}
\renewcommand{\cftpartfont}{\cmssbxparttoc}
\renewcommand{\cftchappagefont}{\cmssbxparttoc}
\renewcommand{\cftchapfont}{\cmssbxchaptoc}
\renewcommand{\cftchapnumwidth}{5.5em}
\renewcommand{\cftchappagefont}{\cmssbxchaptoc}
\renewcommand{\cftsecfont}{\sf}
\renewcommand{\cftsecindent}{0em}
%\renewcommand{\cftsecnumwidth}{0em}
\renewcommand{\cftsecpagefont}{\sf}
\renewcommand{\cftsubsecfont}{\sf}
\renewcommand{\cftsubsecindent}{4em}
%\renewcommand{\cftsubsecnumwidth}{0em}
\renewcommand{\cftsubsecpagefont}{\sf}
\renewcommand{\contentsname}{Contents}


%\setcounter{secnumdepth}{0} 
%\renewcommand{\numberline}[2]{} 
%%%%% end toc commands % Loading a few commands related to the tocloft package




% %
% %
% %
% %
% 
% @lorihughes: Pandoc markdown supports inline LaTeX. We can reuse existing `.tex` components easily.
% @lorihughes: We could also do the opposite! Take a look at https://tex.stackexchange.com/questions/374890
\input{emcpps_frontmatter}

% @lorihughes: For "uncommon" pieces such as pages introducing parts, or even tables, we could use inline LaTeX. We could work on a hybrid Markdown/LaTeX source, which will probably generate a cleaner final LaTeX result. -->
\part{This is the Part Title}
\vspace*{4pt}
\sffamily % this sets this page in sans serif font

\noindent Chapter \ref{ExampleChap}, "\titleref{ExampleChap},"  is followed by a one-sentence description of each chapter. 

Chapter \ref{samplechap2}, "\titleref{samplechap2},"  is followed by a one-sentence description of each chapter and if the sentence is longer what happens when it runs to two lines.

Chapter \ref{samplechap3}, "\titleref{samplechap3}," is followed by a one-sentence description of each chapter and if the sentence is longer what happens when it runs to two lines. 

\normalfont % this resets the fonts to our normal settings






% Chapter heading. This could be converted to a nice Markdown special class, but we could also just embed LaTeX as below.
\input{emcpps_chapter1head}

\hypertarget{c11}{%
\section{C++11}\label{c11}}

Small amount of intro text here. Lorem ipsum dolor sit amet, consectetur
adipiscing elit. Suspendisse non nisi ex. Ut non sollicitudin est. Donec
laoreet, risus pretium egestas condimentum, ex eros tempor diam, quis
congue nisi metus in tellus. Proin tempor ac lectus nec elementum.
Maecenas augue turpis, pellentesque sed eros sit amet, tincidunt pretium
nulla. Nunc lacus ligula, ullamcorper a porta eget, fringilla sed orci.
Maecenas eget ultricies risus. Donec varius vehicula diam.

\hypertarget{auto}{%
\subsection{\texorpdfstring{\texttt{auto}}{auto}}\label{auto}}

Small amount of intro text here. The \passthrough{\lstinline!auto!}
keyword was repurposed\footnote{Footnote. Lorem ipsum dolor sit amet,
  consectetur adipiscing elit.} in C++11 to act as a \emph{placeholder}
type. When used instead of a type as part of a variable declaration, the
{compiler} will use the same rules as {template argument deduction} to
deduce the type of the variable.

\hypertarget{description}{%
\subsubsection{Description}\label{description}}

Description of the feature here. Lorem ipsum dolor sit amet, consectetur
adipiscing elit. Suspendisse non nisi ex. Ut non sollicitudin est. Donec
laoreet, risus pretium egestas condimentum, ex eros tempor diam, quis
congue nisi metus in tellus. Proin tempor ac lectus nec elementum.
Maecenas augue turpis, pellentesque sed eros sit amet, tincidunt pretium
nulla. Nunc lacus ligula, ullamcorper a porta eget, fringilla sed orci.
Maecenas eget ultricies risus. Donec varius vehicula diam.

\hypertarget{automatic-type-deduction-rules}{%
\paragraph{\texorpdfstring{\emph{Automatic type deduction
rules}}{Automatic type deduction rules}}\label{automatic-type-deduction-rules}}

Sub-section that describes a particular aspect of the feature in
abstract terms. Lorem ipsum dolor sit amet, consectetur adipiscing elit.
Suspendisse non nisi ex. Ut non sollicitudin est. Donec laoreet, risus
pretium egestas condimentum, ex eros tempor diam, quis congue nisi metus
in tellus.

\Begin{leftbar}

Proin tempor ac lectus nec elementum. Maecenas augue turpis,
pellentesque sed eros sit amet, tincidunt pretium nulla. Nunc lacus
ligula, ullamcorper a porta eget, fringilla sed orci. Maecenas eget
ultricies risus. Donec varius vehicula diam.

\begin{lstlisting}[language=C++, caption={ example caption for recordcount }, label={ testlabel }, frame=tb]
int recordCount;
while (cursor.next()) { ++recordCount; }
    //                  ^~~~~~~~~~~~~
    //                Undefined behavior.
\end{lstlisting}
    

\End{leftbar}

Lorem ipsum dolor sit amet, consectetur adipiscing elit. Suspendisse non
nisi ex. Ut non sollicitudin est. Donec laoreet, risus pretium egestas
condimentum, ex eros tempor diam, quis congue nisi metus in tellus.
Proin tempor ac lectus nec elementum. Maecenas augue turpis,
pellentesque sed eros sit amet, tincidunt pretium nulla. Nunc lacus
ligula, ullamcorper a porta eget, fringilla sed orci. Maecenas eget
ultricies risus. Donec varius vehicula diam.

\begin{lstlisting}[language=C++, caption={ missing caption }, label={ testlabel }, frame=tb]
auto recordCount; // Compile-time error.
while (cursor.next()) { ++recordCount; }
\end{lstlisting}
    

Lorem ipsum dolor sit amet, consectetur adipiscing elit. Suspendisse non
nisi ex. Ut non sollicitudin est.

\begin{itemize}
\item
  Donec \passthrough{\lstinline!laoreet!}, risus pretium egestas
  condimentum, ex eros tempor diam, quis congue nisi metus in tellus.
\item
  Proin tempor ac lectus nec elementum. Maecenas augue turpis.
\item
  Lorem ipsum dolor sit amet, consectetur adipiscing elit. Suspendisse
  non nisi ex. Ut non sollicitudin est.
\end{itemize}

\hypertarget{use-cases}{%
\subsubsection{Use Cases}\label{use-cases}}

Small amount of intro text can optionally be here. Lorem ipsum dolor sit
amet, consectetur adipiscing elit. Suspendisse non nisi ex. Ut non
sollicitudin est. Donec laoreet, risus pretium egestas condimentum, ex
eros tempor diam.

\hypertarget{avoiding-type-repetition}{%
\paragraph{\texorpdfstring{\emph{Avoiding type
repetition}}{Avoiding type repetition}}\label{avoiding-type-repetition}}

Sub-section that describes a particular aspect of the feature in
concreteterms. Lorem ipsum dolor sit amet, consectetur adipiscing elit.
Suspendisse non nisi ex. Ut non sollicitudin est. Donec laoreet, risus
pretium egestas condimentum, ex eros tempor diam, quis congue nisi metus
in tellus. Proin tempor ac lectus nec elementum\footnote{The relative
  position of \passthrough{\lstinline!decltype(range.sort())!} in the
  signature of \passthrough{\lstinline!sortRangeImpl!} is not
  significant, as long as it is visible to the compiler during template
  substitution. This particular example (shown in the main text) makes
  use of a function parameter that is defaulted to
  \passthrough{\lstinline!nullptr!}. Alternatives involving a trailing
  return type or a default template argument are also viable:}.

\begin{lstlisting}[language=C++, caption={ missing caption }, label={ testlabel }, frame=tb]
int x = 10;
auto y = x;
\end{lstlisting}
    

Maecenas augue turpis, pellentesque sed eros sit amet, tincidunt pretium
nulla. Nunc lacus ligula, ullamcorper a porta eget, fringilla sed orci.
Maecenas eget ultricies risus. Donec varius vehicula diam.\footnote{For
  more information on foobar, see \textbf{lakos20}, section 1.2.3, pp
  208-234, especially Figure 1-35, p.~215.}

\hypertarget{potential-pitfalls}{%
\subsubsection{Potential Pitfalls}\label{potential-pitfalls}}

Lorem ipsum dolor sit amet, consectetur adipiscing elit. Suspendisse non
nisi ex. Ut non sollicitudin est. Donec laoreet, risus pretium egestas
condimentum, ex eros tempor diam, quis congue nisi metus in tellus.
Proin tempor ac lectus nec elementum. Maecenas augue turpis,
pellentesque sed eros sit amet, tincidunt pretium nulla. Nunc lacus
ligula, ullamcorper a porta eget, fringilla sed orci. Maecenas eget
ultricies risus. Donec varius vehicula diam.\footnote{\textbf{lakos20},
  section 0.5, pp 34-42

  \begin{lstlisting}[language=C++, label={ testlabel }, basicstyle=\footnotesize]
  template <typename Range>
  auto sortRangeImpl(Range& range, int) -> decltype(range.sort(), void());
      // The comma operator is used to force the return type to `void`,
      // regardless of the return type of `range.sort()`.

  template <typename Range, typename = decltype(std::declval<Range&>().sort()>
  auto sortRangeImpl(Range& range, int);
      // `std::declval` is used to generate a reference to `Range` that can be
      // used in an unevaluated expression
  \end{lstlisting}
      }

In the example above, \passthrough{\lstinline!localhost!} will be
initialized in the following manner\footnote{Footnote. Lorem ipsum dolor
  sit amet, consectetur adipiscing elit.}:

\begin{enumerate}
\def\labelenumi{\arabic{enumi}.}
\item
  {Constructor} \textbf{(0)} will be invoked;
\item
  On line \textbf{(1)}, execution will be delegated to constructor
  \textbf{(2)};
\item
  The body of constructor \textbf{(2)} will be executed;
\item
  The body of constructor \textbf{(1)} will be executed.
\end{enumerate}

This feature, when used in conjunction with \emph{explicit instantiation
definitions}, can significantly improve compilation times for a set of
translation units that often instantiate common templates:

\begin{longtable}[]{@{}lll@{}}
\toprule
\endhead
\begin{minipage}[t]{0.30\columnwidth}\raggedright
\begin{lstlisting}[language=C++, caption={ missing caption }, label={ testlabel }, frame=tb]
// foo.h

template <typename T>
void foo() { /* ... */ }
    // Template definition







\end{lstlisting}
    \strut
\end{minipage} & \begin{minipage}[t]{0.30\columnwidth}\raggedright
\begin{lstlisting}[language=C++, caption={ missing caption }, label={ testlabel }, frame=tb]
// foo.h

template <typename T>
void foo() { /* ... */ }
    // Template definition







\end{lstlisting}
    \strut
\end{minipage} & \begin{minipage}[t]{0.30\columnwidth}\raggedright
\begin{lstlisting}[language=C++, caption={ missing caption }, label={ testlabel }, frame=tb]
// foo.h

template <typename T>
void foo() { /* ... */ }
    // Template definition







\end{lstlisting}
    \strut
\end{minipage}\tabularnewline
\bottomrule
\end{longtable}

Attempting to compile \passthrough{\lstinline!main.cpp!} on its own will
produce a linker error along the lines of:

\begin{lstlisting}[language=C++, caption={ missing caption }, label={ testlabel }, frame=tb]
undefined reference to `Vector2D<float>::normalize()'
\end{lstlisting}
    

The linker error is expected as the inclusion of
\passthrough{\lstinline!vector2d.h!} suppresses implicit instantiation
of \passthrough{\lstinline!Vector2D<float>!}. Note that
\passthrough{\lstinline!iVec!} is not affected, as the
\passthrough{\lstinline!Vector2D<int>!} instantiation does take place.

\hypertarget{readability-concerns}{%
\paragraph{\texorpdfstring{\emph{Readability
concerns}}{Readability concerns}}\label{readability-concerns}}

Using \passthrough{\lstinline!auto!} can hide all information regarding
a variable's type, increasing cognitive overhead for the readers. In
conjunction with unclear variable naming, disproportionate usage of
\passthrough{\lstinline!auto!} can make code unreadable. E.g.

\begin{lstlisting}[language=C++, caption={ missing caption }, label={ testlabel }, frame=tb]
int main(int argc, char** argv)
{
    const auto args0 = parseArgs(argc, argv);
        // The behavior of `parseArgs` is unclear.

    const std::vector<std::string> args1 = parseArgs(argc, argv);
        // It is obvious what `parseArgs` does.
}
\end{lstlisting}
    

While it may be necessary to read \passthrough{\lstinline!parseArgs!}'s
contract at least once to fully understand its behavior, an explicit
type in the usage site helps readers understand its purpose
\href{http://wg21.link/cwg1655}{CWG1655}.

\begin{lstlisting}
testing column width for code 
123456789A123456789B123456789C123456789D123456789E123456789F123456789G123456789H 
this is 80 characters!}
\end{lstlisting}

C++11 introduces three new types of string literal, which provide strong
guarantees on the encoding of character sequences:

\begin{longtable}[]{@{}lll@{}}
\toprule
Encoding & Example & Character Type\tabularnewline
\midrule
\endhead
UTF-8 & \passthrough{\lstinline!u8"Hello"!} &
\passthrough{\lstinline!char!}\tabularnewline
UTF-16 & \passthrough{\lstinline!u"Hello"!} &
\passthrough{\lstinline!char16\_t!}\tabularnewline
UTF-32 & \passthrough{\lstinline!U"Hello"!} &
\passthrough{\lstinline!char32\_t!}\tabularnewline
\bottomrule
\end{longtable}

Raw string literals enable developers to embed strings in a program's
source code without requiring to escape special character sequences and
preserving whitespace, with the goal of enhancing readability. The
syntax of the feature is easily understood through an example showing a
\emph{regular expression} embedded in source code:

\begin{lstlisting}[language=C++, caption={ missing caption }, label={ testlabel }, frame=tb]
//            delimiter and round parenthesis
//                    v~~~             v~~~
const char* regex = R"xxx([0-9]\(".*"\))xxx";
//                  ^     ^~~~~~~~~~~~~
//                  |      string contents
//                  |
//                  | uppercase R
\end{lstlisting}
    

\hypertarget{lack-of-interface-restrictions}{%
\paragraph{\texorpdfstring{\emph{Lack of interface
restrictions}}{Lack of interface restrictions}}\label{lack-of-interface-restrictions}}

In generic code, even if concrete types are dependent on template
arguments, \passthrough{\lstinline!auto!} is needlessly lax. It is
always possible to identify a \emph{concept}\footnote{Authors' Note: We
  will have some footnotes that are authors' notes.} which provides
information regarding operations allowed on a type to the reader {[}see
@AGE86, pp. 33-35{]}, albeit specifying it in code is
cumbersome.\footnote{Unless explicitly specified, the \emph{underlying
  type} of non-strongly typed enumerations is
  {[}\emph{implementation-defined}{]}.}

In some particular cases, concepts also carry important semantic meaning
that could be lost by using \passthrough{\lstinline!auto!}. E.g.

\begin{lstlisting}[language=C++, caption={ missing caption }, label={ testlabel }, frame=tb]
Packet* PacketCache::findFirstCorruptPacket() const
{
    auto it = std::begin(this->d_packet);

    static_assert(IsRandomAccessIterator<decltype(it)>::value,
                  "`it` must be a random access iterator.");

    return it == std::end(this->d_employees) ? nullptr
                                             : &*it;
}
\end{lstlisting}
    

\hypertarget{list-initialization}{%
\paragraph{\texorpdfstring{\emph{List
initialization}}{List initialization}}\label{list-initialization}}

The meaning of \passthrough{\lstinline!auto!} completely changes when
using \emph{list initialization}:
\passthrough{\lstinline!std::initializer\_list!} is always deduced.

\begin{lstlisting}[language=C++, caption={ missing caption }, label={ testlabel }, frame=tb]
auto example0 = 0; // Copy initialiation, deduced as `int`.
auto example1(0);  // Direct initialiation, deduced as `int`.
auto example2{0};  // List initialiation, deduced as `std::initializer_list<int>`.
\end{lstlisting}
    

\begin{longtable}[]{@{}clrl@{}}
\toprule
\endhead
\begin{minipage}[t]{0.15\columnwidth}\centering
First\strut
\end{minipage} & \begin{minipage}[t]{0.10\columnwidth}\raggedright
row\strut
\end{minipage} & \begin{minipage}[t]{0.20\columnwidth}\raggedleft
12.0\strut
\end{minipage} & \begin{minipage}[t]{0.32\columnwidth}\raggedright
Example of a row that spans multiple lines.\strut
\end{minipage}\tabularnewline
\begin{minipage}[t]{0.15\columnwidth}\centering
Second\strut
\end{minipage} & \begin{minipage}[t]{0.10\columnwidth}\raggedright
row\strut
\end{minipage} & \begin{minipage}[t]{0.20\columnwidth}\raggedleft
5.0\strut
\end{minipage} & \begin{minipage}[t]{0.32\columnwidth}\raggedright
Here's another one. Note the blank line between rows.\strut
\end{minipage}\tabularnewline
\bottomrule
\end{longtable}

This surprising behavior contradicts the idea of ``uniform
initialization'' and has been widely regarded as a mistake and rectified
in C++14.

The \passthrough{\lstinline!decltype!} keyword allows inspecting the
declared type of an entity or the type and value category of an
expression. What \passthrough{\lstinline!decltype!} yields as the result
depends on the provided argument:

\begin{itemize}
\item
  With an unparenthesized \emph{id-expression}\footnote{Footnote. Lorem
    ipsum dolor sit amet, consectetur adipiscing elit.} or
  unparenthesized \emph{class member access expression}\footnote{Footnote.
    Lorem ipsum dolor sit amet, consectetur adipiscing elit.},
  \passthrough{\lstinline!decltype!} yields the \emph{``declared
  type''}\footnote{Footnote. Lorem ipsum dolor sit amet, consectetur
    adipiscing elit.} of the given expression.
\item
  With any other expression of type \passthrough{\lstinline!T!},
  \passthrough{\lstinline!decltype!} yields:

  \begin{itemize}
  \item
    \passthrough{\lstinline!T\&\&!} if the value category of the
    expression is \emph{xvalue};
  \item
    \passthrough{\lstinline!T\&!} if the value category of the
    expression is \emph{lvalue};
  \item
    \passthrough{\lstinline!T!} if the value category of the expression
    is \emph{prvalue}.
  \end{itemize}
\end{itemize}

Similarly to \passthrough{\lstinline!sizeof!}, the provided expression
is not evaluated.

\newpage

\cleardoublepage

\chapter*{Glossary}\label{Glossary}

\gloss{word to be defined}{Definition text follows. At vero eos et accusamus et iusto odio dignissimos ducimus qui blanditiis praesentium voluptatum deleniti atque corrupti quos dolores et quas molestias excepturi sint occaecati cupiditate non provident, similique sunt in culpa qui officia deserunt mollitia animi, id est laborum et dolorum fuga.}

\gloss{word to be defined}{Definition text follows. Itaque earum rerum hic tenetur a sapiente delectus, ut aut reiciendis voluptatibus maiores alias consequatur aut perferendis doloribus asperiores repellat.}

\gloss{word to be defined}{Definition text follows.Nam libero tempore, cum soluta nobis est eligendi optio cumque nihil impedit quo minus id quod maxime placeat facere possimus, omnis voluptas assumenda est, omnis dolor repellendus.  }

\gloss{word to be defined}{Definition text follows. At vero eos et accusamus et iusto odio dignissimos ducimus qui blanditiis praesentium voluptatum deleniti atque corrupti quos dolores et quas molestias excepturi sint occaecati cupiditate non provident, similique sunt in culpa qui officia deserunt mollitia animi, id est laborum et dolorum fuga. }

\cleardoublepage
\bibliographystyle{ieeetr-hacked}

\bibliography{samplebib}
\nocite{*}

\cleardoublepage

\newcommand\see[2]{\emph{\seename} #1}  
\newcommand\also[2]{\emph{See also} #1} 
\newcommand\seename{See} 
\newcommand\seealso[2]{\emph{See also} #1}

\begin{theindex}
\footnotesize
{\sffamily\bfseries{Symbols}}\nopagebreak

  \item \$, 31
  \item \%, 12
  \item \&, 31

  \indexspace
{\sffamily\bfseries\sffamily\bfseries{A}}\nopagebreak

  \item a very long entry to test the column width, 31
  \item anise, 31
  \item apple
    \subitem braebrun, 31
    \subitem cameo, 31
    \subitem fuji, 31
    \subitem gala, 31
    \subitem granny smith, 31
    \subitem red delicious, 31
  \item apricots, 31
  \item avocado, 31

  \indexspace
{\sffamily\bfseries{B}}\nopagebreak

  \item banana, 31
  \item basil, 31
  \item bibendum, 12
    \subitem dapibus, 12
  \item blackberry, 31

  \indexspace
{\sffamily\bfseries{C}}\nopagebreak

  \item cabbage, 31
  \item celery, 31
  \item chervil, 31
  \item chives, 31
  \item cilantro, 31
  \item \protect{\tt codeword}, 12
  \item corn, 31
  \item cucumber, 31

  \indexspace
{\sffamily\bfseries{D}}\nopagebreak

  \item dates, 31
  \item dill, 31

  \indexspace
{\sffamily\bfseries{E}}\nopagebreak

  \item eggplant, 31
  \item endive, 31

  \indexspace
{\sffamily\bfseries{F}}\nopagebreak

  \item fennel, 31
  \item fig, 31
  \item \protect{\tt function}, 31

  \indexspace
{\sffamily\bfseries{G}}\nopagebreak

  \item garlic, 31
  \item grapes, 31

  \indexspace
{\sffamily\bfseries{H}}\nopagebreak

  \item horseradish, 31
  \item how about another long entry, 31
  \item huckleberry, 31

  \indexspace
{\sffamily\bfseries{J}}\nopagebreak

  \item jicama, 31

  \indexspace
{\sffamily\bfseries{K}}\nopagebreak

  \item kale, 31
  \item kiwi, 31

  \indexspace
{\sffamily\bfseries{L}}\nopagebreak

  \item leeks, 31
  \item lemon, 31
  \item lettuce
    \subitem boston bibb, 31
    \subitem iceberg, 31
    \subitem mesclun, 31
    \subitem red leaf, 31
  \item lime, 31
  \item lorem, \see{lobortis}{12}

  \indexspace
{\sffamily\bfseries{M}}\nopagebreak

  \item majoram, 31
  \item mango, 31
  \item maybe another long entry for this test, 31
  \item melon
    \subitem canary, 31
    \subitem cantaloupe, 31
    \subitem honeydew, 31
    \subitem watermelon, 31
  \item mushrooms
    \subitem button, 31
    \subitem porcini, 31
    \subitem portabella, 31
    \subitem shitake, 31

  \indexspace
{\sffamily\bfseries{N}}\nopagebreak

  \item nectarine, 31
  \item nutmeg, 31

  \indexspace
{\sffamily\bfseries{O}}\nopagebreak

  \item okra, 31\pagebreak % you'll need to manually break pages or columns at times to get a nice layout
  \item onion
    \subitem red, 31
    \subitem vidalia, 31
    \subitem yellow, 31
  \item orange, 31

  \indexspace
{\sffamily\bfseries{P}}\nopagebreak

  \item papaya, 31
  \item parsley, 31
  \item peaches, 31
  \item peppers
    \subitem ancho, 31
    \subitem bell, 31
    \subitem habe\~{n}eros, 31
    \subitem jalape\~{n}os, 31
    \subitem pabla\~{n}os, 31
  \item perhaps yet another long entry for this test, 31
  \item plantains, 31
  \item plums, 31
  \item potatoes
    \subitem red-skinned, 31
    \subitem russet, 31
    \subitem yukon gold, 31
  \item pumpkin, 31

  \indexspace
{\sffamily\bfseries{Q}}\nopagebreak

  \item quince, 31

  \indexspace
{\sffamily\bfseries{R}}\nopagebreak

  \item radicchio, 31
  \item radish, 31
  \item raspberry, 31
  \item rosemary, 31
  \item rutabaga, 31\newpage % to end this column short so that it and the last column are balanced

  \indexspace
{\sffamily\bfseries{S}}\nopagebreak

  \item shallots, 31
  \item spinach, 31
  \item squash, 31
  \item still another long entry for column width testing, 31

  \indexspace
{\sffamily\bfseries{T}}\nopagebreak

  \item thyme, 31
  \item tomatillo, 31
  \item tomatoes
    \subitem cherry, 31
    \subitem grape, 31
    \subitem heirloom, 31
    \subitem hybrid, 31
    \subitem roma, 31
  \item \protect{\tt typeof}, 31

  \indexspace
{\sffamily\bfseries{U}}\nopagebreak

  \item ugly fruit, 31

  \indexspace
{\sffamily\bfseries{V}}\nopagebreak

  \item verbena, 31
  \item viverra, \seealso{neque}{12}

  \indexspace
{\sffamily\bfseries{X}}\nopagebreak

  \item xacuti masala, 31

  \indexspace
{\sffamily\bfseries{Y}}\nopagebreak

  \item yams, 31
  \item yet another very long entry for column width test, 31

  \indexspace
{\sffamily\bfseries{Z}}\nopagebreak

  \item zucchini, 31

\end{theindex}


% 
% %
% 
% 
\end{document}
