%%%% CHANGED FOR ROMEO
% Changing code font to Cousine
\setmonofont[Ligatures=NoCommon,Scale=.9,SlantedFont={Cousine-Italic.ttf},BoldFont={Cousine-Bold.ttf},BoldSlantedFont={Cousine-BoldItalic.ttf}]{Cousine-Regular.ttf} % to set our code font
\newfontface{\SecCode}[Ligatures=NoCommon,Scale=1.3]{Cousine-Bold.ttf}
%\newfontface{\SecCode}[Ligatures=NoCommon,Scale=1.105]{Cousine-Bold.ttf}
\newfontface{\SubsecCode}[Ligatures=NoCommon,Scale=1.2]{Cousine-Bold.ttf}
%\newfontface{\SubsecCode}[Ligatures=NoCommon,Scale=1.02]{Cousine-Bold.ttf}
\newfontface{\SubsubsecCode}[Ligatures=NoCommon,Scale=1]{Cousine-Bold.ttf}
%\newfontface{\SubsubsecCode}[Ligatures=NoCommon,Scale=.935]{Cousine-Bold.ttf}
\newfontface{\ParaCode}[Ligatures=NoCommon,Scale=.9]{Cousine-Bold.ttf}
\newfontface{\SubparaCode}[Ligatures=NoCommon,Scale=.9]{Cousine-Regular.ttf}
%\newfontface{\RHCode}[Ligatures=NoCommon,Scale=.81]{Cousine-Regular.ttf}
%\newfontface{\RHCode}[Ligatures=NoCommon,Scale=.85]{Cousine-Bold.ttf}
\newfontface{\RHCode}[Ligatures=NoCommon,Scale=1.17]{Cousine-Bold.ttf} % set at scaled .9 of 13pts
\newfontface{\RHCodeEleven}[Ligatures=NoCommon,Scale=.935]{Cousine-Bold.ttf}
%\newfontface{\RHCodeTwelve}[Ligatures=NoCommon,Scale=1.02]{Cousine-Bold.ttf}
\newfontface{\RHCodeTwelve}[Ligatures=NoCommon,Scale=1.1]{Cousine-Bold.ttf}
\newfontface{\TOCCode}[Ligatures=NoCommon,Scale=.85]{Cousine-Bold.ttf}
%\newfontface{\TOCCode}[Ligatures=NoCommon,Scale=.9]{Cousine-Bold.ttf}
\newfontface{\TOCCodeR}[Ligatures=NoCommon,Scale=.85]{Cousine-Regular.ttf}
\newfontface{\TOCCodeEleven}[Ligatures=NoCommon,Scale=.935]{Cousine-Regular.ttf}
\newfontface{\TOCCodeElevenB}[Ligatures=NoCommon,Scale=.935]{Cousine-Bold.ttf}
%\newfontface{\TOCCodeElevenB}[Ligatures=NoCommon,Scale=1.1]{Cousine-Bold.ttf}
\newfontface{\TOCCodeTwelve}[Ligatures=NoCommon,Scale=1.02]{Cousine-Regular.ttf}
\newfontface{\TOCCodeThirteen}[Ligatures=NoCommon,Scale=1.105]{Cousine-Regular.ttf}

% Changing sans serif font to OpenSans
\setsansfont[Ligatures=Common,Scale=1,SlantedFont={OpenSans-LightItalic.ttf},BoldFont={OpenSans-Bold.ttf},BoldSlantedFont={OpenSans-BoldItalic.ttf}]{OpenSans-Light.ttf} % to set our sans font
%%
\newfontface{\sfbHugeRomeo}[Ligatures=Common,Scale=2.5]{OpenSans-Bold.ttf}
\newfontface{\cmssbxparttocRomeo}[Ligatures=Common,Scale=1.2]{OpenSans-Bold.ttf}
\newfontface{\cmssbxsectionRomeo}[Ligatures=Common,Scale=1.4]{OpenSans-Bold.ttf}
\newfontface{\cmssbxelevenRomeo}[Ligatures=Common,Scale=1.1]{OpenSans-Bold.ttf}
\newfontface{\cmssbxchaptocRomeo}[Ligatures=Common,Scale=1]{OpenSans-Bold.ttf}
\newfontface{\cmssbxchaptitleRomeo}[Ligatures=Common,Scale=3.0]{OpenSans-Bold.ttf}
\newfontface{\cmsschapnameRomeo}[Ligatures=Common,Scale=2.6]{OpenSans-Light.ttf}
\newfontface{\cmssbxpartRomeo}[Ligatures=Common,Scale=4.4]{OpenSans-Bold.ttf}
\newfontface{\cmssparttitleRomeo}[Ligatures=Common,Scale=3.4]{OpenSans-Light.ttf}
\newfontface{\sfiHugeRomeo}[Ligatures=Common,Scale=2.5]{OpenSans-LightItalic.ttf}
\newfontface{\sfititleRomeo}[Ligatures=Common,Scale=2.4]{OpenSans-LightItalic.ttf}
\newfontface{\sfihalftitleRomeo}[Ligatures=Common,Scale=2.0]{OpenSans-LightItalic.ttf}
\newfontface{\sfiauthorRomeo}[Ligatures=Common,Scale=1.6]{OpenSans-LightItalic.ttf}
\newfontface{\sfblargeRomeo}[Ligatures=Common,Scale=1.2]{OpenSans-Bold.ttf}
\newfontface{\sfbelevenRomeo}[Ligatures=Common,Scale=1.1]{OpenSans-Bold.ttf}
\newfontface{\sfbRomeo}[Ligatures=Common,Scale=1]{OpenSans-Bold.ttf}
\newfontface{\sfeightRomeo}[Ligatures=Common,Scale=.8]{OpenSans-Light.ttf}
\newfontface{\sfnineRomeo}[Ligatures=Common,Scale=.9]{OpenSans-Light.ttf}
\newfontface{\defnemRomeo}[Ligatures=Common,Scale=1]{OpenSans-Bold.ttf}
%%%%%
\newfontface{\sfbsectionRomeo}[Ligatures=Common,Scale=1.3]{OpenSans-Bold.ttf}
\newfontface{\sfbsectionitalRomeo}[Ligatures=Common,Scale=1.4]{OpenSans-BoldItalic.ttf}
\newfontface{\sfbsubsecRomeo}[Ligatures=Common,Scale=1.2]{OpenSans-Bold.ttf}
\newfontface{\sfbsubsubRomeo}[Ligatures=Common,Scale=1.1]{OpenSans-Bold.ttf}
\newfontface{\sfbparaRomeo}[Ligatures=Common,Scale=1]{OpenSans-Bold.ttf}
\newfontface{\sfbsubsecitalRomeo}[Ligatures=Common,Scale=1.2]{OpenSans-BoldItalic.ttf}
\newfontface{\sfbsubsubsecitalRomeo}[Ligatures=Common,Scale=1.1]{OpenSans-BoldItalic.ttf}
\newfontface{\sfbiRHRomeo}[Ligatures=Common,Scale=1.3]{OpenSans-BoldItalic.ttf}
\newfontface{\sfiRHRomeo}[Ligatures=Common,Scale=1.3]{OpenSans-LightItalic.ttf}
\newfontface{\sfRHRomeo}[Ligatures=Common,Scale=1.3]{OpenSans-Light.ttf}

%% redefining existing sans serif font commands
\renewcommand{\sfbHuge}{\sfbHugeRomeo}       
\renewcommand{\cmssbxparttoc}{\cmssbxparttocRomeo}
\renewcommand{\cmssbxsection}{\cmssbxsectionRomeo}
\renewcommand{\cmssbxeleven}{\cmssbxelevenRomeo}
\renewcommand{\cmssbxchaptoc}{\cmssbxchaptocRomeo}    
\renewcommand{\cmssbxchaptitle}{\cmssbxchaptitleRomeo}
\renewcommand{\cmsschapname}{\cmsschapnameRomeo}
\renewcommand{\cmssbxpart}{\cmssbxpartRomeo}
\renewcommand{\cmssparttitle}{\cmssparttitleRomeo} 
\newcommand{\sfiHuge}{\sfiHugeRomeo}
\renewcommand{\sfititle}{\sfititleRomeo}
\renewcommand{\sfihalftitle}{\sfihalftitleRomeo} 
\renewcommand{\sfiauthor}{\sfiauthorRomeo} 
\renewcommand{\sfblarge}{\sfblargeRomeo} 
\renewcommand{\sfbeleven}{\sfbelevenRomeo}  
\renewcommand{\sfb}{\sfbRomeo} 
\renewcommand{\sfeight}{\sfeightRomeo}
\renewcommand{\sfnine}{\sfnineRomeo}
\renewcommand{\defnem}{\defnemRomeo}                  
%%%%
\renewcommand{\sfbsection}{\sfbsectionRomeo}
\renewcommand{\sfbsubsec}{\sfbsubsecRomeo}
\renewcommand{\sfbsubsub}{\sfbsubsubRomeo}
\renewcommand{\sfbpara}{\sfbparaRomeo}

%%%%%%% emoji fonts
\newcommand{\martini}{\includegraphics[scale=.0125]{1F378_color}}
% this is an open source emoji image.  Make sure to note for Right & Reprints. 

%%%%%%% strikeout text function
\usepackage[normalem]{ulem}  % used for a meredith citation
%%%%%%%%%%%%%%
