%%%%%%%%%%%%%%%%%%%%%%%%%%%%%
%%% Altered for Lakos & Romeo book, 2020
% Changes marked with ROMEO-


% New commands and/or command redefinitions
%
% You can also place such commands in a macro file (e.g. mydefs.tex)
% and load them in the preamble with "input mydefs"
% Here are some examples:

\newcommand{\be}{\begin{equation}}           %---- begin numbered equation
\newcommand{\ee}{\end{equation}}             %---- end numbered equation
\newcommand{\dst}{\everymath{\displaystyle}} %---- use displaystle in eqs.
\newcommand{\Nuclide}[2]{${}^{#2}$#1}        %---- Nuclide macro
\renewcommand{\SS}[1]{${}^{#1}$}             %---- superscript in text
\newcommand{\STRUT}{\rule{0in}{3ex}}         %--- small strut
\newcommand{\dotprod}{{\scriptscriptstyle \stackrel{\bullet}{{}}}}
\renewcommand{\tilde}{\widetilde}
\renewcommand{\hat}{\widehat}
\renewcommand{\theequation}{\arabic{chapter}.\arabic{equation}}
%\renewcommand{\thefigure}{\arabic{chapter}.\arabic{section}--\arabic{figure}}  %%% Romeo - reset per section


\renewcommand{\thetable}{\arabic{table}}   %%% Romeo - reset per section
\newcommand{\ul}{\underline}
\newcommand{\dul}{\underline{\underline}}
\newcommand{\ol}{\overline}
\newcommand{\beq}{\begin{equation}}
\newcommand{\eeq}{\end{equation}}
\newcommand{\bea}{\begin{eqnarray}}
\newcommand{\eea}{\end{eqnarray}}
\newcommand{\beas}{\begin{eqnarray*}}
\newcommand{\eeas}{\end{eqnarray*}}
\newcommand{\ba}{\begin{array}}
\newcommand{\ea}{\end{array}}

%%%%%%%% set chapter author commands
\newcommand{\chapauthor}[1]{\hspace*{\fill}{\textsf{\large 
 #1}}\\[1.5ex]}
%%%%%%%%%%%%%%

%%%%%%% set epigraph commands
\newcommand{\epigraph}[1]{\vspace*{2ex}{\raggedleft{\textit{#1}}\\[6ex]}}
%%%%%%%%%%%%

%%%%%% set definition commands
\newcounter{definitionctr}[chapter]
\def\thedefinitionctr{\arabic{chapter}--\arabic{definitionctr}}
\newcommand{\definition}[1]{\refstepcounter{definitionctr}\par\vskip2ex \indent\parbox[]{.96\textwidth}{\indent{\textbf{Definition~\thedefinitionctr.}~~#1}\par\vskip2ex}}
%%%%%%%%%%%%%%%%%

%%%%%%%%%%%%%%% create a thick line for tables
\newcommand{\thickhline}{\noalign{\hrule height 1.5pt}} 
%%%%%%%%%%%%%

%%%%%%%%%%%%%% define the unnumbered list
\newenvironment{unnumlist}{\begin{list}{}% empty for no label
{}} % empty since we don't need a counter and the standard lengths/indentations are fine
{\end{list}} 
%%%%%%%%%%%%%%%%%%%

%%%%%%%%%%% define a simple command for the quote environment
\newenvironment{italquote}{\begin{quote}{}% empty for no label
{}\itshape} % empty since we don't need a counter and the standard lengths/indentations are fine
{\end{quote}} 
%%%%%%%%%%%

%%%%%%%%%%% define a simple command for the quote environment
\newenvironment{dialogue}[2]{\begin{quote}{\bfseries #1:~}% empty for no label
{} #2} % empty since we don't need a counter & the standard lengths/indentations and text style are fine
{\end{quote}} 
%%%%%%%%%%%

%%%%%%%%%%%%% define the multicolumn list environment
\newcommand{\mclhead}{\bfseries} % define a multi-column list (mcl) heading
\newlength{\mcltopsep}% define a spacing command
\setlength{\mcltopsep}{\topsep} % 
\newenvironment{mcl}{\par\vspace*{\mcltopsep}
\begin{tabular}{ll}}{\end{tabular}\linebreak[4] \par\vspace*{.5\mcltopsep}}
%%%%%%%%%%%%%%%%

%%%%%%%%%%%% define some sidebar commands
% the sidebar is created via the framed.sty package
\definecolor{shadecolor}{gray}{0.9} % define the color to be used for the sidebar
% define the sidebar head  ---  corrected bug 4/19/2011
\newcommand{\sidebarhead}[1]{\noindent{\cmssbxsection #1}\hspace*{\fill}\linebreak\nopagebreak} 
%%%%%%%%%%%%%%

%%%%%%%%%%% define note, tip, and warning environments
\newcommand{\note}[2]{\pagebreak[3]\par\vspace*{2ex} % define a new note command 
\noindent{\sffamily\bfseries\small NOTE}\linebreak\vspace*{-1.5ex}\nopagebreak
\rule[1.5ex]{\textwidth}{.75pt}\linebreak\nopagebreak
{\sffamily\small  {\bfseries{\raggedright #1}\\}\nopagebreak
\noindent{#2}}\par
\noindent\rule[1ex]{\textwidth}{.75pt}\par\vspace*{1ex}\pagebreak[3] 
}

\newcommand{\tip}[2]{\pagebreak[3]\par\vspace*{2ex} % define a new tip command 
\noindent{\sffamily\bfseries\small TIP}\linebreak\vspace*{-1.5ex}\nopagebreak
\rule[1.5ex]{\textwidth}{.75pt}\linebreak\nopagebreak
{\sffamily\small  {\bfseries{\raggedright #1}\\}\nopagebreak
\nopagebreak
\noindent{#2}}\par
\noindent\rule[1ex]{\textwidth}{.75pt}\par\vspace*{1ex}\pagebreak[3]
}

\newcommand{\warning}[2]{\pagebreak[3]\par\vspace*{2ex} % define a new warning command 
\noindent{\sffamily\bfseries\small WARNING}\linebreak\vspace*{-1.5ex}\nopagebreak
\rule[1.5ex]{\textwidth}{.75pt}\linebreak\nopagebreak
{\sffamily\small  {\bfseries{\raggedright #1}\\}\nopagebreak
\noindent{#2}}\par
\noindent\rule[1ex]{\textwidth}{.75pt}\par\vspace*{1ex}\pagebreak[3]
}
%%%%%%%%%%%%%%

%%%%%%%%%%% define glossary command
\newcommand{\gloss}[2]{\pagebreak[3]\par
\noindent{\sffamily\bfseries #1}\\
\noindent{#2}\hspace*{\fill}\\ 
\par\pagebreak[3]}
%%%%%%%%%%%%%%%%

%%%%%%%%%%%%%% ROMEO - reduce indent of nested itemize
%%%% NOTE than any nested enumerate will need to be set within the list itself
\setlength{\leftmarginii}{4mm}




