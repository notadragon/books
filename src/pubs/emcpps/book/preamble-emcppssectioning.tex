%------------------------------------------------------------------------------
% FEATURE SECTION declaration
%------------------------------------------------------------------------------

\makeatletter

% These keys define the options available to the \emcppsFeature command
\define@cmdkey[emcpps]{section}{tocshort}{}
\define@cmdkey[emcpps]{section}{toclong}{}
\define@cmdkey[emcpps]{section}{rhshort}{}
\define@cmdkey[emcpps]{section}{headinglong}{}
\define@cmdkey[emcpps]{section}{short}{}
\define@cmdkey[emcpps]{section}{long}{}

\presetkeys[emcpps]{section}{ tocshort = {},
                              toclong  = {},
                              rhshort  = {},
                              headinglong = {},
                              short = {},
                              long = {},
                             }{}

% Default display macro for titles, this shoud never be rendered, and will
% only be visible if the corresponding macro has not been set properly.
\newcommand{\emcppsShowTitleDefaultImpl}[1]{%
  --- this should not be seen --- #1 --- \cmdecpps@section@short --- \cmdemcpps@section@long%
}

% ------------------------section headings-------------------------------------
% The title of a section at the start will use the "headinglong" option if
% it is set, otherwise the "long" option will be used
\newcommand{\emcppsShowTitleSectionImpl}{%
  \ifthenelse{\equal{}{\cmdemcpps@section@headinglong}}%
    {\cmdemcpps@section@long}%
    {\cmdemcpps@section@headinglong}%
}

% The runnning head in a section, which uses the alt title from the \section
% command, will use the "rhshort" option if set, otherwise the "long" option.
\newcommand{\emcppsShowTitleRHImpl}{%
  \ifthenelse{\equal{}{\cmdemcpps@section@rhshort}}%
    {\cmdemcpps@section@short}%
    {\cmdemcpps@section@rhshort}%
}

% ----------------------toc contents ------------------------------------------
% The table of contents will place the short title in a box and the long title
% after it, prefer the "toc" options ("tocshort", "toclong") if those are set
% and otherwise using the basic options ("short", "long").
\newcommand{\emcppsShowTitleTocImpl}{%
  \makebox[12em][l]{\sf %
    \ifthenelse{\equal{}{\cmdemcpps@section@tocshort}}%
      {\cmdemcpps@section@short}%
      {\cmdemcpps@section@tocshort}%
  }%
  \ifthenelse{\equal{}{\cmdemcpps@section@toclong}}%
    {\cmdemcpps@section@long}%
    {\cmdemcpps@section@toclong}%
}

%------------------------------------------------------------------------------
% Section title rendering will delegate to the below two macros, which will be
% redefined in different contexts to point to different renderers.  These
% macros will be invoked with the option keys passed to the coresponding
% \emcppsFeatureSection command having already been processed.
\newcommand{\emcppsShowTitleImpl}{\emcppsShowTitleDefaultImpl{TITLE}}
\newcommand{\emcppsShowTitleAltImpl}{\emcppsShowTitleDefaultImpl{ALTTITLE}}

% Prior to calling the \section command this command should be invoked in order
% to configure the title and running head to pick up the proper options.  This
% is handled by \emcppsFeatureSection.
\newcommand{\emcppsSetShowTitleSection}{%
  \renewcommand{\emcppsShowTitleImpl}{\emcppsShowTitleSectionImpl}%
  \renewcommand{\emcppsShowTitleAltImpl}{\emcppsShowTitleRHImpl}%
}

% Prior to rendering the table of contents this macro should be invoked in order
% to have the toc pick up the proper options.
\newcommand{\emcppsSetShowTitleToc}{%
  \renewcommand{\emcppsShowTitleAltImpl}{\emcppsShowTitleTocImpl}%
}

% After the table of contents, or after all feature sections are finished, this
% macro should be invoked to be restore the impl commands have their default
% configurations.
\newcommand{\emcppsRevertShowTitle}{%
  \renewcommand{\emcppsShowTitleImpl}{\emcppsShowTitleDefaultImpl{TITLE}}%
  \renewcommand{\emcppsShowTitleAltImpl}{\emcppsShowTitleDefaultImpl{ALTTITLE}}%
}


% The below two commands will be passed to \section by \emcppsFeatureSection in
% order to render the section title in the appropriate context-sensitive manner.
\newcommand{\emcppsShowTitle}[1]{%
    \setkeys[emcpps]{section}{#1}
    \protect\emcppsShowTitleImpl
}
\newcommand{\emcppsShowTitleAlt}[1]{%
    \setkeys[emcpps]{section}{#1}
    \protect\emcppsShowTitleAltImpl
}


% Each feature should start with the below command, \emcppsFeature.
% The first argument is a list of key=value options, with the following keys
% supported:
%   short:       The short title of the section
%   long:        The long title of the section
%   tocshort, toclong: Override the short and long portions of the table of
%                      contents entry, if specified.
%   rhshort:     Override the running head short title, if specified.
%   headinglong: Override the long title to be placed in the heading, if
%                specified.
% The second argument is the label associated with the feature section.  This
% might potentially be used to generate other references within the section,
% so it should be unique within the book.
\newcommand{\emcppsFeature}[2]{%
  \setkeys[emcpps]{section}{#1}%
  \refstepcounter{section}%
  \newpage% 
  \emcppsSetShowTitleSection%
  \section[\protect\emcppsShowTitleAlt{#1}]%
          {\protect\emcppsShowTitle{#1}}%
  \label{#2}%
}

\makeatother
