% This file should contain the full glossary contents

\newglossaryentry{access specifier}{
  name={Access Specifier},
  description={
TODO
}}

\newglossaryentry{acquire/release memory barrier}{
  name={Acquire/Release Memory Barrier},
  description={
TODO
}}

\newglossaryentry{aggregate}{
  name={Aggregate},
  description={
TODO: from standard: An aggregate is an array or a class (Clause 9) with no user-provided constructors (12.1), [[no brace-or-equal-initializers for non-static data members (9.2)]]{PRIOR TO C++14}, no private or protected non-static data members (Clause 11), no base classes (Clause 10), and no virtual functions (10.3). --- Comments from Josh: *"This should be broken down a bit more per standard version... Specifically, 'no base classes' goes away in c++17, and it should be noted that there was no formal definition prior to c++11. It would also be useful to mention the related concepts of POD, standard-layout type, and literal types"*
}}

\newglossaryentry{aggregate initialization}{
  name={Aggregate Initialization},
  description={
TODO
}}

\newglossaryentry{alias template}{
  name={Alias Template},
  description={
TODO
}}

\newglossaryentry{alignment}{
  name={Alignment},
  description={
The alignment of an \emph{object type} is a \lstinline!std::size\_t! value (always a power of two) representing the number of bytes between successive addresses at which objects of this type can be allocated. It is the reason why \emph{padding} might be introduced between non-static members of a \lstinilne!class!.
}}

\newglossaryentry{architectural insulation}{
  name={Architectural Insulation},
  description={
TODO
}}

\newglossaryentry{as if}{
  name={As If},
  description={
TODO
}}

\newglossaryentry{atomic}{
  name={Atomic},
  description={
TODO
}}

\newglossaryentry{barriers}{
  name={Barriers},
  description={
TODO
}}

\newglossaryentry{basic source character set}{
  name={Basic Source Character Set},
  description={
TODO (SMD: the basic source character set is the abstract character set that must be available for expressing C++ source code. It's not the same as the source file encoding, which is what it is in C

    <!-- from unicode footnote: Implementations are free to map characters outside the basic source character set to sequences of its members, resulting in the possibility of embedding other characters (e.g. emojis) in a C++ source file. -->

    <!-- @SuperV1234 so are you saying that there are holes that can act as escape sequences?  So like 0xfe means the next character is in play and 0xff means the next 3 characters are in play?  Just don't get it. @SMD shift encodings? I'm not sure if we really want to go there?-->

    <!-- @SMD this is about the ways your editor and compiler can conspire to lie to you. A octets in a u8 literal are not interpreted as unicode. They are interpreted however the compiler interprets source in phase 1 of translation. The fact that the common encoding on Windows is 8 bit complete with no shift characters means that even though your editor will display 🍸, the compiler will see 0xF0 0x9F 0x8D 0xB8, which the compiler will emit as that sequence of bytes, even though the 0xBD is not canonically mapped. -->
)
}}

\newglossaryentry{boilerplate code}{
  name={Boilerplate Code},
  description={
TODO
}}

\newglossaryentry{brace elision}{
  name={Brace Elision},
  description={
TODO
}}

\newglossaryentry{bytes}{
  name={Bytes},
  description={
TODO
}}

\newglossaryentry{callable object}{
  name={Callable Object},
  description={
TODO
}}

\newglossaryentry{callback functions}{
  name={Callback Functions},
  description={
TODO
}}

\newglossaryentry{class member access expression}{
  name={Class Member Access Expression},
  description={
TODO (include any expression that is used to refer to a class member, such as \lstinline!object.member!, \lstinline!object->member!, \lstinline!object.*member!.)
}}

\newglossaryentry{closure}{
  name={Closure},
  description={
TODO
}}

\newglossaryentry{code bloat}{
  name={Code Bloat},
  description={
TODO
}}

\newglossaryentry{compile time coupling}{
  name={Compile Time Coupling},
  description={
TODO
}}

\newglossaryentry{compile time dispatch}{
  name={Compile Time Dispatch},
  description={
TODO
}}

\newglossaryentry{complete type}{
  name={Complete Type},
  description={
TODO
}}

\newglossaryentry{component}{
  name={Component},
  description={
TODO
}}

\newglossaryentry{component local}{
  name={Component Local},
  description={
TODO
}}

\newglossaryentry{components}{
  name={Components},
  description={
TODO
}}

\newglossaryentry{concepts}{
  name={Concepts},
  description={
TODO
}}

\newglossaryentry{conditionally supported}{
  name={Conditionally Supported},
  description={
A conforming implementation can choose not to support a feature that is specified as \emcppsglossgloss{conditionally supported}; if used and not supported, however, at least one diagnostic --- stating such lack of support --- is required.
}}

\newglossaryentry{constant expression}{
  name={Constant Expression},
  description={
TODO: An expression that can be evaluated at compile-time.  Mention \lstinline!constexpr! and state that \lstinline!const! variables that are initialized from a compile-time constants are themselves required to be compile-time constants.  New info to me in June 2020, worthwhile to have.
}}

\newglossaryentry{constant initialization}{
  name={Constant Initialization},
  description={
TODO
}}

\newglossaryentry{contextual conversion to bool}{
  name={Contextual Conversion to \lstinline!bool!},
  description={
A property of certain particular language construct (e.g., an \lstinilne!if! expression) that permits conversion from any expression \lstinline!E! to be treated \emcppsglossgloss{as if} a \lstinline!static_cast! to type bool had been applied --- e.g., \lstinline!if (E)! is definitional equivalent to \lstinline!if (static_cast<bool>(E))!. See also [explicit conversion operators]{.xref}.
}}

\newglossaryentry{contextual keyword}{
  name={Contextual Keyword},
  description={
A \emph{``contextual keyword''} is a special identifier that acts like a \emph{keyword} when used in particular contexts. \lstinline!override! is an example as it can be used as a regular identifier outside of member-function declarators.
}}

\newglossaryentry{cookie}{
  name={Cookie},
  description={
TODO
}}

\newglossaryentry{copy assignment operator}{
  name={Copy Assignment Operator},
  description={
TODO
}}

\newglossaryentry{copy constructor}{
  name={Copy Constructor},
  description={
TODO
}}

\newglossaryentry{copy operations}{
  name={Copy Operations},
  description={
TODO
}}

\newglossaryentry{copy semantics}{
  name={Copy Semantics},
  description={
An operation on two objects, a destination and a source, has \emcppsglossgloss{copy semantics}
if, after it completes, the \emcppsglossgloss{value} of the source object remains unchanged and the target object
has the same \emcppsglossgloss{value} as does the source.   ..  for some criteria the source and destination are
the same, for a value semantic type that property is value...
}}

\newglossaryentry{critical section}{
  name={Critical Section},
  description={
TODO
}}

\newglossaryentry{cyclic physical dependency}{
  name={Cyclic Physical Dependency},
  description={
TODO
}}

\newglossaryentry{cyclically dependent}{
  name={Cyclically Dependent},
  description={
TODO
}}

\newglossaryentry{data races}{
  name={Data Races},
  description={
TODO
}}

\newglossaryentry{declaration}{
  name={Declaration},
  description={
TODO
}}

\newglossaryentry{declare}{
  name={Declare},
  description={
TODO
}}

\newglossaryentry{declared type}{
  name={Declared Type},
  description={
The type of the \emcppsglossgloss{entity} named by the given expression.
}}

\newglossaryentry{default initialized}{
  name={Default Initialized},
  description={
TODO
}}

\newglossaryentry{define}{
  name={Define},
  description={
TODO
}}

\newglossaryentry{definition}{
  name={Definition},
  description={
TODO
}}

\newglossaryentry{delegating constructor}{
  name={Delegating Constructor},
  description={
TODO
}}

\newglossaryentry{delete}{
  name={Delete},
  description={
TODO
}}

\newglossaryentry{deleted function}{
  name={Deleted Function},
  description={
TODO
}}

\newglossaryentry{design pattern}{
  name={Design Pattern},
  description={
TODO
}}

\newglossaryentry{disambiguator}{
  name={Disambiguator},
  description={
TODO
}}

\newglossaryentry{dumb data}{
  name={Dumb Data},
  description={
TODO
}}

\newglossaryentry{embedded development}{
  name={Embedded Development},
  description={
TODO
}}

\newglossaryentry{embedded systems}{
  name={Embedded Systems},
  description={
TODO
}}

\newglossaryentry{encapsulation}{
  name={Encapsulation},
  description={
TODO
}}

\newglossaryentry{entity}{
  name={Entity},
  description={
One of the primary building blocks of a C++ program. An entity is one of: \emph{value}, \emph{object}, \emph{reference}, \emph{function}, \emph{enumerator}, \emph{type}, \emph{class member}, \emph{bit-field}, \emph{template}, \emph{template specialization}, \emph{namespace}, \emph{parameter pack}, or \lstinline!this!.
}}

\newglossaryentry{escalating}{
  name={Escalating},
  description={
TODO
}}

\newglossaryentry{excess n}{
  name={Excess $\mathbf{n}$},
  description={
TODO
}}

\newglossaryentry{explicit instantiation declaration}{
  name={Explicit Instantiation Declaration},
  description={
TODO
}}

\newglossaryentry{explicit instantiation definition}{
  name={Explicit Instantiation Definition},
  description={
TODO
}}

\newglossaryentry{explicit instantiation directive}{
  name={Explicit Instantiation Directive},
  description={
TODO
}}

\newglossaryentry{expression sfinae}{
  name={Expression SFINAE},
  description={
TODO
}}

\newglossaryentry{fences}{
  name={Fences},
  description={
TODO
}}

\newglossaryentry{flow of control}{
  name={Flow Of Control},
  description={
TODO
}}

\newglossaryentry{footprint}{
  name={Footprint},
  description={
TODO
}}

\newglossaryentry{forward class declaration}{
  name={Forward Class Declaration},
  description={
TODO
}}

\newglossaryentry{forward declaration}{
  name={Forward Declaration},
  description={
TODO - mention about enums that this was Impossible prior to C++11 as the compiler could not determine the underlying type without visibility of the enumerators.
}}

\newglossaryentry{forward declare}{
  name={Forward Declare},
  description={
TODO
}}

\newglossaryentry{forward declared}{
  name={Forward Declared},
  description={
TODO
}}

\newglossaryentry{free function}{
  name={Free Function},
  description={
TODO
}}

\newglossaryentry{fully constructed}{
  name={Fully Constructed},
  description={
TODO
}}

\newglossaryentry{function scope}{
  name={Function Scope},
  description={
TODO
}}

\newglossaryentry{fundamental integral type}{
  name={Fundamental Integral Type},
  description={
TODO
}}

\newglossaryentry{fundamental types}{
  name={Fundamental Types},
  description={
TODO
}}

\newglossaryentry{general purpose machines}{
  name={General Purpose Machines},
  description={
General-purpose machines are what Big Tech, the financial industry, and many other large companies use exclusively.
}}

\newglossaryentry{hide}{
  name={Hide or Hiding},
  alts={{hiding},
       },
  description={
Function-name \textbf{hiding} occurs when a member function in a derived class has the same name as one in the base class, but it is not overriding it due to a difference in the function signature or because the member function in the base class is not \texttt{virtual}. The hidden member function will \textbf{not} participate in dynamic dispatch; the member function of the base class will be invoked instead when invoked via a pointer or reference to the base class . The same code would have invoked the derived class's implementation had the member function of the base class had been \textbf{overridden} rather than \textbf{hidden}.
}}

\newglossaryentry{higher order function}{
  name={Higher Order Function},
  description={
TODO
}}

\newglossaryentry{IFNDR}{
  name={IFNDR},
  description={
see Ill-formed, No Diagnostic Required
}}

\newglossaryentry{id expression}{
  name={Id Expression},
  description={
TODO
}}

\newglossaryentry{ill formed}{
  name={Ill Formed},
  description={
TODO (`[temp.res]p8`). The program is not valid C++. Generally, the compiler is required to produce a diagnostic (error) message, but see [^ndr](.glossary).
}}

\newglossaryentry{ill formed, no diagnostic required}{
  name={Ill-Formed, No Diagnostic Required},
  description={
TODO *No diagnostic required*. The compiler is not mandated by the Standard to produce a diagnostic and the behavior of the code is *undefined*.
}}

\newglossaryentry{implementation defined}{
  name={Implementation Defined},
  description={
TODO
}}

\newglossaryentry{instantiation time}{
  name={Instantiation Time},
  description={
TODO
}}

\newglossaryentry{insulate}{
  name={Insulate},
  description={
TODO
}}

\newglossaryentry{insulating}{
  name={Insulating},
  description={
TODO
}}

\newglossaryentry{insulation}{
  name={Insulation},
  description={
TODO
}}

\newglossaryentry{integral promotion}{
  name={Integral Promotion},
  description={
TODO
}}

\newglossaryentry{integral type}{
  name={Integral Type},
  description={
TODO
}}

\newglossaryentry{join}{
  name={Join},
  description={
TODO
}}

\newglossaryentry{lambda expressions}{
  name={Lambda Expressions},
  description={
TODO
}}

\newglossaryentry{leaks}{
  name={Leaks},
  description={
TODO
}}

\newglossaryentry{linkage}{
  name={Linkage},
  description={
TODO
}}

\newglossaryentry{literal type}{
  name={Literal Type},
  description={
TODO
}}

\newglossaryentry{local declaration}{
  name={Local Declaration},
  description={
TODO
}}

\newglossaryentry{local scope}{
  name={Local Scope},
  description={
TODO
}}

\newglossaryentry{logical}{
  name={Logical},
  description={
TODO
}}

\newglossaryentry{mantissa}{
  name={Mantissa},
  description={
TODO
}}

\newglossaryentry{mechanism}{
  name={Mechanism},
  description={
TODO
}}

\newglossaryentry{mechanisms}{
  name={Mechanisms},
  description={
TODO
}}

\newglossaryentry{member}{
  name={Member},
  description={
TODO
}}

\newglossaryentry{member initializer list}{
  name={Member Initializer List},
  description={
TODO
}}

\newglossaryentry{memory leak}{
  name={Memory Leak},
  description={
TODO
}}

\newglossaryentry{message}{
  name={Message},
  description={
TODO
}}

\newglossaryentry{messenger}{
  name={Messenger},
  description={
TODO
}}

\newglossaryentry{meyers singleton}{
  name={Meyers Singleton},
  description={
TODO
}}

\newglossaryentry{move operations}{
  name={Move Operations},
  description={
TODO
}}

\longnewglossaryentry{move semantics}{
  name={Move Semantics},
  description={
An operation on two objects, a destination and a source, has \emcppsglossgloss{move semantics}
if, after it completes, the target object has the same \emcppsglossgloss{value} that the source object had before
the operation began.  Note that the source object may be modified, and is left in an unspecified state after
the operation completes.

Also note that any operation that has \emcppsglossgloss{copy semantics} also has \emcppsglossgloss{move semantics}.
}}

\newglossaryentry{multithreading context}{
  name={Multithreading Context},
  description={
TODO
}}

\newglossaryentry{nibbles}{
  name={Nibbles},
  description={
TODO
}}

\newglossaryentry{non trivial constructor}{
  name={Non Trivial Constructor},
  description={
TODO
}}

\newglossaryentry{non trivial special member function}{
  name={Non Trivial Special Member Function},
  description={
TODO
}}

\newglossaryentry{null address}{
  name={Null Address},
  description={
TODO
}}

\newglossaryentry{one definition rule}{
  name={One Definition Rule},
  description={
TODO
}}

\newglossaryentry{opaque declaration}{
  name={Opaque Declaration},
  description={
TODO
}}

\newglossaryentry{opaque enumeration}{
  name={Opaque Enumeration},
  description={
TODO
}}

\newglossaryentry{overload resolution}{
  name={Overload Resolution},
  description={
TODO
}}

\newglossaryentry{POD}{
  name={POD},
  description={
TODO
}}

\newglossaryentry{partially constructed}{
  name={Partially Constructed},
  description={
TODO
}}

\newglossaryentry{physical}{
  name={Physical},
  description={
TODO
}}

\newglossaryentry{physical design}{
  name={Physical Design},
  description={
TODO
}}

\newglossaryentry{placement new}{
  name={Placement New},
  description={
TODO
}}

\newglossaryentry{precondition}{
  name={Precondition},
  description={
TODO
}}

\newglossaryentry{predicate function}{
  name={Predicate Function},
  description={
TODO
}}

\newglossaryentry{pure function}{
  name={Pure Function},
  description={
A function that produces no side effects and always returns the same value given the same sequence of argument values.
}}

\newglossaryentry{RAII}{
  name={RAII},
  description={
See Resource Acquisiition is Initialization
}}

\newglossaryentry{range}{
  name={Range},
  description={
TODO
}}

\newglossaryentry{receiver}{
  name={Receiver},
  description={
TODO
}}

\newglossaryentry{redundant check}{
  name={Redundant Check},
  description={
TODO
}}

\newglossaryentry{reference type}{
  name={Reference Type},
  description={
TODO
}}

\newglossaryentry{return type deduction}{
  name={Return Type Deduction},
  description={
TODO
}}

\newglossaryentry{SFINAE}{
  name={SFINAE},
  description={
See Substitution Failure Is Not An Error
}}

\newglossaryentry{safe bool idiom}{
  name={Safe bool idiom},
  description={
A technique in class design that suppresses unwanted comparisons made available by the presence of a non-\lstinline!explicit! conversion function to \lstinline!bool!. See also [explicit conversion operators]{.xref}. <!-- TODO: "Given a class \lstinline!C!, the idiom requires \lstinilne!C! to provide a (implicit in C++03) conversion operator to return a type that is \emph{contextually convertible to \lstinline!bool!} (e.g., any integral or pointer type) but does not support arithmetic (e.g., a \emcppsglossgloss{pointer to member}), and the presence of non-member template comparison functions that invokes a \lstinline!private! (unimplemented) member function, purposefully causing a compile-time error, should two objects of \lstinline!C! ever be compared -- e.g., using equality or relational operations (valid on \emph{all} pointers types); see [Karlsson04]{.ref}. An early version of a return type designed for this purpose, known as \lstinline!unspecified_bool_type! was implemented in Boost; see [semashev07]{.ref}." -->
}}

\newglossaryentry{section}{
  name={Section},
  description={
TODO
}}

\newglossaryentry{sections}{
  name={Sections},
  description={
TODO
}}

\newglossaryentry{sender}{
  name={Sender},
  description={
TODO
}}

\newglossaryentry{side effects}{
  name={Side Effects},
  description={
TODO
}}

\newglossaryentry{signature}{
  name={Signature},
  description={
TODO
}}

\newglossaryentry{signed integer overflow}{
  name={Signed Integer Overflow},
  description={
TODO
}}

\newglossaryentry{special functions}{
  name={Special Functions},
  description={
TODO
}}

\newglossaryentry{special member function}{
  name={Special Member Function},
  description={
As per the C++17 standard, the following are considered special member functions: \emph{destructors}, \emph{default constructors}, \emph{copy constructors}, \emph{move constructors}, \emph{copy assignment operators}, and \emph{move assignment operators}.
}}

\newglossaryentry{standard conversion}{
  name={Standard Conversion},
  description={
TODO
}}

\newglossaryentry{static assertion declarations}{
  name={Static Assertion Declarations},
  description={
TODO
}}

\newglossaryentry{static duration}{
  name={Static Duration},
  description={
TODO
}}

\newglossaryentry{string literal}{
  name={String Literal},
  description={
TODO
}}

\newglossaryentry{strong typedef idiom}{
  name={Strong Typedef Idiom},
  description={
TODO
}}

\newglossaryentry{structural inheritance}{
  name={Structural Inheritance},
  description={
TODO
}}

\newglossaryentry{sum type}{
  name={Sum Type},
  description={
Abstract data type allowing the representation of one of multiple possible alternative types. Each alternative has its own type (and state), and only one alternative can be ``active" at any given point in time. Sum types automatically keep track of which choice is ``active,'' and properly implement value-sematic special member functions (even for non-trivial types). They can be implemented efficiently as a C++ \texttt{class} using a C++ \texttt{union} and a separate (integral) discriminator. This sort of implementation is commonly referred to as a discriminating (or ``tagged'') union.
}}

\newglossaryentry{template head}{
  name={Template Head},
  description={
TODO
}}

\newglossaryentry{template instantiation time}{
  name={Template Instantiation Time},
  description={
TODO
}}

\newglossaryentry{translation unit}{
  name={Translation Unit},
  description={
TODO
}}

\newglossaryentry{trivial}{
  name={Trivial},
  description={
TODO
}}

\newglossaryentry{trivial operation}{
  name={Trivial Operation},
  description={
TODO
}}

\newglossaryentry{trivial types}{
  name={Trivial Types},
  description={
TODO
}}

\newglossaryentry{trivially constructible}{
  name={Trivially Constructible},
  description={
TODO
}}

\newglossaryentry{trivially copy constructor}{
  name={Trivially Copy Constructor},
  description={
The requirements for the copy constructor of a class \lstinilne!T! to be considered trivial are as follows:

  - It must not be user-provided;
  - If defaulted, its signature must be the same as implicitly-defined (exception specification and qualifiers on the argument must match);
  - \lstinline!T! must have no \lstinline!virtual! member functions or \lstinline!virtual! base classes;
  - The copy constructor for every single data member and base class of \lstinline!T! must be trivial (recursively).
}}

\newglossaryentry{trivially copyable}{
  name={Trivially Copyable},
  description={
TODO
}}

\newglossaryentry{type alias}{
  name={Type Alias},
  description={
TODO
}}

\newglossaryentry{type traits}{
  name={Type Traits},
  description={
TODO
}}

\newglossaryentry{UDT}{
  name={UDT},
  description={
TODO
}}

\newglossaryentry{undefined}{
  name={Undefined},
  description={
TODO
}}

\newglossaryentry{undefined behavior}{
  name={Undefined Behavior},
  description={
TODO
}}

\newglossaryentry{undefined symbols}{
  name={Undefined Symbols},
  description={
TODO
}}

\newglossaryentry{underlying type}{
  name={Underlying Type},
  description={
TODO
}}

\newglossaryentry{unevaluated contexts}{
  name={Unevaluated Contexts},
  description={
TODO
}}

\newglossaryentry{user defined type}{
  name={User Defined Type},
  description={
TODO
}}

\newglossaryentry{user provided}{
  name={User Provided},
  description={
TODO
}}

\longnewglossaryentry{user provided special member function}{
  name={User Provided Special Member Function},
  description={
In the C++11 standard, a special member function is considered \emcppsglossgloss{user-provided} if it has been explicitly declared by the programmer and not explicitly defaulted or deleted in its first declaration:

%begin{emcppslisting}
%struct S0
%{
%    S0(); // User-provided
%};
%
%struct S1
%{
%    S1() = default; // Not user-provided
%};
%\end{emcppslisting}

One of the requirement for a special member function to be considered \emcppsglossgloss{trivial} is that it is not user-provided. Trivial classes (i.e. with all special member function being trivial) have special semantics (e.g. they can be safely used with \lstinline!std::memcpy!, or copied into a suitable array of \lstinline!char! and back). See [Generalized PODs]{.xref} for more information.
}}

\newglossaryentry{user provided special member functions}{
  name={User Provided Special Member Functions},
  description={
TODO
}}

\newglossaryentry{value}{
  name={Value},
  description={
TODO
}}

\newglossaryentry{value category}{
  name={Value Category},
  description={
TODO
}}

\newglossaryentry{value constructor}{
  name={Value Constructor},
  description={
TODO
}}

\newglossaryentry{value semantic}{
  name={Value Semantic},
  description={
TODO
}}

\newglossaryentry{vocabulary type}{
  name={Vocabulary Type},
  description={
TODO
}}

\newglossaryentry{zero initialized}{
  name={Zero Initialized},
  description={
TODO
}}
