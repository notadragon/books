\emcppsFeature{
        short={Local Types '11},
        long={Local/Unnamed Types as Template Arguments},
}{local-types-as-template-arguments}

C++11 allows function-scope and unnamed types to be used as template arguments.

\subsection[Description]{Description}\label{description}

Historically, types without \romeogloss{linkage} (i.e., local and unnamed
types) were forbidden as template arguments due to implementability
concerns using the compiler technology available at that
time.{\cprotect\footnote{TODO: Alisdair}} Modern C++ lifts this
restriction, making use of local or unnamed types consistent with
nonlocal, named ones, thereby obviating the need to gratuitously name
or enlarge the scope of a type.

\begin{lstlisting}[language=C++]
template <typename T>
void f(T) { };            // function template

template <typename T>
class C { };              // class template

struct { } obj;           // object (ù{\codeincomments{obj}}ù) of unnamed C++ type

void g()
{
    struct S { };         // local type

    f(S());               // OK in C++11; was error in C++03
    f(obj);               // OK in C++11; was error in C++03

    C<S>             cs;  // OK in C++11; was error in C++03
    C<decltype(obj)> co;  // OK in C++11; was error in C++03
}
\end{lstlisting}

\noindent Notice that we have used the \emph{\lstinline!decltype!
keyword} (see \featureref{\locationa}{decltype}) to extract the unnamed type of the object \lstinline!obj!.

These new relaxed rules for template arguments are essential to the
ergonomics of \emph{lambda expressions} (see \featureref{\locationc}{lambda}), as such types are both
unnamed and local in typical usage:

\begin{lstlisting}[language=C++]
#include <algorithm> // (ù{\codeincomments{std::sort}}ù)
#include <string>
#include <vector>

struct Person { std::string d_name; };

void sortByName(std::vector<Person>& people)
{
    std::sort(people.begin(), people.end(),
              [](const Person& lhs, const Person& rhs)
              {
                  return lhs.d_name < rhs.d_name;
              });
}
\end{lstlisting}

\noindent In the example above, the lambda expression passed to the
\lstinline!std::sort! algorithm is a local unnamed type, and the algorithm
itself is a function template.

\subsection[Use Cases]{Use Cases}\label{use-cases}

\subsubsection[Encapsulating a type within a function]{Encapsulating a type within a function}\label{encapsulating-a-type-within-a-function}

Limiting the scope and visibility of an \romeogloss{entity} to the body of a
function actively prevents its direct use, even when the function body
is exposed widely --- say, as an \lstinline!inline! function or function
template defined within a header file.

Consider, for instance, an implementation of Dijkstra's algorithm that
uses a local type to keep track of metadata for each vertex in the input
graph:

\begin{lstlisting}[language=C++]
// dijkstra.h

inline int dijkstra(std::vector<Vertex>* path, const Graph& graph)
{
    struct VertexMetadata         // implementation-specific helper class
    {
        int  d_distanceFromSource;
        bool d_inShortestPath;
    };

    std::vector<VertexMetadata> vertexMetadata(graph.numNodes());
        // standard vector of local (ù{\codeincomments{VertexMetadata}}ù) objects -- one per vertex

    // ... (body of algorithm)
}
\end{lstlisting}

\noindent Defining \lstinline!VertexMetadata! outside of the body of
\lstinline!dijkstra! --- e.g., to comply with C++03 restrictions --- would
make that implementation-specific helper class directly accessible to
anyone including the \lstinline!dijkstra.h! header file. As Hyrum's
law{\cprotect\footnote{``With a sufficient number
of users of an API, it does not matter what you promise in the
contract: all observable behaviors of your system will be depended on
  by somebody'': see \cite{wight}.}} suggests, if the
imple\-men\-tation-specific \lstinline!VertexMetadata! detail is defined
outside the function body, it is to be expected that some user somewhere
will depend on it in its current form, making it problematic, if not
impossible, to change.\footnote{The C++20 \emph{modules} facility enables the
encapsulation of helper types (such as \lstinline!metadata! in the \mbox{\lstinline!dijkstra.h!} example on this page) used in
the implementation of other locally defined types or functions, even
when the helper types appear at namespace scope within the module.} Conversely, encapsulating
the type within the function body avoids unintended use by clients,
while improving human cognition by colocating the definition of the
type with its sole purpose.{\cprotect\footnote{For a detailed
  discussion of malleable  versus stable software, see \cite{{lakos20}},
  section 0.5, pp.~29-43.}}

\subsubsection[Instantiating templates with local function objects as type arguments]{Instantiating templates with local function objects as type arguments}\label{instantiating-templates-with-local-function-objects-as-type-arguments}

Suppose that we have a program that makes wide use of an aggregate data
type, \lstinline!City!:

\begin{lstlisting}[language=C++]
#include <iostream>
#include <iterator>
#include <set>
#include <vector>

struct City
{
    int         d_uniqueId;
    std::string d_name;
    std::ostream& operator<<(std::ostream& stream,
                         const City&   object);
};
\end{lstlisting}

\noindent Consider now the task of writing a function to print unique elements of
an\linebreak[4] \lstinline!std::vector<City>!, ordered by name:

\begin{lstlisting}[language=C++]
void printUniqueCitiesOrderedByName(const std::vector<City>& cities)
{
    struct OrderByName
    {
        bool operator()(const City& lhs, const City& rhs) const
        {
            return lhs.d_name < rhs.d_name;
                // increasing order (subject to change)
        }
    };

    const std::set<City, OrderByName> tmp(cities.begin(), cities.end());

    std::copy(tmp.begin(), tmp.end(),
              std::ostream_iterator<City>(std::cout, "\n"));
}
\end{lstlisting}

\noindent Absent reasons to make the \lstinline!OrderByName!
function object more generally available, rendering its definition
alongside the one place where it is used{\cprotect\footnote{Note that using a lambda (see \featureref{\locationc}{lambda}) in such scenario is cumbersome:
\begin{lstlisting}[basicstyle=\ttfamily\footnotesize]
auto compare = [](const City& lhs, const City& rhs) {
    return lhs.d_name < rhs.d_name;
};
const std::set<City, decltype(compare)>
    tmp(cities.begin(), cities.end(), compare);
\end{lstlisting}}} --- i.e., directly within
function scope --- again enforces and readily communicates its tightly
encapsulated (and therefore \emph{malleable}) status.

\subsubsection[Configuring algorithms via lambda expressions]{Configuring algorithms via lambda expressions}\label{configuring-algorithms-via-lambda-expressions}

Suppose we are representing a 3D environment using a \emph{scene
graph}{\cprotect\footnote{A \emph{scene graph} data structure, commonly
used in computer games and 3D-modeling software, represents the
  logical and spatial hierarchy of objects in a scene.}} and managing
the graph's nodes via an \lstinline!std::vector! of \lstinline!SceneNode!
objects. Our \lstinline!SceneNode! class supports a variety of
\lstinline!const! member functions used to query its status (e.g.,
\lstinline!isDirty! and \lstinline!isNew!). Our task is to implement a
\romeogloss{predicate function}, \lstinline!mustRecalculateGeometry!, that
returns \lstinline!true! if and only if at least one of the nodes is either
``dirty'' or ``new.''

These days, we might reasonably elect to implement this functionality
using the (C++11) standard algorithm
\lstinline!std::any_of!{\cprotect\footnote{\cite{cppreferencea}}}:

\begin{lstlisting}[language=C++]
template <typename InputIterator, typename UnaryPredicate>
bool any_of(InputIterator first, InputIterator last, UnaryPredicate pred);
    // Return (ù{\codeincomments{true}}ù) if any of the elements in the range satisfies (ù{\codeincomments{pred}}ù).
\end{lstlisting}

\noindent Prior to C++11, however, use of a function template, such as
\lstinline!any_of!, would have required a separate function or
function object (defined \emph{outside} of the scope of the function):

\begin{lstlisting}[language=C++]
// C++03 (obsolete)
namespace {

struct IsNodeDirtyOrNew
{
    bool operator()(const SceneNode& node) const
    {
        return node.isDirty() || node.isNew();
    }
};

}  // close unnamed namespace

bool mustRecalculateGeometry(const std::vector<SceneNode>& nodes)
{
    return any_of(nodes.begin(), nodes.end(), IsNodeDirtyOrNew());
}
\end{lstlisting}

\noindent Because unnamed types can serve as arguments to this function template, we can also employ a lambda expression instead of a function object that would be required in C++03:

\begin{lstlisting}[language=C++]
#include <algorithm> // (ù{\codeincomments{'std::any\_of'}}ù)
bool mustRecalculateGeometry(const std::vector<SceneNode>& nodes)
{
    return std::any_of(nodes.begin(),             // start of range
                       nodes.end(),               // end of range
                       [](const SceneNode& node)  // lambda expression
                       {
                           return node.isDirty() || node.isNew();
                       }
                      );
}
\end{lstlisting}

\noindent By creating a \romeogloss{closure} of unnamed type via a lambda
expression, unnecessary boilerplate, excessive scope, and even local
symbol visibility are avoided.

\subsection[Potential Pitfalls]{Potential Pitfalls}\label{potential-pitfalls}

None so far

\subsection[Annoyances]{Annoyances}\label{annoyances}

None so far

\subsection[See Also]{See Also}\label{see-also}

\begin{itemize}
\item{\seealsoref{lambda}{\locationc}Conditionally safe C++11 feature providing strong practical motivation for the relaxations discussed here}
\item{\seealsoref{decltype}{\locationa}Safe C++11 feature that allows developers to query the type of any expression or entity, including objects with unnamed types}
\end{itemize}

\subsection[Further Reading]{Further Reading}\label{further-reading}

None so far



