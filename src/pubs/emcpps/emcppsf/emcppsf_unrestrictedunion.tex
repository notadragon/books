\newpage
\section[{\tt union} '11]{Unions Having Non-Trivial Members\sectionmark{{\RHCode union}~'11}}\label{unrestricted-unions}\sectionmark{{\RHCode union}~'11}
%%%%% copyedits in and proofed

Any nonreference type is permitted to be a member of a \lstinline!union!.

\subsection[Description]{Description}\label{unrestrictedunion-description}

Prior to C++11, only \romeogloss{trivial types} --- e.g.,
\romeogloss{fundamental types}, such as \lstinline!int! and \lstinline!double!,
enumerated or pointer types, or a C-style array or \lstinline!struct!
(a.k.a. a \romeogloss{POD}) --- were allowed to be members of a
\lstinline!union!. This limitation prevented any user-defined type having
a \romeogloss{non-trivial special member function} from being a member of a
\lstinline!union!:

\begin{emcppshiddenlisting}[emcppsbatch=e1]
#include <string>
\end{emcppshiddenlisting}
\begin{emcppslisting}[emcppsbatch=e1]
union U0
{
    int         d_i;  // OK
    std::string d_s;  // compile-time error in C++03 (OK as of C++11)
};
\end{emcppslisting}

\noindent C++11 relaxes such restrictions on \lstinline!union! members, such as
\lstinline!d_s! above, allowing any type other than a \romeogloss{reference
type} to be a member of a \lstinline!union!.

A \lstinline!union! type is permitted to have user-defined special member
functions but --- by design --- does not initialize any of its members
automatically. Any member of a \lstinline!union! having a
\romeogloss{non-trivial constructor}, such as \lstinline!struct!~\lstinline!Nt!
below, must be constructed manually (e.g., via \romeogloss{placement
\lstinline!new!}) before it can be used:

\begin{emcppslisting}[emcppsbatch=e2]
struct Nt  // used as part of a (ù{\codeincomments{union}}ù) (below)
{
    Nt();   // non-trivial default constructor
    ~Nt();  // non-trivial destructor

    // Copy construction and assignment are implicitly defaulted.
    // Move construction and assignment are implicitly deleted.
};
\end{emcppslisting}

\noindent As an added safety measure, any non-trivial \romeogloss{special member
function} defined --- either implicitly or explicitly --- for any
\romeogloss{member} of a \lstinline!union! results in the compiler implicitly
deleting (see %Section~\ref{deleted-functions},
``\titleref{deleted-functions}" on page~\pageref{deleted-functions}) the corresponding \romeogloss{special
member function} of the \lstinline!union! itself:

\begin{emcppslisting}[emcppsbatch=e2]
union U1
{
    int d_i;   // fundamental type having all trivial special member functions
    Nt  d_nt;  // user-defined type having non-trivial special member functions

    // Implicitly deleted special member functions of (ù{\codeincomments{U1}}ù):
    /*
        U1()                     = delete; // due to explicit (ù{\codeincomments{Nt::Nt()}}ù)
        U1(const U1&)            = delete; // due to implicit (ù{\codeincomments{Nt::Nt(const Nt\&)}}ù)
        ~U1()                    = delete; // due to explicit (ù{\codeincomments{Nt::}}ù)~(ù{\codeincomments{Nt()}}ù)
        U1& operator=(const U1&) = delete; // due to implicit
                                           // (ù{\codeincomments{Nt::operator=(const Nt\&)}}ù)
    */
};
\end{emcppslisting}

%\noindent This same sort of precautionary deletion also occurs for any class
%containing such a union as a data member (see {\it\titleref{unrestrictedunion-use-cases}: \titleref{implementing-a-sum-type-as-a-discriminating-(or-tagged)-union}} on page~\pageref{implementing-a-sum-type-as-a-discriminating-(or-tagged)-union}).

A special member function of a \lstinline!union! that is implicitly deleted
can be restored via explicit declaration, thereby forcing a programmer
to consider how non-trivial members should be managed. For example,
we can start providing a \emph{value constructor} and corresponding
\emph{destructor}:

%begin{emcppslisting}
%struct U2
%{
%    union
%    {
%        int  d_i;   // fundamental type (trivial)
%        Nt   d_nt;  // non-trivial user-defined type
%    };
%
%    bool d_useInt;  // discriminator
%
%    U2(bool useInt) : d_useInt(useInt)       // value constructor
%    {
%        if (d_useInt) { new (&d_i) int(); }  // value initialized (to (ù{\codeincomments{0}}ù))
%        else          { new (&d_nt) Nt(); }  // default constructed in place
%    }
%
%    ~U2()  // destructor
%    {
%        if (!d_useInt) { d_nt.~Nt(); }
%    }
%};
%\end{emcppslisting}
\begin{emcppslisting}[emcppsbatch=e2]
#include <new>

struct U2
{
    union
    {
        int  d_i;   // fundamental type (trivial)
        Nt   d_nt;  // non-trivial user-defined type
    };

    bool d_useInt;  // discriminator

    U2(bool useInt) : d_useInt(useInt)
    {
        if (d_useInt) { new (&d_i) int(); }  // value initialized (to (ù{\codeincomments{0}}ù))
        else          { new (&d_nt) Nt(); }  // default constructed in place
    }

    ~U2()  // destructor
    {
        if (!d_useInt) { d_nt.~Nt(); }
    }
};
\end{emcppslisting}


\noindent Notice that we have employed \romeogloss{placement \lstinline!new!} syntax to
control the lifetime of both member objects. Although assignment would
be permitted for the trivial \lstinline!int! type, it would be
\romeogloss{undefined behavior} for the non-trivial \lstinline!Nt! type:

%begin{emcppslisting}
%U2(bool useInt) : d_useInt(useInt)  // value constructor
%{
%    if (d_useInt) { d_i = int(); }  // value initialized (to (ù{\codeincomments{0}}ù))
%    else          { d_nt = Nt(); }  // undefined behavior
%}
%\end{emcppslisting}
\begin{emcppshiddenlisting}[emcppsbatch=e3]
// duplicate the requirements for U2 so that we can show an alternate version
// of the constructor
struct Nt  // used as part of a (ù{\codeincomments{union}}ù) (below)
{
    Nt();   // non-trivial default constructor
    ~Nt();  // non-trivial destructor
};
struct U2
{
    union
    {
        int  d_i;   // fundamental type (trivial)
        Nt   d_nt;  // non-trivial user-defined type
    };

    bool d_useInt;  // discriminator
\end{emcppshiddenlisting}
\begin{emcppslisting}[emcppsbatch=e3]
    U2(bool useInt) : d_useInt(useInt)
    {
        if (d_useInt) { d_i = int(); }  // value initialized (to (ù{\codeincomments{0}}ù))
        else          { d_nt = Nt(); }  // BAD IDEA: undefined behavior (no
                                        // lhs object)
    }
\end{emcppslisting}
\begin{emcppshiddenlisting}[emcppsbatch=e3]
};
\end{emcppshiddenlisting}


\noindent Now if we were to try to copy-construct or assign one object of type
\lstinline!U2! to another, the operation would fail because we have not
yet specifically addressed those \romeogloss{special member functions}:

\begin{emcppslisting}[emcppsbatch=e2]
void f()
{
    U2 a(false), b(true);  // OK (construct both instances of (ù{\codeincomments{U2}}ù))
    U2 c(a);               // Error, no (ù{\codeincomments{U2(const U2\&)}}ù)
    a = b;                 // Error, no (ù{\codeincomments{U2\& operator=(const U2\&)}}ù)
}
\end{emcppslisting}

\noindent We can restore these implicitly deleted special member functions too,
simply by adding appropriate copy-constructor and assignment-operator
definitions for \lstinline!U2! explicitly:

%begin{emcppslisting}
%union U2
%{
%    // ... (everything in (ù{\codeincomments{U2}}ù) above)
%
%    U2(const U2& original) : d_useInt(original.d_useInt)
%    {
%        if (d_useInt) { new (&d_i) int(original.d_i);  }
%        else          { new (&d_nt) Nt(original.d_nt); }
%    }
%
%    U2& operator=(const U2& rhs)
%    {
%        if (this == &rhs) // Prevent self-assignment.
%        {
%            return *this;
%        }
%
%        // Resolve all possible combinations of active types between the
%        // left-hand side and right-hand side of the assignment:
%
%        if (d_useInt)
%        {
%            if (rhs.d_useInt) { d_i = rhs.d_i; }
%            else              { new (&d_nt) Nt(rhs.d_nt); }
%        }
%        else
%        {
%            if (rhs.d_useInt) { d_nt.~Nt(); new (&d_i) int(rhs.d_i); }
%            else              { d_nt = rhs.d_nt; }
%        }
%
%        return *this;
%    }
%};
%\end{emcppslisting}
\begin{emcppshiddenlisting}[emcppsbatch=e4]
#include <new>
// duplicate the requirements for U2 so that we can show an alternate version
// of the constructor
struct Nt  // used as part of a (ù{\codeincomments{union}}ù) (below)
{
    Nt();   // non-trivial default constructor
    ~Nt();  // non-trivial destructor
};
// --- Replace
// ... (everything in (ù{\codeincomments{U2}}ù) above)
    union
    {
        int  d_i;   // fundamental type (trivial)
        Nt   d_nt;  // non-trivial user-defined type
    };

    bool d_useInt;  // discriminator    
// --- End
\end{emcppshiddenlisting}
\begin{emcppslisting}[emcppsbatch=e4]
class U2
{
    // ... (everything in (ù{\codeincomments{U2}}ù) above)

    U2(const U2& original) : d_useInt(original.d_useInt)
    {
        if (d_useInt) { new (&d_i) int(original.d_i);  }
        else          { new (&d_nt) Nt(original.d_nt); }
    }

    U2& operator=(const U2& rhs)
    {
        if (this == &rhs) // Prevent self-assignment.
        {
            return *this;
        }

        // Resolve all possible combinations of active types between the
        // left-hand side and right-hand side of the assignment:

        if (d_useInt)
        {
            if (rhs.d_useInt) { d_i = rhs.d_i; }
            else              { new (&d_nt) Nt(rhs.d_nt); }  // (ù{\codeincomments{int}}ù) DTOR trivial
        }
        else
        {
            if (rhs.d_useInt) { d_nt.~Nt(); new (&d_i) int(rhs.d_i); }
            else              { d_nt = rhs.d_nt; }
        } d_useInt = rhs.d_useInt;

        // Resolve all possible combinations of active types between the
        // left-hand side and right-hand side of the assignment.  Use the
        // corresponding assignment operator when they match; otherwise,
        // if the old member is (ù{\codeincomments{d\_nt}}ù), run its non-trivial destructor, and
        // then copy-construct the new member in place:

        return *this;
    }
};
\end{emcppslisting}
Note that in the code example above, we ignore exceptions for exposition simplicity. Note also that attempting to
restore a \lstinline!union!'s implicitly deleted special member
functions by using the \lstinline!=!~\lstinline!default! syntax (see
%``\titleref{Defaulted-Special-Member-Functions}" on page~\pageref{Defaulted-Special-Member-Functions}
\featureref{\locationa}{Defaulted-Special-Member-Functions}) will still result in their being deleted because
the compiler cannot know which member of the union is active.

\subsection[Use Cases]{Use Cases}\label{unrestrictedunion-use-cases}

\subsubsection[Implementing a \romeogloss{sum type} as a discriminated {\tt union}]{Implementing a sum type as a discriminated {\SubsubsecCode union}}\label{implementing-a-sum-type-as-a-discriminating-(or-tagged)-union}

A \romeogloss{sum type} is an algebraic data type that provides a choice
among a fixed set of specific types. A C++11 unrestricted union can serve as a convenient and efficient way to define storage for a sum type (also called a \emph{tagged} or \emph{discriminated} union) because the alignment and size calculations are performed automatically by the compiler.

As an example, consider writing a parsing function \lstinline!parseInteger!
that, given a\linebreak[4] \lstinline!std::string! \lstinline!input!, will return, as a
\romeogloss{sum type} \lstinline!ParseResult! (see below), containing either an
\lstinline!int! result (on success) or an informative error message
on failure:

\begin{emcppshiddenlisting}[emcppsbatch=e5]
#include <sstream>
#include <string>
struct ParseResult {
    explicit ParseResult(int values);
    explicit ParseResult(const std::string& error);
};

// --- Replace
    if (/* Failure case (1). */)
    if (true)
// --- End

// --- Replace
    if (/* Failure case (2). */)
    if (true)
// --- End

\end{emcppshiddenlisting}
\begin{emcppslisting}[emcppsbatch=e5]
ParseResult parseInteger(const std::string& input)  // Return a sum type.
{
    int result;     // accumulate (ù{\codeincomments{result}}ù) as we go
    std::size_t i;  // current character index

    // ...

    if (/* Failure case (1). */)
    {
        std::ostringstream oss;
        oss << "Found non-numerical character '" << input[i]
            << "' at index '" << i << "'.";

        return ParseResult(oss.str());
    }

    if (/* Failure case (2). */)
    {
        std::ostringstream oss;
        oss << "Accumulating '" << input[i]
            << "' at index '" << i
            << "' into the current running total '" << result
            << "' would result in integer overflow.";

        return ParseResult(oss.str());
    }

    // ...

    return ParseResult(result);  // Success!
}
\end{emcppslisting}

\noindent The implementation above relies on \lstinline!ParseResult! being able to
hold a value of type either \lstinline!int! or \lstinline!std::string!. By
encapsulating a C++ \lstinline!union! and a \emph{discriminator} as part
of the \lstinline!ParseResult! \romeogloss{sum type}, we can achieve the
desired semantics:

%begin{emcppslisting}
%class ParseResult
%{
%    union  // storage for either the result or the error
%    {
%        int         d_value;  // trivial result type
%        std::string d_error;  // non-trivial error type
%    };
%
%    bool d_isError;  // discriminator
%
%public:
%    explicit ParseResult(int value);                 // value constructor (1)
%    explicit ParseResult(const std::string& error);  // value constructor (2)
%
%    ParseResult(const ParseResult& rhs);             // copy constructor
%    ParseResult& operator=(const ParseResult& rhs);  // copy assignment
%
%    ~ParseResult();                                  // destructor
%};
%\end{emcppslisting}
\begin{emcppshiddenlisting}[emcppsbatch=e6]
#include <string>
\end{emcppshiddenlisting}
\begin{emcppslisting}[emcppsbatch=e6]
class ParseResult
{
    union  // storage for either the result or the error
    {
        int         d_value; // result type (trivial)
        std::string d_error; // error  type (non-trivial)
    };

    bool d_isError;  // discriminator

public:
    explicit ParseResult(int value);                 // value constructor (1)
    explicit ParseResult(const std::string& error);  // value constructor (2)

    ParseResult(const ParseResult& rhs);             // copy constructor
    ParseResult& operator=(const ParseResult& rhs);  // copy assignment

    ~ParseResult();                                  // destructor
};
\end{emcppslisting}


\noindent If a \romeogloss{sum type} comprised more than two types, the discriminator would be an appropriately-sized integral or enumerated type instead of a Boolean.

As discussed in
%{\it\titleref{unrestrictedunion-description}} on page~\pageref{unrestrictedunion-description}
\intrarefsimple{unrestrictedunion-description}, having a non-trivial
type within a \lstinline!union! forces the programmer to provide each
desired special member function and define it manually; note
that the use of placement \lstinline!new! is not required for either of the
two \emph{value constructors} (above) because the initializer syntax
(below) is sufficient to begin the lifetime of even a non-trivial
object:

%begin{emcppslisting}
%ParseResult::ParseResult(double value) : d_value(value), d_isError(false)
%{
%}
%
%ParseResult::ParseResult(const std::string& error)
%    : d_error(error), d_isError(true)
%    // Note that placement (ù{\codeincomments{new}}ù) was not necessary here because a new
%    // (ù{\codeincomments{std::string}}ù) object will be created as part of the initialization of
%    // (ù{\codeincomments{d\_error}}ù).
%{
%}
%\end{emcppslisting}
\begin{emcppslisting}[emcppsbatch=e6]
ParseResult::ParseResult(int value) : d_value(value), d_isError(false)
{
}

ParseResult::ParseResult(const std::string& error)
    : d_error(error), d_isError(true)
    // Note that placement (ù{\codeincomments{new}}ù) was not necessary here because a new
    // (ù{\codeincomments{std::string}}ù) object will be created as part of the initialization of
    // (ù{\codeincomments{d\_error}}ù).
{
}
\end{emcppslisting}


\noindent Placement \lstinline!new! and explicit destructor calls are still, however,
required for destruction and both copy operations{\cprotect\footnote{For
more information on initiating the lifetime of an object, see \cite{iso14}, section 3.8, ``Object Lifetime," pp. 66--69.}}:

%begin{emcppslisting}
%ParseResult::~ParseResult()
%{
%    if(d_isError)
%    {
%        d_error.std::string::~string();
%            // An explicit destructor call is required for (ù{\codeincomments{d\_error}}ù) because its
%            // destructor is non-trivial.
%    }
%}
%
%ParseResult::ParseResult(const ParseResult& rhs) : d_isError(rhs.d_isError)
%{
%    if (d_isError)
%    {
%        new (&d_error) std::string(rhs.d_error);
%            // Placement (ù{\codeincomments{new}}ù) is necessary here to begin the lifetime of a
%            // (ù{\codeincomments{std::string}}ù) object at the address of (ù{\codeincomments{d\_error}}ù).
%    }
%    else
%    {
%        d_value = rhs.d_value;
%            // Placement (ù{\codeincomments{new}}ù) is not necessary here as (ù{\codeincomments{int}}ù) is a trivial type.
%    }
%}
%
%ParseResult& ParseResult::operator=(const ParseResult& rhs)
%{
%    // Destroy (ù{\codeincomments{lhs}}ù)'s error string if existent:
%    if (d_isError) { d_error.std::string::~string(); }
%
%    // Copy (ù{\codeincomments{rhs}}ù)'s object:
%    if (rhs.d_isError) { new (&d_error) std::string(rhs.d_error); }
%    else               { d_value = rhs.d_value; }
%
%    d_isError = rhs.d_isError;
%    return *this;
%}
%\end{emcppslisting}
\begin{emcppshiddenlisting}[emcppsbatch=e6]
// The calls to d_error.std::string::~string() below do not compile on clang
// before clang 11.  Invoking it as ~basic_string instead works, but that
// is a confusing visible change.  Adding this using fixes the problem invisibly.
using std::string;
\end{emcppshiddenlisting}
\begin{emcppslisting}[emcppsbatch=e6]
ParseResult::~ParseResult()
{
    if (d_isError)
    {
        d_error.std::string::~string();
            // An explicit destructor call is required for (ù{\codeincomments{d\_error}}ù) because its
            // destructor is non-trivial.
    }
}

ParseResult::ParseResult(const ParseResult& rhs) : d_isError(rhs.d_isError)
{
    if (d_isError)
    {
        new (&d_error) std::string(rhs.d_error);
            // Placement (ù{\codeincomments{new}}ù) is necessary here to begin the lifetime of a
            // (ù{\codeincomments{std::string}}ù) object at the address of (ù{\codeincomments{d\_error}}ù).
    }
    else
    {
        d_value = rhs.d_value;
            // Placement (ù{\codeincomments{new}}ù) is not necessary here as (ù{\codeincomments{int}}ù) is a trivial type.
    }
}

ParseResult& ParseResult::operator=(const ParseResult& rhs)
{
    if (this == &rhs) // Prevent self-assignment.
    {
        return *this;
    }
    // Destroy (ù{\codeincomments{lhs}}ù)'s error string if existent:
    if (d_isError) { d_error.std::string::~string(); }

    // Copy (ù{\codeincomments{rhs}}ù)'s object:
    if (rhs.d_isError) { new (&d_error) std::string(rhs.d_error); }
    else               { d_value = rhs.d_value; }

    d_isError = rhs.d_isError;
    return *this;
}
\end{emcppslisting}
In practice, \lstinline!ParseResult! would typically use a more general \romeogloss{sum type}{\cprotect\footnote{\lstinline!std::variant!, introduced
in C++17, is the standard construct used to represent a \romeogloss{sum
type} as a \emph{discriminated union}. Prior to C++17,
\lstinline!boost::variant! was the most widely used \emph{tagged} union
  implementation of a \romeogloss{sum type}.}} abstraction to support arbitrary value types and provide proper exception safety.

%\noindent In practice, \texttt{ParseResult} would typically be defined as a
%template and renamed to allow any arbitrary result type \texttt{T} to be
%returned or else implemented in terms of a more general \textbf{sum
%type} abstraction.{\cprotect\footnote{\texttt{std::variant}, introduced
%in C++17, is the standard construct used to represent a \textbf{sum
%type} as a \emph{discriminating union}. Prior to C++17,
%\texttt{boost::variant} was the most widely used \emph{tagged} union
%  implementation of a \textbf{sum type}.}}

\subsection[Potential Pitfalls]{Potential Pitfalls}\label{potential-pitfalls}

\subsubsection[Inadvertent misuse can lead to latent \romeogloss{undefined behavior} at runtime]{Inadvertent misuse can lead to latent \romeogloss{undefined behavior} at runtime}\label{inadvertent-misuse-can-lead-to-latent-undefined-behavior-at-runtime}

When implementing a type that makes use of an unrestricted union,
forgetting to initialize a non-trivial object (using either a
\emph{member initialization list} or \romeogloss{placement \lstinline!new!}) or
accessing a different object than the one that was actually initialized
can result in tacit \romeogloss{undefined behavior}. Although forgetting to
destroy an object does not necessarily result in \romeogloss{undefined
behavior}, failing to do so for any object that manages a resource such
as dynamic memory will result in a \emph{resource leak} and/or lead to
unintended behavior. Note that destroying an
object having a trivial destructor is never necessary; there are, however, rare cases where
we may choose not to destroy an object having a non-trivial
one.
%{\cprotect\footnote{A specific example of where one might
%deliberately choose \emph{not} to destroy an object occurs when a
%collection of related objects are allocated from the same local memory
%resource and then deallocated unilaterally by releasing the memory
%back to the resource. No issue arises if the only resource that is ``leaked''
%by not invoking each individual destructor is the memory allocated
%from that memory resource, and that memory can be
%reused without resulting in \textbf{undefined behavior} if it
%is not subsequently referenced in the context of the deallocated
%  objects.}}

\subsection[Annoyances]{Annoyances}\label{annoyances}

\hspace*{\fill}

\subsection[See Also]{See Also}\label{see-also}

\begin{itemize}
\item{%Section~\ref{deleted-functions}, ``\titleref{deleted-functions}" on page~\pageref{deleted-functions} —
\seealsoref{deleted-functions}{\locationa}Any special member function of a \lstinline!union! that corresponds to a non-trivial one in any of its member elements will be implicitly deleted.} % "or when" changed to "and when" per JD Garcia and Slava
\end{itemize}

\subsection[Further Reading]{Further Reading}\label{further-reading}

\begin{itemize}
\item{\cite{goldthwaite07}}
\item{\cite{ouellet16}}
\end{itemize}



