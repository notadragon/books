\emcppsFeature{
        short={{\tt constexpr} Functions '14},
        long={Relaxed Restrictions on {\SecCode constexpr} Functions},
}{relaxed-constexpr-restrictions}

C++14 lifts restrictions regarding use of many language features in the
body of a \texttt{constexpr} function (see ``\titleref{constexprfunc}" on page~\pageref{constexprfunc}).

\subsection[Description]{Description}\label{description}

The cautious introduction (in C++11) of \texttt{constexpr}
functions --- i.e., functions eligible for compile-time evaluation
--- was accompanied by a set of strict rules that, despite making life
easier for compiler implementers, severely narrowed the breadth of valid
use cases for the feature. In C++11, \texttt{constexpr}
function bodies were restricted to essentially a single
\texttt{return} statement and were not permitted to have any modifiable
local state (variables) or \textbf{imperative} language constructs
(e.g., assignment), thereby greatly reducing their \mbox{usefulness}:

\begin{lstlisting}[language=C++]
constexpr int fact11(int x)
{
    static_assert(x >= 0, "");
        // Error in C++11/14: (ù{\codeincomments{x}}ù) is not a constant expression.

    static_assert(sizeof(x) >= 4, "");  // OK in C++11/14

    return x < 2 ? 1 : x * fact11(x - 1);  // OK in C++11/14
}
\end{lstlisting}

\noindent Notice that recursive calls were supported, often leading to convoluted
implementations of algorithms (compared to an \textbf{imperative}
counterpart); see \textit{\titleref{use-cases-relaxedconstexpr}: \titleref{non-recursive-constexpr-algorithms}} on page~\pageref{non-recursive-constexpr-algorithms}.

The C++11 \texttt{static\_assert} feature (see ``\titleref{compile-time-assertions-(static_assert)}" on page~\pageref{compile-time-assertions-(static_assert)}) was always
permitted in a C++11 \texttt{constexpr} function body.
However, because the input variable \texttt{x} in \texttt{fact11}
(in the code snippet above) is inherently not a compile-time constant expression, it can
never appear as part of a \texttt{static\_assert} predicate. Note that a
\texttt{constexpr} function returning \texttt{void} was also permitted:

\begin{lstlisting}[language=C++]
constexpr void no_op() { };  // OK in C++11/14
\end{lstlisting}

\noindent Experience gained from the release and subsequent real-world use of
C++11 emboldened the standard committee to lift most of these (now
seemingly arbitrary) restrictions for C++14, allowing use of (nearly)
\emph{all} language constructs in the body of a \texttt{constexpr}
function. In C++14, familiar non-expression-based control-flow
constructs, such as \texttt{if} statements and \texttt{while} loops, are
also available, as are modifiable local variables and assignment
operations:

\begin{lstlisting}[language=C++]
constexpr int fact14(int x)
{
    if (x <= 2)        // error in C++11; OK in C++14
    {
        return 1;
    }

    int temp = x - 1;  // error in C++11; OK in C++14
    return x * fact14(temp);
}
\end{lstlisting}

\noindent Some useful features remain disallowed in C++14; most notably, any form
of dynamic allocation is not permitted, thereby preventing the use of
common standard container types, such as \texttt{std::string} and
\texttt{std::vector}{\cprotect\footnote{In C++20, even more
restrictions were lifted, allowing, for example, some limited forms of
  dynamic allocation, \texttt{try} blocks, and uninitialized variables.}}:
\begin{enumerate}
\item{\texttt{asm} declarations}
\item{\texttt{goto} statements}
\item{Statements with labels other than \texttt{case} and \texttt{default}}
\item{\texttt{try} blocks}
\item{Definitions of variables
\begin{enumerate}
\item{of other than a \textbf{literal type} (i.e., fully processable at compile time)}
\item{decorated with either \texttt{static} or \texttt{thread\_local}}
\item{left uninitialized}
\end{enumerate}
}
\end{enumerate}
The restrictions on what can appear in the body of a \texttt{constexpr}
that remain in C++14 are reiterated here in codified
form\footnote{Note that the degree to which these remaining forbidden features are reported varies substantially from one popular compiler to the next.}:

\begin{lstlisting}[language=C++]
template <typename T>
constexpr void f()
try {                  // Error: (ù{\codeincomments{try}}ù) outside body isn't allowed (until C++20).
    std::ifstream is;  // Error: objects of *non-literal* types aren't allowed.
    int x;             // error: uninitialized vars. disallowed (until C++20)
    static int y = 0;  // Error: (ù{\codeincomments{static}}ù) variables are disallowed.
    thread_local T t;  // Error: (ù{\codeincomments{thread\_local}}ù) variables are disallowed.
    try{}catch(...){}  // error: (ù{\codeincomments{try}}ù)/(ù{\codeincomments{catch}}ù) disallowed (until C++20)
    if (x) goto here;  // Error: (ù{\codeincomments{goto}}ù) statements are disallowed.
    []{};              // Error: lambda expressions are disallowed (until C++17).
here: ;                // Error: labels (except (ù{\codeincomments{case}}ù)/(ù{\codeincomments{default}}ù)) aren't allowed.
    asm("mov %r0");    // Error: (ù{\codeincomments{asm}}ù) directives are disallowed.
} catch(...) { }       // error: (ù{\codeincomments{try}}ù) outside body disallowed (until C++20)
\end{lstlisting}


\subsection[Use Cases]{Use Cases}\label{use-cases-relaxedconstexpr}

\subsubsection[Nonrecursive {\tt constexpr} algorithms]{Nonrecursive {\SubsubsecCode constexpr} algorithms}\label{non-recursive-constexpr-algorithms}

The C++11 restrictions on the use of \texttt{constexpr} functions often
forced programmers to implement algorithms (that would otherwise be
implemented iteratively) in a recursive manner. Consider, as a familiar
example, a naive{\cprotect\footnote{For a more efficient (yet less
intuitive) C++11 algorithm, see \textit{\titleref{appendix:-optimized-c++11-example-algorithms}, \titleref{recursive-fibonacci}} on page~\pageref{recursive-fibonacci}.}}
C++11-compliant \texttt{constexpr} implementation of a function,
\texttt{fib11}, returning the \emph{n}th Fibonacci number\footnote{We used \texttt{long}~\texttt{long} (instead of \texttt{long})
here to ensure a unique C++ type having at least 8 bytes on all
conforming platforms for simplicity of exposition (avoiding an internal
copy). We deliberately chose \emph{not} to make the value returned
\texttt{unsigned} because the extra bit does not justify changing the
\textbf{algebra} (from \texttt{signed} to \texttt{unsigned}). For more
discussion on these specific topics, see ``\titleref{long-long}" on page~\pageref{long-long}.}:

\begin{lstlisting}[language=C++]
constexpr long long fib11(long long x)
{
    return
        x == 0 ? 0
               : (x == 1 || x == 2) ? 1
                                    : fib11(x - 1) + fib11(x - 2);
}
\end{lstlisting}

\noindent The implementation of the \texttt{fib11} function (above) has various
undesirable properties.
\begin{enumerate}
\item{\emph{Reading difficulty} — Because it must be implemented using a single \texttt{return} statement, branching requires a chain of \emph{ternary operators}, leading to a single long expression that might impede human comprehension.}
\item{\emph{Inefficiency and lack of scaling} — The explosion of recursive calls is taxing on compilers: (1) the time to compile is markedly slower for the \emph{recursive} (C++11) algorithm than it would be for its \emph{iterative} (C++14) counterpart, even for modest inputs,{\cprotect\footnote{As an example, Clang 10.0.0, running on an x86-64 machine, required more than 80 times longer to evaluate \texttt{fib(27)} implemented using the \emph{recursive} (C++11) algorithm than to evaluate the same functionality implemented using the \emph{iterative} (C++14) algorithm.}} and (2) the compiler might simply refuse to complete the compile-time calculation if it exceeds some internal (platform-dependent) \emph{threshold} number of operations.{\cprotect\footnote{The same Clang 10.0.0 compiler discussed in the previous footnote failed to compile \texttt{fib11(28)}:

\begin{lstlisting}[language=C++, basicstyle={\ttfamily\footnotesize}]
error: static_assert expression is not an integral constant expression
    static_assert(fib11(28) == 317811, "");
                  ^~~~~~~~~~~~~~~~~~~

note: constexpr evaluation hit maximum step limit; possible infinite loop?
\end{lstlisting}

\noindent GCC 10.x fails at \texttt{fib(36)}, with a similar diagnostic:

\begin{lstlisting}[language=C++, basicstyle={\ttfamily\footnotesize}]
error: 'constexpr' evaluation operation count exceeds limit of 33554432
       (use '-fconstexpr-ops-limit=' to increase the limit)
\end{lstlisting}

\noindent Clang 10.x fails to compile any attempt at constant evaluating \texttt{fib(28)}, with the following diagnostic message:

\begin{lstlisting}[language=C++, basicstyle={\ttfamily\footnotesize}]
note: constexpr evaluation hit maximum step limit; possible infinite loop?
\end{lstlisting}
}} % end footnote, end cprotect
} % end item
\item{\emph{Redundancy} — Even if the recursive implementation were suitable for small input values during compile-time evaluation, it would be unlikely to be suitable for any run-time evaluation, thereby requiring programmers to provide and maintain \emph{two} separate versions of the same algorithm: a compile-time \emph{recursive} one and a runtime \emph{iterative} one.}
\end{enumerate}

In contrast, an \emph{imperative} implementation of a \texttt{constexpr}
function implementing a function returning the \emph{n\/}th Fibonacci
number in C++14, \texttt{fib14}, does not suffer from any of the three
issues discussed above:

\begin{lstlisting}[language=C++]
constexpr long long fib14(long long x)
{
    if (x == 0) { return 0; }

    long long a = 0;
    long long b = 1;

    for (long long i = 2; i <= x; ++i)
    {
        long long temp = a + b;
        a = b;
        b = temp;
    }

    return b;
}
\end{lstlisting}

\noindent As one would expect, the compile time required to evaluate the iterative
implementation (above) is manageable{\cprotect\footnote{Both GCC 10.x
and Clang 10.x evaluated \texttt{fib14(46)} 1836311903 correctly in
under 20ms on a machine running Windows 10 x64 and equipped with a
  Intel Core i7-9700k CPU.}}; of course, far more
computationally efficient (e.g., closed form{\cprotect\footnote{E.g.,
see
  http://mathonline.wikidot.com/a-closed-form-of-the-fibonacci-sequence.}})
solutions to this classic exercise are available.

\subsubsection[Optimized metaprogramming algorithms]{Optimized metaprogramming algorithms}\label{optimized-metaprogramming-algorithms}

C++14's relaxed \texttt{constexpr} restrictions enable the use of
modifiable local variables and \textbf{imperative} language constructs
for metaprogramming tasks that were historically often implemented by
using (Byzantine) recursive template instantiation (notorious for their
voracious consumption of compilation time).

Consider, as the simplest of examples, the task of counting the number
of occurrences of a given type inside a \textbf{type list} represented
here as an empty variadic template (see ``\titleref{variadictemplate}" on page~\pageref{variadictemplate}) that can be
instantiated using a variable-length sequence of arbitrary C++ types{\cprotect\footnote{Variadic templates are a C++11
  feature having many valuable and practical uses. In this case, the
  variadic feature enables us to easily describe a template that takes
  an arbitrary number of C++ type arguments by specifying an ellipsis
  (\texttt{...}) immediately following \texttt{typename}. Emulating such
  functionality in C++98/03 would have required significantly more
  effort: A typical workaround for this use case would have been to
  create a template having some fixed maximum number of arguments (e.g.,
  20), each defaulted to some unused (incomplete) type (e.g.,
  \texttt{Nil}):

  \begin{lstlisting}[language=C++, basicstyle={\ttfamily\footnotesize}]
  struct Nil;  // arbitrary unused (incomplete) type

  template <typename = Nil, typename = Nil, typename = Nil, typename = Nil>
  struct TypeList { };
      // emulates the variadic (ù{\codeincomments{TypeList}}ù) template (ù{\codeincomments{struct}}ù) for up to four
      // type arguments
  \end{lstlisting}

\vspace*{-1ex}
\noindent Another theoretically appealing approach is to implement a Lisp-like
  recursive data structure; the compile-time overhead for such
  implementations, however, often makes them impractical.}}:

\begin{lstlisting}[language=C++]
template <typename...> struct TypeList { };
    // empty variadic template instantiable with arbitrary C++ type sequence
\end{lstlisting}

\noindent Explicit instantiations of this variadic template could be used to
create objects:

\begin{lstlisting}[language=C++]
TypeList<>                 emptyList;
TypeList<int>              listOfOneInt;
TypeList<int, double, Nil> listOfThreeIntDoubleNil;
\end{lstlisting}

\noindent A naive C++11-compliant implementation of a \textbf{metafunction}
\texttt{Count}, used to ascertain the (order-agnostic) number of times a
given C++ type was used when creating an instance of the
\texttt{TypeList} template (above), would usually make recursive use of
(baroque) \textbf{partial class template
specialization}{\cprotect\footnote{The use of class-template
specialization (let alone partial specialization) might be unfamiliar
to those not accustomed to writing low-level template metaprograms, but the point of this use case is to obviate such
unfamiliar use. As a brief refresher, a general class template is what
the client typically sees at the user interface. A specialization is
typically an implementation detail consistent with the
\textbf{contract} specified in the general template but somehow more
restrictive. A partial specialization (possible for \emph{class} but
not \emph{function} templates) is itself a template but with one or
more of the general template parameters resolved. An \textbf{explicit}
or \textbf{full specialization} of a template is one in which
\emph{all} of the template parameters have been resolved and, hence, is
not itself a template. Note that a \textbf{full specialization} is a
stronger candidate for a match than a partial specialization, which is
a stronger match candidate than a simple template specialization,
which, in turn, is a better match than the general template (which, in
  this example, happens to be an \textbf{incomplete type}).}} to satisfy the single-return-statement requirements{\cprotect\footnote{Notice that this \texttt{Count}
\textbf{metafunction} also makes use (in its implementation) of
variadic class templates to parse a \textbf{type list} of unbounded
depth. Had this been a C++03 implementation, we would have been forced
to create an approximation (to the simple class-template
specialization containing the \textbf{parameter pack}
\texttt{Tail...}) consisting of a bounded number (e.g., 20) of simple
(class) template specializations, each one taking an increasing number
of template arguments:

\begin{lstlisting}[language=C++, basicstyle={\ttfamily\footnotesize}]
template <typename X, typename Y>
struct Count<X, TypeList<Y>>
    : std::integral_constant<int, std::is_same<X, Y>::value> { };
    // (class) template specialization for one argument

template <typename X, typename Y, typename Z>
struct Count<X, TypeList<Y, Z>>
    : std::integral_constant<int,
        std::is_same<X, Y>::value + std::is_same<X, Z>::value> { };
    // (class) template specialization for two arguments

template <typename X, typename Y, typename Z, typename A>
struct Count<X, TypeList<Y, Z, A>>
    : std::integral_constant<int,
        std::is_same<X, Y>::value + Count<X, TypeList<Z, A>>::value> { };
    // recursive (class) template specialization for three arguments

// ...
\end{lstlisting}
      }}:

\begin{lstlisting}[language=C++,label={relaxedconstexpr-countcode}]
#include <type_traits>  // (ù{\codeincomments{std::integral\_constant}}ù), (ù{\codeincomments{std::is\_same}}ù)

template <typename X, typename List> struct Count;
    // general template used to characterize the interface for the (ù{\codeincomments{Count}}ù)
    // metafunction
    // Note that this general template is an incomplete type.

template <typename X>
struct Count<X, TypeList<>> : std::integral_constant<int, 0> { };
    // partial (class) template specialization of the general (ù{\codeincomments{Count}}ù) template
    // (derived from the integral-constant type representing a compile-time
    // (ù{\codeincomments{0}}ù)), used to represent the base case for the recursion --- i.e., when
    // the supplied (ù{\codeincomments{TypeList}}ù) is empty
    // The payload (i.e., the enumerated (ù{\codeincomments{value}}ù) member of the base class)
    // representing the number of elements in the list is (ù{\codeincomments{0}}ù).

template <typename X, typename Head, typename... Tail>
struct Count<X, TypeList<Head, Tail...>>
    : std::integral_constant<int,
        std::is_same<X, Head>::value + Count<X, TypeList<Tail...>>::value> { };
    // simple (class) template specialization of the general (ù{\codeincomments{count}}ù) template
    // for when the supplied list is not empty
    // In this case, the second parameter will be partitioned as the first
    // type in the sequence and the (possibly empty) remainder of the
    // (ù{\codeincomments{TypeList}}ù). The compile-time value of the base class will be either the
    // same as or one greater than the value accumulated in the (ù{\codeincomments{TypeList}}ù) so
    // far, depending on whether the first element is the same as the one
    // supplied as the first type to (ù{\codeincomments{Count}}ù).

static_assert(Count<int, TypeList<int, char, int, bool>>::value == 2, "");
\end{lstlisting}

\noindent Notice that we made use of a C++11 \textbf{parameter pack},
\texttt{Tail...} (see ``\titleref{variadictemplate}" on page~\pageref{variadictemplate}), in the
implementation of the simple template specialization to package up and
pass along any remaining types.

As should be obvious by now, the C++11 restriction encourages both
somewhat rarified metaprogramming-related knowledge and a
\emph{recursive} implementation that can be compile-time intensive in
practice.{\cprotect\footnote{For a more efficient C++11 version of
\texttt{Count}, see \textit{\titleref{appendix:-optimized-c++11-example-algorithms}, \titleref{constexpr-typelist-count-algorithm}} on page~\pageref{constexpr-typelist-count-algorithm}.}} By exploiting C++14's relaxed
\texttt{constexpr} rules, a simpler and typically more compile-time
friendly \emph{imperative} solution can be realized:

\begin{lstlisting}[language=C++]
template <typename X, typename... Ts>
constexpr int count()
{
    bool matches[sizeof...(Ts)] = { std::is_same<X, Ts>::value... };
        // Create a corresponding array of bits where (ù{\codeincomments{1}}ù) indicates sameness.

    int result = 0;
    for (bool m : matches)  // (C++11) range-based (ù{\codeincomments{for}}ù) loop
    {
        result += m;        // Add up (ù{\codeincomments{1}}ù) bits in the array.
    }

    return result;  // Return the accumulated number of matches.
}
\end{lstlisting}

\noindent The implementation above --- though more efficient and comprehensible
--- will require some initial learning for those unfamiliar with modern
C++ variadics. The general idea here is to use \textbf{pack expansion}
in a nonrecursive manner{\cprotect\footnote{\textbf{Pack expansion} is
a language construct that expands a \textbf{variadic pack} during
compilation, generating code for each element of the pack. This
construct, along with a \textbf{parameter pack} itself, is a
  fundamental building block of variadic templates,
  introduced in C++11. As a minimal example, consider the variadic
  function template, \texttt{e}:

  \begin{lstlisting}[language=C++, basicstyle={\ttfamily\footnotesize}]
  template <int... Is> void e() { f(Is...); }
  \end{lstlisting}

\noindent \texttt{e} is a function template that can be instantiated
  with an arbitrary number of compile-time-constant integers. The
  \texttt{int...}~\texttt{Is} syntax declares a \textbf{variadic pack}
  of compile-time-constant integers. The \texttt{Is...} syntax (used to
  invoke \texttt{f}) is a basic form of pack expansion that will resolve
  to all the integers contained in the \texttt{Is} pack, separated by
  commas. For instance, invoking
  \texttt{e<0,}~\texttt{1,}~\texttt{2,}~\texttt{3>()} results in the
  subsequent invocation of
  \texttt{f(0,}~\texttt{1,}~\texttt{2,}~\texttt{3)}. Note that --- as
  seen in the \texttt{count} example (which starts on page~\pageref{relaxedconstexpr-countcode}) --- any arbitrary
  expression containing a variadic pack can be expanded:

  \begin{lstlisting}[language=C++, basicstyle={\ttfamily\footnotesize}]
  template <int... Is> void g() { h((Is > 0)...); }
  \end{lstlisting}

\noindent The \texttt{(Is}~\texttt{>}~\texttt{0)...} expansion (above) will
  resolve to \texttt{N} comma-separated Boolean values, where \texttt{N}
  is the number of elements contained in the \texttt{Is}
  \textbf{variadic pack}. As an example of this expansion, invoking
  \texttt{g<5,}~\texttt{-3,}~\texttt{9>()} results in the subsequent invocation of \texttt{h(true,}~\texttt{false,}~\texttt{true)}.}} to
initialize the \texttt{matches} array with a sequence of zeros and ones
(representing, respectively, mismatch and matches between \texttt{X} and
a type in the \texttt{Ts...} pack) and then iterate over the array to
accumulate the number of ones as the final \texttt{result}. This
\texttt{constexpr}-based solution is both easier to understand and
typically faster to compile.{\cprotect\footnote{For a type list
containing 1024 types, the imperative (C++14) solution compiles about
twice as fast on GCC 10.x and roughly 2.6 times faster on Clang
  10.x.}}

\subsection[Potential Pitfalls]{Potential Pitfalls}\label{potential-pitfalls}

None so far

\subsection[Annoyances]{Annoyances}\label{annoyances}

None so far

\subsection[See Also]{See Also}\label{see-also}

\begin{itemize}
\item{``\titleref{constexprfunc}" — Conditionally safe C++11 feature that first introduced compile-time evaluations of functions.}
\item{``\titleref{constexprvar}" — Conditionally safe C++11 feature that first introduced variables usable as constant expressions.}
\item{``\titleref{variadictemplate}" — Conditionally safe C++11 feature allowing templates to accept an arbitrary number of parameters.}
\end{itemize}

\subsection[Further Reading]{Further Reading}\label{further-reading}

None so far

\subsection[Appendix: Optimized C++11 Example Algorithms]{Appendix: Optimized C++11 Example Algorithms}\label{appendix:-optimized-c++11-example-algorithms}

\subsubsection[Recursive Fibonacci]{Recursive Fibonacci}\label{recursive-fibonacci}

Even with the restrictions imposed by C++11, we can write a
more efficient recursive algorithm to calculate the \emph{n\/}th
Fibonacci number:

\begin{lstlisting}[language=C++]
#include <utility>  // (ù{\codeincomments{std::pair}}ù)

constexpr std::pair<long long, long long> fib11NextFibs(
    const std::pair<long long, long long> prev,  // last two calculations
    int count)                                   // remaining steps
{
    return (count == 0) ? prev : fib11NextFibs(
        std::pair<long long, long long>(prev.second,
                                        prev.first + prev.second),
        count - 1);
}

constexpr long long fib11Optimized(long long n)
{
    return fib11NextFibs(
        std::pair<long long, long long>(0, 1), // first two numbers
        n                                      // number of steps
    ).second;
}
\end{lstlisting}


\subsubsection[{\ttfamily constexpr} type list {\ttfamily Count} algorithm]{{\SubsubsecCode constexpr} type list {\SubsubsecCode Count} algorithm}\label{constexpr-typelist-count-algorithm}

As with the \texttt{fib11Optimized} example, providing a more efficient version of the \texttt{Count} algorithm in
C++11 is also possible, by accumulating the final result through recursive
\texttt{constexpr} function invocations:

\begin{lstlisting}[language=C++]
#include <type_traits>  // (ù{\codeincomments{std::is\_same}}ù)

template <typename>
constexpr int count11Optimized() { return 0; }
    // Base case: always return (ù{\codeincomments{0}}ù).

template <typename X, typename Head, typename... Tail>
constexpr int count11Optimized()
    // Recursive case: compare the desired type ((ù{\codeincomments{X}}ù)) and the first type in
    // the list ((ù{\codeincomments{Head}}ù)) for equality, turn the result of the comparison
    // into either (ù{\codeincomments{1}}ù) (equal) or (ù{\codeincomments{0}}ù) (not equal), and recurse with the rest
    // of the type list ((ù{\codeincomments{Tail...}}ù)).
{
    return (std::is_same<X, Head>::value ? 1 : 0)
        + count11Optimized<X, Tail...>();
}
\end{lstlisting}

\noindent This algorithm can be optimized even further in C++11 by using
a technique similar to the one shown for the iterative C++14
implementation. By leveraging a \texttt{std::array} as compile-time
storage for bits where \texttt{1} indicates equality between types, we can compute the final result with a fixed number of template
instantiations:

\begin{lstlisting}[language=C++]
#include <array>        // (ù{\codeincomments{std::array}}ù)
#include <type_traits>  // (ù{\codeincomments{std::is\_same}}ù)

template <int N>
constexpr int count11VeryOptimizedImpl(
    const std::array<bool, N>& bits,  // storage for "type sameness" bits
    int i)                            // current array index
{
    return i < N
        ? bits[i] + count11VeryOptimizedImpl<N>(bits, i + 1)
            // Recursively read every element from the (ù{\codeincomments{bits}}ù) array and
            // accumulate into a final result.
        : 0;
}

template <typename X, typename... Ts>
constexpr int count11VeryOptimized()
{
    return count11VeryOptimizedImpl<sizeof...(Ts)>(
        std::array<bool, sizeof...(Ts)>{ std::is_same<X, Ts>::value... },
            // Leverage pack expansion to avoid recursive instantiations.
        0);
}
\end{lstlisting}

\noindent Note that, despite being recursive, \texttt{count11VeryOptimizedImpl}
will be instantiated only once with \texttt{N} equal to the number of
elements in the \texttt{Ts...} pack.



