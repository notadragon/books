
% 01 Feb, 2021 JMB - packet 4 compilation fixes
% 10 Feb 2021, some cleanup before sending for revisions
% 4 March 2021, revisions entered and proofed
% 5 March 2021, glossary entries updated
% 5 March 2021, sent to copyediting
% 13 March 2021, copyedits in and proofed


\emcppsFeature{
    short={Underlying Type '11},
    long={Explicit Enumeration Underlying Type},
}{explicit-enumeration-underlying-type}
\setcounter{table}{0}
\setcounter{footnote}{0}
\setcounter{lstlisting}{0}
%\section[Underlying Type '11]{Explicit Enumeration Underlying Type}\label{explicit-enumeration-underlying-type}


The underlying type of an enumeration is the fundamental
\emcppsgloss{integral type} used to represent its enumerated values, which
can be specified explicitly in C++11.

\subsection[Description]{Description}\label{description}

Every enumeration employs an integral type, known as its
\emcppsgloss[underlying type (UT)]{underlying type}, to represent its compile-time-enumerated
values. By default, the \emcppsgloss[underlying type (UT)]{underlying type} of a C++03
\lstinline!enum! is chosen by the
implementation to be large enough to represent all of the values in the
enumeration and is allowed to exceed the size of an \lstinline!int!
\emph{only} if there are enumerators having values that cannot be
represented as an \lstinline!int! or \lstinline!unsigned!~\lstinline!int!:

\begin{emcppslisting}
enum RGB { e_RED, e_GREEN, e_BLUE };                  // OK, fits any (ù{\codeincomments{char}}ù)

enum Port { e_LEFT = -81, e_RIGHT = -82 };            // OK, fits (ù{\codeincomments{signed char}}ù)

enum Mask { e_LOW = 32767, e_HIGH = 65535 };          // OK, fits (ù{\codeincomments{unsigned short}}ù)

enum Big { e_31 = 1U<<31 };                           // OK, fits (ù{\codeincomments{unsigned int}}ù)

enum Err { K = 1024, M = K*K, G = M*K, T = G*K };     // Error, (ù{\codeincomments{G*K}}ù) overflows (ù{\codeincomments{int}}ù)...

enum OK { K = 1<<10, M = 1<<20, G = 1<<30, T = 1LL<<40 }; // OK
\end{emcppslisting}

\noindent The default underlying type chosen for an \lstinline!enum! is always
sufficiently large to represent \emph{all} enumerator values defined for
that \lstinline!enum!. If the value doesn't fit in an \lstinline!int!, it will
be selected deterministically as the first type able to represent all
values from the sequence: \lstinline!unsigned!~\lstinline!int!, \lstinline!long!,
\lstinline!unsigned!~\lstinline!long!, \lstinline!long!~\lstinline!long!,
\lstinline!unsigned!~\lstinline!long!~\lstinline!long!; see \featureref{\locationa}{long-long}.

While specifying an enumeration’s underlying type was impossible before C++11, the compiler could be forced to choose at least a 32-bit or 64-bit signed integral type by adding an enumerator having a sufficiently large negative value --- e.g., \lstinline!-1 << 31! for a 32-bit and \lstinline!-1LL << 63! for a 64-bit signed integer (assuming such is available on the target platform).

The above applies only to C++03 \lstinline!enum!s; the default underlying type of an \lstinline!enum! \lstinline!class! is ubiquitously \lstinline!int!, and it is not implementation defined; see \featureref{\locationc}{enumclass}.

Note that \lstinline!char! and \lstinline!wchar_t!, like enumerations, are their own distinct types (as opposed to \lstinline!typedef!-like aliases such as \lstinline!std::uint8_t!) and have their own implementation-defined underlying integral types. With \lstinline!char!, for example, the underlying type will always be either \lstinline!signed! \lstinline!char! or \lstinline!unsigned! \lstinline!char! (both of which are also distinct C++ types). The same is true in C++11 for \lstinline!char16_t! and \lstinline!char32_t!.{\protect\footnote{C++20 adds \lstinline!char8_t!, which is a distinct type that has \lstinline!unsigned! \lstinline!char! as its underlying type.}}

\subsubsection[Specifying underlying type explicitly]{Specifying underlying type explicitly}\label{specifying-underlying-type-explicitly}

As of C++11, we have the ability to specify the \emcppsgloss{integral type}
that is used to represent an \lstinline!enum!. This is achieved by
providing the type explicitly in the \lstinline!enum!'s declaration
following the enumeration's (optional) name and preceded by a colon:

\begin{emcppslisting}
enum Port : unsigned char
{
    // Each enumerator of (ù{\codeincomments{Port}}ù) is represented as an (ù{\codeincomments{unsigned char}}ù) type.

    e_INPUT        =  37,  // OK, would have fit in a (ù{\codeincomments{signed char}}ù) too
    e_OUTPUT       = 142,  // OK, would not have fit in a (ù{\codeincomments{signed char}}ù)
    e_CONTROL      = 255,  // OK, barely fits in an 8-bit unsigned integer
    e_BACK_CHANNEL = 256,  // Error, doesn't fit in an 8-bit unsigned integer
};
\end{emcppslisting}

\noindent If any of the values specified in the definition of the \lstinline!enum! is
outside the boundaries of what the provided \emcppsgloss[underlying type (UT)]{underlying type} is
able to represent, the compiler will emit an error, but see \intraref{potential-pitfalls-underlyingenum}{subtleties-of-integral-promotion}.
%\textit{\titleref{potential-pitfalls-underlyingenum}: \titleref{subtleties-of-integral-promotion}} on page~\pageref{subtleties-of-integral-promotion}.

\subsection[Use Cases]{Use Cases}\label{use-cases}

\subsubsection[Ensuring a compact representation where enumerator values are salient]{Ensuring a compact representation where enumerator values are salient}\label{ensuring-a-compact-representation-where-enumerator-values-are-salient}

When an enumeration needs to have an efficient representation, e.g.,
when it is used as a data member of a widely replicated type,
restricting the width of the underlying type to something smaller than
would occur by default on the target platform might be justified.

As a concrete example, suppose that we want to enumerate the months of
the year, in anticipation of placing that enumeration
inside a date class having an internal representation that maintains the
year as a two-byte signed integer, the month as an enumeration, and the
day as an 8-bit signed integer:

\begin{emcppshiddenlisting}[emcppsbatch=e1]
enum Month { January=1 };
\end{emcppshiddenlisting}
\begin{emcppslisting}[emcppsbatch=e1]
#include <cstdint>  // (ù{\codeincomments{std::int8\_t}}ù), (ù{\codeincomments{std::int16\_t}}ù)

class Date
{
    std::int16_t d_year;
    Month        d_month;
    std::int8_t  d_day;

public:
    Date();
    Date(int year, Month month, int day);

    // ...

    int year() const    { return d_year; }
    Month month() const { return d_month; }
    int day() const     { return d_day; }
};
\end{emcppslisting}

\noindent Suppose that, within an application, a \lstinline!Date! is typically constructed using
values obtained through a GUI, where the month is always selected
from a drop-down menu. The month is
supplied to the constructor as an \lstinline!enum! to avoid recurring
defects where the individual fields of the date are supplied in
month/day/year format. New functionality will be written to expect the
month to be enumerated. Still, the date class will be used in contexts
where the numerical value of the month is significant, such as in calls
to legacy functions that accept the month as an integer. Moreover,
iterating over a range of months is common and requires that the
enumerators convert automatically to their integral \emcppsgloss[underlying type (UT)]{underlying
type}, thus contraindicating use of the more strongly typed
\lstinline!enum! \lstinline!class!:

\begin{emcppslisting}
enum Month // defaulted underlying type (BAD IDEA)
{
    e_JAN = 1, e_FEB, e_MAR, e_APR, e_MAY, e_JUN,
    e_JUL , e_AUG, e_SEP, e_OCT, e_NOV, e_DEC
};
static_assert(sizeof(Month) <= 4 && alignof(Month) <= 4, "");
\end{emcppslisting}

%\begin{emcppslisting}
%enum Month  // defaulted underlying type (BAD IDEA)
%{
%    enum Month {
%    e_JAN = 1, e_FEB, e_MAR, e_APR, e_MAY, e_JUN,
%    e_JUL    , e_AUG, e_SEP, e_OCT, e_NOV, e_DEC
%    };
%};
%
%static_assert(sizeof(Month) <= 4 && alignof(Month) <= 4, "");
%\end{emcppslisting}

%\begin{emcppslisting}
%enum Month  // defaulted underlying type (BAD IDEA)
%{
%    e_JAN = 1, e_FEB, e_MAR,    // winter
%    e_APR    , e_MAY, e_JUN,    // spring
%    e_JUL    , e_AUG, e_SEP,    // summer
%    e_OCT    , e_NOV, e_DEC     // autumn
%};
%
%static_assert(sizeof(Month) == 4 && alignof(Month) == 4, "");
%\end{emcppslisting}

\noindent As it turns out, date values are used widely throughout this codebase,
and the proposed \lstinline!Date! type is expected to be used in large
aggregates. The underlying type of the \lstinline!enum! in the code snippet
above is implementation-defined and could be as small as a \lstinline!char!
or as large as an \lstinline!int! despite all the values fitting in a
\lstinline!char!. Hence, if this enumeration were used as a data member in
the \lstinline!Date! class, \lstinline!sizeof(Date)! would likely balloon to
12 bytes on some relevant platforms due to \emcppsgloss{natural alignment}!
(See \featureref{\locationc}{alignas}.)
%``\titleref{alignas}" on page~\pageref{alignas}.)

While reordering the data members of \lstinline!Date! such that \lstinline!d_year! and
\lstinline!d_day! were adjacent would ensure that \lstinline!sizeof(Date)!
would not exceed 8 bytes, a better approach is to explicitly specify the
enumeration's underlying type to ensure \lstinline!sizeof(Date)! is exactly
the 4 bytes needed to accurately represent the value of a
\lstinline!Date! object. Given that the values in this enumeration fit in
an 8-bit signed integer, we can specify its \emcppsgloss[underlying type (UT)]{underlying type} to
be, e.g., \lstinline!std::int8_t! or \lstinline!signed!~\lstinline!char!, on
every platform:

\begin{emcppslisting}
#include <cstdint>  // (ù{\codeincomments{std::int8\_t}}ù)

enum Month : std::int8_t  // user-provided underlying type (GOOD IDEA)
{
    e_JAN = 1, e_FEB, e_MAR,
    e_APR    , e_MAY, e_JUN,
    e_JUL    , e_AUG, e_SEP,
    e_OCT    , e_NOV, e_DEC
};

static_assert(sizeof(Month) == 1 && alignof(Month) == 1, "");
\end{emcppslisting}

\noindent With this revised definition of \lstinline!Month!, the size of a
\lstinline!Date! class is 4 bytes, which is especially valuable for large
aggregates:

\begin{emcppslisting}[emcppsbatch=e1]
Date timeSeries[1000 * 1000];  // (ù{\codeincomments{sizeof(timeSeries)}}ù) is now 4Mb (not 12Mb)
\end{emcppslisting}


\subsection[Potential Pitfalls]{Potential Pitfalls}\label{potential-pitfalls-underlyingenum}

\subsubsection[External use of opaque enumerators]{External use of opaque enumerators}\label{external-use-of-opaque-enumerators-enumunderlying}

Providing an explicit underlying type to an \lstinline!enum! enables
clients to declare it as a complete type without
its enumerators. Unless the opaque form of its definition is exported in
a header file separate from its full definition, external clients
wishing to exploit the opaque version will be forced to locally declare
it with its \emcppsgloss[underlying type (UT)]{underlying type} but without its enumerator list. If
the underlying type of the full definition were to change, any program
incorporating \emph{its own} original and now inconsistent elided
definition and the \emph{new} full one would become silently \emcppsgloss{ill formed, no diagnostic required (IFNDR)}. (See \featureref{\locationc}{enumopaque}.)
%``\titleref{enumopaque}" on page~\pageref{enumopaque}.)

\subsubsection[Subtleties of integral promotion]{Subtleties of integral promotion}\label{subtleties-of-integral-promotion}

When used in an arithmetic context, one might naturally assume that the
type of a classic \lstinline!enum! will first convert to its
\emcppsgloss[underlying type (UT)]{underlying type}, which is not always the case. When used in a
context that does not explicitly operate on the \lstinline!enum! type
itself, such as a parameter to a function that takes that enum type,
\emcppsgloss{integral promotion} comes into play. For unscoped enumerations
without an explicitly specified underlying type and for character types
such as \lstinline!wchar_t!, \lstinline!char16_t!, and \lstinline!char32_t!,
integral promotion will directly convert the value to the first type in
the list \lstinline!int!, \lstinline!unsigned!~\lstinline!int!, \lstinline!long!,
\lstinline!unsigned!~\lstinline!long!, \lstinline!long!~\lstinline!long!, and
\lstinline!unsigned!~\lstinline!long!~\lstinline!long! that is sufficiently large
to represent all of the values of the underlying type. Enumerations
having a fixed underlying type will, as a first step, behave as if they
had decayed to their underlying type.

In most arithmetic expressions, this difference is irrelevant.
Subtleties arise, however, when one relies on overload resolution for
identifying the underlying type:

\begin{emcppslisting}[emcppsbatch=e2]
void f(signed char x);
void f(short x);
void f(int x);
void f(long x);
void f(long long x);

enum E1         { q, r, s, t, u };
enum E2 : short { v, w, x, y, z };

void test()
{
   f(E1::q);  // always calls (ù{\codeincomments{f(int)}}ù) on all platforms
   f(E2::v);  // always calls (ù{\codeincomments{f(short)}}ù) on all platforms
}
\end{emcppslisting}

\noindent The overload resolution for \lstinline!f! considers the type to which each
\emph{individual} enumerator can be directly integrally promoted. This
conversion for \lstinline!E1! can be only to \lstinline!int!. For \lstinline!E2!,
the conversion will consider \lstinline!int! \emph{and} \lstinline!short!, and
\lstinline!short!, being an exact match, will be selected. Note that even
though both enumerations are small enough to fit into a
\lstinline!signed!~\lstinline!char!, that overload of \lstinline!f! will never be
selected.

One might want to get to the implementation-defined underlying type,
though, and the Standard does provide a trait to do that:
\lstinline!std::underlying_type! in C++11 and the corresponding
\lstinline!std::underlying_type_t! alias in C++14. This trait can safely
be used in a cast without risking loss of value (see \featureref{\locationc}{auto}):
%  ``\titleref{auto}" on page~\pageref{auto}):

\begin{emcppslisting}[emcppsbatch=e2]
#include <type_traits>  // (ù{\codeincomments{std::underlying\_type}}ù)

template <typename E>
typename std::underlying_type<E>::type toUnderlying(E value)
{
    return static_cast<typename std::underlying_type<E>::type>(value);
}

void h()
{
    auto e1 = toUnderlying(E1::q); // might be anywhere from (ù{\codeincomments{signed char}}ù) to (ù{\codeincomments{int}}ù)
    auto e2 = toUnderlying(E2::v); // always deduced as (ù{\codeincomments{short}}ù)
}
\end{emcppslisting}

%\noindent As of C++20, however, the use of a classic enumerator in a context in
%which it is compared to or otherwise used in a binary operation with
%either an enumerator of another type or a nonintegral type (i.e., a
%floating-point type, such as \lstinline!float!, \lstinline!double!, or
%\lstinline!long!~\lstinline!double!) is deprecated, with the the possibility
%of being removed in C++23. Platforms might decide to warn against such
%uses retroactively:
%
%\begin{emcppslisting}
%enum { k_GRAMS_PER_OZ = 28 };  // not the best idea
%
%double gramsFromOunces(double ounces)
%{
%    return ounces * k_GRAMS_PER_OZ;  // deprecated in C++20; might warn
%}
%\end{emcppslisting}
%
%\noindent Casting to the \emcppsgloss{underlying type} is \emph{not} necessarily the
%same as direct integral promotion. In this context, we might want to
%change our \lstinline!enum! to a
%\lstinline!constexpr!~\lstinline!int! in the long
%term (see \featureref{\locationc}{constexprvar}):
%%  ``\titleref{constexprvar}" on page~\pageref{constexprvar}):
%
%\begin{emcppslisting}
%constexpr int k_GRAMS_PER_OZ = 28;  // future proof
%\end{emcppslisting}

Casting to the underlying type is not necessarily the same as direct integral promotion. If
an enumerator is intended to be used in arithmetic operations,\footnote{As of C++20, using an expression of an unscoped enumerated type in a \emph{binary} operation along with
an expression of either some \emph{other} enumerated type or any nonintegral type (e.g.,
\lstinline!double!) is deprecated, with the possibility of being removed in C++23. Platforms
may decide to warn against such uses retroactively.} a \lstinline!constexpr! variable might
be a better alternative (see \featureref{\locationc}{constexprvar}):

\begin{emcppslisting}
// enum { k_GRAMS_PER_OZ = 28 };    // not the best idea
constexpr int k_GRAMS_PER_OZ = 28;  // better idea

double gramsFromOunces(double ounces)
{
    return ounces * k_GRAMS_PER_OZ;
}
\end{emcppslisting}


\subsection[See Also]{See Also}\label{see-also}

\begin{itemize}
\item{\seealsoref{constexprvar}{\seealsolocationc}introduces an alternative way of declaring compile-time constants.}
\item{\seealsoref{enumclass}{\seealsolocationc}introduces a scoped, more strongly typed enumeration for which the default underlying type is \lstinline!int! and can also be specified explicitly.}
\item{\seealsoref{enumopaque}{\seealsolocationc}offers a means to insulate individual enumerators from clients.}
\end{itemize}

\subsection[Further Reading]{Further Reading}\label{further-reading}

\begin{itemize}
\item{``Item 10: Prefer scoped \texttt{enum}s to unscoped \texttt{enum}s," \cite{meyers15}}
\item{\cite{grimm17}}
\end{itemize}

