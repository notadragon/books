\newpage
\section[{\tt explicit} Operators]{Explicit Conversion Operators}\label{explicit-conversion-operators}


Ensure that a (user-defined) type is convertible to another type only in
contexts where the conversion is made obvious in the code.

\subsection[Description]{Description}\label{description-explicitconv}

Though sometimes desirable, implicit conversions achieved via (user-defined) \emph{conversion
functions} --- either (1) \textbf{converting constructors} (accepting a
single argument) or (2) \textbf{conversion operators} --- can also be problematic, especially when the
conversion involves a commonly used type (e.g., \texttt{int} or
\texttt{double}){\cprotect\footnote{Use of a conversion operator to
calculate distance from the origin in this unrealistically simple \texttt{Point}
example is for didactic purposes only. In practice, we would typically
use a named function for this purpose; see {\it\titleref{potential-pitfalls-explicitconv}: \titleref{sometimes-a-named-function-is-better}} on page~\pageref{sometimes-a-named-function-is-better}.}}:

\begin{lstlisting}[language=C++]
class Point  // implicitly convertible from an (ù{\codeincomments{int}}ù) or to a (ù{\codeincomments{double}}ù)
{
int d_x, d_y;

public:
Point(int x = 0, int y = 0);  // default, conversion, & value constructor
// ...
operator double() const;  // Return distance from origin as a (ù{\codeincomments{double}}ù).
};
\end{lstlisting}

\noindent As ever, calling a function \texttt{g} that takes a \texttt{Point} but
accidentally passing an \texttt{int} can lead to surprises:

\begin{lstlisting}[language=C++]
void g0(Point p);         // arbitrary function taking a (ù{\codeincomments{Point}}ù) object by value
void g1(const Point& p);  // arbitrary function taking a (ù{\codeincomments{Point}}ù) by (ù{\codeincomments{const}}ù) reference

void f1(int i)
{
g0(i);  // oops, called (ù{\codeincomments{g0}}ù) with (ù{\codeincomments{Point(i, 0)}}ù) by mistake
g1(i);  // oops, called (ù{\codeincomments{g1}}ù) with (ù{\codeincomments{Point(i, 0)}}ù) by mistake
}
\end{lstlisting}

\noindent This problem could have been solved even in C++98 by declaring the
constructor to be \texttt{explicit}:

\begin{lstlisting}[language=C++]
explicit Point(int x = 0, int y = 0);  // explicit converting constructor
\end{lstlisting}

\noindent If the conversion is desired, it must now be specified explicitly:

\begin{lstlisting}[language=C++]
void f2(int i)
{
g0(i)          // error: could not convert (ù{\codeincomments{i}}ù) from (ù{\codeincomments{int}}ù) to (ù{\codeincomments{Point}}ù)
g1(i);         // error: invalid initialization of reference type
g0(Point(i));  // OK
g1(Point(i));  // OK
}
\end{lstlisting}

\noindent The companion problem stemming from an \emph{implicit conversion
operator}, albeit less severe, remained:

\begin{lstlisting}[language=C++]
void h(double d);

double f3(const P& p)
{
h(p);      // OK? Or maybe called (ù{\codeincomments{h}}ù) with a "hypotenuse" by mistake
return p;  // OK? Or maybe this is a mistake too.
}
\end{lstlisting}

\noindent As of C++11, we can now use the \textbf{\texttt{explicit} specifier}
when declaring \textbf{conversion operators} (as well as
\textbf{converting constructors}), thereby forcing the client to request
conversion explicitly --- e.g., using \textbf{direct initialization} or
\texttt{static\_cast}):

\begin{lstlisting}[language=C++]
struct S0 { explicit operator int(); };

void g()
{
S0 s0;
int i = s0;                    // error (copy initialization)
double d = s0;                 // error (copy initialization)
int j = static_cast<int>(s0);  // OK (static cast)
if (s0) { }                    // error (contextual conversion to (ù{\codeincomments{bool}}ù))
int k(s0);                     // OK (direct initialization)
double e(s0);                  // error (direct initialization)
}
\end{lstlisting}

\noindent In contrast, had the conversion operator above not been declared to be
\texttt{explicit}, all conversions shown above would compile:

\begin{lstlisting}[language=C++]
struct S1 { /* implicit */ operator int(); };

void f()
{
S1 s1;
int i = s1;                    // OK (copy initialization)
double d = s1;                 // OK (copy initialization)
int j = static_cast<int>(s1);  // OK (static cast)
if (s1) { }                    // OK (contextual conversion to (ù{\codeincomments{bool}}ù))
int k(s1);                     // OK (direct initialization)
double e(s1);                  // OK (direct initialization)
}
\end{lstlisting}

\noindent Additionally, the notion of \textbf{contextual convertibility to
\texttt{bool}}{\cprotect\footnote{Since the early days of C++, a common
idiom to test for validity of an object has been to use it in a
context where it can (implicitly) convert itself to a type whose value
can be interpreted (contextually) as a boolean, with \texttt{true}
implying validity (and \texttt{false} otherwise). Implicit conversion
to \texttt{bool} (an integral type) was considered too dangerous,
so the cumbersome \textbf{safe-\texttt{bool} idiom} was used instead,
converting to a type that --- while contextually convertible to
\texttt{bool} --- could not (by design) participate in any other
operations. While making the conversion to \texttt{bool} (or
\texttt{const}~\texttt{bool}) \texttt{explicit} solves the safety
issue, the benefit of the idiom would be entirely lost if an explicit
cast would have to be performed to test for validity. To address this,
C++11 extends contextual conversion to \texttt{bool} for a given
expression \texttt{E} to include an application of
\texttt{static\_cast<const}~\texttt{volatile}~\texttt{bool>} to
\texttt{E}, thus enabling explicit conversion to \texttt{bool} to be
used in lieu of the (now deprecated) \textbf{safe-\texttt{bool} idiom}; see
\textbf{{sharpe13}}.}} applicable to arguments of logical operations
(e.g., \texttt{\&\&}, \texttt{||}, and \texttt{!}) and conditions of
most control-flow constructs (e.g., \texttt{if}, \texttt{while}) was
extended in C++11 to admit \emph{explicit} (user-defined) \texttt{bool}
conversion operators (see {\it\titleref{use-cases-explicitconv}: \titleref{enabling-contextual-conversions-to-bool-as-a-test-for-validity}} on page~\pageref{enabling-contextual-conversions-to-bool-as-a-test-for-validity}):

\begin{lstlisting}[language=C++]
struct S2 { explicit operator bool(); };

void h()
{
S2 s2;
int i = s2;                    // error (copy initialization)
double d = s2;                 // error (copy initialization)
int j = static_cast<int>(s2);  // error (static cast)
if (s2) { }                    // OK (contextual conversion to (ù{\codeincomments{bool}}ù))
int k(s2);                     // error (direct initialization)
double fd(s2);                 // error (direct initialization)
}
\end{lstlisting}

\noindent Prior to C++11, essentially the same effect as having an \emph{explicit}
\texttt{operator}~\texttt{bool()} member was achieved (albeit far less
conveniently) via the \textbf{safe-\texttt{bool} idiom}.

\subsection[Use Cases]{Use Cases}\label{use-cases-explicitconv}

\subsubsection[Enabling contextual conversions to {\tt bool} as a test for validity]{Enabling contextual conversions to {\SubsubsecCode bool} as a test for validity}\label{enabling-contextual-conversions-to-bool-as-a-test-for-validity}

Having a conventional test for validity that involves testing whether the object itself evaluates to \texttt{true} or \texttt{false} is an idiom that goes back to the
origins of C++. The \texttt{<iostream>} library, for example, uses this
idiom to determine if a given stream is valid:

\begin{lstlisting}[language=C++]
// C++03
#include <iostream>  // (ù{\codeincomments{std::ostream}}ù)

std::ostream& printTypeValue(std::ostream& stream, double value)
{
if (stream)  // relies on an implicit conversion to (ù{\codeincomments{bool}}ù)
{
stream << "double(" << value << ')';
}
else
{
// ... (handle stream failure)
}

return stream;
}
\end{lstlisting}

\noindent Implementing the implicit conversion to \texttt{bool} was, however,
problematic as the straight\-forward approach of using a
\textbf{conversion operator} could easily allow accidental misuse to go
undetected:

\begin{lstlisting}[language=C++]
class ostream
{
// ...

/* implicit */ operator bool();  // hypothetical (bad) idea
};

int client(std::ostream& out)
{
// ...
return out + 1;  // likely a latent runtime bug: always returns 1 or 2
}
\end{lstlisting}

\noindent The classic workaround, the \textbf{safe-\texttt{bool} idiom},
was to return some obscure pointer\linebreak[4] %%%%%
type (e.g., \textbf{pointer to
member}) that could not possibly be useful in any context other\linebreak[4] %%%%%
than one in which \texttt{false} and a null pointer-to-member value (e.g.,\linebreak[4] %%%%%
\texttt{static\_cast<(ostream*::operator}~\texttt{bool)()>(0)}) are
treated equivalently.

When implementing this idiom in a user-defined type ourselves, we need not go to such lengths to avoid inviting unintended use
via an \emph{implicit} conversion to \texttt{bool}. As discussed in \textit{\titleref{description-explicitconv}} on page~\pageref{description-explicitconv}, a conversion operator to type
\texttt{bool} that is declared \texttt{explicit} continues to act as if
it were \emph{implicit} only in those places where we might want it to
do so and nowhere else --- i.e., exactly those places that enable
\textbf{contextual conversion to \texttt{bool}}.{\cprotect\footnote{Note that two
consecutive \texttt{!} operators can be used to synthesize a
\textbf{contextual conversion to \texttt{bool}} --- i.e., if \texttt{X} is an
expression that is explicitly convertible to \texttt{bool}, then
\texttt{(!!(X))} will be \texttt{(true)} or \texttt{(false)}
accordingly.}}

As a concrete example, consider a \texttt{ConnectionHandle} class that
can be in either a \emph{valid} or \emph{invalid} state. For the user's
convenience and consistency with other proxy types (e.g., raw pointers)
that have a similar \emph{invalid} state, representing the invalid (or null) state via an explicit conversion to
\texttt{bool} might be desirable:

\begin{lstlisting}[language=C++]
struct ConnectionHandle
{
std::size_t maxThroughput() const;
// Return the maximum throughput (in bytes) of the connection.

explicit operator bool() const;
// Return (ù{\codeincomments{true}}ù) if the handle is valid and (ù{\codeincomments{false}}ù) otherwise.
};
\end{lstlisting}

\noindent Instances of \texttt{ConnectionHandle} will convert to \texttt{bool}
only where one might reasonably want them to do so, say, as the
predicate of an \texttt{if} statement:

\begin{lstlisting}[language=C++]
int ping(const ConnectionHandle& handle)
{
if (handle)  // OK (contextual conversion to bool)
{
// ...
return 0;  // success
}

std::cerr << "Invalid connection handle.\n";
return -1;  // failure
}
\end{lstlisting}

\noindent Having an \texttt{explicit} conversion operator prevents unwanted
conversions to \texttt{bool} that might otherwise happen inadvertently:

\begin{lstlisting}[language=C++]
bool hasEnoughThroughput(const ClientConnection& connection,
const ResourceHandle&   handle)
{
return connection.throughput() <= handle;  // Compilation error, thankfully
//                                    ^~~~~~
}
\end{lstlisting}

\noindent After the relational operator (\texttt{<=}) in the example above, the
programmer mistakenly wrote \texttt{handle} instead of
\mbox{\texttt{handle.maxThroughput()}}. Fortunately the conversion operation of\linebreak[4] %%%%%
\mbox{\texttt{ResourceHandle}} was declared to be \texttt{explicit} and a
compile-time error (thankfully) ensued; if the conversion had
been \emph{implicit}, the example code above would have compiled, and,
if executed, the very same source for the \mbox{\texttt{hasEnoughThroughput}}
function would have silently exhibited well-defined but incorrect
behavior.

\subsection[Potential Pitfalls]{Potential Pitfalls}\label{potential-pitfalls-explicitconv}

\subsubsection[Sometimes implicit conversion is indicated]{Sometimes implicit conversion is indicated}\label{sometimes-implicit-conversion-is-indicated}

Implicit conversions to and from common arithmetic types, especially
\texttt{int}, are generally ill advised given the likelihood of
accidental misuse. However, sometimes \emph{implicit}
conversion is exactly what is needed. Such cases occur frequently with
wrapper and proxy types that might need to interoperate with a large
legacy codebase. Consider, for example, an initial implementation of
memory allocators in which each constructor takes, as an optional
trailing argument, a pointer to an abstract memory resource that itself
provides pure virtual \texttt{allocate} and \texttt{deallocate} member
functions. Later, we decide to move in the direction of the
\texttt{std::pmr} (C++17) standard and wrap those pointers in classes
that support additional operations. Making such constructors on the
wrapper explicit would force every client supplying an allocator to a
container to rework their code (e.g., by using \texttt{static\_cast}).
An implicit conversion in this case is further justified because the
likelihood of accidental spontaneous conversion to an \texttt{Allocator}
is all but nonexistent.

The same sort of stability argument favors implicit conversion for proxy
types intended to be dropped in and used in existing codebases. If, for
example, we wanted to provide a proxy for a writeable
\texttt{std::string} that, say, also logged, we might want an implicit
conversion to a \texttt{std::string\&\&} (perhaps using
\featureref{\locatione}{reference qualifiers}). In such cases, making the
conversion explicit would entirely defeat the purpose of the proxy,
which is to achieve new functionality with minimal effect on existing
client code.

\subsection[Sometimes a named function is better]{Sometimes a named function is better}\label{sometimes-a-named-function-is-better}

Other kinds of overuses of even \emph{explicit} conversion
operators exist. Like any user-defined operator, when the operation being
implemented is not somehow either canonical or ubiquitously idiomatic
for that operator, expressing that operation by a
named (i.e., non-operator) function is often better. Recall from
{\it\titleref{description-explicitconv}} on page~\pageref{description-explicitconv} that we used a conversion operator of
class \texttt{Point} to represent the distance from the origin. This
example serves both to illustrate how conversion operators \emph{can} be
used and also how they probably should \emph{not} be. Consider that (1)
many mathematical operations on a 2-D integral point might return a \texttt{double} (e.g., \texttt{magnitude},
\texttt{angle}) and (2) we might want to represent the same
information but in different units (e.g., \texttt{angleInDegrees},
\texttt{angleInRadians}).{\cprotect\footnote{Another valid design
decision is returning an object of type \texttt{Angle} that captures
the amplitude and provides named accessory to the different units
(e.g., \texttt{asDegrees}, \texttt{asRadians}).}}

Rather than employing any conversion \emph{operator} (\texttt{explicit}
or otherwise), consider instead providing a named function, which (1) is
automatically \texttt{explicit} and (2) affords both flexibility (in
writing) and clarity (in reading) for a variety of domain-specific
functions --- now and in the future --- that might well have had
overlapping return types:

\begin{lstlisting}[language=C++]
class Point  // only explicitly convertible (and from only an (ù{\codeincomments{int}}ù))
{
int d_x, d_y;

public:
explicit Point(int x = 0, int y = 0);  // explicit converting constructor
// ...
double magnitude() const;  // Return distance from origin as a (ù{\codeincomments{double}}ù).
};
\end{lstlisting}

\noindent Note that defining \textbf{nonprimitive functionality}, like
\texttt{magnitude}, in a separate \emph{utility} at a higher level in the
physical hierarchy might be better still.{\cprotect\footnote{For more on
separating out \textbf{nonprimitive functionality}, see
\textbf{{lakos20}}, sections 3.2.7--3.2.8, pp 529--552.}}

\subsection[Annoyances]{Annoyances}\label{annoyances}

None so far

\subsection[See Also]{See Also}\label{see-also}

None so far

\subsection[Further Reading]{Further Reading}\label{further-reading}

None so far

