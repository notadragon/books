%%%% copyedits in and proofed
%% 27 Jan, code checked and replaced as needed by Josh

\emcppsFeature{
    toclong={Operator for Extracting Expression Types},
    tocshort={\TOCCode decltype},
    short={\lstinline!decltype!},
    long={Operator for Extracting Expression Types},
    rhshort={\RHCode decltype},
    }{decltype}
%\section[{\tt decltype}]{Operator for Extracting Expression Types\sectionmark{\RHCode decltype}}\label{decltype}\sectionmark{\RHCode decltype}
\setcounter{table}{0}
\setcounter{footnote}{0}
\setcounter{lstlisting}{0}


The keyword \lstinline!decltype! enables the compile-time inspection of the \emcppsgloss{declared type} of an \emcppsgloss{entity} \emph{or} the type and
\emcppsgloss{value category} of an expression. Note that the special construct \lstinline!decltype(auto)! has a separate meaning; see \featureref{\locationf}{decltypeauto}.
%``\titleref{decltypeauto}" on page~\pageref{decltypeauto}.}

\subsection[Description]{Description}\label{description}

What results from the use of \lstinline!decltype! depends on the nature of
its operand.

\subsubsection[Use with entities]{Use with entities}\label{use-with-(typically-named)-entities}

%If an unparenthesized operand is either an \textbf{id-expression} that names an entity (or a non-type template parameter) or a \textbf{class member access expression}\glossary{class member access expression} (that identifies a class member), \texttt{decltype} yields the \emph{declared type} (the type of the \emph{entity} indicated by the operand):
If the operand is an unparenthesized \emcppsgloss[id expression]{id-expression} or unparenthesized member access, \lstinline!decltype! yields the \emph{declared type}, meaning the type of the \emph{entity} indicated by the operand:


\begin{emcppshiddenlisting}[emcppsbatch=e1]
#include <string>  // (ù{\codeincomments{std::string}}ù)
\end{emcppshiddenlisting}
\begin{emcppslisting}[emcppsbatch=e1]
int i;                // (ù{\codeincomments{decltype(i)}}ù)   -> (ù{\codeincomments{int}}ù)
std::string s;        // (ù{\codeincomments{decltype(s)}}ù)   -> (ù{\codeincomments{std::string}}ù)
int* p;               // (ù{\codeincomments{decltype(p)}}ù)   -> (ù{\codeincomments{int}}ù)*
const int& r = *p;    // (ù{\codeincomments{decltype(r)}}ù)   -> (ù{\codeincomments{const int\&}}ù)
struct { char c; } x; // (ù{\codeincomments{decltype(x.c)}}ù) -> (ù{\codeincomments{char}}ù)
double f();           // (ù{\codeincomments{decltype(f)}}ù)   -> (ù{\codeincomments{double()}}ù)
double g(int);        // (ù{\codeincomments{decltype(g)}}ù)   -> (ù{\codeincomments{double(int)}}ù)
\end{emcppslisting}


\subsubsection[Use with expressions]{Use with expressions}\label{use-with-(unnamed)-expressions}

When \lstinline!decltype! is used with any other expression \lstinline!E! of type \lstinline!T!, including parenthesized \emcppsgloss[id expression]{id-expression} or parenthesized member access, the result incorporates both the expression's type and
its \emcppsgloss{value category} (see \featureref{\locationc}{Rvalue-References}):
%\mclhead{Value category of \texttt{E}} & \mclhead{Result of
%\texttt{decltype(E)}} \\
%\emph{prvalue} & \texttt{T} \\
%\emph{lvalue} & \texttt{T\&} \\
%\emph{xvalue} & \texttt{T\&\&} \\
%%%%%%%%%%% changing this per Greg & design discussion
\begin{center}
{\small \begin{tabular}{c|c}\thickhline
\rowcolor[gray]{.9}   {\sffamily\bfseries Value category of {\ttfamily\bfseries E}}
& {\sffamily\bfseries Result of {\ttfamily\bfseries decltype(E)}} \\ \hline
{\romeovalueinside prvalue} &\lstinline!T! \\ \hline
{\romeovalueinside lvalue} & \lstinline!T\&! \\ \hline
{\romeovalueinside xvalue} & \lstinline!T\&\&! \\ \thickhline
\end{tabular}
}
\end{center}
%\noindent The three integer expressions below illustrate the various value
%categories:
%
%begin{emcppslisting}
%decltype(0)   // -> (ù{\codeincomments{int}}ù)   (*prvalue* category)
%
%int i;
%decltype((i)) // -> (ù{\codeincomments{int\&}}ù)  (*lvalue* category)
%
%int&& g();
%decltype(g()) // -> (ù{\codeincomments{int\&\&}}ù) (*xvalue* category)
%\end{emcppslisting}
%begin{emcppslisting}
%int g();
%decltype (g) g1; // id-expression,  (ù{\codeincomments{int ()();}}ù)
%decltype (g()) g2; // function call, (ù{\codeincomments{int}}ù)
%decltype ((g)) g3 = g; // id expression in parentheses, treat as lvalue,  (ù{\codeincomments{int (&)()}}ù)
%decltype ((g())) g4; // function call, parentheses have no effect, prvalue, (ù{\codeincomments{int}}ù)
%\end{emcppslisting}
%%%% new change from Josh
\noindent In general, \romeovalue{prvalues} can be passed to \lstinline!decltype! in a number of ways, including numeric literals, function calls that return by value, and explicitly created temporaries:

\begin{emcppslisting}
decltype(0)   i; // -> (ù{\codeincomments{int}}ù)
int f();
decltype(f()) j; // -> (ù{\codeincomments{int}}ù)
struct S{};
decltype(S()) k; // -> (ù{\codeincomments{S}}ù)
\end{emcppslisting}

\noindent An entity name passed to \lstinline!decltype!, as mentioned above, produces the type of the entity. If an entity name is enclosed in an additional set of parentheses, however, \lstinline!decltype! interprets its argument as an expression and its result incorporates the value category:

\begin{emcppslisting}[emcppsbatch=e2]
int i;
decltype(i)   l = i; // -> (ù{\codeincomments{int}}ù)
decltype((i)) m = i; // -> (ù{\codeincomments{int\&}}ù)
\end{emcppslisting}

\noindent Similarly, for all other \romeovalue{lvalue} expressions, the result of \lstinline!decltype! will be an \romeovalue{lvalue} reference:

\begin{emcppslisting}[emcppsbatch=e2]
int* pi = &i;
decltype(*pi) j = *pi; // -> (ù{\codeincomments{int\&}}ù)
decltype(++i) k = ++i; // -> (ù{\codeincomments{int\&}}ù)
\end{emcppslisting}

\noindent Finally, the value category of the expression will be an \romeovalue{xvalue} if it is a cast to or a function returning an \romeovalue{rvalue} reference:

\begin{emcppslisting}
int i;
decltype(static_cast<int&&>(i)) j = static_cast<int&&>(i); // -> (ù{\codeincomments{int\&\&}}ù)
int&& g();
decltype(g()) k = g();                                     // -> (ù{\codeincomments{int\&\&}}ù)
\end{emcppslisting}


\noindent Much like the \lstinline!sizeof!
operator (which is also resolved at compile time), the expression operand of \lstinline!decltype! is not evaluated:

%begin{emcppslisting}
%int i = 0;
%decltype(i++) j;  // equivalent to (ù{\codeincomments{int j;}}ù)
%assert(i == 0);   // The function (ù{\codeincomments{next}}ù) is never invoked.
%\end{emcppslisting}
\begin{emcppshiddenlisting}[emcppsbatch=e3]
#include <cassert>  // standard C (ù{\codeincomments{assert}}ù) macro
\end{emcppshiddenlisting}
\begin{emcppslisting}[emcppsbatch=e3]
void test1()
{
    int i = 0;
    decltype(i++) j;  // equivalent to (ù{\codeincomments{int j;}}ù)
    assert(i == 0);   // The expression (ù{\codeincomments{i++}}ù) was not evaluated.
}
\end{emcppslisting}
Note that the choice of using the postfix increment is significant; the prefix increment yields a different type:
\begin{emcppslisting}[emcppsbatch=e3]
void test2()
{
    int i = 0;
    int m = 1;
    decltype(++i) k = m; // equivalent to (ù{\codeincomments{int\& k = m;}}ù)
    assert(i == 0); // The expression (ù{\codeincomments{++1}}ù) is not evaluated.
}
\end{emcppslisting}

\subsection[Use Cases]{Use Cases}\label{use-cases-decltype}

\subsubsection[Avoiding unnecessary use of explicit typenames]{Avoiding unnecessary use of explicit typenames}\label{avoiding-unnecessary-use-of-explicit-typenames}

Consider two logically equivalent ways of declaring a vector of
iterators into a list of \lstinline!Widgets!:

\begin{emcppshiddenlisting}[emcppsbatch=e4]
#include <list>    // (ù{\codeincomments{std::list}}ù)
#include <vector>  // (ù{\codeincomments{std::vector}}ù)
struct Widget {};
\end{emcppshiddenlisting}
\begin{emcppslisting}[emcppsbatch=e4]
std::list<Widget> widgets;
std::vector<std::list<Widget>::iterator> widgetIterators;
    // (1) The full type of (ù{\codeincomments{widgets}}ù) needs to be restated, and (ù{\codeincomments{iterator}}ù)
    // needs to be explicitly named.
\end{emcppslisting}
\begin{emcppshiddenlisting}[emcppsbatch=e4-1]
#include <list>    // (ù{\codeincomments{std::list}}ù)
#include <vector>  // (ù{\codeincomments{std::vector}}ù)
struct Widget {};
\end{emcppshiddenlisting}
\begin{emcppslisting}[emcppsbatch=e4-1]
std::list<Widget> widgets;
std::vector<decltype(widgets.begin())> widgetIterators;
    // (2) Neither (ù{\codeincomments{std::list}}ù) nor (ù{\codeincomments{Widget}}ù) nor (ù{\codeincomments{iterator}}ù) need be named
    // explicitly.
\end{emcppslisting}

\noindent Notice that, when using \lstinline!decltype!, if the C++ type representing
the widget changes (e.g., from \lstinline!Widget! to, say,
\lstinline!ManagedWidget!) or the container used changes (e.g., from
\lstinline!std::list! to \lstinline!std::vector!), the declaration of
\lstinline!widgetIterators! does not necessarily need to change.

\subsubsection[Expressing type-consistency explicitly]{Expressing type-consistency explicitly}\label{expressing-type-consistency-explicitly}

In some situations, repetition of explicit type names might
inadvertently result in latent defects caused by mismatched types during
maintenance. For example, consider a \lstinline!Packet! class exposing a
\lstinline!const! member function that returns a value of type \lstinline!std::uint8_t!
representing the length of the packet's checksum:

\begin{emcppshiddenlisting}[emcppsbatch=e6]
#include <cstdint>  // (ù{\codeincomments{std::uint8\_t}}ù)
\end{emcppshiddenlisting}
\begin{emcppslisting}[emcppsbatch=e6]
class Packet
{
    // ...

public:
    std::uint8_t checksumLength() const;
};
\end{emcppslisting}

\noindent This unsigned 8-bit type was selected to minimize bandwidth usage
as the checksum length is sent over the network. Next, picture a loop
that computes the checksum of a \lstinline!Packet!, using the same type for its iteration variable to match the
type returned by \lstinline!Packet::checksumLength!:

\begin{emcppshiddenlisting}[emcppsbatch=e7]
#include <cstdint>  // (ù{\codeincomments{std::uint8\_t}}ù)
struct Packet
{
public:
    std::uint8_t checksumLength() const;
    std::uint8_t nthByte(std::uint8_t i) const;
};
struct Checksum {
    void appendByte(std::uint8_t b);
};
\end{emcppshiddenlisting}
\begin{emcppslisting}[emcppsbatch=e7]
void f()
{
    Checksum sum;
    Packet data;

    for (std::uint8_t i = 0; i < data.checksumLength(); ++i)  // brittle
    {
        sum.appendByte(data.nthByte(i));
    }
}
\end{emcppslisting}


Now suppose that, over time, the data transmitted by the \lstinline!Packet!
type grows to the point where the range of an \lstinline!std::uint8_t!
value might not be enough to ensure a sufficiently reliable checksum. If
the type returned by \lstinline!checksumLength()! is changed to, say,
\lstinline!std::uint16_t! without updating the type of the iteration
variable \lstinline!i! in lockstep, the loop might
silently{\cprotect\footnote{As of this writing, neither
GCC~9.3 nor Clang~10.0.0 provide
a warning (using \lstinline!-Wall!, \lstinline!-Wextra!, and
\lstinline!-Wpedantic!) for the comparison between \lstinline!std::uint8_t!
and \lstinline!std::uint16_t! --- even if (1) the value returned by
\lstinline!checksumLength! does not fit in a 8-bit integer, and (2) the
body of the function is visible to the compiler. Decorating
\lstinline!checksumLength! with \lstinline!constexpr! causes
\lstinline!clang++! to issue a warning, but this is clearly not a general
solution.}} become infinite.{\cprotect\footnote{The loop
variable is promoted to an \lstinline!unsigned! \lstinline!int! for
comparison purposes but wraps to 0 whenever its value prior to
  being incremented is 255.}}

Had \lstinline!decltype(packet.checksumLength())! been used to express the
type of \lstinline!i!, the types would have remained consistent, and the
ensuing defect would naturally have been avoided:

\begin{emcppshiddenlisting}[emcppsbatch=e7]
void g(const Packet &data)
{
\end{emcppshiddenlisting}
\begin{emcppslisting}[emcppsbatch=e7]
// ...
for (decltype(data.checksumLength()) i = 0; i < data.checksumLength(); ++i)
// ...
\end{emcppslisting}
\begin{emcppshiddenlisting}[emcppsbatch=e7]
{}
} // end g
\end{emcppshiddenlisting}

\subsubsection[Creating an auxiliary variable of generic type]{Creating an auxiliary variable of generic type}\label{creating-an-auxiliary-variable-of-generic-type}

Consider the task of implementing a generic \lstinline!loggedSum! function
template that returns the sum of two arbitrary objects \lstinline!a! and
\lstinline!b! after logging both the operands and the result value (e.g.,
for debugging or monitoring purposes). To avoid computing the possibly
expensive sum twice, we decide to create an auxiliary function-scope
variable, \lstinline!result!. Since the type of the sum depends on both
\lstinline!a! and \lstinline!b!, we can use
\lstinline!decltype(a!~\lstinline!+!~\lstinline!b)! to infer the type for both
the trailing return type of the
function (see %Section~\ref{trailing-function-return-types},
%``\titleref{trailing-function-return-types}" on page~\pageref{trailing-function-return-types})
\featureref{\locationa}{trailing-function-return-types}) and the auxiliary variable:

\begin{emcppshiddenlisting}[emcppsbatch=e8]
struct LogTrace {};
constexpr LogTrace LOG_TRACE; // Quick and dirty syntactic simulation of logger
template <typename T>
const LogTrace& operator<<(const LogTrace& lhs, const T&) { return lhs; }
\end{emcppshiddenlisting}
\begin{emcppslisting}[emcppsbatch=e8]
template <typename A, typename B>
auto loggedSum(const A& a, const B& b)
    -> decltype(a + b)                 // (1) exploiting trailing return types
{
    decltype(a + b) result = a + b;    // (2) auxiliary generic variable
    LOG_TRACE << a << " + " << b << " = " << result;
    return result;
}
\end{emcppslisting}
Using \lstinline!decltype(a!~\lstinline!+!~\lstinline!b)! as a return type is significantly different from relying on automatic \emcppsgloss[return type deduction]{\lstinline!return!-type deduction}; see \featureref{\locationc}{auto-feature}. Note that this particular use involves significant repetition of the expression \lstinline!a+b!.  See
%\textit{\titleref{annoyances-decltype}: \titleref{decltype-mechanical}} on page~\pageref{decltype-mechanical}
\intraref{annoyances-decltype}{decltype-mechanical}
for a discussion of ways in which this might be avoided.

\subsubsection[Determining the validity of a generic expression]{Determining the validity of a generic expression}\label{determining-the-validity-of-a-generic-expression}

In the context of generic-library development, \lstinline!decltype! can be
used in conjunction with \emcppsgloss{SFINAE} (``Substitution Failure Is Not An Error") to validate an expression involving a
template parameter.

For example, consider the task of writing a generic \lstinline!sortRange!
function template that, given a \emcppsgloss{range}, either invokes the
\lstinline!sort! member function of the argument (the one specifically
optimized for that type) if available or falls back to the more
general \lstinline!std::sort!:

\begin{emcppslisting}
template <typename Range>
void sortRange(Range& range)
{
    sortRangeImpl(range, 0);
}
\end{emcppslisting}

\noindent The client-facing \lstinline!sortRange! function (above) delegates its
behavior to an overloaded \lstinline!sortRangeImpl! function (below),
invoking the latter with the \lstinline!range! and a \emcppsgloss{disambiguator}
of type \lstinline!int!. The type of this additional parameter, whose value
is arbitrary, is used to give priority to the \lstinline!sort! member
function at compile time by exploiting overload resolution rules in
the presence of an implicit (\emph{standard}) conversion from
\lstinline!int! to \lstinline!long!:

\begin{emcppshiddenlisting}[emcppsbatch=e9]
#include <algorithm>  // (ù{\codeincomments{std::sort}}ù)
\end{emcppshiddenlisting}
\begin{emcppslisting}[emcppsbatch=e9]
template <typename Range>
void sortRangeImpl(Range& range,
                   long)                  // low priority: standard conversion
{
    // fallback implementation
    std::sort(std::begin(range), std::end(range));
}
\end{emcppslisting}

\noindent The fallback overload of \lstinline!sortRangeImpl! (above) will accept a
\lstinline!long! \emcppsgloss{disambiguator}, requiring a standard conversion from
\lstinline!int!, and will simply invoke \lstinline!std::sort!. The more
specialized overload of \lstinline!sortRangeImpl! (below) will accept an
\lstinline!int! \emcppsgloss{disambiguator} requiring no conversions and thus
will be a better match, provided a range-specific sort is available:

\begin{emcppslisting}
template <typename Range>
void sortRangeImpl(Range& range,
                   int,                          // high priority: exact match
                   decltype(range.sort())* = 0)  // check expression validity
{
    // optimized implementation
    range.sort();
}
\end{emcppslisting}

\noindent Note that, by exposing \lstinline!decltype(range.sort())! as part of
\lstinline!sortRangeImpl!'s declaration, the more specialized overload will
be discarded during template substitution if \lstinline!range.sort()! is
not a valid expression for the deduced \lstinline!Range!
type.{\cprotect\footnote{The technique of exposing a possibly unused
unevaluated expression --- e.g., using \lstinline!decltype! --- in a function's
declaration for the purpose of expression-validity detection prior to
template instantiation is commonly known as \emcppsgloss[expression sfinae]{expression SFINAE},
which is a restricted form of the more general (classical) SFINAE, and acts exclusively on
expressions visible in a function's signature rather than on frequently obscure
  template-based type computations.}}

The relative position of
\lstinline!decltype(range.sort())! in the signature of
\lstinline!sortRangeImpl! is not significant, as long as it is visible to
the compiler
during template substitution. The example shown in the main text uses a function parameter that is defaulted to
\lstinline!nullptr!. Alternatives involving a trailing return type or a
default template argument are also viable:

%\begin{emcppslisting}[basicstyle={\ttfamily\footnotesize}]
%template <typename Range>
%auto sortRangeImpl(Range& range, int) -> decltype(range.sort(), void());
%    // The comma operator is used to force the return type to (ù{\codeincomments{void}}ù),
%    // regardless of the return type of (ù{\codeincomments{range.sort()}}ù).
%
%template <typename Range, typename = decltype(std::declval<Range&>().sort()>
%auto sortRangeImpl(Range& range, int);
%    // (ù{\codeincomments{std::declval}}ù) is used to generate a reference to (ù{\codeincomments{Range}}ù) that can
%    // be used in an unevaluated expression.
%\end{emcppslisting}
\begin{emcppslisting}
#include <utility>  // (ù{\codeincomments{declval}}ù)
template <typename Range>
auto sortRangeImpl(Range& range, int) -> decltype(range.sort(), void());
    // The comma operator is used to force the return type to (ù{\codeincomments{void}}ù),
    // regardless of the return type of (ù{\codeincomments{range.sort()}}ù).

template <typename Range, typename = decltype(std::declval<Range&>().sort())>
auto sortRangeImpl(Range& range, int) -> void;
    // (ù{\codeincomments{std::declval}}ù) is used to generate a reference to (ù{\codeincomments{Range}}ù) that can
    // be used in an unevaluated expression.
\end{emcppslisting}

\noindent Putting it all together, we see that exactly two possible
outcomes exist for the original client-facing \lstinline!sortRange! function
invoked with a range argument of type \lstinline!R!:

\begin{itemize}
\item{If \lstinline!R! does have a \lstinline!sort! member function, the more specialized overload of\linebreak[4] \lstinline!sortRangeImpl! will be viable as \lstinline!range.sort()! is a well-formed expression and preferred because the \emcppsgloss{disambiguator} \lstinline!0! (of type \lstinline!int!) requires no conversion.}
\item{Otherwise, the more specialized overload will be discarded during template substitution as \lstinline!range.sort()! is not a well-formed expression, and the only remaining more general \lstinline!sortRangeImpl! overload will be chosen instead.}
\end{itemize}

\subsection[Potential Pitfalls]{Potential Pitfalls}\label{potential-pitfalls}

Perhaps surprisingly, \lstinline!decltype(x)! and \lstinline!decltype((x))!
will sometimes yield different results for the same expression
\lstinline!x!:

\begin{emcppslisting}[emcppsbatch=e10]
int i = 0; // (ù{\codeincomments{decltype(i)}}ù) yields (ù{\codeincomments{int}}ù).
           // (ù{\codeincomments{decltype((i))}}ù) yields (ù{\codeincomments{int\&}}ù).
\end{emcppslisting}

\noindent In the case where the unparenthesized operand is an entity having a
declared type \lstinline!T! and the parenthesized operand is an expression
whose value category is represented (by \lstinline!decltype!) as the same
type \lstinline!T!, the results will coincidentally be the same:

\begin{emcppslisting}[emcppsbatch=e10]
int& ref = i;  // (ù{\codeincomments{decltype(ref)}}ù) yields (ù{\codeincomments{int\&}}ù).
               // (ù{\codeincomments{decltype((ref))}}ù) yields (ù{\codeincomments{int\&}}ù).
\end{emcppslisting}

\noindent Wrapping its operand with parentheses ensures \lstinline!decltype! yields
the \emcppsgloss{value category} of a given expression. This technique can be
useful in the context of metaprogramming --- particularly in the case of
\emcppsgloss{value category} propagation.

\subsection[Annoyances]{Annoyances}\label{annoyances-decltype}

\subsubsection{Mechanical repetition of expressions might be required}\label{decltype-mechanical}

As mentioned in
%\textit{\titleref{use-cases-decltype}: \titleref{creating-an-auxiliary-variable-of-generic-type}} on page~\pageref{creating-an-auxiliary-variable-of-generic-type},
\intraref{use-cases-decltype}{creating-an-auxiliary-variable-of-generic-type},
using \lstinline!decltype! to capture a value of an expression that is about to be used, or for the return value of an expression, can often lead to repeating the same expression in multiple places (three distinct ones in the earlier example).

An alternate solution to this problem is to capture the result of the \lstinline!decltype! expression in a \lstinline!typedef!, \lstinline!using! type alias, or as a defaulted template parameter --- but that runs into the problem that it can be used only once the expression is valid.  A defaulted \lstinline!template! parameter cannot reference parameter names as it is written before them, and a type alias cannot be introduced prior to the return type being needed.  A solution to this problem lies in using standard library function \lstinline!std::declval! to create expressions of the appropriate type without needing to reference the actual function parameters by name:

\begin{emcppshiddenlisting}[emcppsbatch=e11]
#include <utility>  // (ù{\codeincomments{std::declval}}ù)
struct LogTrace {};
constexpr LogTrace LOG_TRACE; // Quick and dirty syntactic simulation of logger
template <typename T>
const LogTrace& operator<<(const LogTrace& lhs, const T&) { return lhs; }
\end{emcppshiddenlisting}
\begin{emcppslisting}[emcppsbatch=e11]
template <typename A, typename B,
          typename Result = decltype(std::declval<const A&>() +
                                     std::declval<const B&>())>
Result loggedSum(const A& a, const B& b)
{
    Result result = a + b;  // no duplication of the (ù{\codeincomments{decltype}}ù) expression
    LOG_TRACE << a << " + " << b << "=" << result;
    return result;
}
\end{emcppslisting}

Here, \lstinline!std::declval!, a function that cannot be executed at runtime and is only appropriate for use in \emcppsgloss{unevaluated contexts}, produces an expression of the specified type.  When mixed with \lstinline!decltype!, this lets us determine the result types for expressions without needing to construct (or even being able to construct) objects of the needed types.

\subsection[See Also]{See Also}\label{see-also}

\begin{itemize}
\item{%``\titleref{Rvalue-References}" on page~\pageref{Rvalue-References} ---
\seealsoref{Rvalue-References}{\locationc}The \lstinline!decltype! operator yields precise information on whether an expression is an \romeovalue{lvalue} or \romeovalue{rvalue}.}
\item{%``\titleref{alias-declarations-and-alias-templates}" on page~\pageref{alias-declarations-and-alias-templates} ---
\seealsoref{alias-declarations-and-alias-templates}{\locationa}Oftentimes, it is useful to give a name to the type yielded by \lstinline!decltype!, which is done with a \lstinline!using! alias.}
\item{%``\titleref{auto-feature}" on page~\pageref{auto-feature} ---
\seealsoref{auto-feature}{\locationc}The type computed by \lstinline!decltype! is similar to, but distinct from, the type deduction used by \lstinline!auto!.}
\item{%``\titleref{decltypeauto}" on page~\pageref{decltypeauto} ---
\seealsoref{decltypeauto}{\locationf}\lstinline!decltype! type computation rules can be useful in conjunction with an \lstinline!auto! variable.}
\end{itemize}

\subsection[Further Reading]{Further Reading}\label{further-reading}




