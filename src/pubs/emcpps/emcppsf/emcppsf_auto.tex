% 3 March 2021, See Also lines need to be updated in current style; waiting to hear back from Vittorio
% 4 March 2021, ready for Josh's code check




\emcppsFeature{
    short={\lstinline!auto! Variables},
    tocshort={{\TOCCode auto} Variables},
    long={Variables of Automatically Deduced Type},
    rhshort={{\RHCode auto} Variables},
}{auto}
\label{auto-variables}
\label{auto-feature}
\setcounter{table}{0}
\setcounter{footnote}{0}
\setcounter{lstlisting}{0}
%\section[{\tt auto} Variables]{Variables of Automatically Deduced Type}\label{auto}\label{auto-feature}
%\subsection[\tt{auto} Variables]{{\SubsecCode auto} Variables}\label{auto-variables}

The \lstinline!auto! keyword was repurposed{\cprotect\footnote{Prior to
C++11, the \lstinline!auto! keyword could be used as a \emcppsgloss{storage
class specifier} for objects declared at block scope and in function
parameter lists to indicate automatic storage duration, which is the
default for these kinds of declarations. The \lstinline!auto! keyword was
repurposed in C++11 alongside the deprecation (and subsequent removal
in C++17) of the \lstinline!register! keyword as a storage class
  specifier.}} in C++11 to act as a \emcppsgloss{placeholder type}. When used
in place of a type as part of a variable declaration, the compiler will
deduce the variable type from its initializer.

\subsection[Description]{Description}\label{description}

The \lstinline!auto! keyword may be used instead of an explicit type's name
when declaring variables. In such cases, the variable's type is deduced
from its initializer by the compiler applying the \emcppsgloss{placeholder
type} deduction rules, which, apart from a single exception for list
initializers (see \intraref{potential-pitfalls-auto}{surprising-deduction-for-list-initialization}),
are the same as the rules for \emcppsgloss{function template argument type
deduction}:

\begin{emcppslisting}[language=C++]
auto two = 2;     // Type of (ù{\codeincomments{two}}ù) is deduced to be (ù{\codeincomments{int}}ù)
auto pi = 3.14f;  // Type of (ù{\codeincomments{pi}}ù) is deduced to be (ù{\codeincomments{float}}ù).
\end{emcppslisting}
    
\noindent The types of the \lstinline!two! and \lstinline!pi! variables above are
deduced in the same manner as they would be if the same initializer were
passed to a function template taking a single argument of the template
type by value:

\begin{emcppslisting}[language=C++]
template <typename T> void deducer(T);

deducer(2);      // (ù{\codeincomments{T}}ù) is deduced to be (ù{\codeincomments{int}}ù).
deducer(3.14f);  // (ù{\codeincomments{T}}ù) is deduced to be (ù{\codeincomments{float}}ù).
\end{emcppslisting}
    
\noindent If the variable declared with \lstinline!auto! does not have an
initializer, if its name appears in the expression used to initialize
it, or if the initializers of multiple variables in the same declaration
don't deduce the same type, the program is ill formed:

\begin{emcppslisting}[language=C++]
auto x;                // Error, declaration of (ù{\codeincomments{auto x}}ù) has no initializer.
auto n = sizeof(n);    // Error, use of (ù{\codeincomments{n}}ù) before deduction of (ù{\codeincomments{auto}}ù)
auto i = 3, f = 0.3f;  // Error, inconsistent deduction for (ù{\codeincomments{auto}}ù)
\end{emcppslisting}
    
\noindent Just like the function template argument deduction never deduces a
reference type for its by-value argument, a variable declared with
unqualified \lstinline!auto! is never deduced to have a reference type:

\begin{emcppslisting}[language=C++]
int  val = 3;
int& ref = val;
auto tmp = ref;  // Type of (ù{\codeincomments{tmp}}ù) is deduced to be (ù{\codeincomments{int}}ù), not (ù{\codeincomments{int\&}}ù).
\end{emcppslisting}
    
\noindent Augmenting \lstinline!auto! with reference and cv-qualifiers, however,
allows controlling whether the deduced type is a reference and whether
it is \lstinline!const! and/or \lstinline!volatile!:

\begin{emcppslisting}[language=C++]
auto val = 3;
    // Type of (ù{\codeincomments{val}}ù) is deduced to be (ù{\codeincomments{int}}ù).
    // same as argument for (ù{\codeincomments{template <typename T> void deducer(T)}}ù)

const auto cval = val;
    // Type of (ù{\codeincomments{cval}}ù) is deduced to be (ù{\codeincomments{const int}}ù).
    // same as argument for (ù{\codeincomments{template <typename T> void deducer(const T)}}ù)

auto& ref = val;
    // Type of (ù{\codeincomments{ref}}ù) is deduced to be (ù{\codeincomments{int\&}}ù).
    // same as argument for (ù{\codeincomments{template <typename T> void deducer(T\&)}}ù)

auto& cref1 = cval;
    // Type of (ù{\codeincomments{cref1}}ù) is deduced to be (ù{\codeincomments{const int\&}}ù).
    // same as argument for (ù{\codeincomments{template <typename T> void deducer(T\&)}}ù)

const auto& cref2 = val;
    // Type of (ù{\codeincomments{cref2}}ù) is deduced to be (ù{\codeincomments{const int\&}}ù).
    // same as argument for (ù{\codeincomments{template <typename T> void deducer(const T\&)}}ù)
\end{emcppslisting}
    
\noindent Note that qualifying \lstinline!auto! with \lstinline!&&!
does \emph{not} result in deduction of an \romeovalue{rvalue} reference (see
\featureref{\locationc}{Rvalue-References}), but, in line with function template
argument deduction rules, would be treated as a \emph{forwarding
reference} (see \featureref{\locationc}{forwardingref}). This means that a
variable declared with \lstinline!auto&&! will result in an \romeovalue{lvalue} or an
\romeovalue{rvalue} reference depending on the value category of its initializer:

\begin{emcppslisting}[language=C++]
double doStuff();

      int val  = 3;
const int cval = 7;

// Deduction rules are the same as for (ù{\codeincomments{template <typename T> void deducer(T\&\&)}}ù):

auto&& lref1 = val;
    // Type of (ù{\codeincomments{lref1}}ù) is deduced to be (ù{\codeincomments{int\&}}ù).

auto&& lref2 = cval;
    // Type of (ù{\codeincomments{lref2}}ù) is deduced to be (ù{\codeincomments{const int\&}}ù).

auto&& rref = doStuff();
    // Type of (ù{\codeincomments{rref}}ù) is deduced to be (ù{\codeincomments{double\&\&}}ù).
\end{emcppslisting}
    
\noindent Similarly to references, explicitly specifying that a
pointer type is to be deduced is possible. If the supplied initializer is not of a
pointer type, the compiler will issue an error:

\begin{emcppslisting}[language=C++]
const auto* cptr = &cval;
    // Type of (ù{\codeincomments{cptr}}ù) is deduced to be (ù{\codeincomments{const int*}}ù).
    // same as argument for (ù{\codeincomments{template <typename T> void deducer(const T*)}}ù)

auto* cptr2 = cval;  // Error, cannot (ù{\codeincomments{auto*}}ù) from (ù{\codeincomments{cval}}ù)
\end{emcppslisting}
    
\noindent The compiler can also be instructed to deduce pointers to functions,
data members, and member functions (but see \intraref{annoyances-auto}{not-all-template-argument-deduction-constructs-are-allowed-for-auto}):

\begin{emcppslisting}[language=C++]
float freeF(float);

struct S
{
    double d_data;
    int memberF(long);
};

auto (*fptr)(float) = &freeF;
  // Type of (ù{\codeincomments{fptr}}ù) is deduced to be (ù{\codeincomments{float (*)(float)}}ù).
  // same as argument for (ù{\codeincomments{template <typename T> void deducer(T (*)(float))}}ù)

const auto S::* mptr = &S::d_data;
  // Type of (ù{\codeincomments{mptr}}ù) is deduced to be (ù{\codeincomments{const double S::*}}ù).
  // same as argument for (ù{\codeincomments{template <typename T> void deducer(T S::*)}}ù)

auto (S::* mfptr)(long) = &S::memberF;
  // Type of (ù{\codeincomments{mfptr}}ù) is deduced to be (ù{\codeincomments{int (S::*)(long)}}ù).
  // same as argument for (ù{\codeincomments{template <typename T> void deducer(T (S::*)(long))}}ù)
\end{emcppslisting}
    
\noindent Unlike references, pointer types may be deduced by \lstinline!auto! alone.
Therefore, different forms of \lstinline!auto! can be used to declare a
variable of a pointer type:

\begin{emcppslisting}[language=C++]
auto  cptr1 = &cval;  // (ù{\codeincomments{const int*}}ù)
auto *cptr2 = &cval;  //   "     "

auto   fptr1         = &freeF;  // (ù{\codeincomments{float (*)(float)}}ù)
auto  *fptr2         = &freeF;  //    "         "
auto (*fptr3)(float) = &freeF;  //    "         "

auto      mptr1 = &S::d_data;  // (ù{\codeincomments{double S::*}}ù)
auto S::* mptr2 = &S::d_data;  //    "     "

auto       mfptr1        = &S::memberF;  // (ù{\codeincomments{int (S::*)(long)}}ù)
auto (S::* mfptr2)(long) = &S::memberF;  //   "    "     "
\end{emcppslisting}
    
\noindent Note, however, that because regular and member pointers are
incompatible, \lstinline!auto*! cannot be used to deduce pointers to data
members and member functions:

\begin{emcppslisting}[language=C++]
auto *mptr3  = &S::d_data;   // Error, cannot deduce (ù{\codeincomments{auto*}}ù) from (ù{\codeincomments{\&S::d\_data}}ù)
auto *mfptr3 = &S::memberF;  // Error, cannot deduce (ù{\codeincomments{auto*}}ù) from (ù{\codeincomments{\&S::memberF}}ù)
\end{emcppslisting}
    
\noindent In addition, storage class specifiers as well as the \lstinline!constexpr!
(see \featureref{\locationc}{constexprvar}) specifier can be
applied to variables that use \lstinline!auto! in their declaration:

\begin{emcppslisting}[language=C++]
thread_local     const auto* logPrefix = "mylib";
static constexpr       auto  pi        = 3.1415926535f;
\end{emcppslisting}
    
\noindent Finally, \lstinline!auto! variables may be declared in any location that
allows declaring a variable supplied with an initializer with a single
exception of nonstatic data members (see \intraref{annoyances-auto}{auto-not-allowed-for-non-static-data-members}):

\begin{emcppslisting}[language=C++]
// namespace scope
auto globalNamespaceVar = 3.;

namespace ns
{
    static auto nsNamespaceVar = "...";
}

void f()
{
    // block scope
    constexpr auto blockVar = 'a';

    // condition of (ù{\codeincomments{if}}ù), (ù{\codeincomments{switch}}ù), and (ù{\codeincomments{while}}ù) statements
    if     (auto rc        = sendRequest())    { /* ... */ }
    switch (auto status    = responseStatus()) { /* ... */ }
    while  (auto keepGoing = haveMoreWork())   { /* ... */ }

    // init-statement of (ù{\codeincomments{for}}ù) loops
    for (auto it = vec.begin(); it != vec.end(); ++it) { /* ... */ }

    // range declaration of range-based (ù{\codeincomments{for}}ù) loops
    for (const auto& constVal : vec) { /* ... */ }
}

struct S
{
    // static data members
    static const auto k_CONSTANT = 11u;
};
\end{emcppslisting}
    

\subsection[Use Cases]{Use Cases}\label{use-cases-auto}

\subsubsection[Ensuring variable initialization]{Ensuring variable initialization}\label{ensuring-variable-initialization}

Consider a defect introduced due to mistakenly leaving a variable
uninitialized:

\begin{emcppslisting}[language=C++]
int recordCount;
while (cursor.next()) { ++recordCount; }  // Bug, undefined behavior
\end{emcppslisting}
    
\noindent Variables declared with \lstinline!auto! must be initialized. Use of
\lstinline!auto! might, therefore, prevent such defects:

\begin{emcppslisting}[language=C++]
auto recordCount; // Error, declaration of (ù{\codeincomments{recordCount}}ù) has no initializer
while (cursor.next()) { ++recordCount; }
\end{emcppslisting}
    
\noindent In addition, the initialization requirement might encourage a good
practice of reducing the scope of local variables.

\subsubsection[Avoiding redundant type name repetition]{Avoiding redundant type name repetition}\label{avoiding-redundant-type-name-repetition}

Certain function templates require that the caller explicitly specify
the type that the function uses as its return type. For example, the
\lstinline!std::make_shared<TYPE>! function returns a
\lstinline!std::shared_ptr<TYPE>!.{\cprotect\footnote{Often, such
functions are associated with \lstinline!optional! and \lstinline!variant!
types, which were standardized in C++17:

\begin{emcppslisting}[language=C++, style=footcode,]
std::variant<std::string, int, double> valueVariant;

// Without (ù{\fncodeincomments{auto}}ù):
std::optional<std::string> greeting = std::make_optional<std::string>("Hello");
const std::string& valueString = std::get<std::string>(valueVariant);

// With (ù{\fncodeincomments{auto}}ù):
auto greeting = std::make_optional<std::string>("Hello");
const auto& valueString = std::get<std::string>(valueVariant);
\end{emcppslisting}
      }} If a variable's type is specified explicitly, such declarations
redundantly repeat the type. The use of \lstinline!auto! obviates this repetition:

\begin{emcppslisting}[language=C++]
// Without (ù{\codeincomments{auto}}ù):
std::shared_ptr<RequestContext> context = std::make_shared<RequestContext>();
std::unique_ptr<Socket>         socket  = std::make_unique<Socket>();

// With (ù{\codeincomments{auto}}ù):
auto context = std::make_shared<RequestContext>();
auto socket  = std::make_unique<Socket>();
\end{emcppslisting}
    

\subsubsection[Preventing unexpected implicit conversions]{Preventing unexpected implicit conversions}\label{preventing-unexpected-implicit-conversions}

Use of \lstinline!auto! might prevent defects arising from explicitly
specifying a variable's type that is distinct --- yet implicitly
convertible --- from its initializer. As an example, the code below has
a subtle defect that can lead to performance degradation or incorrect
semantics:

\begin{emcppslisting}[language=C++]
std::map<int, User> users{/* ... */};
for (const std::pair<int, User>& idUserPair : users)
{
    // ...
}
\end{emcppslisting}
    
\noindent On every iteration, the \lstinline!idUserPair! will be bound to a
\emph{copy} of the corresponding pair in the \lstinline!users! map. This
happens because the type returned by dereferencing the \lstinline!map!'s
iterator is \lstinline!std::pair<const!~\lstinline!int,!~\lstinline!User>!, which
is implicitly convertible to \lstinline!std::pair<int,!~\lstinline!User>!.
Using \lstinline!auto! would allow the compiler to deduce the correct type
and avoid this unnecessary and potentially expensive copy:

\begin{emcppslisting}[language=C++]
std::map<int, User> users{/* ... */};
for (const auto& idUserPair : users)
{
    // (ù{\codeincomments{auto}}ù) is deduced as (ù{\codeincomments{std::pair<const int, User>}}ù).
}
\end{emcppslisting}
    

\subsubsection[Declaring variables of implementation-defined or compiler-synthesized types]{Declaring variables of implementation-defined or compiler-synthesized types}\label{declaring-variables-of-implementation-defined-or-compiler-synthesized-types}

Using \lstinline!auto! is the only way to declare variables of
implementation-defined or compiler-synthesized types, such as
lambda expressions (see \featureref{\locationc}{lambda}). While in some cases 
using type erasure to avoid the need to spell out the type is possible, doing so typically
comes with additional overhead. For example, storing a lambda closure in
a \lstinline!std::function! might entail an allocation on construction and
virtual dispatch upon every call:

\begin{emcppslisting}[language=C++]
std::function<void()> errorCallback0 = [&]{ saveCurrentWork(); };
    // OK, implicit conversion from anonymous closure type to
    // (ù{\codeincomments{std::function<void()>}}ù), which incurs additional overhead

auto errorCallback1 = [&]{ saveCurrentWork(); };
    // BETTER, deduces the compiler-synthesized type
\end{emcppslisting}
    

\subsubsection[Declaring variables of complex and deeply nested types]{Declaring variables of complex and deeply nested types}\label{declaring-variables-of-complex-and-deeply-nested-types}

\lstinline!auto! can be used to declare variables of types that are complex
or do not convey useful information. A typical example is avoiding the
need for spelling out the iterator type of a container:

\begin{emcppslisting}[language=C++]
void doWork(const std::vector<int>& data)
{
    // without (ù{\codeincomments{auto}}ù):
    for (std::vector<int>::const_iterator it = data.begin();
         it != data.end();
         ++it) {
        /* ... */
    }

    // with (ù{\codeincomments{auto}}ù):
    for (auto it = data.begin(); it != data.end(); ++it) { /* ... */ }
}
\end{emcppslisting}
    
\noindent Furthermore, the need for such types can arise, for example, when
storing intermediate results of \emcppsgloss[expression template]{expression
templates} whose types can
be deeply nested and unreadable and might even differ between versions of
the same library:

\begin{emcppslisting}[language=C++]
// without (ù{\codeincomments{auto}}ù):
TransformRange<
    FilterRange<decltype(employees), JoinedInYear>,
    &Employee::name
> newEmployeeNames =
    employees | filter(JoinedInYear(2019))
              | transform(&Employee::name);

// with (ù{\codeincomments{auto}}ù):
auto newEmployeeNames =
    employees | filter(JoinedInYear(2019))
              | transform(&Employee::name);
\end{emcppslisting}
    

\subsubsection[Improving resilience to library code changes]{Improving resilience to library code changes}\label{improving-resilience-to-library-code-changes}

\lstinline!auto! may be used to indicate that code using the variable
doesn't rely on a specific type but rather on certain requirements that
the type must satisfy. Such an approach might give library implementers
more freedom to change return types without affecting the semantics of
their clients' code in projects where automated large-scale refactoring
tools are not available (but see \intraref{potential-pitfalls-auto}{lack-of-interface-restrictions}). As an example,
consider the following library function:

\begin{emcppslisting}[language=C++]
std::vector<Node> getNetworkNodes();
    // Return a sequence of nodes in the current network.
\end{emcppslisting}
    
\noindent As long as the return value of the \lstinline!getNetworkNodes! function is
only used for iteration, that a \lstinline!std::vector! is
returned is not pertinent. If clients use \lstinline!auto! to initialize
variables storing the return value of this function, the implementers of
\lstinline!getNetworkNodes! can migrate from \lstinline!std::vector! to,
for example, \lstinline!std::deque! without breaking their clients'
code.

\begin{emcppslisting}[language=C++]
// without (ù{\codeincomments{auto}}ù):
const std::vector<Node> nodes = getNetworkNodes();
for (const Node& node : nodes) { /* ... */ }
    // prevents migration

// with (ù{\codeincomments{auto}}ù):
const auto nodes = getNetworkNodes();
for (const Node& node : nodes) { /* ... */ }
    // The return type of (ù{\codeincomments{getNetworkNodes}}ù) can be silently
    // changed while retaining correctness of the user code.
\end{emcppslisting}
    

\subsection[Potential Pitfalls]{Potential Pitfalls}\label{potential-pitfalls-auto}

\subsubsection[Compromised readability]{Compromised readability}\label{compromised-readability}

Use of \lstinline!auto! hides essential semantic information contained in a
variable's type, increasing cognitive load for readers. In conjunction
with unclear variable naming, disproportionate use of \lstinline!auto! can
make code difficult to read and maintain.

\begin{emcppslisting}[language=C++]
int main(int argc, char** argv)
{
    const auto args0 = parseArgs(argc, argv);
        // The behavior of (ù{\codeincomments{parseArgs}}ù) and operations available on (ù{\codeincomments{args0}}ù) is unclear.

    const std::vector<std::string> args1 = parseArgs(argc, argv);
        // It is clear what (ù{\codeincomments{parseArgs}}ù) does and what can be done with (ù{\codeincomments{args1}}ù).
}
\end{emcppslisting}
    
\noindent While reading the contract of the \lstinline!parseArgs!
function at least once may be necessary to fully understand its behavior, using an
explicit type's name at the call site helps readers understand its
purpose.

\subsubsection[Unintentional copies]{Unintentional copies}\label{unintentional-copies}

Since the rules for function template type deduction apply to
\lstinline!auto!, appropriate cv-qualifiers and declarator modifiers
(\lstinline!&!, \lstinline!&&!, \lstinline!*!, etc.) must be applied to avoid
unnecessary copies that might negatively affect both code performance
and correctness. For example, consider a function that capitalizes a
user's name:

\begin{emcppslisting}[language=C++]
void capitalizeName(User& user)
{
    if (user.name().empty())
    {
        return;
    }

    user.name()[0] = std::toupper(user.name()[0]);
}
\end{emcppslisting}
    
\noindent This function was then incorrectly refactored to avoid repetition of
the \lstinline!user.name()! invocation. However, a missing reference
qualification leads not only to an unnecessary copy of the string, but
also to the function failing to perform its job:

\begin{emcppslisting}[language=C++]
void capitalizeName(User& user)
{
    auto name = user.name();  // Bug, unintended copy

    if (name.empty())
    {
         return;
    }

    name[0] = std::toupper(name[0]);  // Bug, changes the copy
}
\end{emcppslisting}
    
\noindent Furthermore, even a fully cv-ref-qualified \lstinline!auto! might still
prove inadequate in cases as simple as introducing a variable for a
returned-temporary value. As an example, consider refactoring the
hypothetical \lstinline!useValue(getValue())! expression by first
initializing an intermediate variable with \lstinline!getValue()!:

\begin{emcppslisting}[language=C++]
auto&& tempValue = getValue();
useValue(tempValue);
\end{emcppslisting}
    
\noindent The above invocation of \lstinline!useValue! is not equivalent to the
original expression; the semantics of the program might have changed because \lstinline!tempValue! is an \romeovalue{lvalue} expression. To get
close to the original semantics, \lstinline!std::forward! and
\lstinline!decltype! must be used to propagate the original value category
of \lstinline!getValue()! to the invocation of \lstinline!useValue! (see
\featureref{\locationc}{forwardingref}):

\begin{emcppslisting}[language=C++]
auto&& tempValue = getValue();
useValue(std::forward<decltype(tempValue)>(tempValue));
\end{emcppslisting}
    
\noindent Note that, even with the latest changes, the code above achieves the same result but in a somewhat different way because
 \lstinline!std::forward<decltype(tempValue)>(tempValue)! is
an \romeovalue{xvalue} expression whereas \lstinline!getValue()! is a
\romeovalue{prvalue} expression; see \featureref{\locationc}{Rvalue-References}.

\subsubsection[Unexpected conversions (and lack of expected ones)]{Unexpected conversions (and lack of expected ones)}\label{unexpected-conversions-(and-lack-of-expected-ones)}

Compulsively declaring variables using \lstinline!auto!, even in cases
where the desired type has to be spelled out in the initializer, allows
explicit conversions to be used where they would not be applicable
otherwise. For an example of unintentional consequences of allowing such
conversions, consider a function template that is intended to combine
two \lstinline!chrono! duration values:

\begin{emcppslisting}[language=C++]
template <typename Duration1, typename Duration2>
std::chrono::seconds combine_durations(Duration1 d1, Duration2 d2)
{
    auto d = std::chrono::seconds{d1 + d2};
    // ...
}
\end{emcppslisting}
    
\noindent This function template will be successfully instantiated and compiled
even when its arguments are two \lstinline!int!s. Were \lstinline!auto! not
used in this situation --- i.e., were \lstinline!d! declared as
\lstinline!std::chrono::seconds!~\lstinline!d!~\lstinline!=!~\lstinline!d1!~\lstinline!+!~\lstinline!d2;!
--- the code would fail to compile because the conversion from
\lstinline!int! to \lstinline!seconds! is explicit, which would indicate a
likely defect in the caller's code.

Conversely, some conversions that would be expected to happen might be
missed when using \lstinline!auto! instead of an explicitly specified type.
For example, \lstinline!auto! might deduce a proxy type that might lead to
difficult-to-diagnose defects:

\begin{emcppslisting}[language=C++]
std::vector<bool> flags = loadFlags();

auto firstFlag = flags[0];  // deduces a proxy type, not (ù{\codeincomments{bool}}ù)
flags.clear();

if (firstFlag) // Bug, use-after-free: (ù{\codeincomments{flags}}ù) vector released its memory.
{
    // ...
}
\end{emcppslisting}
    

\subsubsection[Lack of interface restrictions]{Lack of interface restrictions}\label{lack-of-interface-restrictions}

Lack of any restrictions placed by \lstinline!auto! on the type that is
deduced might result in defects that could otherwise be detected at
compile time. Consider refactoring the \lstinline!getNetworkNodes! function
illustrated in \intraref{use-cases-auto}{improving-resilience-to-library-code-changes} to return \lstinline!std::deque<Node>! instead of
\lstinline!std::vector<Node>!:

\begin{emcppslisting}[language=C++]
std::deque<Node> getNetworkNodes();  // Return type changed from (ù{\codeincomments{std::vector<Node>}}ù).
    // Return a sequence of nodes in the current network.
\end{emcppslisting}
    
\noindent While code that uses \lstinline!auto! to store the result returned by
\lstinline!getNetworkNodes! only to subsequently iterate over it with a
range-based \lstinline!for! wouldn't be affected, the behavior of code that
relied on the contiguous layout of elements in \lstinline!std::vector!
objects \emph{silently} becomes undefined:

\begin{emcppslisting}[language=C++]
auto nodes = getNetworkNodes();
CLibraryProcessNodes(&nodes[0], nodes.size());  
  // exhibits UB after (ù{\codeincomments{std::vector}}ù)-to-(ù{\codeincomments{std::deque}}ù) change
\end{emcppslisting}
    
\noindent While specifying constraints on types deduced by
\lstinline!auto! with \lstinline!static_assert! is possible, doing so is often
cumbersome{\cprotect\footnote{C++20 introduced \emcppsgloss{concepts} ---
named type requirements --- as well as means to constrain
\lstinline!auto! with a specific concept, which can be used instead of
  \lstinline!static_assert! in such circumstances.}}:

\begin{emcppslisting}[language=C++]
const Packet* PacketCache::findFirstCorruptPacket() const
{
    auto it = std::begin(this->d_packets);

    static_assert(
        std::is_base_of<
            std::random_access_iterator_tag,
            std::iterator_traits<decltype(it)>::iterator_category>::value,
        "it must satisfy the requirements of a random access iterator.");

    /* ... */

    return it == std::end(this->d_packets) ? nullptr : &*it;
}
\end{emcppslisting}
    

\subsubsection[Obscuration of important properties of fundamental types]{Obscuration of important properties of fundamental types}\label{obscuration-of-important-properties-of-fundamental-types}

Use of \lstinline!auto! for variables of fundamental types might hide
important, context-sensitive considerations, such as overflow behavior
or a mix of signed and unsigned arithmetic. In the example below, the
\lstinline!lowercaseEncode! function will either work correctly or enter an
infinite loop depending on whether the type returned by
\lstinline!Encoder::encodedLengthFor! is signed.

\begin{emcppslisting}[language=C++]
void lowercaseEncode(std::string* result, const std::string& input)
{
    auto encodedLength = Encoder::encodedLengthFor(input);

    result->resize(encodedLength);
    Encoder::encode(result->begin(), input);

    while (--encodedLength >= 0)  // infinite loop if (ù{\codeincomments{encodedLength}}ù) is unsigned
    {
        (*result)[i] = std::tolower((*result)[i]);
    }
}
\end{emcppslisting}
    

\subsubsection[Surprising deduction for list initialization]{Surprising deduction for list initialization}\label{surprising-deduction-for-list-initialization}

\lstinline!auto! type-deduction rules differ from those of function
templates if brace-enclosed initializer list are used. Function template
argument deduction will always fail, whereas, according to C++11 rules,
\lstinline!std::initializer_list! will be deduced for \lstinline!auto!.

\begin{emcppslisting}[language=C++]
auto example0 = 0; // copy initialization, deduced as (ù{\codeincomments{int}}ù)
auto example1(0);  // direct initialization, deduced as (ù{\codeincomments{int}}ù)
auto example2{0};  // list initialization, deduced as (ù{\codeincomments{std::initializer\_list<int>}}ù)

template <typename T> void func(T);
func(0);   // (ù{\codeincomments{T}}ù) deduced as (ù{\codeincomments{int}}ù)
func({0}); // compilation failure
\end{emcppslisting}
    
\noindent This surprising behavior was, however, widely regarded as a mistake and
was formally rectified in C++17 with, e.g., \lstinline!auto!~\lstinline!i{0}!
deducing \lstinline!int!. Furthermore, mainstream compilers had applied
this deduction-rule change retroactively as early as GCC 5.1 and Clang
3.8, with the revised rule being applied even if \lstinline!-std=c++11!
flag is explicitly supplied.

Nonetheless, even with this retroactive fix, the effects of the
deduction rules when applied to braced initializer lists might be
puzzling. In particular, \lstinline!std::initializer_list! is deduced when
\emcppsgloss{copy initialization} is used instead of
\emcppsgloss{direct initialization}:

\begin{emcppslisting}[language=C++]
auto x1 = 1;    // (ù{\codeincomments{int}}ù)
auto x2(1);     //   "
auto x3{1};     //   "
auto x4 = {1};  // (ù{\codeincomments{std::initializer\_list<int>}}ù)

auto x5{1, 2};     // Error, direct-list-init requires exactly 1 element.
auto x6 = {1, 2};  // (ù{\codeincomments{std::initializer\_list<int>}}ù)
\end{emcppslisting}
    

\subsubsection[Deducing built-in arrays is problematic]{Deducing built-in arrays is problematic}\label{deducing-built-in-arrays-is-problematic}

Deducing built-in array types using \lstinline!auto! presents multiple
challenges. First, declaring an array of \lstinline!auto! is ill formed:

\begin{emcppslisting}[language=C++]
auto arr1[]  = {1, 2};  // Error, array of (ù{\codeincomments{auto}}ù) is not allowed.
auto arr2[2] = {1, 2};  // Error, array of (ù{\codeincomments{auto}}ù) is not allowed.
\end{emcppslisting}
    
\noindent Second, if the array bound is not specified, either the program does not
compile or\linebreak[4]%%%%%%
 \lstinline!std::initializer_list! is deduced instead of a
built-in array:

\begin{emcppslisting}[language=C++]
auto arr3 = {1, 2};  // (ù{\codeincomments{std::initializer\_list<int>}}ù)
auto arr4{1, 2};     // Error, direct-list-init requires exactly 1 element.
\end{emcppslisting}
    
\noindent Finally, attempting to circumvent this deficiency by using an alias
template (see \featureref{\locationa}{alias-declarations-and-alias-templates}) will result in code that
compiles but has undefined behavior:

\begin{emcppslisting}[language=C++]
template <typename TYPE, std::size_t SIZE>
using BuiltInArray = TYPE[SIZE];

auto arr5 = BuiltInArray<int, 2>{1, 2};
    // (ù{\codeincomments{arr5}}ù) is a dangling pointer of type (ù{\codeincomments{int *}}ù).
\end{emcppslisting}
    
\noindent Note that in this case such code also almost entirely defeats the
purpose of \lstinline!auto! since neither the array element's type nor the
array's bound are deduced.

With that said, using \lstinline!auto! to deduce
references to built-in arrays is straightforward:

\begin{emcppslisting}[language=C++]
int data[] = {1, 2};

      auto&  arr6 = data;                        //       (ù{\codeincomments{int (\&) [2]}}ù)
const auto&  arr7 = BuiltInArray<int, 2>{1, 2};  // (ù{\codeincomments{const int (\&) [2]}}ù)
      auto&& arr8 = BuiltInArray<int, 2>{1, 2};  //       (ù{\codeincomments{int (\&\&)[2]}}ù)
\end{emcppslisting}
    
\noindent Note that the \lstinline!arr7! and \lstinline!arr8! references in the code snippet immediately above extend
the lifetime of the temporary arrays that they bind to, so subscripting
them does not have the undefined behavior that subscripting
\lstinline!arr5! (in the previous code snippet) has.

\subsection[Annoyances]{Annoyances}\label{annoyances-auto}

\subsubsection[\lstinline!auto! not allowed for nonstatic data members]{{\SubsubsecCode auto} not allowed for nonstatic data members}\label{auto-not-allowed-for-non-static-data-members}

Despite C++11 allowing nonstatic data members to be initialized within
class definitions, \lstinline!auto! cannot be used to declare them:

\begin{emcppslisting}[language=C++]
class C
{
    auto d_i = 1;  // Error, nonstatic data member is declared with (ù{\codeincomments{auto}}ù).
};
\end{emcppslisting}
    

\subsubsection[Not all template argument deduction constructs are allowed for \lstinline!auto!]{Not all template argument deduction constructs are allowed for {\SubsubsecCode auto}}\label{not-all-template-argument-deduction-constructs-are-allowed-for-auto}

Despite \lstinline!auto! type deduction largely following the template
argument deduction rules, certain constructs that are allowed for
templates are not allowed for \lstinline!auto!. For example, when deducing
a pointer-to-data-member type, templates allow for deducing both the
data member type and the class type, whereas \lstinline!auto! can 
deduce only the former:

\begin{emcppslisting}[language=C++]
struct Node
{
    int   d_data;
    Node* d_next;
};

template <typename TYPE>
void deduceMemberTypeFn(TYPE Node::*);

             deduceMemberTypeFn   (&Node::d_data);  // OK, (ù{\codeincomments{int Node::*}}ù)
auto Node::* deduceMemberTypeVar = &Node::d_data;   // OK,   "     "


template <typename TYPE>
void deduceClassTypeFn(int TYPE::*);

            deduceClassTypeFn   (&Node::d_data);  // OK, (ù{\codeincomments{int Node::*}}ù)
int auto::* deduceClassTypeVar = &Node::d_data;   // Error, not allowed


template <typename TYPE>
void deduceBothTypesFn(TYPE* TYPE::*);

              deduceBothTypesFn   (&Node::d_next);  // OK, (ù{\codeincomments{Node* Node::*}}ù)
auto* auto::* deduceBothTypesVar = &Node::d_next;   // Error, not allowed
\end{emcppslisting}
    
\noindent Furthermore, deducing the parameter of a class template is also not
allowed:

\begin{emcppslisting}[language=C++]
std::vector<int> vectorOfInt;

template <typename TYPE>
void deduceVectorArgFn(const std::vector<TYPE>&);

                  deduceVectorArgFn   (vectorOfInt); // OK, (ù{\codeincomments{TYPE}}ù) is (ù{\codeincomments{int}}ù)
std::vector<auto> deduceVectorArgVar = vectorOfInt;  // Error, not allowed
\end{emcppslisting}
    
\noindent Instead, if \lstinline!auto! type deduction is desired in such cases,
\lstinline!auto! alone is suitable to deduce the type from the initializer:

\begin{emcppslisting}[language=C++]
auto deduceClassTypeVar = &Node::d_data;  // OK, (ù{\codeincomments{int   Node::*}}ù)
auto deduceBothTypesVar = &Node::d_next;  // OK, (ù{\codeincomments{Node* Node::*}}ù)

auto deduceVectorArgVar = vectorOfInt;    // OK, (ù{\codeincomments{std::vector<int>}}ù)
\end{emcppslisting}
    

\subsection[See Also]{See Also}\label{see-also}
%%%% NEED to be updated; waiting to hear back from Vittorio
\begin{itemize}
\item{\seealsoref{trailing-function-return-types}{\seealsolocationa}the \lstinline!auto! placeholder can be used to specify a function’s return type at the end of its signature.}
\item{\seealsoref{genericlambda}{\seealsolocationd}the \lstinline!auto! placeholder can be used in the argument list of a lambda to make its function call operator a template.}
\item{\seealsoref{Function-Return-Type-Deduction}{\seealsolocationf}the \lstinline!auto! placeholder can be used to deduce a function’s return type.}

\end{itemize}

\subsection[Further Reading]{Further Reading}

TODO
 
 