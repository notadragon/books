% 30 April 2021, cleaned up LaTeX and updated commands; to Josh for code check
% 30 April 2021 JMB, code check complete

\emcppsFeature{
    short={{\itshape Generic} Lambdas},
    long={Lambdas Having a Templated Call Operator},
}{genericlambda}
\label{generic-lambdas}
\setcounter{table}{0}
\setcounter{footnote}{0}
\setcounter{lstlisting}{0}
%\section[{\itshape Generic} Lambdas]{Lambdas Having a Templated Call Operator}\label{genericlambda}
%\subsection[Generic Lambdas]{Generic Lambdas}\label{generic-lambdas}



C++14 extends the \emcppsgloss[lambda expressions]{lambda expression} syntax of C++11 to allow a
\emph{templated} definition of the function-call operator belonging to
the \emcppsgloss{closure type}.

\subsection[Description]{Description}\label{description}

\emcppsgloss[generic lambda]{Generic lambdas} are a C++14 extension of C++11 lambda
expressions (see \featureref{}{}\textbf{[ AUs: We have Lambdas or Lambda Captures. There is no Lambda Expressions])}. The relationship
between a \emcppsgloss{generic lambda} and a nongeneric \emcppsgloss[lambda expressions]{lambda
expression} is the same as that between a function template and a
nontemplate function: The argument types are deduced at the point of
invocation rather than at definition.

Consider two \emcppsgloss{lambda expressions}, each of which simply returns
its argument:

\begin{emcppslisting}[emcppsbatch=e1,emcppsstandards={c++14}]
auto identityInt = [](int  a) { return a; };  // nongeneric lambda
auto identity =    [](auto a) { return a; };  // generic lambda
\end{emcppslisting}
    

\noindent Generic lambdas are characterized by the presence of one or more
\lstinline!auto! parameters, accepting arguments of any type. In the
example above, the first version is a nongeneric lambda having a
parameter of concrete type \lstinline!int!. The second version is a
\emcppsgloss{generic lambda} because its parameter uses the placeholder type
\lstinline!auto!. Unlike \lstinline!identityInt!, which is callable only for
\lstinline!int! arguments, \lstinline!identity! can be applied to any type
that can be passed by value:

\begin{emcppslisting}[emcppsbatch=e1]
int         a1 = identityInt(42);    // OK, (ù{\codeincomments{a1 == 42}}ù)
double      a2 = identityInt(3.14);  // Bug, (ù{\codeincomments{a2 == 3}}ù), truncation warning
const char* a3 = identityInt("hi");  // Error, cannot pass (ù{\codeincomments{"hi"}}ù) as (ù{\codeincomments{int}}ù)
int         a4 = identity(42);       // OK, (ù{\codeincomments{a4 == 42}}ù)
double      a5 = identity(3.14);     // OK, (ù{\codeincomments{a5 == 3.14}}ù)
const char* a6 = identity("hi");     // OK, (ù{\codeincomments{strcmp(a6, "hi") == 0}}ù)
\end{emcppslisting}
    

\noindent \emcppsgloss[generic lambda]{Generic lambdas} accomplish this compile-time polymorphism by
defining their function call operator --- \lstinline!operator()! --- as a
\emph{template}. Recall that the result of a \emcppsgloss[lambda expressions]{lambda expression} 
is a \emcppsgloss{closure object}, an object of unique type having a function
call operator; i.e., the \emcppsgloss{closure type} is a unique
\emcppsgloss{functor} class. The arguments defined in the \emcppsgloss[lambda expressions]{lambda
expression} become the arguments to the function call operator. The
following code transformation is roughly equivalent to the definitions
of the \lstinline!identityInt! and \lstinline!identity! \emcppsgloss[closure object]{closure
objects} from the example above:

\begin{emcppslisting}
struct __lambda_1  // compiler-generated name; not visible to the user
{
    int operator()(int x) const { return x; }
    // ...
};

struct __lambda_2  // compiler-generated name; not visible to the user
{
    template <typename __T>
    __T operator()(__T x) const { return x; }
    // ...
};

__lambda_1 identityInt = __lambda_1();
__lambda_2 identity    = __lambda_2();
\end{emcppslisting}
    

\noindent Note that the names \lstinline!__lambda_1!, \lstinline!__lambda_2!, and
\lstinline!__T! are for only descriptive purposes and are not available
to the user; the compiler may choose any name or no name for these
entities.

A \emcppsgloss{generic lambda} is any \emcppsgloss[lambda expressions]{lambda expression} having one
or more parameters declared using the placeholder type \lstinline!auto!.
The compiler generates a template parameter type for each \lstinline!auto!
parameter in the \emcppsgloss{generic lambda} and that type is substituted
for \lstinline!auto! in the function call operator's parameter list. In the
\lstinline!identity! example above, \lstinline!auto! \lstinline!x! is replaced
with \lstinline!__T! \lstinline!x!, where \lstinline!__T! is a new template
parameter type. When user code subsequently calls, e.g.,
\lstinline!identity(42)!, normal template type deduction takes place and
\lstinline!operator()<int>! is instantiated.

\subsubsection[Lambda capture and mutable closures]{Lambda capture and mutable closures}\label{lambda-capture-and-mutable-closures}

The \emcppsgloss{closure type} produced by a \emcppsgloss{generic lambda} is not a
class template. Rather, its function call operator and its
conversion-to-function-pointer operator (as we'll see later in \intrarefsimple{conversion-to-a-pointer-to-function}) are function templates. In
particular, the \emcppsgloss{lambda capture}, which creates member variables
within the \emcppsgloss{closure type}, is treated no differently than any
other \emcppsgloss[lambda expressions]{lambda expression}. Similarly, the \lstinline!mutable!
qualifier has the same effect for \emcppsgloss[generic lambda]{generic lambdas} as for
nongeneric lambdas:

\begin{emcppslisting}[emcppsstandards={c++14}]
#include <algorithm>  // (ù{\codeincomments{std::for\_each}}ù)
#include <iterator>   // (ù{\codeincomments{std::next}}ù)

template <typename FwdIter>
auto secondBiggest(FwdIter begin, FwdIter end)
    // Return the second-largest element in the range [(ù{\codeincomments{begin}}ù), (ù{\codeincomments{end}}ù)),
    // assuming at least two elements and that all values in the range
    // are distinct.
{
    auto ret = *std::next(begin);  // Set to 2nd element
    std::for_each(begin, end,
        [biggest = *begin, &ret](const auto& element) mutable
        {
            if (element < biggest) { return; }
            ret = biggest;
            biggest = element;
        }
    );

    return ret;
}
\end{emcppslisting}
    

\noindent The \lstinline!ret! declaration uses the placeholder \lstinline!auto! (see
\featureref{\locationc}{auto}) to deduce the variable's type to be
\lstinline!*std::next(begin)!, i.e., the element type for the input range.
The return type of \lstinline!secondBiggest! is also declared \lstinline!auto!
and is deduced from the type of \lstinline!ret! (see
\featureref{\locationf}{autoreturn}). The \emcppsgloss{generic lambda} being passed to
\lstinline!std::for_each! uses the C++14 initialized capture (see
\featureref{\locationd}{lambda-capture-expressions}) to initialize \lstinline!biggest! to the
largest value known so far. Because the lambda is declared
\lstinline!mutable!, it can update \lstinline!biggest! each time a larger
element is encountered. The \lstinline!ret! variable is also captured ---
by reference --- and is updated with the previous biggest value when a
new biggest value is encountered. Note that, at the point where
\lstinline!ret! appears in the \emcppsgloss{lambda capture}, its type has
already been deduced. When \lstinline!for_each! invokes the function call
operator, the type of the \lstinline!auto! parameter \lstinline!element! is
conveniently deduced to be the element type for the input range and is
thus the same reference type as \lstinline!ret! except with an added
\lstinline!const! qualifier.

\subsubsection[Constraints on deduced parameters]{Constraints on deduced parameters}\label{constraints-on-deduced-parameters}

A \emcppsgloss{generic lambda} may accept any mix of \lstinline!auto! and
non\lstinline!auto! parameters:

\begin{emcppslisting}[emcppsstandards={c++14}]
void g1()
{
    auto y1 = [](auto& a, int b, auto c) { a += b * c; };

    int    i = 5;
    double d = 1;

    y1(i, 2, 2);    // i is now 9
    y1(d, 3, 0.5);  // d is now 2.5
}
\end{emcppslisting}
    

\noindent If the \lstinline!auto! placeholder in a \emcppsgloss{generic lambda} parameter
is part of a type declaration that forms a potentially cv-qualified
reference, pointer, pointer-to-member, pointer-to-function, or
reference-to-function type, then the allowable arguments will be
restricted accordingly:

\begin{emcppslisting}[emcppsstandards={c++14}]
struct C1 { double d_i; };
double f1(int i);

auto y1 = [](const auto& r) { };  // match anything (read only)
auto y2 = [](auto&& r)      { };  //   "   anything (forwarding reference)
auto y3 = [](auto& r)       { };  //   "   only (ù{\emphincomments{lvalues}}ù)
auto y4 = [](auto* p)       { };  //   "     "  pointers
auto y5 = [](auto(*p)(int)) { };  //   "     "      "    to functions
auto y6 = [](auto C1::* pm) { };  //   "     "      "    to data members of (ù{\codeincomments{C1}}ù)

void g2()
{
    int       i1 = 0;
    const int i2 = 1;

    y1(i1);        // OK, (ù{\codeincomments{r}}ù) has type (ù{\codeincomments{const int\&}}ù)
    y2(i1);        // OK, (ù{\codeincomments{r}}ù) has type (ù{\codeincomments{int\&}}ù)

    y3(5);         // Error, argument is not an (ù{\emphincomments{lvalue}}ù)
    y3(i1);        // OK, (ù{\codeincomments{r}}ù) has type (ù{\codeincomments{int\&}}ù)
    y3(i2);        // OK, (ù{\codeincomments{r}}ù) has type (ù{\codeincomments{const int\&}}ù)

    y4(i2);        // Error, (ù{\codeincomments{i2}}ù) is not a pointer
    y4(&i2);       // OK, (ù{\codeincomments{p}}ù) has type (ù{\codeincomments{const int*}}ù)

    y5(&f1);       // OK, (ù{\codeincomments{p}}ù) has type (ù{\codeincomments{double (*)(int)}}ù)

    y6(&C1::d_i);  // OK, (ù{\codeincomments{pm}}ù) has type (ù{\codeincomments{double C1::*}}ù)
}
\end{emcppslisting}
    

\noindent To understand how \lstinline!y1! and \lstinline!y2! match any argument type,
recall that \lstinline!auto! is a placeholder for a template type argument,
say, \lstinline!__T!. As usual, \lstinline!const! \lstinline!__T&!
\lstinline!r! can bind to a \lstinline!const! or non\lstinline!const!
\romeovalue{lvalue} or a temporary value created from an \romeovalue{rvalue}. The
argument \lstinline!__T&&! \lstinline!r! is a \emcppsgloss{forwarding
reference} (see {\featureref{\locationc}{forwardingref}); \lstinline!__T! will be
deduced to an \romeovalue{rvalue} if the argument is an \romeovalue{rvalue} and to
an \romeovalue{lvalue} reference otherwise. Because the parameter type for
\lstinline!r! is unnamed --- we invented the name \lstinline!__T! for
descriptive purposes only --- we must use \lstinline!decltype(r)! to refer
to the deduced type:

\begin{emcppslisting}[emcppsstandards={c++14}]
#include <utility>  // (ù{\codeincomments{std::move}}ù), (ù{\codeincomments{std::forward}}ù)
#include <cassert>  // standard C (ù{\codeincomments{assert}}ù) macro

struct C2
{
    int d_value;

    explicit C2(int i)     : d_value(i)                { }
    C2(const C2& original) : d_value(original.d_value) { }
    C2(C2&& other)         : d_value(other.d_value)    { other.d_value = 99; }
};

void g3()
{
    auto y1 = [](const auto& a) { C2 v(a); };
    auto y2 = [](auto&&      a) { C2 v(std::forward<decltype(a)>(a)); };

    C2 a(1);

    y1(a);            assert(1  == a.d_value);  // Copies from (ù{\codeincomments{a}}ù)
    y1(std::move(a)); assert(1  == a.d_value);  //    "     "  (ù{\codeincomments{a}}ù)
    y2(a);            assert(1  == a.d_value);  //    "     "  (ù{\codeincomments{a}}ù)
    y2(std::move(a)); assert(99 == a.d_value);  // Moves    "  (ù{\codeincomments{a}}ù)
}
\end{emcppslisting}
    

\noindent In this example, \lstinline!y1! always invokes the copy constructor for
\lstinline!C2! because \lstinline!a! has type \lstinline!const! \lstinline!C2&!
regardless of whether we instantiate it with an \romeovalue{lvalue} or
\romeovalue{rvalue} reference to \lstinline!C2!. Conversely, \lstinline!y2! forwards
the \emcppsgloss{value category} of its argument to the \lstinline!C2!
constructor using \lstinline!std::forward! according to the common idiom
for forwarding references. If passed an \romeovalue{lvalue} reference, the
copy constructor is invoked; otherwise, the move constructor is invoked.
We can tell the difference because \lstinline!C2! has a move constructor
that puts the special value \lstinline!99! into the moved-from object.

The \lstinline!auto! placeholder in a \emcppsgloss{generic lambda} parameter
cannot be a type argument in a template specialization, a parameter type
in the prototype of a function reference or function pointer, or the
class type in a pointer to member{\cprotect\footnote{GCC 10.2 does allow
\lstinline!auto! in both template arguments and function prototype
parameters and deduces the template parameter type in the same way as
for a regular function template. MSVC 19.28 allows \lstinline!auto! in
the parameter list for a function reference or function pointer but
  not in the other two contexts.}}:

\begin{emcppslisting}[emcppsbatch=e2,emcppsstandards={c++14}]
#include <vector>  // (ù{\codeincomments{std::vector}}ù)
auto y7 = [](const std::vector<auto>& x) { };  // Error, invalid use of (ù{\codeincomments{auto}}ù)
auto y8 = [](double (*f)(auto)) { };           // Error,    "     "   "    "
auto y9 = [](int auto::* m) { };               // Error,    "     "   "    "
\end{emcppslisting}
    

\noindent Because of this restriction, there are no contexts where more than one
\lstinline!auto! is allowed to appear in the declaration of a single lambda
parameter. Template parameters \emph{are} allowed in these contexts for
regular function templates, so \emcppsgloss[generic lambda]{generic lambdas} are less
expressive than handwritten functor objects in this respect:

\begin{emcppslisting}[emcppsbatch=e2]
struct manualY7
{
    template <typename T>
    void operator()(const std::vector<T>& x) const { }  // OK, can deduce (ù{\codeincomments{T}}ù)
};

struct manualY8
{
    template <typename T>
    void operator()(double (*f)(T)) const { }  // OK, can deduce (ù{\codeincomments{T}}ù)
};

struct manualY9
{
    template <typename R, typename T>
    void operator()(R T::* m) const { }  // OK, can deduce (ù{\codeincomments{R}}ù) and (ù{\codeincomments{T}}ù)
};
\end{emcppslisting}
    

\noindent Not only do \lstinline!manualY7!, \lstinline!manualY8!, and \lstinline!manualY9!
successfully deduce \lstinline!T! where \lstinline!y7!, \lstinline!y8!, and
\lstinline!y9! could not deduce \lstinline!auto!, \lstinline!manualY9! deduces
two template arguments for a single function parameter. There is a
trade-off between the benefits of \emcppsgloss{lambda expressions}, such as
defining the function in place at the point of use, and the
pattern-matching power of manually written function templates. See
{Annoyances --- Cannot use full power of template-argument deduction}.

A default value on an \lstinline!auto! parameter, while allowed, is not
useful because it defaults only the \emph{value} and not the \emph{type}
of the template parameter. Invocation of such a \emcppsgloss{generic lambda}
requires the programmer to either supply a value for the argument, which
defeats the point of a defaulted argument, or explicitly instantiate
\lstinline!operator()!, which is gratuitously awkward:

\begin{emcppslisting}[emcppsstandards={c++14}]
void g1()
{
    auto y = [](auto a = 3) { return a * 2; };
    y(5);                // OK, returns 10
    y();                 // Error, cannot deduce type for parameter (ù{\codeincomments{a}}ù)
    y.operator()<int>(); // OK, returns 6
}
\end{emcppslisting}
    

\subsubsection[Variadic generic lambdas]{Variadic generic lambdas}\label{variadic-generic-lambdas}

If a placeholder argument to a \emcppsgloss{generic lambda} is followed by an
ellipsis (\lstinline!...!), then the parameter becomes a variadic parameter
pack and the function call operator becomes a variadic function
template; see \featureref{\locationc}{variadictemplate}:

\begin{emcppslisting}[emcppsstandards={c++14}]
#include <tuple>  // (ù{\codeincomments{std::tuple}}ù), (ù{\codeincomments{std::make\_tuple}}ù)
auto y11 = [](int i, auto&&... args)
{
    return std::make_tuple(i, std::forward<decltype(args)>(args)...);
};

std::tuple<int, const char*, double> tpl1 = y11(3, "hello", 1.2);
\end{emcppslisting}
    

\noindent The \lstinline!y11! \emcppsgloss{closure object} forwards all of its arguments
to \lstinline!std::make_tuple!. The first argument must have type
convertible to \lstinline!int!, but the remaining arguments can have any
type. Assuming the invented name \lstinline!__T! for the template type
parameter, the generated function call operator for \lstinline!y11! would
have a variadic template parameter list:

\begin{emcppslisting}
struct __lambda_3
{
    template <typename... __T> void operator()(int, __T&&... args) const
    {
        /* ... */
    }
};
\end{emcppslisting}
    

\noindent The standard limitations on variadic function templates apply. For
example, only a variadic parameter pack at the end of the argument list
will match function-call arguments at the point of invocation. In
addition, because \lstinline!auto! is not permitted in a
template-specialization parameter, the usual methods of defining
function templates with multiple variadic parameter packs do not work
for \emcppsgloss[generic lambda]{generic lambdas}:

\begin{emcppslisting}
// Attempt to define a lambda expression with two variadic parameter packs.
auto y12 = [](std::tuple<auto...>&, auto...args) { };
    // Error, (ù{\codeincomments{auto}}ù) is a template argument in (ù{\codeincomments{tuple}}ù) specialization.
\end{emcppslisting}
    

\subsubsection[Conversion to a pointer to function]{Conversion to a pointer to function}\label{conversion-to-a-pointer-to-function}

A nongeneric \emcppsgloss[lambda expressions]{lambda expression} with an empty \emcppsgloss{lambda
capture} can be converted implicitly to a function pointer with the same
signature. A \emcppsgloss{generic lambda} with an empty \emcppsgloss{lambda
capture} can similarly be converted to a regular function pointer, where
the parameters in the prototype of the target pointer type drive
deduction of the appropriate \lstinline!auto! parameters in the
\emcppsgloss{generic lambda} signature:

\begin{emcppslisting}[emcppsbatch=e3,emcppsstandards={c++14}]
auto y1 = [](int a, char b) { return a; };   // nongeneric lambda
int (*f1)(int, char) = y1;                   // OK, conversion to pointer

auto y2 = [](auto a, auto b) { return a; };  // generic lambda
int    (*f2)(int, int)    = y2;  // OK, instantiates (ù{\codeincomments{operator()<int, int>}}ù)
double (*f3)(double, int) = y2;  // OK, instantiates (ù{\codeincomments{operator()<double, int>}}ù)
char   (*f4)(int, char)   = y2;  // Error, incorrect return type
\end{emcppslisting}
    

\noindent If the function target pointer is a variable template (see
\featureref{\locationb}{variable-templates}), then the deduction of the arguments is
delayed until the variable template itself is instantiated:

\begin{emcppslisting}[emcppsbatch=e3]
template <typename T> int (*f5)(int, T) = y2;  // variable template
int (*f6)(int, short) = f5<short>;             // instantiate (ù{\codeincomments{f5}}ù)
\end{emcppslisting}
    

\noindent Each function pointer is produced by calling a conversion operator on
the \emcppsgloss{closure object}. In the case of the \emcppsgloss{generic lambda},
the conversion operator is also a template. Template-argument deduction
and return-type deduction are performed on the conversion operator, then
the conversion operator instantiates the function call operator.
Intuitively, it is as though the function call operator were a static
member function template of the closure and the conversion operator
returned a pointer to it.

\subsection[Use Cases]{Use Cases}\label{use-cases}

\subsubsection[Reusable lambda expressions]{Reusable lambda expressions}\label{reusable-lambda-expressions}

One of the benefits of \emcppsgloss{lambda expressions} is that they can be
defined within a function, close to the point of use. Saving a
\emcppsgloss[lambda expressions]{lambda expression} in a variable allows it to be reused within
the function. This reusability is greater for \emcppsgloss[generic lambda]{generic lambdas}
than for nongeneric lambdas, just as function templates are more
reusable than ordinary functions. Consider, for example, a function that
partitions a vector of strings and a vector of vectors based on the
length of each element:

\begin{emcppslisting}[emcppsstandards={c++14}]
#include <vector>     // (ù{\codeincomments{std::vector}}ù)
#include <string>     // (ù{\codeincomments{std::string}}ù)
#include <algorithm>  // (ù{\codeincomments{std::partition}}ù)

void partitionByLength(std::size_t                    pivotLen,
                       std::vector<std::vector<int>>& v1,
                       std::vector<std::string>&      v2)
{
    auto condition = [pivotLen](const auto& e) { return e.size() < pivotLen; };

    std::partition(v1.begin(), v1.end(), condition);
    std::partition(v2.begin(), v2.end(), condition);
}
\end{emcppslisting}
    

\noindent The \lstinline!condition! \emcppsgloss{generic lambda} can be used to partition
both \lstinline!vector!s because its function call operator can be
instantiated on either element type. The capture of \lstinline!pivotLen! is
performed only once, when the \emcppsgloss[lambda expressions]{lambda expression} is evaluated to
yield \lstinline!condition!.

\subsubsection[Applying a lambda to each element of a tuple]{Applying a lambda to each element of a tuple}\label{applying-a-lambda-to-each-element-of-a-tuple}

An \lstinline!std::tuple! is a collection of objects having heterogeneous
types. We can apply a functor to each element of a \lstinline!tuple! by
using some metaprogramming features of the C++14 Standard Library:

\begin{emcppslisting}[emcppsbatch=e4,emcppsstandards={c++14}]
#include <utility>  // (ù{\codeincomments{std:::index\_sequence}}ù), (ù{\codeincomments{std::make\_index\_sequence}}ù)
#include <tuple>    // (ù{\codeincomments{std::tuple}}ù), (ù{\codeincomments{std::tuple\_size}}ù), (ù{\codeincomments{std::get}}ù)

template <typename Tpl, typename F, std::size_t... I>
void visitTupleImpl(Tpl& t, F& f, std::index_sequence<I...>)
{
    auto discard = { (f(std::get<I>(t)), 0)... };
}

template <typename Tpl, typename F>
void visitTuple(Tpl& t, F f)
{
    visitTupleImpl(t, f,
                   std::make_index_sequence<std::tuple_size<Tpl>::value>());
}
\end{emcppslisting}
    

\noindent The \lstinline!visitTuple! function uses \lstinline!make_index_sequence! to
generate a compile-time pack of sequential indexes, from 0 up to, but
not including, the number of elements in the \lstinline!tuple!-like
argument \lstinline!t!. The original arguments, along with this pack of
indexes, are passed to \lstinline!visitTupleImpl!, which applies each index
to \lstinline!t! and then calls the functor \lstinline!f! on the resulting
element.{\cprotect\footnote{This pattern for using
\lstinline!make_index_sequence! with \lstinline!tuple!-like objects is
explained in a number of places on the Internet (see \cite{prowl13}),
and a basic knowledge of variadic templates
  \featureref{\locationc}{variadictemplate} is critical to understanding this
  idiom.}} The implementation uses an idiom that discards the sequence
of return values (if any) from the calls to \lstinline!f!: Regardless of
the type of value returned by \lstinline!f(std::get<I>(t))!, even if it's
\lstinline!void!, the comma expression \lstinline!(f(std::get<I>(t)),!
\lstinline!0)! always yields the integer \lstinline!0!. The order of
evaluation of elements in an \lstinline!initializer_list! is guaranteed
left to right. The result of this expansion is an
\lstinline!initializer_list! used to initialize \lstinline!discard!, which,
as its name suggests, is then discarded.

Once we have a \lstinline!visitTuple! function, we can use it to apply a
\emcppsgloss{generic lambda} to the elements of a \lstinline!tuple!:

\begin{emcppslisting}[emcppsbatch=e4]
#include <ostream>  // (ù{\codeincomments{std::ostream}}ù), (ù{\codeincomments{std::endl}}ù)
void f1(std::ostream& os)
{
    std::tuple<int, float, const char*> t{3, 4.5, "six"};
    visitTuple(t, [&os](const auto& v){ os << v << ' '; });
    os << std::endl;
}
\end{emcppslisting}
    

\noindent The first line constructs a \lstinline!tuple! with three different element
types. The second line then calls \lstinline!visitTuple! to visit each
element of \lstinline!t! and apply a lambda function to it. The
\emcppsgloss{lambda capture} stores a reference to the output stream, and the
\emcppsgloss{lambda body} prints the current element to the output stream.
This code would not work if the lambda were not generic because the type
of \lstinline!v! is different for each element in \lstinline!t!.

\subsubsection[Terse, robust lambdas]{Terse, robust lambdas}\label{terse,-robust-lambdas}

Often, function objects are more convenient to write as \emcppsgloss[generic lambda]{generic
lambdas}, rather than nongeneric lambdas, even if they are used only
once and their genericity is not fully exploited. Consider the case
where a lambda's parameters have long, elaborate types:

\begin{emcppslisting}[emcppsbatch=e5,emcppsstandards={c++14}]
#include <iterator>   // (ù{\codeincomments{std::iterator\_traits}}ù)
#include <algorithm>  // (ù{\codeincomments{std::sort}}ù)

template <typename Iterator>
void f1(Iterator begin, Iterator end)
    // Sort [(ù{\codeincomments{begin}}ù), (ù{\codeincomments{end}}ù)) using a nongeneric lambda
{
    std::sort(begin, end,
              [](typename std::iterator_traits<Iterator>::reference a,
                 typename std::iterator_traits<Iterator>::reference b)
              {
                  return a < b;
              });
}
\end{emcppslisting}
    

\noindent The types for parameters \lstinline!a! and \lstinline!b! of the comparison
\emcppsgloss[lambda expressions]{lambda expression}
 are not easy to type or read. Compare this
code with similar code using \emcppsgloss[generic lambda]{generic lambdas}:

\begin{emcppslisting}[emcppsbatch=e5]
template <typename Iterator>
void f2(Iterator begin, Iterator end)
    // Sort [(ù{\codeincomments{begin}}ù), (ù{\codeincomments{end}}ù)) using a generic lambda
{
    std::sort(begin, end,
              [](const auto& a, const auto& b) { return a < b; });
}
\end{emcppslisting}
    

\noindent The code is simpler to write and read because it takes advantage of the
\emcppsgloss[lambda expressions]{lambda expression}'s being defined at the point of use; even
though the argument types are not written out, their meaning is still
clear. A \emcppsgloss{generic lambda} is also more robust when the types
involved change, even in a small way, as it adapts to changes in
argument types as long as they support the same interface.

\subsubsection[Recursive lambdas]{Recursive lambdas}\label{recursive-lambdas}

Since neither generic nor nongeneric \emcppsgloss{lambda expressions} have
names, defining a lambda that calls itself recursively is tricky. One
way to accomplish recursion is by passing the lambda as a parameter to
itself. This technique requires a \emcppsgloss{generic lambda} so that it can
deduce its own type from its argument:

\begin{emcppslisting}[emcppsstandards={c++14}]
auto fib = [](auto self, int n) -> int  // Compute the (ù{\codeincomments{n}}ù)th Fibonacci number.
{
    if (n < 2) { return n; }
    return self(self, n - 1) + self(self, n - 2);
};

int fib7 = fib(fib, 7);  // returns 13
\end{emcppslisting}
    

\noindent Notice that when we invoke the recursive lambda, we pass it as an
argument to itself, both to the external call and to the internal
recursive calls. To avoid this somewhat awkward interface, a special
function object called a \emcppsgloss[y combinator]{Y Combinator} can be
used.{\cprotect\footnote{\cite{hindley86}}} The \emcppsgloss[y combinator]{Y Combinator}
object holds the \emcppsgloss{closure object} to be invoked recursively and
passes it to itself:

\begin{emcppslisting}[emcppsbatch=e6,emcppsstandards={c++14}]
#include <utility>  // (ù{\codeincomments{std::move}}ù), (ù{\codeincomments{std::forward}}ù)

template <typename Lambda>
class Y_Combinator {
    Lambda d_lambda;

public:
    Y_Combinator(Lambda&& lambda) : d_lambda(std::move(lambda)) { }

    template <typename... Args>
    decltype(auto) operator()(Args&&...args) const
    {
        return d_lambda(*this, std::forward<Args>(args)...);
    }
};

template <typename Lambda>
Y_Combinator<Lambda> Y(Lambda&& lambda) { return std::move(lambda); }
\end{emcppslisting}
    

\noindent The function call operator for \lstinline!Y_Combinator! is a
\emcppsgloss{variadic function template} (see
\featureref{\locationc}{variadictemplate}) that passes itself to the stored
\emcppsgloss{closure object}, \lstinline!d_lambda!, along with zero or more
additional arguments supplied by the caller. Thus, \lstinline!d_lambda!
and the \lstinline!Y_Combinator! are mutually-recursive functors. The
\lstinline!Y! function template constructs a \lstinline!Y_Combinator! from a
\emcppsgloss[lambda expressions]{lambda expression}.

To use a \lstinline!Y_Combinator!, pass a recursive \emcppsgloss{generic
lambda} to \lstinline!Y!; the resulting object is the one that you would
call from code:

\begin{emcppslisting}[emcppsbatch=e6]
auto fib2 = Y([](auto self, int n) -> int
{
    if (n < 2) { return n; }
    return self(n - 1) + self(n - 2);
});

int fib8 = fib2(8);  // returns 21
\end{emcppslisting}
    

\noindent Note that the recursive lambda still needs to take \lstinline!self! as an
argument, but, because \lstinline!self! is a \lstinline!Y_Combinator!, it
does not need to pass \lstinline!self! to itself. Unfortunately, we must
now specify the return type of the lambda because the compiler cannot
deduce the return type of the mutually recursive invocations of
\lstinline!self!. The usefulness of a \emcppsgloss[y combinator]{Y Combinator} in C++ is
debatable given alternative, often simpler ways to achieve recursion,
including using ordinary function templates instead of \emcppsgloss{lambda
expressions}.{\cprotect\footnote{Derevenets suggested that
\lstinline!y_combinator! should be part of the C++ Standard Library (see
\cite{derevenets16}), but the proposal was rejected for addressing
  a problem deemed not worth solving.}}

\subsubsection[Conditional instantiation]{Conditional instantiation}\label{conditional-instantiation}

Because a \emcppsgloss{generic lambda} defines a function template that is
not instantiated unless it is called, it is possible to put code into a
\emcppsgloss{generic lambda} that would not compile for certain argument
types. We can, thus, selectively instantiate calls to \emcppsgloss[generic lambda]{generic
lambdas} based on some compile-time conditional expression similar to
the \lstinline!if! \lstinline!constexpr! feature introduced in C++17:

\begin{emcppslisting}[emcppsbatch=e7,emcppsstandards={c++14}]
#include <iostream>     // (ù{\codeincomments{std::cout}}ù)
#include <type_traits>  // (ù{\codeincomments{std::is\_pointer}}ù)
#include <functional>   // (ù{\codeincomments{std::reference\_wrapper}}ù)
#include <cassert>      // standard C (ù{\codeincomments{assert}}ù) macro

// Identity functor: Each call to (ù{\codeincomments{operator()}}ù) returns its argument unchanged.
struct Identity
{
    template <typename T>
    decltype(auto) operator()(T&& x) { return std::forward<T>(x); }
};

template <typename F1, typename F2>
decltype(auto) ifConstexprImpl(std::true_type, F1&& f1, F2&&)
    // call (ù{\codeincomments{f1}}ù), which is the "then" branch of (ù{\codeincomments{ifConstexpr}}ù)
{
    return std::forward<F1>(f1)(Identity{});
}

template <typename F1, typename F2>
decltype(auto) ifConstexprImpl(std::false_type, F1&&, F2&& f2)
    // call (ù{\codeincomments{f2}}ù), which is the "else" branch of (ù{\codeincomments{ifConstexpr}}ù)
{
    return std::forward<F2>(f2)(Identity{});
}

template <bool Cond, typename F1, typename F2>
decltype(auto) ifConstexpr(F1&& f1, F2&& f2)
    // If the compile-time condition (ù{\codeincomments{Cond}}ù) is true, return the result of
    // invoking (ù{\codeincomments{f1}}ù), else return the result of invoking (ù{\codeincomments{f2}}ù).  The
    // invocations of (ù{\codeincomments{f1}}ù) and (ù{\codeincomments{f2}}ù) are each passed an instance of (ù{\codeincomments{Identity}}ù)
    // as their sole argument.
{
    using CondT = std::integral_constant<bool, Cond>;
    return ifConstexprImpl(CondT{}, std::forward<F1>(f1), std::forward<F2>(f2));
}
\end{emcppslisting}
    

\noindent The \lstinline!Identity! functor class, as its name suggests, returns its
argument unchanged, preserving both its type and \emcppsgloss{value category}
(\romeovalue{lvalue} or \romeovalue{rvalue}). This functor exists solely to be
passed to a \emcppsgloss{generic lambda}, which will then invoke it as
described below.

The two overloads of \lstinline!ifConstexprImpl! each take two functor
arguments, only one of which is used. The first overload invokes only
its first argument; the second overload invokes only its second
argument. In both cases, an \lstinline!Identity! object is passed as the
only argument to the functor invocation.

The \lstinline!ifConstexpr! function template, in addition to taking two
functor arguments, must be instantiated with an explicit compile-time
Boolean value. It calls one of the two \lstinline!ifConstexprImpl!
functions such that its first argument is instantiated and invoked if
the condition is \lstinline!true! --- the ``then'' clause of the simulated
\lstinline!if! \lstinline!constexpr! --- and the second argument is
instantiated and invoked otherwise --- the ``else'' clause of the
simulated \lstinline!if! \lstinline!constexpr!.

To see how \lstinline!ifConstexpr! achieves conditional instantiation,
let's write a function that dereferences an object. The object being
dereferenced can be a pointer, an \lstinline!std::reference_wrapper!, or a
class that behaves like \lstinline!reference_wrapper!. We present both a
C++17 version that uses \lstinline!if! \lstinline!constexpr! and a C++14
version that uses \lstinline!ifConstexpr!, above:

\begin{emcppslisting}[emcppsbatch=e7]
template <typename T>
decltype(auto) objectAt(T ref)
    // Generalized dereference for pointers and (ù{\codeincomments{reference\_wrapper}}ù)-like
    // objects: If (ù{\codeincomments{T}}ù) is a pointer type, return (ù{\codeincomments{*ref}}ù); otherwise, return
    // (ù{\codeincomments{ref.get()}}ù).
{
#if __cplusplus >= 201703
    // C++17 version
    if constexpr (std::is_pointer<T>::value) { return *ref; }
    else                                     { return ref.get(); }
#else
    // C++14 version
    return ifConstexpr<std::is_pointer<T>::value>(
        [=](auto dependent) -> decltype(auto) { return *dependent(ref); },
        [=](auto dependent) -> decltype(auto) { return dependent(ref).get(); });
#endif
}
\end{emcppslisting}
    

\noindent Looking at the C++17 version first, we see that if \lstinline!ref! is a
pointer, then we return \lstinline!*ref!; otherwise, we return
\lstinline!ref.get()!. In both cases, the code would not compile if the
branch not taken were instantiated; i.e., \lstinline!*ref! is ill formed
for \lstinline!std::reference_wrapper!, and \lstinline!ref.get()! is ill
formed for pointers. The code depends on the ill-formed branch being
discarded at compile time.

The C++14 version needs to follow a very specific idiom whereby
each conditional branch is represented as a \emcppsgloss{generic lambda} with
one \lstinline!auto! parameter that is expected to be an \lstinline!Identity!
object. Because it is a template parameter, \lstinline!dependent! is a
\emcppsgloss{dependent type}. In C++ template parlance, a \emcppsgloss{dependent
type} is one that cannot be known until template instantiation.
Critically, the compiler will not check semantic correctness of an
expression containing a value of \emcppsgloss{dependent type} unless and
until the template is instantiated with a concrete type. Any such
expression that uses a value of \emcppsgloss{dependent type} also has
\emcppsgloss{dependent type}. By wrapping \lstinline!ref! in a call to
\lstinline!dependent!, therefore, we ensure that the compiler will not test
if \lstinline!*ref! or \lstinline!ref.get()! is valid until the
\emcppsgloss{generic lambda} is instantiated. Our implementation of
\lstinline!ifConstexpr! ensures that only the \emph{correct} lambda is
instantiated, thus preventing a compiler error.{\cprotect\footnote{Paul
Fultz suggested the idea for using an identity functor to mark
  expressions as having \emcppsgloss{dependent type} (\cite{fultz14}).}}

A simple test for \lstinline!objectAt! instantiates it with both pointer
and \lstinline!reference_wrapper! arguments and verifies that the value
returned has both the correct value and address:

\begin{emcppslisting}[emcppsbatch=e7]
void f1()
{
    int i = 8;

    int&       i1 = objectAt(&i);
    int&       i2 = objectAt(std::reference_wrapper<int>(i));
    const int& i3 = objectAt(std::reference_wrapper<const int>(i));

    assert(8 == i1);
    assert(&i1 == &i);
    assert(&i2 == &i);
    assert(&i3 == &i);
}
\end{emcppslisting}
    

\subsection[Potential Pitfalls]{Potential Pitfalls}\label{potential-pitfalls}

Although the generic feature is generally safe, \emcppsgloss[generic lambda]{generic lambdas}
are an incremental advance over regular \emcppsgloss{lambda expressions}. All
pitfalls of the nongeneric feature apply to this feature as well; see
\featureref{\locationc}{lambda}.

\subsection[Annoyances]{Annoyances}\label{annoyances}

\subsubsection[Cannot use full power of template-argument deduction]{Cannot use full power of template-argument deduction}\label{cannot-use-full-power-of-template-argument-deduction}

A normal function template can constrain template parameters to patterns
that include argument prototypes and template instantiations:

\begin{emcppslisting}[emcppsbatch=e8,emcppsstandards={c++14}]
#include <vector>  // (ù{\codeincomments{std::vector}}ù)
template <typename T> void f1(std::vector<T>& v) { }  // (ù{\codeincomments{T}}ù) is the element type
\end{emcppslisting}
    

\noindent Not only is the argument \lstinline!v! constrained to being a
\lstinline!vector!, but the deduced element type \lstinline!T! is available
for use within the function. The same sort of pattern matching is not
available portably for \emcppsgloss[generic lambda]{generic lambdas}:

\begin{emcppslisting}[emcppsbatch=e8]
auto y1 = [](std::vector<auto>& v) { };  // Error, (ù{\codeincomments{auto}}ù) as template parameter
\end{emcppslisting}
    

\noindent Constraining an \lstinline!auto! parameter using metaprogramming, e.g.,
through the use of \lstinline!std::enable_if!, is sometimes possible:

\begin{emcppslisting}[emcppsbatch=e8]
#include <type_traits>  // (ù{\codeincomments{std::enable\_if\_t}}ù), (ù{\codeincomments{std::is\_same}}ù),
                        // (ù{\codeincomments{std::remove\_reference\_t}}ù)
auto y2 = [](auto& v) -> std::enable_if_t<
    std::is_same<
        std::vector<typename std::remove_reference_t<decltype(v)>::value_type>&,
        decltype(v)
    >::value> { };
\end{emcppslisting}
    

\noindent The \lstinline!y2! \emcppsgloss{closure} can be called only with a vector. Any
other type will fail substitution because \lstinline!is_same! will return
\lstinline!false! if substitution even gets that far; substitution might
fail earlier if the type for \lstinline!v! does not have a nested
\lstinline!value_type!. Passing non\lstinline!vector! arguments to this
constrained lambda will now fail at the call site, rather than,
presumably, failing during instantiation of \lstinline!y2(v)!:

\begin{emcppslisting}[emcppsbatch=e8]
void g1()
{
    int                i;
    std::vector<int>   v1;
    std::vector<float> v2;

    y2(i);   // Error, cannot call (ù{\codeincomments{y2}}ù) on a non(ù{\codeincomments{vector}}ù)
    y2(v1);  // OK, (ù{\codeincomments{v1}}ù) is a vector
    y2(v2);  // OK, (ù{\codeincomments{v2}}ù) is a vector
}
\end{emcppslisting}
    

\noindent For all of the additional complication on \lstinline!y2!, the element type
for our \lstinline!vector! is still not available within the \emcppsgloss{lambda
body}, as it was for the function body for \lstinline!f1!, above; we would
need to repeat the type name \lstinline!typename!
\lstinline!std::remove_reference_t<decltype(v)>::value_type! if the
element type became necessary.

This annoyance is mitigated, however, because \emcppsgloss{lambda
expressions} cannot be overloaded. In the absence of overloading, there
is little benefit to removing a call from the overload set compared to
simply letting the instantiation fail, especially as most \emcppsgloss{lambda
expressions} are defined at the point of use, making it comparatively
easy to diagnose a compilation problem if one occurs. Moreover, this
point-of-use definition is already tuned to its expected use case, so
constraints are often redundant, adding little additional safety to the
code.

\subsubsection[Difficulty in constraining multiple arguments]{Difficulty in constraining multiple arguments}\label{difficulty-in-constraining-multiple-arguments}

Often, we want to restrict function template arguments such that two or
more arguments have related types. For example, an operation might
require two iterators of the same type or one argument to be a pointer
to the type of the other argument. \emcppsgloss[generic lambda]{Generic lambdas} provide only
limited support for interargument patterns. Each parameter to a
\emcppsgloss{generic lambda} that contains \lstinline!auto! is completely
independent of the other parameters, so the normal mechanism used for
function templates --- using the same named type parameter (e.g.,
\lstinline!T! or \lstinline!U!) in multiple places --- is unavailable:

\begin{emcppslisting}[emcppsstandards={c++14}]
void g1()
{
    auto y = [](auto a, auto b) { };  // No interargument constraints
    y(1, 2);          // OK, arguments of type (ù{\codeincomments{int}}ù) and (ù{\codeincomments{int}}ù)
    y("one", "two");  // OK,     "      "   "  (ù{\codeincomments{const char*}}ù) and (ù{\codeincomments{const char*}}ù)
    y(1, "two");      // OK,     "      "   "  (ù{\codeincomments{int}}ù) and (ù{\codeincomments{const char*}}ù)
}
\end{emcppslisting}
    

\noindent Limited constraints are possible --- e.g., requiring that arguments have
the same type --- by using the \lstinline!decltype! operator to declare
later parameters based on the types of earlier parameters; see
\featureref{\locationa}{decltype}:

\begin{emcppslisting}[emcppsstandards={c++14}]
void g2()
{
    auto y = [](auto a, decltype(a) b) { };  // (ù{\codeincomments{a}}ù) and (ù{\codeincomments{b}}ù) have the same type.
    y(1, 2);          // OK, both arguments are of type (ù{\codeincomments{int}}ù).
    y("one", "two");  // OK,   "      "      "   "   "  (ù{\codeincomments{const char*}}ù).
    y(1, "two");      // Error, mixed argument types (ù{\codeincomments{int}}ù) and (ù{\codeincomments{const char*}}ù)
}
\end{emcppslisting}
    

\noindent If the relationship between the parameter types is even slightly
different than an exact match, expressing it can become complicated. For
example, if parameter \lstinline!a! is a pointer and we want parameter
\lstinline!b! to be a value of the type pointed to by \lstinline!a!, our first
approach might be to try \lstinline!decltype(*a)!:

\begin{emcppslisting}[emcppsbatch=e9,emcppsstandards={c++14}]
int i = 0;

void g3()
{
    auto y = [](auto* a, decltype(*a) b) { *a = b; };
    y(&i, 5);  // Error, can't bind (ù{\emphincomments{rvalue}}ù) (ù{\codeincomments{5}}ù) to (ù{\codeincomments{int\& b}}ù)
}
\end{emcppslisting}
    

\noindent Unfortunately, as we see, \lstinline!decltype(*a)! yields a
non\lstinline!const! \romeovalue{lvalue} reference, which cannot bind to the
\romeovalue{rvalue} \lstinline!5!. Our next attempt is to \lstinline!const!-qualify
\lstinline!b!, since \lstinline!const! references \emph{can} bind to
\romeovalue{rvalues}:

\begin{emcppslisting}[emcppsbatch=e9]
void g4()
{
    auto y = [](auto* a, const decltype(*a) b) { *a = b; };
    y(&i, 5);  // Error, (ù{\codeincomments{const}}ù) applied to reference is ignored.
}
\end{emcppslisting}
    

\noindent This approach fails because \lstinline!const! can be applied only to the
\emph{referred-to} type; applying \lstinline!const! to the reference has no
effect. If we use the standard metafunction
\lstinline!std::remove_reference_t!, we can finally get to the type to
which \lstinline!a! points, without any reference specifiers:

\begin{emcppslisting}[emcppsbatch=e9]
#include <type_traits>  // (ù{\codeincomments{std::remove\_reference\_t}}ù)

void g5()
{
    auto y = [](auto* a, std::remove_reference_t<decltype(*a)> b)
    {
        *a = b;
    };
    y(&i, 5); // OK, pass (ù{\codeincomments{5}}ù) by value for (ù{\codeincomments{int b}}ù).
}
\end{emcppslisting}
    

\noindent Note that \lstinline!y! takes argument \lstinline!b! by value, which can be
inefficient for types having an expensive copy constructor. When the
type of an argument is unknown, we usually pass it by \lstinline!const!
reference:

\begin{emcppslisting}[emcppsbatch=e9]
void g6()
{
    auto y = [](auto* a, const std::remove_reference_t<decltype(*a)>& b)
    {
        *a = b;
    };
    y(&i, 5); // OK, bind (ù{\codeincomments{5}}ù) to (ù{\codeincomments{const int\& b}}ù).
}
\end{emcppslisting}
    

\noindent In the last example, parameter \lstinline!b! is always a \lstinline!const!
\romeovalue{lvalue} reference, never an \romeovalue{rvalue} reference. When
constraining a type this way, we forfeit the possibility of perfect
forwarding.

\subsection[See Also]{See Also}\label{see-also}

\begin{itemize}
\item{\seealsoref{decltype}{\seealsolocationa}provides the only way to name the type of an \lstinline!auto! parameter in a \emcppsgloss{generic lambda}.}
%
\item{\seealsoref{auto}{\seealsolocationc}allows a \emcppsgloss[lambda expressions]{lambda expression}, which has unnamed type, to be saved in a variable and also details the rules for deducing an \lstinline!auto! parameter from its argument expression.}
%
\item{\seealsoref{forwardingref}{\seealsolocationc}provide the most general way to declare a \emcppsgloss{generic lambda}’s parameters, preserving both the parameters’ types and their \emcppsgloss[value category]{value categories}.}
%
\item{\seealsoref{lambda}{\seealsolocationc}introduced the facility for locally defined anonymous function objects that \emcppsgloss[generic lambda]{generic lambdas} expand on.}
%
\item{\seealsoref{variadictemplate}{\seealsolocationc} allow a template, including a \emcppsgloss{generic lambda}, to have a variable number of parameters.}
%
\item{\seealsoref{lambda-capture-expressions}{\seealsolocationd}describes the initialized capture syntax added to \emcppsgloss[lambda capture]{lambda captures} in C++14.}
%
\item{\seealsoref{autoreturn}{\seealsolocationf}describes how a function, including the function call operator for a \emcppsgloss[lambda expressions]{lambda expression}, can deduce its return type from its \lstinline!return! statements.}
\end{itemize}

\subsection[Further Reading]{Further Reading}\label{further-reading}

%@LORI: please ask Pablo or Andrei for these sources, these are mentioned in the Markdown but could not find the actual refs

\begin{itemize}
\item{\cite{hindley86}}
\item{\cite{derevenets16}}
\item{\cite{fultz14}}
\end{itemize}


