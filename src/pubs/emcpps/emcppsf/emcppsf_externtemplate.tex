
% 01 Feb, 2021 JMB - packet 4 compilation fixes

\emcppsFeature{
    short={\lstinline!extern!~\lstinline!template!},
    tocshort={{\TOCCode extern}~{\TOCCode template}},
    long={Explicit Instantiation Declarations},
    rhshort={{\RHCode extern}~{\RHCode template}},
}{explicit-template-instantiations}
\setcounter{table}{0}
\setcounter{footnote}{0}
\setcounter{lstlisting}{0}
%\section[{\tt extern} {\tt template}]{Explicit Instantiation Declarations}\label{explicit-template-instantiations}

The \lstinline!extern!~\lstinline!template! prefix can be used to suppress
\emph{implicit} generation of local object code for the definitions of
particular specializations of class, function, or
variable templates used within a translation unit, with the
expectation that any suppressed object-code-level definitions will be
provided elsewhere within the program by template definitions that are
instantiated \emph{explicitly}.

\subsection[Description]{Description}\label{description}

Inherent in the current ecosystem for supporting template programming in
C++ is the need to generate redundant definitions of fully specified
function and variable templates within \lstinline!.o! files. For common
instantiations of popular templates, such as \lstinline!std::vector!, the
increased object-file size --- a.k.a. \emcppsgloss{code bloat} --- and
potentially extended link times might become significant:

\begin{emcppslisting}
#include <vector>    // (ù{\codeincomments{std::vector}}ù) is a popular template.
std::vector<int> v;  // (ù{\codeincomments{std::vector<int>}}ù) is a common instantiation.

#include <string>    // (ù{\codeincomments{std::basic\_string}}ù) is a popular template.
std::string s;       // (ù{\codeincomments{std::string}}ù), an alias for (ù{\codeincomments{std::basic\_string<char>}}ù), is
                     // a common instantiation.
\end{emcppslisting}

\noindent The intent of the \lstinline!extern!~\lstinline!template! feature is to
\emph{suppress} the implicit generation of duplicative object code
within each and every translation unit in which a fully specialized
class template, such as \lstinline!std::vector<int>! in the code snippet
above, is used. Instead, \lstinline!extern!~\lstinline!template! allows
developers to choose a single translation unit in which to explicitly
\emph{generate} object code for all the definitions pertaining to that
specific template specialization as explained in the next
subsection, \intrarefnopage{explicit-instantiation-definition}.

\subsubsection[Explicit-instantiation definition]{Explicit-instantiation definition}\label{explicit-instantiation-definition}

The ability to create an \emcppsgloss[explicit instantiation definition]{explicit-instantiation definition} has been available since C++98.{\cprotect\footnote{The
  C++03 Standard term for the syntax used to create an
  \emcppsgloss[explicit instantiation definition]{explicit-instantiation \emph{definition}}, though rarely used,
  was \emcppsgloss[explicit instantiation directive]{explicit-instantiation \emph{directive}}. The term
  \emcppsgloss[explicit instantiation directive]{explicit-instantiation directive} was clarified in
  C++11 and can now also refer to syntax that is used to create a
  \emph{declaration} --- i.e., \emcppsgloss[explicit instantiation declaration]{explicit-instantiation \emph{declaration}}.}} The requisite syntax is to place the keyword
\lstinline!template! in front of the name of the fully specialized class
template, function template, or --- in C++14 --- variable
template (see \featureref{\locationb}{variable-templates}):

\begin{emcppslisting}
#include <vector>  // (ù{\codeincomments{std::vector}}ù) (general template)

template class std::vector<int>;
    // Deposit all definitions for this specialization into the (ù{\codeincomments{.o}}ù) for this
    // translation unit.
\end{emcppslisting}

\noindent This \emcppsgloss[explicit instantiation directive]{explicit-instantiation directive} compels the compiler to
instantiate \emph{all} functions defined by the named
\lstinline!std::vector! class template having the specified \lstinline!int!
template argument; any collateral object code resulting from these
instantiations will be deposited in the resulting \lstinline!.o! file for
the current translation unit. Importantly, even functions that are never
used are still specialized, so this might not be the correct solution
for many classes; see \intraref{potential-pitfalls-externtemplate}{accidentally-making-matters-worse}.

\subsubsection[Explicit-instantiation declaration]{Explicit-instantiation declaration}\label{explicit-instantiation-declaration}

C++11 introduced the \emcppsgloss[explicit instantiation declaration]{explicit-instantiation \emph{declaration}},
complement to the \emcppsgloss[explicit instantiation definition]{explicit-instantiation \emph{definition}}. The
newly provided syntax allows us to place
\lstinline!extern!~\lstinline!template! in front of the declaration of the
explicit specialization of a class template, a function template, or a
variable template: %(see \featureref{\locationb}{variable-templates}):

\begin{emcppslisting}
#include <vector>  // (ù{\codeincomments{std::vector}}ù) (general template)

extern template class std::vector<int>;
    // Suppress depositing of any object code for (ù{\codeincomments{std::vector<int>}}ù) into the
    // (ù{\codeincomments{.o}}ù) file for this translation unit.
\end{emcppslisting}

\noindent The use of the modern \lstinline!extern!~\lstinline!template! syntax above
instructs the compiler to specifically \emph{not} deposit any object
code for the named specialization in the current translation unit and
instead to rely on some other translation unit to provide any missing
object-level definitions that might be needed at link time; see
\intraref{annoyances-externtemplate}{no-good-place-to-put-definitions-for-unrelated-classes}.

Note, however, that declaring an explicit instantiation to be an
\lstinline!extern!~\lstinline!template! \emph{in no way} affects the ability
of the compiler to instantiate and to inline visible function-definition
bodies for that template specialization in the translation unit:

\begin{emcppslisting}
// client.cpp:
#include <vector>  // (ù{\codeincomments{std::vector}}ù) (general template)

extern template class std::vector<int>;  // specialization for (ù{\codeincomments{int}}ù) elements

void client(std::vector<int>& inOut)  // fully specialized instance of a (ù{\codeincomments{vector}}ù)
{
    if (inOut.size())          // This invocation of (ù{\codeincomments{size}}ù) can inline.
    {
        int value = inOut[0];  // This invocation of (ù{\codeincomments{operator[]}}ù) can inline.
    }
}
\end{emcppslisting}

\noindent In the example above, the two tiny member functions of \lstinline!vector!,
namely \lstinline!size! and \lstinline!operator[]!, will typically be
substituted inline --- in precisely the same way they would have been
had the \lstinline!extern!~\lstinline!template! declaration been omitted. The
\emph{only} purpose of an \lstinline!extern!~\lstinline!template! declaration
is to suppress object-code generation for this particular template
instantiation for the current translation unit.

Finally, note that the use of \emcppsgloss[explicit instantiation directive]{explicit-instantiation \emph{directives}} have absolutely no affect on the logical meaning of a
well-formed program; in particular, when applied to specializations of
function templates, they have no affect whatsoever on overload
resolution:

\begin{emcppslisting}
template <typename T> bool f(T v) {/*...*/}  // general template definition

extern template bool f(char c);  // specialization of (ù{\codeincomments{f}}ù) for (ù{\codeincomments{char}}ù)
extern template bool f(int v);   // specialization of (ù{\codeincomments{f}}ù) for (ù{\codeincomments{int}}ù)

char     c;
short    s;
int      i;
unsigned u;

bool bc = f(c);  // exact match: Object code is suppressed locally.
bool bs = f(s);  // not exact match: Object code is generated locally.
bool bi = f(i);  // exact match: Object code is suppressed locally.
bool bu = f(u);  // not exact match: Object code is generated locally.
\end{emcppslisting}

\noindent As the example above illustrates, overload resolution and template
parameter deduction occur independently of any
\emcppsgloss[explicit instantiation declaration]{explicit-instantiation declarations}. Only \emph{after}
the template to be instantiated is determined does the
\lstinline!extern!~\lstinline!template! syntax take effect; see also \intraref{potential-pitfalls-externtemplate}{corresponding-explicit-instantiation-declarations-and-definitions}.

\subsubsection[A more complete illustrative example]{A more complete illustrative example}\label{a-more-complete-illustrative-example}

So far, we have seen the use of \emcppsgloss[explicit instantiation declaration]{explicit-instantiation declarations} and \emcppsgloss[explicit instantiation definition]{explicit-instan\-tiation definitions} applied to only a
(standard) \emph{class} template, \lstinline!std::vector!. The same syntax
shown in the previous code snippet applies also to full specializations of individual
function templates and variable templates. % (see \featureref{\locationb}{variable-templates}).

As a more comprehensive, albeit largely pedagogical example, consider
the overly simplistic \lstinline!my::Vector! class template along with
other related templates defined within a header file
\lstinline!my_vector.h!:

\begin{emcppslisting}[emcppsbatch={e1,e1b}]
// my_vector.h:
#ifndef INCLUDED_MY_VECTOR  // internal include guard
#define INCLUDED_MY_VECTOR

#include <cstddef>  // (ù{\codeincomments{std::size\_t}}ù)
#include <utility>  // (ù{\codeincomments{std::swap}}ù)

namespace my  // namespace for all entities defined within this component
{

template <typename T>
class Vector
{
    static std::size_t s_count; // track number of objects constructed
    T* d_data_p;                // pointer to dynamically allocated memory
    std::size_t d_length;       // current number of elements in the vector
    std::size_t d_capacity;     // number of elements currently allocated

public:
    // ...

    std::size_t length() const { return d_length; }
        // return the number of elements

    // ...
};

// ...          Any partial or full specialization definitions            ...
// ...          of the class template (ù{\codeincomments{Vector}}ù) go here.                     ...

template <typename T>
void swap(Vector<T> &lhs, Vector<T> &rhs) { return std::swap(lhs, rhs); }
    // free function that operates on objects of type (ù{\codeincomments{my::Vector}}ù) via ADL

// ...            Any [full] specialization definitions                   ...
// ...            of free function (ù{\codeincomments{swap}}ù) would go here.                    ...

template <typename T>
const std::size_t vectorSize = sizeof(Vector<T>);  // C++14 variable template
    // This nonmodifiable (ù{\codeincomments{static}}ù) variable holds the size of a (ù{\codeincomments{my::Vector<T>}}ù).

// ...           Any [full] specialization definitions                    ...
// ...           of variable (ù{\codeincomments{vectorSize}}ù) would go here.                    ...

template <typename T>
std::size_t Vector<T>::s_count = 0;
    // definition of (ù{\codeincomments{static}}ù) counter in general template

// ... We may opt to add explicit-instantiation declarations here;
//     see the next code example.

}  // close (ù{\codeincomments{my}}ù) namespace

#endif  // close internal include guard
\end{emcppslisting}

\noindent In the \lstinline!my_vector! component in the code snippet above, we
have defined the following, in the \lstinline!my! namespace:
\begin{enumerate}
\item{a \lstinline!class! template, \lstinline!Vector!, parameterized on element type}
\item{a free-function template, \lstinline!swap!, that operates on objects of corresponding specialized \lstinline!Vector! type}
\item{a \lstinline!const! C++14 variable template,
%(see \featureref{\locationb}{variable-templates}),
\lstinline!vectorSize!, that represents the number of bytes in the \emcppsgloss{footprint} of an object of the corresponding specialized \lstinline!Vector! type}
\end{enumerate}
Any use of these templates by a client might and typically will trigger
the depositing of equivalent definitions as object code in the client
translation unit's resulting \lstinline!.o! file, irrespective of whether
the definition being used winds up getting inlined.

To eliminate object code for specializations of entities in the
\lstinline!my_vector! component, we must first decide where the unique
definitions will go; see \intraref{annoyances-externtemplate}{no-good-place-to-put-definitions-for-unrelated-classes}. In this
specific case, however, we own the component that requires
specialization, and the specialization is for a ubiquitous built-in type;
hence, the natural place to generate the specialized definitions is in a
\lstinline!.cpp! file corresponding to the component's header:

\begin{emcppslisting}[emcppsbatch=e1,emcppsstandards=c++14]
// my_vector.cpp:
#include <my_vector.h>  // We always include the component's own header first.
    // By including this header file, we have introduced the general template
    // definitions for each of the explicit-instantiation declarations below.

namespace my  // namespace for all entities defined within this component
{

template class Vector<int>;
    // Generate object code for all non(ù{\codeincomments{template}}ù) member-function and (ù{\codeincomments{static}}ù)
    // member-variable definitions of template (ù{\codeincomments{my::Vector}}ù) having (ù{\codeincomments{int}}ù) elements.

template std::size_t Vector<double>::length() const;  // BAD IDEA
    // In addition, we could generate object code for just a particular member
    // function definition of (ù{\codeincomments{my::Vector}}ù) (e.g., (ù{\codeincomments{length}}ù)) for some other
    // parameter type (e.g., (ù{\codeincomments{double}}ù)), which is shown here for pedagogy only.

template void swap(Vector<int>& lhs, Vector<int>& rhs);
    // Generate object code for the full specialization of the (ù{\codeincomments{swap}}ù) free-
    // function template that operates on objects of type (ù{\codeincomments{my::Vector<int>}}ù).

template const std::size_t vectorSize<int>;  // C++14 variable template
    // Generate the object-code-level definition for the specialization of the
    // C++14 variable template instantiated for built-in type (ù{\codeincomments{int}}ù).

//template std::size_t Vector<int>::s_count = 0;
    // Generate the object-code-level definition for the specialization of the
    // (ù{\codeincomments{static}}ù) member variable of (ù{\codeincomments{Vector}}ù) instantiated for built-in type (ù{\codeincomments{int}}ù).

}; // close (ù{\codeincomments{my}}ù) namespace
\end{emcppslisting}

\noindent Each of the constructs introduced by the keyword \lstinline!template!
within the \lstinline!my! namespace in the code snippet above represents a
separate \emcppsgloss[explicit instantiation definition]{explicit-instantiation definition}. These
constructs instruct the compiler to generate object-level definitions
for general templates declared in \lstinline!my_vector.h! specialized on
the built-in type \lstinline!int!.

Having installed the necessary \emcppsgloss[explicit instantiation definition]{explicit-instantiation definitions} in the component's \lstinline!my_vector.cpp! file, we
must now go back to its \lstinline!my_vector.h! file and, without altering
any of the previously existing lines of code, \emph{add} the
corresponding \emcppsgloss[explicit instantiation declaration]{explicit-instantiation declarations} to
suppress redundant local code generation:

\begin{emcppshiddenlisting}[emcppsbatch=e1b]
#undef INCLUDED_MY_VECTOR
\end{emcppshiddenlisting}
\begin{emcppslisting}[emcppsbatch=e1b]
// my_vector.h:
#ifndef INCLUDED_MY_VECTOR  // internal include guard
#define INCLUDED_MY_VECTOR

namespace my  // namespace for all entities defined within this component
{

// ...
// ...  everything that was in the original (ù{\codeincomments{my}}ù) namespace
// ...

extern template class Vector<int>;
    // Suppress object code for this class template specialized for (ù{\codeincomments{int}}ù).

extern template std::size_t Vector<double>::size() const;  // BAD IDEA
    // Suppress object code for this member, only specialized for (ù{\codeincomments{double}}ù).

extern template void swap(Vector<int>& lhs, Vector<int>& rhs);
    // Suppress object code for this free function specialized for (ù{\codeincomments{int}}ù).

extern template std::size_t vectorSize<int>;  // C++14
    // Suppress object code for this variable template specialized for (ù{\codeincomments{int}}ù).

extern template std::size_t Vector<int>::s_count;
    // Suppress object code for this (ù{\codeincomments{static}}ù) member definition w.r.t. (ù{\codeincomments{int}}ù).

}  // close (ù{\codeincomments{my}}ù) namespace

#endif  // close internal include guard
\end{emcppslisting}

\noindent Each of the constructs that begin with \lstinline!extern!~\lstinline!template!
in the example above are \emcppsgloss[explicit instantiation declaration]{explicit-instantiation declarations}, which serve only to suppress the generation of any
object code emitted to the \lstinline!.o! file of the current translation
unit in which such specializations are used. These added
\lstinline!extern!~\lstinline!template! declarations must appear in the
\lstinline!my_header.h! source file \emph{after} the declaration of the
corresponding general template and, importantly, before whatever
relevant definitions are ever used.

\subsubsection[The effect on various \lstinline!.o! files]{The effect on various {\SubsubsecCode .o} files}\label{the-effect-on-various-.o-files}

To illustrate the effect of \emcppsgloss[explicit instantiation declaration]{explicit-instantiation declarations} and \emcppsgloss[explicit instantiation definition]{explicit-instantiation definitions} on the contents of object and executable files,
we'll use a simple \lstinline!lib_interval! library \emcppsgloss{component}
consisting of a header file, \lstinline!lib_interval.h!, and an
implementation file, \lstinline!lib_interval.cpp!. The latter, apart from
including its corresponding header, is effectively empty:

\begin{emcppslisting}[emcppsbatch=e2]
// lib_interval.h:
#ifndef INCLUDED_LIB_INTERVAL  // internal include guard
#define INCLUDED_LIB_INTERVAL

namespace lib  // namespace for all entities defined within this component
{

template <typename T>  // elided sketch of a class template
class Interval
{
    T d_low;   // interval's low value
    T d_high;  // interval's high value

public:
    explicit Interval(const T& p) : d_low(p), d_high(p) { }
        // Construct an empty interval.

    Interval(const T& low, const T& high) : d_low(low), d_high(high) { }
        // Construct an interval having the specified boundary values.

    const T& low() const { return d_low; }
        // Return this interval's low value.

    const T& high() const { return d_high; }
        // Return this interval's high value.

    int length() const { return d_high - d_low; }
        // Return this interval's length.

    // ...
};

template <typename T>                   // elided sketch of a function template
bool intersect(const Interval<T>& i1, const Interval<T>& i2)
    // Determine whether the specified intervals intersect ...
{
    bool result = false;  // nonintersecting until proven otherwise
    // ...
    return result;
}

}  // close (ù{\codeincomments{lib}}ù) namespace

#endif  // INCLUDED_LIB_INTERVAL
\end{emcppslisting}
\vspace*{2ex}   %%%% extra space to highlight different snippets
\begin{emcppslisting}[emcppsbatch=e2]
// lib_interval.cpp:
#include <lib_interval.h>
\end{emcppslisting}

\noindent This library component defines, in the namespace \lstinline!lib!, a heavily
elided sketch of an implementation of (1) a class template,
\lstinline!Interval!, and (2) a function template, \lstinline!intersect!, the
only practical purpose of which is to provide some sample template
source code to compile.

Let's also consider a trivial application that uses this library
\emcppsgloss{component}:

\begin{emcppslisting}[emcppsbatch=e2]
// app.cpp:
#include <lib_interval.h>  // Include the library component's header file.

int main(int argv, const char** argc)
{
    lib::Interval<double> a(0, 5);  // instantiate with (ù{\codeincomments{double}}ù) type parameter
    lib::Interval<double> b(3, 8);  // instantiate with (ù{\codeincomments{double}}ù) type parameter
    lib::Interval<int>    c(4, 9);  // instantiate with (ù{\codeincomments{int}}ù)    type parameter

    if (lib::intersect(a, b))  // instantiate deducing (ù{\codeincomments{double}}ù) type parameter
    {
        return 0;  // return "success" as (0.0, 5.0) does intersect (3.0, 8.0)
    }

    return 1;  // Return "failure" status as function apparently doesn't work.
}
\end{emcppslisting}

\noindent The purpose of this application is merely to exhibit a couple of
instantiations of the library \emph{class} template,
\lstinline!lib::Interval!, for type parameters \lstinline!int! and
\lstinline!double!, and of the library \emph{function} template,
\lstinline!lib::intersect!, for just \lstinline!double!.

Next, we compile the application and library translation units,
\lstinline!app.cpp! and\linebreak[4]
\lstinline!lib_interval.cpp!, and inspect the symbols
in their respective corresponding object files, \lstinline!app.o! and
\lstinline!lib_interval.o!:

\begin{lstlisting}[language=bash]
$ gcc -I. -c app.cpp lib_interval.cpp
$ nm -C app.o lib_interval.o

app.o:
0000000000000000 W lib::Interval<double>::Interval(double const&, double const&)
0000000000000000 W lib::Interval<int>::Interval(int const&, int const&)
0000000000000000 W bool lib::intersect<double>(lib::Interval<double> const&,
                                               lib::Interval<double> const&)
0000000000000000 T main

lib_interval.o:
\end{lstlisting}

\noindent Looking at \lstinline!app.o! in the previous example, the class and
function templates used in the \lstinline!main! function, which is
defined in the \lstinline!app.cpp! file, were instantiated
\emph{implicitly} and the relevant code was added to the resulting
object file, \lstinline!app.o!, with each instantiated function definition
in its own separate \emcppsgloss{section}. In the \lstinline!Interval!
\emph{class} template, the generated symbols correspond to the two
unique instantiations of the constructor, i.e., for \lstinline!double! and
\lstinline!int!, respectively. The \lstinline!intersect! function
template, however, was implicitly instantiated for only type
\lstinline!double!. Note importantly that all of the implicitly
instantiated functions have the \lstinline!W! symbol type, indicating that
they are \emph{weak} symbols, which are permitted to be present in
multiple object files. By contrast, this file defines the strong symbol
\lstinline!main!, marked here by a \lstinline!T!. Linking this file with any
other file containing such a symbol would cause the linker to report a
multiply-defined-symbol error. On the other hand, the
\lstinline!lib_interval.o! file corresponds to the
\lstinline!lib_interval! library component, whose \lstinline!.cpp! file
served only to include its own \lstinline!.h! file, and is again
effectively empty.

Let's now link the two object files, \lstinline!app.o! and
\lstinline!lib_interval.o!, and inspect the symbols in the resulting
executable, \lstinline!app!{\cprotect\footnote{We have stripped out
extraneous unrelated information that the \lstinline!nm! tool produces;
note that the \lstinline!-C! option invokes the symbol demangler, which
turns encoded names like \lstinline!_ZN3lib8IntervalIdEC1ERKdS3_! into
something more readable like
  \lstinline!lib::Interval<double>::Interval(double!~\lstinline!const&,!~\lstinline!double!~\lstinline!const&)!.}}:

\begin{lstlisting}[language=bash]
$ gcc -o app app.o lib_interval.o
$ nm -C app
000000000040056e T lib::Interval<double>::Interval(double const&, double const&)
00000000004005a2 T lib::Interval<int>::Interval(int const&, int const&)
00000000004005ce T bool lib::intersect<double>(lib::Interval<double> const&,
                                               lib::Interval<double> const&)
00000000004004b7 T main
\end{lstlisting}

\noindent As the textual output above confirms, each of the needed \emph{weak}
template symbols, previously marked with a \lstinline!W!, is bound into the
final program as a \emph{strong} symbol, now --- like \lstinline!main! ---
marked with a \lstinline!T!.{\cprotect\footnote{Whether the symbol is
marked \lstinline!W! or \lstinline!T! in the final executable is
implementation specific and of no consequence here. We present these
  concepts in this particular way to aid cognition.}} In this tiny
illustrative example, only one set of weak symbols appeared in the
combined \lstinline!.o! files.

More generally, if the application comprises multiple object files, each
file will potentially contain their own set of weak symbols,
often leading to duplicate code \emcppsgloss{sections} for implicitly
instantiated class, function, and variable
templates instantiated on the same parameters. When the linker combines
object files, it will arbitrarily choose at most one of each of these
respective and hopefully identical weak-symbol \emcppsgloss{sections} to
include in the final executable, now marked as a strong symbol
(\lstinline!T!).

Imagine now that our program includes a large number of \lstinline!.o!
files, many of which make use of our \lstinline!lib_interval! component,
particularly to operate on \lstinline!double! intervals. Suppose, for now,
we decide we would like to suppress the generation of object code for
templates related to just \lstinline!double! types with the intent of later putting them all in one place, i.e., the currently
empty \lstinline!lib_interval.o!. Achieving this objective is precisely what the
\lstinline!extern!~\lstinline!template! syntax is designed to accomplish.

Returning to our \lstinline!lib_interval.h! file, we need not change one
line of code; we need only to \emph{add} two
\emcppsgloss[explicit instantiation declaration]{explicit-instantiation declarations} --- one for the
template \emph{class}, \lstinline!Interval<double>!, and one for the
template \emph{function},
\lstinline!intersect<double>(const!~\lstinline!double&,!~\lstinline!const!~\lstinline!double&)!
--- to the header file anywhere \emph{after} their respective
corresponding general template declaration and definition:

\begin{emcppshiddenlisting}[emcppsbatch=e3]
// lib_interval.h:

#include <cstddef>  // (ù{\codeincomments{std::size\_t}}ù)

namespace lib {
template <typename T>
class Interval
{};

template <typename T>
bool intersect(const Interval<T>&, const Interval<T>&)
{
    return false;
}

}

\end{emcppshiddenlisting}
\begin{emcppslisting}[emcppsbatch=e3]
// lib_interval.h:  // (no change to existing code)
#ifndef INCLUDED_LIB_INTERVAL  // internal include guard
#define INCLUDED_LIB_INTERVAL

namespace lib  // namespace for all entities defined within this component
{

template <typename T>                    // elided sketch of a class template
class Interval;                          // ...
    // ...                       (same as before)

template <typename T>                    // elided sketch of a function template
bool intersect(const Interval<T>& i1, const Interval<T>& i2);
    // ...                       (same as before)

extern template class Interval<double>;  // explicit-instantiation declaration

extern template                          // explicit-instantiation declaration
bool intersect(const Interval<double>&, const Interval<double>&);

}  // close (ù{\codeincomments{lib}}ù) namespace

#endif  // INCLUDED_LIB_INTERVAL
\end{emcppslisting}

\noindent Let's again compile the two \lstinline!.cpp! files and inspect the
corresponding \lstinline!.o! files:

\begin{lstlisting}[language=bash]
$ gcc -I. -c app.cpp lib_interval.cpp
$ nm -C app.o lib_interval.o

app.o:
                 U lib::Interval<double>::Interval(double const&, double const&)
0000000000000000 W lib::Interval<int>::Interval(int const&, int const&)
                 U bool lib::intersect<double>(lib::Interval<double> const&,
                                               lib::Interval<double> const&)
0000000000000000 T main

lib_interval.o:
\end{lstlisting}

\noindent Notice that this time some of the symbols --- specifically those
relating to the \lstinline!class! and \lstinline!function! templates
instantiated for type \lstinline!double! --- have changed from \lstinline!W!,
indicating a \emph{weak} symbol, to \lstinline!U!, indicating an
\emph{undefined} one. What this means is that, instead of generating a
weak symbol for the explicit specializations for \lstinline!double!, the
compiler left those symbols undefined, as if only the
\emph{declarations} of the member and free-function templates had been
available when compiling \lstinline!app.cpp!, yet inlining of the
instantiated definitions is in no way affected. \emcppsgloss[undefined symbols]{Undefined symbols} are symbols that are expected to be made available to the
linker from other object files. Attempting to link this application
expectedly fails because no object files being linked contain the needed
definitions for those instantiations:

\begin{lstlisting}[language=bash]
$ gcc -o app app.o lib_interval.o

app.o: In function ('main':}
app.cpp:(.text+0x38): undefined reference to
  `lib::Interval<double>::Interval(double const&, double const&)'
app.cpp:(.text+0x69): undefined reference to
  `lib::Interval<double>::Interval(double const&, double const&)'
app.cpp:(.text+0xa1): undefined reference to
  `bool lib::intersect<double>(lib::Interval<double> const&,
                               lib::Interval<double> const&)'

collect2: error: ld returned 1 exit status
\end{lstlisting}

\noindent To provide the missing definitions, we will need to instantiate them
explicitly. Since the type for which the class and function are being
specialized is the ubiquitous built-in type, \lstinline!double!, the ideal
place to sequester those definitions would be within the \lstinline!.o!
file of the \lstinline!lib_interval! library component itself, but see \intraref{annoyances-externtemplate}{no-good-place-to-put-definitions-for-unrelated-classes}. To force the needed
template definitions into the \lstinline!lib_interval.o! file, we will
need to pull out our trusty \emcppsgloss[explicit instantiation definition]{explicit-instantiation definition} syntax, i.e., the \lstinline!template! prefix:

\begin{emcppslisting}[emcppsbatch=e3]
// lib_interval.cpp:
#include <lib_interval.h>

template class lib::Interval<double>;
    // example of an explicit-instantiation definition for a class

template bool lib::intersect(const Interval<double>&, const Interval<double>&);
    // example of an explicit-instantiation definition for a function
\end{emcppslisting}

\noindent We recompile once again and inspect our newly generated object files:

\begin{lstlisting}[language=bash]
$ gcc -I. -c app.cpp lib_interval.cpp
$ nm -C app.o lib_interval.o

app.o:
                 U lib::Interval<double>::Interval(double const&, double const&)
0000000000000000 W lib::Interval<int>::Interval(int const&, int const&)
                 U bool lib::intersect<double>(lib::Interval<double> const&,
                                               lib::Interval<double> const&)
0000000000000000 T main

lib_interval.o:
0000000000000000 W lib::Interval<double>::Interval(double const&)
0000000000000000 W lib::Interval<double>::Interval(double const&, double const&)
0000000000000000 W lib::Interval<double>::low() const
0000000000000000 W lib::Interval<double>::high() const
0000000000000000 W lib::Interval<double>::length() const
0000000000000000 W bool lib::intersect<double>(lib::Interval<double> const&,
                                               lib::Interval<double> const&)
\end{lstlisting}

\noindent The application \lstinline!.o! file, \lstinline!app.o!, naturally remained
unchanged. What's new here is that the functions that were missing from
the \lstinline!app.o! file are now available in the
\lstinline!lib_interval.o! file, again as \emph{weak} (\lstinline!W!), as
opposed to strong (\lstinline!T!), symbols. Notice, however, that explicit
instantiation forces the compiler to generate code for all of the member
functions of the class template for a given specialization. These
symbols might all be linked into the resulting executable unless we take
explicit precautions to exclude those that aren't
needed{\cprotect\footnote{To avoid including the explicitly generated
definitions that are being used to resolve undefined symbols, we have
instructed the linker to remove all unused code \emcppsgloss{sections} from the
executable. The \lstinline!-Wl! option passes comma-separated options to
the linker. The \lstinline!--gc-sections! option instructs the compiler
to compile and assemble and instructs the linker to omit individual
unused \emcppsgloss{sections}, where each section contains, for example, its own
  instantiation of a function template.}}:

\begin{lstlisting}[language=bash]
$ gcc -o app app.o lib_interval.o -Wl,--gc-sections
$ nm -C app
00000000004005ca T lib::Interval<double>::Interval(double const&, double const&)
000000000040056e T lib::Interval<int>::Interval(int const&, int const&)
000000000040063d T bool lib::intersect<double>(lib::Interval<double> const&,
                                               lib::Interval<double> const&)
00000000004004b7 T main
\end{lstlisting}

\noindent tl;dr: This \lstinline!extern!~\lstinline!template! feature is provided to
enable software architects to reduce code bloat in individual
\lstinline!.o! files for common instantiations of class,
function, and, as of C++14, variable templates in
large-scale C++ software systems. The practical benefit is in reducing
the physical size of libraries, which \emph{might} lead to improved link
times. \emcppsgloss[explicit instantiation declaration]{Explicit-instantiation declarations} do \emph{not}
(1) affect the meaning of a program, (2) suppress template
instantiation, (3) impede the compiler's ability to \lstinline!inline!, or
(4) meaningfully improve compile time. To be clear, the \emph{only}
purpose of the \lstinline!extern!~\lstinline!template! syntax is to suppress
object-code generation for the current translation unit, which is then
selectively overridden in the translation unit(s) of choice.

\subsection[Use Cases]{Use Cases}\label{use-cases}

\subsubsection[Reducing template code bloat in object files]{Reducing template code bloat in object files}\label{reducing-template-code-bloat-in-object-files}

The motivation for the \lstinline!extern!~\lstinline!template! syntax is as a
purely \emcppsgloss{physical} (not \emcppsgloss{logical}) optimization, i.e., to
reduce the amount of redundant code within individual object files
resulting from common template instantiations in client code. As an
example, consider a fixed-size-array class template,
\lstinline!FixedArray!, that is used widely, i.e., by many clients from
separate translation units, in a large-scale \lstinline!game! project for
both integral and floating-point calculations, primarily with type
parameters \lstinline!int! and \lstinline!double! and array sizes of either
\lstinline!2! or \lstinline!3!:

\begin{emcppslisting}[emcppsbatch=e4]
// game_fixedarray.h:
#ifndef INCLUDED_GAME_FIXEDARRAY  // *internal* include guard
#define INCLUDED_GAME_FIXEDARRAY

#include <cstddef>  // (ù{\codeincomments{std::size\_t}}ù)
namespace game  // namespace for all entities defined within this component
{

template <typename T, std::size_t N>
class FixedArray                                  // widely used class template
{
    // ... (elided private implementation details)
public:
    FixedArray()                                 { /*...*/ }
    FixedArray(const FixedArray<T, N>& other)    { /*...*/ }
    T& operator[](std::size_t index)             { /*...*/ }
    const T& operator[](std::size_t index) const { /*...*/ }
};

template <typename T, std::size_t N>
T dot(const FixedArray<T, N>& a, const FixedArray<T, N>& b) { /*...*/ }

// Explicit-instantiation declarations for full template specializations
// commonly used by the (ù{\codeincomments{game}}ù) project are provided below.

extern template class FixedArray<int, 2>;              // class template
extern template int dot(const FixedArray<int, 2>& a,   // function template
                        const FixedArray<int, 2>& b);  // for (ù{\codeincomments{int}}ù) and (ù{\codeincomments{2}}ù)

extern template class FixedArray<int, 3>;              // class template
extern template int dot(const FixedArray<int, 3>& a,   // function template
                        const FixedArray<int, 3>& b);  // for (ù{\codeincomments{int}}ù) and (ù{\codeincomments{3}}ù)

extern template class FixedArray<double, 2>;           // for (ù{\codeincomments{double}}ù) and (ù{\codeincomments{2}}ù)
extern template double dot(const FixedArray<double, 2>& a,
                           const FixedArray<double, 2>& b);

extern template class FixedArray<double, 3>;           // for (ù{\codeincomments{double}}ù) and (ù{\codeincomments{3}}ù)
extern template double dot(const FixedArray<double, 3>& a,
                           const FixedArray<double, 3>& b);

}  // close (ù{\codeincomments{game}}ù) namespace

#endif  // INCLUDED_GAME_FIXEDARRAY
\end{emcppslisting}

\noindent Specializations commonly used by the \lstinline!game! project are
provided by the \lstinline!game! library. In the component header in the
example above, we have used the \lstinline!extern!~\lstinline!template! syntax
to suppress object-code generation for instantiations of both the class
template \lstinline!FixedArray! and the function template \lstinline!dot! for
element types \lstinline!int! and \lstinline!double!, each for array sizes
\lstinline!2! and \lstinline!3!. To ensure that these specialized definitions
are available in every program that might need them, we use the
\lstinline!template! syntax counterpart to \emph{force} object-code
generation within just the one \lstinline!.o! corresponding to the
\lstinline!game_fixedarray! library component{\cprotect\footnote{Notice
  that we have chosen \emph{not} to nest the explicit specializations
  (or any other definitions) of entities already declared directly
  within the \lstinline!game! namespace, preferring instead to qualify each
  entity explicitly to be consistent with how we render free-function
  definitions (to avoid self-declaration); see \cite{lakos20}, section~2.5, ``Component Source-Code Organization," pp.~333--342, specifically Figure 2-36b, p.~340. See also \intraref{potential-pitfalls-externtemplate}{corresponding-explicit-instantiation-declarations-and-definitions}.}}:

\begin{emcppslisting}[emcppsbatch=e4]
// game_fixedarray.cpp:
#include <game_fixedarray.h>  // included as first substantive line of code

// Explicit-instantiation definitions for full template specializations
// commonly used by the (ù{\codeincomments{game}}ù)] project are provided below.

template class game::FixedArray<int, 2>;              // class template
template int game::dot(const FixedArray<int, 2>& a,   // function template
                       const FixedArray<int, 2>& b);  // for (ù{\codeincomments{int}}ù) and (ù{\codeincomments{2}}ù)

template class game::FixedArray<int, 3>;              // class template
template int game::dot(const FixedArray<int, 3>& a,   // function template
                       const FixedArray<int, 3>& b);  // for (ù{\codeincomments{int}}ù) and (ù{\codeincomments{3}}ù)

template class game::FixedArray<double, 2>;           // for (ù{\codeincomments{double}}ù) and (ù{\codeincomments{2}}ù)
template double game::dot(const FixedArray<double, 2>& a,
                          const FixedArray<double, 2>& b);

template class game::FixedArray<double, 3>;           // for (ù{\codeincomments{double}}ù) and (ù{\codeincomments{3}}ù)
template double game::dot(const FixedArray<double, 3>& a,
                          const FixedArray<double, 3>& b);
\end{emcppslisting}

\noindent Compiling \lstinline!game_fixedarray.cpp! and examining the resulting
object file shows that the code for all explicitly instantiated classes
and free functions was generated and placed into the object file,
\lstinline!game_fixedarray.o!{\cprotect\footnote{Note that only a subset
  of the relevant symbols have been retained.}}:

\begin{lstlisting}[language=bash]
$ gcc -I. -c game_fixedarray.cpp
$ nm -C game_fixedarray.o
0000000000000000 W game::FixedArray<double, 2ul>::FixedArray(
  game::FixedArray<double, 2ul> const&)
0000000000000000 W game::FixedArray<double, 2ul>::FixedArray()
0000000000000000 W game::FixedArray<double, 2ul>::operator[](unsigned long)
0000000000000000 W game::FixedArray<double, 3ul>::FixedArray(
  game::FixedArray<double, 3ul> const&)
0000000000000000 W game::FixedArray<int, 3ul>::FixedArray()
                                    :
0000000000000000 W double game::dot<double, 2ul>(
  game::FixedArray<double, 2ul> const&, game::FixedArray<double, 2ul> const&)
0000000000000000 W double game::dot<double, 3ul>(
  game::FixedArray<double, 3ul> const&, game::FixedArray<double, 3ul> const&)
0000000000000000 W int game::dot<int, 2ul>(
  game::FixedArray<int, 2ul> const&, game::FixedArray<int, 2ul> const&)
                                    :
0000000000000000 W game::FixedArray<int, 2ul>::operator[](unsigned long) const
0000000000000000 W game::FixedArray<int, 3ul>::operator[](unsigned long) const
\end{lstlisting}

\noindent This \lstinline!FixedArray! class template is used in multiple translation
units within the \lstinline!game! project. The first one contains a set
of geometry utilities:

\begin{emcppshiddenlisting}[emcppsbatch=e5]
// game_fixedarray.h:
#pragma once

#include <cstddef>  // (ù{\codeincomments{std::size\_t}}ù)

template <typename T, std::size_t N>
class FixedArray {};

template <typename T, std::size_t N>
T dot(const FixedArray<T, N>& a, const FixedArray<T, N>& b) { /*...*/ }

// game_unit.h:
#include <game_fixedarray.h>

namespace game {

struct Unit {
      FixedArray<double,3> centerOfMass() const;
};

}
\end{emcppshiddenlisting}
\begin{emcppslisting}[emcppsbatch=e5]
// app_geometryutil.cpp:

#include <game_fixedarray.h>
#include <game_unit.h>

using namespace game;

void translate(game::Unit* object, const FixedArray<double, 2>& dst)
    // Perform precise movement of the object on 2D plane.
{
    FixedArray<double, 2> objectProjection;
    // ...
}

void translate(game::Unit* object, const FixedArray<double, 3>& dst)
    // Perform precise movement of the object in 3D space.
{
    FixedArray<double, 3> delta;
    // ...
}

bool isOrthogonal(const FixedArray<int, 2>& a1, const FixedArray<int, 2>& a2)
    // Return true if 2d arrays are orthogonal.
{
    return 0 == dot(a1, a2);
}

bool isOrthogonal(const FixedArray<int, 3>& a1, const FixedArray<int, 3>& a2)
    // Return true if 3d arrays are orthogonal.
{
    return 0 == dot(a1, a2);
}
\end{emcppslisting}

\noindent The second one deals with physics calculations:

\begin{emcppslisting}[emcppsbatch=e5]
// app_physics.cpp:

#include <game_fixedarray.h>
#include <game_unit.h>

using namespace game;

void collide(game::Unit* objectA, game::Unit* objectB)
    // Calculate the result of object collision in 3D space.
{
    FixedArray<double, 3> centerOfMassA = objectA->centerOfMass();
    FixedArray<double, 3> centerOfMassB = objectB->centerOfMass();
    // ..
}

void accelerate(game::Unit* object, const FixedArray<double, 3>& force)
    // Calculate the position after applying a specified force for the
    // duration of a game tick.
{
    // ...
}
\end{emcppslisting}

\noindent Note that the object files for the application components throughout the
\lstinline!game! project do not contain any of the implicitly
instantiated definitions that we had chosen to uniquely sequester
externally, i.e., within the \lstinline!game_fixedarray.o! file:

\begin{lstlisting}[language=bash]
$ nm -C app_geometryutil.o
000000000000003e T isOrthogonal(game::FixedArray<int, 2ul> const&,
  game::FixedArray<int, 2ul> const&)
0000000000000068 T isOrthogonal(game::FixedArray<int, 3ul> const&,
  game::FixedArray<int, 3ul> const&)
0000000000000000 T translate(game::Unit*, game::FixedArray<double, 2ul> const&)
000000000000001f T translate(game::Unit*, game::FixedArray<double, 3ul> const&)
                 U game::FixedArray<double, 2ul>::FixedArray()
                 U game::FixedArray<double, 3ul>::FixedArray()
                 U int game::dot<int, 2ul>(game::FixedArray<int, 2ul> const&,
  game::FixedArray<int, 2ul> const&)
                 U int game::dot<int, 3ul>(game::FixedArray<int, 3ul> const&,
  game::FixedArray<int, 3ul> const&)

$ nm -C app_physics.o
0000000000000039 T accelerate(game::Unit*,
  game::FixedArray<double, 3ul> const&)
0000000000000000 T collide(game::Unit*, game::Unit*)
                 U game::FixedArray<double, 3ul>::FixedArray()
0000000000000000 W game::Unit::centerOfMass()
\end{lstlisting}

\noindent Whether optimization involving \emcppsgloss[explicit instantiation directive]{explicit-instantiation directives} reduces library sizes on disc has no noticeable effect or
actually makes matters worse will depend on the particulars of the
system at hand. Having this optimization applied to frequently used
templates across a large organization has been known to decrease object
file sizes, storage needs, link times, and overall build times, but see \intraref{potential-pitfalls-externtemplate}{accidentally-making-matters-worse}.

\subsubsection[Insulating template definitions from clients]{Insulating template definitions from clients}\label{insulating-template-definitions-from-clients}

Even before the introduction of \emcppsgloss[explicit instantiation declaration]{explicit-instantiation \emph{declarations}}, strategic use of \emcppsgloss[explicit instantiation definition]{explicit-instantiation \emph{definitions}} made it possible to \emcppsgloss{insulate} the
\emph{definition} of a template from client code, presenting instead
just a limited set of instantiations against which clients may link.
Such insulation enables the definition of the template to change without
forcing clients to recompile. What's more, new explicit instantiations
can be added without affecting existing clients.

As an example, suppose we have a single free-function template,
\lstinline!transform!, that operates on only floating-point values:

\begin{emcppslisting}[emcppsbatch=e6]
// transform.h:
#ifndef INCLUDED_TRANSFORM_H
#define INCLUDED_TRANSFORM_H

template <typename T>  // declaration (only) of free-function template
T transform(const T& value);
    // Return the transform of the specified floating-point (ù{\codeincomments{value}}ù).

#endif
\end{emcppslisting}

\noindent Initially, this function template will support just two built-in types,
\lstinline!float! and \lstinline!double!, but it is anticipated to eventually
support the additional built-in type \lstinline!long!~\lstinline!double! and
perhaps even supplementary user-defined types (e.g., \lstinline!Float128!)
to be made available via separate headers (e.g., \lstinline!float128.h!).
By placing only the declaration of the \lstinline!transform! function
template in its component's header, clients will be able to link against
only two supported explicit specializations provided in the
\lstinline!transform.cpp! file:

\begin{emcppslisting}[emcppsbatch=e6]
// transform.cpp:
#include <transform.h>  // Ensure consistency with client-facing declaration.

template <typename T>   // redeclaration/definition of free-function template
T transform(const T& value)
{
    // insulated implementation of transform function template
}

// explicit-instantiation *definitions*
template float transform(const float&);    // Instantiate for type (ù{\codeincomments{float}}ù).
template double transform(const double&);  // Instantiate for type (ù{\codeincomments{double}}ù).
\end{emcppslisting}

\noindent Without the two \emcppsgloss[explicit instantiation definition]{explicit-instantiation definitions} in
the \lstinline!transform.cpp! file above, its corresponding object file,
\lstinline!transform.o!, would be empty.

Note that, as of C++11, we \emph{could} place the corresponding
\emcppsgloss[explicit instantiation definition]{explicit-instantiation definitions} in the header file
for, say, documentation purposes:

\begin{emcppslisting}
// transform.h:
#ifndef INCLUDED_TRANSFORM_H
#define INCLUDED_TRANSFORM_H

template <typename T>  // declaration (only) of free-function template
T transform(const T& value);
    // Return the transform of the specified floating-point (ù{\codeincomments{value}}ù).

// explicit-instantiation declarations, available as of C++11
extern template float transform(const float&);    // user documentation only;
extern template double transform(const double&);  // has no effect whatsoever

#endif
\end{emcppslisting}

\noindent But because no definition of the \lstinline!transform! free-function
template is visible in the header, no \emph{implicit} instantiation can
result from client use; hence, the two \emcppsgloss[explicit instantiation declaration]{explicit-instantiation declarations} above for \lstinline!float! and \lstinline!double!,
respectively, do nothing.

\subsection[Potential Pitfalls]{Potential Pitfalls}\label{potential-pitfalls-externtemplate}

\subsubsection[Corresponding explicit-instantiation declarations and definitions]{Corresponding explicit-instantiation declarations and definitions}\label{corresponding-explicit-instantiation-declarations-and-definitions}

To realize a reduction in object-code size for individual translation
units and yet still be able to link all valid programs successfully into
a well-formed program, several moving parts have to be brought together
correctly:
\begin{enumerate}
\item{Each general template, \lstinline!C<T>!, whose object code bloat is to be optimized must be declared within some designated component’s header file, \lstinline!c.h!.}
\item{The specific definition of each \lstinline!C<T>! relevant to an explicit specialization being optimized — including general, partial-specialization, and full-specialization definitions — must appear in the header file prior to its corresponding \emcppsgloss[explicit instantiation declaration]{explicit-instantiation declaration}.}
\item{Each \emcppsgloss[explicit instantiation declaration]{explicit-instantiation declaration} for each specialization of each separate top-level — i.e., class, function, or variable — template must appear in the component’s \lstinline!.h! file \emph{after} the corresponding general template declaration and the relevant general, partial-specialization, or full-specialization definition, but, in practice, always after \emph{all} such definitions, not just the relevant one.}
\item{Each template specialization having an \emcppsgloss[explicit instantiation declaration]{explicit-instantiation declaration} in the header file must have a corresponding \emcppsgloss[explicit instantiation definition]{explicit-instantiation definition} in the component’s implementation file, \lstinline!c.cpp!.}
\end{enumerate}
Absent items (1) and (2), clients would have no way to safely separate
out the usability and inlineability of the template definitions yet
consolidate the otherwise redundantly generated object-level definitions
within just a single translation unit. Moreover, failing to provide the
relevant definition would mean that any clients using one of these
specializations would either fail to compile or, arguably worse, pick up
the general definitions when a more specialized definition was intended,
likely resulting in an ill-formed program.

Failing item (3), the object code for that particular specialization of
that template will be generated locally in the client's translation unit
as usual, negating any benefits with respect to local object-code size,
irrespective of what is specified in the \lstinline!c.cpp! file.

Finally, unless we provide a matching \emcppsgloss[explicit instantiation definition]{explicit-instantiation definition} in the \lstinline!c.cpp! file for each and every
corresponding \emcppsgloss[explicit instantiation declaration]{explicit-instantiation declaration} in the
\lstinline!c.h! file as in item (4), our optimization attempts might well
result in a library component that compiles, links, and even passes some
unit tests but, when released to our clients, fails to link.
Additionally, any \emcppsgloss[explicit instantiation definition]{explicit-instantiation definition} in
the \lstinline!c.cpp! file that is not accompanied by a corresponding
\emcppsgloss[explicit instantiation declaration]{explicit-instantiation declaration} in the \lstinline!c.h!
file will inflate the size of the \lstinline!c.o! file with no possibility
of reducing code bloat in client code{\cprotect\footnote{Fortunately,
these extra instantiations do not result in multiply-defined symbols
because they still reside in their own \emcppsgloss{sections} and are marked
  as \emph{weak} symbols.}}:

\begin{emcppslisting}[emcppsbatch=e7]
// c.h:
#ifndef INCLUDED_C                            // internal include guard
#define INCLUDED_C

template <typename T> void f(T v) {/*...*/};  // general template definition

extern template void f<int>(int v);           // OK, matched in (ù{\codeincomments{c.cpp}}ù)
extern template void f<char>(char c);         // Error, unmatched in (ù{\codeincomments{.cpp}}ù) file

#endif
\end{emcppslisting}
\vspace*{2ex}   %%%% extra space to highlight different snippets
\begin{emcppslisting}[emcppsbatch=e7]
// c.cpp:
#include <c.h>                                 // incorporate own header first

extern template void f<int>(int v);            // OK, matched in (ù{\codeincomments{c.h}}ù)
extern template void f<double>(double v);      // Bug, unmatched in (ù{\codeincomments{c.h}}ù) file
\end{emcppslisting}
\vspace*{2ex}   %%%% extra space to highlight different snippets
\begin{emcppslisting}[emcppsbatch=e7]
// client.cpp:
#include <c.h>

void client()
{
    int    i = 1;
    char   c = 'a';
    double d = 2.0;

    f(i);  // OK, matching explicit-instantiation directives
    f(c);  // Link-Time Error, no matching explicit-instantiation definition
    f(d);  // Bug, size increased due to no matching explicit-instantiation
           // declaration
}
\end{emcppslisting}

\noindent In the example above, \lstinline!f(i)! works as expected, with the linker
finding the definition of \lstinline!f<int>! in \lstinline!c.o!; \lstinline!f(c)!
fails to link, because no definition of \lstinline!f<c>! is guaranteed to
be found anywhere; and \lstinline!f(d)! accidentally works by silently
generating a \emph{redundant} local copy of \lstinline!f<double>! in
\lstinline!client.o! while another, identical definition is generated
explicitly in \lstinline!c.o!. Importantly, note that
\lstinline!extern!~\lstinline!template! has \emph{absolutely no affect} on
overload resolution because the call to \lstinline!f(c)! did \emph{not}
resolve to \lstinline!f<int>!.

\subsubsection[Accidentally making matters worse]{Accidentally making matters worse}\label{accidentally-making-matters-worse}

When making the decision to preinstantiate common specializations of
popular templates within some designated \lstinline!.o! file, one must
consider that not all programs necessarily need every (or even any) such
instantiation. Special consideration should be given to classes that
have many methods but typically use only a few.

The language feature is sufficiently flexible that one can suppress and
preinstantiate just one or a handful of member functions of such a type. Intuition is all well and good, but measurement simply has no
substitute.

If one suspects that \emcppsgloss[explicit instantiation directive]{explicit-instantiation directives}
might profitably reduce the size of libraries resulting from object code
that is bloated due to redundant local reinstantiations of popular
templates on common types, measuring before and after and retaining the
change \emph{only} if it offers a significant --- at least measurable
--- improvement avoids complicating the codebase without a verifiable
return on the investment. Finally, remember that one might need to
explicitly tell the linker to strip unused \emcppsgloss{sections} resulting,
for example, from forced instantiation of common template
specializations, to avoid inadvertently bloating executables, which
could adversely affect load times.

\subsection[Annoyances]{Annoyances}\label{annoyances-externtemplate}

\subsubsection[No good place to put definitions for unrelated classes]{No good place to put definitions for unrelated classes}\label{no-good-place-to-put-definitions-for-unrelated-classes}

When we consider the implications of physical
dependency,\footnote{See \cite{lakos96}}\footnote{See \cite{lakos20}} determining in which component to deposit the
specialized definitions can be problematic. For example, consider a
codebase implementing a core library that provides both a nontemplated
\lstinline!String! class and a \lstinline!Vector! container class template.
These fundamentally unrelated entities would ideally live in separate
physical \emcppsgloss{components} --- i.e., \lstinline!.h!/\lstinline!.cpp! pairs
--- neither of which depends physically on the other. That is, an
application using just one of these components could ideally be
compiled, linked, tested, and deployed entirely independently of the
other. Now, consider a large codebase that makes heavy use of
\lstinline!Vector<String>!: In what component should the object-code-level
definitions for the \lstinline!Vector<String>! specialization
reside?{\cprotect\footnote{Note that the problem of determining in which
component to instantiate the object-level implementation of a template
for a user-defined type is similar to that of specializing an
  arbitrary user-defined trait for a user-defined type.}} There are two
obvious alternatives:
\begin{enumerate}
\item{\lstinline!vector!: In this case, \lstinline!vector.h! would hold  \lstinline!extern!~\lstinline!template!~\lstinline!class!~\lstinline!Vector<String>;! --- the \emcppsgloss[explicit instantiation declaration]{explicit-instantiation \emph{declaration}} --- and \lstinline!vector.cpp! would hold\linebreak[4]%%%%%
\lstinline!template!~\lstinline!class!~\lstinline!Vector<String>;! --- the \emcppsgloss[explicit instantiation definition]{explicit-instantiation \emph{definition}}. With this approach, we would create a physical dependency of the \lstinline!vector! component on \lstinline!string!. Any client program wanting to use a \lstinline!Vector! would also depend on \lstinline!string! regardless of whether it was needed.}
\item{\lstinline!string!: In this case, \lstinline!string.h! and \lstinline!string.cpp! would instead be modified so as to depend on \lstinline!vector!. Clients wanting to use a \lstinline!string! would also be forced to depend physically on \lstinline!vector! \emph{at} \emph{compile} \emph{time}.}
\end{enumerate}
Another possibility might be to create a third component, call it
\lstinline!stringvector!, that itself depends on both \lstinline!vector!
and \lstinline!string!. By
\emcppsgloss{escalating}\footnote{\cite{lakos20}, section~3.5.2, ``Escalation," pp.~604--614} the mutual dependency
to a higher level in the physical hierarchy, we avoid forcing any client
to depend on more than what is actually needed. The practical drawback
to this approach is that only those clients that proactively include the
composite \lstinline!stringvector.h! header would realize any benefit;
fortunately, in this case, there is no \emcppsgloss[one definition rule]{one-definition rule (ODR)} issue if they don't.

Finally, complex machinery could be added to both \lstinline!string.h! and
\lstinline!vector.h! to conditionally include \lstinline!stringvector.h!
whenever both of the other headers are included; such heroic efforts
would, nonetheless, involve a \emcppsgloss{cyclic physical dependency} among
all three of these components. Circular intercomponent collaborations
are best avoided.\footnote{\cite{lakos20}, section~3.4, ``Avoiding Cyclic Link-Time Dependencies," pp.~592--601}

\subsubsection[All members of an explicitly defined template class must be valid]{All members of an explicitly defined template class must be valid}\label{all-members-of-an-explicitly-defined-template-class-must-be-valid}

In general, when using a template class, only those members that are
actually used get implicitly instantiated. This hallmark allows class
templates to provide functionality for parameter types having certain
capabilities (e.g., default constructible) while also providing partial
support for types lacking those same capabilities. When providing an
\emcppsgloss[explicit instantiation definition]{explicit-instantiation definition}, however, \emph{all}
members of a template class are instantiated.

Consider a simple class template having a data member that can be either
default-initialized (via the template's default constructor) or
initialized with an instance of the member's type (supplied at
construction):

\begin{emcppslisting}[emcppsbatch=e8]
template <typename T>
class W
{
    T d_t;  // a data member of type (ù{\codeincomments{T}}ù)

public:
    W() : d_t() {}
        // Create an instance of (ù{\codeincomments{W}}ù) with a default-constructed (ù{\codeincomments{T}}ù) member.

    W(const T& t) : d_t(t) {}
        // Create an instance of (ù{\codeincomments{W}}ù) with a copy of the specified (ù{\codeincomments{t}}ù).

    void doStuff() { /* do stuff */ }
};
\end{emcppslisting}

\noindent This class template can be used successfully with a type, such as
\lstinline!U! in the code snippet below, that is not default constructible:

\begin{emcppslisting}[emcppsbatch=e8]
struct U
{
    U(int i) { /* do something with (ù{\codeincomments{i}}ù) */ }
    // ...
};

void useWU()
{
    W<U> wu1(U(17));  // OK, using copy constructor for (ù{\codeincomments{U}}ù)
    wu1.doStuff();
}
\end{emcppslisting}

\noindent As it stands, the code above is well formed even though
\lstinline!W<U>::W()! would fail to compile if instantiated. Consequently,
although providing an \emcppsgloss[explicit instantiation declaration]{explicit-instantiation declaration}
for \lstinline!W<U>! is valid, a corresponding
\emcppsgloss[explicit instantiation definition]{explicit-instantiation definition} for \lstinline!W<U>!
fails to compile, as would an implicit instantiation of
\lstinline!W<U>::W()!:

\begin{emcppslisting}[emcppsbatch=e8]
extern template class W<U>;  // Valid: Suppress implicit instantiation of (ù{\codeincomments{W<U>}}ù).

template class W<U>;         // Error, (ù{\codeincomments{U::U()}}ù) not available for (ù{\codeincomments{W<U>::W()}}ù)

void useWU0()
{
    W<U> wu0{};       // Error, (ù{\codeincomments{U::U()}}ù) not available for (ù{\codeincomments{W<U>::W()}}ù)
}
\end{emcppslisting}

\noindent Unfortunately, the only workaround to achieve a comparable reduction in
code bloat is to provide member-specific \emcppsgloss[explicit instantiation directive]{explicit-instantiation directives} for each valid member of \lstinline!W<U>!, an approach
that would likely carry a significantly greater maintenance burden:

\begin{emcppslisting}[emcppsbatch=e8]
extern template W<U>::W(const U& u);   // suppress individual member
extern template void W<U>::doStuff();  //    "         "        "
// ... Repeat for all other functions in (ù{\codeincomments{W}}ù) except (ù{\codeincomments{W<U>::W()}}ù).

template W<U>::W(const U& u);          // instantiate individual member
template void W<U>::doStuff();         //       "         "        "
// ... Repeat for all other functions in (ù{\codeincomments{W}}ù) except (ù{\codeincomments{W<U>::W()}}ù).
\end{emcppslisting}

\noindent The power and flexibility to make it all work --- albeit annoyingly ---
are there nonetheless.

\subsection[See Also]{See Also}\label{see-also}

\begin{itemize}
\item{\seealsoref{variable-templates}{\locationb}Extension of the template syntax for defining a family of like-named variables or static data members that can be instantiated explicitly}
\end{itemize}

\subsection[Further Reading]{Further Reading}\label{further-reading}

\begin{itemize}
\item{For a different perspective on this feature, see \cite{lakos20},
section 1.3.16, ``\texttt{extern} Templates,'' pp.~183--185.}
\item{For a more complete discussion of how compilers and linkers work with
respect to C++, see \cite{lakos20}, Chapter 1, ``Compilers, Linkers, and
Components,'' pp.~123--268.}
\end{itemize}
