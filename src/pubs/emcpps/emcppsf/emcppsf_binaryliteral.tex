\emcppsFeature{
    short={Binary Literals},
    long={Binary Literals: The {\SecCode 0b} Prefix},
}{binary-literals}
%% copyedits in and proofed
%% 27 Jan, code checked and replaced as needed by Josh

\emcppsFeature{
    toclong={Binary Literals: The \lstinline!0b! Prefix},
    short={Binary Literals},
    long={Binary Literals: The {\SecCode 0b} Prefix},
}{binary-literals}
%\section[Binary Literals]{Binary Literals}\label{binary-literals}\sectionmark{Binary Literals}
%\section[Binary Literals]{Binary Literals: The {\SecCode 0b} Prefix}\label{binary-literals}
\setcounter{table}{0}
\setcounter{footnote}{0}
\setcounter{lstlisting}{0}


\textit{Binary literals} are \textbf{integer literals}\glossary{integer literals} representing their values in base 2.

\subsection[Description]{Description}\label{description-binary}

A binary literal is an integral value represented in code in a binary numeral system. A binary literal consists of a \lstinline!0b! or \lstinline!0B! prefix followed by a nonempty sequence of binary digits, namely, \lstinline!0! and \lstinline!1!{\cprotect\footnote{Prior to being introduced in C++14, GCC supported binary literals (with the same syntax as the standard feature) as a nonconforming extension since version~4.3.0, released in March 2008;
%4.3 (released between March 2008 and May 2010);
for more details, see https://gcc.gnu.org/gcc-4.3/.}}:

%A \emph{binary literal} (e.g., \texttt{0b1110}) --- much like a
%hexadecimal literal (e.g., \texttt{0xE}) or an octal literal (e.g.,
%\texttt{016}) --- is a kind of \emph{integer literal} (in this case,
%having the \emph{decimal} value \texttt{14}). A binary literal consists
%of a \texttt{0b} (or \texttt{0B}) prefix followed by a nonempty
%sequence of binary digits (\texttt{0} or \texttt{1})

\begin{emcppslisting}[emcppsbatch=e1]
int i = 0b11110000;  // equivalent to (ù{\codeincomments{240}}ù), (ù{\codeincomments{0360}}ù), or (ù{\codeincomments{0xF0}}ù)
int j = 0B11110000;  // same value as above
\end{emcppslisting}

\noindent The first digit after the \lstinline!0b! prefix is the most significant
one:

\begin{emcppslisting}[emcppsbatch=e1]
static_assert(0b0     ==  0, "");  // 0*2^0
static_assert(0b1     ==  1, "");  // 1*2^0
static_assert(0b10    ==  2, "");  // 1*2^1 + 0*2^0
static_assert(0b11    ==  3, "");  // 1*2^1 + 1*2^0
static_assert(0b100   ==  4, "");  // 1*2^2 + 0*2^1 + 0*2^0
static_assert(0b101   ==  5, "");  // 1*2^2 + 0*2^1 + 1*2^0
// ...
static_assert(0b11010 == 26, "");  // 1*2^4 + 1*2^3 + 0*2^2 + 1*2^1 + 0*2^0
\end{emcppslisting}


\noindent Leading zeros --- as with octal and hexadecimal (but not decimal)
literals --- are ignored but can be added for readability:

\begin{emcppslisting}
static_assert(0b00000000 ==   0, "");
static_assert(0b00000001 ==   1, "");
static_assert(0b00000010 ==   2, "");
static_assert(0b00000100 ==   4, "");
static_assert(0b00001000 ==   8, "");
static_assert(0b10000000 == 128, "");
\end{emcppslisting}


\noindent The type of a binary literal
%{\cprotect\footnote{Its \emph{value category} is \emph{prvalue} like every other integer literal.}}
is by
default an \lstinline!int! unless that value cannot fit in an
\lstinline!int!. In that case, its type is the first type in the sequence
\{\lstinline!unsigned!~\lstinline!int!, %\pagebreak%%%%%
 \lstinline!long!,
 \lstinline!unsigned!~\lstinline!long!, \lstinline!long!~\lstinline!long!,
\lstinline!unsigned!~\lstinline!long!~\lstinline!long!\} in which it will fit. This
same type list applies for both \lstinline!octal! and \lstinline!hex!
literals but not for decimal literals, which, if initially
\lstinline!signed!, skip over any \lstinline!unsigned! types, and vice versa;
see \intrarefsimple{description-binary}.
  If neither of those is applicable, the compiler may use implementation-defined extended integer types such as \lstinline!__int128! to represent the literal if it fits --- otherwise the program is ill-formed:

% If neither of those is applicable, then the program is \emph{ill-formed}

\begin{emcppslisting}[emcppsignore={Invalid Descriptive Literals}]
// example platform 1:
// ((ù{\codeincomments{sizeof(int)}}ù): 4; (ù{\codeincomments{sizeof(long)}}ù): 4; (ù{\codeincomments{sizeof(long long)}}ù): 8)
auto i32  = 0b0111...[ 24 1-bits]...1111;  // (ù{\codeincomments{i32}}ù) is (ù{\codeincomments{int}}ù).
auto u32  = 0b1000...[ 24 0-bits]...0000;  // (ù{\codeincomments{u32}}ù) is (ù{\codeincomments{unsigned int}}ù).
auto i64  = 0b0111...[ 56 1-bits]...1111;  // (ù{\codeincomments{i64}}ù) is (ù{\codeincomments{long long}}ù).
auto u64  = 0b1000...[ 56 0-bits]...0000;  // (ù{\codeincomments{u64}}ù) is (ù{\codeincomments{unsigned long long}}ù).
auto i128 = 0b0111...[120 1-bits]...1111;  // Error, integer literal too large
auto u128 = 0b1000...[120 0-bits]...0000;  // Error, integer literal too large

// example platform 2:
// ((ù{\codeincomments{sizeof(int)}}ù): 4; (ù{\codeincomments{sizeof(long)}}ù): 8; (ù{\codeincomments{sizeof(long long)}}ù): 16)
auto i32  = 0b0111...[ 24 1-bits]...1111;  // (ù{\codeincomments{i32}}ù)  is (ù{\codeincomments{int}}ù).
auto u32  = 0b1000...[ 24 0-bits]...0000;  // (ù{\codeincomments{u32}}ù)  is (ù{\codeincomments{unsigned int}}ù).
auto i64  = 0b0111...[ 56 1-bits]...1111;  // (ù{\codeincomments{i64}}ù)  is (ù{\codeincomments{long}}ù).
auto u64  = 0b1000...[ 56 0-bits]...0000;  // (ù{\codeincomments{u64}}ù)  is (ù{\codeincomments{unsigned long}}ù).
auto i128 = 0b0111...[120 1-bits]...1111;  // (ù{\codeincomments{i128}}ù) is (ù{\codeincomments{long long}}ù).
auto u128 = 0b1000...[120 0-bits]...0000;  // (ù{\codeincomments{u128}}ù) is (ù{\codeincomments{unsigned long long}}ù).
\end{emcppslisting}


\noindent (Purely for convenience of
exposition, we have employed the C++11 \lstinline!auto! feature to
conveniently capture the type implied by the literal itself; see \featureref{\locationc}{auto-feature}.) Separately, the precise initial type of a binary literal, like any other
literal, can be controlled explicitly using the common integer-literal
suffixes \{\lstinline!u!, \lstinline!l!, \lstinline!ul!, \lstinline!ll!,
\lstinline!ull!\} in either lower- or uppercase:

\begin{emcppslisting}
auto i   = 0b101;        // type: (ù{\codeincomments{int}}ù);                 value: 5
auto u   = 0b1010U;      // type: (ù{\codeincomments{unsigned int}}ù);        value: 10
auto l   = 0b1111L;      // type: (ù{\codeincomments{long}}ù);                value: 15
auto ul  = 0b10100UL;    // type: (ù{\codeincomments{unsigned long}}ù);       value: 20
auto ll  = 0b11000LL;    // type: (ù{\codeincomments{long long}}ù);           value: 24
auto ull = 0b110101ULL;  // type: (ù{\codeincomments{unsigned long long}}ù);  value: 53
\end{emcppslisting}


\noindent Finally, note that affixing a minus sign to a binary
literal (e.g., \lstinline!-b1010!) --- just like any other integer literal
(e.g., \lstinline!-10!, \lstinline!-012!, or \lstinline!-0xa!) --- is parsed as a
non-negative value first, after which a unary minus is applied:

%\begin{emcppslisting}
%static_assert(sizeof(int) == 4, "");  // true on virtually all machines today
%static_assert(-0b1010 == -10, "");    // as if: (ù{\codeincomments{0 - 0b1010 == 0 - 10}}ù)
%static_assert(0x7fffffff != -0x7fffffff, "");  // Both values are (ù{\codeincomments{signed int}}ù).
%static_assert(0x80000000 == -0x80000000, "");  // Both values are (ù{\codeincomments{unsigned int}}ù).
%\end{emcppslisting}
\begin{emcppslisting}[emcppsignore={Invalid Descriptive Literals}]
static_assert(sizeof(int) == 4, "");  // true on virtually all machines today
static_assert(-0b1010 == -10, "");    // as if: (ù{\codeincomments{0 - 0b1010 == 0 - 10}}ù)
static_assert( 0b0111...[ 24 1-bits]...1111       //   signed
           != -0b0111...[ 24 1-bits]...1111, ""); //   signed

static_assert( 0b1000...[ 24 0-bits]...0000       // unsigned
           != -0b1000...[ 24 0-bits]...0000, ""); // unsigned
\end{emcppslisting}



%\newpage%%%%%

\subsection[Use Cases]{Use Cases}\label{use-cases}

\subsubsection[Bit masking and bitwise operations]{Bit masking and bitwise operations}\label{bit-masking-and-bitwise-operations}

Prior to the introduction of binary literals, hexadecimal and octal literals were commonly used to represent bit masks or
specific bit constants in source code.
As an example, consider a
function that returns the least significant four bits of a given
\lstinline!unsigned!~\lstinline!int! value:

\begin{emcppslisting}
unsigned int lastFourBits(unsigned int value)
{
    return value & 0xFu;
}
\end{emcppslisting}


\noindent The correctness of the \emph{bitwise and} operation above might not be
immediately obvious to a developer inexperienced with
hexadecimal literals. In contrast, using a binary literal more directly
states our intent to mask all but the four least-significant bits of the
input:

%\begin{emcppslisting}
%unsigned int lastFourBits(unsigned int value)
%{
%    return value & 0b1111u;  // The (ù{\codeincomments{u}}ù) literal suffix here is entirely optional.
%}
%\end{emcppslisting}
\begin{emcppslisting}
unsigned int lastFourBits(unsigned int value)
{
    return value & 0b1111u;
}
\end{emcppslisting}



Similarly, other bitwise operations, such as setting or getting
individual bits, might benefit from the use of binary literals. For
instance, consider a set of flags used to represent the state of an
avatar in a game:
%\enlargethispage{2ex}

%\begin{emcppslisting}
%struct AvatarStateFlags
%{
%    enum Enum
%    {
%        e_ON_GROUND    = 0b0001,
%        e_INVULNERABLE = 0b0010,
%        e_INVISIBLE    = 0b0100,
%        e_SWIMMING     = 0b1000,
%    };
%};
%
%class Avatar
%{
%    unsigned char d_state;  // power set of possible state flags
%
%public:
%    bool isOnGround() const
%    {
%        return d_flags & AvatarStateFlags::e_ON_GROUND;
%    }
%
%    // ...
%};
%\end{emcppslisting}
\begin{emcppslisting}
struct AvatarStateFlags
{
    enum Enum
    {
        e_ON_GROUND    = 0b0001,
        e_INVULNERABLE = 0b0010,
        e_INVISIBLE    = 0b0100,
        e_SWIMMING     = 0b1000,
    };
};

class Avatar
{
    unsigned char d_state;

public:
    bool isOnGround() const
    {
        return d_state & AvatarStateFlags::e_ON_GROUND;
    }

    // ...
};
\end{emcppslisting}

\noindent Note that the choice of using a nested classic \lstinline!enum! was deliberate;
see\linebreak[4] \featureref{\locationc}{enumclass}.


\subsubsection[Replicating constant binary data]{Replicating constant binary data}\label{replicating-constant-binary-data}

Especially in the context of \romeogloss{embedded development} or emulation,
a programmer will commonly write code that needs to deal
with specific ``magic'' constants (e.g., provided as part of the
specification of a CPU or virtual machine) that must be incorporated in
the program's source code. Depending on the original format of such
constants, a binary representation can be the most convenient or most
easily understandable one.

As an example, consider a function decoding instructions of a virtual
machine whose opcodes are specified in binary format:

%\begin{emcppslisting}
%#include <cstdint>  // (ù{\codeincomments{std::uint8\_t}}ù)
%
%void VirtualMachine::decodeInstruction(std::uint8_t instruction)
%{
%    switch(instruction)
%    {
%        case 0b00000000u:  // no-op
%            break;
%
%        case 0b00000001u:  // (ù{\codeincomments{add(register0, register1)}}ù)
%            d_register0 += d_register1;
%            break;
%
%        case 0b00000010u:  // (ù{\codeincomments{jmp(register0)}}ù)
%            jumpTo(d_register0);
%            break;
%
%        // ...
%    }
%}
%\end{emcppslisting}
\begin{emcppshiddenlisting}[emcppsbatch=e2]
// needed here for virtual machine, also below for visibility
#include <cstdint>  // // (ù{\codeincomments{std::uint8\_t}}ù)
struct VirtualMachine {
    std::uint8_t d_register0;
    std::uint8_t d_register1;
    void decodeInstruction(std::uint8_t instruction);
    void jumpTo(std::uint8_t address);
};
\end{emcppshiddenlisting}
\begin{emcppslisting}[emcppsbatch=e2]
#include <cstdint>  // (ù{\codeincomments{std::uint8\_t}}ù)

void VirtualMachine::decodeInstruction(std::uint8_t instruction)
{
    switch (instruction)
    {
        case 0b00000000u:  // no-op
            break;

        case 0b00000001u:  // (ù{\codeincomments{add(register0, register1)}}ù)
            d_register0 += d_register1;
            break;

        case 0b00000010u:  // (ù{\codeincomments{jmp(register0)}}ù)
            jumpTo(d_register0);
            break;

        // ...
    }
}
\end{emcppslisting}

\noindent Replicating the same binary constant specified as part of the CPU's or
virtual machine's manual or documentation directly in the source avoids the need to
mentally convert such constant data to and from, say, a hexadecimal
number.

Binary literals are also suitable for capturing bitmaps. For instance,
consider a bitmap representing the uppercase letter \textit{C}\/:

\begin{emcppslisting}
const unsigned char letterBitmap_C[] =
{
    0b00011111,
    0b01100000,
    0b10000000,
    0b10000000,
    0b10000000,
    0b01100000,
    0b00011111
};
\end{emcppslisting}


\noindent Using \emph{binary} literals makes the shape of the image that the
bitmap represents apparent directly in the source code.

\subsection[Potential Pitfalls]{Potential Pitfalls}\label{potential-pitfalls}

\hspace*{\fill}

\subsection[Annoyances]{Annoyances}\label{annoyances}

\hspace*{\fill}

\subsection[See Also]{See Also}\label{see-also}

\begin{itemize}
\item{%``\titleref{digitseparator}" on page~\pageref{digitseparator}{\seealsodelim}
%\seealsoref{digitseparator}{\locationb}Long binary literals are made much more readable by grouping digits visually.}
\seealsoref{digitseparator}{\seealsolocationb}explains grouping digits visually to make long binary literals much more readable.}
\end{itemize}

\subsection[Further Reading]{Further Reading}\label{further-reading}

\begin{itemize}
\item{\cite{informit}}
\end{itemize}


