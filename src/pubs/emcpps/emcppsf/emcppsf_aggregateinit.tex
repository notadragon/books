\emcppsFeature{
    short={Aggregate Init '14},
    long={Aggregates Having Default Member Initializers},
}{aggregate-member-initialization-relaxation}
%%% 2 Feb 2021, copyedits entered and proofed


\emcppsFeature{
    short={Aggregate Init '14},
    long={Aggregates Having Default Member Initializers},
}{aggregate-member-initialization-relaxation}
\setcounter{table}{0}
\setcounter{footnote}{0}
\setcounter{lstlisting}{0}
%\section[Aggregate Init '14]{Aggregates Having Default Member Initializers}\label{aggregate-member-initialization-relaxation}


C++14 enables the use of \romeogloss{aggregate initialization} with classes
employing default member initializers (see \featureref{\locationa}{Default-Member-Initializers}).

\subsection[Description]{Description}\label{description}

Prior to C++14, classes that used default member initializers (see \featureref{\locationa}{Default-Member-Initializers}) --- i.e., initializers that appear directly within the
scope of the class --- were not considered \romeogloss{aggregate} types:

\begin{emcppslisting}[language=C++]
struct S                // aggregate type in C++14 but not C++11
{
    int i;
    bool b = false;     // uses default member initializer
};

struct A                // aggregate type in C++11 and C++14
{
    int  i;
    bool b;             // does not use default member initializer
};
\end{emcppslisting}

\noindent Because \lstinline!A! (but not \lstinline!S!) is considered an \romeogloss{aggregate} in
C++11, instances of \lstinline!A! can be created via \romeogloss{aggregate
initialization} (whereas instances of \lstinline!S! cannot):

\begin{emcppslisting}[language=C++]
A a={100, true};  // OK in both C++11 and C++14
S s={100, true};  // error in C++11; OK in C++14
\end{emcppslisting}


\noindent Note that since C++11, direct-list-initialization may be used to perform aggregate initialization (see \featureref{\locationc}{bracedinit}):
\begin{emcppslisting}[language=C++]
A a{100, true};  // OK in both C++11 and C++14 but not in C++03
\end{emcppslisting}
As of C++14, the requirements for a type to be categorized as an
\romeogloss{aggregate} are relaxed, allowing classes employing default
member initializers to be considered as such; hence, both \lstinline!A! and
\lstinline!S! are considered \romeogloss{aggregates} in C++14 and eligible for
\romeogloss{aggregate initialization}:

\begin{emcppslisting}[language=C++]
void f()
{
    S s0{100, true};        // OK in C++14 but not in C++11
    assert(s0.i == 100);    // set via explicit aggregate initialization (above)
    assert(s0.b == true);   // set via explicit aggregate initialization (above)

    S s1{456};              // OK in C++14 but not in C++11
    assert(s1.i == 456);    // set via explicit aggregate initialization (above)
    assert(s1.b == false);  // set via default member initializer
}
\end{emcppslisting}

\noindent In the code snippet above, the C++14 aggregate \lstinline!S! is initialized
in two ways: \lstinline!s0! is created using aggregate initialization for
both data members, and \lstinline!s1! is created using aggregate
initialization for only the first data member (and the second is set via its
default member \nobreak{initializer}).

\subsection[Use Cases]{Use Cases}\label{use-cases}

\subsubsection[Configuration \lstinline!struct!s]{Configuration {\SubsubsecCode struct}s}\label{configuration-structs}

\romeogloss{Aggregates} in conjunction with default member initializers (see \featureref{\locationa}{Default-Member-Initializers}) can be used to provide concise customizable
configuration \lstinline!struct!s, packaged with typical default values. As
an example, consider a configuration \lstinline!struct! for an HTTP request
handler:

\begin{emcppslisting}[language=C++]
struct HTTPRequestHandlerConfig
{
    int maxQueuedRequests = 1024;
    int timeout           = 60;
    int minThreads        = 4;
    int maxThreads        = 8;
};
\end{emcppslisting}

\noindent \romeogloss{Aggregate initialization} can be used when creating objects of
type \nobreak{\lstinline!HTTPRequestHandlerConfig!} (above) to override one or more
of the defaults in definition order{\cprotect\footnote{In C++20, the
designated initializers feature adds flexibility (e.g., for
configuration \lstinline!struct!s, such as
\lstinline!HTTPRequestHandlerConfig!) by enabling explicit specification
of the names of the data members:

\begin{emcppslisting}[language=C++, basicstyle={\ttfamily\footnotesize}]
HTTPRequestHandlerConfig lowTimeout{.timeout = 15};
    // (ù{\codeincomments{maxQueuedRequests}}ù), (ù{\codeincomments{minThreads}}ù), and (ù{\codeincomments{maxThreads}}ù) have their default value.

HTTPRequestHandlerConfig highPerformance{.timeout = 120, .maxThreads = 16};
    // (ù{\codeincomments{maxQueuedRequests}}ù) and (ù{\codeincomments{minThreads}}ù) have their default value.
\end{emcppslisting}
      }}:

\begin{emcppslisting}[language=C++]
HTTPRequestHandlerConfig getRequestHandlerConfig(bool inLowMemoryEnvironment)
{
    if (inLowMemoryEnvironment)
    {
        return HTTPRequestHandlerConfig{128};
            // (ù{\codeincomments{timeout}}ù), (ù{\codeincomments{minThreads}}ù), and (ù{\codeincomments{maxThreads}}ù) have their default value.
    }
    else
    {
        return HTTPRequestHandlerConfig{2048, 120};
            // (ù{\codeincomments{minThreads}}ù), and (ù{\codeincomments{maxThreads}}ù) have their default value.
    }
}

// ...
\end{emcppslisting}


\subsection[Potential Pitfalls]{Potential Pitfalls}\label{potential-pitfalls}

None so far

\subsection[Annoyances]{Annoyances}\label{annoyances}

\subsubsection[Syntactical ambiguity in the presence of brace elision]{Syntactical ambiguity in the presence of brace elision}\label{syntactical-ambiguity-in-the-presence-of-brace-elision}

During the initialization of multilevel \romeogloss{aggregates}, braces
around the initialization of a nested aggregate can be omitted
(\romeogloss{brace elision}):

\begin{emcppslisting}[language=C++]
struct S
{
    int arr[3];
};

S s0{{0, 1, 2}};  // OK, nested (ù{\codeincomments{arr}}ù) initialized explicitly
S s1{0, 1, 2};    // OK, brace elision for nested (ù{\codeincomments{arr}}ù)
\end{emcppslisting}

\noindent The possibility of \romeogloss{brace elision} creates an interesting
syntactical ambiguity when used alongside \romeogloss{aggregates} with
default member initializers (see \featureref{\locationa}{Default-Member-Initializers}). Consider a
\lstinline!struct!~\lstinline!X! containing three data members, one of which
has a default value:

\begin{emcppslisting}[language=C++]
struct X
{
    int a;
    int b;
    int c = 0;
};
\end{emcppslisting}

\noindent Now, consider various ways in which an array of elements of type
\lstinline!X! can be initialized:

\begin{emcppslisting}[language=C++]
X xs0[] = {{0, 1}, {2, 3}, {4, 5}};
    // OK, clearly 3 elements having the respective values:
    // (ù{\codeincomments{\{0, 1, 0\}}}ù), (ù{\codeincomments{\{2, 3, 0\}}}ù), (ù{\codeincomments{\{4, 5, 0\}}}ù)

X xs1[] = {{0, 1, 2}, {3, 4, 5}};
    // OK, clearly 2 elements with values:
    // (ù{\codeincomments{\{0, 1, 2\}}}ù), (ù{\codeincomments{\{3, 4, 5\}}}ù)

X xs2[] = {0, 1, 2, 3, 4, 5};
    // ...?
\end{emcppslisting}

\noindent Upon seeing the definition of \lstinline!xs2!, a programmer not versed in
the details of the C++ Language Standard might be unsure as to whether the
initializer of \lstinline!xs2! is three elements (like \lstinline!xs0!) or two
elements (like \lstinline!xs1!). The Standard is, however, clear that the
compiler will interpret \lstinline!xs2! the same as \lstinline!xs1!, and,
thus, the default values of \lstinline!X::c! for the two array elements
will be replaced with \lstinline!2! and \lstinline!5!, respectively.

\subsection[See Also]{See Also}\label{see-also}

\begin{itemize}
\item{\seealsoref{Default-Member-Initializers}{\seealsolocationa}allows developers to provide a default initializer for a data member directly in the definition of a class.}
\item{\seealsoref{bracedinit}{\seealsolocationc}introduces a syntactically similar feature for initializing objects in a uniform manner.}
\end{itemize}

\subsection[Further Reading]{Further Reading}\label{further-reading}



