% 23 April 2021, to Josh for code check



\emcppsFeature{
    short={\lstinline!decltype(auto)!},
    tocshort={\TOCCode decltype(auto)},
    long={Deducing Types Using {\SecCode decltype} Semantics},
    toclong={Deducing Types Using \lstinline!decltype! Semantics},
    rhshort={\RHCode decltype(auto)},
}{decltypeauto}
\label{deducing-types-using-decltype-semantics}
\setcounter{table}{0}
\setcounter{footnote}{0}
\setcounter{lstlisting}{0}
%\section[{\tt decltypeauto}]{Deducing Types Using {\SecCode decltype} Semantics}\label{decltypeauto}
%\subsection[Deducing Types Using \tt{decltype} Semantics]{Deducing Types Using {\SubsecCode decltype} Semantics}\label{deducing-types-using-decltype-semantics}


In a C++14 variable declaration, \lstinline!decltype(auto)! can act as a
\emcppsgloss{placeholder type} that is deduced to exactly match the type of
the variable's initializer, preserving the initializer's \emcppsgloss{value
category}, unlike the \lstinline!auto! placeholder.

\subsection[Description]{Description}\label{description}

The type specifier, \lstinline!auto! (see \featureref{\locationc}{auto}),
can be used in C++11 as a placeholder to declare a variable whose type
is deduced from the variable's initializer:

\begin{emcppslisting}[emcppsbatch=e1]
class C { /*...*/ };

C f1();

auto a1 = 0;    // deduced type (ù{\codeincomments{int}}ù)
auto a2{f1()};  // deduced type (ù{\codeincomments{C}}ù)
\end{emcppslisting}
    

C++14 introduced a new placeholder, \lstinline!decltype(auto)!, which can
be used in most of the same contexts as \lstinline!auto!. For the example
above, \lstinline!decltype(auto)! behaves identically to \lstinline!auto!:

\begin{emcppslisting}[emcppsbatch=e1,emcppsstandards={c++14}]
decltype(auto) a3 = 0;    // deduced type (ù{\codeincomments{int}}ù)
decltype(auto) a4{f1()};  // deduced type (ù{\codeincomments{C}}ù)
\end{emcppslisting}
    

The literal, \lstinline!0!, has type \lstinline!int!; initializing \lstinline!a3!
with \lstinline!0! thus yields a variable of type \lstinline!int!. Similarly,
the expression \lstinline!f1()! has type \lstinline!C!, yielding a variable of
type \lstinline!C! when used to initialize \lstinline!a4!.

Unlike plain \lstinline!auto!, the deduced type of a variable declared with
type \lstinline!decltype(auto)! is determined not by using
template-argument deduction rules, but by applying the \lstinline!decltype!
operator to the initialization expression. In practice, this semantic
means that the cv-qualifiers and \emcppsgloss{value category} (see Appendix~\ref{appendix-value-categories} on page~\pageref{appendix-value-categories}) of the initializer are
preserved for \lstinline!decltype(auto)! when they would be discarded for
plain \lstinline!auto!:

\begin{emcppslisting}[emcppsbatch=e1]
int&    f2();
C&&     f3();
C       c1;
const C cc1;
C*      p1 = &c1;

auto           i1  = 4;      // deduced as (ù{\codeincomments{int}}ù)
decltype(auto) i2  = 4;      //    "     " (ù{\codeincomments{int}}ù)

auto           c7  = f2();   //    "     " (ù{\codeincomments{int}}ù)
decltype(auto) c8  = f2();   //    "     " (ù{\codeincomments{int\&}}ù)

auto           i3  = f3();   //    "     " (ù{\codeincomments{C}}ù)
decltype(auto) i4  = f3();   //    "     " (ù{\codeincomments{C\&\&}}ù)

auto           c9  = cc1;    //    "     " (ù{\codeincomments{C}}ù)
decltype(auto) c10 = cc1;    //    "     " (ù{\codeincomments{const C}}ù)

auto           c11 = (cc1);  //    "     " (ù{\codeincomments{C}}ù)
decltype(auto) c12 = (cc1);  //    "     " (ù{\codeincomments{const C\&}}ù)
\end{emcppslisting}
    

As with the \lstinline!decltype! operator, \lstinline!decltype(auto)! is one
of the few constructs that preserves the distinction between the two
categories of \romeovalue{rvalue} --- \romeovalue{prvalues} (e.g., literal
\lstinline!4!) and \romeovalue{xvalues} (e.g., \lstinline!f3()!).

Both \lstinline!auto! and \lstinline!decltype(auto)! can be used as
placeholders for a function return type, indicating that the return type
should be deduced from the function's \lstinline!return! statement(s). The
deduced return-type feature is covered in its own section (see
\featureref{\locationf}{autoreturn}). Note that the deduction rules described for
function return types in that section refer to the ones described for
variables here. Readers are, therefore, advised to read this section
first.

\subsubsection[Specification]{Specification}\label{specification}

The \lstinline!decltype(auto)! placeholder can appear in most places where
\lstinline!auto! can appear:
\begin{enumerate}
\item{As the type in the declaration of an initialized variable, including variables defined in the header of a loop or \lstinline!switch! statement}
\item{As the type of object allocated and initialized by a \lstinline!new! expression}
\item{As the type returned by a function or conversion operator}
\end{enumerate}
The last is the most common use of \lstinline!decltype(auto)! and is
described in detail in \featureref{\locationf}{autoreturn}. Note that
\lstinline!decltype(auto)! cannot be used to declare a parameter of a
generic \emcppsgloss[lambda expressions]{lambda expression}; see \featureref{\locationd}{genericlambda}.

For a variable, \lstinline!v!, declared with \lstinline!decltype(auto)! and
initialized with an expression, \lstinline!expr!, the type of \lstinline!v! is
deduced to be the type denoted by \lstinline!decltype(expr)!; see
\featureref{\locationa}{decltype}. This semantic means that the deduced
type might be cv-qualified and/or a reference:

\begin{emcppslisting}[emcppsbatch=e2,emcppsstandards={c++14}]
struct C1 { /*...*/ };

int&     f2();
C1&&     f3();
C1       c1;
const C1 cc1;

decltype(f2())  i1  = f2();   // deduced as (ù{\codeincomments{int\&}}ù)
decltype(auto)  i2  = f2();   // equivalent to (ù{\codeincomments{i1}}ù)

decltype(f3())  c2  = f3();   // deduced as (ù{\codeincomments{C1\&\&}}ù)
decltype(auto)  c3  = f3();   // equivalent to (ù{\codeincomments{c2}}ù)

decltype(c1)    c4  = c1;     // deduced as (ù{\codeincomments{C1}}ù)
decltype(auto)  c5  = c1;     // equivalent to (ù{\codeincomments{c4}}ù)

decltype((c1))  c6  = c1;     // deduced as (ù{\codeincomments{C1\&}}ù)
decltype(auto)  c7  = (c1);   // equivalent to (ù{\codeincomments{c6}}ù)

decltype(cc1)   cc2 = cc1;    // deduced as (ù{\codeincomments{const C1}}ù)
decltype(auto)  cc3 = cc1;    // equivalent to (ù{\codeincomments{cc2}}ù)
decltype((cc1)) cc4 = cc1;    // deduced as (ù{\codeincomments{const C1\&}}ù)
decltype(auto)  cc5 = (cc1);  // equivalent to (ù{\codeincomments{cc4}}ù)

decltype({ 3 }) x1 = { 3 };   // Error, not an expression
decltype(auto)  x2 = { 3 };   // Error, not an expression
\end{emcppslisting}
    

The semantics of the \lstinline!decltype! operator, when applied to an
expression consisting of a single variable, cause \lstinline!decltype(c1)!
to yield type \lstinline!C1! and \lstinline!decltype((c1))! to yield reference
type \lstinline!C1&!, as in the definitions of \lstinline!c4! and
\lstinline!c6!, respectively; variables \lstinline!c5! and \lstinline!c7!,
therefore, also have the types \lstinline!C1! and \lstinline!C1&!. A
braced-initializer list such as \lstinline!{!~\lstinline!3!~\lstinline!}! is not
an expression; thus, \lstinline!x1! and \lstinline!x2! are both invalid.

Note that functions returning scalars discard top-level cv-qualifiers on
their return types, so a type deduced from a call to such a function
will not reflect top-level cv-qualifiers even when defined with
\lstinline!decltype(auto)!:

\begin{emcppslisting}[emcppsbatch=e2]
template <typename T> T f4();

decltype(auto) v1 = f4<const C1>();           // deduced as (ù{\codeincomments{const C1}}ù)
decltype(auto) v2 = f4<const int>();          //    "     " (ù{\codeincomments{int}}ù)
decltype(auto) v3 = f4<const int&>();         //    "     " (ù{\codeincomments{const int\&}}ù)
decltype(auto) v4 = f4<const char* const>();  //    "     " (ù{\codeincomments{const char*}}ù)
\end{emcppslisting}
    

The top-level \lstinline!const! qualifier on the class type,
\lstinline!const!~\lstinline!C1!, and on the reference type,
\lstinline!const!~\lstinline!int&!, are preserved but not on the scalar type,
\lstinline!const!~\lstinline!int!. The \lstinline!const!ness of the pointer
itself, in \lstinline!const!~\lstinline!char*!~\lstinline!const!, is similarly
discarded, as it is the top-level cv-qualifier on a scalar type.

When a function name is used as the initializer expression, it
automatically decays to a pointer type when initializing a variable
declared with type \lstinline!auto! but does not decay when the variable is
declared with type \lstinline!decltype(auto)!; thus, the deduced type of
\lstinline!fx2! is a function type, which is not an allowed type for a
variable. Initializer expressions having pointer-to-function type or
reference-to-function type do not pose a problem, as seen with
\lstinline!fx3! and \lstinline!fx4!.

\begin{emcppslisting}[emcppsbatch=e2]
auto           fx1 = f2;     // OK, deduced as (decayed type) (ù{\codeincomments{int\& (*)()}}ù)
decltype(auto) fx2 = f2;     // Error, cannot define variable of type (ù{\codeincomments{int\&()}}ù)

auto           fx3 = &f3;    // OK, deduced as (ù{\codeincomments{C1\&\& (*)()}}ù)
decltype(auto) fx4 = &f3;    // OK,    "     " (ù{\codeincomments{C1\&\& (*)()}}ù)

auto&          fx5 = *fx3;   // OK,    "     " (ù{\codeincomments{C1\&\& (\&)()}}ù)
decltype(auto) fx6 = *fx3;   // OK,    "     " (ù{\codeincomments{C1\&\& (\&)()}}ù)
\end{emcppslisting}
    

Note that \lstinline!fx5! uses \lstinline!&! to force the variable type to be
a reference. The ability to add reference specifiers and cv-qualifiers
to a placeholder is not available for \lstinline!decltype(auto)!, as
described in the next section.

\subsubsection[Syntactic restrictions]{Syntactic restrictions}\label{syntactic-restrictions}

When used as a type placeholder, \lstinline!decltype(auto)! must appear
alone, unadorned by cv-qualifiers, reference type specifiers, pointer
type specifiers, or function parameter lists:

\begin{emcppslisting}[emcppsbatch=e3,emcppsstandards={c++14}]
int&& f1();
int i1 = 5;

decltype(auto)       i2     = f1();  // OK, deduced as (ù{\codeincomments{int\&\&}}ù)
const decltype(auto) i3     = f1();  // Error, (ù{\codeincomments{const}}ù) qualifier not allowed
decltype(auto)&&     i4     = f1();  // Error, reference operator "     "
decltype(auto)*      i5     = &i1;   // Error, pointer operator   "     "
decltype(auto)     (*fx1)() = &f1;   // Error, function parameters "    "
\end{emcppslisting}
    

All of the above definitions would be valid if \lstinline!decltype(auto)!
were replaced with \lstinline!auto!:

\begin{emcppslisting}[emcppsbatch=e3]
auto                 i6     = f1();  // OK, (ù{\codeincomments{i6}}ù) deduced as (ù{\codeincomments{int}}ù)
const auto           i7     = f1();  // OK, (ù{\codeincomments{i7}}ù)    "     " (ù{\codeincomments{const int}}ù)
auto&&               i8     = f1();  // OK, (ù{\codeincomments{i8}}ù)    "     " (ù{\codeincomments{int\&\&}}ù)
auto*                i9     = &i1;   // OK, (ù{\codeincomments{i9}}ù)    "     " (ù{\codeincomments{int*}}ù)
auto               (*fx2)() = &f1;   // OK, (ù{\codeincomments{fx2}}ù)   "     " (ù{\codeincomments{int\&\& (*)()}}ù)
\end{emcppslisting}
    

The \lstinline!decltype(auto)! placeholder cannot be used to define a
variable without an initializer because there would be no way to deduce
its type. If multiple variables are defined in a single
\lstinline!decltype(auto)! definition, they must all have initializers of
exactly the same type:

\begin{emcppslisting}[emcppsbatch=e3]
#include <utility>  // (ù{\codeincomments{std::move}}ù)

decltype(auto) v1;                             // Error, no initializer
decltype(auto) v2 = f1(), v3 = std::move(i1);  // OK, deduced as (ù{\codeincomments{int\&\&}}ù)
decltype(auto) v4 = 5, v5 = f1();              // Error, ambiguous deduction
\end{emcppslisting}
    

A non\lstinline!static! member variable cannot be declared using
\lstinline!decltype(auto)!, even if provided with a default member
initializer (see \featureref{\locationc}{default-member-initializers}):

\begin{emcppslisting}
struct C1
{
    decltype(auto) d_data = f();  // Error, (ù{\codeincomments{decltype(auto)}}ù) for member variable
};
\end{emcppslisting}
    

A \lstinline!constexpr! static member variable (see
\featureref{\locationc}{constexprvar}) that is initialized at the point of
declaration can be declared using \lstinline!decltype(auto)!, but a
non\lstinline!constexpr! static member variable cannot, simply because
non\lstinline!constexpr! static members cannot be inline-initialized at the
point of declaration, independently of the \lstinline!decltype(auto)!
feature:

\begin{emcppslisting}[emcppsbatch=e4,emcppsstandards={c++14}]
constexpr int f2() { return 5; }

struct C2
{
    static constexpr decltype(auto) s_mem1 = f2();  // OK
    static           decltype(auto) s_mem2 = f2();  // Error, inline init
};
\end{emcppslisting}
    

A variable with static storage duration (either in global or class
scope) can be declared using an explicit type then redeclared and
initialized using \lstinline!decltype(auto)!. Note, however, that some
popular compilers reject these redeclarations{\cprotect\footnote{GCC~10.2 and MSVC~19.28, among many other compilers, reject \lstinline!auto!
redeclaration of previously declared variables; see
GCC bug report~60352. However, nothing in the C++14 Standard appears
to disallow such redeclarations, and an example in the C++20 Standard
  indicates that they are valid.}}:

\begin{emcppslisting}[emcppsbatch=e4,emcppserrorlines={8,9}]
extern int gi;  // forward declaration

struct C3
{
    static decltype(f2()) s_mem1;  // type (ù{\codeincomments{int}}ù)
};

decltype(auto) gi =         f2();  // OK, compatible redeclaration
decltype(auto) C3::s_mem1 = f2();  // OK, compatible redeclaration
\end{emcppslisting}
    

\subsubsection[\lstinline!new! expressions]{{\SubsubsecCode new} expressions}\label{new-expressions}

When used in a \lstinline!new! expression, \lstinline!decltype(auto)! offers
little benefit over plain \lstinline!auto! and will sometimes cause
otherwise-valid code to not compile:

\begin{emcppslisting}[emcppsstandards={c++14}]
int   i;
int&& f1();

auto* p1 = new auto(5);               // OK, equivalent to (ù{\codeincomments{new int(5)}}ù)
auto* p2 = new decltype(auto)(5);     // OK, equivalent to (ù{\codeincomments{new int(5)}}ù)

auto* p3 = new auto(i);               // OK, equivalent to (ù{\codeincomments{new int(i)}}ù)
auto* p4 = new decltype(auto)(i);     // OK, equivalent to (ù{\codeincomments{new int(i)}}ù)

auto* p5 = new auto(f1());            // OK, equivalent to (ù{\codeincomments{new int(f1())}}ù)
auto* p6 = new decltype(auto)(f1());  // Error, equivalent to (ù{\codeincomments{new int\&\&(f1())}}ù)

auto* p7 = new auto((i));             // OK, equivalent to (ù{\codeincomments{new int(i)}}ù)
auto* p8 = new decltype(auto)((i));   // Error, equivalent to (ù{\codeincomments{new int\&(i)}}ù)
\end{emcppslisting}
    

In all of the examples above, the variable type is declared as
\lstinline!auto*! so that it could be deduced from the return type of the
\lstinline!new! expression; plain \lstinline!auto!, \lstinline!decltype(auto)!,
or \lstinline!int*! would all have been equivalent. The initializers for
\lstinline!p6! and \lstinline!p8! fail to compile because
\lstinline!decltype(auto)! deduces to a reference type in each of those
cases, causing the \lstinline!new! expression to generate a
pointer-to-reference type. The \lstinline!auto! specifiers used to
initialize \lstinline!p5! and \lstinline!p7!, conversely, discard the
reference qualifiers, yielding valid types.

\subsection[Use Cases]{Use Cases}\label{use-cases}

\subsubsection[Exact capture of an expression’s type and value category]{Exact capture of an expression’s type and value category}\label{exact-capture-of-an-expression’s-type-and-value-category}

Both \lstinline!auto&&! and \lstinline!decltype(auto)! can be used to
declare a variable initialized to the result of any expression, but only
\lstinline!decltype(auto)! will capture the exact \emcppsgloss{value category}
of initializing expression:

\begin{emcppslisting}[emcppsstandards={c++14}]
int   f1();
int&  f2();
int&& f3();
int   i;

auto&&         v1 = f1();  // type (ù{\codeincomments{int\&\&}}ù)
decltype(auto) v2 = f1();  // type (ù{\codeincomments{int}}ù)

auto&&         v3 = f2();  // type (ù{\codeincomments{int\&}}ù)
decltype(auto) v4 = f2();  // type (ù{\codeincomments{int\&}}ù)

auto&&         v5 = f3();  // type (ù{\codeincomments{int\&\&}}ù)
decltype(auto) v6 = f3();  // type (ù{\codeincomments{int\&\&}}ù)

auto&&         v7 = i;     // type (ù{\codeincomments{int\&}}ù)
decltype(auto) v8 = i;     // type (ù{\codeincomments{int}}ù)
\end{emcppslisting}
    

Variables \lstinline!v1! and \lstinline!v5! have the same \emcppsgloss{value
category} even though \lstinline!f1()! is a \romeovalue{prvalue} and
\lstinline!f3()! is an \romeovalue{xvalue}, illustrating the limitation of
\lstinline!auto&&!, whereas \lstinline!v2! and \lstinline!v6! correctly capture
the distinction. In addition, \lstinline!auto&&! deduces \lstinline!v7! as a
reference whereas \lstinline!decltype(auto)! correctly deduces it as an
object.

\subsubsection[Return type of a proxy iterator or moving iterator]{Return type of a proxy iterator or moving iterator}\label{return-type-of-a-proxy-iterator-or-moving-iterator}

When an iterator is dereferenced, it usually returns an \romeovalue{lvalue}
reference to an element within a sequence. It might, however, return an
\romeovalue{rvalue} object of proxy type, as in the case of
\lstinline!std::vector<bool>!. Alternatively, the iterator dereference
operator might return an \romeovalue{rvalue} reference to a sequence element,
as in the case of a \emph{moving iterator} --- i.e., for a sequence that
will not be used again after it has been traversed.
\lstinline!decltype(auto)! can be used in generic code to faithfully
capture a dereferenced iterator when the \emcppsgloss{value category} of the
iterator's \lstinline!reference! type is unknown:

\begin{emcppslisting}[emcppsstandards={c++14}]
#include <vector>  // (ù{\codeincomments{std::vector}}ù)

template <typename C, typename V>
void fill(C& container, const V& val)
    // Replace the value of every element in (ù{\codeincomments{container}}ù) with a copy of (ù{\codeincomments{val}}ù).
{
    for (typename C::iterator iter = container.begin();
         iter != container.end();
         ++iter)
    {
        // auto& element = *iter;  // won't work for proxy or moving iterators
        decltype(auto) element = *iter;
        element = val;
    }
}

void f1(std::vector<bool>& v)
{
    fill(v, false);
    // ...
}
\end{emcppslisting}
    

Instead of \lstinline!decltype(auto)!, we could have used \lstinline!auto&&!
and gotten the same effect, although the semantics of
\lstinline!decltype(auto)! are slightly simpler to understand in this case.
Although more verbose, it would be more descriptive to declare
\lstinline!element! as having type
\lstinline!typename!~\lstinline!std::iterator_traits<decltype(iter)>::reference!.

\subsection[Potential Pitfalls]{Potential Pitfalls}\label{potential-pitfalls}

\subsubsection[Hidden dangling references]{Hidden dangling references}\label{hidden-dangling-references}

An operation on an \romeovalue{rvalue} will sometimes return an \romeovalue{rvalue}
reference that is valid only for the lifetime of the original
\romeovalue{rvalue}. When the returned reference is saved in a named variable,
there is a danger of the variable binding to a temporary object that
will go out of scope before it can be used:

\begin{emcppslisting}[emcppsstandards={c++14}]
#include <list>  // (ù{\codeincomments{std::list}}ù)

template <typename T>
T&& first(std::list<T>&& s) { return std::move(*s.begin()); }
    // Return an (ù{\emphincomments{rvalue}}ù) reference to the first element in (ù{\codeincomments{s}}ù).

std::list<int> collection();
    // Return (by value) a (ù{\codeincomments{list}}ù) of (ù{\codeincomments{int}}ù) values.

void f()
{
    int&&          r1 = first(collection());
    auto&&         r2 = first(collection());
    decltype(auto) r3 = first(collection());

    // Bug, (ù{\codeincomments{r1}}ù), (ù{\codeincomments{r2}}ù), and (ù{\codeincomments{r3}}ù) are all dangling references to destroyed
    // objects.

    // ...
}
\end{emcppslisting}
    

The variables \lstinline!r1!, \lstinline!r2!, and \lstinline!r3! all have type
\lstinline!int&&! and all are \emcppsgloss[dangling reference]{dangling references} because they
refer to an element of a \lstinline!list! that goes out of scope
immediately after the reference is initialized. When a reference
variable is initialized from a reference expression, the programmer
should be wary of the lifetime of the object being referenced. In this
regard, \lstinline!decltype(auto)! adds no new hazard. Note, however, that
\lstinline!r1! and \lstinline!r2! are both \emph{declared} as reference types,
whereas \lstinline!r3! is only \emph{deduced} to be a reference type. The
fact that the reference is \emph{hidden} makes this pitfall harder to
avoid for \lstinline!decltype(auto)! than for the other two cases.

\subsubsection[Poor signaling of intent]{Poor signaling of intent}\label{poor-signaling-of-intent}

Since C++11, a common set of idioms has emerged for the use of
\lstinline!auto! in generic code:

\begin{emcppshiddenlisting}[emcppsbatch=e5]
int expr1;
int expr2;
int expr3;
\end{emcppshiddenlisting}
\begin{emcppslisting}[emcppsbatch=e5]
auto        copyVar     = expr1;  // copy of (ù{\codeincomments{expr1}}ù)
const auto& readonlyVar = expr2;  // read-only reference to (ù{\codeincomments{expr1}}ù)
auto&&      mutableVar  = expr3;  // possibly mutable reference to (ù{\codeincomments{expr1}}ù)
\end{emcppslisting}
    

The \lstinline!copyVar! object will be initialized from \lstinline!expr1!
using direct-initialization; if \lstinline!expr1! yields an \romeovalue{rvalue}
or \romeovalue{lvalue} reference, then the copy or move constructor,
respectively, is invoked. The \lstinline!readonlyVar! reference provides
read-only access to the object produced by \lstinline!expr2!; if
\lstinline!expr2! returns an \romeovalue{rvalue}, then \emcppsgloss{lifetime
extension} ensures that it remains valid until \lstinline!readonlyVar! goes
out of scope. Finally, \lstinline!mutableVar! allows modifying or moving
from \lstinline!expr3! (unless \lstinline!expr3! is \lstinline!const!); as in the
case of \lstinline!const!~\lstinline!auto&!, \emcppsgloss{lifetime extension}
might come into play. Faithful use of these idioms provides safety and
signals the programmer's intent regarding the expected use of the
variable.

There is no way to create a similar set of idioms for
\lstinline!decltype(auto)! because \lstinline!decltype(auto)! cannot be
combined with \lstinline!const! or reference type specifiers; see\linebreak%%%%
\intraref{annoyances-decltypeauto}{decltype(auto)-stands-alone}. For variable
declarations, therefore, using \lstinline!auto! in this idiomatic way might
be preferable.

The situation is somewhat different for \emph{function return} types,
where the expected use of the return value is not always known at the
point of declaration; see \featureref{\locationf}{autoreturn}.

\subsection[Annoyances]{Annoyances}\label{annoyances-decltypeauto}

\subsubsection[\lstinline!decltype(auto)! stands alone]{{\SubsubsecCode decltype(auto)} stands alone}\label{decltype(auto)-stands-alone}

When defining a variable with \lstinline!auto!, we can add cv-qualifiers
and reference type specifiers to the deduced type, even if the
initializer expression has simpler qualifiers:

\begin{emcppslisting}
int f();

const auto& v1 = f();  // (ù{\codeincomments{v1}}ù) is (ù{\codeincomments{const int\&}}ù).
\end{emcppslisting}
    

Unfortunately, \lstinline!decltype(auto)! must stand alone; the type of the
variable is always exactly that of the initializer expression, with no
extra decorations:

\begin{emcppslisting}
const decltype(auto) v2 = f();  // Error, (ù{\codeincomments{const}}ù) with (ù{\codeincomments{decltype(auto)}}ù)
\end{emcppslisting}
    

Thus, deducing a variable type that is always read-only, for example, is
not possible with \lstinline!decltype(auto)!.

\subsection[See Also]{See Also}\label{see-also}

\begin{itemize}
\item{\seealsoref{auto}{\seealsolocationa}describes the C++11 feature on which \lstinline!decltype(auto)! is based.}
\item{Appendix~\ref{appendix-value-categories} describes complex world of \emcppsgloss[value category]{value categories} that distinguish\linebreak[4] \lstinline!decltype(auto)! from \lstinline!auto!.}
\item{\seealsoref{autoreturn}{\seealsolocationf}uses the deduction rules described in this section and applies them to function return types instead of variables declarations.}
\item{\seealsoref{decltype}{\seealsolocationa}describes the \lstinline!decltype! operator, which determines the semantics of \lstinline!decltype(auto)!.}
\end{itemize}

\subsection[Further Reading]{Further Reading}\label{further-reading}

None.


