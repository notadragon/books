\emcppsFeature{
    short={Trailing Return},
    long={Trailing Function Return Types},
}{trailing-function-return-types}
%%%% 15 Jan: copyedits in and proofed
%% 27 Jan, code checked and replaced as needed by Josh

\emcppsFeature{
    short={Trailing Return},
    long={Trailing Function Return Types},
}{trailing-function-return-types}
%\section[Trailing Return]{Trailing Function Return Types}\label{trailing-function-return-types}
\setcounter{table}{0}
\setcounter{footnote}{0}
\setcounter{lstlisting}{0}

Trailing return types provide a new alternate syntax in which the return type of a function is specified at
the end of a function declaration as opposed to at the beginning,
thereby allowing it to reference function parameters by name and to reference class
or namespace members without explicit qualification.

\subsection[Description]{Description}\label{description}

C++11 offers an alternative function-declaration syntax in which the
return type of a function is located to the right of its
\romeogloss{signature} (name, parameters, and qualifiers), offset by the
arrow token (\lstinline!->!); the function itself is introduced by the
keyword \lstinline!auto!, which acts as a type placeholder:
%\cprotect\footnote{Note that, when using the alternative trailing return syntax for a function (e.g., one returning a \texttt{double}), the override keyword would be inserted after call qualifiers and before the arrow:
%
%\begin{emcppslisting}[basicstyle=\ttfamily\footnotesize]
%virtual f(int value) const override -> double;
%\end{emcppslisting}\vspace*{-1ex}}:

\begin{emcppslisting}
auto f() -> void;  // equivalent to (ù{\codeincomments{void f();}}ù)
\end{emcppslisting}

When using the alternative, trailing-\lstinline!return!-type syntax, any \lstinline!const!,
\lstinline!volatile!, and reference qualifiers (see
%``\titleref{refqualifiers}" on page~\pageref{refqualifiers}
\featureref{\locatione}{refqualifiers}) are placed to the left of
the \lstinline!->!~\lstinline!*<return-type>*!, and any contextual keywords, such as
\lstinline!override! and \lstinline!final! (see
%``\titleref{override}" on page~\pageref{override}
\featureref{\locationa}{override}
and
%``\titleref{final}" on page~\pageref{final}
\featureref{\locatione}{final}), are placed to its right:

\begin{emcppslisting}
struct Base
{
    virtual int e() const;     // (ù{\codeincomments{const}}ù) qualifier
    virtual int f() volatile;  // (ù{\codeincomments{volitile}}ù) qualfier
    virtual int g() &;         // *lvalue*-reference qualifier
    virtual int h() &&;        // *rvalue*-reference qualifier
};

struct Derived : Base
{
    auto e() const    -> int override;  //  (ù{\codeincomments{override}}ù) contextual keyword
    auto f() volatile -> int final;     //  (ù{\codeincomments{final}}ù)         "        "
    auto g() &        -> int override;  //  (ù{\codeincomments{override}}ù)      "        "
    auto h() &&       -> int final;     //  (ù{\codeincomments{final}}ù)         "        "
};
\end{emcppslisting}

\noindent Using a trailing return type allows the parameters of a function to be
named as part of the specification of the return type, which can be
useful in conjunction with \lstinline!decltype!:

\begin{emcppslisting}
auto g(int x) -> decltype(x);  // equivalent to (ù{\codeincomments{int g(int x);}}ù)
\end{emcppslisting}

\noindent When using the trailing-return-type syntax in a member function
definition outside the class definition, names appearing in the return
type, unlike with the classic notation, will be looked up in class scope
by default:

\begin{emcppslisting}
struct S
{
    typedef int T;
    auto h1() -> T;  // trailing syntax for member function
    T h2();          // classical syntax for member function
};

auto S::h1() -> T { /*...*/ }  // equivalent to (ù{\codeincomments{S::T S::h1() \{ /\*...\*/ \}}}ù)
T    S::h2()      { /*...*/ }  // Error, (ù{\codeincomments{T}}ù) is unknown in this context.
\end{emcppslisting}

\noindent The same advantage would apply to a nonmember
function{\cprotect\footnote{A \lstinline!static! member function of a
\lstinline!struct! can be a viable alternative implementation to a free
function declared within a namespace; see \cite{lakos20}, section~1.4,
pp.~190--201, especially Figure~1-37c (p.~199), and section~2.4.9, pp.~312--321, especially Figure~2-23 (p.~316).}} defined outside of the namespace in
which it is declared:

\begin{emcppslisting}
namespace N
{
    typedef int T;
    auto h3() -> T;  // trailing syntax for free function
    T h4();          // classical syntax for free function
};

auto N::h3() -> T { /*...*/ }  // equivalent to (ù{\codeincomments{N::T N::h3() \{ /\*...\*/ \}}}ù)
T    N::h4()      { /*...*/ }  // Error, (ù{\codeincomments{T}}ù) is unknown in this context.
\end{emcppslisting}


Finally, since the syntactic element to be provided after the arrow
token is a separate type unto itself, return types involving pointers to
functions are somewhat simplified. Suppose, for example, we want to
describe a \romeogloss{higher-order function}, \lstinline!f!, that takes as its
argument a \lstinline!long!~\lstinline!long! and returns a pointer to a
function that takes an \lstinline!int! and returns a
\lstinline!double!{\cprotect\footnote{Co-author John Lakos first used the shown verbose declaration notation
while teaching Advanced Design and Programming using C++ at Columbia
  University (1991--1997).}}:

\begin{emcppslisting}
// [function(long long) returning]
//     [pointer to] [function(int x) returning] double   f;
//     [pointer to] [function(int x) returning] double   f(long long);
//                  [function(int x) returning] double  *f(long long);
//                                              double (*f(long long))(int x);
\end{emcppslisting}

\noindent Using the alternate trailing syntax, we can conveniently break the
declaration of \lstinline!f! into two parts: (1) the declaration of the
function's signature, \lstinline!auto!~\lstinline!f(long!~\lstinline!long)!, and (2) that of the return type, say, \lstinline!R! for now:

\begin{emcppslisting}
// [pointer to] [function (int) returning] double   R;
//              [function (int) returning] double  *R;
//                                         double (*R)(int);
\end{emcppslisting}

\noindent The two equivalent forms of the same declaration are shown below:

\begin{emcppslisting}
double (*f(long long))(int x);         // classic return-type syntax
auto f(long long) -> double (*)(int);  // trailing return-type syntax
\end{emcppslisting}

\noindent Note that both syntactic forms of the same declaration may appear
together within the same scope. Note also that not all functions that
can be represented in terms of the trailing syntax have a convenient
equivalent representation in the classic one:

\begin{emcppslisting}
#include <utility>  // (ù{\codeincomments{declval}}ù)

template <typename A, typename B>
auto foo(A a, B b) -> decltype(a.foo(b));
    // trailing return-type syntax

template <typename A, typename B>
decltype(std::declval<A&>().foo(std::declval<B&>())) foo(A a, B b);
    // classic return-type syntax (using C++11's (ù{\codeincomments{std::declval}}ù))
\end{emcppslisting}

\noindent In the example above, we were essentially forced to use the C++11
standard library template \lstinline!std::declval!\footnote{\cite{cpprefdeclval}} to express our intent with the classic
return-type syntax.

\subsection[Use Cases]{Use Cases}\label{use-cases}

\subsubsection[Function template whose return type depends on a parameter type]{Function template whose return type depends on a parameter type}\label{function-template-whose-return-type-depends-on-a-parameter-type}

Declaring a function template whose return type depends on the types of
one or more of its parameters is not uncommon in generic programming.
For example, consider a mathematical function that linearly interpolates
between two values of possibly different types:

\begin{emcppslisting}
template <typename A, typename B, typename F>
auto linearInterpolation(const A& a, const B& b, const F& factor)
    -> decltype(a + factor * (b - a))
{
    return a + factor * (b - a);
}
\end{emcppslisting}

\noindent The return type of \lstinline!linearInterpolation! is the type of
expression inside the \emph{\lstinline!decltype! specifier}, which is
identical to the expression returned in the body of the function. Hence,
this interface necessarily supports any set of input types for which
\lstinline!a!~\lstinline!+!~\lstinline!factor!~\lstinline!*!~\lstinline!(b!~\lstinline!-!~\lstinline!a)!
is valid, including types such as mathematical vectors, matrices, or
expression templates. As an added benefit, the presence of the
expression in the function's declaration enables \romeogloss{expression
SFINAE}, which is typically desirable for generic template functions
(see %Section~\ref{decltype}, ``\titleref{decltype}" on page~\pageref{decltype}
\featureref{\locationa}{decltype}).

\subsubsection[Avoiding having to qualify names redundantly in return types]{Avoiding having to qualify names redundantly in return types}\label{avoiding-having-to-qualify-names-redundantly-in-return-types}

When defining a function outside the \lstinline!class!, \lstinline!struct!, or
\lstinline!namespace! in which it is first declared, any unqualified names
present in the return type might be looked up differently depending on
the particular choice of function-declaration syntax used. When the
return type precedes the qualified name of the function definition as
is the case with classic syntax, all references to types declared in
the same scope where the function itself is declared must also be
qualified. By contrast, when the return type follows the
qualified name of the function,
the return type is looked up in the same scope in which the
function was first declared, just like its parameter types would. Avoiding redundant qualification of the return type  can be beneficial, especially when the qualifying name is long.

As an illustration, consider a class representing an abstract syntax
tree node that exposes a type alias:

\begin{emcppslisting}[emcppsbatch=e1]
struct NumericalASTNode
{
    using ElementType = double;
    auto getElement() -> ElementType;
};
\end{emcppslisting}

\noindent Defining the \lstinline!getElement! member function using traditional
function-declaration syntax would require repetition of the
\lstinline!NumericalASTNode! name:

\begin{emcppslisting}[emcppsbatch=e1]
NumericalASTNode::ElementType NumericalASTNode::getElement() { /*...*/ }
\end{emcppslisting}

\noindent Using the trailing-return-type syntax handily avoids the repetition:

\begin{emcppshiddenlisting}[emcppsbatch=e2]
// duplicated as hidden to show alternate implementation below
struct NumericalASTNode
{
    using ElementType = double;
    auto getElement() -> ElementType;
};
\end{emcppshiddenlisting}
\begin{emcppslisting}[emcppsbatch=e2]
auto NumericalASTNode::getElement() -> ElementType { /*...*/ }
\end{emcppslisting}

\noindent By ensuring that name lookup within the return type is the same as for
the parameter types, we avoid needlessly having to qualify names that
should be found correctly by default.

\subsubsection[Improving readability of declarations involving function pointers]{Improving readability of declarations involving function pointers}\label{improving-readability-of-declarations-involving-function-pointers}

Declarations of functions returning a pointer to either a function, a member function, or a data member are notoriously hard to parse --- even for seasoned programmers. As an example, consider a function called
\lstinline!getOperation! that takes a \lstinline!kind! of
enumerated \lstinline!Operation! as its argument and returns a pointer to a member
function of \lstinline!Calculator! that takes a \lstinline!double! and
returns a \lstinline!double!:

\begin{emcppshiddenlisting}[emcppsbatch=e3]
struct Calculator;
struct Operation;
\end{emcppshiddenlisting}
\begin{emcppslisting}[emcppsbatch=e3]
double (Calculator::*getOperation(Operation kind))(double);
\end{emcppslisting}

\noindent As we saw in the description, such declarations can be constructed
systematically but do not exactly roll off the fingers. On the other
hand, by partitioning the problem into (1) the declaration of the
function itself and (2) the type it returns, each individual problem
becomes far simpler than the original:

\begin{emcppslisting}[emcppsbatch=e3]
auto getOperation(Operation kind)  // (1) function taking a kind of (ù{\codeincomments{Operation}}ù)
    -> double (Calculator::*)(double);
        // (2) returning a pointer to a (ù{\codeincomments{Calculator}}ù) member function taking a
        //     (ù{\codeincomments{double}}ù) and returning a (ù{\codeincomments{double}}ù)
\end{emcppslisting}

\noindent Using this divide-and-conquer approach, writing such functions becomes fairly straightforward. Declaring a \romeogloss{higher-order function} that
takes a function pointer as an argument might be even easier to read
if a type alias is used via \lstinline!typedef! or, as of C++11,
  \lstinline!using!.

\subsection[Potential Pitfalls]{Potential Pitfalls}\label{potential-pitfalls}

\hspace{\fill}

\subsection[Annoyances]{Annoyances}\label{annoyances}

\hspace{\fill}

\subsection[See Also]{See Also}\label{see-also}

\begin{itemize}
\item{%Section~\ref{decltype}, ``\titleref{decltype}" on page~\pageref{decltype} —
\seealsoref{decltype}{\locationa}Function declarations may use \lstinline!decltype! either in conjunction with, or as an alternative to, trailing return types.}
\item{%Section~\ref{Function-Return-Type-Deduction}, ``\titleref{Function-Return-Type-Deduction}" on page~\pageref{Function-Return-Type-Deduction} —
\seealsoref{Function-Return-Type-Deduction}{\locationf}Leaving the return type to deduction shares syntactical similarities with trailing return types but brings with it significant pitfalls when migrating from C++11 to C++14.}
\end{itemize}

\subsection[Further Reading]{Further Reading}\label{further-reading}

\begin{itemize}
\item{\cite{mertz18}}
\end{itemize}


