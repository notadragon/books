% 19 Feb 2021, ready for Josh's code fixes



\emcppsFeature{
    short={\lstinline!carries_dependency!},
    tocshort={\TOCCode carries\_dependency},
    long={The {\SecCode [[carries\_dependency]]} Attribute},
    toclong={The \lstinline![[carries\_dependency]]! Attribute},
    rhshort={\RHCode carries\_dependency},
}{carriesdependency}
\label{the-carries_dependency-attribute}
\setcounter{table}{0}
\setcounter{footnote}{0}
\setcounter{lstlisting}{0}
 %\section[{\tt carries\_dependency}]{The {\SecCode [[carries\_dependency]]} Attribute}\label{carriesdependency}
%\subsection[The \lstinline![[carries_dependency]]! Attribute]{The {\SecCode [[carries\_dependency]]} Attribute}\label{the-[[carries_dependency]]-attribute}

The\lstinline![[carries_dependency]]! attribute provides a means to
manually identify function parameters and \lstinline!return! values as
components of \emcppsgloss[data dependency chain]{data dependency chains} to enable (including
across translation units) use of the lighter-weight
\emcppsgloss[release consume]{release-consume} \emcppsgloss{synchronization paradigm} as an
optimization over the more conservative \emcppsgloss[release acquire]{release-acquire}
paradigm.{\cprotect\footnote{The authors would like to thank Michael
Wong, Paul McKenney, and Maged Michael for reviewing and contributing to this feature section.}}

\subsection[Description]{Description}\label{description}

C++11 ushered in support for multithreading by introducing a rigorously
specified memory model. The Standard Library provides support for
managing threads, including their execution, synchronization, and
intercommunication. As a part of the new memory model, the Standard
defines various \emcppsgloss[synchronization operation]{synchronization operations}, which can be \emph{sequentially
consistent}, \emph{release}, \emph{acquire}, 
\emph{release-and-acquire}, or \emph{consume} operations. These
operations play a key role in making changes in data in one thread
visible in another.

The modern C++ memory model describes two \emcppsgloss[synchronization
paradigm]{synchronization paradigms}{\cprotect\footnote{The current suite of supported
\emcppsgloss[synchronization paradigm]{synchronization paradigms} comprise \emcppsgloss[release acquire]{release-acquire} and \emcppsgloss[release consume]{release-consume}, although in practice
\emcppsgloss[release consume]{release-consume} is implemented in terms of \emcppsgloss[release acquire]{release-acquire} in all known implementations.}} that are used
to coordinate data flow across concurrent threads of execution. In
particular, the \emcppsgloss[release consume]{release-consume} paradigm requires that the
compiler is given fine-grained understanding of the \emcppsgloss[intra thread dependency]{intra-thread dependencies} among the \emph{reads} and \emph{writes} within a program
and relates those to atomic \emph{release stores} and \emph{consume
loads} that happen concurrently across multiple threads of execution.
Dependency chains in the \emcppsgloss[release consume]{release-consume} synchronization
paradigm specify which evaluations following the \emph{consume load} are
\emcppsgloss{ordered after} a corresponding \emph{release store}.

\subsubsection[The release-acquire paradigm]{The release-acquire paradigm}\label{the-release-acquire-paradigm}

A \emph{release} operation writes a value to a memory location, and an
\emph{acquire} operation reads a value from a memory location. In a \emcppsgloss[release acquire]{release-acquire}{\cprotect\footnote{Although many have
referred to it as \emph{acquire-release}, the proper, standard,
  \emph{time-ordered} nomenclature is \emph{release-acquire}.}} pair, the acquire operation reads the value written by
the release operation, which means that all of the reads and writes to
\emph{any} memory location \emph{before the release operation} happen
before \emph{all} of the reads and writes \emph{after the acquire operation}. Note that this paradigm does \emph{not} use 
dependency chains or the \lstinline![[carries_dependency]]! attribute. See \intraref{use-cases-carriesdependency}{producer-consumer-programming-pattern} 
%{\intraref{{Use Cases: A producer-consumer implementation}}} below 
for a
complete example that implements this paradigm.

\subsubsection[Data dependency]{Data dependency}\label{data-dependency}

In the current revisions of C++, \emcppsgloss{data dependency} is defined as
existing whenever the output of one evaluation is used as the input of
another. When one evaluation has a data dependency on another
evaluation, the second evaluation is said to \emcppsgloss{carry dependency}
to the other.{\cprotect\footnote{Note that the Standard Library function
\lstinline!std::kill_dependency! is also related and can be used to
  \emph{break} a data dependency chain.}} Naturally the compiler must
ensure that any evaluation that depends on another must not be started
until the first evaluation is complete. A \emcppsgloss{data dependency chain}
is formed when multiple evaluations carry dependency transitively; the
output of one evaluation is used as the input of the next evaluation in
the chain.

\subsubsection[The release-consume paradigm]{The release-consume paradigm}\label{the-release-consume-paradigm}

Some systems use the read-copy-update (RCU) synchronization mechanism.
This approach preserves the order of \emph{loads} and \emph{stores} that
form in a \emcppsgloss{data dependency chain}, which is a sequence of
\emph{loads} and \emph{stores} in which the input to one operation is an
output of another. A compiler can use guaranteed order of loads
and stores provided by the RCU synchronization mechanism for performance purposes by omitting certain
\emcppsgloss[memory fence instruction]{memory-fence instructions} that would otherwise be required to
enforce the correct ordering. In such cases, however, ordering is
guaranteed only between those operations making up the relevant
\emcppsgloss{data dependency chain}. {\cprotect\footnote{The C++ definition
of data dependency is intended to mimic the data dependency on RCU
systems. However, note that C++ currently defines data dependency in
terms of evaluations, while RCU data dependency is defined in terms of
  loads and stores.}}

This optimization was intended to be available in C++ through use of a
\emph{release-consume} pair, which, as its name
suggests, consists of a \emph{release-store} operation and a
\emph{consume-load} operation. A \emph{consume} operation is much like
an \emph{acquire} operation, except that it guarantees only the ordering
of those evaluations in a \emcppsgloss{data dependency chain}, starting with
the consume-load operation.

Note, however, that currently no known implementation is able to take
advantage of the current C++ \emph{consume} semantics; hence, all
current compilers promote \emph{consume} loads to \emph{acquire} loads,
effectively making the \lstinline![[carries_dependency]]! attribute
redundant. Revisions to render this feature implementable and therefore
usable are currently under consideration by the C++ Standards Committee.
Prototypes for various approaches have been produced. When a usable
feature with real implementations is delivered, it quite possibly will
not work exactly as described in the examples here; see \intrarefsimple{use-cases-carriesdependency}. 
% {\intraref{{Use Cases}}}, below.

\subsubsection[Using the {\protect\lstinline![[carries_dependency]]!} attribute]{Using the {\SubsubsecCode [[carries\_dependency]]} attribute}\label{using-the-[[carries_dependency]]-attribute}

\emcppsgloss[data dependency chain]{Data dependency chains} can and do propagate into and out of
called functions. If one of these interoperating functions is in a
separate translation unit, the compiler will have no way of seeing the
dependency chain. In such cases, the user can apply the
\lstinline![[carries_dependency]]! attribute to imbue the necessary
information for the compiler to track the propagation of dependency
chains into and out of functions across translation units, thus possibly
avoiding unnecessary memory-fence instructions; see \intrarefsimple{use-cases-carriesdependency}.{\cprotect\footnote{Note that the Standard Library function
\lstinline!std::kill_dependency! is also related and can be used to
  \emph{break} a data dependency chain.}} 
% {\intrarefsub{{Using the \lstinline![[carries_dependency]]! attribute}}}, below.

The \lstinline![[carries_dependency]]! attribute can be applied to a
function declaration as a whole by placing it in front of the function,
in which case the attribute applies to the \lstinline!return! value:

\begin{emcppslisting}
[[carries_dependency]] int* f();  // attribute applied to entire function (ù{\codeincomments{f}}ù)
\end{emcppslisting}
    
\noindent In the example above, the \lstinline![[carries_dependency]]! attribute was
applied to the declaration of function \lstinline!f! to indicate that the
\lstinline!return! value carries a dependency out of the function. The
compiler may now be able to avoid emitting a memory-fence instruction
for the return value of \lstinline!f!.

This same \lstinline![[carries_dependency]]! attribute can also be applied
to one or more of the function's parameter declarations by placing it
immediately after the parameter name:

\begin{emcppslisting}
void g(int* input [[carries_dependency]]); // attribute applied to (ù{\codeincomments{input}}ù)
\end{emcppslisting}
    
\noindent In the declaration of function \lstinline!g! in the example above, the
\lstinline![[carries_dependency]]! attribute is applied to the
\lstinline!input! parameter to indicate that a dependency is carried
through that parameter into the function, which may obviate the
compiler's having to emit an unnecessary memory-fence instruction for
the \lstinline!input! parameter; see \featureref{\locationa}{attributes}.
%{\intraref{{Attribute Syntax}}}.

In both cases, if a function or a parameter declaration specifies the
\lstinline![[carries_dependency]]! attribute, the first declaration of
that function shall specify that \lstinline![[carries_dependency]]!
attribute. Similarly, if the first declaration of a function or one of
its parameters specifies the \lstinline![[carries_dependency]]! attribute
in one translation unit and the first declaration of the same function
in another translation unit doesn't, the program is IFNDR.

The user is responsible for ensuring that an existing dependency chain
is available if needed for synchronization purposes. The
\lstinline![[carries_dependency]]! attribute will not create a \mbox{dependency}.

\subsection[Use Cases]{Use Cases}\label{use-cases-carriesdependency}

\subsubsection[Producer-consumer programming pattern]{Producer-consumer programming pattern}\label{producer-consumer-programming-pattern}

The popular producer-consumer programming pattern uses
\emph{release-acquire} pairs to synchronize between threads:

\begin{emcppslisting}
#include <cassert>  // standard C (ù{\codeincomments{assert}}ù) macro
#include <atomic>   // (ù{\codeincomments{std::atomic}}ù), (ù{\codeincomments{std::memory\_order\_release}}ù)
                    // (ù{\codeincomments{std::memory\_order\_acquire}}ù)

struct S
{
    int i;
    char c;
    double d;
};

S data;
std::atomic<int> guard;

void producerThread()
{
    data.i = 42;
    data.c = 'c';
    data.d = 5.0;
    guard.store(1, std::memory_order_release);
}

void consumerThread()
{
    if (guard.load(std::memory_order_acquire) == 1)
    {
        // By dint of the release-acquire guarantee, we know that all the
        // data changes are visible if the guard change is visible.
        assert(data.i == 42);
        assert(data.c == 'c');
        assert(data.d == 5.0);
    }
}
\end{emcppslisting}
    
\noindent When this \emph{release-acquire} \emph{synchronization paradigm} is
used, the compiler must maintain the ordering of the statements to avoid
breaking the \emph{release-acquire} guarantee; the compiler will also
need to insert memory-fence instructions to prevent the hardware from
breaking this guarantee.

If we wanted to modify the example above to use \emph{release-consume}
semantics, we would somehow need to make the \lstinline!assert! statements
a part of the dependency chain on the \lstinline!load! from the
\lstinline!guard! object. We can accomplish this because reading data
through a pointer establishes a dependency chain between the reading of
that pointer value and the reading of the referenced data

\begin{emcppslisting}[emcppsbatch=e1]
#include <cassert>  // standard C (ù{\codeincomments{assert}}ù) macro
#include <atomic>   // (ù{\codeincomments{std::atomic}}ù), (ù{\codeincomments{std::memory\_order\_release}}ù)
                    // (ù{\codeincomments{std::memory\_order\_consume}}ù)

struct S
{
    int i;
    char c;
    double d;
};

S data;
std::atomic<S*> guard(nullptr);

void producerThread()
{
    data.i = 42;
    data.c = 'c';
    data.d = 5.0;
    guard.store(&data, std::memory_order_release);
}

void consumerThread()
{
    S* setData = guard.load(std::memory_order_consume);
    if (setData)
    {
        assert(setData->i == 42);
        assert(setData->c == 'c');
        assert(setData->d == 5.0);
    }
}
\end{emcppslisting}
    
\noindent Finally, if we want to start to refactor the work of
\lstinline!consumerThread! into multiple\linebreak%%%%
 functions across different
translation units, we would want to carefully apply the\linebreak%%%%
\lstinline![[carries_dependency]]! attribute to the newly refactored
functions, so calling into these functions might conceivably be better
optimized, like this:

\begin{emcppslisting}[emcppsbatch=e1]
[[carries_dependency]] S* loadData()
{
    return guard.load(std::memory_order_consume);
}

void checkData(S* s [[carries_dependency]])
{
    assert(s->i == 42);
    assert(s->c == 'c');
    assert(s->d == 5.0);
}

void betterThreadB()
{
    S* setData = loadData();
    if (setData)
    {
        checkData(setData);
    }
}
\end{emcppslisting}
    
\noindent Again, as of this writing, all known compilers implement \emph{consume}
loads as \emph{acquire} loads and thus fail to provide the desired
optimization.

\subsection[Potential Pitfalls]{Potential Pitfalls}\label{potential-pitfalls}

\subsubsection[No practical use on current platforms]{No practical use on current platforms}\label{no-practical-use-on-current-platforms}

All known compilers promote \emph{consume} loads to \emph{acquire}
loads, thus failing to omit superfluous memory-fence instructions.
Developers writing code with the expectation that it will be run under
the more efficient \emcppsgloss[release consume]{release-consume} \emcppsgloss{synchronization paradigm} will find that their code will continue to work --- as
expected --- under the more conservative \emcppsgloss[release acquire]{release-acquire}
guarantees until such time as a theoretical, not-yet-existent compiler
that properly supports the \emcppsgloss[release consume]{release-consume}
\emcppsgloss{synchronization paradigm} becomes widely available. In the
meantime, applications that require the potential performance benefits
of \emph{consume} semantics typically make careful --- and potentially
very implementation-specific --- use of the \lstinline!volatile! keyword or
handcrafted inline assembly instead.{\cprotect\footnote{Since C++17, the
use of \lstinline!memory_order_consume! has been explicitly
  discouraged after the acceptance of \cite{boehm16}. The specific
  note in the standard now says, ``Prefer
  \lstinline!memory_order::acquire!, which provides stronger guarantees
  than \lstinline!memory_order::consume!'' (\cite{iso17}, section~32.4 ``Order and Consistency," paragraph~1.3, Note~1, p.~1346). Implementations have found it infeasible to provide performance better
  than that of \lstinline!memory_order::acquire!. Specification revisions
  are under consideration by the Standards Committee.}}

\subsection[Annoyances]{Annoyances}\label{annoyances}

\subsubsection[Ill-formed when inconsistently applied]{Ill-formed when inconsistently applied}\label{ill-formed-when-inconsistently-applied}

Like many aspects of a declaration, such as the
\lstinline![[noreturn]]! attribute (see\linebreak%%%%%
\featureref{\locationa}{the-noreturn-attribute}),
\lstinline!alignas! specifier (see \featureref{\locationa}{alignas}), language linkage, and so on,
the \lstinline![[carries_dependency]]! attribute must be applied to a
function's declaration consistently across all translation units.
Failing to apply it on the first declaration in a translation unit and
then later to a (re)declaration is ill-formed. Such uniqueness issues
are readily dispatched when (1) each function's owner supplies a
corresponding header having the canonical declarations for that function
and (2) every client includes that corresponding header rather than
attempting to redefine the function locally.

\subsection[See Also]{See Also}\label{see-also}

\begin{itemize}
\item{%\intraref{Attribute Syntax}
\seealsoref{attributes}{\seealsolocationa}provides an in-depth discussion of how attributes pertain to C++ language entities.}
\item{%\intraref{\texttt{[[noreturn]]} Attribute}
\seealsoref{the-noreturn-attribute}{\seealsolocationa}offers an example of another \emph{attribute} that \emph{is} implemented ubiquitously.}
\end{itemize}

\subsection[Further Reading]{Further Reading}\label{further-reading}

TODO


