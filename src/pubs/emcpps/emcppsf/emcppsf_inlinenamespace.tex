% 14 Feb 2021, revisions in; ready for Josh's code fixes

\emcppsFeature{
    short={\lstinline!inline!~\lstinline!namespace!},
    tocshort={{\TOCCode inline}~{\TOCCode namespace}},
    long={Transparently Nested Namespaces},
    rhshort={{\RHCode inline}~{\RHCode namespace}},
}{inline-namespaces}
\setcounter{table}{0}
\setcounter{footnote}{0}
\setcounter{lstlisting}{0}
%\section[{\tt inline} {\tt namespace}]{Transparently Nested Namespaces}\label{inline-namespaces}%


An \lstinline!inline! namespace is a nested namespace whose member entities
closely behave as if they were declared directly within the enclosing
namespace.

\subsection[Description]{Description}\label{description-inlinenamespace}

To a first approximation, an \romeogloss{\lstinline!inline!~\lstinline!namespace!}
(e.g., \lstinline!v2! in the code snippet below) acts a lot like a
conventional nested namespace (e.g., \lstinline!v1!) followed by a
\lstinline!using! directive for that namespace in its enclosing namespace{\cprotect\footnote{C++17 allows developers to concisely declare nested
namespaces with shorthand notation:
\begin{emcppslisting}[language=C++, style=footcode]
namespace a::b { /* ... */ }
// is the same as
namespace a { namespace b { /* ... */ } }
\end{emcppslisting}
C++20 expands on the above syntax by allowing the insertion of the
\lstinline!inline! keyword in front of any of the namespaces, except the
first one:
\begin{emcppslisting}[language=C++, style=footcode]
namespace a::inline b::inline c { /* ... */ }
// is the same as
namespace a { inline namespace b { inline namespace c { /* ... */ } } }

inline namespace a::b { }  // Error, cannot start with (ù{\codeincomments{inline}}ù) for compound namespace names
namespace inline a::b { }  // Error, (ù{\codeincomments{inline}}ù) at front of sequence explicitly disallowed
\end{emcppslisting}
      }}:

\begin{emcppslisting}
// example.cpp:
namespace n
{
    namespace v1  // conventional nested namespace followed by (ù{\codeincomments{using}}ù) directive
    {
        struct T { };     // nested type declaration (identified as (ù{\codeincomments{::n::v1::T}}ù))
        int d;            // (ù{\codeincomments{::n::v1::d}}ù) at, e.g., (ù{\codeincomments{0x01a64e90}}ù)
    }

    using namespace v1;   // import names (ù{\codeincomments{T}}ù) and (ù{\codeincomments{d}}ù) into (ù{\codeincomments{namespace n}}ù)
}

namespace n
{
    inline namespace v2   // similar to being followed by (ù{\codeincomments{using namespace v2}}ù)
    {
        struct T { };     // nested type declaration (identified as (ù{\codeincomments{::n::v2::T}}ù))
        int d;            // (ù{\codeincomments{::n::v2::d}}ù) at, e.g., (ù{\codeincomments{0x01a64e94}}ù)
    }

    // using namespace v2;  // redundant when used with an (ù{\codeincomments{inline namespace}}ù)
}
\end{emcppslisting}
    
 
%%%%% because the long list of code terms in the FN below cause a persistent end-of-line error, those long lists are being set as a command:
%\newcommand{inlinelonglistone}{\lstinline!namespace!~\lstinline!n!~\lstinline!{!~\lstinline!inline!~\lstinline!namespace~\lstinline!v!~\lstinline!{!~\lstinline!int!~\lstinline!d;!~\lstinline!}!~\lstinline!}!}
%\newcommand{inlinelonglisttwo}{\lstinline!namespace!~\lstinline!n!~\lstinline!{!~\lstinline!namespace}!~\lstinline!v!~\lstinline!{!~\lstinline!int!~\lstinline!d;!~\lstinline!}!~\lstinline!using!~\lstinline!namespace!~\lstinline!v;!~\lstinline!}! }
\noindent Four subtle details distinguish these approaches:
\begin{enumerate}
\item{Name collisions with existing names behave differently due to differing name-lookup rules.}
\item{\romeogloss{Argument-dependent lookup} (ADL) gives special treatment to \lstinline!inline! namespaces.}
\item{Template specializations can refer to the primary template in an \lstinline!inline! namespace even if written in the enclosing namespace.}
\item{Reopening namespaces might reopen an \lstinline!inline! namespace.}
\end{enumerate}

One important aspect that all forms of namespaces share, however, is
that (1) nested symbolic names (e.g., \lstinline!n::v1::T!) at the
\romeogloss{API} level, (2) \romeogloss{mangled names} (e.g.,
\lstinline!_ZN1n2v11dE!, \lstinline!_ZN1n2v21dE!), and (3) assigned
relocatable addresses (e.g., \lstinline!0x01a64e90!, \lstinline!0x01a64e94!)
at the \romeogloss{ABI} level remain unaffected by the use of either
\lstinline!inline! or \lstinline!using! or both.{\cprotect\footnote{Compiling
source files containing, alternately, 
\lstinline!namespace!~\lstinline!n!~\lstinline!{!~\lstinline!inline!~\lstinline!namespace!~\lstinline!v!~\lstinline!{!~\lstinline!int!~\lstinline!d;!~\lstinline!}!~\lstinline!}! 
and\linebreak[4]%%%%%% 
\lstinline!namespace!~\lstinline!n!~\lstinline!{!~\lstinline!namespace!~\lstinline!v!~\lstinline!{!~\lstinline!int!~\lstinline!d;!~\lstinline!}!~\lstinline!using!~\lstinline!namespace!~\lstinline!v;!~\lstinline!}!, will produce identical assembly. This can be seen with GCC by running \lstinline!g++!~\lstinline!-S!~\lstinline!<file>.cpp! and viewing the contents of the generated \lstinline!<file>.s!. Note that Compiler Explorer is another valuable tool for learning about what comes out the other end of a C++ compiler: see https://godbolt.org/.}} Note that a \lstinline!using! directive immediately following an \lstinline!inline!
namespace is superfluous; name lookup will always consider names in
\lstinline!inline! namespaces before those imported by a \lstinline!using!
directive. Such a directive can, however, be used to import the contents
of an \lstinline!inline! namespace to some other namespace, albeit only in
the conventional, \romeogloss{\lstinline!using! directive} sense; see \intraref{annoyances-inlinenamespace}{only-one-namespace-can-contain-any-given-inline-namespace}. 

%%%%%%%%%% code for FN
%\begin{emcppslisting}[style=footcode]
%namespace n { inline namespace v { int d; } }
%\end{emcppslisting}
%
%and 
%
%\begin{emcppslisting}[style=footcode]
%namespace n { namespace} v { int! d; } using namespace v; }
%\end{emcppslisting}
%%%%%%%%%%%% Previous code in FN
%\texttt{namespace}~\texttt{n}~\texttt{\{}~\texttt{inline}~\texttt{namespace}~\texttt{v}~\texttt{\{}~\texttt{int}~\texttt{d;}~\texttt{\}}~\texttt{\}} and\linebreak[4]%%%%%% 
%\texttt{namespace}~\texttt{n}~\texttt{\{}~\texttt{namespace\}}~\texttt{v}~\texttt{\{}~\texttt{int}~\texttt{d;}~\texttt{\}}~\texttt{using}~
%\texttt{namespace}~\texttt{v;}~
%\texttt{\}},
%
%\lstinline!namespace!~\lstinline!n!~\lstinline!{!~\lstinline!inline!~\lstinline!namespace!~\lstinline!v!~\lstinline!{!~\lstinline!int!~\lstinline!d;!~\lstinline!}!~\lstinline!}! 
%and\linebreak[4]%%%%%% 
%\lstinline!namespace!~\lstinline!n!~\lstinline!{!~\lstinline!namespace}!~\lstinline!v!~\lstinline!{!~\lstinline!int!~\lstinline!d;!~\lstinline!}!~\lstinline!using!~\lstinline!namespace!~\lstinline!v;!~\lstinline!}!,
% 
%
%%%%%%%%%%%% Previous xref
%\textit{\titleref{annoyances-inlinenamespace}: \titleref{only-one-namespace-can-contain-any-given-inline-namespace}} on page~\pageref{only-one-namespace-can-contain-any-given-inline-namespace}.

More generally, each namespace has what is called its
\emph{\lstinline!inline! namespace set}, which is the transitive closure of
all \lstinline!inline! namespaces within the namespace. All names in the
\lstinline!inline! namespace set are roughly intended to behave as if they
are defined in the enclosing namespace. Conversely, each \lstinline!inline!
namespace has an \emph{enclosing namespace set} that comprises all
enclosing namespaces up to and including the first non-\lstinline!inline!
namespace.

\subsubsection[Loss of access to duplicate names in enclosing namespace]{Loss of access to duplicate names in enclosing namespace}\label{loss-of-access-to-duplicate-names-in-enclosing-namespace}

When both a type and a variable are declared with the same name in the
same scope, the variable name hides the type name --- such behavior can
be demonstrated by using the form of \lstinline!sizeof! that accepts a
nonparenthesized \emph{expression}{\cprotect\footnote{The form of
\lstinline!sizeof! that accepts a \emph{type} as its argument
  specifically requires parentheses.}}:

\begin{emcppslisting}
struct A { double d; };  static_assert(sizeof(  A) == 8, "");  // type
                      // static_assert(sizeof   A  == 8, "");  // Error

int A;                   static_assert(sizeof(  A) == 4, "");  // data
                         static_assert(sizeof   A  == 4, "");  // OK
\end{emcppslisting}
    
\noindent Unless both type and variable entities are declared within the same
scope, no preference is given to variable names; the name of an entity
in an inner scope hides a like-named entity in an enclosing scope:

\begin{emcppslisting}
void f()
{
    double B;                 static_assert(sizeof(B) == 8, "");  // variable
    {                         static_assert(sizeof(B) == 8, "");  // variable
        struct B { int d; };  static_assert(sizeof(B) == 4, "");  // type
    }                         static_assert(sizeof(B) == 8, "");  // variable
}
\end{emcppslisting}
    
\noindent When an entity is declared in an enclosing \lstinline!namespace! and
another entity having the same name hides it in a \emph{lexically}
nested scope, then (apart from \lstinline!inline! namespaces) access to a
hidden element can generally be recovered by using scope resolution:

\begin{emcppslisting}
struct C { double d; };  static_assert(sizeof(  C) == 8, "");

void g()
{                        static_assert(sizeof(  C) == 8, "");  // type
    int C;               static_assert(sizeof(  C) == 4, "");  // variable
                         static_assert(sizeof(::C) == 8, "");  // type
}                        static_assert(sizeof(  C) == 8, "");  // type
\end{emcppslisting}
    
\noindent A conventional nested namespace behaves as one might expect:

\begin{emcppslisting}
namespace outer
{
    struct D { double d; }; static_assert(sizeof(       D) == 8, ""); // type

    namespace inner
    {                       static_assert(sizeof(       D) == 8, ""); // type
        int D;              static_assert(sizeof(       D) == 4, ""); // var
    }                       static_assert(sizeof(       D) == 8, ""); // type
                            static_assert(sizeof(inner::D) == 4, ""); // var
                            static_assert(sizeof(outer::D) == 8, ""); // type
    using namespace inner;//static_assert(sizeof(       D) == 0, ""); // Error
                            static_assert(sizeof(inner::D) == 4, ""); // var
                            static_assert(sizeof(outer::D) == 8, ""); // type
}                           static_assert(sizeof(outer::D) == 8, ""); // type
\end{emcppslisting}
    
\noindent In the example above, the inner variable name, \lstinline!D!, hides the
outer type with the same name, starting from the point of \lstinline!D!'s
declaration in \lstinline!inner! until \lstinline!inner! is closed, after
which the unqualified name \lstinline!D! reverts back to the type in the
\lstinline!outer! namespace. Then, right after the subsequent
\lstinline!using!~\lstinline!namespace!~\lstinline!inner;! directive, the meaning
of the unqualified name \lstinline!D! in \lstinline!outer! becomes ambiguous,
shown here with a \lstinline!static_assert! that is commented out; any
attempt to refer to an unqualified \lstinline!D! from here to the end of
the scope of \lstinline!outer! will fail to compile. The type entity
declared as \lstinline!D! in the \lstinline!outer! namespace can, however,
still be accessed --- from inside or outside of the \lstinline!outer!
namespace, as shown in the example --- via its qualified name,
\lstinline!outer::D!.

If an \lstinline!inline! namespace were used instead of a nested namespace
followed by a \mbox{\lstinline!using!} directive, however, the ability to recover
by name the hidden entity in the enclosing namespace is lost.
Unqualified name lookup considers the inline namespace set and the used
namespace set simultaneously. Qualified name lookup first considers the
\lstinline!inline! namespace set, and \emph{then} goes on to look into used
namespaces. This means we can still refer to \lstinline!outer::D! in the
example above, but doing so would still be ambiguous if \lstinline!inner!
were an inline namespace. This subtle difference in behavior is a
byproduct of the highly specific use case that motivated this feature
and for which it was explicitly designed; see \intraref{use-cases-inlinenamespace}{link-safe-abi-versioning}. 
%\textit{\titleref{use-cases-inlinenamespace}: \titleref{link-safe-abi-versioning}} on page~\pageref{link-safe-abi-versioning}.

\subsubsection[Argument-dependent–lookup interoperability across \lstinline!inline! \lstinline!namespace! boundaries]{Argument-dependent–lookup interoperability across\\[0.5ex] {\SubsubsecCode inline} {\SubsubsecCode namespace} boundaries}\label{argument-dependent–lookup-interoperability-across-inline-namespace-boundaries}

Another important aspect of \lstinline!inline! namespaces is that they
allow \romeogloss{ADL} to work seamlessly across \lstinline!inline! namespace
boundaries. Whenever unqualified function names are being resolved, a
list of \emph{associated namespaces} is built for each argument of the
function. This list of associated namespaces comprises the namespace of the
argument, its enclosing namespace set, plus the \lstinline!inline!
namespace set.

Consider the case of a type, \lstinline!U!, defined in an \lstinline!outer!
namespace, and a function, \lstinline!f(U)!, declared in an \lstinline!inner!
namespace nested within \lstinline!outer!. A second type, \lstinline!V!, is
defined in the \lstinline!inner! namespace, and a function, \lstinline!g!, is
declared, after the close of \lstinline!inner!, in the \lstinline!outer!
namespace:

\begin{emcppslisting}
namespace outer
{
    struct U { };

    // inline               // Uncommenting this line fixes the problem.
    namespace inner
    {
         void f(U) { }
         struct V { };
    }

    using namespace inner;  // If we inline (ù{\codeincomments{inner}}ù), we don't need this line.

    void g(V) { }
}

void client()
{
    f(outer::U());         // Error, (ù{\codeincomments{f}}ù) is not declared in this scope.
    g(outer::inner::V());  // Error, (ù{\codeincomments{g}}ù) is not declared in this scope.
}
\end{emcppslisting}
    
\noindent In the example above, a \lstinline!client! invoking \lstinline!f! with an
object of type \lstinline!outer::U! fails to compile because
\lstinline!f(outer::U)! is declared in the nested \lstinline!inner! namespace,
which is not the same as declaring it in \lstinline!outer!. Because
\romeogloss{ADL} does not look into namespaces added with the \lstinline!using!
directive, \romeogloss{ADL} does not find the needed
\lstinline!outer::inner::f! function. Similarly, the type \lstinline!V!,
defined in namespace \lstinline!outer::inner!, is not declared in the same
namespace as the function \lstinline!g! that operates on it. Hence, when
\lstinline!g! is invoked from within \lstinline!client! on an object of type
\lstinline!outer::inner::V!, \romeogloss{ADL} again does not find the needed
function \lstinline!outer::g(outer::V)!.

Simply making the \lstinline!inner! namespace \lstinline!inline! solves both
of these \romeogloss{ADL}-related problems. All transitively nested
\lstinline!inline! namespaces --- up to and including the most proximate
non-\lstinline!inline! enclosing namespace --- are treated as one with
respect to \romeogloss{ADL}.

\subsubsection[The ability to specialize templates declared in a nested \lstinline!inline! namespace]{The ability to specialize templates declared in a nested {\SubsubsecCode inline} namespace}\label{the-ability-to-specialize-templates-declared-in-a-nested-inline-namespace}

The third property that distinguishes \lstinline!inline! namespaces from
conventional ones, even when followed by a \lstinline!using! directive, is
the ability to specialize a class template defined within an
\lstinline!inline! namespace from within an enclosing one; this ability
holds transitively up to and including the most proximate
non-\lstinline!inline! namespace:

\begin{emcppslisting}[emcppserrorlines={24}]
namespace out                      // proximate non-(ù{\codeincomments{inline}}ù) outer namespace
{
    inline namespace in1           // first-level nested (ù{\codeincomments{inline}}ù) namespace
    {
        inline namespace in2       // second-level nested (ù{\codeincomments{inline}}ù) namespace
        {
            template <typename T>  // primary class template general definition
            struct S { };

            template <>            // class template *full* specialization
            struct S<char> { };
        }

        template <>                // class template *full* specialization
        struct S<short> { };
    }

    template <>                    // class template *full* specialization
    struct S<int> { };
}

using namespace out;               // conventional using directive

template <>
struct S<int> { };                 // Error, cannot specialize from this scope
\end{emcppslisting}
    
\noindent Note that the conventional nested namespace \lstinline!out! followed by a
\lstinline!using! directive in the enclosing namespace does not admit
specialization from that outermost namespace, whereas all of the
\lstinline!inline! namespaces do. Function templates behave similarly
except that --- unlike class templates, whose definitions must reside
entirely within the namespace in which they are declared --- a function
template can be \emph{declared} within a nested namespace and then be
\emph{defined} from anywhere via a \romeogloss{qualified name}:

\begin{emcppslisting}
namespace out                      // proximate non-(ù{\codeincomments{inline}}ù) outer namespace
{
    inline namespace in1           // first-level nested (ù{\codeincomments{inline}}ù) namespace
    {
        template <typename T>      // function template declaration
        void f();

        template <>                // function template (full) specialization
        void f<short>() { }
    }

    template <>                    // function template (full) specialization
    void f<int>() { }
}

template <typename T>              // function template general definition
void out::in1::f() { }
\end{emcppslisting}
    
\noindent An important takeaway from the examples above is that every template
entity --- be it class or function --- \emph{must} be declared
in \emph{exactly} one place within the collection of namespaces that
comprise the \lstinline!inline! namespace set. In particular, declaring a
class template in a nested \lstinline!inline! namespace and then
subsequently defining it in a containing namespace is not possible
because, unlike a function definition, a type definition cannot be
placed into a namespace via name qualification alone:

\begin{emcppslisting}[emcppserrorlines={17}]
namespace outer
{
    inline namespace inner
    {
        template <typename T>      // class template declaration
        struct Z;                  // (if defined, must be within same namespace)

        template <>                // class template full specialization
        struct Z<float> { };
    }

    template <typename T>          // inconsistent declaration (and definition)
    struct Z { };                  // (ù{\codeincomments{Z}}ù) is now ambiguous in namespace (ù{\codeincomments{outer}}ù).

    const int i = sizeof(Z<int>);  // Error, Reference to (ù{\codeincomments{Z}}ù) is ambiguous.

    template <>                    // attempted class template full specialization
    struct Z<double> { };          // Error, (ù{\codeincomments{outer::Z}}ù) or (ù{\codeincomments{outer::inner::Z}}ù)?
}
\end{emcppslisting}
    

\subsubsection[Reopening namespaces can reopen nested \lstinline!inline! ones]{Reopening namespaces can reopen nested {\SubsubsecCode inline} ones}\label{reopening-namespaces-can-reopen-nested-inline-ones}

Another subtlety specific to \lstinline!inline! namespaces is related to
reopening namespaces. Consider a namespace \lstinline!outer! that declares
a nested namespace \lstinline!outer::m! and an \lstinline!inline!
namespace \lstinline!inner! that, in turn, declares a nested namespace
\lstinline!outer:inner::m!. In this case, subsequent attempts to reopen
namespace \lstinline!m! cause an ambiguity error:

\begin{emcppshiddenlisting}[emcppsbatch=e1]
#include <type_traits>  // (ù{\codeincomments{std::is\_same}}ù)
\end{emcppshiddenlisting}
\begin{emcppslisting}[emcppsbatch=e1,emcppserrorlines={17,18,19}]
namespace outer
{
    namespace m { }      // opens and closes (ù{\codeincomments{::outer::m}}ù)

    inline namespace inner
    {
        namespace n { }  // opens and closes (ù{\codeincomments{::outer::inner::n}}ù)
        namespace m { }  // opens and closes (ù{\codeincomments{::outer::inner::m}}ù)
    }

    namespace n          // OK, reopens (ù{\codeincomments{::outer::inner::n}}ù)
    {
        struct S { };    // defines (ù{\codeincomments{::outer::inner::n::S}}ù)
    }

    namespace m          // Error, (ù{\codeincomments{namespace m}}ù) is ambiguous
    {
        struct T { };    // with clang defines (ù{\codeincomments{::outer::m::T}}ù)
    }
}

static_assert(std::is_same<outer::n::S, outer::inner::n::S>::value, "");
\end{emcppslisting}
    
\noindent In the code snippet above, no issue occurs with reopening
\lstinline!outer::inner::n! and no issue would have occurred with reopening
\lstinline!outer::m! but for the \lstinline!inner! namespaces having been
declared \lstinline!inline!. When a new namespace declaration is
encountered, a lookup determines if a matching namespace having that
name appears anywhere in the \emph{\lstinline!inline! namespace set} of the
current namespace. If the namespace is ambiguous, as is the case with
\lstinline!m! in the example above, one can get the surprising error shown.{\cprotect\footnote{Note that reopening already declared
namespaces, such as \lstinline!m! and \lstinline!n! in the \lstinline!inner!
and \lstinline!outer! example, is handled incorrectly on several
popular platforms. Clang, for example, will perform a name lookup when
encountering a new namespace declaration and give preference to the
outermost namespace found, causing the last declaration of \lstinline!m!
to reopen \lstinline!::outer::m! instead of being ambiguous. GCC, prior
to version~8.1, will not perform name lookup and will place \emph{any}
nested namespace declarations directly within their enclosing
namespace. This compiler defect causes the last declaration of
\lstinline!m! to reopen \lstinline!::outer::m! instead of
\lstinline!::outer::inner::m! and the last declaration of \lstinline!n! to
open a new namespace, \lstinline!::outer::n!, instead of reopening
  \lstinline!::outer::inner::n!.}} If a matching namespace is found
unambiguously inside an \lstinline!inline! namespace, \lstinline!n! in this
case, then it is that nested namespace that is reopened --- here,
\lstinline!::outer::inner::n!. The inner namespace is reopened even though
the last declaration of \lstinline!n! is not lexically scoped within
\lstinline!inner!. Notice that the definition of \lstinline!S! is perhaps
surprisingly defining \lstinline!::outer::inner::n::S!, not
\lstinline!::outer::n::S!. For more on what is \emph{not} supported by this
feature, see \intraref{annoyances-inlinenamespace}{inability-to-redeclare-across-namespaces-impedes-code-factoring}. 
%\textit{\titleref{annoyances-inlinenamespace}: \titleref{inability-to-redeclare-across-namespaces-impedes-code-factoring}} on page~\pageref{inability-to-redeclare-across-namespaces-impedes-code-factoring}.

\subsection[Use Cases]{Use Cases}\label{use-cases-inlinenamespace}

\subsubsection[Facilitating API migration]{Facilitating API migration}\label{facilitating-api-migration}

Getting a large codebase to \emph{promptly} upgrade to a new version of
a library in any sort of timely fashion can be challenging. As a
simplistic illustration, imagine that we have just developed a new
library, \lstinline!parselib!, comprising a class template, \lstinline!Parser!,
and a function template, \lstinline!analyze!, that takes a \lstinline!Parser!
object as its only argument:

\begin{emcppslisting}[emcppsbatch=e2]
namespace parselib
{
    template <typename T>
    class Parser
    {
        // ...

    public:
        Parser();
        int parse(T* result, const char* input);
            // Load (ù{\codeincomments{result}}ù) from null-terminated (ù{\codeincomments{input}}ù); return (ù{\codeincomments{0}}ù) (on
            // success) or nonzero (with no effect on (ù{\codeincomments{result}}ù)).
    };

    template <typename T>
    double analyze(const Parser<T>& parser);
}
\end{emcppslisting}
    
\noindent To use our library, clients will need to specialize our \lstinline!Parser!
class directly within the \lstinline!parselib! namespace:

\begin{emcppslisting}[emcppsbatch=e2]
struct MyClass { /*...*/ };  // end-user-defined type

namespace parselib  // necessary to specialize (ù{\codeincomments{Parser}}ù)
{
    template <>            // Create *full* specialization of class
    class Parser<MyClass>  // (ù{\codeincomments{Parser}}ù) for user-type (ù{\codeincomments{MyClass}}ù).
    {
        // ...

    public:
        Parser();
        int parse(MyClass* result, const char* input);
            // The *contract* for a specialization typically remains the same.
    };

    double analyze(const Parser<MyClass>& parser);
};
\end{emcppslisting}
    
\noindent Typical \lstinline!client! code will also look for the \lstinline!Parser!
class directly within the \lstinline!parselib! namespace:

\begin{emcppslisting}[emcppsbatch=e2]
void client()
{
    MyClass result;
    parselib::Parser<MyClass> parser;

    int status = parser.parse(&result, "...( MyClass value )...");
    if (status != 0)
    {
        return;
    }

    double value = analyze(parser);
    // ...
}
\end{emcppslisting}
    
\noindent Note that invoking \lstinline!analyze! on objects of some instantiated type
of the \lstinline!Parser! class template will rely on \romeogloss{ADL} to find
the corresponding overload.

We anticipate that our library's API will evolve over time so we want to
enhance the design of \lstinline!parselib! accordingly. One of our goals is
to somehow encourage clients to move essentially all at once, yet also to
accommodate both the early adopters and the inevitable stragglers that
make up a typical adoption curve. Our approach will be to create, within
our outer \lstinline!parselib! namespace, a nested \lstinline!inline!
namespace, \lstinline!v1!, which will hold the current implementation of
our library software:

\begin{emcppslisting}[emcppsbatch=e3]
namespace parselib
{
    inline namespace v1             // Note our use of (ù{\codeincomments{inline}}ù) namespace here.
    {
        template <typename T>
        class Parser
        {
            // ...

        public:
            Parser();
            int parse(T* result, const char* input);
                // Load (ù{\codeincomments{result}}ù) from null-terminated (ù{\codeincomments{input}}ù); return 0 (on
                // success) or nonzero (with no effect on (ù{\codeincomments{result}}ù)).
        };

        template <typename T>
        double analyze(const Parser<T>& parser);
    }
}
\end{emcppslisting}
    
\noindent As suggested by the name \lstinline!v1!, this namespace serves primarily as
a mechanism to support library evolution through \romeogloss{API} and
\romeogloss{ABI} versioning (see \intraref{use-cases-inlinenamespace}{link-safe-abi-versioning} 
%\textit{\titleref{use-cases-inlinenamespace}: \titleref{link-safe-abi-versioning}} on page~\pageref{link-safe-abi-versioning} 
and \intraref{use-cases-inlinenamespace}{build-modes-and-abi-link-safety} ). 
%\textit{\titleref{use-cases-inlinenamespace}: \titleref{build-modes-and-abi-link-safety}} on page~\pageref{build-modes-and-abi-link-safety}). 
The need to specialize \lstinline!class!~\lstinline!Parser! and,
independently, the reliance on ADL to find the free function template
\lstinline!analyze! require the use of \lstinline!inline! namespaces, as
opposed to a conventional namespace followed by a \lstinline!using!
directive.

Note that, whenever a subsystem starts out directly in a first-level
namespace and is subsequently moved to a second-level nested namespace
for the purpose of versioning, declaring the inner namespace
\lstinline!inline! is the most reliable way to avoid inadvertently
destabilizing existing clients; see also \intraref{potential-pitfalls-inlinenamespace}{enabling-selective-using-directives-for-short-named-entities}.  
%\textit{\titleref{potential-pitfalls-inlinenamespace}: \titleref{enabling-selective-using-directives-for-short-named-entities}} on page~\pageref{enabling-selective-using-directives-for-short-named-entities}.

Now suppose we decide to enhance \lstinline!parselib! in a
non--backwards-compatible manner, such that the signature of
\lstinline!parse! takes a second argument \lstinline!size! of type
\lstinline!std::size_t! to allow parsing of non--null-terminated strings
and to reduce the risk of buffer overruns. Instead of unilaterally
removing all support for the previous version in the new release, we can
create a second namespace, \lstinline!v2!, containing the new
implementation and then, at some point, make \lstinline!v2! the
\lstinline!inline! namespace instead of \lstinline!v1!:

\begin{emcppslisting}
#include <cstddef>  // (ù{\codeincomments{std::size\_t}}ù)

namespace parselib
{
    namespace v1  // Notice that (ù{\codeincomments{v1}}ù) is now just a nested namespace.
    {
        template <typename T>
        class Parser
        {
            // ...

        public:
            Parser();
            int parse(T* result, const char* input);
                // Load (ù{\codeincomments{result}}ù) from null-terminated (ù{\codeincomments{input}}ù); return 0 (on
                // success) or nonzero (with no effect on (ù{\codeincomments{result}}ù)).
        };

        template <typename T>
        double analyze(const Parser<T>& parser);
    }

    inline namespace v2    // Notice that use of (ù{\codeincomments{inline}}ù) keyword has moved here.
    {
        template <typename T>
        class Parser
        {
            // ...

        public:  // note incompatible change to (ù{\codeincomments{Parser}}ù)'s essential API
            Parser();
            int parse(T* result, const char* input, std::size_t size);
                // Load (ù{\codeincomments{result}}ù) from (ù{\codeincomments{input}}ù) of specified (ù{\codeincomments{size}}ù); return 0
                // on success) or nonzero (with no effect on (ù{\codeincomments{result}}ù)).
        };

        template <typename T>
        double analyze(const Parser<T>& parser);
    }
}
\end{emcppslisting}
    
\noindent When we release this new version with \lstinline!v2! made \lstinline!inline!,
all existing clients that rely on the version supported directly in
\lstinline!parselib! will, by design, break when they recompile. At
that point, each client will have two options. The first one is
to upgrade the code immediately by passing in the size of the input
string (e.g., \lstinline!23!) along with the address of its first
character:

\begin{emcppslisting}[emcppserrorlines={4}]
void client()
{
    // ...
    int status = parser.parse(&result, "...( MyClass value )...", 23);
    // ...                                                      ^^^^ Look here!
}
\end{emcppslisting}
    
\noindent The second option is to change all references to
\lstinline!parselib! to refer to the original version in \lstinline!v1!
explicitly:

\begin{emcppshiddenlisting}[emcppsbatch=e3]
struct MyClass { /*...*/ };  // end-user-defined type
\end{emcppshiddenlisting}
\begin{emcppslisting}[emcppsbatch=e3]
namespace parselib
{
    namespace v1  // specializations moved to nested namespace
    {
        template <>
        class Parser<MyClass>
        {
            // ...

        public:
            Parser();
            int parse(MyClass* result, const char* input);
        };

        double analyze(const Parser<MyClass>& parser);
    }
};

void client1()
{
    MyClass result;
    parselib::v1::Parser<MyClass> parser;  // reference nested namespace (ù{\codeincomments{v1}}ù)

    int status = parser.parse(&result, "...( MyClass value )...");
    if (status != 0)
    {
        return;
    }

    double value = analyze(parser);
    // ...
}
\end{emcppslisting}
    
\noindent Providing the updated version in a new \lstinline!inline! namespace
\lstinline!v2! provides a more flexible migration path --- especially for a
large population of independent client programs --- compared to manual
targeted changes in client code.

Although new users would pick up the latest version automatically either
way, existing users of \lstinline!parselib! will have the option of
converting immediately by making a few small syntactic changes or
opting to remain with the original version for a while longer by making
all references to the library namespace refer explicitly to the desired
version. If the library is released before the \lstinline!inline! keyword
is moved, early adopters will have the option of opting in by referring
to \lstinline!v2! explicitly until it becomes the default. Those who have
no need for enhancements can achieve stability by referring to a
particular version in perpetuity or until it is physically removed from
the library source.

Although this same functionality can sometimes be realized without the
use of \lstinline!inline! namespaces (i.e., by adding a
\lstinline!using!~\lstinline!namespace! directive at the end of the
\lstinline!parselib! namespace), the use of ADL and the ability to
specialize templates from within the enclosing \lstinline!parselib!
namespace itself would be lost.{\cprotect\footnote{Note that, because
specialization doesn't kick in until overload resolution is completed,
specializing overloaded functions is dubious at best; see \intraref{potential-pitfalls-inlinenamespace}{specializing-templates-in-std-can-be-problematic}.}}
%\textit{\titleref{potential-pitfalls-inlinenamespace}: \titleref{specializing-templates-in-std-can-be-problematic}} on page~\pageref{specializing-templates-in-std-can-be-problematic}.}}

Providing separate namespaces for each successive version has an
additional advantage in an entirely separate dimension. Though not
demonstrated by this specific example,{\cprotect\footnote{For distinct
nested namespaces to effectively guard against accidental link-time
errors, the symbols involved have to (1) reside in object code (e.g.,
a \romeogloss{header-only library} would fail this requirement) and (2)
have the same \romeogloss{name mangling} (i.e., linker symbol) in both
versions. In this particular instance, however, the signature of the
\lstinline!parse! member function of \lstinline!parser! did change, and its
mangled name will consequently change as well; hence the same
  \lstinline!undefined!~\lstinline!symbol! link error would result either way.}}
cases do arise where simply changing which of the version namespaces is
declared \lstinline!inline! might lead to an \romeogloss{ill-formed, no-diagnostic
required (IFNDR)} program. This might happen when one or more of
its translation units that use the library are not recompiled before the
program is relinked to the new static or dynamic library containing the
updated version of the library software; see \intraref{use-cases-inlinenamespace}{link-safe-abi-versioning}. 
%\textit{\titleref{use-cases-inlinenamespace}: \titleref{link-safe-abi-versioning}} on page~\pageref{link-safe-abi-versioning}.

\subsubsection[Link-safe ABI versioning]{Link-safe ABI versioning}\label{link-safe-abi-versioning}

\lstinline!inline! namespaces are not intended as a mechanism for
source-code versioning; instead, they prevent programs from being
\romeogloss{ill-formed} due to linking some version of a library with client
code compiled using some other, typically older version of the same
library. Below, we present two examples: a simple pedagogical example
to illustrate the principle followed by a more real-world example.
Suppose we have a library component \lstinline!my_thing! that implements
an example type, \lstinline!Thing!, which wraps an \lstinline!int! and
initializes it with some value in its default constructor defined
out-of-line in the \lstinline!cpp! file:

\begin{emcppslisting}
struct Thing  // version 1 of class (ù{\codeincomments{Thing}}ù)
{
    int i;    // integer data member (size is 4)
    Thing();  // original non-(ù{\codeincomments{inline}}ù) constructor (defined in (ù{\codeincomments{.cpp}}ù) file)
};
\end{emcppslisting}
    
\noindent Compiling a source file with this version of the header included might produce an object file that can be incompatible yet linkable with an object file resulting from compiling a different source file with an a different version of this header included:

\begin{emcppslisting}
struct Thing   // version 2 of class (ù{\codeincomments{Thing}}ù)
{
    double d;  // double-precision floating-point data member (size is 8)
    Thing();   // updated non-(ù{\codeincomments{inline}}ù) constructor (defined in (ù{\codeincomments{.cpp}}ù) file)
};
\end{emcppslisting}
    
\noindent To make the problem that we are illustrating concrete, let's represent
the client as a \lstinline!main! program that does nothing but create a
\lstinline!Thing! and print the value of its only data member, \lstinline!i!.

\begin{emcppslisting}[emcppsbatch=e4]
// main.cpp:
#include <my_thing.h>  // (ù{\codeincomments{my::Thing}}ù) (version 1)
#include <iostream>    // (ù{\codeincomments{std::cout}}ù)

int main()
{
     my::Thing t;
     std::cout << t.i << '\n';
}
\end{emcppslisting}
    
\noindent If we compile this program, a reference to a locally undefined linker
symbol, such as
\lstinline!_ZN2my7impl_v15ThingC1Ev!,{\cprotect\footnote{On a Unix
machine, typing \lstinline!nm!~\lstinline!main.o! reveals the symbols used
in the specified object file. A symbol prefaced with a capital
\lstinline!U! represents an undefined symbol that must be resolved by the
linker. Note that the linker symbol shown here incorporates an
intervening \lstinline!inline! namespace, \lstinline!impl_v1!, as will be
  explained shortly.}} which represents the \lstinline!my::Thing::Thing!
constructor, will be generated in the \lstinline!main.o! file:

\begin{lstlisting}[language=bash,style=plain]
$ g++ -c main.cpp
\end{lstlisting}
    
\noindent Without explicit intervention, the spelling of this linker symbol would
be unaffected by any subsequent changes made to the implementation of
\lstinline!my::Thing!, such as its data members or implementation of its
default constructor, even after recompiling. The same, of course,
applies to its definition in a separate translation unit.

We now turn to the translation unit implementing type
\lstinline!my::Thing!. The \lstinline!my_thing! \romeogloss{component} consists
of a \lstinline!.h!/\lstinline!.cpp! pair: \lstinline!my_thing.h! and
\lstinline!my_thing.cpp!. The header file \lstinline!my_thing.h! provides
the physical interface, such as the definition of the principal type,
\lstinline!Thing!, its member and associated free function declarations,
plus definitions for inline functions and function templates, if any:

\begin{emcppslisting}[emcppsbatch=e4]
// my_thing.h:
#ifndef INCLUDED_MY_THING
#define INCLUDED_MY_THING

namespace my                  // outer namespace (used directly by clients)
{
    inline namespace impl_v1  // inner namespace (for implementer use only)
    {
        struct Thing
        {
            int i;    // original data member, size = 4
            Thing();  // default constructor (defined in (ù{\codeincomments{my\_thing.cpp}}ù))
        };
    };
}

#endif
\end{emcppslisting}
    
\noindent The implementation file \lstinline!my_thing.cpp! contains all of the
non-\lstinline!inline! function bodies that will be translated separately
into the \lstinline!my_thing.o! file:

\begin{emcppslisting}[emcppsbatch=e4]
// my_thing.cpp:
#include <my_thing.h>

namespace my                   // outer namespace (used directly by clients)
{
    inline namespace impl_v1   // inner namespace (for implementer use only)
    {
        Thing::Thing() : i(0)  // load a 4-byte value into (ù{\codeincomments{Thing}}ù)'s data member
        {
        }
    }
}
\end{emcppslisting}
    
\noindent Observing common good practice, we include the header file of the component
as the first substantive line of code to ensure that --- irrespective of
anything else --- the header always compiles in isolation, thereby
avoiding insidious include-order dependencies.{\cprotect\footnote{See
  \cite{lakos20}, section~1.6.1, ``Component Property 1," pp.~210--212.}} When we compile the source file \lstinline!my_thing.cpp!,
we produce an object file \lstinline!my_thing.o! containing the definition
of the very same linker symbol, such as
\lstinline!_ZN2my7impl_v15ThingC1Ev!, for the default constructor of
\lstinline!my::Thing! needed by the client:

\begin{lstlisting}[language=bash,style=plain]
$ g++ -c my_thing.cpp
\end{lstlisting}
    
\noindent We can then link \lstinline!main.o! and \lstinline!my_thing.o! into an
executable and run it:

\begin{lstlisting}[language=bash,style=plain]
$ g++ -o prog main.o my_thing.o
$ ./prog

0
\end{lstlisting}
    
\noindent Now, suppose we were to change the definition of \lstinline!my::Thing! to hold a \lstinline!double! instead of an \lstinline!int!, recompile
\lstinline!my_thing.cpp!, and then relink with the original
\lstinline!main.o! without recompiling \lstinline!main.cpp! first. None of the
relevant linker symbols would change, and the code would recompile and link
just fine, but the resulting binary \lstinline!prog! would be
\romeogloss{IFNDR}: the client would be trying to print a 4-byte,
\lstinline!int! data member, \lstinline!i!, in \lstinline!main.o! that was loaded
by the library component as an 8-byte, \lstinline!double! into \lstinline!d!
in \lstinline!my_thing.o!. We can resolve this problem by changing --- or,
if we didn't think of it in advance, by adding --- a new \lstinline!inline!
namespace and making that change there:

\begin{emcppshiddenlisting}[emcppsbatch=e5]
// my_thing.h:
#ifndef INCLUDED_MY_THING
#define INCLUDED_MY_THING

namespace my                  // outer namespace (used directly by clients)
{
    inline namespace impl_v2  // inner namespace (for implementer use only)
    {
        struct Thing
        {
            double d; // new data member, size = 8
            Thing();  // default constructor (defined in (ù{\codeincomments{my\_thing.cpp}}ù))
        };
    };
}

#endif
\end{emcppshiddenlisting}
\begin{emcppslisting}[emcppsbatch=e5]
// my_thing.cpp:
#include <my_thing.h>

namespace my                     // outer namespace (used directly by clients)
{
    inline namespace impl_v2     // inner namespace (for implementer use only)
    {
        Thing::Thing() : d(0.0)  // load 8-byte value into (ù{\codeincomments{Thing}}ù)'s data member
        {
        }
    }
}
\end{emcppslisting}
    
\noindent Now clients that attempt to link against the new library will not find
the linker symbol, such as \lstinline!_Z...impl_v1...v!, and the link
stage will fail. Once clients recompile, however, the undefined linker
symbol will match the one available in the new \lstinline!my_thing.o!,
such as \lstinline!_Z...impl_v2...v!, the link stage will succeed, and
the program will again work as expected. What's more, we have the option
of keeping the original implementation. In that case, existing clients
that have not as yet recompiled will continue to link against the old
version until it is eventually removed after some suitable deprecation
period.

As a more realistic second example of using \lstinline!inline! namespaces
to guard against linking incompatible versions, suppose we have two
versions of a \lstinline!Key! class in a security library in the enclosing
namespace, \lstinline!auth! --- the original version in a regular nested
namespace \lstinline!v1!, and the new current version in an \lstinline!inline!
nested namespace \lstinline!v2!:

\begin{emcppslisting}
#include <cstdint>  // (ù{\codeincomments{std::uint32\_t}}ù), (ù{\codeincomments{std::unit64\_t}}ù)

namespace auth      // outer namespace (used directly by clients)
{
    namespace v1    // inner namespace (optionally used by clients)
    {
        class Key
        {
        private:
            std::uint32_t d_key;
                // (ù{\codeincomments{sizeof(Key)}}ù) is 4 bytes

        public:
            std::uint32_t key() const;  // stable interface function

            // ...
        };
    }

    inline namespace v2    // inner namespace (default current version)
    {
        class Key
        {
        private:
            std::uint64_t d_securityHash;
            std::uint32_t d_key;
                // (ù{\codeincomments{sizeof(Key)}}ù) is 16 bytes

        public:
            std::uint32_t key() const;  // stable interface function

            // ...
        };
    }
}
\end{emcppslisting}
    
\noindent Attempting to link together older binary artifacts built against version
1 with binary artifacts built against version 2 will result in a
link-time error rather than allowing an \romeogloss{ill-formed} program to
be created. Note, however, that this approach works only if functionality essential
to typical use is defined out of line in a \lstinline!.cpp! file. For example, it would add absolutely no value for libraries that are
shipped entirely as header files, since the versioning offered here
occurs strictly at the binary level (i.e., between object files)
during the link stage.

\subsubsection[Build modes and ABI link safety]{Build modes and ABI link safety}\label{build-modes-and-abi-link-safety}

In certain scenarios, a class might have two different memory layouts
depending on compilation flags. For instance, consider a low-level
\lstinline!ManualBuffer! class template in which an additional data member
is added for debugging purposes{\cprotect\footnote{Note that we have
  employed the C++11 \lstinline!alignas! attribute (see \featureref{\locationc}{alignas})   
 % ``\titleref{alignas}" on page~\pageref{alignas}) 
  here
  because it is exactly what's needed for this usage example.}}:

\begin{emcppslisting}
template <typename T>
struct ManualBuffer
{
private:
    alignas(T) char d_data[sizeof(T)];  // aligned and big enough to hold a (ù{\codeincomments{T}}ù)

#ifndef NDEBUG
    bool d_engaged;  // tracks whether buffer is full (debug builds only)
#endif

public:
    void construct(const T& obj);
        // Emplace (ù{\codeincomments{obj}}ù). (Engage the buffer.) The behavior is undefined unless
        // the buffer was not previously engaged.

    void destroy();
        // Destroy the current (ù{\codeincomments{obj}}ù). (Disengage the buffer.) The behavior is
        // undefined unless the buffer was previously engaged.

    // ...
};
\end{emcppslisting}
    
\noindent The \lstinline!d_engaged! flag in the example above serves as a way to detect misuse of
the \mbox{\lstinline!ManualBuffer!} class but only in debug builds. The extra
space and run time required to maintain this Boolean flag is undesirable
in a release build because \mbox{\lstinline!ManualBuffer!} is intended to be an
efficient, lightweight abstraction over the direct use of
\romeogloss{placement \lstinline!new!} and explicit destruction.

The linker symbol names generated for the methods of
\lstinline!ManualBuffer! are the same irrespective of the chosen build
mode. If the same program links together two object files where
\lstinline!ManualBuffer! is used --- one built in debug mode and one built
in release mode --- the \romeogloss{one definition rule} will be violated
and the program will again be \romeogloss{IFNDR}.

One way of avoiding these sorts of incompatibilities at link time is to
introduce two \lstinline!inline! namespaces, the entire purpose of which is
to change the ABI-level names of the linker symbols associated with
\lstinline!ManualBuffer! depending on the build mode{\cprotect\footnote{Prior
to \lstinline!inline! namespaces, it was possible to control the
ABI-level name of linked symbols by creating separate template
instantiations on a per-build-mode basis:

\begin{emcppslisting}[language=C++, style=footcode]
#ifndef NDEBUG
enum { is_debug_build = 1 };
#else
enum { is_debug_build = 0 };
#endif

template <typename T, bool Debug = is_debug_build>
struct ManualBuffer { /* ... */ };
\end{emcppslisting}
    
\noindent While the code above changes the interface of \lstinline!ManualBuffer! to
accept an additional template parameter, it also allows debug and
release versions of the same class to coexist in the same program,
  which might prove useful, e.g., for testing.}}:

\begin{emcppslisting}
#ifndef NDEBUG            // perhaps a BAD IDEA
inline namespace release
#else
inline namespace debug
#endif
{
    template <typename T>
    struct ManualBuffer
    {
        // ... (same as above)
    };
}
\end{emcppslisting}
    
\noindent The approach demonstrated in this example tries to ensure that a linker
error will occur if any attempt is made to link objects built with a
build mode different from that of \mbox{\lstinline!manualbuffer.o!}. Tying it to
the \lstinline!NDEBUG! flag, however, might have unintended consequences;
we might introduce unwanted restrictions in what we call
\romeogloss{mixed-mode builds}. Most modern platforms support the notion of
linking a collection of object files irrespective of their
optimization levels. The same is certainly true for whether or not
C-style \lstinline!assert! is enabled. In other words, we may want to have
a mixed-mode build where we link object files which differ in their optimization and assertion options, as long as they are binary compatible --- i.e., in this case, they all must be uniform with respect to the implementation of \lstinline!ManualBuffer!.  Hence, a more general, albeit more complicated and
manual, approach would be to tie the non-interoperable behavior
associated with this ``safe'' or ``defensive'' build mode to a different
switch entirely. Another consideration would be to avoid ever inlining a
namespace into the global namespace since no method is available to
recover a symbol when there is a collision:

\begin{emcppslisting}
namespace buflib  // GOOD IDEA: enclosing namespace for nested (ù{\codeincomments{inline}}ù) namespace
{
#ifdef SAFE_MODE  // GOOD IDEA: separate control of non-interoperable versions
    inline namespace safe_build_mode
#else
    inline namespace normal_build_mode
#endif
    {
        template <typename T>
        struct ManualBuffer
        {
        private:
            alignas(T) char d_data[sizeof(T)];  // aligned/sized to hold a (ù{\codeincomments{T}}ù)

#ifdef SAFE_MODE
            bool d_engaged;  // tracks whether buffer is full (safe mode only)
#endif

        public:
            void construct(const T& obj);  // sets (ù{\codeincomments{d\_engaged}}ù) (safe mode only)
            void destroy();                // sets (ù{\codeincomments{d\_engaged}}ù) (safe mode only)
            // ...
        };
    }
}
\end{emcppslisting}
    
\noindent And, of course, the appropriate conditional compilation within the
function bodies would need to be in the corresponding \lstinline!.cpp!
file.

Finally, if we have two implementations of a particular entity that are
sufficiently distinct, we might choose to represent them in their
entirety, controlled by their own bespoke conditional-compilation
switches, as illustrated here using the \lstinline!my::VersionedThing! type
(see \intraref{use-cases-inlinenamespace}{link-safe-abi-versioning}): 
% \textit{\titleref{use-cases-inlinenamespace}: \titleref{link-safe-abi-versioning}} on page~\pageref{link-safe-abi-versioning}:

\begin{emcppshiddenlisting}[emcppsbatch=e7]
// my_versionedthing.h:
#define MY_THING_VERSION_1
\end{emcppshiddenlisting}
\begin{emcppslisting}[emcppsbatch=e7]
// my_versionedthing.h:
#ifndef INCLUDED_MY_VERSIONEDTHING
#define INCLUDED_MY_VERSIONEDTHING

namespace my
{
#ifdef MY_THING_VERSION_1  // bespoke switch for this component version
    inline
#endif
    namespace v1
    {
        struct VersionedThing
        {
            int d_i;
            VersionedThing();
        };
    }

#ifdef MY_THING_VERSION_2  // bespoke switch for this component version
    inline
#endif
    namespace v2
    {
        struct VersionedThing
        {
            double d_i;
            VersionedThing();
        };
    }
}
#endif
\end{emcppslisting}
    
\noindent However, see \intraref{potential-pitfalls-inlinenamespace}{inline-namespace-based-versioning-doesn’t-scale}. 
%\textit{\titleref{potential-pitfalls-inlinenamespace}: \titleref{inline-namespace-based-versioning-doesn’t-scale}} on page~\pageref{inline-namespace-based-versioning-doesn’t-scale}.

\subsubsection[Enabling selective \lstinline!using! directives for short-named entities]{Enabling selective {\SubsubsecCode using} directives for short-named entities}\label{enabling-selective-using-directives-for-short-named-entities}

Introducing a large number of small names into client code that doesn't
follow rigorous nomenclature can be problematic. Hoisting these names
into one or more nested namespaces so that they are easier to identify
as a unit and can be used more selectively by clients, such as through
explicit qualification or using directives, can sometimes be an
effective way of organizing shared codebases. For example,
\lstinline!std::literals! and its nested namespaces, such as
\lstinline!chrono_literals!, were introduced as \lstinline!inline! namespaces
in C++14. As it turns out, clients of these nested namespaces have no
need to specialize any templates defined in these namespaces nor do they
define types that must be found through \romeogloss{ADL}, but one can at
least imagine special circumstances in which such tiny-named entities
are either templates that require specialization or operator-like
functions, such as \lstinline!swap!, defined for local types within those
nested namespaces. In those cases, \lstinline!inline! namespaces would be
required to preserve the desired ``as if'' properties.

Even without either of these two needs, another property of an
\lstinline!inline! namespace differentiates it from a non-\lstinline!inline!
one followed by a \lstinline!using! directive. Recall from\intraref{description-inlinenamespace}{loss-of-access-to-duplicate-names-in-enclosing-namespace}  
%\textit{\titleref{description-inlinenamespace}: \titleref{loss-of-access-to-duplicate-names-in-enclosing-namespace}} on page~\pageref{loss-of-access-to-duplicate-names-in-enclosing-namespace} 
that a name in an outer namespace will
hide a duplicate name imported via a \lstinline!using! directive, whereas
any access to that duplicate name within the enclosing namespace would
be ambiguous when that symbol is installed by way of an \lstinline!inline!
namespace. To see why this more forceful clobbering behavior might be
preferred over hiding, suppose we have a communal namespace \lstinline!abc!
that is shared across multiple disparate headers. The first header,
\lstinline!abc_header1.h!, represents a collection of logically related
small functions declared directly in \lstinline!abc!:

\begin{emcppslisting}[emcppsbatch=e6]
// abc_header1.h:
namespace abc
{
    int i();
    int am();
    int smart();
}
\end{emcppslisting}
    
\noindent A second header, \lstinline!abc_header2.h!, creates a suite of many
functions having tiny function names. In a perhaps misguided effort to
avoid clobbering other symbols within the \lstinline!abc! namespace having
the same name, all of these tiny functions are sequestered within a
\lstinline!nested! namespace:

\begin{emcppslisting}[emcppsbatch=e6]
// abc_header2.h:
namespace abc
{
    namespace nested  // Should this instead have been an (ù{\codeincomments{inline}}ù) namespace?
    {
        int a();  // lots of functions with tiny names
        int b();
        int c();
        // ...
        int h();
        int i();  // might collide with another name declared in (ù{\codeincomments{abc}}ù)
        // ...
        int z();
    }

    using namespace nested;  // becomes superfluous if (ù{\codeincomments{nested}}ù) is made (ù{\codeincomments{inline}}ù)
}
\end{emcppslisting}
    
\noindent Now suppose that a client application includes
both of these headers to accomplish some task:

\begin{emcppslisting}[emcppsbatch=e6]
// client.cpp:
#include <abc_header1.h>
#include <abc_header2.h>

int function()
{
    if (abc::smart() < 0) { return -1; }  // uses (ù{\codeincomments{smart()}}ù) from (ù{\codeincomments{abc\_header1.h}}ù)
    return abc::z() + abc::i() + abc::a() + abc::h() + abc::c();  // Oops!
        // Bug, silently uses the (ù{\codeincomments{abc::i()}}ù) defined in (ù{\codeincomments{abc\_header1.h}}ù)
}
\end{emcppslisting}
    
\noindent In trying to cede control to the client as to whether the declared or
imported \lstinline!abc::i()! function is to be used, we have, in effect,
invited the defect illustrated in the above example whereby the client
was expecting the \lstinline!abc::i()! from \lstinline!abc_header2.h! and yet
picked up the one from \lstinline!abc_header1.h! by default. Had the
\lstinline!nested! namespace in \lstinline!abc_header2.h! been declared
\lstinline!inline!, the qualified name \lstinline!abc::i()! would have
automatically been rendered \emph{ambiguous} in namespace \lstinline!abc!,
the translation would have failed \emph{safely}, and the defect would
have been exposed at compile time. The downside, however, is that no
method would be available to recover nominal access to the
\lstinline!abc::i()! defined in \lstinline!abc_header1.h! once
\lstinline!abc_header2.h! is included, even though the two functions
(e.g., including their \romeogloss{mangled names} at the ABI level) remain
distinct.

\subsection[Potential Pitfalls]{Potential Pitfalls}\label{potential-pitfalls-inlinenamespace}

\subsubsection[\lstinline!nline!-namespace-based versioning doesn’t scale]{{\SubsubsecCode inline}-namespace-based versioning doesn’t scale}\label{inline-namespace-based-versioning-doesn’t-scale}

The problem with using \lstinline!inline! namespaces for ABI link safety is
that the protection they offer is only partial; in a few major places,
critical problems can linger until run time instead of being caught at
compile time.

Controlling which namespace is \lstinline!inline! using macros, such as was
done in the\linebreak[4] \lstinline!my::VersionedThing! example in \intraref{use-cases-inlinenamespace}{link-safe-abi-versioning},  
%\textit{\titleref{use-cases-inlinenamespace}: \titleref{link-safe-abi-versioning}} on page~\pageref{link-safe-abi-versioning}, 
will result in code that
directly uses the unversioned name, \lstinline!my::VersionedThing! being
bound directly to the versioned name \lstinline!my::v1::VersionedThing! or
\lstinline!my::v2::VersionedThing!, along with the class layout of that
particular entity. Sometimes details of the use of the \lstinline!inline!
namespace member are not resolved by the linker, such as the object
layout when we use types from that namespace as member variables in
other objects:

\begin{emcppslisting}[emcppsbatch=e7]
// my_thingaggregate.h:

// ...
#include <my_versionedthing.h>
// ...

namespace my
{
    struct ThingAggregate
    {
        // ...
        VersionedThing d_thing;
        // ...
    };
}
\end{emcppslisting}
    
\noindent This new \lstinline!ThingAggregate! type does not have the versioned
\lstinline!inline! namespace as part of its mangled name; it does, however,
have a completely different layout if built with
\lstinline!MY_THING_VERSION_1! defined
versus \lstinline!MY_THING_VERSION_2! defined. Linking a program with mixed
versions of these flags will result in runtime failures that are
decidedly difficult to diagnose.

This same sort of problem will arise for functions taking arguments of
such types; calling a function from code that is wrong about the layout
of a particular type will result in stack corruption and other undefined
and unpredictable behavior. This macro-induced problem will also arise in cases where an old object
file is linked against new code that changes which namespace is
\lstinline!inline!d but still provides the definitions for the old version
namespace. The old object file for the client can still link, but new
object files using the headers for the old objects might attempt to
manipulate those objects using the new namespace.

The only viable workaround for this approach is to propagate the
\lstinline!inline! namespace hierarchy through the entire software stack.
Every object or function that uses\linebreak[4] \lstinline!my::VersionedThing! needs to
also be in a namespace that differs based on the same control macro. In
the case of \lstinline!ThingAggregate!, one could just use the same
\lstinline!my::v1! and \lstinline!my::v2! namespaces, but higher-level
libraries would need their own \lstinline!my!-specific nested namespaces.
Even worse, for higher-level libraries, every lower-level library having
a versioning scheme of this nature would need to be considered,
resulting in having to provide the full cross-product of nested
namespaces to get link-time protection against mixed-mode builds.

This need for layers above a library to be aware of and to integrate
into their own structure the same namespaces the library has removes all
or most of the benefits of using \lstinline!inline! namespaces for
versioning. For an authentic real-world case study of heroic industrial use --- and
eventual disuse --- of \lstinline!inline!-namespaces for versioning, see \intrarefsimple{appendix:-case-study-of-using-inline-namespaces-for-versioning}. 
% \textit{\titleref{appendix:-case-study-of-using-inline-namespaces-for-versioning}} on page~\pageref{appendix:-case-study-of-using-inline-namespaces-for-versioning}.

\subsubsection[Relying on \lstinline!inline! namespaces to solve library evolution]{Relying on {\SubsubsecCode inline} namespaces to solve library evolution}\label{specializing-templates-in-std-can-be-problematic}

%Inline namespaces might be perceived as a complete solution for the owner of a library to evolve its API. Such, however, is not the case. As an example, consider the case of the C++ Standard Library, which does not use \lstinline!inline! namespaces for versioning. Instead, to allow for its necessary evolution, the Standard
%Library --- the \lstinline!std! namespace, in particular --- carries
%certain special restrictions on what one can do with it that are
%enforced by deeming certain constructs \romeogloss{ill formed} or
%engendering \romeogloss{undefined behavior}.

Inline namespaces might be misperceived as a complete solution for the owner of a library to evolve its API. As an especially relevant example, consider the C++ Standard Library, which itself does not use inline namespaces for versioning. Instead, to allow for its anticipated essential evolution, the Standard Library imposes certain special restrictions on what is permitted to occur within its own \lstinline!std! namespace by dint of deeming certain problematic uses as either \romeogloss{ill formed} or otherwise engendering \romeogloss{undefined behavior}.

Since C++11, several restrictions related to the Standard Library were
put in place:
\begin{itemize}
\item{Users may not add any new declarations within namespace \lstinline!std!. This means that users cannot add new \emph{functions}, \emph{overloads}, \emph{types}, or \emph{templates} to \lstinline!std!. This restriction gives the Standard Library freedom to add new \emph{names} in future versions of the Standard.}
\item{Users may not specialize member functions, member function templates, or member class templates. Specializing any of those entities might significantly inhibit a Standard Library vendor’s ability to maintain its otherwise encapsulated implementation details.}
\item{Users may add specializations of top-level Standard Library templates only if the declaration depends on the name of a nonstandard user-defined type and only if that user-defined type meets all requirements of the original template. Specialization of function templates is allowed but generally discouraged because this practice doesn’t scale since function templates cannot be partially specialized. Specializing of standard class templates when the specialization names a nonstandard user-defined type, such as \lstinline!std::vector<MyType*>!, is allowed but also problematic when not explicitly supported. While certain specific types, such as \lstinline!std::hash!, are designed for user specialization, steering clear of the practice for any other type helps to avoid surprises.}
\end{itemize}

Several other good practices facilitate smooth evolution for the
Standard Library{\cprotect\footnote{These restrictions are normative in
C++20, having finally formalized what were long identified as best
practices. Though these restrictions might not be codified in the
Standard for pre-C++20 software, they have been recognized best
practices for as long as the Standard Library has existed and
adherence to them will materially improve the ability of software to
migrate to future language standards irrespective of what version of
  the language standard is being targeted.}}:
\begin{itemize}
\item{Avoid specializing variable templates, even if dependent on user-defined types, except for those variable templates where specialization is explicitly allowed.\cprotect\footnote{C++20 limits the specialization of variable templates to only those instances where specialization is explicitly allowed and does so only for the mathematical constants in \lstinline!<numbers>!.}}
\item{Other than a few very specific exceptions, avoiding the forming of pointers to Standard Library functions — either explicitly or implicitly — allows the library to add overloads, either as part of the Standard or as an implementation detail for a particular Standard Library, without breaking user code.\cprotect\footnote{C++20 identifies these functions as \lstinline!addressable! and gives that property to only \lstinline!iostream! manipulators since those are the only functions in the Standard Library for which taking their address is part of normal usage.}}
\item{Overloads of Standard Library functions that depend on user-defined types are permitted, but, as with specializing Standard Library templates, users must still meet the requirements of the Standard Library function. Some functions, such as \lstinline!std::swap!, are designed to be customization points via overloading, but leaving functions not specifically designed for this purpose to vendor implementations only helps to avoid surprises.}
\end{itemize}

Finally, upon reading about this \lstinline!inline! namespace feature, one
might think that all names in namespace \lstinline!std! could be made
available at a global scope simply by inserting an\linebreak[4]%%%%%%
\lstinline!inline!~\lstinline!namespace!~\lstinline!std!~\lstinline!{}! before
including any standard headers. This practice is, however, explicitly
called out as ill-formed within the C++11 Standard.\footnote{Although not uniformly diagnosed as an error by
all compilers, attempting this forbidden practice is apt to lead to
surprising problems even if not diagnosed as an error immediately.}

\subsubsection[Inconsistent use of \lstinline!inline! keyword is ill formed, no diagnostic required]{Inconsistent use of {\SubsubsecCode inline} keyword is ill formed, no diagnostic required}\label{inconsistent-use-of-inline-keyword-is-ifndr}

It is an \romeogloss{ODR} violation, \romeogloss{IFNDR}, for a nested namespace
to be \lstinline!inline! in one translation unit and non-\lstinline!inline! in
another. And yet, the motivating use case of this feature relies on the
linker to actively complain whenever different, incompatible versions
--- nested within different, possibly \lstinline!inline!-inconsistent,
namespaces of an ABI --- are used within a single executable. Because
declaring a nested namespace \lstinline!inline! does not, by design, affect
linker-level symbols, developers must take appropriate care, such as
effective use of header files, to defend against such preventable
inconsistencies.

\subsection[Annoyances]{Annoyances}\label{annoyances-inlinenamespace}

\subsubsection[Inability to redeclare across namespaces impedes code factoring]{Inability to redeclare across namespaces impedes code factoring}\label{inability-to-redeclare-across-namespaces-impedes-code-factoring}

An essential feature of an \lstinline!inline! namespace is the ability to
declare a template within a nested \lstinline!inline! namespace and then
specialize it within its enclosing namespace. For example, we can
declare
\begin{itemize}
\item{a type template, \lstinline!S0!}
\item{a couple of function templates, \lstinline!f0! and \lstinline!g0!}
\item{and a member function template \lstinline!h0!, which is similar to \lstinline!f0!}
\end{itemize}
in an \lstinline!inline! namespace, \lstinline!inner!, and specialize each of
them, such as for \lstinline!int!, in the enclosing namespace,
\lstinline!outer!:

\begin{emcppslisting}
namespace outer                                          // enclosing namespace
{
    inline namespace inner                               // nested namespace
    {
        template<typename T> struct S0;                  // declarations of
        template<typename T> void f0();                  // various class
        template<typename T> void g0(T v);               // and function
        struct A0 { template <typename T> void h0(); };  // templates
    }

    template<> struct S0<int> { };                       // specializations
    template<> void f0<int>() { }                        // of the various
    void g0(int) { }  /* overload not specialization */  // class and function
    template<> void A0::h0<int>() { }                    // declarations above
}                                                        // in (ù{\codeincomments{outer}}ù) namespace
\end{emcppslisting}
    
\noindent Note that, in the case of \lstinline!g0! in this example, the
``specialization'' \lstinline!void!~\lstinline!g0(int)! is a non-template
\emph{overload} of the function template \lstinline!g0! rather than a
specialization of it. We \emph{cannot}, however,
portably{\cprotect\footnote{GCC provides the \lstinline!-fpermissive! flag,
which allows the example containing specializations within the inner
namespace to compile with warnings. Note again that \lstinline!g1(int)!,
being an \emph{overload} and not a \emph{specialization}, wasn't an
  error and, therefore, isn't a warning either.}} declare these
templates within the \lstinline!outer! namespace and then specialize them
within the \lstinline!inner! one, even though the \lstinline!inner! namespace
is \lstinline!inline!:

\begin{emcppslisting}
namespace outer                                     // enclosing namespace
{
    template<typename T> struct S1;                 // class template
    template<typename T> void f1();                 // function template
    template<typename T> void g1(T v);              // function template

    struct A1 { template <typename T> void h1(); }; // member function template

    inline namespace inner                          // nested namespace
    {                                               // BAD IDEA
        template<> struct S1<int> { };              // Error, (ù{\codeincomments{S1}}ù) not a template
        template<> void f1<int>() { }               // Error, (ù{\codeincomments{f1}}ù) not a template
        void g1(int) { }                            // OK, overloaded function
        template<> void A1::h1<int>() { }           // Error, (ù{\codeincomments{h1}}ù) not a template
    }
}
\end{emcppslisting}
    
\noindent Attempting to declare a template in the \lstinline!outer! namespace and
then define, effectively redeclaring, it in an \lstinline!inline! inner one
causes the name to be inaccessible within the \lstinline!outer! namespace:

\begin{emcppslisting}
namespace outer                                          // enclosing namespace
{                                                        // BAD IDEA
    template<typename T> struct S2;                      // declarations of
    template<typename T> void f2();                      // various class and
    template<typename T> void g2(T v);                   // function templates

    inline namespace inner                               // nested namespace
    {
        template<typename T> struct S2 { };              // definitions of
        template<typename T> void f2() { }               // unrelated class and
        template<typename T> void g2(T v) { }            // function templates
    }

    template<> struct S2<int> { };     // Error, (ù{\codeincomments{S2}}ù) is ambiguous in (ù{\codeincomments{outer}}ù).
    template<> void f2<int>() { }      // Error, (ù{\codeincomments{f2}}ù) is ambiguous in (ù{\codeincomments{outer}}ù).
    void g2(int) { }                   // OK, (ù{\codeincomments{g2}}ù) is an overload definition.
}
\end{emcppslisting}
    
\noindent Finally, declaring a template in the nested \lstinline!inline! namespace
\lstinline!inner! in the example above and then subsequently defining it in
the enclosing \lstinline!outer! namespace has the same effect of making
declared symbols ambiguous in the \lstinline!outer! namespace:

\begin{emcppslisting}[emcppsbatch=e8]
namespace outer                                          // enclosing namespace
{                                                        // BAD IDEA
    inline namespace inner                               // nested namespace
    {
        template<typename T> struct S3;                  // declarations of
        template<typename T> void f3();                  // various class
        template<typename T> void g3(T v);               // and function
        struct A3 { template <typename T> void h3(); };  // templates
    }

    template<typename T> struct S3 { };                  // definitions of
    template<typename T> void f3() { }                   // unrelated class
    template<typename T> void g3(T v) { }                // and function
    template<typename T> void A3::h3() { };              // templates

    template<> struct S3<int> { };     // Error, (ù{\codeincomments{S3}}ù) is ambiguous in (ù{\codeincomments{outer}}ù).
    template<> void f3<int>() { }      // Error, (ù{\codeincomments{f3}}ù) is ambiguous in (ù{\codeincomments{outer}}ù).
    void g3(int) { }                   // OK, (ù{\codeincomments{g3}}ù) is an *overload* definition.
    template<> void A3::h3<int>() { }  // Error, (ù{\codeincomments{h2}}ù) is ambiguous in (ù{\codeincomments{outer}}ù).
}
\end{emcppslisting}
    
\noindent Note that, although the definition for a member function template must
be located directly within the namespace in which it is declared, a
class or function template, once declared, may be defined in a different
scope by using an appropriate name qualification:

\begin{emcppslisting}[emcppsbatch=e8]
template <typename T> struct outer::S3 { };        // OK, enclosing namespace
template <typename T> void outer::inner::f3() { }  // OK, nested namespace
template <typename T> void outer::g3(T v) { }      // OK, enclosing namespace
template <typename T> void outer::A3::h3<T>() { }  // Error, ill-formed

namespace outer
{
    inline namespace inner
    {
        template <typename T> void A3::h3<T>() { } // OK, within same namespace
    }
}
\end{emcppslisting}
    
\noindent Also note that, as ever, the corresponding definition of the declared
template must have been seen before it can be used in a context
requiring a complete type. The importance of ensuring that all
specializations of a template have been seen before it is used
substantively (i.e., \romeogloss{ODR-used}) cannot be overstated, giving
rise to the only limerick, which is actually part of the normative text,
in the C++ Language Standard{\cprotect\footnote{See \cite{iso11},
  section~14.7.3.7, pp.~375--375, specifically p.~376.}}:
\begin{quote}
When writing a specialization,\\
be careful about its location;\\
or to make it compile\\
will be such a trial\\
as to kindle its self-immolation.
\end{quote}

\subsubsection[Only one namespace can contain any given \lstinline!inline! namespace]{Only one namespace can contain any given {\SubsubsecCode inline} namespace}\label{only-one-namespace-can-contain-any-given-inline-namespace}

Unlike conventional \lstinline!using! directives, which can be used to
generate arbitrary many-to-many relationships between different
namespaces, \lstinline!inline! namespaces can be used only to contribute
names to the sequence of enclosing namespaces up to the first
non-\lstinline!inline! one. In cases in which the names from a namespace
are desired in multiple other namespaces, the classical \lstinline!using!
directive must be used, with the subtle differences between the two
modes properly addressed.

As an example, the C++14 Standard Library provides a hierarchy of nested
\lstinline!inline! namespaces for literals of different sorts within
namespace\linebreak[4] \lstinline!std!: \lstinline!std::literals::complex_literals!,
\lstinline!std::literals::chrono_literals!,\linebreak[4]
\lstinline!std::literals::string_literals!, and
\lstinline!std::literals::string_view_literals!.\linebreak[4] 
These namespaces can
be imported to a local scope in one shot via a
\lstinline!using!~\lstinline!std::literals! or instead, more selectively, by
\lstinline!using! the nested namespaces directly. This separation of the
types used with user-defined literals, which are all in namespace
\lstinline!std!, from the user-defined literals that can be used to create
those types led to some frustration; those who had a
\lstinline!using!~\lstinline!namespace!~\lstinline!std;! could reasonably have
expected to get the user-defined literals associated with their
\lstinline!std! types. However, the types in the nested namespace
\lstinline!std::chrono! did \emph{not} meet this
expectation.{\cprotect\footnote{\cite{hinnant17}}}

Eventually \emph{both} solutions for incorporating literal namespaces,
\lstinline!inline! from\linebreak[4] \lstinline!std::literals! and non-\lstinline!inline! from
\lstinline!std::chrono!, were pressed into service when, in C++17, a
\mbox{\lstinline!using!~\lstinline!namespace!~\lstinline!literals::chrono_literals;!}
was added to the\linebreak[4]
 \lstinline!std::chrono! namespace. The Standard does not, however, benefit in any objective way from any of
these namespaces being \lstinline!inline! since the artifacts in the
\lstinline!literals! namespace neither depend on ADL nor are templates in
need of user-defined specializations; hence having all
non-\lstinline!inline! namespaces with appropriate \lstinline!using!
declarations would have been functionally indistinguishable from the
bifurcated approach taken.

\subsection[See Also]{See Also}\label{see-also}

\begin{itemize}
\item{%``\titleref{alignas}" on page~\pageref{alignas} — 
\seealsoref{alignas}{\seealsolocationc}Safe C++11 feature used in the example in  \textit{\titleref{use-cases-inlinenamespace}: \titleref{build-modes-and-abi-link-safety}} on page~\pageref{build-modes-and-abi-link-safety} to provide properly aligned storage for an object of arbitrary type \lstinline!T!.}
\end{itemize}

\subsection[Further Reading]{Further Reading}\label{further-reading}

TODO, TBD

\subsection[Appendix: Case study of using \lstinline!inline! namespaces for versioning]{Appendix: Case study of using {\SubsecCode inline} namespaces for versioning}\label{appendix:-case-study-of-using-inline-namespaces-for-versioning}

\noindent\textbf{By Niall Douglas}\\[.5ex]

\noindent Let me tell you what I (don't) use them for. It is not a conventional
opinion.

At a previous well-regarded company, they were shipping no less than
forty-three copies of Boost in their application. Boost was not on the
approved libraries list, but the great thing about header-only libraries
is that they don't obviously appear in final binaries, unless you look
for them. So each individual team was including bits of Boost quietly
and without telling their legal department. Why? Because it saved time.
(This was C++98, and \lstinline!boost::shared_ptr! and
\lstinline!boost::function! are both extremely attractive facilities).

Here's the really interesting part: Most of these copies of Boost were
not the same version. They were varying over a five-year release period.
And, unfortunately, Boost makes no API or ABI guarantees. So,
theoretically, you could get two different incompatible versions of
Boost appearing in the same program binary, and BOOM! there goes memory
corruption.

I advocated to Boost that a simple solution would be for Boost to wrap
up their implementation into an internal inline namespace. That inline
namespace ought to mean something:
\begin{itemize}
\item{\lstinline!lib::v1! is the \emph{stable}, version-1 ABI, which is guaranteed to be compatible with all past and future \lstinline!lib::v1! ABIs, forever, as determined by the ABI-compliance-check tool that runs on \romeogloss{CI}. The same goes for \lstinline!v2!, \lstinline!v3!, and so on.}
\item{\lstinline!lib::v2_a7fe42d! is the \emph{unstable}, version-2 ABI, which may be incompatible with any other \lstinline!lib::*!~\lstinline!ABI!; hence the seven hex chars after the underscore are the git short \romeogloss{SHA}, permuted by every commit to the git repository but, in practice, per CMake configure, because nobody wants to rebuild everything per commit. This ensures that no symbols from any revision of \lstinline!lib! will \emph{ever} silently collide or otherwise interfere with any other revision of \lstinline!lib!, when combined into a single binary by a dumb linker.}
\end{itemize}

I have been steadily making progress on getting Boost to avoid putting
anything in the global namespace, so a straightforward find-and-replace
can let you ``fix'' on a particular version of Boost.

That's all the same as the pitch for \lstinline!inline! namespaces. You'll
see the same technique used in \lstinline!libstdc++! and many other major modern C++
codebases.

But I'll tell you now, I don't use \lstinline!inline! namespaces any more.
Now what I do is use a macro defined to a uniquely named namespace. My
build system uses the git SHA to synthesize namespace macros for my
namespace name, beginning the namespace and ending the namespace.
Finally, in the documentation, I teach people to always use a namespace
alias to a macro to denote the namespace:

\begin{emcppslisting}
namespace output = OUTCOME_V2_NAMESPACE;
\end{emcppslisting} 

\noindent That macro expands to something like \lstinline!::outcome_v2_ee9abc2!,
that is, I don't use \lstinline!inline! namespaces any more.

Why?

Well, for \emph{existing} libraries that don't want to break backward
source compatibility, I think \lstinline!inline! namespaces serve a need.
For \emph{new} libraries, I think a macro-defined namespace is clearer.
\begin{itemize}
\item{It causes users to publicly commit to ``I know what you’re doing here, what it means, and what its consequences are.''}
\item{It declares to \emph{other} users that something unusual (i.e., go read the documentation) is happening here, instead of silent magic behind the scenes.}
\item{It prevents accidents that interfere with ADL and other customization points, which induce surprise, such as accidentally injecting a customization point into \lstinline!lib!, not into \lstinline!lib::v2!.}
\item{Using macros to denote namespace lets us reuse the preprocessor machinery to generate C++ modules using the exact same codebase; C++ modules are used if the compiler supports them, else we fall back to inclusion.}
\end{itemize}

Finally, and here's the real rub, because we now have namespace aliases,
if I were tempted to use an \lstinline!inline! namespace, nowadays I probably would
instead use a uniquely named namespace instead, and, in the \lstinline!include! file,
I'd alias a user-friendly name to that uniquely named namespace. I think
that approach is less likely to induce surprise in the typical
developer's likely use cases than \lstinline!inline! namespaces, such as
injecting customization points into the wrong namespace.

So now I hope you've got a good handle on \lstinline!inline! namespaces: I
was once keen on them, but after some years of experience, I've gone off
them in favor of better-in-my-opinion alternatives. Unfortunately, if your type \lstinline!x::S! has members of type
\lstinline!a::T! and macros decide if that is \lstinline!a::v1::T! or
\lstinline!a::v2::T!, then no linker protects the higher-level types from
ODR bugs, unless you also version \lstinline!x!.


