%%% 2 Feb 2021, copyedits in and proofed


\emcppsFeature{
    short={\lstinline!using! Aliases},
    tocshort={{\TOCCode using} Aliases},
    long={Type/Template Aliases (Extended {\SecCode typedef})},
    toclong={Type/Template Aliases (Extended \lstinline!typedef!)},
    rhshort={{\RHCode using} Aliases},
}{alias-declarations-and-alias-templates}
\setcounter{table}{0}
\setcounter{footnote}{0}
\setcounter{lstlisting}{0}
%\section[{\tt using} Aliases]{Type/Template Aliases (Extended {\SecCode typedef})}\label{alias-declarations-and-alias-templates}



%Alias declarations and alias templates provide an expanded use of the
%\lstinline!using! keyword, offering an alternative syntax (to
%\lstinline!typedef!) for creating a \romeogloss{type alias} that can itself be
%a template.

Alias declarations and alias templates provide an expanded use of the \lstinline!using! keyword to introduce aliases for types and templates, thus providing a more general alternative to \lstinline!typedef!.

\subsection[Description]{Description}\label{description}

The keyword \lstinline!using! has historically supported the introduction
of an alias for a named entity (e.g., type, function, or data) from some
named scope into the current one.
%see {\it\titleref{appendix:-brief-review-of-(c++03)-using-declarations}} on page~\pageref{appendix:-brief-review-of-(c++03)-using-declarations}.
%%%%% This appendix is removed and saved for a future edition. See file cpp03appendix.tex.
As of C++11, we can
employ the \lstinline!using! keyword to achieve everything that could
previously be accomplished with a \lstinline!typedef! declaration but in a
syntactic form that many people find more natural and intuitive (but
that offers nothing profoundly new):

\begin{emcppslisting}
using Type1 = int;     // equivalent to (ù{\codeincomments{typedef int Type1;}}ù)
using Type2 = double;  // equivalent to (ù{\codeincomments{typedef double Type2;}}ù)
\end{emcppslisting}

\noindent In contrast to \lstinline!typedef!, the name of the synonym created via the
\lstinline!using! syntax always appears on the left side of the \lstinline!=!
token and separate from the type declaration itself --- the advantage of
which becomes apparent with more involved types, such as
\emph{pointer-to-function}, \emph{pointer-to-member-function}, or
\emph{pointer-to-data-member}:

\begin{emcppslisting}
struct S { int i; void f(); };  // user-defined type (ù{\codeincomments{S}}ù) defined at file scope

using Type3 = void(*)();        // equivalent to (ù{\codeincomments{typedef void(*Type3)();}}ù)
using Type4 = void(S::*)();     // equivalent to (ù{\codeincomments{typedef void(S::*Type4)();}}ù)
using Type5 = int S::*;         // equivalent to (ù{\codeincomments{typedef int S::*Type5;}}ù)
\end{emcppslisting}

\noindent Just as with a \lstinline!typedef!, the name representing the type can be
qualified, but the symbol representing the synonym cannot:

\begin{emcppslisting}
namespace N { struct S { }; }  // original type (ù{\codeincomments{S}}ù) defined with namespace (ù{\codeincomments{N}}ù)

using Type6 = N::S;            // equivalent to (ù{\codeincomments{typedef N::S Type6;}}ù)
using ::Type7 = int;           // Error, the alias's name must be unqualified.
\end{emcppslisting}

\noindent Unlike a \lstinline!typedef!, however, a type alias introduced via
\lstinline!using! can itself be a template, known as an \emph{alias
template}:

\begin{emcppslisting}
template <typename T>
using Type8 = T;  // "identity" alias template

Type8<int>    i;  // equivalent to (ù{\codeincomments{int i;}}ù)
Type8<double> d;  // equivalent to (ù{\codeincomments{double d;}}ù)
\end{emcppslisting}

\noindent Note, however, that neither partial nor full specialization of alias
templates is supported:

\begin{emcppslisting}[emcppserrorlines={4,7}]
template <typename, typename>   // general alias template
using Type9 = char;             // OK

template <typename T>           // attempted partial specialization of above
using Type9<T, int> = char;     // Error, expected (ù{\codeincomments{=}}ù) before (ù{\codeincomments{<}}ù) token

template <>                     // attempted full specialization of above
using Type10<int, int> = char;  // Error, expected unqualified-id before (ù{\codeincomments{using}}ù)
\end{emcppslisting}

\noindent Used in conjunction with existing class templates, alias templates allow
programmers to \emph{bind} one or more template parameters to a fixed
type, while leaving others open:

\begin{emcppslisting}
#include <utility>  // (ù{\codeincomments{std::pair}}ù)

template <typename T>
using PairOfCharAnd = std::pair<char, T>;
    // alias template that binds (ù{\codeincomments{char}}ù) to the first type parameter of (ù{\codeincomments{std::pair}}ù)

PairOfCharAnd<int>    pci;  // equivalent to (ù{\codeincomments{std::pair<char, int> pci;}}ù)
PairOfCharAnd<double> pcd;  // equivalent to (ù{\codeincomments{std::pair<char, double> pcd;}}ù)
\end{emcppslisting}

\noindent Finally, note that similar functionality can be
achieved in C++03, it suppresses type deduction and requires additional boilerplate code at both the
point of definition and the call site:

\begin{emcppshiddenlisting}[emcppsbatch=e1]
#include <utility>  // (ù{\codeincomments{std::pair}}ù)
\end{emcppshiddenlisting}
\begin{emcppslisting}[emcppsbatch=e1]
// C++03 (obsolete)
template <typename T>
struct PairOfCharAnd
    // template class holding an alias, (ù{\codeincomments{Type}}ù), to (ù{\codeincomments{std::pair<char, T>}}ù)
{
    typedef std::pair<char, T> Type;
        // type alias binding (ù{\codeincomments{char}}ù) to the first type parameter of (ù{\codeincomments{std::pair}}ù)
};

PairOfCharAnd<int>::Type    pci;  // equivalent to (ù{\codeincomments{std::pair<char, int> pci;}}ù)
PairOfCharAnd<double>::Type pcd;  // equivalent to (ù{\codeincomments{std::pair<char, double> pcd;}}ù)
\end{emcppslisting}


\subsection[Use Cases]{Use Cases}\label{use-cases}

\subsubsection[Simplifying convoluted \lstinline!typedef! declarations]{Simplifying convoluted {\SubsubsecCode typedef} declarations}\label{simplifying-convoluted-typedef-declarations}

Complex \lstinline!typedef! declarations involving pointers to functions,
member functions, or data members require looking in the middle of the
declaration to find the alias name. As an example, consider a
\emph{callback} type alias intended to be used with asynchronous functions:

\begin{emcppslisting}
typedef void(*CompletionCallback)(void* userData);
\end{emcppslisting}

\noindent Developers coming from a background other than C or C++03 might find the
above declaration hard to parse since the name of the alias
(\lstinline!CompletionCallback!) is embedded in the function pointer type.
Replacing \lstinline!typedef! with \lstinline!using! results in a simpler,
more consistent formulation of the same alias:

\begin{emcppslisting}
using CompletionCallback = void(*)(void* userData);
\end{emcppslisting}

\noindent The \lstinline!CompletionCallback! alias declaration (above) reads almost
completely left-to-right, and the name of the alias is clearly
specified after the \lstinline!using! keyword. To make the\linebreak[4]
\lstinline!CompletionCallback! alias read left-to-right, a
  trailing return (see \featureref{\locationa}{trailing-function-return-types})
%  Section~\ref{trailing-function-return-types}, ``\titleref{trailing-function-return-types}")
can be used:

  \begin{emcppslisting}
  using CompletionCallback = auto(*)(void* userData) -> void;
  \end{emcppslisting}

\noindent The alias declaration above can be read as,
  ``\lstinline!CompletionCallback! is an alias for a pointer to a
  function taking a \lstinline!void*! parameter named \lstinline!userData! and
  returning \lstinline!void!.''

\subsubsection[Binding arguments to template parameters]{Binding arguments to template parameters}\label{binding-template-arguments}

An alias template can be used to \emph{bind} one or more template
parameters of, say, a commonly used class template, while leaving the
other parameters open to variation. Suppose, for example, we have a class, \lstinline!UserData!, that contains
several distinct instances of \lstinline!std::map! --- each
having the same key type, \lstinline!UserId!, but with different payloads:

\begin{emcppshiddenlisting}[emcppsbatch={e2,e3}]
#include <map>  // (ù{\codeincomments{std::map}}ù)
#include <set>  // (ù{\codeincomments{std::set}}ù)
struct UserId {};
struct Message {};
struct Photos {};
struct Article {};
\end{emcppshiddenlisting}
\begin{emcppslisting}[emcppsbatch=e2]
class UserData  // class having excessive code repetition (BAD IDEA)
{
private:
    std::map<UserId, Message>          d_messages;
    std::map<UserId, Photos>           d_photos;
    std::map<UserId, Article>          d_articles;
    std::map<UserId, std::set<UserId>> d_friends;
};
\end{emcppslisting}

\noindent The example above, though clear and regular, involves significant
repetition, making it more difficult to maintain should we later opt to
change data structures. If we were to instead use an \romeogloss{alias
template} to bind the \lstinline!UserId! type to the first type
parameter of \lstinline!std::map!, we could both reduce code repetition
and enable the programmer to consistently replace \lstinline!std::map!
to another container
(e.g., \lstinline!std::unordered_map!{\cprotect\footnote{An
\lstinline!std::unordered_map! is an STL container type that became
available on all conforming platforms along with C++11. The
functionality is similar except that since it is not required to
support ordered traversal or (worst case) \lstinline!O[log(n)]! lookups
and \lstinline!O[n*log(n)]! insertions, \lstinline!std::unordered_map! can
be implemented as a hash table instead of a balanced tree, yielding
significantly faster average access times. See
  \cite{cpprefb}.}}) by performing the change in only
one place:

\begin{emcppslisting}[emcppsbatch=e3]
class UserData  // class with well-factored implementation (GOOD IDEA)
{
private:
    template <typename V>                  // using a template alias to bind
    using Mapping = std::map<UserId, V>;   // (ù{\codeincomments{UserId}}ù) as the key type

    Mapping<Message>          d_messages;
    Mapping<Photos>           d_photos;
    Mapping<Article>          d_articles;
    Mapping<std::set<UserId>> d_friends;
};
\end{emcppslisting}


\subsubsection[Providing a shorthand notation for type traits]{Providing a shorthand notation for type traits}\label{providing-a-shorthand-notation-for-type-traits}

Alias templates can provide a shorthand notation for \romeogloss{type
traits}, avoiding \romeogloss{boilerplate code} in the usage site. As an
example, consider a simple type trait that adds a pointer to a given
type (akin to \lstinline!std::add_pointer!):

\begin{emcppslisting}[emcppsbatch=e4]
template <typename T>
struct AddPointer
{
    typedef T* Type;
};
\end{emcppslisting}

\noindent To use the trait above, the \lstinline!AddPointer! class template must be instantiated, and its nested
\lstinline!Type! alias must be accessed. Furthermore, in the generic context, it has to be prepended with the \lstinline!typename! keyword::

\begin{emcppslisting}[emcppsbatch=e4]
template <typename T>void f()
{
    T t;
    typename AddPointer<T>::Type p = t;
}
\end{emcppslisting}

\noindent The syntactical overhead of \lstinline!AddPointer! can be removed by
creating an alias template for its nested type alias, such as
\lstinline!AddPointer_t!:

\begin{emcppslisting}[emcppsbatch=e4]
template <typename T>
using AddPointer_t = typename AddPointer<T>::Type;
\end{emcppslisting}

\noindent Using \lstinline!AddPointer_t! instead of \lstinline!AddPointer! results in
shorter code devoid of boilerplate:

\begin{emcppslisting}[emcppsbatch=e4]
void g()
{
    int i;
    AddPointer_t<int> p = &i;
}
\end{emcppslisting}

\noindent Note that, since C++14, all
the standard type traits defined in the \lstinline!<type_traits>! header
provide a corresponding alias template with the goal of reducing
boilerplate code. For instance, C++14 introduces the
\lstinline!std::remove_reference_t! alias template for the C++11
\lstinline!std::remove_reference! type trait:

\begin{emcppshiddenlisting}[emcppsbatch=e5]
#include <type_traits>  // (ù{\codeincomments{std::remove\_reference}}ù)
\end{emcppshiddenlisting}
\begin{emcppslisting}[emcppsbatch=e5,emcppsstandards={c++14}]
typename std::remove_reference<int&>::type i0 = 5; // OK in both C++11 and C++14
std::remove_reference_t<int&> i1 = 5;              // OK in C++14
\end{emcppslisting}


\subsection[Potential Pitfalls]{Potential Pitfalls}\label{potential-pitfalls}

\hspace*{\fill}

\subsection[Annoyances]{Annoyances}\label{annoyances}

\hspace*{\fill}

\subsection[See Also]{See Also}\label{see-also}

\begin{itemize}
\item{\seealsoref{ctorinheriting}{\seealsolocationc}provides another meaning for the \lstinline!using! keyword to allow base-class constructors to be invoked as part of the derived class.}
\item{\seealsoref{trailing-function-return-types}{\seealsolocationa}provides an alternative syntax for function declaration, which can help improve readability in type aliases and alias templates involving function types.}
\end{itemize}

\subsection[Further Reading]{Further Reading}\label{further-reading}





