\emcppsFeature{
    toclong={The {\tt [[deprecated]]} Attribute},
    short={{\tt deprecated}},
    long={The {\SecCode [[deprecated]]} Attribute},
}{deprecated}
\label{the-standard-[[deprecated]]-attribute}

The \texttt{[[deprecated]]} attribute discourages the use of a decorated
\textbf{entity}, typically via the emission of a compiler warning.

\subsection[Description]{Description}\label{description}

The standard \texttt{[[deprecated]]} attribute is used to portably
indicate that a particular \textbf{entity} is no longer recommended and
to actively discourage its use. Such deprecation typically follows the
introduction of alternative constructs that, in (ideally) all ways, are
superior to the original one, providing time for clients to migrate to
them (\emph{asynchronously}{\cprotect\footnote{A process for ongoing
improvement of legacy codebases, sometimes known as
\textbf{continuous refactoring}, often allows time for clients to
migrate --- on their own respective schedules and time frames --- from
existing \emph{deprecated} constructs to newer ones, rather than
having every client change in lock step. Allowing clients time to move
\emph{asynchronously} to newer alternatives is often the only viable
approach unless (1) the codebase is a closed system, (2) all of the
relevant code is governed by a single authority, and (3) there is some
  sort of mechanical way to make the change.}}) before the deprecated
one is (in some subsequent release) removed. Although not strictly
required, the Standard explicitly encourages{\cprotect\footnote{The C++
Standard characterizes what constitutes a well-formed program, but
compiler vendors require a great deal of leeway to facilitate the
needs of their users. In case any feature induces warnings,
command-line options are typically available to disable those warnings
(\texttt{-Wno-deprecated} in GCC) or methods are in place to suppress those warnings
locally (e.g.,
  \texttt{\#pragma}~\texttt{GCC}~\texttt{diagnostic}~\texttt{ignored}~\texttt{"-Wdeprecated"}).}}
conforming compilers to produce a diagnostic message in case a program
refers to any \textbf{entity} to which the \texttt{[[deprecated]]}
attribute pertains. For instance, most popular compilers emit a
warning whenever a \texttt{[[deprecated]]} function or
object{\cprotect\footnote{The \texttt{[[deprecated]]} attribute can be
used portably to decorate other entities: \texttt{class},
\texttt{struct}, \texttt{union}, type alias, variable, data member,
function, enumeration, template specialization. Applying
\texttt{[[deprecated]]} to a specific enumerator or namespace,
however, is guaranteed to be supported only since C++17; see
  \textbf{smith15a} for more information.}} is used:

\begin{lstlisting}[language=C++]
               void f();
[[deprecated]] void g();

               int a;
[[deprecated]] int b;

void h()
{
    f();
    g();  // Warning: (ù{\codeincomments{g}}ù) is deprecated.
    a;
    b;    // Warning: (ù{\codeincomments{b}}ù) is deprecated.
}
\end{lstlisting}

\noindent A programmer can (optionally) supply a \textbf{string literal} as an
argument to the\linebreak[4] \mbox{\texttt{[[deprecated]]}} attribute (e.g.,
\mbox{\texttt{[[deprecated("message")]]}}) to inform human users regarding the
reason for the deprecation:

\begin{lstlisting}[language=C++]
[[deprecated("too slow, use (ù{\codeincomments{algo1}}ù) instead")]] void algo0();
                                                void algo1();

void f()
{
    algo0();  // Warning: (ù{\codeincomments{algo0}}ù) is deprecated; too slow, use (ù{\codeincomments{algo1}}ù) instead.
    algo1();
}
\end{lstlisting}

\noindent An \textbf{entity} that is initially \emph{declared} without
\texttt{[[deprecated]]} can later be redeclared with the attribute and
vice versa:

\begin{lstlisting}[language=C++]
void f();
void g0() { f(); }  // OK, likely no warnings

[[deprecated]] void f();
void g1() { f(); }  // Warning: (ù{\codeincomments{f}}ù) is deprecated.

void f();
void g2() { f(); }  // Warning: (ù{\codeincomments{f}}ù) is deprecated (still).
\end{lstlisting}

\noindent As seen in \texttt{g2} (above), redeclaring an \textbf{entity} that was
previously decorated with\linebreak[4] \mbox{\texttt{[[deprecated]]}} without the attribute
does not un-deprecate the entity.

\subsection[Use Cases]{Use Cases}\label{use-cases}

\subsubsection[Discouraging use of an obsolete or unsafe \textbf{entity}]{Discouraging use of an obsolete or unsafe \textbf{entity}}\label{discouraging-use-of-an-obsolete-or-unsafe-entity}

Decorating any \textbf{entity} with \texttt{[[deprecated]]} serves both
to indicate a particular feature should not be used in the future and
to actively encourage migration of existing uses to a better
alternative. Obsolescence, lack of safety, and poor performance are
common motivators for deprecation.

As an example of productive deprecation, consider the
\texttt{RandomGenerator} class having a static \texttt{nextRandom}
member function to generate random numbers:

\begin{lstlisting}[language=C++]
struct RandomGenerator
{
    static int nextRandom();
        // Generate a random value between 0 and 32767 (inclusive).
};
\end{lstlisting}

\noindent Although such a simple random number generator can be very useful, it
might become unsuitable for heavy use because good pseudorandom number
generation requires more state (and the overhead of synchronizing such
state for a single \texttt{static} function can be a significant
performance bottleneck) while good random number generation requires
potentially very high overhead access to external sources of
entropy.{\cprotect\footnote{The C Standard Library provides
\texttt{rand}, available in C++ through the \texttt{<cstdlib>} header.
It has similar issues to our \texttt{RandomGenerator::nextRandom}
function, and similarly developers are guided to use the facilities
  provided in the \texttt{<random>} header since C++11.}} One
solution is to provide an alternative random number generator that
maintains more state, allows users to decide where to store that state
(the random number generator objects), and overall offers more
flexibility for clients. The downside of such a change is that it comes
with a functionally distinct API, requiring that users update their code
to move away from the inferior solution:

\begin{lstlisting}[language=C++]
class BetterRandomGenerator
{
    // ... (internal state of a quality pseudorandom number generator) ...

public:
    int nextRandom();
       // Generate a quality random value between 0 and 32767 (inclusive).
};
\end{lstlisting}

\noindent Any user of the original random number generator can migrate to the new
facility with little effort, but that is not a completely trivial
operation, and migration will take some time before the original feature is no
longer in use. The empathic maintainers of \texttt{RandomGenerator} can
decide, instead of removing it completely, to use the
\texttt{[[deprecated]]} attribute to (gently) discourage continued use
of \texttt{RandomGenerator::nextRandom()}:

\begin{lstlisting}[language=C++]
struct RandomGenerator
{
    [[deprecated("Use (ù{\codeincomments{BetterRandomGenerator::nextRandom()}}ù) instead.")]]
    static int nextRandom();
        // ...
};
\end{lstlisting}

\noindent By using \texttt{[[deprecated]]} as shown above, existing clients of
\texttt{RandomGenerator} are informed that a superior
alternative, \texttt{BetterRandomGenerator}, is available, yet they are granted time to
migrate their code to the new solution (rather than their code being broken by the
removal of the old solution). When clients are notified of the
deprecation (thanks to a compiler diagnostic), they can schedule time to
(eventually) rewrite their applications to consume the new interface.{\cprotect\footnote{All joking aside, \textbf{continuous refactoring} is
an essential responsibility of a development organization, and
deciding when to go back and fix what's suboptimal instead of writing
new code that will please users and contribute more immediately to the
bottom line will forever be a source of tension. Allowing disparate
development teams to address such improvements in their own respective
time frames (perhaps subject to some reasonable overall deadline
date) is a proven real-world practical way of ameliorating this
  tension.}}

\subsection[Potential Pitfalls]{Potential Pitfalls}\label{potential-pitfalls}

\subsubsection[Interaction with {\tt -Werror} (e.g., GCC, Clang) or {\tt /WX} (MSVC)]{Interaction with {\SubsubsecCode -Werror} (e.g., GCC, Clang) or {\SubsubsecCode /WX} (MSVC)}\label{interaction-with--werror-(e.g,-gcc,-clang)-or-/wx-(msvc)}

To prevent warnings from being overlooked, the \texttt{-Werror} flag
(\texttt{/WX} on MSVC) is sometimes used, which promotes warnings to
errors. Consider the case where a project has been successfully using
\texttt{-Werror} for years, only to one day face an unexpected
compilation failure due to one of the project's dependencies using
\texttt{[[deprecated]]} as part of their API.

Having the compilation process completely stopped due to use of a
deprecated \textbf{entity} defeats the purpose of the attribute because users of such an \textbf{entity} are given no time to adapt their code
to use a newer alternative. On GCC and Clang, users can selectively demote deprecation errors back to warnings by using the
\texttt{-Wno-error=deprecated-declarations} compiler flag. On MSVC,
however, such demotion of warnings is not possible: The (unsatisfactory)
workarounds are to disable (entirely) either \texttt{/WX} or deprecation
diagnostics (using the \texttt{-wd4996} flag).

Furthermore, this interaction between \texttt{[[deprecated]]} and
\texttt{-Werror} makes it impossible for owners of a low-level library
to deprecate a function when releasing their code requires that they do
not break the ability for \textit{any} of their higher-level clients to compile;
a single client using the to-be-deprecated function along with
\texttt{-Werror} prevents the release of the code with the
\texttt{[[deprecated]]} attribute on it. With the default behaviors of
compilers and the frequent advice given in practice to use
\texttt{-Werror} aggressively, this can make any use of
\texttt{[[deprecated]]} completely unfeasible.

\subsection[Annoyances]{Annoyances}\label{annoyances}

None so far

\subsection[See Also]{See Also}\label{see-also}

None so far

\subsection[Further Reading]{Further Reading}\label{further-reading}

None so far



