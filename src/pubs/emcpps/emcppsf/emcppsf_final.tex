% 6 May 2021, cleaned up LaTeX and updated commands; to Josh for code check
% 7 May 2021 JMB, code compiles, comments addressed


\emcppsFeature{
    short={\lstinline!final!},
    tocshort={\TOCCode final},
    long={Preventing Overriding and Derivation},
    rhshort={\RHCode final},
}{final}
\label{preventing-overriding-and-derivation}
\setcounter{table}{0}
\setcounter{footnote}{0}
\setcounter{lstlisting}{0}
%\section[{\tt final}]{Preventing Overriding and Derivation}\label{final}
% \subsection[Preventing Overriding and Derivation]{Preventing Overriding and Derivation}\label{preventing-overriding-and-derivation}


The \lstinline!final! specifier can be used to disallow either (1)
overriding one or more virtual member functions from within derived
types or (2) deriving from a type altogether.

\subsection[Description]{Description}\label{description}

The ability to extend an arbitrary (\lstinline!class! or \lstinline!struct!)
user-defined type (\emcppsgloss{UDT}) via inheritance and then to override
any virtual functions declared therein is a hallmark of the C++
object-oriented model. There may, however, be cases where the author of
such a \emcppsgloss{UDT} will find a legitimate need to intentionally
restrict clients' abilities in this regard. The \lstinline!final! specifier
serves such a purpose.

When applied to a virtual-function declaration, \lstinline!final! prevents
derived-class authors from overriding that specific function. When used
on a virtual function, \lstinline!final! is syntactically similar to the
override specifier (see \featureref{\locationa}{override}) but
with different semantics. Separately, \lstinline!final! can be applied to
the declaration of a user-defined type as a whole, thereby preventing
prospective clients from deriving from it. Finally, note that
\lstinline!final!, like \lstinline!override!, is only a \emcppsgloss{contextual
keyword} --- i.e., an identifier with special meaning --- and can still
be used as a C++ identifier wherever other identifiers can be
syntactically used, e.g., to name types, objects, functions, and so on:

\begin{emcppshiddenlisting}
// This is the below example with all of the latex markup removed.
struct final final                             // struct named "final"
{
    final() = default;                         // default constructor
    virtual ~final() final;                    // `final` destructor
};          
struct S1 { ::final *final; };                 // data member named "final"
struct S2 { virtual ::final final() final; };  // function named "final"
final final;                                   // object named "final" 
\end{emcppshiddenlisting}
\begin{emcppslisting}[emcppsignore={too much markup, checked above}]
struct (ù{\texttt{final}}ù) final                             // struct named "final"
{
    (ù{\texttt{final}}ù)() = default;                         // default constructor
    virtual ~(ù{\texttt{final}}ù)() final;                    // (ù{\codeincomments{final}}ù) destructor
};          
struct S1 { ::(ù{\texttt{final}}ù) *(ù{\texttt{final}}ù); };                 // data member named "final"
struct S2 { virtual ::(ù{\texttt{final}}ù) (ù{\texttt{final}}ù)() final; };  // function named "final"
(ù{\texttt{final}}ù) (ù{\texttt{final}}ù);                                   // object named "final" 
\end{emcppslisting}
    

\noindent The example above is syntactically legal but not recommended. The
example defines a number of different entities named \texttt{final}, %NOTE: not lstlinline, as this is an identifier being talked about, not the keyword
many of mhich have the \lstinline!final! specifier attached to them.

\subsubsection[\lstinline!final! virtual member functions]{{\SubsubsecCode final} virtual member functions}\label{final-virtual-member-functions}

When applied to a \lstinline!virtual!-function declaration, the
\lstinline!final! specifier prevents derived-class authors from
\emph{overriding} that function, i.e., from within a derived class:

\begin{emcppslisting}[emcppsbatch=e1]
struct B0  // Each function in (ù{\codeincomments{B0}}ù) is explicitly declared (ù{\codeincomments{virtual}}ù).
{
    virtual void f();
    virtual void g() final;  // prevents overriding in derived classes
    virtual void g() const;
};

struct D0 : B0       // (ù{\codeincomments{D0}}ù) inherits (publicly) from (ù{\codeincomments{B0}}ù).
{
    void f();        // OK, explicitly overrides (ù{\codeincomments{void B0::f()}}ù)
    void g();        // Error, (ù{\codeincomments{void B0::g()}}ù) is (ù{\codeincomments{final}}ù).
    void g() const;  // OK, (ù{\codeincomments{void B0::g() const}}ù) is (ù{\emphincomments{not}}ù) (ù{\codeincomments{final}}ù).
};
\end{emcppslisting}
    

\noindent As the simple example above illustrates, decorating a virtual member
function --- e.g., \lstinline!B::g()! --- with \lstinline!final! precludes
overriding only that specific function signature. Note that when
redeclaring a \lstinline!final! function (e.g., to define it), the
\lstinline!final! specifier is not permitted:

\begin{emcppslisting}[emcppsbatch=e1]
void B0::g() final { }  // Error, (ù{\codeincomments{final}}ù) not permitted outside class definition
void B0::g() { }        // OK
\end{emcppslisting}
    

\subsubsection[\lstinline!final! on destructors]{{\SubsubsecCode final} on destructors}\label{final-on-destructors}

The use of \lstinline!final! on a virtual destructor precludes inheritance
entirely, as any derived class must have either an implicit or explicit
destructor, which will attempt to override the \lstinline!final! base class
destructor:

\begin{emcppslisting}[emcppsbatch=e1,emcppserrorlines={8,9,11}]
struct B1
{
    virtual ~B1() final;
};

struct D1a : B1 { };    // Error, implicitly tries to override (ù{\codeincomments{B1::\texttt{\~{}}B1()}}ù)

struct D1b : B1
{
    virtual ~D1b() { }  // Error, explicitly tries to override (ù{\codeincomments{B1::\texttt{\~{}}B1()}}ù)
};
\end{emcppslisting}
    

\noindent Any attempt to suppress the destructor in the derived class, e.g., using
\lstinline!=!\,\lstinline!delete! (see {\featureref{\locationa}{delete}), will
be in vain. If the intent is to suppress derivation entirely, a more
expressive way would be to declare the type itself \lstinline!final!; see
{\featureref{\locatione}{final}.

\subsubsection[\lstinline!final! pure virtual functions]{{\SubsubsecCode final} pure virtual functions}\label{final-pure-virtual-functions}

Although declaring a \emcppsgloss{pure virtual function} \lstinline!final! is
valid, doing so makes the type an \emcppsgloss{abstract class} and also
prevents making any derived type a \emcppsgloss{concrete class}:

\begin{emcppslisting}[emcppsbatch=e1]
struct B2  // abstract class
{
    virtual void f() final = 0;  // OK, but probably not useful
};

B2 b;  // Error, (ù{\codeincomments{B2}}ù) is an abstract type

struct D2a : B2  // also an abstract class
{
};

D2a d;  // Error, (ù{\codeincomments{D2a}}ù) is an abstract type.

struct D2b : B2
{
    void f() {};  // Error, (ù{\codeincomments{void B2::f()}}ù) is (ù{\codeincomments{final}}ù).
};
\end{emcppslisting}
    

\noindent By declaring the pure \lstinline!virtual! member function, \lstinline!B2::f()!
in the example above, to be \lstinline!final!, we have effectively
precluded ever extending the uninstantiable \emcppsgloss{abstract class},
\lstinline!B2!, to \emph{any} instantiable \emcppsgloss{concrete class}.

\subsubsection[\lstinline!final! and its interactions with \lstinline!virtual! and \lstinline!override!]{{\SubsubsecCode final} and its interactions with {\SubsubsecCode virtual} and {\SubsubsecCode override}}\label{final-and-its-interactions-with-virtual-and-override}

In contrast, when we apply \lstinline!final! to non\lstinline!virtual!
functions, the \lstinline!final! specifier will always force a compilation
error:

\begin{emcppslisting}[emcppsbatch=e1]
struct B3a
{
    void f() final;  // Error, (ù{\codeincomments{f}}ù) is not (ù{\codeincomments{virtual}}ù).
};

struct B3b
{
    void g(); // OK, (ù{\codeincomments{g}}ù) is (ù{\emphincomments{not}}ù) virtual.
};

struct D3 : B3b
{
    void g() final; // Error, (ù{\codeincomments{g}}ù) is not (ù{\codeincomments{virtual}}ù) and hides (ù{\codeincomments{B3b::g}}ù).
};
\end{emcppslisting}
    

\noindent The \lstinline!final! keyword combines with the \lstinline!virtual! and
\lstinline!override! keywords to produce various effects. For example,
functions that are declared virtual in a base class, e.g., all of the
functions below in \lstinline!B4!, are automatically considered virtual by
a function having the same signature in a derived class, e.g., the
corresponding functions in \lstinline!D4!, irrespective of whether the
\lstinline!virtual! keyword is repeated:

\begin{emcppslisting}[emcppsbatch=e1]
struct B4  // Each of the functions in (ù{\codeincomments{B4}}ù) is explicitly declared virtual.
{
    virtual void f();  // explicitly declared virtual
    virtual void g();  //      "        "       "
    virtual void h();  //      "        "       "
};

struct D4 : B4  // Each of the functions in (ù{\codeincomments{D4}}ù) is explicitly declared (ù{\codeincomments{final}}ù).
{
    void f() final;          // OK, because (ù{\codeincomments{B4::f}}ù) is declared virtual
    virtual void g() final;  // OK, explicitly declared virtual (no effect)
    void h() final override; // OK, because (ù{\codeincomments{B4::h}}ù) is declared virtual
    virtual void i() final;  // OK, explicitly declared virtual (necessary)
    void j() final;          // Error, nonvirtual function (ù{\codeincomments{j}}ù) declared (ù{\codeincomments{final}}ù)
};
\end{emcppslisting}
    

\noindent Notice that \lstinline!D4::g()! is annotated with the keyword
\lstinline!virtual! but \lstinline!D4::f()! is not. Adding \lstinline!virtual! to
the declaration of a function in a derived class when there is a
matching function declared \lstinline!virtual! in a base class has no
effect. In a manner similar to \lstinline!override!, leaving off the
explicit \lstinline!virtual! will prevent removing the base class function
or altering its signature.

Given the availability of the override specifier (see
\featureref{\locationa}{override}), however, a common coding
standard has emerged: Use \lstinline!virtual! only to \emph{introduce} a
\lstinline!virtual! function in a class hierarchy and then \emph{require}
that any functions attempting to \emph{override} a \lstinline!virtual!
function in a derived class be decorated with \emph{either}
\lstinline!override! or \lstinline!final! and explicitly \emph{not}
\lstinline!virtual!:

\begin{emcppslisting}[emcppsbatch=e1]
struct B5  // base class consisting of virtual and nonvirtual functions
{
    virtual void f1();  // OK, (ù{\codeincomments{virtual}}ù) function
    virtual void f2();  // OK, (ù{\codeincomments{virtual}}ù) function
            void g1();  // OK, non(ù{\codeincomments{virtual}}ù) function
            void g2();  // OK, non(ù{\codeincomments{virtual}}ù) function
};

struct D5  // "derived class" attempting to override virtual functions (ù{\codeincomments{f1}}ù) and (ù{\codeincomments{f2}}ù)
{
    void f1() override;  // Error, (ù{\codeincomments{f1}}ù) marked (ù{\codeincomments{override}}ù) but doesn't override.
    void f2() final;     // Error, (ù{\codeincomments{f2}}ù) marked (ù{\codeincomments{final}}ù) but isn't virtual
};
\end{emcppslisting}
    

\noindent The astute reader will have already noticed that in the example above we
failed to make \lstinline!D5! inherit publicly from \lstinline!B5!. Catching
this flaw early is one of the awesome benefits of always following this
convention. If we try again, this time with \lstinline!D5b!, we will
observe a second benefit of always supplying either \lstinline!override! or
\lstinline!final!:

\begin{emcppslisting}[emcppsbatch=e1]
struct D5b : B5  // This time we remembered to inherit from (ù{\codeincomments{B5}}ù).
{
    void f1() override;  // OK
    void f2() final;     // OK
    void g1() override;  // Error, (ù{\codeincomments{g1}}ù) marked (ù{\codeincomments{override}}ù) but doesn't override
    void g2() final;     // Error, (ù{\codeincomments{g2}}ù) marked (ù{\codeincomments{final}}ù) but isn't virtual
};
\end{emcppslisting}
    

\noindent Finally, one could imagine deliberately declaring a base-class function
to be both \lstinline!virtual! and \lstinline!final! --- e.g., just to prevent
a derived class from hiding it:

\begin{emcppslisting}[emcppsbatch=e1]
struct B6
{
    virtual void f() final;  // OK
};

static_assert(sizeof(B6) == 1, "");  // Error, (ù{\codeincomments{B6}}ù) holds a vtable pointer.

struct D6a : B6
{
    void f() const;     // OK, (ù{\codeincomments{D6a::f}}ù) doesn't override.
    void f();           // Error, (ù{\codeincomments{B6::f}}ù) is (ù{\codeincomments{final}}ù).
};

struct D6b : B6
{
    void f(int i = 0);  // OK, even though it hides (ù{\codeincomments{B6::f}}ù)
};
\end{emcppslisting}
    

\noindent Declaring, for the first time, the member function \lstinline!B6::f! in the
example above to be both \lstinline!virtual! and \lstinline!final! has limited
practical effect. Attempts to hide \lstinline!f! in a subclass will be
blocked but only when the hiding function has \emph{exactly} the same
signature; a function that hides \lstinline!f! can still be written with
different member function qualifiers or even slightly different,
possibly optional parameters. Adding \lstinline!virtual! to \lstinline!f! also
makes \lstinline!B6! a \emcppsgloss{polymorphic type}, bringing with it the need
for a \emcppsgloss{vtable pointer} in every object and making it
non-\emcppsgloss{trivial}; see \featureref{\locationc}{gpods}. Even
though the compiler will likely be able to \emcppsgloss{devirtualize} calls
to \lstinline!B6::f!, that it will not is still possible, and there will be
the additional overhead of \emcppsgloss{dynamic dispatch} when invoking calls
to \lstinline!B6::f!; see \intraref{potential-pitfalls-final}{attempting-to-defend-against-the-hiding-of-a-nonvirtual-function}.

\subsubsection[\lstinline!final! user-defined types]{{\SubsubsecCode final} user-defined types}\label{final-user-defined-types}

The use of \lstinline!final! is not limited to individual member functions
and can also be applied to an entire user-defined type to explicitly
disallow \emph{any} other type from inheriting from it. Preventing a
type from being inheritable closes the gap between what is possible with
built-in types, such as \lstinline!int! and \lstinline!double!, and what can
additionally be done with typical user-defined ones --- specifically,
inherit from them. See \intraref{use-cases-final}{suppressing-derivation-to-ensure-portability}, i.e., Hyrum's Law.

Although other uses are plausible, widespread use can run afoul of
stable \emcppsgloss{reuse} in general and \emcppsgloss{hierarchical reuse} in
particular; see \intraref{potential-pitfalls-final}{systemic-lost-opportunities-for-reuse}. Hence the decision
to use \lstinline!final! on an entire class even rarely --- let alone
routinely --- is not to be taken lightly.

Prior to C++11's introduction of the \lstinline!final! specifier, there was
no convenient way to ensure that a user-defined type (UDT) was
\emph{uninheritable}, although some byzantine idioms to approximate this
existed. For example, a virtual base class needs to be initialized in
each constructor of all concrete derived types, and that can be levered
to prevent useful inheritance. Consider a trio of classes, the first of
which, \lstinline!UninheritableGuard!, has a private constructor and
befriends its only intended derived class; the second,
\lstinline!Uninheritable!, derives privately and virtually from
\lstinline!UninheritableGuard!; and the third, \lstinline!Inheriting!, is a
misguided class that tries in vain to inherit from
\lstinline!Uninheritable!:

\begin{emcppslisting}[emcppserrorlines={13,14,15,16,17,18}]
struct UninheritableGuard  // private, virtual base class
{
private:
    UninheritableGuard();         // private constructor
    friend struct Uninheritable;  // constructible only by (ù{\codeincomments{Uninheritable}}ù)
};

struct Uninheritable : private virtual UninheritableGuard
{
    Uninheritable() : UninheritableGuard() { /* ... */ }
};

struct Inheriting : Uninheritable  // (ù{\codeincomments{Uninheritable}}ù) is effectively (ù{\codeincomments{final}}ù).
{
    Inheriting()
    : UninheritableGuard()  // Error, (ù{\codeincomments{UninheritableGuard()}}ù) is inaccessible.
    { /* ... */ }
};
\end{emcppslisting}
    

\noindent Any attempt to define --- either implicitly or explicitly --- a
constructor for \lstinline!Inheriting! will fail with the same error due to
the inaccessibility of the constructor for \lstinline!UninheritableGuard!.
Using \lstinline!virtual! inheritance typically requires each object of
type \lstinline!Uninheritable! to maintain a virtual table pointer; hence,
this solution does not come without overhead. Note also that this
workaround prior to \lstinline!final! does not prevent the derivation
itself, but merely the instantiation of any ill-fated derived classes.

In the special case where all of the data members of the type are
\emcppsgloss{trivial}, i.e., have no user-provided \emcppsgloss{special member
functions} (see \featureref{\locationc}{gpods}), we could have
instead created a type, e.g., \lstinline!Uninheritable2!, that is
implemented as a \lstinline!union! consisting of just a single
\lstinline!struct!:

\begin{emcppslisting}
union Uninheritable2  // C++ does not yet permit inheritance from (ù{\codeincomments{union}}ù) types.
{
    struct  // anonymous class type
    {
        int    i;
        double d;
    } s;

    Uninheritable2()
    {
       s.i = 0;
       s.d = 0;
    }
};

struct S : Uninheritable2 { };  // Error, unions cannot be base classes.
\end{emcppslisting}
    

\noindent With the introduction of the \lstinline!final! specifier, no such
contortions are needed. When added to the \emph{definition} of either a
\lstinline!class! or \lstinline!struct!, the \lstinline!final! specifier adeptly
prevents prospective clients from deriving from that type:

\begin{emcppslisting}
struct S1 { };        // non(ù{\codeincomments{final}}ù) user-defined type
struct S2 final { };  // (ù{\codeincomments{final}}ù) user-defined type

struct D1 : S1 { };   // OK, S1 is not declared (ù{\codeincomments{final}}ù).
struct D2 : S2 { };   // Error, S2 is declared (ù{\codeincomments{final}}ù).
\end{emcppslisting}
    

\noindent The \lstinline!final! specifier may be applied to a type's declaration as
long as that declaration is a definition:

\begin{emcppslisting}
class C1;            // OK,    (ù{\codeincomments{C1}}ù) is (as of now) an incomplete type.
class C1 final;      // Error, attempt to declare variable of incomplete type
class C1 final { };  // OK, (ù{\codeincomments{C1}}ù) is now a complete type.
class C1;            // OK, (ù{\codeincomments{C1}}ù) is known to be a (ù{\codeincomments{final}}ù) type.
\end{emcppslisting}
    

\noindent Once a type is complete, an attempt to redeclare the type using the
final specifier will be a valid declaration of an object of that type
named \lstinline!final!. This also means that once a type has been defined
as \emph{non}\/\lstinline!final!, it cannot subsequently be redeclared as
\lstinline!final!:

\begin{emcppslisting}
class C2 { };        // OK, (ù{\codeincomments{C2}}ù) is a non(ù{\codeincomments{final}}ù) (ù{\emphincomments{complete}}ù) type.
class C2 final;      // Bug??, (ù{\codeincomments{C2}}ù) object named (ù{\codeincomments{final}}ù).
\end{emcppslisting}
    

\noindent As the final line above illustrates, \lstinline!final!, being a contextual
keyword, is a valid name for a object of non\lstinline!final! class
\lstinline!C2!; see \intraref{potential-pitfalls-final}{contextual-keywords-are-contextual}.

\pagebreak%%%%%  to keep fn on one page so reviewers don't freak out

The \lstinline!final! specifier, in addition to decorating a \lstinline!class!
or \lstinline!struct!, can also be applied to a \lstinline!union!. Since
derivation from a \lstinline!union! isn't permitted anyway, declaring a
union to be \lstinline!final! has no effect.{\cprotect\footnote{In C++14,
the Standard Library header \lstinline!<type_traits>! defines the trait
\lstinline!std::is_final!, which can be used to determine whether a
given user-defined, class type is specified as \lstinline!final!. This
trait can be used to distinguish between a non\lstinline!final! and
\lstinline!final!~\lstinline!union!. The argument to this trait, however,
  must be a complete type.}}

\subsection[Use Cases]{Use Cases}\label{use-cases-final}

\subsubsection[Suppressing derivation to ensure portability]{Suppressing derivation to ensure portability}\label{suppressing-derivation-to-ensure-portability}

A rare but compelling use of \lstinline!final! for a user-defined type as a
whole occurs when that type is used to simulate a feature that is or
might some day be implemented as a \emcppsgloss{fundamental type} on this or
some other platform. By specifying the user-defined type to be
\lstinline!final!, we avoid locking ourselves into forever being
represented as a user-defined type.

As a concrete and very real-world example, consider an early
implementation of a family of floating-point decimal types:
\lstinline!Decimal32_t!, \lstinline!Decimal64_t!, and
\lstinline!Decimal128_t!. Initially these types (as type aliases) are
implemented in terms of user-defined types:

\begin{emcppslisting}[emcppsbatch=e2]
class DecimalFloatingPoint32  { /* ... */ };
class DecimalFloatingPoint64  { /* ... */ };
class DecimalFloatingPoint128 { /* ... */ };

typedef DecimalFloatingPoint32  Decimal32_t;
typedef DecimalFloatingPoint64  Decimal64_t;
typedef DecimalFloatingPoint128 Decimal128_t;
\end{emcppslisting}
    

\noindent Once released in this form, there is nothing to stop users from
inheriting from these \lstinline!typedef!s to derive their own custom
types, and, in fact, \emcppsgloss[Hyrum's law]{Hyrum's Law} all but guarantees that such
inheritance will happen:

\begin{emcppslisting}[emcppsbatch=e2]
class MyDecimal32     : public Decimal32_t     { /* ... */ };
class YourDecimal64   : private Decimal64_t    { /* ... */ };
class TheirDecimal128 : protected Decimal128_t { /* ... */ };
\end{emcppslisting}
    

\noindent Hardware support will someday arrive --- and on some platforms it
already has --- that will likely allow these \lstinline!typedef!s to become
aliases to fundamental types rather than aliases to user-defined types:

\begin{emcppshiddenlisting}[emcppsbatch=e3]
typedef int __decimal32_t;
typedef int __decimal64_t;
typedef int __decimal128_t;
\end{emcppshiddenlisting}
\begin{emcppslisting}[emcppsbatch=e3]
typedef __decimal32_t  Decimal32_t;
typedef __decimal64_t  Decimal64_t;
typedef __decimal128_t Decimal128_t;
\end{emcppslisting}
    

\noindent When that day comes and we try to leverage the possibly huge performance
benefits of using a natively supported type, a small number of bad
actors amongst clients of this library will prevent the majority of
users from benefitting from the new platform. The end result is that
person years of effort might be required to unravel the code that now
depends on inheritance. This effort is a price that could have been
avoided had we just declared our \emph{not necessarily} user-defined
types \lstinline!final! from the start:

\begin{emcppslisting}
class DecimalFloatingPoint32 final  { /* ... */ };
class DecimalFloatingPoint64 final  { /* ... */ };
class DecimalFloatingPoint128 final { /* ... */ };
\end{emcppslisting}
    

\noindent Using \lstinline!final! here would have avoided advertising more than we
were prepared to support in perpetuity.

\subsubsection[Improving performance of concrete classes]{Improving performance of concrete classes}\label{improving-performance-of-concrete-classes}

Object-Oriented programming (OOP) comprises two important aspects of
software design: (1) inheritance and (2) \emcppsgloss{dynamic binding}. For a
language to claim that it supports OOP, it must support
both.{\cprotect\footnote{\cite{stroustrup91}, section~0.x, pp.~y--z}} No
overhead is associated with inheritance \emph{per se}, but the same
cannot be said in general for dynamic binding (a.k.a., \emcppsgloss{virtual
dispatch}).

Without commenting on the wisdom of an implementation consisting of a
mixture of \emph{interface}, \emph{implementation}, and
\emph{structural} inheritance, consider the classic \lstinline!Shape!,
\lstinline!Circle!, and \lstinline!Rectangle! example illustrating
\emcppsgloss{object orientation}:

\begin{emcppslisting}[emcppsbatch=e4]
class Shape  // abstract base class
{
    int d_x;  // x-coordinate of origin
    int d_y;  // y-coordinate of origin

public:
    Shape(int x, int y) : d_x(x), d_y(y) { }    // value constructor

    void move(int dx, int dy) { d_x += dx; d_y += dy; }
                                                // concrete manipulator

    int xOrigin() const { return d_x; }         // concrete accessor
    int yOrigin() const { return d_y; }         // concrete accessor

    virtual double area() const = 0;            // (ù{\emphincomments{abstract}}ù) accessor
};

class Circle : public Shape  // concrete derived class
{
    int d_radius;  // radius of this circle

public:
    Circle(int x, int y, int radius) : Shape(x, y), d_radius(radius) { }
                                                // value constructor

    double area() const { return 3.14 * d_radius; }
                                                // (ù{\emphincomments{concrete}}ù) accessor
};

class Rectangle: public Shape   // concrete derived class
{
    int d_length;  // length of this rectangle
    int d_width;   // width of this rectangle

public:
    Rectangle(int x, int y, int length, int width) :
                       Shape(x, y), d_length(length), d_width(width) { }
                                                // value constructor

    double area() const final { return d_length * d_width; }
                                                // (ù{\emphincomments{concrete}}ù) accessor
};
\end{emcppslisting}
    

\noindent Note that \lstinline!Shape! acts as both an interface and as a base class
offering a proper value constructor. Inheriting classes have to override
\lstinline!Shape::area! to provide a \emph{concrete accessor} for their
respective area values. The two classes inheriting from \lstinline!Shape!,
namely \lstinline!Circle! and \lstinline!Rectangle!, differ substantially in
just one aspect: \lstinline!Rectangle::area()! is annotated with
\lstinline!final! whereas \lstinline!Circle::area()! is not.

Now imagine a client accessing the \emph{concrete implementations} of
\lstinline!Shape::area! through a base-class reference:

\begin{emcppslisting}[emcppsbatch=e4]
void client1(const Shape& shape)
{
    int x = shape.xOrigin();     // inlines (e.g., Clang -O2, GCC -O1)
    double area = shape.area();  // inline requires whole program optimization
}
\end{emcppslisting}
    

\noindent The call to \lstinline!Shape::xOrigin! can be inlined at a fairly low level
of optimization whereas the call to \lstinline!Shape::area! cannot because
it is subject to \emph{virtual dispatch}. The same ability to inline the
call to get the \lstinline!xOrigin! applies for objects of either derived
type --- \lstinline!Circle! as well as \lstinline!Rectangle! --- irrespective
of whether the type's implementation of \lstinline!Shape::area! has been
annotated with \lstinline!final!.

If the compiler can somehow locally infer the runtime type of the
derived object, the function calls can be inlined at a fairly low
optimization level too, again irrespective of any annotation with
\lstinline!final!:

\begin{emcppslisting}[emcppsbatch=e4]
void client2()
{
    Circle c(3, 2, 1);
    Rectangle r(4, 3, 2, 1);
    const Shape& s1 = c;
    const Shape& s2 = r;

    double cArea = c.area();  // inlines (e.g., Clang -O2, GCC -O1)
    double rArea = r.area();  // inlines (e.g., Clang -O2, GCC -O1)

    double s1Area = s1.area();  // inlines (e.g., Clang -O2, GCC -O1)
    double s2Area = s2.area();  // inlines (e.g., Clang -O2, GCC -O1)
}
\end{emcppslisting}
    

\noindent The real difference comes to light when accepting references of these
types in a separate function. As only \lstinline!Rectangle! prohibits
further overriding of the \lstinline!area! function, it is the only one of
the three types that can know at compile time the runtime type of its
object and therefore bypass virtual dispatch:

\begin{emcppslisting}[emcppsbatch=e4]
void client3(const Shape& s, const Circle& c, const Rectangle& r)
{
    double sArea = s.area();  // must undergo virtual dispatch
    double cArea = c.area();  // must undergo virtual dispatch
    double rArea = r.area();  // inlines (Clang -O2, GCC -O1)
}
\end{emcppslisting}
    

\noindent Note that a compilation system that is aware of the entirety of a
program might be able to turn that into knowledge that \lstinline!Circle!
is effectively final --- e.g., by observing that no other classes derive
from it --- and thus similarly bypass virtual dispatch in
\lstinline!client3!. Such optimization, however, is expensive, doesn't
scale, and would be thwarted by common practices, such as the
possibility of loading shared objects at run time, which might contain
classes derived from \lstinline!Circle! that are then passed to
\lstinline!client3!.

\subsubsection[Restoring performance lost to mocking]{Restoring performance lost to mocking}\label{restoring-performance-lost-to-mocking}

In most cases, a component cannot be fully isolated from its
dependencies in order to be able to test it in complete isolation. An
object-oriented approach widely used to artificially circumvent such
limitations is often colloquially referred to as \emcppsgloss{mocking}.

As an example, consider a custom file-handling class, \lstinline!File!,
that depends on a custom local file system, \lstinline!LocalFS!:

\begin{emcppshiddenlisting}[emcppsbatch={e5,e6}]
class FileHandle;
\end{emcppshiddenlisting}
\begin{emcppslisting}[emcppsbatch=e5]
class LocalFS  // lower-level file system
{
public:
    FileHandle* open(const char* path, int* errorStatus = 0);
};

class File  // higher-level file representation
{
    LocalFS* d_lfs;  // pointer to concrete filesystem object

public:
    File(LocalFS* lfs) : d_lfs(lfs) { }

    int open(const char *path);
    // ...
};
\end{emcppslisting}
    

\noindent In the testing-resistant design suggested above, class \lstinline!File!
depends directly on a local file system; hence, there is no convenient
way for us to test what happens when rarely seen (e.g., error) events
occur within the file system.

To be able to test our \lstinline!File! class thoroughly, we will need some
way to control what the \lstinline!File! class does when it receives input
back from what it thinks is the local file system. One approach would be
to extract a pure abstract (a.k.a. \emcppsgloss{protocol}) class, e.g.,
\lstinline!AbstractFileSystem!, from \lstinline!LocalFS!. We can then make our
\lstinline!File! class depend on the protocol instead of \lstinline!LocalFS!
and make \lstinline!LocalFS! implement it:

\begin{emcppslisting}[emcppsbatch=e6]
class AbstractFileSystem  // lower-level pure abstract interface
{
public:
    virtual FileHandle *open(const char *path, int *errorStatus = 0) = 0;
};


class File  // uses lower-level abstract file system
{
    AbstractFileSystem *d_afs;  // pointer to abstract filesystem object

public:
    File(AbstractFileSystem *afs) : d_afs(afs) { }

    int open(const char *path);
    // ...
};

class LocalFS : public AbstractFileSystem  // implements abstract interface
{
public:
    FileHandle *open(const char *path, int *errorStatus = 0);
};
\end{emcppslisting}
    

\noindent With this artificially complicated design, we are now able to create an
independent concrete, ``mock'' implementation derived from
\lstinline!AbstractFileSystem!, which we can then use to orchestrate
arbitrary behavior, thereby enabling us to test \lstinline!File! both (1)
in isolation and (2) under abnormal circumstances:

\begin{emcppshiddenlisting}[emcppsbatch=e6]
// --- Replace
    FileHandle *open(const char *path, int *errorStatus = 0) { /* mocked */ }
    FileHandle *open(const char *path, int *errorStatus = 0) { return nullptr; }
// --- End
\end{emcppshiddenlisting}
\begin{emcppslisting}[emcppsbatch=e6]
class MockedFS : public AbstractFileSystem  // test-engineer-controllable class
{
public:
    FileHandle *open(const char *path, int *errorStatus = 0) { /* mocked */ }
};

void test()  // test driver that orchestrates mock implementation to test (ù{\codeincomments{File}}ù)
{
    MockedFS mfs;  // mock AbstractFileSystem used to thoroughly test (ù{\codeincomments{File}}ù)
    File f(&mfs);  // (ù{\codeincomments{f}}ù) installed with (ù{\emphincomments{mock}}ù) instead of actual (ù{\codeincomments{LocalFS}}ù)

    int rc = f.open("dummyPath");  // (ù{\emphincomments{mock}}ù) used to supply handle
    // ...
}
\end{emcppslisting}
    

\noindent Although this technique does enable independent testing, it comes with
the performance cost of dynamic dispatch on all calls to the underlying
\lstinline!AbstractFileSystem! object. In certain situations where the
concrete implementation doesn't have to be exchangeable for the mock,
performance can be recovered by declaring some or all of the concrete,
now-virtual functions of \lstinline!localFS!~\lstinline!final!. This way, when
we pass the concrete implementation --- e.g., \lstinline!LocalFS! --- to a
function explicitly, at least the functions that are declared
\lstinline!final! and were previously inline can again be inline:

\begin{lstlisting}[language=bash,style=plain]
client          Client  MockMechanism   Mechanism   (final virtual functions)
   o                 o        |          /
   |       =>         \       |         /
   |                   \      V        V
Mechanism             MechanismInterface

--> implies an *Is-A* relationship
o-- implies a *Uses-In-The-Interface* relationship
\end{lstlisting}
    

\noindent Alternatively, we might create a new component that adapts an existing
library to a new interface suitable for mocking without having to alter
it in any way:

\begin{lstlisting}[language=bash,style=plain]
client          Client   MockLibrary      Adapter
   o                 o        |          /       *
   |       =>         \       |         /         \
   |                   \      V        V           \
Library                 LibraryInterface         Library   (left unmodified)

*-- implies a *Uses-In-The-Implementation* (only) relationship
\end{lstlisting}
    

\noindent Sadly, it has become commonplace in the industry ``to make a mockery of
one's design through mocking''{\cprotect\footnote{Jonathan Wakely has
expressed to John Lakos that indiscriminate and excessive use of
  mocking ``is to make a mockery of one's design.''}} by artificially
declaring all nonconstructor member functions of a class virtual
(without extracting a protocol) just to be able to provide a ``mock''
implementation derived from the original concrete class:

\pagebreak%%%%%%% to keep image on one page so reviewers don't freak out

\begin{lstlisting}[language=bash,style=plain]
client          Client  MockMechanism  (making these final is pointless)
   o                 o       |
   |       =>         \      |  (BAD IDEA)
   |                   \     V
Mechanism             Mechanism   (now with all public virtual functions)
\end{lstlisting}
    

\noindent As the previous diagrams above suggest, much of the virtual function
overhead associated with mocking can be reduced significantly by
\emph{not} deriving the mock from the original class. Instead, we can
either (1) extract a \emcppsgloss{protocol} for all of the public
nonconstructor member functions of that class and mark those specific
functions \lstinline!final! or (2) adapt the original concrete
implementation unchanged to the new protocol for use by \lstinline!File!
and any other facilities that choose to opt in for the purposes of
thoroughly testing their code. With either of these approaches, all
clients that continue to use the concrete, possibly now-derived class by
its original name do not need to pay for \emph{any} virtual function
overhead to support the ability to mock that class's interface where
needed.

\subsubsection[Improving performance in a protocol hierarchy]{Improving performance in a protocol hierarchy}\label{improving-performance-in-a-protocol-hierarchy}

A \emcppsgloss{protocol
hierarchy}{\cprotect\footnote{\cite{lakos96}, Appendix~A, pp.~737--768}}{\cprotect\footnote{\cite{lakos2y}, section~2.7 [AUs: The citation is lakos2y, do you mean Vol II of Large-Scale C++? Response: Yes, volume II]}} is a generalization of the \emcppsgloss{composite
pattern}{\cprotect\footnote{\cite{gamma95}, Chapter 2, section ``Composite,''
  pp.~163--174}} in which a pure abstract interface (a.k.a.
\emcppsgloss{protocol}) sits at the root (\lstinline!leve!}~\lstinline!1!) of a
public inheritance hierarchy (\emph{Is-A} relationship) of abstract
interfaces:

\begin{lstlisting}[language=bash,style=plain]
   level 3          ( yet wider interface )
                             |         f(x), g(y), h(z)
                             |
                             |
                             | *Is-A*
                             V
   level 2:           ( wider interface )
                             |         f(x), g(y) [[ y >= 0 ]]
                             |
                             |
                             | *Is-A*
                             V
   level 1:            ( interface )
                                       f(x)
\end{lstlisting}
    

\noindent At each successive level, the \emph{pure} interface is a proper superset
of the one at the previous level from which it derives. None of the
classes in the hierarchy contributes any implementation; each member
function --- except for the destructor --- is implemented as
\emcppsgloss[pure virtual function]{pure virtual}. In particular, observe that \lstinline!g(y)! is not
accessible at \lstinline!level!~\lstinline!1!, can be used only with
non-negative values at \lstinline!level!~\lstinline!2! (i.e., \lstinline!g(y)!
has a \emcppsgloss{narrow contract}), and is usable with all syntactically
legal values at \lstinline!level!~\lstinline!3! (i.e., \lstinline!g(y)! has a
\emcppsgloss{wide contract}). Note that this same sort of interface widening
can apply in the absence of virtual functions through judicious use of
hiding nonvirtual functions; see \intraref{potential-pitfalls-final}{systemic-lost-opportunities-for-reuse}.

Concrete leaf nodes can then be derived from the protocol hierarchy to
implement the desired level of service as efficiently as practical. In
cases where multiple concrete nodes need to share the same
implementation of one or more functions, we can derive an intermediate
node from the appropriate \emph{protocol} that doesn't widen the
interface at all but does implement one or more of the pure abstract
functions; such an \emph{impure} abstract node is known as a
\emcppsgloss{partial implementation}. When the implementation of one of these
functions is trivial, declaring that \lstinline!virtual! function to be
\lstinline!inline! in addition might make sense. Since there will, by
design, be no need to further override that function, we can declare it
to be \lstinline!final! as well.

For performance-critical clients that would otherwise consume the
concrete object via the pure abstract interface from which this partial
implementation derives, we might decide to instead take the partial
implementation itself as the reference type. Because one or more
functions are both \lstinline!inline! and \lstinline!final!, the client can
dispense with runtime dispatch and inline the virtual functions directly
as discussed in {Restoring performance lost to mocking}.

As a real-world example, consider a simplified \emcppsgloss{protocol
hierarchy} for memory allocation:

\begin{emcppslisting}[emcppsbatch=e7]
#include <cstddef>  // (ù{\codeincomments{std::size\_t}}ù)

struct Allocator
{
    virtual void *allocate(std::size_t numBytes) = 0;
        // Allocate a block of memory of at least the specified (ù{\codeincomments{numBytes}}ù).

    virtual void deallocate(void *address) = 0;
        // Deallocate the block at the specified (ù{\codeincomments{address}}ù).
};

struct ManagedAllocator : Allocator
{
    void *allocate(std::size_t numBytes) = 0;
        // Allocate a block of memory of at least the specified (ù{\codeincomments{numBytes}}ù).

    void deallocate(void *address) = 0;
        // Deallocate the block at the specified (ù{\codeincomments{address}}ù).

    virtual void release() = 0;
        // Reclaim all memory currently allocated from this allocator.
};
\end{emcppslisting}
    

\noindent A \emcppsgloss{monotonic allocator} is a kind of \emcppsgloss{managed allocator}
that allocates memory sequentially in a buffer subject to alignment
requirements. In this class of allocators, the \lstinline!deallocate!
method is always a no-op; memory is reclaimed only when the managed
allocator is destroyed or its \lstinline!release! method is invoked:

\begin{emcppslisting}[emcppsbatch=e7]
struct MonotonicAllocatorPartialImp : ManagedAllocator
{
    void *allocate(std::size_t numBytes) = 0;
        // Allocate a block of memory of at least the specified (ù{\codeincomments{numBytes}}ù).

    inline void deallocate(void *address) final { /* empty */ }
        // Deallocate the block at the specified (ù{\codeincomments{address}}ù).

    void release() = 0;
        // Reclaim all memory currently allocated from this allocator.
};
\end{emcppslisting}
    

\noindent Notice that we have specified the empty
\lstinline!inline!~\lstinline!deallocate! member function of\linebreak[4]
\lstinline!MonotonicAllocatorPartialImp! to be \lstinline!final!. A concrete
monotonic allocator --- e.g., a \lstinline!BufferedSequentialAllocator! ---
can then derive from this partial implementation:

\begin{emcppslisting}[emcppsbatch=e7]
struct BufferedSequentialAllocator : MonotonicAllocatorPartialImp
{
    BufferedSequentialAllocator();
        // Create a default version of a buffered-sequential allocator.

    void *allocate(std::size_t numBytes);
        // Allocate a block of memory of at least the specified (ù{\codeincomments{numBytes}}ù).

    void release();
        // Reclaim all memory currently allocated from this allocator.
};
\end{emcppslisting}
    

\noindent Now consider two allocator-aware types, \lstinline!TypeA! and
\lstinline!TypeB!, each of which is always constructed with some flavor of
monotonic managed allocator:

\begin{emcppslisting}[emcppsbatch=e7]
struct TypeA
{
    TypeA(ManagedAllocator *a);
    // ...
};

struct TypeB
{
    TypeB(MonotonicAllocatorPartialImp *a);
    // ...
};
\end{emcppslisting}
    

\enlargethispage{2ex}%%%% so the close bracket is not on the next page       
\noindent We now construct each of the types using the same concrete allocator
object:

\begin{emcppslisting}[emcppsbatch=e7]
void client()
{
    BufferedSequentialAllocator a;  // Concrete monotonic allocator object

    TypeA ta(&a);  // (ù{\codeincomments{deallocate}}ù) is an empty (ù{\codeincomments{virtual}}ù) function call.
    TypeB tb(&a);  // (ù{\codeincomments{deallocate}}ù) is an empty (ù{\codeincomments{inline}}ù) function call.
}
\end{emcppslisting}
    

\noindent When \lstinline!TypeA! invokes \lstinline!deallocate!, it goes through the
non\lstinline!final!, \lstinline!virtual! function interface of
\lstinline!ManagedAllocator! and is subject to the runtime overhead of
dynamic dispatch. Note that even if the virtual \lstinline!deallocate!
function were inline, unless it is declared \lstinline!final! or the
runtime type is somehow known at compile time, there is no sure way for
the compiler to know that the function isn't overridden by a derived
type.

In the case of \lstinline!TypeB!, however, the function is both
\emph{declared} \lstinline!final! and \emph{defined} \lstinline!inline!; hence
the virtual dispatch can be reliably sidestepped, the empty function can
be inlined, and a true no-op is achieved with absolutely no runtime
overhead.

\subsection[Potential Pitfalls]{Potential Pitfalls}\label{potential-pitfalls-final}

\subsubsection[Contextual keywords are contextual]{Contextual keywords are contextual}\label{contextual-keywords-are-contextual}

The Standards Committee has, historically, taken very different
approaches to adding new keywords to the language. C++11 added ten new
keywords to the language --- \lstinline!alignas!, \lstinline!alignof!,
\lstinline!char16_t!, \lstinline!char32_t!, \lstinline!constexpr!,
\lstinline!decltype!, \lstinline!noexcept!, \lstinline!nullptr!,
\lstinline!static_assert!, and
\lstinline!thread_local!{\cprotect\footnote{C++14 and C++17 added exactly
zero new keywords in total. C++20 added \lstinline!char8_t!,
\lstinline!co_await!, \lstinline!co_return!, \lstinline!co_yield!,
\lstinline!concept!, \lstinline!consteval!, \lstinline!constinit!, and
\lstinline!requires!, notably mixing some potentially already used
identifiers (\lstinline!concept! and \lstinline!requires!) with a collection
of more obscure new words that had little chance of significantly
  conflicting with existing codebases.}} --- and thus made ten potential
tokens no longer usable as identifiers. When considering new keywords, a
great deal of effort is generally extended to determine the impact of
this change in status on existing codebases. Two identifiers,
\lstinline!override! and \lstinline!final!, were not made keywords and were
instead given special meaning when used in contexts where previously
identifiers were not syntactically allowed. This approach avoided
possible code breakage for any existing codebases using these words as
identifiers, at the cost of occasional confusion.

When used after a function declaration, \lstinline!override! and
\lstinline!final! do not add any significant parsing ambiguity to the
language; arbitrary identifiers were not syntactically valid in that
position anyway, so confusion is minimal. When used on a \lstinline!class!
declaration, however, \lstinline!final!'s meaning is not determined until
tokens after it are parsed to distinguish between a variable declaration
and a class definition:

\begin{emcppslisting}
struct S1 final;      // Error, variable named (ù{\codeincomments{final}}ù) of (ù{\emphincomments{incomplete}}ù) type
struct S2 final { };  // OK, (ù{\codeincomments{final}}ù) class definition
struct S2 final;      // OK, variable named (ù{\codeincomments{final}}ù) of (ù{\emphincomments{complete}}ù) type (ù{\codeincomments{S2}}ù)
\end{emcppslisting}
    

\noindent Notice that the variable declarations in the example above both look
like they might be an attempt to forward-declare a \lstinline!struct! that
is final but are instead a totally different language construct.

\subsubsection[Systemic lost opportunities for reuse]{Systemic lost opportunities for reuse}\label{systemic-lost-opportunities-for-reuse}

Both \lstinline!final! and \lstinline!override! are similar in their
complexity yet very different in the potential adverse implications that
widespread use can impose. Such ubiquitous use will depend heavily on
the scale and nature of the development process employed. In some
development environments, such as a small organization overseeing a
closed-source codebase where clients are able to request timely code
changes, encouraging liberal use of \lstinline!final! might not be
problematic. Instead of promising everything up front, even when much of
what is offered is not immediately useful, the default development
approach might reasonably be to provide only what is immediately
necessary and then quickly expose more if and as needed.

For some organizations, however, request-based code changes may not be a
viable option and can result in unacceptable delays in responding to
client needs. The extensive use of \lstinline!final! inherently prevents
clients from reusing library components in unanticipated ways and can
therefore lead to redundant and highly undesirable forks, often through
thoughtless copying and pasting the original code. Gratuitously
forbidding clients from doing what they feel they need to do and would
otherwise be able to do might be perceived as unnecessary
nannyism.{\cprotect\footnote{``Unnecessary nannyism'' is a phrase Bjarne
Stroustrup used to characterize his initial decision to restrict
operators \lstinline![]!, \lstinline!()!, and \lstinline!->! to be members; see
\cite{stroustrup94}, Chapter~x, section~3.6.2, ``Overloading: Members and
Friends,'' pp.~81--83, particularly p.~83.
%  
% LORI
%section 3 is ``The Birth of C++'', 3.6 is ``Overloading'', 3.6.2 is
%``Members and Friends''. Just saying ``members and friends'' is silly to
%me, it amounts to a page title\ldots{} maybe ``Overloading - members and
%friends'' or leave the section title out entirely.
}}

Consider, for example, the Standard Template Library (STL) and, in
particular, \lstinline!std::vector!. One might argue that
\lstinline!std:vector! was designed to facilitate generic programming, has
no virtual functions, and therefore should be specified as
\lstinline!final! to ensure its ``proper'' use and no other. Suppose, on
the other hand, that teachers wanting to teach their students the value
of \emcppsgloss{defensive programming} were to create an exercise to
implement a \lstinline!CheckedVector<T>!, derived publicly from
\lstinline!std::vector<T>!.{\cprotect\footnote{Bjarne Stroustrup himself
has employed as a class exercise in which the only two functions in
the derived type (that he typically calls ``Vector'') hide the
\lstinline!operator[]! overloads of \lstinline!std::vector!. These
implementations perform additional checking so that if they are ever
called out of their valid range, instead of resulting in
\emcppsgloss{undefined behavior}, they do something sensible, e.g., throw
an exception or, even better, print an error message and then call
\lstinline!abort! to terminate the program. By not employing
\lstinline!assert!, as we do in our example, Stroustrup avoids using
conditional compilation, which is not essential to the didactic
  purpose of this exercise.}} By inheriting constructors (see
\featureref{\locationc}{ctorinheriting}), it is very simple to implement
this derived class with an alternate implementation for just
\lstinline!operator[]!:

\begin{emcppslisting}[emcppsbatch=e8]
#include <vector>   // (ù{\codeincomments{std::vector}}ù)
#include <cassert>  // standard C (ù{\codeincomments{assert}}ù) macro

template <typename T>
class CheckedVector : public std::vector<T>
{
public:
    using std::vector<T>::vector;  // Inherit all ctors of (ù{\codeincomments{std::vector<T>}}ù).

    using reference       = typename std::vector<T>::reference;
    using const_reference = typename std::vector<T>::const_reference;
    using size_type       = typename std::vector<T>::size_type;

    reference operator[](size_type pos)              // hide base class function
    {
        assert(pos >= 0 && pos < this->size());      // Check bounds.
        return this->std::vector<T>::operator[](pos);
    }

    const_reference operator[](size_type pos) const  // hide base class function
    {
        assert(pos >= 0 && pos < this->size());      // Check bounds.
        return this->std::vector<T>::operator[](pos);
    }
};
\end{emcppslisting}
    

\noindent In the implementation above, we have chosen to use the standard
\lstinline!C!~\lstinline!assert! macro instead of hard-coding the check and
then, if needed, explicitly printing a message and calling
\lstinline!abort!. Note that this check will occur in only certain build
modes, i.e., when \lstinline!NDEBUG! is not defined for the current
translation unit. Note also that the checking for
\lstinline!pos!~\lstinline!>=!~\lstinline!0! can be omitted here as
\lstinline!std::vector::size! returns an integral value of type
\lstinline!std::size_t!, which is guaranteed to be of some
implementation-defined \lstinline!unsigned! integral type on all conforming
platforms. Finally the use of \lstinline!this->! is purely stylistic to
show that we are invoking a member function on this object and can be
omitted in every case above where it is used with no change in defined
behavior.

For local use with the goal of exploring and learning, allowing such
\emcppsgloss{structural inheritance} --- involving no virtual functions ---
can be an expeditious way of uncovering client misuse. This disciplined
use of \emcppsgloss{structural inheritance} adds no data members; it merely
widens some of the narrow interfaces in the base class, thereby benignly
enhancing the defined behavior already provided yet leaving unchanged
the behavior of all other functions of \lstinline!vector!. We can now
deploy our derived \lstinline!CheckedVector! by replacing instances of
\lstinline!std::vector! with those of our derived \lstinline!CheckedVector!,
including those in interfaces where potential misuse might occur; simple
interaction with the parts of the system not so modified will continue
to operate as before. Functions that take an \lstinline!std::vector! passed
by value are not recommended but will continue to work as before via
\emcppsgloss{slicing}, as will those passed by pointer or reference via
implicit standard conversion to base type:

\begin{emcppslisting}[emcppsbatch=e8]
void myApi(const CheckedVector<int> &data);       // checked local API
void otherApi1(const std::vector<int> &data);     // by-reference standard API
void otherApi2(std::vector<int> data);            // by-value standard API
template <typename T>
void genericApi(const T&data);                    // generic API

void myFunction()
{
    CheckedVector<int> myData;
    // ...                  ( populate (ù{\codeincomments{myData}}ù) )

    otherApi1(myData);   // normal usage, unchecked operation
    otherApi2(myData);   // normal usage, unchecked operation, slices on copy
    myApi(myData);       // checked operations within implementation

    std::vector<int>&  uncheckedData = myData;
    genericApi(myData);        // checked call to templated API
    genericApi(uncheckedData); // unchecked call to same API
}
\end{emcppslisting}
    

\noindent Were \lstinline!std::vector! declared \lstinline!final!, this form of
investigation would be entirely prevented, harming those who wished to
learn or needed to diagnose defects in their systems using
\lstinline!std::vector!. Such compulsive use of \lstinline!final!, even in
local libraries, often leads to what is arguably the worst possible
result: The client makes a local copy of the library class. Such
gratuitously forced duplication of source code systemically exacerbates
the already high cost of software maintenance and denies the use of any
future enhancements or bug fixes made to that library class.

Finally, contrary to popular belief, the strict notion of
substitutability afforded by \emcppsgloss{structural inheritance} is far more
in keeping with those characterized by Barbara Liskov in her pioneering
work on subtyping {\cprotect\footnote{\cite{liskov87}}} than the
\emph{variation in behavior}{\cprotect\footnote{Tom Cargill observed
that data members are for variation in value, whereas virtual
functions are for variation in behavior; see \cite{cargill92}, section ``Value versus
Behavior,'' pp.~16--19, specifically p.~17 (see also p.~83 and p.~182)}}
 afforded by virtual functions. Thinking that a
class has no business being derived from just because it doesn't sport a
virtual destructor is misguided. Much of what is done in metaprogramming
relies heavily on structural inheritance; for example, the class
template \lstinline!std::integral_constant! serves as the base class for
most C++ type traits. Structurally inheriting a \lstinline!const_iterator!
from an \lstinline!iterator! achieves implicit convertibility without
consuming a user-defined conversion and thereby avoids needless
asymmetry; see \intraref{annoyances-final}{making-empty-types-final-precludes-the-empty-base-class-optimization}.

The decision to make systemic use of \lstinline!final! the default ---
absent any specific engineering reason --- is dubious as it directly
precludes \emcppsgloss{reuse} in general and \emcppsgloss{hierarchical reuse} in
particular. In any event, such a policy, to achieve its intended
purpose, fairly belongs with an organization as a whole rather than with
each individual developer within that organization. If, however, some
specific reason precludes overriding a virtual function or inheritance
in general, then use of \lstinline!final! is indicated if only to actively
document our intent. Lacking such forceful documentation initially,
\emcppsgloss{Hyrum's law} will ultimately subvert our ability to make the
choice to prevent overriding and/or inheritance at a later date.

\subsubsection[Attempting to defend against the hiding of a non\lstinline!virtual! function]{Attempting to defend against the hiding of a non{\SubsubsecCode virtual} function}\label{attempting-to-defend-against-the-hiding-of-a-nonvirtual-function}

Deliberately hiding virtual functions is not typically
recommended{\cprotect\footnote{\cite{meyers96}, Chapter~6, item~37, ``Avoid
  Hiding Inherited Names,'' pp.~131--132}}; when it is done, it is often
by accident. Declaring an otherwise nonvirtual function to be both
virtual and final does in fact prevent a derived class author from
overriding that specific function; a derived class's version of the
function will never be invoked through dynamic dispatch. \texttt{final}
does \emph{not}, however, prevent a function of the same name having any
distinct signature from inadvertently hiding it.

As an illustrative example, suppose we were to have a simple
output-device type, \texttt{Printer}, that contains a set of overloaded
non\texttt{virtual} print functions to display various types of
information on a common device:

\begin{emcppslisting}[emcppsbatch=e9]
struct Printer
{
    void print(int number);
    void print(bool boolean);
};
\end{emcppslisting}
    

\noindent Now suppose an inexperienced programmer is about to extend that type to
support another parameter type and accidentally hides the base-class
functions:

\begin{emcppslisting}[emcppsbatch=e9]
struct ExtendedPrinter : public Printer
{
    void print(long c);
};
\end{emcppslisting}
    

\noindent Calling \lstinline!ExtendedPrinter::print! with a parameter having a type
that is implicitly convertible to \lstinline!long! --- such as both of the
existing supported types of \lstinline!int! and \lstinline!bool! --- will
still compile but fail to lead to the expected results.

To defend against such misuse, we might incorrectly try to defend
against such misuse by declaring all functions of the base class both
\lstinline!virtual! and \lstinline!final! as well:

\begin{emcppslisting}[emcppsbatch=e10]
struct Printer0
{
    virtual void print(int number) final;
    virtual void print(bool boolean) final;
};
\end{emcppslisting}
    
\enlargethispage*{2ex}%% so reviewers don't freak about code
\noindent Such machinations do, in fact, make it a compile-time error should we
accidentally supply a function matching \emph{exactly} that signature in
a derived class, but it does not prevent a function of the same name
having any other signature from doing so:

\begin{emcppslisting}[emcppsbatch=e10]
struct ExtendedPrinter0 : public Printer0
{
    void print(int number);    // Error, (ù{\codeincomments{Printer0::print(int)}}ù) is (ù{\codeincomments{final}}ù)
    void print(char c) const;  // OK, still hides base-class functions
};
\end{emcppslisting}
    

\noindent Making a function virtual in a class where previously there was none
also forces the compiler to maintain a pointer to a \lstinline!static!
virtual-function table in each object. Consequently, \emph{any} use of
\lstinline!virtual! and \lstinline!final! to decorate the same function with a
class is contraindicated.

\subsection[Annoyances]{Annoyances}\label{annoyances-final}

\subsubsection[Making empty types \lstinline!final! precludes the empty base-class optimization]{Making empty types {\SubsubsecCode final} precludes the empty base-class optimization}\label{making-empty-types-final-precludes-the-empty-base-class-optimization}

Whenever a user-defined type derives from another that has no data
members, that base type does not typically consume any additional memory
in the derived type. This optimization is called \emcppsgloss[empty base optimization]{empty
base-class optimization (EBO)} and is often exploited when applying
policy-based design. Consider this slightly modified version of a
classic example{\cprotect\footnote{\cite{alexandrescu01}, Chapter~1, section~1.5, ``Policies and Policy Classes,'' pp.~8--11}} in which an
\lstinline!ObjectCreator! relies on a specific \lstinline!CreationPolicy! type
for implementing the acquisition of memory and construction of objects:

\begin{emcppslisting}[emcppsbatch=e11]
#include <cstddef>  // (ù{\codeincomments{std::size\_t}}ù)

template<typename TYPE, template<typename> class CreationPolicy>
class ObjectCreator : CreationPolicy<TYPE>
{
    std::size_t objectCount = 0;  // Keep track of allocated objects.

public:
    TYPE *create()
    {
        ++objectCount;
        return CreationPolicy<TYPE>().create();  // Delegate to (ù{\codeincomments{CreationPolicy}}ù).
    }
};
\end{emcppslisting}
    

\noindent Each of the associated policies are implemented as empty classes, i.e.,
classes having no data members:

\begin{emcppslisting}[emcppsbatch=e11]
template<typename TYPE>
class OpNewCreator // (ù{\codeincomments{sizeof(OpNewCreator)}}ù) by itself is 1 byte.
{
public:
    TYPE *create()
    {
        // Allocate memory using placement new and return address.
    }
};

template<typename TYPE>
class MallocCreator // (ù{\codeincomments{sizeof(MallocCreator)}}ù) by itself is 1 byte.
{
public:
    TYPE *create()
    {
        // Allocate memory using malloc and return address.
    }
};

static_assert(sizeof(ObjectCreator<int,OpNewCreator>) ==sizeof(std::size_t),"");
static_assert(sizeof(ObjectCreator<int,MallocCreator>)==sizeof(std::size_t),"");
\end{emcppslisting}
    

\noindent Since \lstinline!OpNewCreator! and \lstinline!MallocCreator! do not have any
data members, inheriting from either of them does not increase the size
of \lstinline!ObjectCreator!. If someone later decides to declare them as
\lstinline!final!, inheriting becomes impossible, even if just privately as
an optimization:

\begin{emcppslisting}[emcppsbatch=e12,emcppserrorlines={4,6}]
class OpNewCreator final { /* ... */ };   // subsequently declared (ù{\codeincomments{final}}ù)
class MallocCreator final { /* ... */ };  //      "           "       "

template<typename TYPE, template<typename> class CreationPolicy>
class ObjectCreator : CreationPolicy<TYPE>  // Error, derivation is disallowed.
{ /* ... */ };
\end{emcppslisting}
    

\noindent By declaring the empty bases class \lstinline!final!, a valid use case is
needlessly prevented. Using containment instead of \emcppsgloss{private
inheritance} consumes at least one extra byte in the footprint of
\lstinline!ObjectCreator!,{\cprotect\footnote{C++20 adds a new
attribute, \lstinline![[no_unique_address]]!, that allows the compiler
to avoid consuming addition storage for data objects of empty classes:

\begin{emcppslisting}[style=footcode,emcppsstandards={c++17}]
struct A final { /* no data members */ };
struct S {
    [[no_unique_address]] A a;  static_assert(sizeof(a) == 1, "");
    int x;                      static_assert(sizeof(x) == 4, "");
};                              static_assert(sizeof(S) == 4, "");
\end{emcppslisting}
      }} which will inevitably also come at the cost of additional
padding imposed by alignment requirements:

\begin{emcppslisting}[emcppsbatch=e11]
template<typename TYPE, template<typename> class CreationPolicy>
class LargeObjectCreator
{
    CreationPolicy<TYPE> policy;  // now consumes an extra byte &
    std::size_t objectCount = 0;  // with padding 8 extra bytes

public:
    TYPE *create()
    {
        ++objectCount;
        return policy.allocate();
    }
};

static_assert(
    sizeof(LargeObjectCreator<int, OpNewCreator>) > sizeof(std::size_t), "");

static_assert(
    sizeof(LargeObjectCreator<int, MallocCreator>) > sizeof(std::size_t), "");
\end{emcppslisting}
    

\noindent Alternatively, the author of \lstinline!OpNewCreator! and
\lstinline!MallocCreator! might reconsider and remove \lstinline!final!.

\subsection[See Also]{See Also}\label{see-also}

\begin{itemize}
\item{\seealsoref{override}{\seealsolocationa} is a related contextual
keyword that verifies the existence of matching virtual functions in
base classes instead of preventing matching virtual functions in derived
classes.}
\end{itemize}

\subsection[Further Reading]{Further Reading}\label{further-reading}

\begin{itemize}
\item{Barbra Liskov discusses in her seminal 1987 keynote paper a remarkable number of issues relevant to the ongoing design and development of modern C++; see \cite{liskov87}.}
\item{Barbara Liskov and Jeanette Wing followed up with a precise notion of subtyping in which any property provable about objects of a supertype would necessarily hold for objects of proper subtypes; see \cite{liskov94}. This notion of proper subtyping (which is manifestly distinct from C++-style inheritance) would later come to be know as the Liskov Substitution Principal (LSP)\cprotect\footnote{\cite{liskov94}, section~1, ``Introduction,'' p.~1812}:
\begin{quote}
Let φ(x) be a property provable about objects x of type T.
Then φ(y) should be true for objects y of type S where S is a subtype of T.
\end{quote}
}
\end{itemize}
    

