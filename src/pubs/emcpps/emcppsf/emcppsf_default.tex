\emcppsFeature{
        short={Defaulted Functions},
        long={Using {\SecCode =default} for Special Member Functions},
}{Defaulted-Special-Member-Functions}

% 01 Feb, 2021 JMB - minor fixes to file heading within listings


Use of \texttt{=}~\texttt{default} in a \textbf{special member function}'s
declaration instructs the compiler to attempt generating the function
automatically.

\subsection[Description]{Description}\label{description}

An important aspect of C++ class design is the understanding that the
compiler will attempt to generate certain member functions to
\emph{create}, \emph{copy}, \emph{destroy}, and now
\emph{move} (see Section~\ref{Rvalue-References}, ``\titleref{Rvalue-References}") an object unless
developers implement some or all of these functions
themselves. Determining which of the \textbf{special member functions}
will continue to be generated and which will be suppressed in the
presence of \textbf{user-provided special member functions}
requires remembering the numerous rules the compiler uses.

\subsubsection[Declaring a special member function explicitly]{Declaring a special member function explicitly}\label{declaring-a-special-member-function-explicitly}

The rules specifying what happens in the presence of one or more
user-provided special member functions are inherently complex and not
necessarily intuitive; in fact, some have been
deprecated.{\cprotect\footnote{Even in the presence of a user-provided
destructor, both the copy constructor and copy-assignment operator
have historically been generated implicitly. Relying on such generated
behavior is not recommended because it is unlikely that a class
requiring a user-provided destructor will function correctly without
corresponding user-provided copy operations. As of C++11, reliance on
  such dubious implicitly generated behavior is deprecated.}} Here, we
will briefly illustrate a few common cases and then refer you to Howard
Hinnant's now famous table (see page~\pageref{default-table1} of {\it\titleref{appendix:-implicit-generation-of-special-member-functions}}) to
demystify what's going on under the hood.

\paragraph[Example 1: Providing just the default constructor]{Example 1: Providing just the default constructor}\label{example-1:-providing-just-the-default-constructor}

Consider a \texttt{struct} with a user-provided default constructor:

\begin{lstlisting}[language=C++]
struct S1
{
    S1();  // user-provided default constructor
};
\end{lstlisting}

\noindent A user-provided default constructor has no effect on other special
member functions. Providing any other constructor, however, will
suppress automatic generation{\cprotect\footnote{A nondeclared
function is nonexistent, which means that it will \emph{not}
participate in overload resolution at all. In contrast, a
\textbf{deleted function} participates in overload resolution and, if
selected, results in a compilation failure; see Section~\ref{deleted-functions}, ``\titleref{deleted-functions}."}} of the default constructor. We can, however, use
\texttt{=}~\texttt{default} to restore the constructor as a \textbf{trivial
operation}; see {\it\titleref{default-use-cases}: \titleref{restoring-the-generation-of-a-special-member-function-suppressed-by-another}} on page~\pageref{restoring-the-generation-of-a-special-member-function-suppressed-by-another}.

\paragraph[Example 2: Providing just a copy constructor]{Example 2: Providing just a copy constructor}\label{example-2:-providing-just-a-copy-constructor}

Now, consider a \texttt{struct} with a user-provided copy constructor:

\begin{lstlisting}[language=C++]
struct S2
{
    S2(const S2&);  // user-provided copy constructor
};
\end{lstlisting}

\noindent A user-provided copy constructor (1) suppresses the generation of the
default constructor and both move operations and (2) allows implicit
generation of both the copy-assignment operator and the destructor.
Similarly, providing just the copy-assignment operator would allow the
compiler to implicitly generate both the copy constructor and the
destructor, but, in this case, it would also generate the default
constructor (see \texttt{S6} in the \texttt{example.h} code snippet on page~\pageref{default-exampleh-code}). Note that --- in either of these
cases --- relying on the compiler's implicitly generated copy operation
is deprecated.

\paragraph[Example 3: Providing just the destructor]{Example 3: Providing just the destructor}\label{example-3:-providing-just-the-destructor}

Finally, consider a \texttt{struct} with a user-provided destructor:


\begin{lstlisting}[language=C++]
struct S3
{
    ~S3();  // user-provided destructor
};
\end{lstlisting}

\noindent A user-provided destructor suppresses the generation of move
operations but still allows copy operations to be generated. Again,
relying on either of these implicitly compiler-generated copy
operations is deprecated.

\paragraph[Example 4: Providing more than one special member function]{Example 4: Providing more than one special member function}\label{example-4:-providing-more-than-one-special-member}

When more than one special member function is declared explicitly, the
\emph{union} of their respective declaration suppressions and the
\emph{intersection} of their respective implicit generations pertain ---
e.g., if just the default constructor and destructor are provided
(\texttt{S1} + \texttt{S3} in Examples 1 and 3), then the declarations of both move
operations are suppressed and both copy operations are generated
implicitly.

\subsubsection[Defaulting the first declaration of a special member function explicitly]{Defaulting the first declaration of a special member function explicitly}\label{defaulting-the-first-declaration-of-a-special-member-function-explicitly}

Using the \texttt{=}~\texttt{default} syntax with the first declaration
of a special member function instructs the compiler to synthesize such a
function automatically without treating it as being
user provided.{\cprotect\footnote{The compiler-generated version for a
special member function is required to call the corresponding special
member functions on every base class in base-class-declaration
order and then every data member of the encapsulating type in
declaration order (regardless of any access specifiers). Note that the
destructor calls will be in exactly the opposite order of the
  other special-member-function calls.}}

For example, consider struct \texttt{S4} (in the code snippet below) in which we have chosen
to make explicit that the copy operations are to be autogenerated by
the compiler; note, in particular, that implicit declaration and generation
of each of the other special member functions is left unaffected.

\begin{lstlisting}[language=C++]
struct S4
{
    S4(const S4&) = default;             // copy constructor
    S4& operator=(const S4&) = default;  // copy-assignment operator

    // has no effect on other other four special member functions, i.e.,
    // implicitly generates the default constructor, the destructor,
    // the move constructor, and the move-assignment operator
};
\end{lstlisting}

\noindent A defaulted declaration may appear with any \textbf{access specifier} ---
i.e., \texttt{private},\linebreak[4] \texttt{protected}, or \texttt{public} --- and access to that generated function will be regulated accordingly:

\begin{lstlisting}[language=C++]
struct S5
{
private:
    S5(const S5&) = default;             // private copy constructor
    S5& operator=(const S5&) = default;  // private copy-assignment operator

protected:
    ~S5() = default;                     // protected destructor

public:
     S5() = default;                     // public default constructor
};
\end{lstlisting}

\noindent In the example above, copy operations exist for use by \emph{member} and
\emph{friend} functions only. Declaring the destructor
\texttt{protected} or \texttt{private} limits which functions can create
automatic variables of the specified type to those functions with the
appropriately privileged access to the class. Declaring the default
constructor \texttt{public} is necessary to avoid its declaration's
being suppressed by another constructor (e.g., the private copy
constructor in the code snippet above) or \emph{any} move operation.

In short, use of \texttt{=}~\texttt{default} on first declaration
denotes that a special member function is intended to be generated by
the compiler --- irrespective of any user-provided declarations; in
conjunction with {\texttt{=}~\texttt{delete}} (see Section~\ref{deleted-functions}, ``\titleref{deleted-functions}"),
use of
\texttt{=}~\texttt{default} affords the fine-grained control over which
special member functions are to be generated and/or made publicly
available.

\subsubsection[Defaulting the implementation of a user-provided special member function]{Defaulting the implementation of a user-provided special member function}\label{defaulting-the-implementation-of-a-user-provided-special-member-function}

The \texttt{=}~\texttt{default} syntax can also be used after the first
declaration, but with a distinctly different meaning: The compiler will
treat the first declaration as a \textbf{user-provided special member
function} and thus will suppress the generation of other
\textbf{special member functions} accordingly.

\begin{lstlisting}[language=C++,label=default-exampleh-code]
// example.h:

struct S6
{
    S6& operator=(const S6&);  // user-provided copy-assignment operator

    // suppresses the declaration of both move operations
    // implicitly generates the default and copy constructors, and destructor
};

inline S6& S6::operator=(const S6&) = default;
    // Explicitly request the compiler to generate the default implementation
    // for this copy-assignment operator. This request might fail (e.g., if (ù{\codeincomments{S6}}ù)
    // were to contain a non-copy-assignable data member).
\end{lstlisting}

\noindent Alternatively, an explicitly defaulted non-inline implementation of this
copy-assignment operator may appear in a separate (\texttt{.cpp}) file;
see {\it\titleref{default-use-cases}: \titleref{physically-decoupling-the-interface-from-the-implementation}} on page~\pageref{physically-decoupling-the-interface-from-the-implementation}.

\subsection[Use Cases]{Use Cases}\label{default-use-cases}

\subsubsection[Restoring the generation of a special member function suppressed by another]{Restoring the generation of a special member function suppressed by another}\label{restoring-the-generation-of-a-special-member-function-suppressed-by-another}

Incorporating \texttt{=}~\texttt{default} in the declaration of a
special member function instructs the compiler to generate its
definition regardless of any other user-provided special member
functions. As an example, consider a \textbf{value-semantic}
\texttt{SecureToken} class that wraps a standard string
(\texttt{std::string}) and an arbitrary-precision-integer
(\texttt{BigInt}) token code that together satisfy certain invariants:

\begin{lstlisting}[language=C++]
class SecureToken
{
    std::string d_value;  // The default-constructed value is the empty string.
    BigInt      d_code;   // The default-constructed value is the integer zero.

public:
    // All six special member functions are (implicitly) defaulted.

    void setValue(const char* value);
    const char* value() const;
    BigInt code() const;
};
\end{lstlisting}

\noindent By default, a secure token's \texttt{value} will be the empty-string
value and the token's \texttt{code} will be the numerical value of zero
(because those are, respectively, the \textbf{default-initialized} values
of the two data members, \texttt{d\_value} and \texttt{d\_code}):

\begin{lstlisting}[language=C++,label=default-voidf-code]
void f()
{
    SecureToken token;                       // default constructed         (1)
    assert(token.value() == std::string());  // default value: empty string (2)
    assert(token.code() == BigInt());        // default value: zero         (3)
}
\end{lstlisting}

\noindent Now suppose that we get a request to add a \textbf{value constructor}
that creates and initializes a \texttt{SecureToken} from a specified
token string:

\begin{lstlisting}[language=C++]
class SecureToken
{
    std::string d_value;  // The default-constructed value is the empty string.
    BigInt      d_code;   // The default-constructed value is the integer zero.

public:
    SecureToken(const char* value);  // newly added value constructor

    // suppresses the declaration of just the default constructor --- i.e.,
    // implicitly generates all of the other five special member functions

    void setValue(const char* value);
    const char* value() const;
    const BigInt& code() const;
};
\end{lstlisting}

\noindent Attempting to compile function \texttt{f} (from page~\pageref{default-voidf-code}) would now fail on the
first line, where it attempts to default-construct the token. Using
the \texttt{=}~\texttt{default} feature, however, we can reinstate the
default constructor to work trivially, just as
it did before:

\begin{lstlisting}[language=C++]
class SecureToken
{
    std::string d_value;  // The default-constructed value is the empty string.
    BigInt d_code;        // The default-constructed value is the integer zero.

public:
    SecureToken() = default;         // newly defaulted default constructor
    SecureToken(const char *value);  // newly added value constructor

    // implicitly generates all of the other five special member functions

    void setValue(const char *value);
    const char *value() const;
    const BigInt& code() const;
};
\end{lstlisting}


\subsubsection[Making class APIs explicit at no runtime cost]{Making class APIs explicit at no runtime cost}\label{making-class-apis-explicit-at-no-runtime-cost}

In the early days of C++, coding standards sometimes
required that each special member function be declared explicitly so
that it could be documented or even just to know that it hadn't been
forgotten:

\begin{lstlisting}[language=C++]
class C1
{
     // ...

public:
   C1();
       // Create an empty object.

   C1(const C1& rhs);
       // Create an object having the same value as the specified (ù{\codeincomments{rhs}}ù) object.

   ~C1();
       // Destroy this object.

   C1& operator=(const C1& rhs);
       // Assign to this object the value of the specified (ù{\codeincomments{rhs}}ù) object.
};
\end{lstlisting}

\noindent Over time, explicitly writing out what the compiler
could do more reliably itself became more clearly an inefficient use of developer time and a maintenance burden.
What's more, even if the function definition was empty, implementing it
explicitly often degraded performance compared to a \textbf{trivial} default. Hence, such
standards tended to evolve toward conventionally commenting out (e.g.,
using \texttt{//!}) the declarations of functions having an empty
body rather than providing it explicitly:

\begin{lstlisting}[language=C++]
class C2
{
     // ...

public:
   //! C2();
       // Create an empty object.

   //! C2(const C2& rhs);
       // Create an object having the same value as the specified (ù{\codeincomments{rhs}}ù) object.

   //! ~C2();
       // Destroy this object.

   //! C2& operator=(const C2& rhs);
       // Assign to this object the value of the specified (ù{\codeincomments{rhs}}ù) object.
};
\end{lstlisting}

\noindent Note, however, that the compiler does not check the commented code, which is easily susceptible to copy-paste and other errors. By
uncommenting the code and defaulting it explicitly in class scope, we
regain the compiler's syntactic checking of the function signatures
without incurring the cost of turning what would have been
\textbf{trivial} functions into equivalent
non-\textbf{trivial} ones:

\begin{lstlisting}[language=C++]
class C3
{
     // ...

public:
   C3() = default;
       // Create an empty object.

   C3(const C3& rhs) = default;
       // Create an object having the same value as the specified (ù{\codeincomments{rhs}}ù) object.

   ~C3() = default;
       // Destroy this object.

   C3& operator=(const C3& rhs) = default;
       // Assign to this object the value of the specified (ù{\codeincomments{rhs}}ù) object.
};
\end{lstlisting}


%\subsubsection[Preserving trivial copyability]{Preserving trivial copyability}\label{preserving-trivial-copyability}
%
%In some situations, a particular type \textit{must} be
%usable with \texttt{std::memcpy} (e.g., runtime performance,
%serialization to binary, or interoperability with C code). Only
%\textbf{trivially copyable} types are safe to use with
%\texttt{std::memcpy}; use with any other types results in
%\textbf{undefined behavior}. A type \texttt{T} is \textbf{trivially
%copyable} if it exposes a \textbf{trivial copy constructor}:
%\begin{enumerate}
%\item{the copy constructor for \texttt{T} is not user provided}
%\item{the type \texttt{T} itself has no \texttt{virtual} member functions or
%\texttt{virtual} base classes}
%\item{any member or base class of
%\texttt{T} is itself \textbf{trivially copyable} (recursively).}
%\end{enumerate}
%
%As an example, the \texttt{EntityHandle} class (in the code snippet below) represents an
%integer handle (to an entity of opaque type) that must be usable with
%\texttt{std::memcpy} for the purpose of efficient serialization (the
%capacity of the encapsulated fundamental integral type is subject to
%change){\cprotect\footnote{Objects of this type are sometimes said to
%  hold ``dumb data''; see \textbf{{lakos20}}, section~3.5.5,
%  pp.~629--633.}}:
%
%\begin{lstlisting}[language=C++]
%class EntityHandle
%{
%    short int d_id;  // Note: Implementation size may increase over time.
%
%public:
%    EntityHandle(int id);  // value constructor
%
%    // suppresses the declaration of just the default constructor --- i.e.,
%    // implicitly generates all of the other five special member functions
%
%    // ...
%}
%\end{lstlisting}
%
%\noindent The presence of any other constructor, except a \emph{move constructor},
%never affects implicit generation of a copy constructor, and
%\texttt{short}~\texttt{int} (like all \emph{enumerated}, \emph{pointer},
%and other \emph{fundamental} types) is a \textbf{trivial type}, thus
%establishing the \emph{triviality} of copying an \texttt{EntityHandle}.
%Now imagine that, to monitor the places in the codebase where
%\emph{temporary} entity handles are exchanged (with the goal of
%ultimately optimizing those), a user-provided \emph{move constructor} is
%added{\cprotect\footnote{Note that a move constructor will be preferred
%over a copy constructor when the type category of the argument is an
%\emph{xvalue} (i.e., expiring value) or \emph{prvalue} (i.e., pure rvalue), which
%are the value categories to which a temporary can pertain. See Section~\ref{Rvalue-References}, ``\titleref{Rvalue-References}," for more information.}}:
%
%\begin{lstlisting}[language=C++]
%class EntityHandle
%{
%    short int d_id;  // Note: Implementation size may increase over time.
%
%public:
%    EntityHandle(int id);              // value constructor
%    EntityHandle(EntityHandle&& rhs);  // user-provided move constructor
%
%    // implicitly generates only the destructor
%    // suppresses synthesis of all four other special member functions
%
%    // ...
%};
%\end{lstlisting}
%
%\noindent As illustrated by Table~\ref{default-table1} on page~\pageref{default-table1}, the
%presence of a user-provided \emph{move constructor} suppressed the
%automatic generation of a copy constructor along with the destructor
%and both the copy- and move-assignment operators, thereby rendering the
%\texttt{EntityHandle} unusable. Replacing these four previously
%generated functions with seemingly equivalent user-provided ones might
%appear to work as intended:
%
%\begin{lstlisting}[language=C++]
%class EntityHandle
%{
%    short int d_id;  // Note: Implementation size may increase over time.
%
%public:
%    EntityHandle(int id);  // value constructor
%
%    EntityHandle(const EntityHandle& rhs);  // user-provided copy constructor
%    EntityHandle(EntityHandle&& rhs);       // user-provided move constructor
%
%    EntityHandle& operator=(const EntityHandle& rhs);
%        // user-provided copy-assignment operator
%
%    EntityHandle& operator=(EntityHandle&& rhs);
%        // user-provided move-assignment operator
%
%    // implicitly generates only the destructor
%    // suppresses synthesis of the default constructor
%
%    // ...
%};
%\end{lstlisting}
%
%\noindent The user-provided nature of the copy constructor, however, renders the
%\texttt{EntityHandle} class ineligible for copy triviality --- even if
%the definitions are identical! Hence, any direct use of
%\texttt{std::memcpy} with an \texttt{EntityHandle} object will result in
%\textbf{undefined behavior}. We could have instead explicitly requested
%that these four special member functions be generated using
%\texttt{=}~\texttt{default}:
%
%\begin{lstlisting}[language=C++]
%class EntityHandle
%{
%    short int d_id;  // Note: Implementation size may increase over time.
%
%public:
%    EntityHandle(int id);  // value constructor
%
%    EntityHandle(const EntityHandle& rhs) = default;
%        // defaulted (trivial) copy constructor
%
%    EntityHandle(EntityHandle&& rhs);
%        // user-provided move constructor
%
%    EntityHandle& operator=(const EntityHandle& rhs) = default;
%        // default (trivial) copy-assignment operator
%
%    EntityHandle& operator=(EntityHandle&& rhs);
%        // user-provided move-assignment operator
%
%    // Implicitly generates only the destructor.
%    // suppresses synthesis of the default constructor
%
%    // ...
%};
%\end{lstlisting}
%
%\noindent By explicitly defaulting these three special member functions {in class
%scope}, we (1) re-enable their generation and (2) preserve the
%\emph{copy triviality} of the class.


%%%% new material from Slava
\subsubsection{Preserving type triviality}

It can be beneficial if a particular type is \romeogloss{trivial}.  The type is considered \romeogloss{trivial} if its default constructor is \romeogloss{trivial} and it is \romeogloss{trivially copyable} --- i.e., it has no non-trivial copy or move constructors, no non-trivial copy or move assignment operators, has at least one of those non-deleted, and has a trivial destructor.  As an example, consider a simple \romeogloss{trivial}  \lstinline!Metrics! type in the code snippet below containing certain collected metrics for our application:

\begin{lstlisting}
struct Metrics
{
    int d_numRequests;  // number of requests to the service
    int d_numErrors;    // number of error responses

    // All special member functions are generated implicitly.
};
\end{lstlisting}

\noindent Now imagine that we would like to add a constructor to this struct to make its use more convenient:

\begin{lstlisting}
struct Metrics
{
    int d_numRequests;  // number of requests to the service
    int d_numErrors;    // number of error responses

    Metrics(int, int);  // user-provided value constructor

    // Generation of default constructor is suppressed.
};
\end{lstlisting}

\noindent As illustrated in \intrarefsimple{appendix:-implicit-generation-of-special-member-functions}, the presence of a user-provided constructor suppressed the implicit generation of the default constructor.  Replacing the default constructor with seemingly equivalent user-provided one might appear to work as intended:

\begin{lstlisting}
struct Metrics
{
    int d_numRequests;  // number of requests to the service
    int d_numErrors;    // number of error responses

    Metrics(int, int);  // user-provided value constructor
    Metrics() {}        // user-provided default constructor

    // Default constructor is user-provided: (ù{\codeincomments{Metrics}ù) is (ù{\emphincomments{not}ù) trivial.
};
\end{lstlisting}

\noindent The user-provided nature of the default constructor, however, renders \lstinline!Metrics! type non-trivial --- even if the definitions are identical!  In contrast, explicitly requesting the default constructor be generated using \lstinline!=!\,\lstinline!default! restores the triviality of the type:

\begin{lstlisting}
struct Metrics
{
    int d_numRequests;  // number of requests to the service
    int d_numErrors;    // number of error responses

    Metrics(int, int);    // user-provided value constructor
    Metrics() = default;  // defaulted (trivial) default constructor

    // Default constructor is defaulted: (ù{\codeincomments{Metrics}ù) (ù{\emphincomments{is}ù) trivial.
};
\end{lstlisting}
%%%%%%%%%%%%% new material ends

\subsubsection[Physically decoupling the interface from the implementation]{Physically decoupling the interface from the implementation}\label{physically-decoupling-the-interface-from-the-implementation}

Sometimes, especially during large-scale development, avoiding compile-time coupling clients to the
implementations of individual methods offers distinct maintenance advantages. Specifying that a special member
function is defaulted on its first declaration (i.e., in class scope)
implies that making any change to this implementation will force all
clients to recompile{\cprotect\footnote{The issue here is not just
compile time, per se, but compile-time \emph{coupling}; see
  \textbf{{lakos20}}, section 3.10.5, pp.~783--789.}}:

\begin{lstlisting}[language=C++]
// smallscale.h:

struct SmallScale
{
    SmallScale() = default;  // explicitly defaulted default constructor
};
\end{lstlisting}

\noindent Alternatively, we can choose to declare the function but deliberately
\emph{not} default it in class scope (or anywhere in the \texttt{.h}
file):

\begin{lstlisting}[language=C++]
// largescale.h:

struct LargeScale
{
    LargeScale();  // user-provided default constructor
};
\end{lstlisting}

\noindent We can then default just the (non-inline) implementation in a
corresponding{\cprotect\footnote{In practice, every \texttt{.cpp} file
(other than the one containing \texttt{main}) typically has a unique
associated header (\texttt{.h}) file and often vice versa  (with the \texttt{.cpp} and \texttt{.h} pair of files constituting a component); see \textbf{{lakos20}}, sections 1.6 and 1.11,
  pages 209--216 and 256--259, respectively.}}~\texttt{.cpp} file:

\begin{lstlisting}[language=C++]
// largescale.cpp:
#include <largescale.h>

LargeScale::LargeScale() = default;
    // Generate the default implementation for this default destructor.
\end{lstlisting}

\noindent Using this \emph{insulation} technique, we are free to change our minds
and implement the default constructor ourselves in any way we see fit
without necessarily forcing our clients to recompile.

\subsection[Potential Pitfalls]{Potential Pitfalls}\label{potential-pitfalls}

\subsubsection[Generation of defaulted functions is not guaranteed]{Generation of defaulted functions is not guaranteed}\label{generation-of-defaulted-functions-is-not-guaranteed}

Use of \texttt{=}~\texttt{default} does not guarantee that the special
member function of a type, \texttt{T}, will be generated. For example, a
noncopyable member variable (or base class) of \texttt{T} will inhibit
generation of \texttt{T}'s copy constructor even when
\texttt{=}~\texttt{default} is used. Such behavior can be observed in
the presence of a \texttt{std::unique\_ptr}{\cprotect\footnote{\texttt{std::unique\_ptr<T>}
is a move-only (movable but noncopyable) class template introduced in
C++11. It models unique ownership over a dynamically allocated
  \texttt{T} instance, leveraging rvalue references (see Section~\ref{Rvalue-References}, ``\titleref{Rvalue-References}") to represent ownership transfer between instances:
  %%% question

  \begin{lstlisting}[language=C++, basicstyle={\ttfamily\footnotesize}]
  int* p = new int(42);
  std::unique_ptr<int> up(p);                   // OK, take ownership of (ù{\codeincomments{p}}ù).
  std::unique_ptr<int> upCopy = up;             // compile-time error
  std::unique_ptr<int> upMove = std::move(up);  // OK, transfer ownership.
  \end{lstlisting}
      }} data member:

\begin{lstlisting}[language=C++]
class Connection
{
private:
    std::unique_ptr<Impl> d_impl;  // noncopyable data member

public:
    Connection() = default;
    Connection(const Connection&) = default;
};
\end{lstlisting}

\noindent Despite the defaulted copy constructor, \texttt{Connection} will not be
copy-constructible as\linebreak[4] \texttt{std::unique\_ptr} is a noncopyable type.
Some compilers \emph{may} produce a warning on the declaration of
{\texttt{Connection(const} \texttt{Connection\&)}, but they are not
required to do so since the example code above is well formed and would produce a
compilation failure only if an attempt were made to default-construct or
copy a \texttt{Connection}.{\cprotect\footnote{Clang 8.x
and later produces a diagnostic with no warning flags specified.
MSVC produces a diagnostic if \texttt{/Wall} is specified. As of this writing, GCC produces no warning, even with both
  \texttt{-Wall} and \texttt{-Wextra} enabled.}}

If desired, a possible way to ensure that a defaulted special member
function has indeed been generated is to use
{\texttt{static\_assert}} (see Section~\ref{compile-time-assertions-(static_assert)}, ``\titleref{compile-time-assertions-(static_assert)}")
in conjunction with an
appropriate trait from the \texttt{<type\_traits>} header:

\begin{lstlisting}[language=C++]
class IdCollection
{
    std::vector<int> d_ids;

public:
    IdCollection() = default;
    IdCollection(const IdCollection&) = default;
    // ...
};

static_assert(std::is_default_constructible<IdCollection>::value,
              "IdCollection must be default constructible.");

static_assert(std::is_copy_constructible<IdCollection>::value,
              "IdCollection must be copy constructible.");

// ...
\end{lstlisting}

\noindent Routine use of such compile-time testing techniques can help to ensure
that a type will continue to behave as expected (at no additional
runtime cost) even when member and base types evolve as a result
of ongoing software maintenance.

\subsection{Annoyances}

\subsubsection[Defaulted special member functions cannot restore trivial copyability]{Defaulted special member functions cannot restore trivial copyability}

Library classes often rely on whether the type on which they are operating is eligible for being copied with \lstinline!memcpy! for optimization purposes.  Such could be the case for implementing, say, \lstinline!vector!, which would make a single call to \lstinline!memcpy! when growing its buffer.  For the \lstinline!memcpy! or \lstinline!memmove! to be well-defined, however, the type of the object that is stored in the buffer must be \romeogloss{trivially copyable}.  One might assume that this trait means that, as long as the copy constructor of the type is trivial, this optimization will apply.  Defaulting the copy operations would then allow us to achieve this goal, while allowing the type to have a non-trivial destructor or move operation.  Such, however, is not the case.

The requirements for a type to be considered \romeogloss{trivially copyable} --- and thus eligible for use with \lstinline!memcpy! --- include triviality of all of its non-deleted copy and move operations as well as of its destructor.  Furthermore, library authors cannot perform fine-grained dispatch based on which operations on the type are in fact trivial.  Even if we detect that the type is trivially copy-constructible with the \lstinline!std::is_trivially_copy_constructible! trait and know that our code would use only copy constructors (and not copy assignment nor any move operations), we still would not be able to use \lstinline!memcpy! unless the more restrictive \lstinline!std::is_trivially_copyable! trait is also \lstinline!true!.

\subsection[See Also]{See Also}\label{see-also}

\begin{itemize}
\item{Section~\ref{Rvalue-References}, ``\titleref{Rvalue-References}" — Conditionally Safe C++11 feature that is the foundation of \textbf{move semantics} — the move-constructor and move-assignment special member functions can be defaulted}
\item{Section~\ref{deleted-functions}, ``\titleref{deleted-functions}" — Safe C++11 feature that, among other use cases, allows the prevention of generation of special member functions, providing fine-grained control over the interface of a class if used in conjunction with \texttt{=}~\texttt{default}}
\item{Section~\ref{compile-time-assertions-(static_assert)}, ``\titleref{compile-time-assertions-(static_assert)}" — Safe C++11 feature that checks a predicate at compile time; useful to verify that a class’s special copy and move operations are available as expected}
\end{itemize}

\subsection[Further Reading]{Further Reading}\label{further-reading}

\begin{itemize}
\item{Howard Hinnant, ``Everything You Ever Wanted to Know About Move Semantics (and Then Some)," \textbf{hinnant14}}
\item{Howard Hinnant, ``Everything You Ever Wanted to Know About Move Semantics," \textbf{hinnant16}}
\end{itemize}

\subsection[Appendix: Implicit Generation of Special Member Functions]{Appendix: Implicit Generation of Special Member Functions}\label{appendix:-implicit-generation-of-special-member-functions}

The rules a compiler uses to decide if a special member function
should be generated implicitly are not entirely intuitive. Howard
Hinnant, lead designer and author of the C++11 proposal for move
semantics\footnote{\textbf{hinnant02}} (among other proposals), produced a tabular representation of
such rules in the situation where the user provides a single special
member function and leaves the rest to the compiler. To understand Table~\ref{default-table1}, after picking a special member function in the first
column, the corresponding row will show what is implicitly generated by
the compiler.

\begin{table}[h!]  %%% PRODUCTION: REQUIRES REPRINT PERMISSION;
\begin{center}
\begin{threeparttable}
\caption{Implicit generation of special member functions. NEEDS A CREDIT LINE TO HINNANT.}\label{default-table1}
{\footnotesize %%% setting smaller to fit
\begin{tabular}{p{.127\textwidth}|p{.1\textwidth}|p{.11\textwidth}|p{.1\textwidth}|p{.127\textwidth}|p{.1\textwidth}|p{.127\textwidth}}
\thickhline
% header
\rowcolor[gray]{.9}    & {\sffamily\bfseries Default\linebreak[4] Ctor} & {\sffamily\bfseries Destructor} & {\sffamily\bfseries Copy\linebreak[4] Ctor} & {\sffamily\bfseries Copy\linebreak[4] Assignment} & {\sffamily\bfseries Move\linebreak[4] Ctor} & {\sffamily\bfseries Move\linebreak[4] Assignment} \tabularnewline \hline
 % row 1
\cellcolor[gray]{.9}{\sffamily\bfseries Nothing} & Defaulted & Defaulted & Defaulted & Defaulted
& Defaulted & Defaulted \\ \hline
 % row 2
 \cellcolor[gray]{.9}{\sffamily\bfseries Any\linebreak[4] Ctor} & Not\linebreak[4] Declared &
Defaulted & Defaulted & Defaulted & Defaulted & Defaulted \\ \hline
 % row 3
 \cellcolor[gray]{.9}{\sffamily\bfseries Default\linebreak[4] Ctor} & User\linebreak[4] Declared & Defaulted & Defaulted &
Defaulted & Defaulted & Defaulted \\ \hline
 % row 4
 \cellcolor[gray]{.9}{\sffamily\bfseries Destructor} &
Defaulted & User\linebreak[4] Declared & Defaulted\tnote{a} & Defaulted\tnote{a} & Not\linebreak[4] Declared &
Not\linebreak[4] Declared \tabularnewline \hline
 % row 5
 \cellcolor[gray]{.9}{\sffamily\bfseries Copy\linebreak[4] Ctor} & Not\linebreak[4] Declared & Defaulted &
User\linebreak[4] Declared & Defaulted\tnote{a} & Not\linebreak[4] Declared & Not\linebreak[4] Declared \tabularnewline \hline
 % row 6
 \cellcolor[gray]{.9}{\sffamily\bfseries Copy\linebreak[4] Assignment} & Defaulted & Defaulted & Defaulted\tnote{a} & User\linebreak[4]
Declared & Not\linebreak[4] Declared & Not\linebreak[4] Declared \tabularnewline \hline
 % row 7
 \cellcolor[gray]{.9}{\sffamily\bfseries Move\linebreak[4] Ctor} &
Not\linebreak[4] Declared & Defaulted & Deleted & Deleted & User\linebreak[4] Declared & Not\linebreak[4]
Declared \tabularnewline \hline
 % row 8
 \cellcolor[gray]{.9}{\sffamily\bfseries Move\linebreak[4] Assignment} & Defaulted & Defaulted & Deleted
& Deleted & Not\linebreak[4] Declared & User\linebreak[4] Declared \tabularnewline
\thickhline
\end{tabular}
} % end small
\begin{tablenotes}{\footnotesize
\item[a]{Deprecated behavior: compilers might warn upon reliance of this
implicitly generated member \mbox{function}.}
} % end footnotesize
\end{tablenotes} % we end the table footnotes
\end{threeparttable}
\end{center}
\end{table}

As an example, explicitly declaring a \emph{copy-assignment} operator
would result in the \emph{default constructor}, \emph{destructor}, and
\emph{copy constructor} being defaulted and in the \emph{move}
operations not being declared. If more than one \emph{special member
function} is user declared (regardless of whether or how it is
implemented), the remaining generated member functions are those in the
intersection of the corresponding rows. For example, explicitly
declaring both the destructor and the default constructor would still
result in the copy constructor and the copy-assignment operator being
defaulted and both move operations not being declared. Relying on the
compiler-generated copy operations when the destructor is anything but
defaulted is dubious; if correct, defaulting them explicitly makes both
their existence and intended definition clear.



