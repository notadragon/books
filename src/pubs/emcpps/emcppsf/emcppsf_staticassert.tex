%%%% 15 Jan: copyedits in and proofed
%% 27 Jan, code checked and replaced as needed by Josh
% 22 March 2021, in FPPs

\emcppsFeature{
    short={\lstinline!static_assert!},
    tocshort={\TOCCode static\_assert},
    long={Compile-Time Assertions},
    rhshort={\RHCode static\_assert},
}{compile-time-assertions-(static_assert)}
%\section[{\tt static\_assert}]{Compile-Time Assertions\sectionmark{\RHCode static\_assert}}\label{compile-time-assertions-(static_assert)}\sectionmark{\RHCode static\_assert}
\setcounter{table}{0}
\setcounter{footnote}{0}
\setcounter{lstlisting}{0}


The \lstinline!static_assert! keyword allows programmers to intentionally terminate compilation whenever a given compile-time predicate
 evaluates to \lstinline!false!.

\subsection[Description]{Description}\label{description}

Assumptions are
inherent in every program, whether we explicitly document them or not. A common way of validating certain
assumptions at run time is to use the classic \lstinline!assert! macro found
in \lstinline!<cassert>!. Such runtime assertions are not always ideal
because (1) the program must already be built and running for them to
even have a chance of being triggered and (2) executing a
\emcppsgloss{redundant check} at run time typically{\cprotect\footnote{It is
not unheard of for a program having runtime assertions to run faster with them
enabled than disabled. For example, asserting that a pointer is not
null enables the optimizer to elide all code branches that
  can be reached only if that pointer were null.}} results in a slower
program. Being able to validate an assertion at compile time avoids
several drawbacks:

\begin{enumerate}
\item{Validation occurs at compile time within a single translation unit and therefore doesn’t need to wait until a complete program is linked and executed.}
\item{Compile-time assertions can exist in many more places than runtime assertions and are unrelated to program control flow.}
\item{No runtime code will be generated due to a \lstinline!static_assert!, so program performance will not be impacted.}
\end{enumerate}

\subsubsection[Syntax and semantics]{Syntax and semantics}\label{syntax-and-semantics}

We can use \emcppsgloss{static assertion declarations} to conditionally trigger
controlled compilation failures depending on the truthfulness of a
\emcppsgloss{constant expression}. Such declarations are introduced by the
\lstinline!static_assert! keyword, followed by a parenthesized list
consisting of (1) a constant Boolean expression and (2) a mandatory (see
%{\it\titleref{static-annoyances}: \titleref{mandatory-string-literal}} on page~\pageref{mandatory-string-literal})
\intraref{static-annoyances}{mandatory-string-literal})
\emcppsgloss{string literal}, which will be
part of the compiler diagnostics if the compiler determines that the
assertion fails to hold:

\begin{emcppslisting}[emcppserrorlines=2]
static_assert(true, "Never fires.");
static_assert(false, "Always fires.");
\end{emcppslisting}

\newpage%%%%
\noindent Static assertions can be placed anywhere in the scope of a namespace,
block, or class:

\begin{emcppshiddenlisting}[emcppsbatch=e1]
template <typename T> struct Predicate
{
   constexpr static int value = 1;
};
\end{emcppshiddenlisting}
\begin{emcppslisting}[emcppsbatch=e1,emcppserrorlines=8]
static_assert(1 + 1 == 2, "Never fires.");  // (global) namespace scope

template <typename T>
struct S
{
    void f0()
    {
        static_assert(1 + 1 == 3, "Always fires.");  // block scope
    }

    static_assert(!Predicate<T>::value, "Might fire.");  // class scope
};
\end{emcppslisting}

\noindent Providing a nonconstant expression to a \lstinline!static_assert! is
itself a compile-time error:

\begin{emcppslisting}
extern bool x;
static_assert(x, "Nice try.");  // Error, (ù{\codeincomments{x}}ù) is not a compile-time constant.
\end{emcppslisting}


\subsubsection[Evaluation of static assertions in templates]{Evaluation of static assertions in templates}\label{evaluation-of-static-assertions-in-templates}

The C++ Standard does not explicitly specify at precisely what point during
the compilation process the expressions tested by static assertion declarations are
evaluated. In particular, when used within the body of a template, the expression tested by a
\lstinline!static_assert! declaration might not be evaluated until
\emcppsgloss{template instantiation time}. In practice, however, a
\lstinline!static_assert! that does not depend on any template parameters
is essentially always{\cprotect\footnote{E.g.,
GCC~10.1, Clang~10.0, and MSVC~19.24}} evaluated immediately --- i.e., as
soon as it is parsed and irrespective of whether any subsequent template
instantiations occur:

\begin{emcppslisting}[emcppserrorlines={3,9}]
void f1()
{
    static_assert(false, "Impossible!");  // always evaluated immediately...
}                                         // even if (ù{\codeincomments{f1()}}ù) is never invoked

template <typename T>
void f2()
{
    static_assert(false, "Impossible!");  // always evaluated immediately...
}                                         // even if (ù{\codeincomments{f2()}}ù) is never instantiated
\end{emcppslisting}

\noindent The evaluation of a static assertion that is located within the
body of a class or function template and depends on at least one
template parameter is almost always bypassed during its initial parse since
the assertion predicate might evaluate to true or false depending on the template argument:

\begin{emcppshiddenlisting}[emcppsbatch=e2]
#include <complex>  // (ù{\codeincomments{std::complex}}ù)
\end{emcppshiddenlisting}
\begin{emcppslisting}[emcppsbatch=e2]
template <typename T>
void f3()
{
    static_assert(sizeof(T) >= 8, "Size < 8.");  // depends on (ù{\codeincomments{T}}ù)
}
\end{emcppslisting}

\noindent However, see
%{\it\titleref{static-potential-pitfalls}: \titleref{static-assertions-in-templates-can-trigger-unintended-compilation-failures}} on page~\pageref{static-potential-pitfalls}.
\intraref{static-potential-pitfalls}{static-assertions-in-templates-can-trigger-unintended-compilation-failures}.
In the example above, the compiler has no choice but to wait until each
time \lstinline!f3! is instantiated because the truth of the predicate will
vary depending on the type provided:

\begin{emcppslisting}[emcppsbatch=e2]
void g()
{
    f3<double>();               // OK
    f3<long double>();          // OK
    f3<std::complex<float>>();  // OK
    f3<char>();                 // Error, static assertion failed: Size < 8.
}
\end{emcppslisting}

\noindent The standard does, however, specify that a program containing any
template definition for which no valid specialization exists is
\emcppsgloss{ill formed, no diagnostic required (IFNDR)}, which was the case for
\lstinline!f2! but not \lstinline!f3!, above. Contrast each of the
\lstinline!h*!\lstinline[basicstyle=\ttfamily\itshape]!n!\lstinline!*! definitions below with its correspondingly numbered
\lstinline!f*!\lstinline[basicstyle=\ttfamily\itshape]!n!\lstinline!*!  definition above\footnote{The formula used --- \lstinline!int!~\lstinline!a[-1];! --- leads to $-1$, not $0$, to avoid a nonconforming extension to GCC that allows \lstinline!a[0]!.}:

\begin{emcppslisting}
void h1()
{
    int a[!sizeof(int) - 1];  // Error, same as (ù{\codeincomments{int a[-1];}}ù)
}

template <typename T>
void h2()
{
    int a[!sizeof(int) - 1];  // Error, always reported
}

template <typename T>
void h3()
{
    int a[!sizeof(T) - 1];    // typically reported only if instantiated
}
\end{emcppslisting}

\noindent Both \lstinline!f1! and \lstinline!h1! are ill-formed, nontemplate functions,
and both will always be reported at compile time, albeit typically with
decidedly different error messages as demonstrated by GCC 10.x's output:

\enlargethispage*{1ex}
\begin{lstlisting}[language=bash,style=plain]
f1: error: static assertion failed: Impossible!
h1: error: size (ù{\codeincomments{-1}}ù) of array (ù{\codeincomments{a}}ù) is negative
\end{lstlisting}

\noindent Both \lstinline!f2! and \lstinline!h2! are ill-formed template functions; the cause of their being ill-formed has nothing to do with the
template type and hence will always be reported as a compile-time error
in practice. Finally, \lstinline!f3! can be only contextually ill-formed,
whereas \lstinline!h3! is always necessarily ill-formed, and yet neither is
reported by typical compilers as such unless and until it has been
instantiated. Reliance on a compiler not to notice that a program is
ill-formed is dubious; see
%{\it\titleref{static-potential-pitfalls}: \titleref{static-assertions-in-templates-can-trigger-unintended-compilation-failures}} on page~\pageref{static-assertions-in-templates-can-trigger-unintended-compilation-failures}.
\intraref{static-potential-pitfalls}{static-assertions-in-templates-can-trigger-unintended-compilation-failures}.

\subsection[Use Cases]{Use Cases}\label{use-cases}

\subsubsection[Verifying assumptions about the target platform]{Verifying assumptions about the target platform}\label{verifying-assumptions-about-the-target-platform}

Some programs rely on specific properties of the native types provided
by their target platform. Static assertions can help ensure portability
and prevent such programs from being compiled into a malfunctioning
binary on an unsupported platform. As an example, consider a
program that relies on the size of an \lstinline!int! to be exactly 32
bits (e.g., due to the use of inline \lstinline!asm! blocks). Placing a
\lstinline!static_assert! in namespace scope in any of the program's
translation units will ensure that the assumption regarding the size
of \lstinline!int! is valid, and also serve as documentation for readers:

\begin{emcppslisting}
#include <climits>  // (ù{\codeincomments{CHAR\_BIT}}ù)

static_assert(sizeof(int) * CHAR_BIT == 32,
    "An (ù{\codeincomments{int}}ù) must have exactly 32 bits for this program to work correctly.");
\end{emcppslisting}

\noindent More typically, statically asserting the \emph{size} of an \lstinline!int!
avoids having to write code to handle an \lstinline!int! type's having
greater or fewer bytes when no such platforms are likely ever to
materialize:

\begin{emcppslisting}
static_assert(sizeof(int) == 4, "An (ù{\codeincomments{int}}ù) must have exactly 4 bytes.");
\end{emcppslisting}


\subsubsection[Preventing misuse of class and function templates]{Preventing misuse of class and function templates}\label{preventing-misuse-of-class-and-function-templates}

Static assertions are often used in practice to constrain class or function templates to prevent their being instantiated with unsupported types. If a type is not syntactically compatible with the template, static assertions provide clear customized error messages that replace compiler-issued diagnostics, which are often absurdly long and notoriously hard-to-read. More critically, static assertions actively avoid erroneous runtime behavior.
%%%  VR WANTS TO COME BACK TO THIS AND ADD A SEE ALSO.

As an example, consider the \lstinline!SmallObjectBuffer<N>! class
templates, which provide storage, aligned properly using \lstinline!alignas! (see
%``\titleref{alignas}" on page~\pageref{alignas}),
\featureref{\locationc}{alignas}),
for arbitrary objects whose size does
not exceed \lstinline!N!{\cprotect\footnote{A \lstinline!SmallObjectBuffer! is
  similar to C++17's \lstinline!std::any! (\cite{cpprefstdany}) in
  that it can store any object of any type. Instead of performing
  dynamic allocation to support arbitrarily sized objects, however,
  \lstinline!SmallObjectBuffer! uses an internal fixed-size buffer, which
  can lead to better performance and cache locality provided the
  maximum size of all of the types involved is known.}}:

%begin{emcppslisting}
%template <std::size_t N>
%class SmallObjectBuffer
%{
%private:
%    char d_buffer[N];
%
%public:
%    template <typename T>
%    void set(const T& object);
%
%    // ...
%};
%\end{emcppslisting}
\begin{emcppslisting}[emcppsbatch=e3]
#include <cstddef> // (ù{\codeincomments{std::size\_t}}ù), (ù{\codeincomments{std::max\_align\_t}}ù)
#include <new>     // placement (ù{\codeincomments{new}}ù)

template <std::size_t N>
class SmallObjectBuffer
{
private:
    alignas(std::max_align_t) char d_buffer[N];

public:
    template <typename T>
    void set(const T& object);

    // ...
};
\end{emcppslisting}


\noindent To prevent buffer overruns, it is important that \lstinline!set! accepts
only those objects that will fit in \lstinline!d_buffer!. The use of a
static assertion in the \lstinline!set! member function template catches
--- at compile time --- any such misuse:

%begin{emcppslisting}
%template <std::size_t N>
%template <typename T>
%void SmallObjectBuffer<N>::set(const T& object)
%{
%    static_assert(sizeof(T) <= N, "(ù{\codeincomments{object}}ù) does not fit in the small buffer.");
%    new (&d_buffer) T(object);
%}
%\end{emcppslisting}
\begin{emcppslisting}[emcppsbatch=e3]
template <std::size_t N>
template <typename T>
void SmallObjectBuffer<N>::set(const T& object)
{
    static_assert(sizeof(T) <= N, "(ù{\codeincomments{object}}ù) does not fit in the small buffer.");
    // Destroy existing object, if any; store how to destroy this new object of
    // type (ù{\codeincomments{T}}ù) later; then...
    new (&d_buffer) T(object);
}
\end{emcppslisting}



The principle of constraining inputs can be applied to most class and
function templates. \lstinline!static_assert! is particularly useful in
conjunction with standard \emcppsgloss[type trait]{type traits} provided in
\lstinline!<type_traits>!. In the \lstinline!rotateLeft! function template
(below), we have used two static assertions to ensure that only unsigned
integral types will be accepted:

\begin{emcppslisting}
#include <climits>      // (ù{\codeincomments{CHAR\_BIT}}ù)
#include <type_traits>  // (ù{\codeincomments{std::is\_integral}}ù), (ù{\codeincomments{std::is\_unsigned}}ù)

template <typename T>
T rotateLeft(T x)
{
    static_assert(std::is_integral<T>::value, "(ù{\codeincomments{T}}ù) must be an integral type.");
    static_assert(std::is_unsigned<T>::value, "(ù{\codeincomments{T}}ù) must be an unsigned type.");

    return (x << 1) | (x >> (sizeof(T) * CHAR_BIT - 1));
}
\end{emcppslisting}


\subsection[Potential Pitfalls]{Potential Pitfalls}\label{static-potential-pitfalls}

\subsubsection[Static assertions in templates can trigger unintended compilation failures]{Static assertions in templates can trigger unintended compilation failures}\label{static-assertions-in-templates-can-trigger-unintended-compilation-failures}

As mentioned in the description, any program containing a template for
which no valid specialization can be generated is
\emcppsgloss[ill formed, no diagnostic required (IFNDR)]{IFNDR}. Attempting to prevent the
use of, say, a particular function template overload by using a static
assertion that never holds produces such a program:

\begin{emcppslisting}[emcppserrorlines=10]
template <bool>
struct SerializableTag { };

template <typename T>
void serialize(char* buffer, const T& object, SerializableTag<true>);  // (1)

template <typename T>
void serialize(char* buffer, const T& object, SerializableTag<false>)  // (2a)
{
    static_assert(false, "T must be serializable.");  // independent of (ù{\codeincomments{T}}ù)
        // too obviously ill-formed: always a compile-time error
}
\end{emcppslisting}

\noindent In the example above, the second overload (2a) of \lstinline!serialize! is
provided with the intent of eliciting a meaningful compile-time error
message in the event that an attempt is made to serialize a
nonserializable type. The program, however, is technically
\emcppsgloss{ill formed} and, in this simple case, will likely result in a
compilation failure --- irrespective of whether either overload of
\lstinline!serialize! is ever instantiated.

A commonly attempted workaround
is to make the predicate of the assertion somehow dependent on a
template parameter, ostensibly forcing the compiler to withhold
evaluation of the \lstinline!static_assert! unless and until the template
is actually instantiated (a.k.a. \emcppsgloss{instantiation time}):

\begin{emcppshiddenlisting}[emcppsbatch=e4]
template <bool>
struct SerializableTag { };
\end{emcppshiddenlisting}
\begin{emcppslisting}[emcppsbatch=e4]
template <typename>  // N.B., we make no use of the (nameless) type parameter:
struct AlwaysFalse   // This class exists only to "outwit" the compiler.
{
    enum { value = false };
};

template <typename T>
void serialize(char* buffer, const T& object, SerializableTag<false>)  // (2b)
{
    static_assert(AlwaysFalse<T>::value, "T must be serializable.");  // OK
        // less obviously ill-formed: compile-time error when instantiated
}
\end{emcppslisting}

\noindent To implement this version of the second overload, we have provided an
intermediary class template \lstinline!AlwaysFalse! that, when instantiated
on any type, contains an enumerator named \lstinline!value!, whose value is
\lstinline!false!. Although this second implementation is more likely to
produce the desired result (i.e., a controlled compilation failure
only when \lstinline!serialize! is invoked with unsuitable arguments),
sufficiently ``smart'' compilers looking at just the current translation
unit would still be able to know that no valid instantiation of
\lstinline!serialize! exists and would therefore be well within their
rights to refuse to compile this still technically \emcppsgloss{ill formed}
program.

Equivalent workarounds achieving the same result without a
helper class are possible.

\begin{emcppshiddenlisting}[emcppsbatch=e5]
template <bool>
struct SerializableTag { };
\end{emcppshiddenlisting}
\begin{emcppslisting}[emcppsbatch=e5]
template <typename T>
void serialize(char* buffer, const T& object, SerializableTag<false>)  // (2c)
{
    static_assert(0 == sizeof(T), "T must be serializable.");  // OK
         // not too obviously ill-formed: compile-time error when instantiated
}
\end{emcppslisting}

\noindent Using this sort of obfuscation is not guaranteed to be either
portable or future-proof.

\subsubsection[Misuse of static assertions to restrict overload sets]{Misuse of static assertions to restrict overload sets}\label{misuse-of-static-assertions-to-restrict-overload-sets}

Even if we are careful to \emph{fool} the compiler into thinking that a
specialization is wrong \emph{only} if instantiated, we still cannot use
this approach to remove a candidate from an overload set because translation
will terminate if the static assertion is triggered. Consider this
flawed attempt at writing a \lstinline!process! function that will behave
differently depending on the size of the given argument:

\begin{emcppslisting}[emcppserrorlines={8,9,10,11,12,13}]
template <typename T>
void process(const T& x)  // (1) first definition of (ù{\codeincomments{process}}ù) function
{
    static_assert(sizeof(T) <= 32, "Overload for small types");  // BAD IDEA
    // ... (process small types)
}

template <typename T>
void process(const T& x)  // (2) compile-time error: redefinition of function
{
    static_assert(sizeof(T) > 32, "Overload for big types");     // BAD IDEA
    // ... (process big types)
}
\end{emcppslisting}

\noindent While the intention of the developer might have been to statically
dispatch to one of the two mutually exclusive overloads, the ill-fated
implementation above will not compile because the signatures of the two
overloads are identical, leading to a redefinition error. The semantics
of \lstinline!static_assert! are not suitable for the purposes of
\emcppsgloss[compile time dispatch]{compile-time dispatch}, and \emcppsgloss{SFINAE}-based approaches should be used instead.

\enlargethispage*{1ex}
To achieve the goal of removing up
front a specialization from consideration, we will need~to employ
\emcppsgloss{SFINAE}. To do that, we must instead find a way to get the
failing compile-time expression to be part of the function's
\emcppsgloss{declaration}:

\begin{emcppslisting}[emcppsbatch=e6]
template <bool> struct Check { };
    // helper class template having a (non-type) boolean template parameter
    // representing a compile-time predicate

template <> struct Check<true> { typedef int Ok; };
    // specialization of (ù{\codeincomments{Check}}ù) that makes the type (ù{\codeincomments{Ok}}ù) manifest *only* if
    // the supplied predicate (boolean template argument) evaluates to (ù{\codeincomments{true}}ù)

template <typename T,
          typename Check<(sizeof(T) <= 32)>::Ok = 0>  // SFINAE
void process(const T& x)  // (1)
{
    // ... (process small types)
}

template <typename T,
          typename Check<(sizeof(T) > 32)>::Ok = 0>  // SFINAE
void process(const T& x)  // (2)
{
    // ... (process big types)
}
\end{emcppslisting}

\noindent The empty \lstinline!Check! helper class template above in conjunction with
just one of its two possible specializations conditionally
exposes the \lstinline!Ok! type alias \emph{only} if the provided boolean
template parameter evaluates to \lstinline!true!. (Otherwise, by default, it
does not.) C++11 provides a library function, \lstinline!std::enable_if!, that more directly addresses this use case.\footnote{\emcppsgloss[concepts]{Concepts} --- a
language feature introduced in C++20 --- provides a far less baroque
alternative to \emcppsgloss{SFINAE} that allows for overload sets to be governed by
  the syntactic properties of their (compile-time) template arguments.}

During the substitution phase of template instantiation,
exactly one of the two overloads of the \lstinline!process! function will
attempt to access a nonexisting \lstinline!Ok! type alias via the
\lstinline!Check<false>! instantiation, which again, by default, is
nonexistent. Although such an error would typically result in a
compilation failure, in the context of template argument substitution it
will instead result in only the offending overload's being discarded,
giving other valid overloads a chance to be selected:

\begin{emcppshiddenlisting}[emcppsbatch=e6]
struct SmallType { char d_data[16]; };
struct BigType   { char d_data[32]; };
\end{emcppshiddenlisting}
\begin{emcppslisting}[emcppsbatch=e6]
void client()
{
    process(SmallType());  // discards (2), selects (1)
    process(BigType());    // discards (1), selects (2)
}
\end{emcppslisting}

\noindent This general technique of pairing template specializations is used widely
in modern C++ programming. For another, often more convenient way of
constraining overloads using \emcppsgloss{expression SFINAE}, see
%Section~\ref{trailing-function-return-types}, ``\titleref{trailing-function-return-types} on page~\pageref{trailing-function-return-types}."
\featureref{\locationa}{trailing-function-return-types}.

\subsection[Annoyances]{Annoyances}\label{static-annoyances}

\subsubsection[Mandatory string literal]{Mandatory string literal}\label{mandatory-string-literal}

Many compilation failures caused by static assertions are
self-explanatory since the offending line (which necessarily contains the
predicate code) is displayed as part of the compiler diagnostic. In
those situations, the message required{\cprotect\footnote{As of C++17,
  the message argument of a static assertion is optional.}} as part of
\lstinline!static_assert!'s grammar is redundant:

\begin{emcppshiddenlisting}[emcppsbatch=e7]
#include <type_traits>  // (ù{\codeincomments{std::is\_integral}}ù)
using T = int;
\end{emcppshiddenlisting}
\begin{emcppslisting}[emcppsbatch=e7]
static_assert(std::is_integral<T>::value, "(ù{\codeincomments{T}}ù) must be an integral type.");
\end{emcppslisting}

\noindent Developers commonly provide an empty string literal in these
cases:

\begin{emcppslisting}[emcppsbatch=e7]
static_assert(std::is_integral<T>::value, "");
\end{emcppslisting}

There is no universal consensus as to the ``parity" of the user-supplied error message.  Should it restate the asserted condition, or should it state what went amiss?
\begin{emcppshiddenlisting}[emcppsbatch=e8]
constexpr int x = 1;
\end{emcppshiddenlisting}
\begin{emcppslisting}[emcppsbatch=e8]
static_assert(0 < x, "x is negative");
   // misleading when (ù{\codeincomments{0 == x}}ù)
\end{emcppslisting}

\subsection[See Also]{See Also}\label{see-also}

\begin{itemize}
\item{%Section~\ref{trailing-function-return-types}, ``\titleref{trailing-function-return-types}" on page~\pageref{trailing-function-return-types} —
\seealsoref{trailing-function-return-types}{\locationa}Enabling expression \emcppsgloss{SFINAE} directly as part of a function's \emcppsgloss{declaration}  allows simple and fine-grained control over overload resolution.}
\end{itemize}

\subsection[Further Reading]{Further Reading}\label{further-reading}

\begin{itemize}
\item{\cite{klarer04}}
\end{itemize}


