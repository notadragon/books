% 13 Jan: copyedits in and proofed
%% 27 Jan, code checked and replaced as needed by Josh

\emcppsFeature{
    short={{\lstinline!nullptr!}},
    tocshort={{\TOCCode nullptr}},
    long={The Null-Pointer-Literal Keyword},
    rhshort={\RHCode nullptr},
}{null-pointer-literal-(nullptr)}
%\section[{\tt nullptr}]{The Null-Pointer-Literal Keyword\sectionmark{\RHCode nullptr}}\label{null-pointer-literal-(nullptr)}\sectionmark{\RHCode nullptr}
\setcounter{table}{0}
\setcounter{footnote}{0}
\setcounter{lstlisting}{0}



The keyword \lstinline!nullptr! unambiguously denotes the
null-pointer-value literal.

\subsection[Description]{Description}\label{description}

The \lstinline!nullptr! keyword is a \romeovalue{prvalue} (pure rvalue) of type
\lstinline!std::nullptr_t! representing the implementation-defined
bit pattern corresponding to a \emcppsgloss{null address} on the host platform;
\lstinline!nullptr! and other values of type \lstinline!std::nullptr_t!, along with the integer literal \lstinline!0! and the macro \lstinline!NULL!, can be converted implicitly to any pointer or pointer-to-member type:

\begin{emcppslisting}
#include <cstddef> // (ù{\codeincomments{NULL}}ù)
int data;  // nonmember data

int *pi0 = &data;    // Initialize with non-null address.
int *pi1 = nullptr;  // Initialize with null address.
int *pi2 = NULL;     //  "          "    "    "
int *pi3 = 0;        //  "          "    "    "

double f(int x);  // nonmember function

double (*pf0)(int) = &f;       // Initialize with non-null address.
double (*pf1)(int) = nullptr;  // Initialize with null address.

struct S
{
    short d_data;    // member data
    float g(int y);  // member function
};

short S::*pmd0 = &S::d_data;  // Initialize with non-null address.
short S::*pmd1 = nullptr;     // Initialize with null address.

float (S::*pmf0)(int) = &S::g;    // Initialize with non-null address.
float (S::*pmf1)(int) = nullptr;  // Initialize with null address.
\end{emcppslisting}


\noindent Because \lstinline!std::nullptr_t! is its own distinct type, overloading on
it is possible:

\begin{emcppslisting}
#include <cstddef>  // (ù{\codeincomments{std::nullptr\_t}}ù)

void g(void*);           // (1)
void g(int);             // (2)
void g(std::nullptr_t);  // (3)

void f()
{
    char buf[] = "hello";
    g(buf);      // OK, (1) (ù{\codeincomments{void g(void*)}}ù)
    g(0);        // OK, (2) (ù{\codeincomments{void g(int)}}ù)
    g(nullptr);  // OK, (3) (ù{\codeincomments{void g(std::nullptr\_t)}}ù)
    g(NULL);     // Error, ambiguous --- (1), (2), or (3)
}
\end{emcppslisting}


\subsection[Use Cases]{Use Cases}\label{use-cases}

\subsubsection[Improvement of type safety]{Improvement of type safety}\label{improve-type-safety}

In pre-C++11 codebases, using the \lstinline!NULL!
macro was a common way of indicating, mostly to the human
reader, that the literal value the macro conveys is intended specifically to
represent a \emph{null address} rather than the literal \lstinline!int!
value \lstinline!0!. In the C Standard, the macro \lstinline!NULL! is
defined as an \emcppsgloss[implementation defined]{implementation-defined} integral or \lstinline!void*!
constant. Unlike C, C++ forbids conversions from \lstinline!void*! to
arbitrary pointer types and instead, prior to C++11, defined
\lstinline!NULL! as an ``integral constant expression rvalue of
integer type that evaluates to zero''; any integer literal (e.g.,
\lstinline!0!, \lstinline!0L!, \lstinline!0U!, \lstinline!0LLU!) satisfies this
  criterion. From a type-safety perspective, its
implementation-defined definition, however,
makes using \lstinline!NULL! only marginally better suited than a raw
literal \lstinline!0! to represent a null pointer. It is worth noting that as of C++11, the definition of \lstinline!NULL! has been expanded to --- in theory --- permit \lstinline!nullptr! as a conforming definition; as of this writing, however, no major compiler vendors do so.{\cprotect\footnote{Both GCC and
Clang default to \lstinline!0L! (\lstinline!long!~\lstinline!int!), while MSVC
defaults to \lstinline!0! (\lstinline!int!). Such definitions are unlikely
to change since existing code could cease to compile or (possibly
  silently) present altered runtime behavior.}}

As just one specific illustration of the added type safety provided by
\lstinline!nullptr!, imagine that the coding standards of a large software company historically required that
values returned via output parameters (as opposed to a \lstinline!return!
statement) are always returned via pointer to a modifiable
object. Functions that return via argument typically do so to reserve the function’s return value to communicate status.{\cprotect\footnote{See \cite{lakos96}, section~9.1.11, pp.~621--628, specifically the \emph{Guideline} at the bottom of p.~621: ``Be consistent about returning values through arguments (e.g., avoid declaring non\lstinline!const! reference parameters)."}} A function in this codebase might ``zero" the output parameter's local pointer variable to indicate and ensure that nothing more is to be written. The function below illustrates three different ways of doing this:

\begin{emcppshiddenlisting}[emcppsbatch=e1]
#include <cstdlib>  // (ù{\codeincomments{NULL}}ù)
// --- Replace
    if (/*...*/)
    if (true)
// --- End
\end{emcppshiddenlisting}
\begin{emcppslisting}[emcppsbatch=e1]
int illustrativeFunction(int* x)   // pointer to modifiable integer
{
    // ...
    if (/*...*/)
    {
        x = 0;       // OK, Set pointer (ù{\codeincomments{x}}ù) to null address.
        x = NULL;    // OK, Set pointer (ù{\codeincomments{x}}ù) to null address.
        x = nullptr; // Bug, Set pointer (ù{\codeincomments{x}}ù) to null address.
    }
    // ...
    return 0;    // success
}
\end{emcppslisting}


Now suppose that the function signature is changed (e.g., due to a
change in coding standards in the organization) to accept a reference
instead of a pointer:

\begin{emcppshiddenlisting}[emcppsbatch=e2]
#include <cstdlib>  // (ù{\codeincomments{NULL}}ù)
// --- Replace
    if (/*...*/)
    if (true)
// --- End
\end{emcppshiddenlisting}
\begin{emcppslisting}[emcppsbatch=e2]
int illustrativeFunction(int& x)  // reference to modifiable integer
{
    // ...
    if (/*...*/)
    {
        x = 0;       // OK, always compiles; makes what (ù{\codeincomments{x}}ù) refers to 0
        x = NULL;    // OK, implementation-defined (might warn)
        x = nullptr; // Error, always a compile-time error
    }
    // ...
    return 0;    // SUCCESS
}
\end{emcppslisting}


As the example above demonstrates, how we represent the notion of a
null address matters:
\begin{enumerate}
\item{\lstinline!0! — Portable across all implementations but minimal type safety}
\item{\lstinline!NULL! — Implemented as a macro; added type safety (if any) is platform specific}
\item{\lstinline!nullptr! — Portable across all implementations and fully type-safe}
\end{enumerate}
Using \lstinline!nullptr! instead of \lstinline!0! or \lstinline!NULL! to denote
a null address maximizes type safety and readability, while avoiding
both macros and implementation-defined behavior.

\subsubsection[Disambiguation of \lstinline!(int)0! vs. \lstinline!(T*)0! during overload resolution]{Disambiguation of {\SubsubsecCode (int)0} vs. {\SubsubsecCode (T*)0} during overload resolution}\label{disambiguation-of-(int)-0-versus-(t*)-0-during-overload-resolution}

The platform-dependent nature of \lstinline!NULL! presents additional
challenges when used to call a function whose overloads differ only in
accepting a pointer or an integral type as the same positional argument, which might be the case, e.g., in a poorly designed third-party
library:

\begin{emcppshiddenlisting}[emcppsbatch=e3]
#include <cstdlib>  // (ù{\codeincomments{NULL}}ù)
\end{emcppshiddenlisting}
\begin{emcppslisting}[emcppsbatch=e3]
void uglyLibraryFunction(int* p);  // (1)
void uglyLibraryFunction(int  i);  // (2)
\end{emcppslisting}


\noindent Calling this function with the literal \lstinline!0! will always invoke
overload (2), but that might not always be what casual clients expect:

\begin{emcppslisting}[emcppsbatch=e3]
void f()
{
    uglyLibraryFunction(0);         // unambiguously invokes (2)
    uglyLibraryFunction((int*) 0);  // unambiguously invokes (1)
    uglyLibraryFunction(nullptr);   // unambiguously invokes (1)
    uglyLibraryFunction(NULL);      // Error, anything! (platform-defined)
    uglyLibraryFunction(0L);        // Error, ambiguous call (on all platforms)
    uglyLibraryFunction(0U);        // Error, ambiguous call (on all platforms)
}
\end{emcppslisting}


\noindent\lstinline!nullptr! is especially useful when such problematic overloads
are unavoidable because it obviates explicit
casts. (Note that explicitly casting \lstinline!0! to an
appropriately typed pointer --- other than \lstinline!void*! --- was at one
  time considered by some to be a best practice, especially in C.)

\subsubsection[Overloading for a literal null pointer]{Overloading for a literal null pointer}\label{overloading-for-a-literal-null-pointer}

Being a distinct type, \lstinline!std::nullptr_t! can itself participate
in an overload set:

\begin{emcppslisting}
#include  <cstddef> // (ù{\codeincomments{std::nullptr\_t}}ù)
void f(int* v);          // (1)
void f(std::nullptr_t);  // (2)

void g()
{
    int* ptr = nullptr;
    f(ptr);      // unaemmbiguously invokes (1)
    f(nullptr);  // unambiguously invokes (2)
}
\end{emcppslisting}


\noindent Given the relative ease with which a \lstinline!nullptr! can be
converted to a typed pointer having the same null-address value, such
overloads are dubious when used to control essential behavior.
Nonetheless, we can envision such use to, say, aid in compile-time
diagnostics when passing a \emcppsgloss{null address} would otherwise result in
a runtime error (see
%``\titleref{deleted-functions}" on page~\pageref{deleted-functions}
\featureref{\locationb}{deleted-functions}):

\begin{emcppshiddenlisting}[emcppsbatch=e4]
#include <cstddef>  // (ù{\codeincomments{std::size\_t}}ù)
\end{emcppshiddenlisting}
\begin{emcppslisting}[emcppsbatch=e4]
std::size_t strlen(const char* s);
    // The behavior is undefined unless s is null-terminated.

std::size_t strlen(std::nullptr_t) = delete;
    // Function is not defined but still participates in overload resolution.
\end{emcppslisting}


\noindent Another arguably safe use of such an overload for a \lstinline!nullptr! is
to avoid a null-pointer check. However, for cases where the client knows the address is null at compile time, better ways typically exist for avoiding the (often
insignificant) overhead of testing for a null pointer at run time.

\subsection[Potential Pitfalls]{Potential Pitfalls}\label{potential-pitfalls}

\hspace*{\fill}

\subsection[Annoyances]{Annoyances}\label{annoyances}

\hspace*{\fill}

\subsection[See Also]{See Also}\label{see-also}

\hspace*{\fill}

\subsection[Further Reading]{Further Reading}\label{further-reading}

\begin{itemize}
\item{\cite{meyers15a}}
\end{itemize}

