% 12 Feb 2021, revisions in; ready for Josh's code fixes




\emcppsFeature{
    short={\lstinline!friend! '11},
    tocshort={{\TOCCode friend} '11},
    long={Extended {\SecCode friend} Declarations},
    toclong={Extended \lstinline!friend! Declarations},
    rhshort={{\RHCode friend} '11},
}{extended-friend-declarations}
\setcounter{table}{0}
\setcounter{footnote}{0}
\setcounter{lstlisting}{0}
%\section[{\tt friend} '11]{Extended {\SecCode friend} Declarations}\label{extended-friend-declarations}


Extended \lstinline!friend! declarations enable a class's author to
designate a type alias, a template parameter, or any other previously
declared type as a \lstinline!friend! of that class.

\subsection[Description]{Description}\label{description-extendedfriend}

A \lstinline!friend! declaration located within a given
\romeogloss{user-defined type} (UDT) grants a designated type (or
\emph{free} function) access to private and protected members of that
class. Because the extended \lstinline!friend! syntax does not affect
\emph{function} friendships, this feature section addresses extended
friendship only between \emph{types}.

Prior to C++11, the Standard required an \emph{elaborated type
specifier} to be provided after the \lstinline!friend! keyword to designate
some other \emph{class} as being a \lstinline!friend! of a given type. An
elaborated type specifier for a class is a syntactical element having the form\linebreak[4]
\mbox{\lstinline!<class|struct|union>!~\lstinline!<identifier>!}. Elaborated
type specifiers can be used to refer to a previously declared entity or
to declare a new one, with the restriction that such an entity is
one of \lstinline!class!, \lstinline!struct!, or \lstinline!union!:

\begin{emcppslisting}[language=C++]
// C++03

struct S;
class C;
enum E { };

struct X0
{
    friend S;         // Error: not legal C++98/03
    friend struct S;  // OK, refers to (ù{\codeincomments{S}}ù) above
    friend class S;   // OK, refers to (ù{\codeincomments{S}}ù) above (might warn)
    friend class C;   // OK, refers to (ù{\codeincomments{C}}ù) above
    friend class C0;  // OK, declares (ù{\codeincomments{C0}}ù) in (ù{\codeincomments{X0}}ù)'s namespace
    friend union U0;  // OK, declares (ù{\codeincomments{U0}}ù) in (ù{\codeincomments{X0}}ù)'s namespace
    friend enum E;    // Error: (ù{\codeincomments{enum}}ù) cannot be a friend.
    friend enum E2;   // Error: (ù{\codeincomments{enum}}ù) cannot be forward-declared.
};
\end{emcppslisting}
    
\noindent This restriction prevents other potentially useful entities, e.g., type
aliases and template parameters, from being designated as friends:

\begin{emcppslisting}[language=C++]
// C++03

struct S;
typedef S SAlias;

struct X1
{
    friend struct SAlias;
        // Error: using typedef-name after (ù{\codeincomments{struct}}ù)
};

template <typename T>
struct X2
{
    friend class T;
        // Error: using template type parameter after (ù{\codeincomments{class}}ù)
};
\end{emcppslisting}
    
\noindent Furthermore, even though an entity belonging to a namespace other than
the class containing a \lstinline!friend! declaration might be visible,
explicit qualification is required to avoid unintentionally declaring a
brand-new type:

\begin{emcppslisting}[language=C++]
// C++03

struct S;  // This (ù{\codeincomments{S}}ù) resides in the global namespace.

namespace ns
{
    class X3
    {
        friend struct S;
            // OK, but declares a new (ù{\codeincomments{ns::S}}ù) instead of referring to (ù{\codeincomments{::S}}ù)
    };
}
\end{emcppslisting}
    
\noindent C++11 relaxes the aforementioned \emph{elaborated type specifier}
requirement and extends the classic \lstinline!friend! syntax by instead
allowing either a \emph{simple type specifier}, which is any unqualified
type or type alias, or a \emph{typename specifier}, e.g., the name of a
template \emph{type} parameter or dependent type thereof:

\begin{emcppslisting}[language=C++]
struct S;
typedef S SAlias;

namespace ns
{
    template <typename T>
    struct X4
    {
        friend T;           // OK
        friend S;           // OK, refers to (ù{\codeincomments{::S}}ù)
        friend SAlias;      // OK, refers to (ù{\codeincomments{::S}}ù)
        friend decltype(0); // OK, equivalent to (ù{\codeincomments{friend int;}}ù)
        friend C;           // Error: (ù{\codeincomments{C}}ù) does not name a type.
    };
}
\end{emcppslisting}
    
\noindent Notice that now it is again possible to declare as a \lstinline!friend! a
type that is expected to have already been declared, e.g., \lstinline!S!,
without having to worry that a typo in the spelling of the type would
silently introduce a new type declaration, e.g., \lstinline!C!, in the
enclosing scope.

Finally, consider the hypothetical case in which a class template,
\lstinline!C!, befriends a \emph{dependent} (e.g., nested) type, \lstinline!N!,
of its type parameter, \lstinline!T!:

\begin{emcppslisting}[language=C++]
template <typename T>
class C
{
    friend typename T::N;       // (ù{\codeincomments{N}}ù) is a *dependent* *type* of parameter (ù{\codeincomments{T}}ù).
    enum { e_SECRET = 10022 };  // This information is (ù{\codeincomments{private}}ù) to class (ù{\codeincomments{C}}ù).
};

struct S
{
    struct N
    {
        static constexpr int f()  // (ù{\codeincomments{f}}ù) is eligible for compile-time computation.
        {
            return C<S>::e_SECRET;  // Type (ù{\codeincomments{S::N}}ù) is a (ù{\codeincomments{friend}}ù) of (ù{\codeincomments{C<S>}}ù).
        }
    };
};

static_assert(S::N::f() == 10022, "");  // (ù{\codeincomments{N}}ù) has (ù{\codeincomments{private}}ù) access to (ù{\codeincomments{C<S>}}ù).
\end{emcppslisting}
    
\noindent In the example above, the nested type \lstinline!S::N! --- but not
\lstinline!S! itself --- has private access to \lstinline!C<S>::e_SECRET!.
{\cprotect\footnote{Note that the need for \lstinline!typename! in the \lstinline!friend!
declaration in the example above to introduce the dependent type \lstinline!N! is relaxed
in C++20. For information on other contexts in which
\lstinline!typename! will eventually no longer be required, see~\cite{meredith20}.}}

\subsection[Use Cases]{Use Cases}\label{use-cases}

\subsubsection[Safely declaring a previously declared type to be a friend]{Safely declaring a previously declared type to be a friend}\label{safely-declaring-a-previously-declared-type-to-be-a-friend}

In C++98/03, to befriend a type that was already declared required
\emph{redeclaring} it. If the type were misspelled in the friend
declaration, a new type would be declared:

\begin{emcppslisting}[language=C++]
class Container { /* ... */ };

class ContainerIterator
{
    friend class Contianer;  // Compiles but wrong: (ù{\codeincomments{ia}}ù) should have been (ù{\codeincomments{ai}}ù).
    // ...
};
\end{emcppslisting}
    
\noindent The code above will compile and appear to be correct until
\lstinline!ContainerIterator! attempts to access a \lstinline!private! or
\lstinline!protected! member of \lstinline!Container!. At that point, the
compiler will surprisingly produce an error. As of C++11, we have the
option of preventing this mistake by using extended \lstinline!friend!
declarations:

\begin{emcppslisting}[language=C++]
class Container { /* ... */ };

class ContainerIterator
{
    friend Contianer;  // Error: (ù{\codeincomments{Contianer}}ù) not found
    // ...
};
\end{emcppslisting}
    

\subsubsection[Befriending a type alias used as a customization point]{Befriending a type alias used as a customization point}\label{befriending-a-type-alias-used-as-a-customization-point}

In C++03, the only option for friendship was to specify a particular
\lstinline!class! or \lstinline!struct! when granting private access. Let's
begin by considering a scenario in which we have an
\romeogloss{in-process}{\cprotect\footnote{When used to qualify a VST, the
term \romeogloss{in-process}, also called \emph{in-core}, refers to a type
that has typical value-type--like operations but does not refer to a
value that is meaningful outside of the current process; see
  \cite{lakos2a}, section~4.2.}} \romeogloss{value-semantic type (VST)}
that serves as a \emph{handle} to a platform-specific object, such as a
\lstinline!Window! in a graphical application. Large parts of a codebase
may seek to interact with \lstinline!Window! objects without needing or
obtaining access to the internal representation.

A very small part of the codebase that handles platform-specific window
management, however, needs privileged access to the internal
representation of \lstinline!Window!. One way to achieve this goal is to
make the platform-specific \lstinline!WindowManager! a \lstinline!friend! of
the \lstinline!Window! class; however, see \intraref{potential-pitfalls-extendedfriend}{long-distance-friendship}. 
%\textit{\titleref{potential-pitfalls-extendedfriend}: \titleref{long-distance-friendship}} on page~\pageref{long-distance-friendship}.

\begin{emcppslisting}[language=C++]
class WindowManager;  // forward declaration enabling extended (ù{\codeincomments{friend}}ù) syntax

class Window
{
private:
    friend class WindowManager;  // could instead use (ù{\codeincomments{friend WindowManager;}}ù)
    int d_nativeHandle;          // in-process (only) value of this object

public:
    // ... all the typical (e.g., special) functions we expect of a value type
};
\end{emcppslisting}
    
\noindent In the example above, class \lstinline!Window! befriends class
\lstinline!WindowManager!, granting it private access. Provided that the
implementation of \lstinline!WindowManager! resides in the same physical
\romeogloss{component} as that of class \lstinline!Window!, no
\romeogloss{long-distance friendship} results. The consequence of such a
monolithic design would be that every client that makes use of the
otherwise lightweight \lstinline!Window! class would necessarily depend
physically on the presumably heavier-weight \lstinline!WindowManager! class.

Now consider that the \lstinline!WindowManager! implementations on
different platforms might begin to diverge significantly. To keep the
respective implementations maintainable, one might choose to factor them
into distinct C++ types, perhaps even defined in separate files, and to
use a \emph{type alias} determined using platform-detection preprocessor
macros to configure that alias:

%%%%%%%%%%%%%%%%%%%%% all separate. intended? 
%%% yes, and VR requests some separation

\begin{emcppslisting}[language=C++]
// windowmanager_win32.h

#ifdef WIN32
class Win32WindowManager { /* ... */ };
#endif
\end{emcppslisting}

\vspace*{2ex}    

\begin{emcppslisting}[language=C++]
// windowmanager_unix.h

#ifdef UNIX
class UnixWindowManager { /* ... */ };
#endif
\end{emcppslisting}
    
\vspace*{2ex}

\begin{emcppslisting}[language=C++]
// windowmanager.h

#ifdef WIN32
#include <windowmanager_win32.h>
typedef Win32WindowManager WindowManager;
#else
#include <windowmanager_unix.h>
typedef UnixWindowManager WindowManager;
#endif
\end{emcppslisting}
    
\vspace*{2ex}

\begin{emcppslisting}[language=C++]
// window.h
#include <windowmanager.h>

class Window
{
private:
    friend WindowManager;  // C++11 extended friend declaration
    int d_nativeHandle;

public:
    // ...
};
\end{emcppslisting}
   
%%%%%%%%%%%%%%%%%%%%%%%      

In this example, class \lstinline!Window! no longer befriends a specific
class named \lstinline!WindowManager!; instead, it befriends the
\lstinline!WindowManager! type alias, which in turn has been set to the
correct platform-specific window manager implementation. Such extended
use of \lstinline!friend! syntax was not available in C++03.

Note that this use case involves \romeogloss{long-distance friendship}
inducing an implicit cyclic dependency between the \romeogloss{component}
implementing \lstinline!Window! and those implementing
\lstinline!WindowManager!; see \intraref{potential-pitfalls-extendedfriend}{long-distance-friendship}. 
%\textit{\titleref{potential-pitfalls-extendedfriend}: \titleref{long-distance-friendship}} on page~\pageref{long-distance-friendship}. 
Such designs, though
undesirable, can result from an emergent need to add new platforms while
keeping tightly related code sequestered within smaller, more manageable
physical units. An alternative design would be to obviate the
\romeogloss{long-distance friendship} by widening the API for the
\lstinline!Window! class, the natural consequence of which would be to
invite public client abuse vis-a-vis \romeogloss{Hyrum's law}.

\subsubsection[Using the \lstinline!PassKey! idiom to enforce initialization]{Using the {\SubsubsecCode PassKey} idiom to enforce initialization}\label{using-the-passkey-idiom-to-enforce-initialization}

Prior to C++11,
efforts to grant private access to a class defined in a separate
physical unit required declaring the higher-level type itself to be a
\lstinline!friend!, resulting in this highly undesirable form of
friendship; see \intraref{potential-pitfalls-extendedfriend}{long-distance-friendship}.  
%\textit{\titleref{potential-pitfalls-extendedfriend}: \titleref{long-distance-friendship}} on page~\pageref{long-distance-friendship}. 
The ability in C++11 to
declare a template \emph{type} parameter or any other type specifier to
be a friend affords new opportunities to enforce selective
private access (e.g., to one or more individual functions) without
explicitly declaring another type to be a \lstinline!friend!; see also \intrarefsimple{granting-a-specific-type-access-to-a-single-private-function}. 
%\textit{\titleref{description-extendedfriend}: \titleref{granting-a-specific-type-access-to-a-single-private-function}} on page~\pageref{granting-a-specific-type-access-to-a-single-private-function}. 
In this use case, however, our use of extended
\lstinline!friend! syntax to befriend a template parameter is unlikely to run
afoul of sound physical design.

Let's say we have a commercial library, and we want it to verify a
software-license key in the form of a C-style string, prior to allowing use of 
other parts of the API:

\begin{emcppslisting}[language=C++]
// simplified pseudocode
LibPassKey initializeLibrary(const char* licenseKey);
int utilityFunction1(LibPassKey object /*, ... (other parameters) */);
int utilityFunction2(LibPassKey object /*, ... (other parameters) */);
\end{emcppslisting}
    
\noindent Knowing full well that this is not a \emph{secure} approach and that
innumerable deliberate, malicious ways exist to get around the C++ type
system, we nonetheless want to create a plausible regime where no
\emph{well-formed} code can \emph{accidentally} gain access
to library functionality other than by legitimately initializing the
system using a valid license key. We could easily cause a
function to \lstinline!throw!, \lstinline!abort!, and so on at run time when
the function is called prior to the client's license key being
authenticated. However, part of our goal, as a friendly library vendor, is to
ensure that clients do not \emph{inadvertently} call other library
functions prior to initialization. To that end, we propose the following protocol:
\begin{enumerate}
\item{use an instantiation of the \lstinline!PassKey! class template\cprotect\footnote{\cite{mayrand15}} that only our API \emph{utility}\linebreak[4] \emph{\lstinline!struct!}\cprotect\footnote{\cite{lakos20}, section~2.4.9, pp.~312--321, specifically Figure~2-23, p.~316} can create}
\item{return a constructed object of this type only upon successful validation of the license key}
\item{require that clients present this (constructed) passkey \emph{object} every time they invoke any other function in the API}
\end{enumerate}
Here's an example that encompasses all three aforementioned points:

\begin{emcppslisting}[language=C++]
template <typename T>
class PassKey  // reusable standard utility type
{
    PassKey() { }  // private default constructor (no aggregate initialization)
    friend T;      // Only (ù{\codeincomments{T}}ù) is allowed to create this object.
};

struct BestExpensiveLibraryUtil
{
    class LicenseError { /*...*/ };  // thrown if license string is invalid

    using LibPassKey = PassKey<BestExpensiveLibraryUtil>;
        // This is the type of the (ù{\codeincomments{PassKey}}ù) that will be returned when this
        // utility is initialized successfully, but only this utility is able 
        // to construct an object of this type. Without a valid license string,
        // the client will have no way to create such an object and thus no way
        // to call functions within this library.

    static LibPassKey initializeLibrary(const char* licenseKey)
        // This function must be called with a valid (ù{\codeincomments{licenseKey}}ù) string prior
        // to using this library; if the supplied license is valid, a
        // (ù{\codeincomments{LibPassKey}}ù) *object* will be returned for mandatory use in *all*
        // subsequent calls to useful functions of this library. This function
        // throws (ù{\codeincomments{LicenseError}}ù) if the supplied (ù{\codeincomments{licenseKey}}ù) string is invalid.
    {
        if (isValid(licenseKey))
        {
            // Initialize library properly.

            return LibPassKey();
                // Return a default-constructed (ù{\codeincomments{LibPassKey}}ù). Note that only
                // this utility is able to construct such a key.
        }

        throw LicenseError();  // supplied license string was invalid
    }

    static int doUsefulStuff(LibPassKey key /*,...*/);
        // The function requires a (ù{\codeincomments{LibPassKey}}ù) object, which can be constructed
        // only by invoking the (ù{\codeincomments{static initializeLibrary}}ù) function, to be
        // supplied as its first argument. ...

private:
    static bool isValid(const char* key);
        // externally defined function that returns true if (ù{\codeincomments{key}}ù) is valid
};
\end{emcppslisting}
    
\noindent Other than going outside the language with invalid constructs or circumventing the
type system with esoteric tricks, this approach, among other things,
prevents invoking the\linebreak[4]%%%%%%%%  
\lstinline!doUsefulStuff! function without a proper
license. What's more, the C++ type system \emph{at compile time} forces
a prospective client to have initialized the library before any attempt
is made to use any of its other functionality.

\subsubsection[Granting a specific type access to a single \lstinline!private! function]{Granting a specific type access to a single {\SubsubsecCode private} function}\label{granting-a-specific-type-access-to-a-single-private-function}

When designing in purely logical terms, wanting to grant some other
logical entity special access to a type that no other entity enjoys is a
common situation. Doing so does not necessarily become problematic until
that friendship spans physical boundaries; see \intraref{potential-pitfalls-extendedfriend}{long-distance-friendship}. 
%\textit{\titleref{potential-pitfalls-extendedfriend}: \titleref{long-distance-friendship}} on page~\pageref{long-distance-friendship}.

As a simple approximation to a real-world use
case,\footnote{For an example of a real-world database
implementation that requires managed objects to befriend that database
manager, see \cite{cst15}, section~2.1.} suppose we have a lightweight object-database class, \lstinline!Odb!, that is designed to operate
collaboratively with objects, such as \lstinline!MyWidget!, that are
themselves designed to work collaboratively with \lstinline!Odb!. Every compliant UDT
suitable for management by \lstinline!Odb! will need to maintain an integer
object ID that is read/write accessible by an \lstinline!Odb! object. Under
no circumstances is any other object permitted to access, let alone
modify, that ID independently of the \lstinline!Odb! API.

Prior to C++11, the design of such a feature might require every
participating class to define a data member named \lstinline!d_objectId!
and to declare the \lstinline!Odb! class a \lstinline!friend! (using old-style
\lstinline!friend! syntax):

\begin{emcppslisting}[language=C++]
class MyWidget  // grants just (ù{\codeincomments{Odb}}ù) access to *all* of its private data
{
    int d_objectId;    // required by our collaborative-design strategy
    friend class Odb;  //    "     "   "        "        "       "
    // ...

public:
    // ...
};

class Odb
{
    // ...

public:
    template <typename T>
    void processObject(T& object)
        // This function template is generally callable by clients.
    {
        int& objId = object.d_objectId;
        // ... (process as needed)
    }

    // ...
};
\end{emcppslisting}
    
\noindent In this example, the \lstinline!Odb! class implements the public member
function template,\linebreak[4] \lstinline!processObject!, which then extracts the
\lstinline!objectId! field for access. The collateral damage is that we
have exposed all of our private details to \lstinline!Odb!, which is at
best a gratuitous widening of our sphere of encapsulation.

Using the \lstinline!PassKey! pattern allows us to be more selective with
what we share:

\begin{emcppslisting}[language=C++]
template <typename T>
class Passkey
    // Implement this eminently reusable (ù{\codeincomments{Passkey}}ù) class template again here.
{
    Passkey() { }  // prevent aggregate initialization
    friend T;      // Only the (ù{\codeincomments{T}}ù) in (ù{\codeincomments{PassKey<T>}}ù) can create a (ù{\codeincomments{PassKey}}ù) object.
    Passkey(const Passkey&) = delete;             // no copy/move construction
    Passkey& operator=(const Passkey&) = delete;  // no copy/move assignment
};
\end{emcppslisting}
    
\noindent We are now able to adjust the design of our systems such that only the
minimum private functionality is exposed to \lstinline!Odb!:

\begin{emcppslisting}[language=C++]
class Odb;      // Objects of this class have special access to other objects.

class MyWidget  // grants just (ù{\codeincomments{Odb}}ù) access to only its (ù{\codeincomments{objectId}}ù) member function
{
    int d_objectId;  // must have an (ù{\codeincomments{int}}ù) data member of any name we choose
    // ...

public:
    int& objectId(Passkey<Odb>) { return d_objectId; }
        // Return a non-(ù{\codeincomments{const}}ù) reference to the mandated (ù{\codeincomments{int}}ù) data member.
        // (ù{\codeincomments{objectId}}ù) is callable only within the scope of (ù{\codeincomments{Odb}}ù).

    // ...
};

class Odb
{
    // ...

public:
    template <typename T>
    void processObject(T& object)
        // This function template is generally callable by clients.
    {
        int& objId = object.objectId(Passkey<Odb>());
        // ...
    }

    // ...
};
\end{emcppslisting}
    
\noindent Instead of granting \lstinline!Odb! private access to \emph{all}
encapsulated implementation details of\linebreak[4] \mbox{\lstinline!MyWidget!}, this example
uses the \lstinline!PassKey! idiom to enable just \lstinline!Odb! to call the
(syntactically \lstinline!public!) \lstinline!objectId! member function of
\lstinline!MyWidget! with no private access whatsoever. As a further
demonstration of the efficacy of this approach, consider that we are
able to create and invoke the \lstinline!processObject! method of an
\lstinline!Odb! object from a function, \lstinline!f!, but we are blocked from
calling the \lstinline!objectId! method of a \lstinline!MyWidget! object
directly:

\begin{emcppslisting}[language=C++]
void f()
{
    Odb mgr;          // object receiving fine-grained privileged access
    MyWidget widget;  // object granting selective private access to just (ù{\codeincomments{Odb}}ù)
    mgr.processObject(widget);

    int& objId = widget.objectId(PassKey<Odb>());  // cannot call out of (ù{\codeincomments{Odb}}ù)
       // Error: (ù{\codeincomments{Passkey<T>::Passkey()}}ù) [with(ù{\codeincomments{T}}ù) = (ù{\codeincomments{Odb}}ù)] is private within
       // this context.
}
\end{emcppslisting}
    
\noindent Notice that use of the extended \lstinline!friend! syntax to befriend a
template parameter and thereby enable the \lstinline!PassKey! idiom here
improved the granularity with which we effectively grant privileged
access to an individually named type but didn't fundamentally alter the
testability issues that result when private access to specific C++
types is allowed to extend across physical boundaries; again, see \intraref{potential-pitfalls-extendedfriend}{long-distance-friendship}. 
%\textit{\titleref{potential-pitfalls-extendedfriend}: \titleref{long-distance-friendship}} on page~\pageref{long-distance-friendship}.

\subsubsection[Curiously recurring template pattern]{Curiously recurring template pattern}\label{curiously-recurring-template-pattern}

Befriending a template parameter via extended \lstinline!friend!
declarations can be helpful when implementing the \romeogloss{curiously
recurring template pattern (CRTP)}. For use-case examples and more
information on the pattern itself, see \intrarefsimple{appendix:-crtp-use-cases}. 
%\textit{\titleref{appendix:-crtp-use-cases}} on page~\pageref{appendix:-crtp-use-cases}.

\subsection[Potential Pitfalls]{Potential Pitfalls}\label{potential-pitfalls-extendedfriend}

\subsubsection[Long-distance friendship]{Long-distance friendship}\label{long-distance-friendship}

Since before C++ was standardized, granting private access via a
\lstinline!friend! declaration across physical boundaries, known as
\romeogloss{long-distance friendship}, was
observed{\cprotect\footnote{\cite{lakos96}, section~3.6.1
  pp.~141--144}}{\cprotect\footnote{\cite{lakos20}, section~2.6,
  pp.~342--370, specifically p.~367 and p.~362}} to potentially lead
to designs that are qualitatively more difficult to understand, test,
and maintain. When a user-defined type, \lstinline!X!, befriends some other
specific type, \lstinline!Y!, in a separate, higher-level translation unit,
testing \lstinline!X! thoroughly without also testing \lstinline!Y! is no
longer possible. The effect is a test-induced cyclic dependency between
\lstinline!X! and \lstinline!Y!. Now imagine that \lstinline!Y! depends on a
sequence of other types, \lstinline!C1!, \lstinline!C2!, \ldots,
\lstinline!CN-2!, each defined in its own physical \romeogloss{component},
\lstinline!CI!, where \lstinline!CN-2! depends on \lstinline!X!. The result is a
physical design cycle of size \emph{N}. As \emph{N} increases, the
ability to manage complexity quickly becomes intractable. Accordingly,
the two design imperatives that were most instrumental in shaping the
C++20 \romeogloss{modules} feature were (1) to have no cyclic module
dependencies and (2) to avoid intermodule friendships.

\subsection[See Also]{See Also}\label{see-also}

TODO

\subsection[Further Reading]{Further Reading}\label{further-reading}

\begin{itemize}
\item{For yet more potential uses of the extended friend pattern in
metaprogramming contexts, such as using CRTP, see
\cite{alexandrescu01}.}
\item{\cite{lakos96}, section~3.6, pp.~136--146, is dedicated to the
classic use (and misuse) of friendship.}
\item{\cite{miller05}}
\item{\cite{{lakos20}} provides extensive advice on \emph{sound}
\romeogloss{physical design}, which generally precludes \romeogloss{long-distance
friendship}.}
\end{itemize}

\subsection[Appendix: Curiously Recurring Template Pattern Use Cases]{Appendix: Curiously Recurring Template Pattern Use Cases}\label{appendix:-crtp-use-cases}

\subsubsection[Refactoring using the curiously recurring template pattern]{Refactoring using the curiously recurring template pattern}\label{refactoring-using-the-curiously-recurring-template-pattern}

Avoiding code duplication across disparate classes can sometimes be
achieved using a strange template pattern first recognized in the
mid-90s, which has since become known as the \romeogloss{curiously recurring
template pattern (CRTP)}. The pattern is \emph{curious} because it
involves the surprising step of declaring as a base class, such as
\lstinline!B!, a template that \emph{expects} the derived class, such as
\lstinline!C!, as a template argument, such as \lstinline!T!:

\begin{emcppslisting}[language=C++]
template <typename T>
class B
{
    // ...
};

class C : public B<C>
{
   // ...
};
\end{emcppslisting}
    
\noindent As a trivial illustration of how the CRTP can be used as a refactoring tool,
suppose that we have several classes for which we would like to track,
say, just the number of active instances:

\begin{emcppslisting}[language=C++]
class A
{
    static int s_count;  // declaration
    // ...

public:
    static int count() { return s_count; }

    A()          { ++s_count; }
    A(const A&)  { ++s_count; }
    A(const A&&) { ++s_count; }
    ~A()         { --s_count; }

    A& operator=(A&)  = default;  // see special members
    A& operator=(A&&) = default;  //  "     "       "
    // ...
};

int A::s_count;  // definition (in (ù{\codeincomments{.cpp}}ù) file)

class B { /* similar to A (above) */ }
// ...

void test()
{          // (ù{\codeincomments{A::s\_count}}ù) = (ù{\codeincomments{0}}ù), (ù{\codeincomments{B::s\_count}}ù) = (ù{\codeincomments{0}}ù)
    A a1;  // (ù{\codeincomments{A::s\_count}}ù) = (ù{\codeincomments{1}}ù), (ù{\codeincomments{B::s\_count}}ù) = (ù{\codeincomments{0}}ù)
    B b1;  // (ù{\codeincomments{A::s\_count}}ù) = (ù{\codeincomments{1}}ù), (ù{\codeincomments{B::s\_count}}ù) = (ù{\codeincomments{1}}ù)
    A a2;  // (ù{\codeincomments{A::s\_count}}ù) = (ù{\codeincomments{2}}ù), (ù{\codeincomments{B::s\_count}}ù) = (ù{\codeincomments{1}}ù)
}          // (ù{\codeincomments{A::s\_count}}ù) = (ù{\codeincomments{0}}ù), (ù{\codeincomments{B::s\_count}}ù) = (ù{\codeincomments{0}}ù)
\end{emcppslisting}
    
\noindent In this example, we have multiple classes, each repeating the same
common machinery. Let's now explore how we might refactor this example
using the CRTP:

\begin{emcppslisting}[language=C++]
template <typename T>
class InstanceCounter
{
protected:
    static int s_count;  // declaration

public:
    static int count() { return s_count; }
};

template <typename T>
int InstanceCounter<T>::s_count;  // definition (in same file as declaration)

struct A : InstanceCounter<A>
{
    A()          { ++s_count; }
    A(const A&)  { ++s_count; }
    A(const A&&) { ++s_count; }
    ~A()         { --s_count; }

    A& operator=(const A&)  = default;
    A& operator=(A&&)       = default;
    // ...
};
\end{emcppslisting}
    
\noindent Notice that we have factored out a common counting mechanism into an
\lstinline!InstanceCounter! class template and then derived our
representative class \lstinline!A! from \lstinline!InstanceCounter<A>!, and we
would do similarly for classes \lstinline!B!, \lstinline!C!, and so on. This
approach works because the compiler does not need to see the derived
type until the point at which the template is instantiated, which will
be \emph{after} it has seen the derived type.

Prior to C++11, however, there was plenty of room for user error.
Consider, for example, forgetting to change the base-type parameter when
copying and pasting a new type:

\begin{emcppslisting}[language=C++]
struct B : InstanceCounter<A>  // Oops! We forgot to change (ù{\codeincomments{A}}ù) to (ù{\codeincomments{B}}ù) in
                               // (ù{\codeincomments{InstanceCounter}}ù): The wrong count will be
                               // updated!
{
    B() { ++s_count; }
};
\end{emcppslisting}
    
\noindent Another problem is that a client deriving from our class can mess with
our protected \lstinline!s_count!:

\begin{emcppslisting}[language=C++]
struct AA : A
{
    AA() { s_count = -1; }  // Oops! *Hyrum's Law* is at work again!
};
\end{emcppslisting}
    
\noindent We could inherit from the \lstinline!InstanceCounter! class privately, but
then \lstinline!InstanceCounter! would have no way to add to the derived
class's public interface, for example, the public \lstinline!count! static
member function.

As it turns out, however, both of these missteps can be erased simply by
making the internal mechanism of the \lstinline!InstanceCounter! template
private and then having \mbox{\lstinline!InstanceCounter!} befriend its template
parameter, \lstinline!T!:

\begin{emcppslisting}[language=C++]
template <typename T>
class InstanceCounter
{
    static int s_count;  // Make this static data member private.
    friend T;            // Allow access only from the derived (ù{\codeincomments{T}}ù).

public:
    static int count() { return s_count; }
};

template <typename T>
int InstanceCounter<T>::s_count;
\end{emcppslisting}
    
\noindent Now if some other class does try to derive from this type, it cannot
access this type's counting mechanism. If we want to suppress even that
possibility, we can declare and default (see \featureref{\locationa}{defaulted-special-member-functions}) 
% ``\titleref{defaulted-special-member-functions}" on page~\pageref{defaulted-special-member-functions}) 
the \lstinline!InstanceCounter!
class constructors to be private as well.

\subsubsection[Synthesizing equality using the curiously recurring template pattern]{Synthesizing equality using the curiously recurring template pattern}\label{synthesizing-equality-using-crtp}

As a second example of code factoring using the CRTP, suppose that we want
to create a factored way of synthesizing \lstinline!operator==! for types
that implement just an \lstinline!operator<!.{\cprotect\footnote{This
example is based on a similar one found on stackoverflow.com:\linebreak[3]
  https://\linebreak[3]stackoverflow.\linebreak[3]com/\linebreak[3]questions/\linebreak[3]4173254/what-is-the-curiously-recurring-template-pattern-crtp}}
In this example, the CRTP base-class template, \lstinline!E!, will
synthesize the homogeneous \lstinline!operator==! for its parameter type,
\lstinline!D!, by returning \lstinline!false! if either argument is \emph{less
than} the other:

\begin{emcppslisting}[language=C++]
template <typename D>
class E { }; // CRTP base class used to synthesize (ù{\codeincomments{operator==}}ù) for (ù{\codeincomments{D}}ù)

template <typename D>
bool operator==(const E<D>& lhs, const E<D>& rhs)
{
    const D& d1 = static_cast<const D&>(lhs);  // derived type better be (ù{\codeincomments{D}}ù)
    const D& d2 = static_cast<const D&>(rhs);  //    "     "     "    "  "
    return !(d1 < d2) && !(d2 < d1);           // assuming (ù{\codeincomments{D}}ù) has an (ù{\codeincomments{operator<}}ù)
}
\end{emcppslisting}
    
\noindent A client that implements an \lstinline!operator<! can now reuse this CRTP
base case to synthesize an \lstinline!operator==!:

\begin{emcppslisting}[language=C++]
struct S : E<S>
{
    int d_size;
};

bool operator<(const S& lhs, const S& rhs)
{
    return lhs.d_size < rhs.d_size;
}

#include <cassert>

void test1()
{
    S s1; s1.d_size = 10;
    S s2; s2.d_size = 10;

    assert(s1 == s2);  // compiles and passes
}
\end{emcppslisting}
    
\noindent As this code snippet suggests, the base-class template, \lstinline!E!, is
able to use the template parameter, \lstinline!D! (representing the derived
class, \lstinline!S!), to synthesize the homogeneous free
\lstinline!operator==! function for \lstinline!S!.

Prior to C++11, no means existed to guard against accidents, such as
inheriting from the wrong base and then perhaps even forgetting to
define the \lstinline!operator<!:

\begin{emcppslisting}[language=C++]
struct P : E<S>  // Oops! should have been (ù{\codeincomments{E(P)}}ù) -- a serious latent defect
{
    int d_x;
    int d_y;
};

void test2()
{
    P p1; p1.d_x = 10; p1.d_y = 15;
    P p2; p2.d_x = 10; p2.d_y = 20;

    assert( !(p1 == p2) );  // Oops! This fails because of (ù{\codeincomments{E(S)}}ù) above.
}
\end{emcppslisting}
    
\noindent Again, thanks to C++11's extended \lstinline!friend! syntax, we can defend
against these defects at compile time simply by making the CRTP base
class's default constructor \emph{private} and befriending its template
parameter:

\begin{emcppslisting}[language=C++]
template <typename D>
class E
{
     E() = default;
     friend D;
};
\end{emcppslisting}
    
\noindent Note that the goal here is not security but simply guarding against
accidental typos, copy-paste errors, and other occasional  human errors. By making this change, we will soon realize that there
is no \lstinline!operator<! defined for \lstinline!P!.

\subsubsection[Compile-time polymorphism using the curiously recurring template pattern]{Compile-time polymorphism using the curiously recurring template pattern}\label{compile-time-polymorphism-using-crtp}

Object-oriented programming provides certain flexibility that at times
might be supererogatory. Here we will exploit the familiar domain of
abstract/concrete shapes to demonstrate a mapping between runtime
polymorphism using virtual functions and compile-time polymorphism using
the CRTP. We begin with a simple abstract \lstinline!Shape! class that
implements a single, pure, virtual \lstinline!draw! function:

\begin{emcppslisting}[language=C++]
class Shape
{
public:
    virtual void draw() const = 0;  // abstract (ù{\codeincomments{draw}}ù) function (interface)
};
\end{emcppslisting}
    
\noindent From this abstract \lstinline!Shape! class, we now derive two concrete
shape types, \lstinline!Circle! and \mbox{\lstinline!Rectangle!}, each implementing
the \emph{abstract} \lstinline!draw! function:

\begin{emcppslisting}[language=C++]
#include <iostream>  // (ù{\codeincomments{std::cout}}ù)

class Circle : public Shape
{
    int d_radius;

public:
    Circle(int radius) : d_radius(radius) { }

    void draw() const  // concrete implementation of abstract (ù{\codeincomments{draw}}ù) function
    {
        std::cout << "Circle(radius = " << d_radius << ")\n";
    }
};

class Rectangle : public Shape
{
    int d_length;
    int d_width;

public:
    Rectangle(int length, int width) : d_length(length), d_width(width) { }

    void draw() const  // concrete implementation of abstract (ù{\codeincomments{draw}}ù) function
    {
        std::cout << "Rectangle(length = " << d_length << ", "
                                "width = " << d_width  << ")\n";
    }
};
\end{emcppslisting}
    
\noindent Notice that a \lstinline!Circle! is constructed with a single integer
argument, i.e., \lstinline!radius!, and a \lstinline!Rectangle! is constructed
with two integers, i.e., \lstinline!length! and \lstinline!width!.

We now implement a function that takes an arbitrary shape, via a
\lstinline!const! \romeovalue{lvalue} reference to its abstract base class, and
prints it:

\begin{emcppslisting}[language=C++]
void print(const Shape& shape)
{
    shape.draw();
}

void testShape()
{
    print(Circle(1));        // prints: (ù{\codeincomments{Circle(radius = 1)}}ù)
    print(Rectangle(2, 3));  // prints: (ù{\codeincomments{Rectangle(length = 2, width = 3)}}ù)
    print(Shape());          // Compile-time error: (ù{\codeincomments{Shape}}ù) is an abstract class.
}
\end{emcppslisting}
    
\noindent Now suppose that we didn't need all the runtime flexibility offered by
this system and wanted to map just what we have in the previous code
snippet onto templates that avoid the spatial and runtime overhead of
virtual-function tables and dynamic dispatch. Such transformation again
involves creating a CRTP base class, this time in lieu of our abstract
interface:

\begin{emcppslisting}[language=C++]
template <typename T>
struct Shape
{
    void draw() const
    {
        static_cast<const T*>(this)->draw();  // assumes (ù{\codeincomments{T}}ù) derives from (ù{\codeincomments{Shape}}ù)
    }
};
\end{emcppslisting}
    
\noindent Notice that we are using a \lstinline!static_cast! to the address of an
object of the \lstinline!const! template parameter type, \lstinline!T!,
assuming that the template argument is of the same type as some derived
class of this object's type. We now define our types as before, the only
difference being the form of the base type:

\begin{emcppslisting}[language=C++]
class Circle : public Shape<Circle>
{
    // same as above
};

class Rectangle : public Shape<Rectangle>
{
    // same as above
};
\end{emcppslisting}
    
\noindent We now define our \lstinline!print! function, this time as a function
template taking a \lstinline!Shape! of arbitrary type \lstinline!T!:

\begin{emcppslisting}[language=C++]
template <typename T>
void print(const Shape<T>& shape)
{
    shape.draw();
}
\end{emcppslisting}
    
\noindent The result of compiling and running \lstinline!testShape! above is the
same, including that \lstinline!Shape()! doesn't compile.

However, opportunities for undetected failure remain. Suppose we decide
to add a third shape, \lstinline!Triangle!, constructed with three sides:

\begin{emcppslisting}[language=C++]
class Triangle : public Shape<Rectangle>  // Oops!
{
    int d_side1;
    int d_side2;
    int d_side3;

public:
    Triangle(int side1, int side2, int side3)
        : d_side1(side1), d_side2(side2), d_side3(side3) { }

    void draw() const
    {
        std::cout << "Triangle(side1 = " << d_side1 << ", "
                              "side2 = " << d_side2 << ", "
                              "side3 = " << d_side3 << ")\n";
    }
};
\end{emcppslisting}
    
\noindent Unfortunately we forgot to change the base-class type parameter when we
copy-pasted from \lstinline!Rectangle!.

Let's now create a new test that exercises all three and see what
happens on our platform:

\begin{emcppslisting}[language=C++]
void test2()
{
    print(Circle(1));          // prints: (ù{\codeincomments{Circle(radius = 1)}}ù)
    print(Rectangle(2, 3));    // prints: (ù{\codeincomments{Rectangle(length = 2, width = 3)}}ù)
    print(Triangle(4, 5, 6));  // prints: (ù{\codeincomments{Rectangle(length = 4, width = 5)}}ù) ?!
    Shape<int> bug;            // Compiles?!
}
\end{emcppslisting}
    
\noindent As should by now be clear, a defect in our \lstinline!Triangle!
implementation results in \emph{hard} \romeogloss{undefined behavior} that
could have been prevented at compile time by using the extended
\lstinline!friend! syntax. Had we defined the CRTP base-class template's
default constructor to be \emph{private} and made its type parameter a
\lstinline!friend!, we could have prevented the copy-paste error with
\lstinline!Triangle! and suppressed the ability to create a \lstinline!Shape!
object without deriving from it (e.g., see \lstinline!bug! in the previous
code snippet):

\begin{emcppslisting}[language=C++]
template <typename T>
class Shape
{
    Shape() = default;  // Default the (ù{\codeincomments{default}}ù) constructor to be (ù{\codeincomments{private}}ù).
    friend T;           // Ensure only a type derived from (ù{\codeincomments{T}}ù) has access.
}
\end{emcppslisting}
    
\noindent Generally, whenever we are using the CRTP, making just the
default constructor of the base-class template \lstinline!private! and
having it befriend its type parameter is typically a trivial local
change, is helpful in avoiding various forms of accidental misuse, and
is unlikely to induce long-distance friendships where none previously
existed: Applying extended \lstinline!friend! syntax to an existing CRTP
is typically \emph{safe}.

\subsubsection[Compile-time visitor using the curiously recurring template pattern]{Compile-time visitor using the curiously recurring template pattern}\label{compile-time-visitor-using-crtp}

As more real-world applications of compile-time polymorphism using the CRTP,
consider implementing traversal and visitation of complex data
structures. In particular, we want to facilitate employing
\emph{default-action} functions, which allow for simpler code from the
point of view of the programmer who needs the results of the traversal.
We illustrate our compile-time visitation approach using binary trees as
our data structure.

We begin with the traditional node structure of a binary tree, where
each node has a left and right subtree plus a label:

\begin{emcppslisting}[language=C++]
struct Node
{
    Node* d_left;
    Node* d_right;
    char  d_label;  // (ù{\codeincomments{label}}ù) will be used in the pre-order example.

    Node() : d_left(0), d_right(0), d_label(0) { }
};
\end{emcppslisting}
    
\noindent Now we wish to have code that traverses the tree in one of the three
traditional ways: \emph{pre-order}, \emph{in-order}, \emph{post-order}.
Such traversal code is often intertwined with the actions to be taken.
In our implementation, however, we will write a CRTP-like base-class
template, \lstinline!Traverser!, that implements empty stub functions for
each of the three traversal types, relying on the CRTP-derived
type to supply the desired functionality:

\begin{emcppslisting}[language=C++]
template <typename T>
class Traverser
{
private:
    Traverser() = default;  // Make the default constructor (ù{\codeincomments{private}}ù).
    friend T;               // Grant access only to the derived class.

public:
    void visitPreOrder(Node*)  { }  // stub-functions & placeholders
    void visitInOrder(Node*)   { }  // (Each of these three functions
    void visitPostOrder(Node*) { }  // defaults to an inline "no-op.")

    void traverse(Node* n)  // factored subfunctionality
    {
        T *t = static_cast<T*>(this);  // Cast (ù{\codeincomments{this}}ù) to the derived type.

        if (n) { t->visitPreOrder(n);     }  // optionally defined in derived
        if (n) { t->traverse(n->d_left);  }  //     "         "    "     "
        if (n) { t->visitInOrder(n);      }  //     "         "    "     "
        if (n) { t->traverse(n->d_right); }  //     "         "    "     "
        if (n) { t->visitPostOrder(n);    }  //     "         "    "     "
    }
};
\end{emcppslisting}
    
\noindent The factored traversal mechanism is implemented in the
\lstinline!Traverser! base-class template. A proper subset of the four
customization points, that is, the four member functions invoked from
the \emph{public} \lstinline!traverse! function of the \lstinline!Traverser!
base class, are implemented as appropriate in the derived class,
identified by \lstinline!T!. Each of these customization functions is
invoked in order. Notice that the \lstinline!traverse! function is safe to call on a
\lstinline!nullptr! as each individual customization-function invocation
will be independently bypassed if its supplied \lstinline!Node! pointer is
null. If a customization function is defined in the derived class, that
version of it is invoked; otherwise, the corresponding empty
\lstinline!inline! base-class version of that function is invoked instead.
This approach allows for any of the three traversal orders to be
implemented simply by supplying an appropriately configured derived type
where clients are obliged to implement only the portions they need. Even
the traversal itself can be modified, as we will soon see, where we
create the very data structure we're traversing.

Let's now look at how derived-class authors might use this pattern.
First, we'll write a traversal class that fully populates a tree to a
specified depth:

\begin{emcppslisting}[language=C++]
struct FillToDepth : Traverser<FillToDepth>
{
    using Base = Traverser<FillToDepth>;  // similar to a local (ù{\codeincomments{typedef}}ù)

    int d_depth;            //  final "height" of the tree
    int d_currentDepth;     //  current distance from the root

    FillToDepth(int depth) : d_depth(depth), d_currentDepth(0) { }

    void traverse(Node*& n)
    {
        if (d_currentDepth++ < d_depth && !n)  // descend; if not balanced...
        {
            n = new Node;     // Add (ù{\codeincomments{node}}ù) since it's not already there.
        }

        Base::traverse(n);    // Recurse by invoking the *base* version.

        --d_currentDepth;     // Ascend.
    }
};
\end{emcppslisting}
    
\noindent The derived class's version of the \lstinline!traverse! member function
acts as if it overrides the \lstinline!traverse! function in the base-class
template and then, as part of its re-implementation, defers to the base-class version to perform the actual traversal.

Importantly, note that we have re-implemented \lstinline!traverse! in the
derived class with a function by the same name but having a
\emph{different} \romeogloss{signature} that has more capability (i.e., it's able
to modify its immediate argument) than the one in the base-class
template. In practice, this signature modification is something we would
do rarely, but part of the flexibility of this design pattern, as with
templates in general, is that we can take advantage of \romeogloss{duck
typing} to achieve useful functionality in somewhat unusual ways. For
this pattern, the designers of the base-class template and the designers
of the derived classes are, at least initially, likely to be the same
people, and they will arrange for these sorts of signature variants to
work correctly if they need such functionality. Or they may decide that
overridden methods should follow a proper contract and signature that
they determine is appropriate, and they may declare improper overrides
to be undefined behavior. In this example, we aim for illustrative
flexibility over rigor.

\begin{emcppslisting}[language=C++]
void traverse(Node* n);   // as declared in the (ù{\codeincomments{Traverser}}ù) base-class template
void traverse(Node*& n);  // as declared in the (ù{\codeincomments{FillToDepth}}ù) derived class
\end{emcppslisting}
    
\noindent Unlike virtual functions, the signatures of corresponding functions in
the base and derived classes need not match exactly \emph{provided} the
derived-class function can be called in the same way as the
corresponding one in the base class. In this case, the compiler has all
the information it needs to make the call properly:

\begin{emcppslisting}[language=C++]
static_cast<FillToDepth *>(this)->traverse(n)  // what the compiler sees
\end{emcppslisting}
    
\noindent Suppose that we now want to create a type that labels a \emph{small}
tree, balanced or not, according to its pre-order traversal:

\begin{emcppslisting}[language=C++]
struct PreOrderLabel : Traverser<PreOrderLabel>
{
    char d_label;

    PreOrderLabel() : d_label('a') { }

    void visitPreOrder(Node* n)  // This choice controls traversal order.
    {
        n->d_label = d_label++;   
            // Each successive label is sequential alphabetically.
    }
};
\end{emcppslisting}
    
\noindent The simple pre-order traversal class, \lstinline!PreOrderLabel!, labels the
nodes such that it visits each parent \emph{before} it visits either of
its two children.

Alternatively, we might want to create a read-only derived class,
\lstinline!InOrderPrint!, that simply prints out the sequence of labels
resulting from an \emph{in-order} traversal of the, e.g., previously
pre-ordered, labels:

\begin{emcppslisting}[language=C++]
#include <cstdio>  // (ù{\codeincomments{std::putchar}}ù)

struct InOrderPrint : Traverser<InOrderPrint>
{
    ~InOrderPrint()
    {
        std::putchar('\n');  // print single newline at end of string
    }

    void visitInOrder(const Node* n) const
    {
        std::putchar(n->label);  // Print the label character exactly as is.
    }
};
\end{emcppslisting}
    
\noindent The simple \lstinline!InOrderPrint!-derived class, shown in the example above, prints out the
labels of a tree \emph{in order}: left subtree, then node, then right
subtree. Notice that since we are only examining the tree here --- not
modifying it --- we can declare the overriding method to take a
\lstinline!const!~\lstinline!Node*! rather than a \lstinline!Node*! and make the
method itself \lstinline!const!. Once again, compatibility of signatures,
not identity, is the key.

Finally, we might want to clean up the tree. We do so in
\emph{post-order} since we do not want to delete a node before we have
cleaned up its children!

\begin{emcppslisting}[language=C++]
struct CleanUp : Traverser<CleanUp>
{
    void visitPostOrder(Node*& n)
    {
        delete n;  // always necessary
        n = 0;     // might be omitted in a "raw" version of the type
    }
};
\end{emcppslisting}
    
\noindent Putting it all together, we can create a \lstinline!main! program that
creates a balanced tree to a depth of four and then labels it in
\emph{pre-order}, prints those labels in \emph{in-order}, and destroys
it in \emph{post-order}:

\begin{emcppslisting}[language=C++]
int main()
{
    Node* n = 0;                  // tree handle

    FillToDepth(4).traverse(n);   // (1) Create balanced tree.
    PreOrderLabel().traverse(n);  // (2) Label tree in pre-order.
    InOrderPrint().traverse(n);   // (3) Print labels in order.
    CleanUp().traverse(n);        // (4) Destroy tree in post-order.
    return 0
}
\end{emcppslisting}
    
\noindent Running this program results in a binary tree of height 4, as
illustrated in the code snippet below, and has reliably consistent output:

\begin{emcppslisting}[language=C++]
dcebgfhakjlinmo
\end{emcppslisting}
    
\begin{emcppslisting}[language=C++]
Level 0:                                a
                              .    '         '    .
Level 1:                b '                           ' i
                   .  '    '  .                   .  '    '  .
Level 2:        c                f              j              m
              /   \            /   \          /   \          /   \
Level 3:    d       e        g       h      k       l      n       o
\end{emcppslisting}
    
\noindent This use of the CRTP for traversal truly shines when the data structure to
be traversed is especially complex, such as an abstract-syntax-tree
(AST) representation of a computer program, where tree nodes have many
different types, with each type having custom ways of representing the
subtrees it contains. For example, a translation unit is a sequence of
declarations; a declaration can be a type, a variable, or a function;
functions have return types, parameters, and a compound statement; the
statement has substatements, expressions and so on. We would not want
to rewrite the traversal code for each new application. Given a reusable
CRTP-based traverser for our AST, we don't have to.

For example, consider writing a type that visits each integer literal
node in a given AST:

\begin{emcppslisting}[language=C++]
struct IntegerLiteralHandler : AstTraverser<IntegerLiteralHandler>
{
    void visit(IntegerLiteral* iLit)
    {
        // ... (do something with this integer literal)
    }
}
\end{emcppslisting}
    
\noindent The AST traverser, which would implement a separate empty \lstinline!visit!
overload for each syntactic node type in the grammar, would invoke our
derived \lstinline!visit! member function with every integer literal in the
program, regardless of where it appeared. This CRTP-based traverser
would also call many other \lstinline!visit! methods, but each of those would perform no action at all by default and would likely be elided at even modest compiler-optimization levels. Be aware, however, that, although we ourselves are
not rewriting the traversal code each time, the compiler is still doing
it because every CRTP instantiation produces a new copy of the traversal
code. If the traversal code is large and complex, the consequence might
be increased program size, that is, \romeogloss{code bloat}.

Finally, the CRTP can be used in a variety of situations for many
purposes,\footnote{\cite{fluentcpp17}} hence its \emph{curiously recurring} nature. Those uses invariably
benefit from (1) declaring the base-class template's default constructor
\emph{private} and (2) having that template befriend its type parameter,
which is possible only by means of the extended \lstinline!friend! syntax.
Thus, the CRTP base-class template can ensure, at compile time, that its
type argument is actually derived from the base class as required by the
pattern.


