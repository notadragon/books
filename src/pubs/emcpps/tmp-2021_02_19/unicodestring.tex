% 12 Feb 2021, revisions in; ready for Josh's code fixes



\emcppsFeature{
    short={Unicode Literals},
    long={Unicode String Literals},
}{unicode-string-and-character-literals}
\setcounter{table}{0}
\setcounter{footnote}{0}
\setcounter{lstlisting}{0}
%\section[Unicode Literals]{Unicode String Literals}\label{unicode-string-and-character-literals}

C++11 introduces a portable mechanism for ensuring that a literal is
encoded as UTF-8, UTF-16, or UTF-32.

\subsection[Description]{Description}\label{description-unicodestring}

According to the C++ Standard, the character encoding of string literals
is unspecified and can vary with the target platform or the configuration of
the compiler. In essence, the C++ Standard does not guarantee that the
string literal \lstinline!"Hello"! will be encoded as the
ASCII{\cprotect\footnote{In fact, C++ still fully supports platforms
  using EBCDIC, a rarely used alternative encoding to ASCII, as their primary text encoding.}} sequence
\lstinline!0x48,! \lstinline!0x65,! \lstinline!0x6C,! \lstinline!0x6C,! \lstinline!0x6F!
or that the character literal \lstinline!'X'! has the value
\lstinline!0x58!.

Table~\ref{unicodestring-table1} illustrates three new kinds of \emph{Unicode}-compliant
\emph{string literals}, each delineating the precise encoding of each
character.
 \begin{table}[h!]
\begin{center}
\begin{threeparttable}
\caption{Three new Unicode-compliant\\ literal strings}\label{unicodestring-table1}\vspace{1.5ex}
{\small \begin{tabular}{c|c|c}\thickhline
\rowcolor[gray]{.9}{\sffamily\bfseries Encoding} & {\sffamily\bfseries Syntax} & {\sffamily\bfseries Underlying Type}\\ \hline
UTF-8   &  \lstinline!u8"Hello"! & \lstinline!char!\tnote{a}\\ \hline
UTF-16 & \lstinline!u"Hello"!  & \lstinline!char16_t! \\ \hline
UTF-32  & \lstinline!U"Hello"! & \lstinline!char32_t!\\ \thickhline
\end{tabular}
} % end small
\begin{tablenotes}{\footnotesize
\item[a]{\lstinline!char8_t! in C++20}
}
\end{tablenotes}
\end{threeparttable}
\end{center} 
\end{table} 

\noindent A Unicode literal value is guaranteed to be encoded in UTF-8, UTF-16, or
UTF-32, for \lstinline!u8!, \lstinline!u!, and \lstinline!U! literals,
respectively:

\begin{emcppslisting}[language=C++]
char s0[] = "Hello";
    // unspecified encoding (albeit very likely ASCII)

char s1[] = u8"Hello";
    // guaranteed to be encoded as (ù{\codeincomments{\{0x48, 0x65, 0x6C, 0x6C, 0x6F, 0x0\}}}ù)
\end{emcppslisting}
    
\noindent C++11 also introduces \emph{universal character names} that provide a
reliably portable way of embedding Unicode code points in a C++ program.
They can be introduced by the \lstinline!\u! character
sequence followed by four hexadecimal digits or by the
\lstinline!\U! character sequence followed by eight
hexadecimal digits:

\begin{emcppslisting}[language=C++]
#include <cstdio>  // (ù{\codeincomments{std::puts}ù)                                               
void f()                                                                        
{                                                                               
    std::puts(u8"\U0001F378"); // Unicode code point in a UTF8 encoded literal  
}
\end{emcppslisting}
    
\noindent This output statement is guaranteed to emit the cocktail emoji
(\martini) to \lstinline!stdout!, assuming that the receiving end is configured to
interpret output bytes as UTF-8.

\subsection[Use Cases]{Use Cases}\label{use-cases}

\subsubsection[Guaranteed-portable encodings of literals]{Guaranteed-portable encodings of literals}\label{guaranteed-portable-encodings-of-literals}

The encoding guarantees provided by the Unicode literals can be useful,
such as in communication with other programs or network/IPC protocols that
expect character strings having a particular encoding.

As an example, consider an instant-messaging program in which both the
client and the server expect messages to be encoded in UTF-8. As part of
broadcasting a message to all clients, the server code uses UTF-8
Unicode literals to guarantee that every client will receive a sequence
of bytes they are able to interpret and display as human-readable text:

\begin{emcppshiddenlisting}
#include <string>                                                               
#include <ostream>                                                              
                                                                                
class Packet                                                                    
{                                                                               
    // ...                                                                      
public:                                                                         
    // ...                                                                      
    std::ostream&  operator<<(const char *message);                             
};                                                                              
                                                                                
                                                                                
class Server                                                                    
{                                                                               
    void broadcastPacket(const Packet& data);                                   
    // ...                                                                      
public:                                                                         
    void broadcastServerMessage(const std::string& message);                    
    // ...                                                                      
                                                                                
};                                                                              
                                                                                
void Server::broadcastServerMessage(const std::string& message)                 
{                                                                               
    Packet data;                                                                
    data << u8"Message from the server: '" << message << u8"'\n";               
    broadcastPacket(data);                                                      
}
\end{emcppshiddenlisting}
\begin{emcppslisting}[language=C++]
void Server::broadcastServerMessage(const std::string& utf8Message)
{
    Packet data;
    data << u8"Message from the server: '" << utf8Message << u8"'\n";

    broadcastPacket(data);
}
\end{emcppslisting}
    
\noindent Not using \lstinline!u8! literals in the code snippet above could potentially result in
nonportable behavior and might require compiler-specific flags to
ensure that the source is UTF-8 encoded.

\subsection[Potential Pitfalls]{Potential Pitfalls}\label{potential-pitfalls}

\subsubsection[Embedding Unicode graphemes]{Embedding Unicode graphemes}\label{embedding-unicode-graphemes}

The addition of Unicode string literals to the language did not bring
along an extension of the \romeogloss{basic source character set}: Even in C++11, the
default \romeogloss{basic source character set} is a subset of
ASCII.{\cprotect\footnote{Implementations are free to map characters
outside the basic source character set to sequences of its members,
resulting in the possibility of embedding other characters, such as emojis, in a C++ source file.}}

Developers might incorrectly assume that \lstinline!u8"!\martini\lstinline!"! is a
portable way of embedding a string literal representing the cocktail
emoji in a C++ program. The representation of the string literal, however, depends on what
encoding the compiler assumes for the source file, which can generally
be controlled through compiler flags. The only portable way of embedding
the cocktail emoji is to use its corresponding Unicode code point escape
sequence (\lstinline!u8"\U0001F378"!).

\subsubsection[Lack of library support for Unicode]{Lack of library support for Unicode}\label{lack-of-library-support-for-unicode}

Essential \romeogloss{vocabulary types}, such as \lstinline!std::string!, are
completely unaware of encoding. They treat any stored string as a
sequence of bytes. Even when correctly using Unicode string literals,
programmers unfamiliar with Unicode might be surprised by seemingly
innocent operations, such as asking for the size of a string
representing the cocktail emoji:

\begin{emcppslisting}[language=C++]
#include <cassert>                                                              
#include <string>                                                               
                                                                                
void f()                                                                        
{                                                                               
    std::string cocktail(u8"\U0001F378"); // big character (!)                  
    assert(cocktail.size() == 1);         // assertion failure (!)              
}
\end{emcppslisting}
    
\noindent Even though the cocktail emoji is a \emph{single} code point,
\lstinline!std::string::size! returns the number of code units (bytes) required to
encode it. The lack of Unicode-aware vocabulary types and utilities in
the Standard Library can be a source of defects and misunderstandings,
especially in the context of international program localization.

\subsubsection[Problematic treatment of UTF-8 in the type system]{Problematic treatment of UTF-8 in the type system}\label{utf-8-quirks}

UTF-8 string literals use \lstinline!char! as their \romeogloss{underlying type}. Such a
choice is inconsistent with UTF-16 and UTF-32 literals, which provide
their own distinct character types (\lstinline!char16_t! and
\lstinline!char32_t!). This precludes providing distinct behavior for UTF-8 encoded strings using function overloading or template specialization because they are indistinguishable from strings having the encoding of the execution character set. Furthermore, whether the underlying
type of \lstinline!char! is a \lstinline!signed! or \lstinline!unsigned! type is
itself implementation defined.{\cprotect\footnote{Note that
\lstinline!char! is distinct from both \lstinline!signed!~\lstinline!char! and
\lstinline!unsigned!~\lstinline!char!, but its behavior is guaranteed to be
  the same as one of those.}}
  
C++20
fundamentally changes how UTF-8 string literals work, by introducing a
new nonaliasing \lstinline!char8_t! character type whose representation
is guaranteed to match \lstinline!unsigned!~\lstinline!char!. The new
character type provides several benefits:
\begin{itemize}
\item{Ensures an \lstinline!unsigned! and distinct type for UTF-8 character data}
\item{Enables overloading for regular string literals versus UTF-8 string literals}
\item{Potentially achieves better performance due to the lack of special aliasing rules}
\end{itemize}
Unfortunately, the changes brought by C++20 are not
backward-compatible and might cause code targeting previous versions
of the language using \lstinline!u8! literals either to fail to compile
or to silently change its behavior when targeting C++20:

\begin{emcppslisting}[language=C++]
template <typename T> void print(const T*);  // (0)
void print(const char*);                     // (1)

void f()
{
    print(u8"text");  // invokes (1) prior to C++20, (0) afterwards
}
\end{emcppslisting}
        

\subsection[Annoyances]{Annoyances}\label{annoyances}

None so far

\subsection[See Also]{See Also}\label{see-also}

None so far

\subsection[Further Reading]{Further Reading}\label{further-reading}

None so far

