

\emcppsFeature{
    short={\lstinline!alignof!},
    tocshort={\TOCCode alignof},
    long={The (Compile-Time) {\SecCode alignof} Operator},
    toclong={The (Compile-Time) \lstinline!alignof! Operator},
    rhshort={\RHCode alignof},
}{alignof}
\setcounter{table}{0}
\setcounter{footnote}{0}
\setcounter{lstlisting}{0}
%\section[{\tt alignof}]{The (Compile-Time) {\SecCode alignof} Operator}\label{alignof}

The keyword \texttt{alignof} serves as a compile-time operator used to
query the \textbf{alignment requirements} of a type on the current
platform.

\subsection[Description]{Description}\label{description}

The \texttt{alignof} operator, when applied to a type, evaluates to an
\textbf{integral constant expression} that represents the
\textbf{alignment requirements} of its argument type. Similar to
\texttt{sizeof}, the (compile-time) value of \texttt{alignof} is of type
\texttt{std::size\_t}; unlike \texttt{sizeof} (which can accept an
arbitrary expressions), \texttt{alignof} is defined (in the C++
Standard) on only a type identifier but often works on expressions
anyway (see \textit{\titleref{annoyances-alignof}} on page~\pageref{annoyances-alignof}). The argument type,
\texttt{T}, supplied to \texttt{alignof} must be either a
\textbf{complete type}, a \textbf{reference type}, or an \textbf{array
type}. If \texttt{T} is a \textbf{complete type}, the result is the
alignment requirement for \texttt{T}. If \texttt{T} is a
\textbf{reference type}, the result is the alignment requirement for the
referenced type. If \texttt{T} is an \textbf{array type}, the
result is the alignment requirement for every element in the array{\cprotect\footnote{According to the C++11 Standard, ``An object of \textbf{array type} contains a contiguously allocated
  non-empty set of \texttt{N} subobjects of type \texttt{T}''  (\textbf{cpp11}, section 8.3.4, ``Arrays," paragraph 1, p. 188). Note
  that, for every type \texttt{T}, \texttt{sizeof(T)} is always a
  multiple of \texttt{alignof(T)}; otherwise, storing multiple \texttt{T} instances in an array would be impossible without padding,
  and the Standard explicitly prohibits padding between array elements.}}:

\begin{lstlisting}[language=C++]
static_assert(alignof(short)    == 2, "");  // complete type   ((ù{\codeincomments{sizeof}}ù) is 2)
static_assert(alignof(short&)   == 2, "");  // reference type  ((ù{\codeincomments{sizeof}}ù) is 2)
static_assert(alignof(short[5]) == 2, "");  // array type      ((ù{\codeincomments{sizeof}}ù) is 2)
static_assert(alignof(short[])  == 2, "");  // array type      ((ù{\codeincomments{sizeof}}ù) fails)
\end{lstlisting}
    

\subsubsection[\tt{alignof} Fundamental Types]{{\SubsubsecCode alignof} Fundamental Types}\label{alignof-fundamental-types}

Like their size, the alignment requirements of a \texttt{char},
\texttt{signed}~\texttt{char}, and \texttt{unsigned}~\texttt{char} are
all guaranteed to be 1 (i.e., 1-byte aligned) on every conforming
platform. For any other fundamental or pointer type \texttt{FPT},
\texttt{alignof(FPT)} (like \texttt{sizeof(FPT)}) is platform-dependent but is typically approximated well by the type's \textbf{natural
alignment} --- i.e.,
\texttt{sizeof(FPT)}~\texttt{==}~\texttt{alignof(FPT)}:

\begin{lstlisting}[language=C++]
static_assert(alignof(char)   == 1, "");  // guaranteed to be 1
static_assert(alignof(short)  == 2, "");  // platform-dependent
static_assert(alignof(int)    == 4, "");  //    "         "
static_assert(alignof(double) == 8, "");  //    "         "
static_assert(alignof(void*)  >= 4, "");  //    "         "
\end{lstlisting}
    

\subsubsection[\tt{alignof} User-Defined Types]{{\SubsubsecCode alignof} User-Defined Types}\label{alignof-user-defined-types}

When applied to user-defined types, alignment is always at least that of
the strictest alignment of any of its arguments' base or member objects.
Empty types are defined to have a size (and alignment) of 1 to
ensure that every object has a unique address.{\cprotect\footnote{An
exception is made for an object of a type derived from an empty (base)
class in that neither the size nor the alignment of the derived object
is affected by the derivation:

\begin{lstlisting}[language=C++, basicstyle={\ttfamily\footnotesize}]
struct S { int i; }      // size = 4; alignment = 4
struct E { };            // size = 1; alignment = 1
struct D : E { int i };  // size = 4; alignment = 4
\end{lstlisting}
      }} Compilers will (by default) avoid nonessential padding because any
extra padding would be wasteful of (e.g., cache)
memory{\cprotect\footnote{Compilers are permitted to increase alignment
(e.g., in the presence of virtual functions) but have certain restrictions
on padding. For example, they must ensure that each comprised type is
itself sufficiently aligned and that the alignment of the parent type
divides its size. This ensures that the fundamental identity for
arrays holds for all types, \texttt{T}, and positive integers,
\texttt{N}:

\begin{lstlisting}[language=C++, basicstyle={\ttfamily\footnotesize}]
T a[N]; static_assert(n == sizeof(a) / sizeof(*a));  // guaranteed
\end{lstlisting}
    
\noindent The alignment of user-defined types can be made artificially stricter
  (but not weaker) using the {\tt alignas} (see ``\titleref{alignas}" on page~\pageref{alignas}) specifier.
  Also note that, for \textbf{standard-layout types}, the address of the
  first member object is guaranteed to be the same as that of the parent
  object:

  \begin{lstlisting}[language=C++, basicstyle={\ttfamily\footnotesize}]
  struct S { int i; }
  class T { public: S s; }
  T t;
  static_assert(&t.s == &t,     "");  // guaranteed
  static_assert(&t.s == &t.s.i, "");  // guaranteed
  \end{lstlisting}
      
\noindent This property also holds for (e.g., anonymous) unions:

  \begin{lstlisting}[language=C++, basicstyle={\ttfamily\footnotesize}]
  struct { union { char c; float f; double d; } } u;
  static_assert(&u == &u.c, "");  // guaranteed
  static_assert(&u == &u.f, "");  // guaranteed
  static_assert(&u == &u.d, "");  // guaranteed
  \end{lstlisting}
      }}:

\begin{lstlisting}[language=C++]
struct S0 { };                          // (ù{\codeincomments{sizeof(S0)}}ù) is  (ù{\codeincomments{1}}ù); (ù{\codeincomments{alignof(S0)}}ù) is  (ù{\codeincomments{1}}ù)
struct S1 { char c; };                  // (ù{\codeincomments{sizeof(S1)}}ù) is  (ù{\codeincomments{1}}ù); (ù{\codeincomments{alignof(S1)}}ù) is  (ù{\codeincomments{1}}ù)
struct S2 { short s; };                 // (ù{\codeincomments{sizeof(S2)}}ù) is  (ù{\codeincomments{2}}ù); (ù{\codeincomments{alignof(S2)}}ù) is  (ù{\codeincomments{2}}ù)
struct S3 { char c; short s; };         // (ù{\codeincomments{sizeof(S3)}}ù) is  (ù{\codeincomments{4}}ù); (ù{\codeincomments{alignof(S3)}}ù) is  (ù{\codeincomments{2}}ù)
struct S4 { short s1; short s2; };      // (ù{\codeincomments{sizeof(S4)}}ù) is  (ù{\codeincomments{4}}ù); (ù{\codeincomments{alignof(S4)}}ù) is  (ù{\codeincomments{2}}ù)
struct S5 { int i; char c; };           // (ù{\codeincomments{sizeof(S5)}}ù) is  (ù{\codeincomments{8}}ù); (ù{\codeincomments{alignof(S5)}}ù) is  (ù{\codeincomments{4}}ù)
struct S6 { char c1; int i; char c2};   // (ù{\codeincomments{sizeof(S6)}}ù) is (ù{\codeincomments{12}}ù); (ù{\codeincomments{alignof(S6)}}ù) is  (ù{\codeincomments{4}}ù)
struct S7 { char c; short s; int i; };  // (ù{\codeincomments{sizeof(S7)}}ù) is  (ù{\codeincomments{8}}ù); (ù{\codeincomments{alignof(S7)}}ù) is  (ù{\codeincomments{4}}ù)
struct S8 { double d; };                // (ù{\codeincomments{sizeof(S8)}}ù) is  (ù{\codeincomments{8}}ù); (ù{\codeincomments{alignof(S8)}}ù) is  (ù{\codeincomments{8}}ù)
struct S9 { double d; char c};          // (ù{\codeincomments{sizeof(S9)}}ù) is (ù{\codeincomments{16}}ù); (ù{\codeincomments{alignof(S9)}}ù) is  (ù{\codeincomments{8}}ù)
struct SA { long double; };             // (ù{\codeincomments{sizeof(SA)}}ù) is (ù{\codeincomments{16}}ù); (ù{\codeincomments{alignof(SA)}}ù) is (ù{\codeincomments{16}}ù)
struct SB { long double; char c};       // (ù{\codeincomments{sizeof(SB)}}ù) is (ù{\codeincomments{32}}ù); (ù{\codeincomments{alignof(SB)}}ù) is (ù{\codeincomments{16}}ù)
\end{lstlisting}
    

\subsection[Use Cases]{Use Cases}\label{use-cases}

\subsubsection[Probing the alignment of a type during development]{Probing the alignment of a type during development}\label{probing-the-alignment-of-a-type-during-development}

Both \texttt{sizeof} and \texttt{alignof} are often used informally
during development and debugging to confirm the compiler's understanding
of those attributes for a given type on the current platform. For
example:

\begin{lstlisting}[language=C++]
#include <iostream>

void f()
{
    std::cout << " sizeof(double): " <<  sizeof(double) << '\n';  //  always 8
    std::cout << "alignof(double): " << alignof(double) << '\n';  // usually 8
}
\end{lstlisting}
    
\noindent Printing the size and alignment of a \texttt{struct} along with those of
each of its individual data members can lead to the discovery of
suboptimal ordering of data members (resulting in wasteful extra
padding). As an example, consider two \texttt{struct}s,
\texttt{Wasteful} and \texttt{Optimal}, having the same three data members but
in different order:

\begin{lstlisting}[language=C++]
struct Wasteful
{
    char   d_c;  // size =  1;  alignment = 1
    double d_d;  // size =  8;  alignment = 8
    int    d_i;  // size =  4;  alignment = 4
};               // size = 24;  alignment = 8

struct Optimal
{
    double d_d;  // size =  8;  alignment = 8
    int    d_i;  // size =  4;  alignment = 4
    char   d_c;  // size =  1;  alignment = 1
};               // size = 16;  alignment = 8
\end{lstlisting}
    
\noindent Both \texttt{alignof(Wasteful)} and \texttt{alignof(Optimal)} are
\texttt{8} but \texttt{sizeof(Wasteful)} is \texttt{24}, whereas
\texttt{sizeof(Optimal)} is only \texttt{16}. Even though these two
\texttt{struct}s contain the very same data members, the individual
alignment requirements of these members forces the compiler to insert
more total padding between the data members in \texttt{Wasteful} than is
necessary in \texttt{Optimal}:

\begin{lstlisting}[language=C++]
struct Wasteful
{
    char   d_c;           // size =  1;  alignment = 1
    char   padding_0[7];  // size =  7
    double d_d;           // size =  8;  alignment = 8
    int    d_i;           // size =  4;  alignment = 4
    char   padding_1[4];  // size =  4
};                        // size = 24;  alignment = 8

struct Optimal
{
    double d_d;           // size =  8;  alignment = 8
    int    d_i;           // size =  4;  alignment = 4
    char   d_c;           // size =  1;  alignment = 1
    char   padding_0[3];  // size =  3
};                        // size = 16;  alignment = 8
\end{lstlisting}
    

\subsubsection[Determining if a given buffer is sufficiently aligned]{Determining if a given buffer is sufficiently aligned}\label{determining-if-a-given-buffer-is-sufficiently-aligned}

The \texttt{alignof} operator can be used to determine if a given (e.g.,
\texttt{char}) buffer is suitably aligned for storing an object of
arbitrary type. As an example, consider the task of creating a
\textbf{value-semantic} class, \texttt{MyAny}, that represents an object
of arbitrary type{\cprotect\footnote{The C++17 Standard Library provides
the (nontemplate) class \texttt{std::any}, which is a type-safe
container for single values of \emph{any} \textbf{regular type}. The
implementation strategies surrounding alignment for \texttt{std::any}
in both \texttt{libstdc++} and \texttt{libc++} closely mirror those
used to implement the simplified \texttt{MyAny} class presented here.
Note that \texttt{std::any} also records the current \texttt{typeid}
(on construction or assignment) so that it can implement a
\texttt{const} template member function,
\texttt{bool}~\texttt{is<T>()}~\texttt{const}, to query, at runtime,
whether a specified type is currently the active one:

\begin{lstlisting}[language=C++, basicstyle={\ttfamily\footnotesize}]
void f(const std::any& object)
{
    if (object.is<int>()) { /* ... */ }
}
\end{lstlisting}
      }}:

\begin{lstlisting}[language=C++]
void f()
{
    MyAny obj = 10;               // can be initialized with values of any type
    assert(obj.as<int>() == 10);  // inner data can be retrieved at runtime

    obj = std::string{"hello"};   // can be reassigned from a value of any type
    assert(obj.as<std::string>() == "hello");
}
\end{lstlisting}
    
\noindent A straightforward implementation of \texttt{MyAny} would be to allocate
an appropriately sized block of dynamic memory each time a value of a
new type is assigned. Such a naive implementation would force memory
allocations even though the vast majority of values assigned in practice
are small (e.g., fundamental types), most of which would fit within the
space that would otherwise be occupied by just the pointer needed to
refer to dynamic memory. As a practical optimization, we might instead
consider reserving a small buffer (say, roughly{\cprotect\footnote{We
would likely choose a slightly larger value, e.g., 35 or 39, if that
space would otherwise be filled with essential padding due to overall
  alignment requirements.}} 32 bytes) within the footprint of the
\texttt{MyAny} object to hold the value provided (1) it will fit and (2)
the buffer is sufficiently aligned. The natural implementation of this
type --- the union of a char array and a \texttt{struct} (containing a
\texttt{char} pointer and a size) --- will naturally result in the minimal
alignment requirement of the \texttt{char*} (i.e., 4 on a 32-bit
platform and 8 on a 64-bit one){\cprotect\footnote{We could, in
  addition, use the {\tt alignas} attribute to ensure that
  the minimal alignment of \texttt{d\_buffer} was at least 8 (or even
  16):

  \begin{lstlisting}[language=C++, basicstyle={\ttfamily\footnotesize}]
  // ...
  alignas(8) char d_buffer[39];  // small buffer aligned to (at least) 8
  // ...
  \end{lstlisting}
      }}:

\begin{lstlisting}[language=C++]
class MyAny  // nontemplate class
{
    union
    {
        struct
        {
            char*       d_buf_p;  // pointer to dynamic memory if needed
            std::size_t d_size;   // for (ù{\codeincomments{d\_buf\_p}}ù); same alignment as (ù{\codeincomments{(char*)}}ù)
        } d_imp;  // Size/alignment of (ù{\codeincomments{d\_imp}}ù) is (ù{\codeincomments{sizeof(d\_buf\_p)}}ù) (e.g., 4 or 8).

        char d_buffer[39];         // small buffer aligned as a (ù{\codeincomments{(char*)}}ù)
    };  // Size of (ù{\codeincomments{union}}ù) is (ù{\codeincomments{39}}ù); alignment of (ù{\codeincomments{union}}ù) is (ù{\codeincomments{alignof(char*)}}ù).

    bool d_onHeapFlag;             // boolean (discriminator) for union (above)

public:
    template <typename T>
    MyAny(const T& x);               // (member template) constructor

    template <typename T>
    MyAny& operator=(const T& rhs);  // (member template) assignment operator

    template <typename T>
    const T& as() const;             // (member template) accessor

    // ...

};  // Size of (ù{\codeincomments{MyAny}}ù) is (ù{\codeincomments{40}}ù); alignment of (ù{\codeincomments{MyAny}}ù) is (ù{\codeincomments{alignof(char*)}}ù) (e.g., 8).
\end{lstlisting}
    
\noindent The (templated) constructor{\cprotect\footnote{In a real-world
implementation, a \emph{forwarding reference} would be used as the
parameter type of \texttt{MyAny}'s constructor to
\emph{perfectly forward} the argument object into the appropriate
  storage; see ``\titleref{forwardingref}" on page~\ref{forwardingref}.}} of \texttt{MyAny}
can then decide (potentially at compile time) whether to store the given
object \texttt{x} in the internal small buffer storage or on the heap,
depending on \texttt{x}'s size and alignment:

\begin{lstlisting}[language=C++]
template <typename T>
MyAny::MyAny(const T& x)
{
    if (sizeof(x) <= 39 && alignof(T) <= alignof(char*))
    {
        // Store (ù{\codeincomments{x}}ù) in place in the small buffer.
        new(d_buffer) T(x);
        d_onHeapFlag = false;
    }
    else
    {
        // Store (ù{\codeincomments{x}}ù) on the heap and a pointer to it in the small buffer.
        d_imp.d_buf_p = new T(x);
        d_imp.d_size = sizeof(x);
        d_onHeapFlag = true;
    }
}
\end{lstlisting}
    
\noindent Using the (compile-time) \texttt{alignof} operator in the constructor
above to check whether the alignment of \texttt{T} is compatible with
the alignment of the small buffer is necessary to avoid attempting to
store overly aligned objects in place --- even if they would fit in the
39-byte buffer. As an example, consider \texttt{long}~\texttt{double},
which on typical platforms has both a size and alignment of
\texttt{16}. Even though \texttt{sizeof(long}~\texttt{double)} (16) is not greater
than 39, \texttt{alignof(long}~\texttt{double)} (16) is greater than
that of \texttt{d\_buffer} (8); hence, attempting to store an instance
of \texttt{long}~\texttt{double} in the small buffer, \texttt{d\_buffer}, might
--- depending on where the \texttt{MyAny} object resides in memory --- result
in \textbf{undefined behavior}. User-defined types that either contain
a \texttt{long}~\texttt{double} or have had their alignments
artificially extended beyond 8 bytes are also unsuitable candidates for the
internal buffer even if they might otherwise fit:

\begin{lstlisting}[language=C++]
struct Unsuitable1 { long double d_value };
    // Size is 16 (<= 39), but alignment is 16 (> 8).

struct alignas(32) Unsuitable2 { };
    // Size is  1 (<= 39), but alignment is 32 (> 8).
\end{lstlisting}

\subsubsection[Monotonic memory allocation]{Monotonic memory allocation}\label{monotonic-memory-allocation}

A common pattern in software --- e.g., request/response in client/server
architectures --- is to quickly build up a complex data structure,
use it, and then quickly destroy it. A \textbf{monotonic allocator} is
a special-purpose memory allocator that returns a monotonically
increasing sequence of addresses into an arbitrary buffer, subject to
specific size and alignment requirements.{\cprotect\footnote{C++17
introduces an alternate interface to supply memory allocators via an
abstract base class. The C++17 Standard Library provides a complete
version of standard containers using this more interoperable design in
a sub-namespace, \texttt{std::pmr}, where \texttt{pmr} stands for
\textbf{polymorphic memory resource}. Also adopted as part of C++17
are two concrete memory resources,
\texttt{std::pmr::monotonic\_buffer\_resource} and
  \texttt{std::pmr::unsynchronized\_pool\_resource}.}} Especially when
the memory is allocated by a single thread, there are prodigious\footnote{see
\textbf{lakos16}} performance benefits to having unsynchronized raw
memory be taken directly off the (always hot) program stack. In what
follows, we will provide the building blocks of a monotonic memory
allocator wherein the \texttt{alignof} operator plays an essential role.

As a practically useful example, suppose that we want to create a
lightweight\linebreak[4] \mbox{\texttt{MonotonicBuffer}} class template that will allow us
to allocate raw memory directly from the footprint of the object. Just
by creating an object of an (appropriately sized) instance of this type
on the program stack, memory will naturally come from the stack. For
didactic reasons, we will start with a first pass at this class ---
ignoring alignment --- and then go back and fix it using
\texttt{alignof} so that it returns properly aligned memory:

\begin{lstlisting}[language=C++]
template <std::size_t N>
struct MonotonicBuffer  // first pass at a monotonic memory buffer
{
    char  d_buffer[N];  // fixed-size buffer
    char* d_top_p;      //  next available address

    MonotonicBuffer() : d_top_p(d_buffer) { }
       // Initialize the next available address to be the start of the buffer.

    template <typename T>
    void* allocate()             // BAD IDEA --- doesn't address alignment
    {
        void* result = d_top_p;  // Remember the current next-available address.
        d_top_p += sizeof(T);    // Reserve just enough space for this type.
        return result;           // Return the address of the reserved space.
    }
};
\end{lstlisting}
    
\noindent \texttt{MonotonicBuffer} is a class template with one integral template
parameter that controls the size of the \texttt{d\_buffer} member from
which it will dispense memory. Note that, while \texttt{d\_buffer} has
an alignment of 1, the \texttt{d\_top\_p} member, used to keep track of
the next available address, has an alignment that is typically 4 or 8
(corresponding to 32-bit and 64-bit architectures, respectively). The
constructor merely initializes the next-address pointer,
\texttt{d\_top\_p}, to the start of the local memory pool,
\texttt{d\_buffer[N]}. The interesting part is how the \texttt{allocate}
function manages to return a sequence of addresses corresponding to
objects allocated sequentially from the local pool:

\begin{lstlisting}[language=C++]
MonotonicBuffer<20> mb;  // On a 64-bit platform, the alignment will be 8.
char*   cp = static_cast<char*  >(mb.allocate<char  >());  // &d_buffer[ 0]
double* dp = static_cast<double*>(mb.allocate<double>());  // &d_buffer[ 1]
short*  sp = static_cast<short* >(mb.allocate<short >());  // &d_buffer[ 9]
int*    ip = static_cast<int*   >(mb.allocate<int   >());  // &d_buffer[11]
float*  fp = static_cast<float* >(mb.allocate<float >());  // &d_buffer[15]
\end{lstlisting}
    
\noindent The predominant problem with this first attempt at an implementation of
\texttt{allocate} is that the addresses returned do not necessarily
satisfy the minimum alignment requirements of the supplied type. A
secondary concern is that there is no internal check to see if
sufficient room remains. To patch this faulty implementation,
we will need a function that, given an initial address and an alignment
requirement, returns the amount by which the address must be rounded up
(i.e., necessary padding) for an object having that alignment
requirement to be properly aligned:

\begin{lstlisting}[language=C++]
std::size_t calculatePadding(const char* address, std::size_t alignment)
    // Requires: alignment is a (non-negative, integral) power of 2.
{
    // rounding up (ù{\codeincomments{X}}ù) to (ù{\codeincomments{N}}ù) (where (ù{\codeincomments{N}}ù) is a power of 2): (ù{\codeincomments{(x + N - 1) \& \textasciitilde(N - 1)}}ù)
    const std::size_t maxA = alignof(std::max_align_t);
    const std::size_t a = reinterpret_cast<std::size_t>(address) & (maxA - 1);
    const std::size_t am1 = alignment - 1;
    const std::size_t alignedAddress = (a + am1) & ~am1;  // round up
    return alignedAddress - a;                            // return padding
}
\end{lstlisting}
    
\noindent Armed with the
\texttt{calculatePadding} helper function (above), we are all set to write the final (correct) version of the \texttt{allocate} method of the \texttt{MonotonicBuffer} class template:

\begin{lstlisting}[language=C++]
template <typename T>
void* MonotonicBuffer::allocate()
{
    // Calculate just the padding space needed for alignment.
    const std::size_t padding = calculatePadding(d_top_p, alignof(T));

    // Calculate the total amount of space needed.
    const std::size_t delta = padding + sizeof(T);

    // Check to make sure the properly aligned object will fit.
    if (delta > d_buffer + N - d_top_p)  // if (Needed > Total - Used)
    {
        return 0;  // not enough properly aligned unused space remaining
    }

    // Reserve needed space; return the address for a properly aligned object.
    void* alignedAddress = d_top_p + padding;  // Align properly for (ù{\codeincomments{T}}ù) object.
    d_top_p += delta;                          // Reserve memory for (ù{\codeincomments{T}}ù) object.
    return alignedAddress;                     // Return memory for (ù{\codeincomments{T}}ù) object.
}
\end{lstlisting}
    
\noindent Using this corrected implementation that uses \texttt{alignof} to pass
the alignment of the supplied type \texttt{T} to the
\texttt{calculatePadding} function, the addresses returned from the
benchmark example (above) would be different{\cprotect\footnote{Note
that on a 32-bit architecture, the \texttt{d\_top\_p} character
pointer would be only four-byte aligned, which means that the entire
buffer might be only four-byte aligned. In that case, the respective
offsets for \texttt{cp}, \texttt{dp}, \texttt{sp}, \texttt{ip}, and
\texttt{bp} in the example for the aligned use case might sometimes
instead be 0, 4, 12, 16, and \texttt{nullptr}, respectively. If
  desired, we can use the \texttt{alignas}
  attribute/keyword to artificially constrain the \texttt{d\_buffer}
  data member always to reside on a maximally aligned address boundary,
  thereby improving consistency of behavior, especially on 32-bit
  platforms.}}:

\begin{lstlisting}[language=C++]
MonotonicBuffer<20> mb;  // Assume 64-bit platform (8-byte aligned).
char*   cp = static_cast<char*  >(mb.allocate<char  >());  // &d_buffer[ 0]
double* dp = static_cast<double*>(mb.allocate<double>());  // &d_buffer[ 8]
short*  sp = static_cast<short* >(mb.allocate<short >());  // &d_buffer[16]
int*    ip = static_cast<int*   >(mb.allocate<int   >());  // 0 (out of space)
bool*   bp = static_cast<bool*  >(mb.allocate<bool  >());  // &d_buffer[18]
\end{lstlisting}
    
\noindent In practice, an object that allocates memory, such as a \texttt{vector}
or a \texttt{list}, will be constructed with an object that allocates and
deallocates memory that is guaranteed to be either \textbf{maximally
aligned}, \textbf{naturally aligned}, or sufficiently aligned to satisfy
an optionally specified alignment requirement.

Finally, instead of returning a null pointer when the buffer was
exhausted, we would typically have the concrete allocator fall back to a
geometrically growing sequence of dynamically allocated blocks; the
\texttt{allocate} method would then fail (i.e., a
\texttt{std::bad\_alloc} exception would somehow be thrown) only if all
available memory were exhausted and the \textbf{\texttt{new} handler}
were unable to acquire more memory yet still opted to return control
to its caller.

\subsection[Annoyances]{Annoyances}\label{annoyances-alignof}

\subsubsection[\tt{alignof} (unlike \tt{sizeof}) is defined only on types]{{\SubsubsecCode alignof} (unlike {\SubsubsecCode sizeof}) is defined only on types}\label{alignof-(unlike-sizeof)-is-defined-only-on-types}

The (compile-time) \texttt{sizeof} operator comes in two different
forms: one accepting a \emph{type} and the other accepting an
\emph{expression}. The C++ Standard currently requires that
\texttt{alignof} support only the former{\cprotect\footnote{Although
the Standard does not require \texttt{alignof} to work on arbitrary
expressions, \texttt{alignof} is a common GNU extension and most compilers support
  it. Both Clang and GCC will warn only if \texttt{-Wpedantic} is set.}}:

\begin{lstlisting}[language=C++]
static_assert(sizeof(int)  == 4, "");     // OK, (ù{\codeincomments{int}}ù) is a type.
static_assert(alignof(int) == 4, "");     // OK, (ù{\codeincomments{int}}ù) is a type.
static_assert(sizeof(3 + 2))  == 4, "");  // OK, (ù{\codeincomments{3 + 2}}ù) is an expression.
static_assert(alignof(3 + 2)) == 4, "");  // Error, (ù{\codeincomments{3 + 2}}ù) is not a type.
\end{lstlisting}
    
\noindent This asymmetry can result in a need to leverage
\texttt{decltype} (see ``\titleref{decltype}" on page~\pageref{decltype}) when inspecting an expression instead
of a type:

\begin{lstlisting}[language=C++]
int f()
{
    enum { e_SUCCESS, e_FAILURES } result;
    std::cout << "size: " << sizeof(result) << '\n';
    std::cout << "alignment:" << alignof(decltype(result)) << '\n';
}
\end{lstlisting}
    
\noindent The same sort of issue occurs in conjunction with modern \textbf{type
inference} features such as \texttt{auto} (see ``\titleref{auto-feature}" on page~\pageref{auto-feature}) and
generic lambdas (see ``\titleref{genericlambda}" on page~\pageref{genericlambda}). As a real-world example, consider the
generic lambda (C++14) being used to introduce a small
\emph{local function} that prints out information regarding the size and
alignment of a given \texttt{object}, likely for debugging purposes:

\begin{lstlisting}[language=C++]
auto printTypeInformation = [](auto object)
{
    std::cout << "     size: " << sizeof(object) << '\n'
              << "alignment: " << alignof(decltype(object)) << '\n';
};
\end{lstlisting}
    
\noindent Because there is no explicit type available within the body of the
\texttt{printTypeInformation} lambda,{\cprotect\footnote{In C++20, referring to the type of a generic lambda parameter explicitly is possible (due to the addition to lambdas of some familiar template syntax):

\begin{lstlisting}[language=C++, basicstyle={\ttfamily\footnotesize}]
auto printTypeInformation = []<typename T>(T object)
{
    std::cout << "     size: " << sizeof(T) << '\n'
              << "alignment: " << alignof(T) << '\n';
};
\end{lstlisting}
      }} a programmer wishing to remain entirely within the C++
Standard{\cprotect\footnote{Note that \texttt{alignof(object)} will work
on every major compiler (GCC~10.x, Clang~10.x, and MSVC~19.x) as a
  nonstandard extension.}} is forced to use the
\texttt{decltype} construct explicitly to first obtain
the type of \texttt{object} before passing it on to \texttt{alignof}.

\subsection[See Also]{See Also}\label{see-also}

\begin{itemize}
\item{``\titleref{alignas}" — Safe C++11 feature that can be used to provide an artificially stricter alignment (e.g., more than \textbf{natural alignment}).}
\item{``\titleref{decltype}" — Safe C++11 feature that helps work around \texttt{alignof}'s limitation of accepting only a type, not an expression (see \textit{\titleref{annoyances-alignof}} on page~\pageref{annoyances-alignof}).}
\end{itemize}

\subsection[Further Reading]{Further Reading}\label{further-reading}

None so far



