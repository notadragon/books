% 5 Feb 2021, copyedits in and proofed


\emcppsFeature{
    short={Delegating Ctors},
    long={Constructors Calling Other Constructors},
}{delegating-constructors}
\setcounter{table}{0}
\setcounter{footnote}{0}
\setcounter{lstlisting}{0}
%\section[Delegating Ctors]{Constructors Calling Other Constructors}\label{delegating-constructors}

Delegating constructors are constructors of a class that delegate initialization to another
constructor of the same class.

\subsection[Description]{Description}\label{description}

A \romeoglossf{delegating constructor} is a constructor of a
\romeoglossf{user-defined type (UDT)} --- i.e., \lstinline!class!, \lstinline!struct!, or
\lstinline!union! --- that invokes another constructor defined for the same
\romeogloss{UDT} as part of its initialization of an object of that type.
The syntax for invoking another constructor is to specify
the name of the type as the only element in the \romeoglossf{member
initializer list}:

%\begin{lstlisting}[language=C++]
%struct S0
%{
%    int d_i;
%
%    S0(int i) : d_i(i)        { }  // non-delegating constructor
%    S0()      : S0(0)         { }  // OK, delegates to (ù{\codeincomments{S0(int)}}ù)
%    S0(bool)  : S0(0), d_i(0) { }  // error: delegation must be on its own
%};
%\end{lstlisting}
\begin{emcppslisting}
#include <string>  // std::string

struct S0
{
  int         d_i;
  std::string d_s;

  S0(int i)         : d_i(i)        {} // nondelegating constructor
  S0()              : S0(0)         {} // OK, delegates to (ù{\codeincomments{S0(int)}}ù)
  S0(const char *s) : S0(0), d_s(s) {} // Error, delegation must be on its own
};  
\end{emcppslisting}
    
\noindent Multiple delegating constructors can be chained together (one calling
exactly one other) so long as cycles are avoided (see \intraref{ctordelegrating-potential-pitfalls}{delegation-cycles}).
%{\it\titleref{ctordelegrating-potential-pitfalls}:} {\it\titleref{delegation-cycles}} on page~\pageref{delegation-cycles}). 
Once a \emph{target} (i.e., invoked via delegation) constructor returns,
the body of the delegator is invoked:

\begin{emcppslisting}[language=C++]
#include <iostream>  // (ù{\codeincomments{std::cout}}ù)

struct S1
{
    S1(int, int)            { std::cout << 'a'; }
    S1(int)      : S1(0, 0) { std::cout << 'b'; }
    S1()         : S1(0)    { std::cout << 'c'; }
};

void f()
{
    S1 s;  // OK, prints (ù{\codeincomments{"abc"}}ù) to (ù{\codeincomments{stdout}}ù)
}
\end{emcppslisting}
    
\noindent If an exception is thrown while executing a nondelegating constructor,
the object being initialized is considered only \romeoglossf{partially
constructed} (i.e., the object is not yet known to be in a valid state),
and hence its destructor will \emph{not} be
invoked:

\begin{emcppslisting}[language=C++]
#include <iostream>  // (ù{\codeincomments{std::cout}}ù)

struct S2
{
    S2()  { std::cout << "S2() ";  throw 0; }
    ~S2() { std::cout << "~S2() ";          }
};

void f() try { S2 s; } catch(int) { }
    // prints only "(ù{\codeincomments{S2() }}ù)" to (ù{\codeincomments{stdout}}ù) (the destructor of (ù{\codeincomments{S2}}ù) is never invoked)
\end{emcppslisting}   
    
\noindent Although the destructor of a \romeogloss{partially
constructed} object will not be invoked, the destructors of
each successfully constructed base and of data members will still be
invoked:

\begin{emcppslisting}[language=C++]
#include <iostream>

using std::cout;
struct A { A() { cout << "A() "; } ~A() { cout << "~A() "; } };
struct B { B() { cout << "B() "; } ~B() { cout << "~B() "; } };

struct C : B
{
    A d_a;

    C()  { cout << "C() "; throw 0; }  // nondelegating constructor that throws
    ~C() { cout << "~C() ";         }  // destructor that never gets called
};

void f() try { C c; } catch(int) { }
    // prints (ù{\codeincomments{"B() A() C() \textasciitilde A() \textasciitilde B()"}}ù) to (ù{\codeincomments{stdout}}ù)
\end{emcppslisting}
    
\noindent Notice that base-class \lstinline!B! and member \lstinline!d_a! of type
\lstinline!A! were fully constructed, and so their respective destructors
are called, even though the destructor for class \lstinline!C! itself is never \mbox{executed}. 
However, if an exception is thrown in the body of a delegating
constructor, the object being initialized is considered \romeoglossf{fully
constructed}, as the target constructor must have returned control to
the delegator; hence, the object's destructor \emph{is} invoked:

\begin{emcppslisting}[language=C++]
#include <iostream>  // (ù{\codeincomments{std::cout}}ù)

struct S3
{
    S3()           { std::cout << "S3() "              }
    S3(int) : S3() { std::cout << "S3(int) "; throw 0; }
    ~S3()          { std::cout << "~S3() "             }
};

void f() try { S3 s(0); } catch(int) { }
    // prints "(ù{\codeincomments{S3() S3(int) \texttt{\~{}}S3() }}ù)" to (ù{\codeincomments{stdout}}ù)
\end{emcppslisting}
    

\subsection[Use Cases]{Use Cases}\label{ctordelegating-use-cases}

\subsubsection[Avoiding code duplication among constructors]{Avoiding code duplication among constructors}\label{avoiding-code-duplication-among-constructors}

Avoiding gratuitous code duplication is considered by many to be a best
practice. Having one ordinary member function call another has always
been an option, but having one constructor invoke another constructor
directly has not. Classic workarounds included repeating the code or
else factoring the code into a private member function that would be
called from multiple constructors. The drawback with this workaround is
that the private member function, not being a constructor, would be unable to
make use of \romeogloss{member initialization lists} to initialize base classes and data members efficiently. As of C++11, \emph{delegating
constructors} can be used to minimize code duplication when some of
the same operations are performed across multiple constructors without
having to forgo efficient initialization.

As an example, consider an \lstinline!IPV4Host! class representing a
network endpoint that can be constructed either (1) by a 32-bit address
and a 16-bit port or (2) by an IPV4 string with
\lstinline!XXX.XXX.XXX.XXX:XXXXX! format{\cprotect\footnote{Note that
this initial design might itself be suboptimal in that the
representation of the IPV4 address and port value might profitably be
factored out into a separate \romeoglossf{value-semantic} class, say,
\lstinline!IPV4Address!, that itself might be constructed in multiple ways;
  see \intraref{ctordelegrating-potential-pitfalls}{suboptimal-factoring}.}}:
%  {\it\titleref{ctordelegrating-potential-pitfalls}:} {\it\titleref{suboptimal-factoring}} on page~\pageref{suboptimal-factoring}.}}:

\begin{emcppslisting}[language=C++]
#include <cstdint> // (ù{\codeincomments{std::uint16\_t}}ù), (ù{\codeincomments{std::uint32\_t}}ù)
#include <string>

class IPV4Host
{
     // ...
private:                                                                      
    int connect(std::uint32_t address, std::uint16_t port);

public:
    IPV4Host(std::uint32_t address, std::uint16_t port)
    {
        if (!connect(address, port))  // code duplication: BAD IDEA
        {
            throw ConnectionException{address, port};
        }
    }

    IPV4Host(const std::string& ip)
    {
        std::uint32_t address = extractAddress(ip);
        std::uint16_t port = extractPort(ip);

        if (!connect(address, port))  // code duplication: BAD IDEA
        {
            throw ConnectionException{address, port};
        }
    }
};
\end{emcppslisting}
    
\noindent Prior to C++11, working around such code duplication would require the
introduction of a separate, private helper function that
would be called by each of the constructors:

\begin{emcppslisting}[language=C++]
// C++03 (obsolete)
#include <cstdint>  // (ù{\codeincomments{std::uint16\_t}}ù), (ù{\codeincomments{std::uint32\_t}}ù)

class IPV4Host
{
    // ...

private:                                                                      
    int connect(std::uint32_t address, std::uint16_t port);
    void init(std::uint32_t address, std::uint16_t port)  // helper function
    {
        if (!connect(address, port))  // factored implementation of needed logic
        {
            throw ConnectionException{address, port};
        }
    }

public:
    IPV4Host(std::uint32_t address, std::uint16_t port)
    {
        init(address, port);  // Invoke factored private helper function.
    }

    IPV4Host(const std::string& ip)
    {
        std::uint32_t address = extractAddress(ip);
        std::uint16_t port = extractPort(ip);

        init(address, port);  // Invoke factored private helper function.
    }
};
\end{emcppslisting}
    
\noindent With C++11 delegating constructors, the constructor accepting a string can be rewritten to
delegate to the one accepting \lstinline!address! and \lstinline!port!,
avoiding repetition without having to use a private function:

\begin{emcppslisting}[language=C++]
#include <cstdint> // (ù{\codeincomments{std::uint16\_t}}ù), (ù{\codeincomments{std::uint32\_t}}ù)
#include <string>

class IPV4Host
{
     // ...
private:                                                                      
    int connect(std::uint32_t address, std::uint16_t port);

public:
    IPV4Host(std::uint32_t address, std::uint16_t port)
    {
        if(!connect(address, port))
        {
            throw ConnectionException{address, port};
        }
    }

    IPV4Host(const std::string& ip)
        : IPV4Host{extractAddress(ip), extractPort(ip)}
    {
    }
};
\end{emcppslisting}
    
\noindent Using delegating constructors results in less boilerplate and fewer runtime
operations, as data members and base classes can be initialized
directly through the \romeogloss{member initialization list}. 

\subsection[Potential Pitfalls]{Potential Pitfalls}\label{ctordelegrating-potential-pitfalls}

\subsubsection[Delegation cycles]{Delegation cycles}\label{delegation-cycles}

If a constructor delegates to itself either directly or indirectly, the
program is \romeoglossf{ill formed, no diagnostic required (IFNDR)}. While some compilers can, under certain conditions, detect delegation cycles at compile time, they are neither 
required nor necessarily able to do so. For example, even the simplest delegation cycles might not result in a diagnostic from a compiler{\cprotect\footnote{GCC 10.x does not detect this delegation
cycle at compile time and produces a binary that, if run, will
necessarily exhibit \romeoglossf{undefined behavior}. Clang 10.x, on the
other hand, halts compilation with a helpful error message:

\begin{emcppslisting}[style=plain, basicstyle={\ttfamily\footnotesize}]
error: constructor for S creates a delegation cycle
\end{emcppslisting}\vspace*{-1ex}
      }}:

\begin{emcppslisting}[language=C++]
struct S  // Object
{
    S(int)  : S(true) { }  // delegating constructor
    S(bool) : S(0)    { }  // delegating constructor
};
\end{emcppslisting}
    
\subsubsection[Suboptimal factoring]{Suboptimal factoring}\label{suboptimal-factoring}

The need for delegating constructors might result from initially
suboptimal factoring --- e.g., in the case where the same \romeoglossf{value}
is being presented in different forms to a variety of different
\romeoglossf{mechanisms}. For example, consider the \lstinline!IPV4Host! class
in \intrarefsimple{ctordelegating-use-cases}. %{\titleref{ctordelegating-use-cases}} 
%(which starts on page~\pageref{ctordelegating-use-cases}). 
While having two constructors to
initialize the host might be appropriate, if either (1) the number of
ways of expressing the same value increases or (2) the number of
consumers of that value increases, we might be well advised to create a
separate \romeoglossf{value-semantic} type, e.g., \lstinline!IPV4Address!, to
represent that value{\cprotect\footnote{The notion that each component
in a subsystem ideally performs one focused function well is sometimes
referred to as separation of (logical) concerns or
  fine-grained (physical) factoring; see \cite{dijkstra82} and see \cite{lakos20},
  sections 0.4, 3.2.7, and 3.5.9, pp.~20--28, 529--530, and 674--676,
  respectively.}}:

\begin{emcppslisting}[language=C++]
#include <cstdint> // (ù{\codeincomments{std::uint16\_t}}ù), (ù{\codeincomments{std::uint32\_t}}ù)
#include <string>

class IPV4Address
{
    std::uint32_t d_address;
    std::uint16_t d_port;

public:
    IPV4Address(std::uint32_t address, std::uint16_t port)
        : d_address{address}, d_port{port}
    {
    }

    IPV4Address(const std::string& ip)
        : IPV4Address{extractAddress(ip), extractPort(ip)}
    {
    }
};
\end{emcppslisting}
    
\noindent Note that \lstinline!IPV4Address! itself makes use of delegating
constructors but as a purely private, encapsulated implementation
detail. With the introduction of \lstinline!IPV4Address! into the codebase,
\lstinline!IPV4Host! (and similar components requiring an
\lstinline!IPV4Address! value) can be redefined to have a single
constructor (or other previously overloaded member function) taking
an \lstinline!IPV4Address! object as an argument.

\subsection[Annoyances]{Annoyances}\label{annoyances}

\hspace*{\fill}

\subsection[See Also]{See Also}\label{see-also}

\begin{itemize}
\item{\seealsoref{forwardingref}{\seealsolocationc}provides perfect forwarding of arguments from one ctor to another.}
\item{\seealsoref{variadictemplate}{\seealsolocationc}describes how to implement constructors that forward an arbitrary list of arguments to other constructors.}
\end{itemize}

\subsection[Further Reading]{Further Reading}\label{further-reading}

\hspace*{\fill}



