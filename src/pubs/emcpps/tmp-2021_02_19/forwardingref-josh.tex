
% 01 Feb, 2021 JMB - packet 4 compilation fixes
% 10 Feb 2021, cleanup before sending for revisions

\emcppsFeature{
    short={Forwarding References},
    long={Forwarding {\SecCode \&\&} References},
    toclong={Forwarding \lstinline!\&\&! References},
}{forwardingref}
\setcounter{table}{0}
\setcounter{footnote}{0}
\setcounter{lstlisting}{0}
%\section[Forwarding References]{Forwarding {\SecCode \&\&} References}\label{forwardingref}

A forwarding reference (\lstinline!T&&!) --- distinguishable from an
\romeovalue{rvalue} reference (\lstinline!&&!) (see \featureref{\locationc}{Rvalue-References}) 
%``\titleref{Rvalue-References}" on page~\pageref{Rvalue-References}) 
based only on context
--- is a distinct, special kind of reference that (1) binds
universally to the result of an expression of \emph{any} \romeogloss{value
category} and (2) preserves aspects of that \romeogloss{value category} so
that the bound object can be \emph{moved from}, if appropriate.

\subsection[Description]{Description}\label{description-forwardingref}

Sometimes we want the same reference to be able to bind to either an
\romeovalue{lvalue} or an \romeovalue{rvalue} and then later be able to discern,
from the reference itself, whether the result of the original expression
was eligible to be \emph{moved from}. A \emph{forwarding reference}
(e.g., \lstinline!forRef! in the example below) used in the interface of a function
\emph{template} (e.g., \lstinline!myFunc!) affords precisely this
capability and will prove invaluable for the purpose of conditionally
moving, or else copying, an object from within the function template's
body:

\begin{emcppslisting}
template <typename T>
void myFunc(T&& forRef)
{
    // It is possible to check if (ù{\codeincomments{forRef}}ù) is eligible to be moved from or not
    // from within the body of (ù{\codeincomments{myFunc}}ù).
}
\end{emcppslisting}

\noindent In the definition of the \lstinline!myFunc! function template in the example above, the
parameter \lstinline!forRef! syntactically appears to be a
non-\lstinline!const! reference to an \romeovalue{rvalue} of type \lstinline!T!; in
this very precise context, however, the same \lstinline!T&&! syntax
designates a \romeogloss{forwarding reference}, with the effect of retaining
the original value category of the object bound to the argument
\lstinline!forRef!; see \intraref{description-forwardingref}{identifying-forwarding-references}. 
%\textit{\titleref{description-forwardingref}: \titleref{identifying-forwarding-references}} on page~\pageref{identifying-forwarding-references}. 
The \lstinline!T&&! syntax
represents a \emph{forwarding} reference --- as opposed to an
\romeovalue{rvalue} reference --- whenever an \emph{individual} function template has a type parameter (e.g., \lstinline!T!) and
an unqualified function parameter of type that is exactly
\lstinline!T&&! (e.g., \lstinline!const!~\lstinline!T&&! would be an
\romeovalue{rvalue} reference, not a forwarding reference).

Consider, for example, a function \lstinline!f! that takes a single
argument by reference and then attempts to use it to invoke one of two
overloads of a function \lstinline!g!, depending on whether the original
argument was an \romeovalue{lvalue} or \romeovalue{rvalue}:

\begin{emcppslisting}[emcppsbatch=e1]
struct S { /* some UDT that might benefit from being able to be moved */ };

void g(const S&);  // target function - overload for (ù{\codeincomments{const}}ù) (ù{\codeincomments{int}}ù) lvalues
void g(S&&);       // target function - overload for (ù{\codeincomments{int}}ù) rvalues only

template <typename T>
void f(T&& forRef); // forwards to target overload (ù{\codeincomments{g}}ù) based on value category
\end{emcppslisting}

\noindent In
theory, we could have chosen a non-\lstinline!const! \romeovalue{lvalue}
reference along with a modifiable \romeovalue{rvalue} reference here for
\emph{pedagogical} symmetry; such an inherently unharmonious overload
set would, however, not typically occur in practice; see \featureref{\locationc}{Rvalue-References}.
%``\titleref{Rvalue-References}" on page~\pageref{Rvalue-References}. 
In this specific case --- where \lstinline!f! is a function template,
\lstinline!T! is a template type parameter, and the type of the parameter
itself is exactly \lstinline!T&&! --- the \lstinline!forRef! function
parameter (in the code snippet above) denotes a \emph{forwarding reference}. If \lstinline!f!\label{f-invoked-example}
is invoked with an \romeovalue{lvalue}, \lstinline!forRef! is an \romeovalue{lvalue}
reference; otherwise \lstinline!forRef! is an \romeovalue{rvalue} reference.
Given the dual nature of \lstinline!forRef!, one rather verbose way of
determining the original value category of the passed argument would be
to use the \lstinline!std::is_lvalue_reference! \romeogloss{type trait} on
\lstinline!forRef! itself:

\begin{emcppslisting}[emcppsbatch=e1]
#include <type_traits>  // (ù{\codeincomments{std::is\_lvalue\_reference}}ù)
#include <utility>  // (ù{\codeincomments{std::move}}ù)

template <typename T>
void f(T&& forRef)      // (ù{\codeincomments{forRef}}ù) is a forwarding reference.
{
    if (std::is_lvalue_reference<T>::value)  // using a C++11 type trait
    {
        g(forRef);             // propagates (ù{\codeincomments{forRef}}ù) as an *lvalue*
    }                          // invokes (ù{\codeincomments{g(const S\&)}}ù)
    else
    {
        g(std::move(forRef));  // propagates (ù{\codeincomments{forRef}}ù) as an *rvalue*
    }                          // invokes (ù{\codeincomments{g(S\&\&)}}ù)
}
\end{emcppslisting}

\noindent The \lstinline!std::is_lvalue_reference<T>::value! predicate above asks
the question, ``Did the object bound to \lstinline!fRef! originate
from an \romeovalue{lvalue} expression?'' and allows the developer to branch on the
answer. A more concise but otherwise equivalent implementation is
generally preferred; see \intraref{description-forwardingref}{the-std::forward-utility}:
%\textit{\titleref{description-forwardingref}: \titleref{the-std::forward-utility}} on page~\pageref{the-std::forward-utility}:

\begin{emcppshiddenlisting}[emcppsbatch=e2]
struct S {};
void g(const S&);
void g(S&&);
\end{emcppshiddenlisting}
\begin{emcppslisting}[emcppsbatch=e2]
#include <utility>  // (ù{\codeincomments{std::forward}}ù)

template <typename T>
void f(T&& forRef)
{
    g(std::forward<T>(forRef));
        // same as (ù{\codeincomments{g(std::move(forRef))}}ù) if and only if (ù{\codeincomments{forRef}}ù) is an *rvalue*
        // reference; otherwise, equivalent to (ù{\codeincomments{g(forRef)}}ù)
}
\end{emcppslisting}

\noindent A \lstinline!client! function invoking \lstinline!f! will enjoy the same
behavior with either of the two implementation alternatives offered
above:

\begin{emcppslisting}[emcppsbatch=e2]
void client()
{
    S s;
    f(s);    // Instantiates (ù{\codeincomments{f<S\&>}}ù) -- (ù{\codeincomments{fRef}}ù) is an lvalue reference ((ù{\codeincomments{S\&}}ù)).
             // The function (ù{\codeincomments{f<S\&>}}ù) will end up invoking (ù{\codeincomments{g(S\&)}}ù).

    f(S());  // Instantiates (ù{\codeincomments{f<S\&\&>}}ù) -- (ù{\codeincomments{fRef}}ù) is an rvalue reference ((ù{\codeincomments{S\&\&}}ù)).
             // The function (ù{\codeincomments{f<S\&\&>}}ù) will end up invoking (ù{\codeincomments{g(S\&\&)}}ù).
}
\end{emcppslisting}

\noindent Use of \lstinline!std::forward! in combination with forwarding references
is typical in the implementation of industrial-strength generic
libraries; see \intrarefsimple{use-cases-forwardingref}.
%\textit{\titleref{use-cases-forwardingref}} on page~\pageref{use-cases-forwardingref}.

\subsubsection[A brief review of function template argument deduction]{A brief review of function template argument deduction}\label{a-brief-review-of-function-template-argument-deduction}

Invoking a function template without explicitly providing template
arguments at the call site will compel the compiler to attempt, if
possible, to \emph{deduce} those template \emph{type} arguments from the
function arguments:

\begin{emcppslisting}
template <typename T> void f();
template <typename T> void g(T x);
template <typename T> void h(T y, T z);

void example0()
{
    f();        // Error, couldn't infer template argument (ù{\codeincomments{T}}ù)
    g(0);       // OK, (ù{\codeincomments{T}}ù) deduced as (ù{\codeincomments{int}}ù) from literal (ù{\codeincomments{0}}ù) -- (ù{\codeincomments{x}}ù) is an (ù{\codeincomments{int}}ù).
    h(0, 'a');  // Error, deduced conflicting types for (ù{\codeincomments{T}}ù) ((ù{\codeincomments{int}}ù) vs. (ù{\codeincomments{char}}ù))
}
\end{emcppslisting}

\noindent Any \romeogloss{cv-qualifiers} (\lstinline!const!, \lstinline!volatile!, or both)
on a \emph{deduced} function parameter will be applied \emph{after} type
deduction is performed:

\begin{emcppslisting}
template <typename T> void cf(const T x);
template <typename T> void vf(volatile T y);
template <typename T> void wf(const volatile T z);

void example1()
{
    cf(0);  // OK, (ù{\codeincomments{T}}ù) deduced as (ù{\codeincomments{int}}ù) -- (ù{\codeincomments{x}}ù) is a (ù{\codeincomments{const int}}ù).
    vf(0);  // OK, (ù{\codeincomments{T}}ù) deduced as (ù{\codeincomments{int}}ù) -- (ù{\codeincomments{y}}ù) is a (ù{\codeincomments{volatile int}}ù).
    wf(0);  // OK, (ù{\codeincomments{T}}ù) deduced as (ù{\codeincomments{int}}ù) -- (ù{\codeincomments{z}}ù) is a (ù{\codeincomments{const}}ù) (ù{\codeincomments{volatile}}ù) (ù{\codeincomments{int}}ù).
}
\end{emcppslisting}

\noindent Similarly, \romeogloss{ref-qualifiers} other than \lstinline!&&!
(\lstinline!&! or \lstinline!&&! along with any cv-qualifier) do not
alter the deduction process, and they too are applied after deduction:

\begin{emcppslisting}
template <typename T> void rf(T& x);
template <typename T> void crf(const T& x);

void example2()
{
    int i;
    rf(i);   // OK, (ù{\codeincomments{T}}ù) is deduced as (ù{\codeincomments{int}}ù) -- (ù{\codeincomments{x}}ù) is an (ù{\codeincomments{int\&}}ù).
    crf(i);  // OK, (ù{\codeincomments{T}}ù) is deduced as (ù{\codeincomments{int}}ù) -- (ù{\codeincomments{x}}ù) is a (ù{\codeincomments{const int\&}}ù).

    rf(0);   // Error, expects an lvalue for 1st argument
    crf(0);  // OK, (ù{\codeincomments{T}}ù) is deduced as (ù{\codeincomments{int}}ù) -- (ù{\codeincomments{x}}ù) is a (ù{\codeincomments{const int\&}}ù).
}
\end{emcppslisting}

\noindent Type deduction works differently for \emph{forwarding} references
where the only qualifier on the template argument is \lstinline!&&!. For
the sake of exposition, consider a function template declaration,
\lstinline!f!, accepting a forwarding reference, \lstinline!forRef!:

\begin{emcppslisting}[emcppsbatch=e3]
template <typename T> void f(T&& forRef);
\end{emcppslisting}

 %%%%%%%% PRODUCTION: make sure we are as precise as we can be with this xref on FPPs.
\noindent We have seen in the example on page~\pageref{f-invoked-example} that,
when \lstinline!f! is invoked with an
\romeovalue{lvalue} of type \lstinline!S!, then \lstinline!T! is deduced as
\lstinline!S&! and \lstinline!forRef! becomes an \romeovalue{lvalue} reference.
When \lstinline!f! is instead invoked with an \romeovalue{xvalue} of type
\lstinline!S!, then \lstinline!T! is deduced as \lstinline!S! and
\lstinline!forRef! becomes an \romeovalue{rvalue} reference. The underlying
process that results in this ``duality'' relies on a special rule (known
as \romeogloss{reference collapsing}; see the next section) introduced as part of
\romeogloss{type deduction}. When the type \lstinline!T! of a
\emph{forwarding} reference is being deduced from an expression
\lstinline!E!, \lstinline!T! itself will be deduced as an \romeovalue{lvalue}
reference if \lstinline!E! is an \romeovalue{lvalue}; otherwise normal
type-deduction rules will apply and \lstinline!T! will be deduced as an \romeovalue{rvalue}
reference:

\begin{emcppslisting}[emcppsbatch=e3]
void g()
{
    int i;
    f(i);  // (ù{\codeincomments{i}}ù) is an *lvalue* expression.
           // (ù{\codeincomments{T}}ù) is therefore deduced as (ù{\codeincomments{int\&}}ù) -- special rule!
           // (ù{\codeincomments{T\&\&}}ù) becomes (ù{\codeincomments{int\& \&\&}}ù), which collapses to (ù{\codeincomments{int\&}}ù).

    f(0);  // (ù{\codeincomments{0}}ù) is an *rvalue* expression.
           // (ù{\codeincomments{T}}ù) is therefore deduced as (ù{\codeincomments{int}}ù).
           // (ù{\codeincomments{T\&\&}}ù) becomes (ù{\codeincomments{int\&\&}}ù), which is an *rvalue* reference.
}
\end{emcppslisting}

\noindent For more on general type deduction, see \featureref{\locationc}{auto-feature}.
%``\titleref{auto-feature}" on page~\pageref{auto-feature}.

\subsubsection[Reference collapsing]{Reference collapsing}\label{reference-collapsing}

As we saw in the previous section, when a function having a
\emph{forwarding} reference parameter, \lstinline!forRef!, is invoked with
a corresponding \romeovalue{lvalue} argument, an interesting phenomenon
occurs: After type deduction, we temporarily get what appears
syntactically to be an \romeovalue{rvalue} reference to an \romeovalue{lvalue}
reference. As references to references are \emph{not} allowed in C++,
the compiler employs \romeogloss{reference collapsing} to resolve the
\emph{forwarding}-reference parameter, \lstinline!forRef!, into a single
reference, thus providing a way to infer, from \lstinline!T! itself, the
original \romeogloss{value category} of the argument passed to \lstinline!f!.

The process of \romeogloss{reference collapsing} takes place automatically
in any situation where a reference to a reference is formed. Table~\ref{forwardingref-table1}
illustrates the simple rules for collapsing ``unstable" references
into ``stable" ones. Notice, in particular, that an \romeovalue{lvalue}
reference always overpowers an \romeovalue{rvalue} reference. The only
situation in which two references collapse into an \romeovalue{rvalue}
reference is when they are both \romeovalue{rvalue} references.
\begin{table}[h!]
\begin{center}
\begin{threeparttable}
\caption{Collapsing ``unstable'' reference pairs into a single
``stable'' one}\label{forwardingref-table1}\vspace{1.5ex}
{\small \begin{tabular}{c|c|c}\thickhline
\rowcolor[gray]{.9}   {\sffamily\bfseries 1st Reference Type} & {\sffamily\bfseries 2nd Reference Type} &
{\sffamily\bfseries Result of Reference Collapsing} \\ \hline
\lstinline!&! &\lstinline!&! & \lstinline!&! \\ \hline
\lstinline!&! & \lstinline!&&! &\lstinline!&! \\ \hline
\lstinline!&&! & \lstinline!&! & \lstinline!&! \\ \hline
\lstinline!&&! & \lstinline!&&! & \lstinline!&&! \\ \thickhline
\end{tabular}
}
\end{threeparttable}
\end{center}
\end{table}

Finally, note that it is not possible to write a reference-to-reference
type in C++\linebreak%%%%%
 \mbox{explicitly}:

\begin{emcppslisting}
int    i   = 0;   // OK
int&   ir  = i;   // OK
int& & irr = ir;  // Error, (ù{\codeincomments{irr}}ù) declared as a reference to a reference
\end{emcppslisting}

\noindent It is, however, easy to do so with type aliases and template parameters,
and that is where reference collapsing comes into play:

\begin{emcppslisting}[emcppsbatch=e4]
#include <type_traits>  // (ù{\codeincomments{std::is\_same)}ù)
using i = int&;  // OK
using j = i&;    // OK, (ù{\codeincomments{int\& \&}}ù) becomes (ù{\codeincomments{int\&}}ù).
static_assert(std::is_same<j,int&>::value);
\end{emcppslisting}

\noindent During computations involving \romeogloss{metafunctions}, or as part of
language rules (such as type deduction), however, references to
references can occur spontaneously:

\begin{emcppslisting}
template <typename T>
struct AddLvalueRef { typedef T& type; };
    // metafunction that transforms to an *lvalue* reference to (ù{\codeincomments{T}}ù)

template <typename T>
void f(T input)
{
    typename AddLvalueRef<T>::type ir1 = input;    // OK, adds (ù{\codeincomments{\&}}ù) to make (ù{\codeincomments{T\&}}ù)
    typename AddLvalueRef<T&>::type ir2 = input;   // OK, collapses to (ù{\codeincomments{T\&}}ù)
    typename AddLvalueRef<T&&>::type ir3 = input;  // OK, collapses to (ù{\codeincomments{T\&}}ù)
}
\end{emcppslisting}

\noindent Notice that we are using the \lstinline!typename! keyword in the example above as a
generalized way of indicating, during \romeogloss{template instantiation},
that a dependent name is a type (as opposed to a value).

\subsubsection[Identifying forwarding references]{Identifying forwarding references}\label{identifying-forwarding-references}

The syntax for a \emph{forwarding} reference (\lstinline!&&!) is the same
as that for \romeovalue{rvalue} references; the only way to discern one from
the other is by observing the surrounding context. When used in a manner
where \romeogloss{type deduction} can take place, the \lstinline!T&&! syntax
does \emph{not} designate an \romeovalue{rvalue} reference; instead, it
represents a \emph{forwarding} reference; for type deduction to be in
effect, an \emph{individual} (possibly member) function \emph{template}
must have a type parameter (e.g., \lstinline!T!) and a function parameter of
type that exactly matches that parameter followed by \lstinline!&&!
(e.g., \lstinline!T&&!):

\begin{emcppslisting}
struct S0
{
    template <typename T>
    void f(T&& forRef);
        // OK, fully eligible for template-argument type deduction: (ù{\codeincomments{forRef}}ù)
        // is a forwarding reference.
};
\end{emcppslisting}

\noindent Note that if the function parameter is qualified, the syntax reverts to
the usual meaning of \romeovalue{rvalue} reference:

\begin{emcppslisting}
struct S1
{
    template <typename T>
    void f(const T&& crRef);
        // Eligible for type deduction but is not a forwarding reference: due
        // to the (ù{\codeincomments{const}}ù) qualifier, (ù{\codeincomments{crRef}}ù) is an *rvalue* reference.
};
\end{emcppslisting}

\noindent If a member function of a class template is not itself also a template,
then its template type parameter will not be deduced:

\begin{emcppslisting}
template <typename T>
struct S2
{
    void f(T&& rRef);
       // Not eligible for type deduction because (ù{\codeincomments{T}}ù) is fixed and known as part
       // of the instantiation of (ù{\codeincomments{S2}}ù): (ù{\codeincomments{rRef}}ù) is an *rvalue* reference.
};
\end{emcppslisting}

\noindent More generally, note that the \lstinline!&&! syntax can \emph{never}
imply a \emph{forwarding} reference for a function that is not itself a
template; see \intraref{annoyances-forwardingref}{forwarding-references-look-just-like-rvalue-references}.
%\textit{\titleref{annoyances-forwardingref}:
%\titleref{forwarding-references-look-just-like-rvalue-references}} on page~\pageref{forwarding-references-look-just-like-rvalue-references}.

\subsubsection[\lstinline!auto&&! --- a forwarding reference in a non-parameter context]{{\SubsubsecCode auto\&\&}~{\subsubemdash}~a forwarding reference in a non-parameter context}\label{auto-a-forwarding-reference-in-a-non-parameter-context}

Outside of template function parameters, \emph{forwarding} references
can also appear in the context of variable definitions using the
\lstinline!auto! variable (see \featureref{\locationc}{auto-feature}) 
%``\titleref{auto-feature}" on page~\pageref{auto-feature}) 
because they too are subject to type
deduction:

\begin{emcppslisting}
void f()
{
    auto&& i = 0;  // (ù{\codeincomments{i}}ù) is a forwarding reference because the type of (ù{\codeincomments{i}}ù) must
                   // be deduced from the initialization expression (ù{\codeincomments{0}}ù).
}
\end{emcppslisting}

\noindent Just like function parameters, \lstinline!auto&&! resolves to either an
\romeovalue{lvalue} reference or \romeovalue{rvalue} reference depending on the
\romeogloss{value category} of the initialization expression:

\begin{emcppslisting}
void g()
{
    int i;
    auto&& lv = i;  // (ù{\codeincomments{lv}}ù) is an (ù{\codeincomments{int\&}}ù).

    auto&& rv = 0;  // (ù{\codeincomments{rv}}ù) is an (ù{\codeincomments{int\&\&}}ù).
}
\end{emcppslisting}

\noindent Similarly to \lstinline!const!~\lstinline!auto&!, the \lstinline!auto&&!
syntax binds to anything. In \lstinline!auto&&!, however, the
\lstinline!const!-ness of the bound object is \emph{preserved} rather than
always enforced:

\begin{emcppslisting}
void h()
{
    int        i = 0;
    const int ci = 0;

    auto&& lv  = i;   // (ù{\codeincomments{lv}}ù) is an (ù{\codeincomments{int\&}}ù).
    auto&& clv = ci;  // (ù{\codeincomments{clv}}ù) is a (ù{\codeincomments{const int\&}}ù).
}
\end{emcppslisting}

\noindent Just as with function parameters, the original \romeogloss{value category}
of the expression used to initialize a \emph{forwarding} reference
variable can be propagated during subsequent function invocation --
e.g., using \lstinline!std::forward! (see \intraref{description-forwardingref}{the-std::forward-utility}):
%\textit{\titleref{description-forwardingref}: \titleref{the-std::forward-utility}} on page~\pageref{the-std::forward-utility}):

\begin{emcppshiddenlisting}[emcppsbatch=e5]
#include <utility>  // (ù{\codeincomments{std::forward}}ù)
template <typename T>
T get();
\end{emcppshiddenlisting}
\begin{emcppslisting}[emcppsbatch=e5]
template <typename T>
void use(T&& t); // Here (ù{\codeincomments{use}}ù) also takes a forwarding reference parameter
                 // to do with as it pleases.

template <typename T>
void l()
{
    auto&& fr = get<T>();
        // (ù{\codeincomments{get<T>()}}ù) might be either an *lvalue* or *rvalue* depending on (ù{\codeincomments{T}}ù).

    use(std::forward<decltype(fr)>(fr));  // (ù{\codeincomments{decltype}}ù) is a C++11 feature.
        // Propagate the original value category of (ù{\codeincomments{get<T>()}}ù) into (ù{\codeincomments{use}}ù).
}
\end{emcppslisting}

\noindent Notice that, because (1) \lstinline!std::forward! (see the next section) requires the
type of the object that's going to be forwarded as a user-provided
template argument and (2) it is not possible to name the type of
\lstinline!fr!, \lstinline!decltype! (see \featureref{\locationa}{decltype}) 
%``\titleref{decltype}" on page~\pageref{decltype}) 
was used in the example above
to retrieve the type of \lstinline!fr!.

\subsubsection[Forwarding references without forwarding]{Forwarding references without forwarding}\label{forwarding-references-without-forwarding}

Sometimes deliberately \emph{not} forwarding (see \intraref{description-forwardingref}{the-std::forward-utility}) 
%\textit{\titleref{description-forwardingref}: \titleref{the-std::forward-utility}} on page~\pageref{the-std::forward-utility}) 
an \lstinline!auto&&! variable or
a forwarding reference function parameter at all can be useful, instead employing
\emph{forwarding} references solely for their \lstinline!const!-preserving
and universal binding semantics. As an example, consider the task of
obtaining iterators over a range of an unknown \romeogloss{value category}:

\begin{emcppshiddenlisting}[emcppsbatch=e6]
template <typename T>
T getRange();
\end{emcppshiddenlisting}
\begin{emcppslisting}[emcppsbatch=e6]
#include <iterator>  // (ù{\codeincomments{std::begin}}ù), (ù{\codeincomments{std::end}}ù),

template <typename T>
void m()
{
    auto&& r = getRange<T>();
        // (ù{\codeincomments{getRange<T>()}}ù) might be either an lvalue or rvalue depending on (ù{\codeincomments{T}}ù).

    auto b = std::begin(r);
    auto e = std::end(r);

    traverseRange(b, e);
}
\end{emcppslisting}

\noindent Using \lstinline!std::forward! in the initialization of both \lstinline!b! and
\lstinline!e! might result in moving from \lstinline!r! twice, which is
potentially unsafe (see \featureref{\locationc}{Rvalue-References}). 
%``\titleref{Rvalue-References}" on page~\pageref{Rvalue-References}). 
Forwarding
\lstinline!r! only in the initialization of \lstinline!e! might avoid issues
caused by moving an object twice but might result in inconsistent
behavior with \lstinline!b!, especially if the implementation of \lstinline!r!
makes use of reference qualifiers (see \featureref{\locatione}{refqualifiers}). 
%``\titleref{refqualifiers}" on page~\pageref{refqualifiers}).

\subsubsection[The \lstinline!std::forward! utility]{The {\SubsubsecCode std::forward} utility}\label{the-std::forward-utility}

The final piece of the forwarding reference puzzle is the
\lstinline!std::forward! utility function. Since the expression naming a
forwarding reference \lstinline!x! is always an \romeovalue{lvalue} --- due to its
reachability either by name or address in virtual memory --- and since our
intention is to move \lstinline!x! in case it was an \romeovalue{rvalue} to begin
with, we need a conditional \emph{move} operation that will move
\lstinline!x! only in that case and otherwise let \lstinline!x! pass through as
an \romeovalue{lvalue}.

The declaration for \lstinline!std::forward<T>! is as follows (in
\lstinline!<utility>!):

\begin{emcppshiddenlisting}[emcppsbatch=e7]
#include <type_traits>  // (ù{\codeincomments{std::remove\_reference}}ù)
using namespace std; // to avoid qualifying remove_reference
\end{emcppshiddenlisting}
\begin{emcppslisting}[emcppsbatch=e7]
template <class T> T&& forward(typename remove_reference<T>::type& t) noexcept;
template <class T> T&& forward(typename remove_reference<T>::type&& t) noexcept;
\end{emcppslisting}

\noindent The second overload is ill-formed if invoked when \lstinline!T! is an
\romeovalue{lvalue} reference type.

Remember that the type \lstinline!T! associated with a forwarding reference
is deduced as a reference type if given an \romeovalue{lvalue} reference and
as a non-reference type otherwise. So for a forwarding reference
\lstinline!forRef! of type \lstinline!T&&!, we have two cases:
\begin{itemize}
\item{An \romeovalue{lvalue} of type \lstinline!U! was used for initializing \lstinline!forRef!, so \lstinline!T! is \lstinline!U&!, thus the first overload of \lstinline!forward! will be selected and will be of the form \lstinline!U&!~\lstinline!forward(U&!~\lstinline!u)!~\lstinline!noexcept!, thus just returning the original \romeovalue{lvalue} reference.}
\item{An \romeovalue{rvalue} of type \lstinline!U! was used for initializing \lstinline!forRef!, so \lstinline!T! is \lstinline!U!, so the second overload of \lstinline!forward! will be selected and will be of the form \lstinline!U&&!~\lstinline!forward(U&!~\lstinline!u)!~\lstinline!noexcept!, essentially equivalent to \lstinline!std::move!.}
\end{itemize}
Note that, in the body of a function template accepting a forwarding
reference \lstinline!T&&! named \lstinline!x!, \lstinline!std::forward<T>(x)!
could be replaced with \lstinline!static_cast<T&&>(x)! to achieve the
same effect. Due to \romeogloss{reference collapsing} rules, \lstinline!T&&!
will resolve to \lstinline!T&! whenever the original \romeogloss{value
category} of \lstinline!x! was an \romeovalue{lvalue} and to \lstinline!T&&!
otherwise, thus achieving the \emph{conditional move} behavior
elucidated in \intrarefsimple{description-forwardingref}.
%\textit{\titleref{description-forwardingref}} on page~\pageref{description-forwardingref}. 
Using \lstinline!std::forward! over \lstinline!static_cast!
will, however, ensure that the types of \lstinline!T! and \lstinline!x! match,
preventing accidental unwanted conversions and, separately, perhaps also
more clearly expressing the programmer's intent.

\subsection[Use Cases]{Use Cases}\label{use-cases-forwardingref}

\subsubsection[Perfectly forwarding an expression to a downstream consumer]{Perfectly forwarding an expression to a downstream consumer}\label{perfectly-forwarding-an-expression-to-a-downstream-consumer}

A frequent use of forwarding references and \lstinline!std::forward! is to
propagate an object, whose \romeogloss{value category} is
invocation-dependent, down to one or more service providers that will
behave differently depending on the \romeogloss{value category} of the
original argument.

As an example, consider an overload set for a function, \lstinline!sink!,
that accepts a \lstinline!std::string! either by \lstinline!const!
\romeovalue{lvalue} reference (e.g., with the intention of \textit{copying} from it) or
\romeovalue{rvalue} reference (e.g., with the intention of \textit{moving} from it):

\begin{emcppshiddenlisting}[emcppsbatch=e8]
#include <string>   // (ù{\codeincomments{std::string}}ù)
#include <utility>  // (ù{\codeincomments{std::move}}ù)
std::string target;
\end{emcppshiddenlisting}
\begin{emcppslisting}[emcppsbatch=e8]
void sink(const std::string& s) { target = s; }
void sink(std::string&& s)      { target = std::move(s); }
\end{emcppslisting}

\noindent Now, let's assume that we want to create an intermediary function
template, \lstinline!pipe!, that will accept an \lstinline!std::string! of any
\romeogloss{value category} and will dispatch its argument to the
corresponding overload of \lstinline!sink!. By accepting a
\emph{forwarding} reference as a function parameter and invoking
\lstinline!std::forward! as part of \lstinline!pipe!'s body, we can achieve
our original goal without any code duplication:

\begin{emcppslisting}[emcppsbatch=e8]
template <typename T>
void pipe(T&& x)
{
    sink(std::forward<T>(x));
}
\end{emcppslisting}

\noindent Invoking \lstinline!pipe! with an \romeovalue{lvalue} will result in \lstinline!x!
being an \romeovalue{lvalue} reference and thus\linebreak[4]%%%%%%%%
\mbox{\lstinline!sink(const!~\lstinline!std::string&)!'s} being called. Otherwise,
\lstinline!x! will be an \romeovalue{rvalue} reference and
\lstinline!sink(std::string&&)! will be called. This idea of enabling
\emph{move} operations without code duplication (as \lstinline!pipe! does)
is commonly referred to as \intraref{use-cases-forwardingref}{perfect-forwarding-for-generic-factory-functions}.
%\textit{\titleref{use-cases-forwardingref}: \titleref{perfect-forwarding-for-generic-factory-functions}} on page~\pageref{perfect-forwarding-for-generic-factory-functions}.

\subsubsection[Handling multiple parameters concisely]{Handling multiple parameters concisely}\label{handling-multiple-parameters-concisely}

Suppose you have a \romeogloss{value-semantic type (VST)} that holds a
collection of attributes where some (not necessarily proper) subset of
them need to be changed together{\cprotect\footnote{This type of
value-semantic type can be classified more specifically as a
\emph{complex}-\emph{constrained} attribute class; see \cite{lakos2a}, section~4.2.}}:

\begin{emcppslisting}
#include <type_traits>  // (ù{\codeincomments{std::enable\_if}}ù)
#include <utility>      // (ù{\codeincomments{std::forward}}ù)

struct Person { /* UDT that benefits from move semantics */ };

class StudyGroup
{
    Person d_a;
    Person d_b;
    Person d_c;
    Person d_d;
    // ...

public:
    bool isValid(const Person& a, const Person& b, const Person& c, const Person& d);
        // Return (ù{\codeincomments{true}}ù) if these specific people form a valid study group under
        // the guidelines of the study-group commission, and false otherwise.
    // ...

    template <typename PA, typename PB, typename PC, typename PD,
        typename = typename std::enable_if<
            std::is_same<typename std::decay<PA>::type, Person>::value &&
            std::is_same<typename std::decay<PB>::type, Person>::value &&
            std::is_same<typename std::decay<PC>::type, Person>::value &&
            std::is_same<typename std::decay<PD>::type, Person>::value>::type>
    int setPersonsIfValid(PA&& a, PB&& b, PC&& c, PD&& d)
    {
        enum { e_SUCCESS = 0, e_FAIL };

        if (!isValid(a, b, c, d))
        {
            return e_FAIL;  // bad choice; no change
        }

        // Move or copy each person into this object's (ù{\codeincomments{Person}}ù) data members:
        d_a = std::forward<PA>(a);
        d_b = std::forward<PB>(b);
        d_c = std::forward<PC>(c);
        d_d = std::forward<PD>(d);

        return e_SUCCESS;  // Study group was updated successfully.
    }
};
\end{emcppslisting}

\noindent The \lstinline!setPersonsIfValid! function is producing the full crossproduct of instantiations for every variation of qualifiers that can be
on a \lstinline!Person! object. Any combination of \romeovalue{lvalue} and
\romeovalue{rvalue} \lstinline!Person!s can be passed, and a template will be
instantiated that will copy the \romeovalue{lvalues} and move from the
\romeovalue{rvalues}. To make sure the \lstinline!Person! objects are
created externally, the function is restricted, using
\lstinline!std::enable_if!, to instantiate only for types that decay to
\lstinline!Person! (i.e., types that are cv-qualified or ref-qualified
\lstinline!Person!). Because each parameter is a forwarding reference, they
can all implicitly convert to \lstinline!const!~\lstinline!Person&! to pass
to \lstinline!isValid!, creating no additional temporaries. Finally,
\lstinline!std::forward! is then used to do the actual moving or copying as
appropriate to data members.

\subsubsection[Perfect forwarding for generic factory functions]{Perfect forwarding for generic factory functions}\label{perfect-forwarding-for-generic-factory-functions}

Consider the prototypical standard-library generic factory function,
\lstinline!std::make_shared<T>!. On the surface, the requirements for this
function are fairly simple --- allocate a place for a \lstinline!T! and
then construct it with the same arguments that were passed to
\lstinline!make_shared!. This, however, gets reasonably complex to
implement efficiently when \lstinline!T! can have a wide variety of ways in
which it might be initialized.

For simplicity, we will show how a two-argument \lstinline!my::make_shared!
might be defined, knowing that a full implementation would employ
variadic template arguments for this purpose --- see \featureref{\locationc}{variadictemplate}. 
%``\titleref{variable-templates}" on page~\pageref{variable-templates}. 
We will also implement a simpler
version of \lstinline!make_shared! that simply creates the element on the
heap with \lstinline!new! and constructs a \lstinline!std::shared_ptr! to
manage the lifetime of that element. The declaration of this form of
make shared would be structured like this:

\begin{emcppshiddenlisting}[emcppsbatch=e9]
#include <memory>  // (ù{\codeincomments{std::shared\_ptr}}ù)
\end{emcppshiddenlisting}
\begin{emcppslisting}[emcppsbatch=e9]
namespace my {
template <typename ELEMENT_TYPE, typename ARG1, typename ARG2>
std::shared_ptr<ELEMENT_TYPE> make_shared(ARG1&& arg1, ARG2&& arg2);
}
\end{emcppslisting}

\noindent As you can see, we have two forwarding reference arguments ---
\lstinline!arg1! and \lstinline!arg2! --- with deduced types \lstinline!ARG1! and
\lstinline!ARG2!. Now, the body of our function needs to carefully
construct our \lstinline!ELEMENT_TYPE! object on the heap and then create
our output \lstinline!shared_ptr!:

\begin{emcppslisting}[emcppsbatch=e9]
template <typename ELEMENT_TYPE, typename ARG1, typename ARG2>
std::shared_ptr<ELEMENT_TYPE> my::make_shared(ARG1&& arg1, ARG2&& arg2)
{
    ELEMENT_TYPE *element_p = new ELEMENT_TYPE(std::forward<ARG1>(arg1),
                                               std::forward<ARG2>(arg2));
    try
    {
        return std::shared_ptr<ELEMENT_TYPE>(element_p);
    }
    catch (...)
    {
        delete element_p;
        throw;
    }
}
\end{emcppslisting}

\noindent Note that this simplified implementation needs to take care that the
constructor for the return value does not throw, cleaning up the
allocated element if that should happen; normally a \romeogloss{RAII}
proctor to manage this ownership would be a more robust solution to this
problem.

Importantly, the use of \lstinline!std::forward! to construct the element
means that the arguments passed to \lstinline!make_shared! will be used to
find the appropriate matching two-argument constructor of
\lstinline!ELEMENT_TYPE!. When those arguments are \romeovalue{rvalues}, the
constructor found will again search for one that takes an \romeovalue{rvalue}
and the arguments will be moved from. Even more, because this function
wants to forward exactly the \lstinline!const!-ness and reference type of
the input arguments, we would have to write 12 distinct overloads for
each argument if we were not using perfect forwarding -- the full cross
product of \lstinline!const! (or not), \lstinline!volatile! (or not), and
\lstinline!&!, \lstinline!&&!, (or not). This would mean a full
implementation of just this two-argument variation would require 144
distinct overloads, all almost identical and most never actually
instantiated. The use of forwarding references reduces that to just 1
overload for each number of arguments.

\subsubsection[Wrapping initialization in a generic factory function]{Wrapping initialization in a generic factory function}\label{wrapping-initialization-in-a-generic-factory-function}

Occasionally we might want to initialize an object with an intervening
function call wrapping the actual construction of that object. Suppose
we have a tracking system that we want to use to monitor how many times
certain initializers have been invoked:

\begin{emcppshiddenlisting}[emcppsbatch=e10]
#include <utility>      // (ù{\codeincomments{std::forward}}ù)
struct BigObject {
    BigObject(const char *, const char *);
};
\end{emcppshiddenlisting}
\begin{emcppslisting}[emcppsbatch=e10]
struct TrackingSystem
{
    template <typename T>
    static void trackInitialization(int numArgs);
        // Track the creation of a (ù{\codeincomments{T}}ù) with a constructor taking (ù{\codeincomments{numArgs}}ù)
        // arguments.
};
\end{emcppslisting}

\noindent Now we want to write a general utility function that can be used to
construct an arbitrary object and notify the tracking system of the
construction for us. Here we will use a variadic pack (see \featureref{\locationc}{variadictemplate}) 
%``\titleref{variable-templates}" on page~\pageref{variable-templates}) 
of forwarding references to handle
calling the constructor for us:

\begin{emcppslisting}[emcppsbatch=e10]
template <class ELEMENT_TYPE, typename... ARGS>
ELEMENT_TYPE trackConstruction(ARGS&&... args)
{
    TrackingSystem::trackInitialization<ELEMENT_TYPE>(sizeof...(args));
    return ELEMENT_TYPE(std::forward<ARGS>(args)...);
}
\end{emcppslisting}

\noindent This lets us add tracking easily to convert any initialization to a
tracked one by inserting a call to this function around the constructor
arguments:

\begin{emcppslisting}[emcppsbatch=e10]
void myFunction()
{
    BigObject untracked("Hello", "World");
    BigObject tracked = trackConstruction<BigObject>("Hello","World");
}
\end{emcppslisting}

\noindent On the surface there does seem to be a difference between how
\lstinline!untracked! and \lstinline!tracked! are initialized. The first
variable is having its constructor directly invoked, while the second is
being constructed from an object being returned by-value from
\lstinline!trackConstruction!. This construction, however, has long been
something that has been optimized away to avoid any additional objects
and construct the object in question just once. In this case, because
the element being returned is initialized by the \lstinline!return!
statement of \lstinline!trackConstruction!, the optimization is called
\romeogloss{return value optimization (RVO)}. C++ has always
allowed this optimization by enabling \romeogloss{copy elision}. In C++17,
this elision can even be guaranteed and is allowed to be done for
objects that have no copy or move constructors. Prior to C++17, this
elision can still be guaranteed (on all compilers that the authors are
aware of) by \romeogloss{declaring} but not \romeogloss{defining} the copy
constructor for \lstinline!BigObject!. You'll find that this code will
still compile and link with such an object, providing observable proof
that the copy constructor is never actually invoked with this pattern.

\subsubsection[Emplacement]{Emplacement}\label{emplacement}

Prior to C++11, inserting an object into a standard library container
always required the programmer to first create such an object and then
copy it inside the container's storage. As an example, consider
inserting a temporary \lstinline!std::string! object in a\linebreak[4]
\mbox{\lstinline!std::vector<std::string>!}:

\begin{emcppshiddenlisting}[emcppsbatch=e11]
#include <vector>  // (ù{\codeincomments{std::vector}}ù)
#include <string>  // (ù{\codeincomments{std::string}}ù)
\end{emcppshiddenlisting}
\begin{emcppslisting}[emcppsbatch=e11]
void f(std::vector<std::string>& v)
{
    v.push_back(std::string("hello world"));
        // invokes (ù{\codeincomments{std::string::string(const char*)}}ù) and the copy-constructor
}
\end{emcppslisting}

\noindent In the function above, a temporary \lstinline!std::string! object is
created in the stack frame of \lstinline!f! and is then copied to the
dynamically allocated buffer managed by \lstinline!v!. Additionally, the
buffer might have insufficient capacity and hence might require
reallocation, which would in turn require every element of \lstinline!v! to
be somehow copied from the old buffer to the new, larger one.

In C++11, the situation is significantly better thanks to
\romeovalue{rvalue} references. The temporary will be moved into
\lstinline!v!, and any buffer reallocation will \emph{move} the elements
between buffers rather than copy them, assuming that the element's
move-constructor is a noexcept specifier (see \featureref{\locatione}{noexcept-specifier}). 
%``\titleref{noexcept-specifier}" on page~\pageref{noexcept-specifier}). 
The amount of work
can, however, be further minimized: What if, instead of first creating
an object externally, we constructed the new \lstinline!std::string}! object
directly in \lstinline!v!'s buffer?

This is where \romeogloss{emplacement} comes into play. All standard library
containers, including \lstinline!std::vector!, now provide an
\romeogloss{emplacement} API powered by variadic templates (see \featureref{\locationc}{variadictemplate}) 
%``\titleref{variadictemplate}" on page~\pageref{variadictemplate})
and perfect forwarding (see \intraref{use-cases-forwardingref}{perfect-forwarding-for-generic-factory-functions}). 
%\textit{\titleref{use-cases-forwardingref}: \titleref{perfect-forwarding-for-generic-factory-functions}} on page~\pageref{perfect-forwarding-for-generic-factory-functions}). 
Rather than accepting a
fully-constructed element, \romeogloss{emplacement} operations accept an
arbitrary number of arguments, which will in turn be used to construct a
new element directly in the container's storage, thereby avoiding
unnecessary copies or even moves:

\begin{emcppslisting}[emcppsbatch=e11]
void g(std::vector<std::string>& v)
{
    v.emplace_back("hello world");
        // invokes only the (ù{\codeincomments{std::string::string(const char*)}}ù) constructor
}
\end{emcppslisting}

\noindent Calling \lstinline!std::vector<std::string>::emplace_back! with a
\lstinline!const!~\lstinline!char*! argument results in a new
\lstinline!std::string! object being created in-place in the next suitable
spot of the vector's storage. Internally,
\lstinline!std::allocator_traits::construct! is invoked, which typically
employs \romeogloss{placement \lstinline!new!} to construct the object in raw
dynamically allocated memory. As previously mentioned,
\lstinline!emplace_back! makes use of both variadic
templates and forwarding references; it accepts any
number of forwarding references and internally \textit{perfectly forwards} them
to the constructor of \lstinline!T! via \lstinline!std::forward!:

\begin{emcppshiddenlisting}[emcppsbatch=e12]
#include <new>     // placement (ù{\codeincomments{new}}ù)
namespace std {
template <typename T>
class vector {
    template <typename... Args>
    void emplace_back(Args&&... args);
};
char freeLocationInBuffer[10];
}
\end{emcppshiddenlisting}
\begin{emcppslisting}[emcppsbatch=e12]
template <typename T>
template <typename... Args>
void std::vector<T>::emplace_back(Args&&... args)
{
    // ...
    new (&freeLocationInBuffer) T(std::forward<Args>(args)...);  // pseudocode
    // ...
}
\end{emcppslisting}

\noindent \romeogloss{Emplacement} operations remove the need for copy or move
operations when inserting elements into containers, potentially
increasing the performance of a program and sometimes --- depending on the
container --- even allowing even noncopyable or nonmovable objects to be stored
in a container.

As previously mentioned,
declaring (but not defining) the \emph{copy} or \emph{move} ctor of a
noncopyable or nonmovable type to be private is often a way to guarantee that
a C++11/14 compiler constructed an object in place.
Containers that might need to move elements around for other operations
(such as \lstinline!std::vector! or \lstinline!std::deque!) will still need
movable elements, while node-based containers that never move the
elements themselves after initial construction (such as
\lstinline!std::list! or \lstinline!std::map!) can use emplace along with
noncopyable or nonmovable objects.

\subsubsection[Decomposing complex expressions]{Decomposing complex expressions}\label{decomposing-complex-expressions}

Many modern C++ libraries have adopted a more ``functional'' style of
programming, chaining the output of one function as the arguments of
another function to produce very complex expressions that accomplish a
great deal in relatively concise fashion. Consider the way in which the
C++20 ranges library encapsulates containers and arbitrary pairs of
iterators into objects that can be adapted and manipulated through long
chains of functions. Let's say you have a function that reads a file,
does some spellchecking for every unique word in the file, and gives you a
list of incorrect words and corresponding suggested proper spellings,
and you have a range-like library with common utilities similar to
standard UNIX processing utilities:

\begin{emcppshiddenlisting}[emcppsbatch={e13,e14},emcppsstandards=c++14]
#include <map>     // (ù{\codeincomments{std::map}}ù)
#include <string>  // (ù{\codeincomments{std::string}}ù)
#include <tuple>   // (ù{\codeincomments{std::tuple}}ù)

class SpellingSuggestion {
    bool isCorrect() const;
};
class LineSeq {};
LineSeq openFile(const std::string&);
LineSeq splitRegex(LineSeq seq, const std::string &);
LineSeq filterRegex(LineSeq seq, const std::string &);
LineSeq sort(LineSeq seq);
LineSeq uniq(LineSeq seq);
template <typename F>
LineSeq transform(LineSeq seq, F&& func) { return {}; }
template <typename F>
LineSeq filter(LineSeq seq, F&& pred) { return {}; }
std::map<std::string,SpellingSuggestion> makeMap(LineSeq seq);
SpellingSuggestion checkSpelling(const std::string& word);
\end{emcppshiddenlisting}
\begin{emcppslisting}[emcppsbatch=e13]
SpellingSuggestion checkSpelling(const std::string& word);

std::map<std::string, SpellingSuggestion> checkFileSpelling(
                                                   const std::string& filename)
{
    return makeMap(
        filter(transform(
           uniq(sort(filterRegex(splitRegex(openFile(filename),"\\S+"),"\\w+"))),
        [](const std::string& x)
        {
            return std::tuple<std::string, SpellingSuggestion>(x,
                                                             checkSpelling(x));
        }
   ), [](auto&& x) { return !std::get<1>(x).isCorrect(); }));
}
\end{emcppslisting}

\noindent Upon doing code review for this amazing use of a modern library produced
by the smart, new programmer on your team, you discover that you actually
have a very hard time understanding what is going on. On top of that,
the usual tools you have to poke and prod at the code by adding
\lstinline!printf! statements or even breakpoints in your debugger are very
hard to apply to the complex set of nested templates involved.

Each of the functions in this range library -- \lstinline!makeMap!,
\lstinline!transform!, \lstinline!uniq!, \lstinline!sort!, \lstinline!filterRegex!,
\lstinline!splitRegex!, and \lstinline!openFile! -- is a set of complex
templated overloads and deeply subtle metaprogramming that becomes hard
to unravel for a nonexpert C++ programmer. On the other hand, you have
also looked at the code generated for this function and the abstractions
amazingly get compiled away to a very robust implementation.

To better understand, document, and debug what is happening
here, you want to decompose this expression into many, capturing the
implicit temporaries returned by all of these functions and ideally not
changing the actual semantics of what is being done. To do that
properly, you need to capture the type and value category of each
subexpression appropriately, without necessarily being able to easily
decode it manually from the expression. Here is where \lstinline!auto&&!
forwarding references can be used effectively to decompose and document
this expression while achieving the same:

\begin{emcppslisting}[emcppsbatch=e14]
std::map<std::string, SpellingSuggestion> checkFileSpelling(
                                                   const std::string& filename)
{
    // Create a range over the contents of (ù{\codeincomments{filename}}ù).
    auto&& openedFile = openFile(filename);

    // Split the file by whitespace.
    auto&& potentialWords = splitRegex(
        std::forward<decltype(openedFile)>(openedFile), "\\S+");

    // Filter out only words made from word-characters.
    auto&& words = filterRegex(
        std::forward<decltype(potentialWords)>(potentialWords), "\\w+");

    // Sort all words.
    auto&& sortedWords = sort(std::forward<decltype(words)>(words));

    // Skip adjacent identical words. (This is now a sequence of unique words.)
    auto&& uniqueWords = uniq(std::forward<decltype(sortedWords)>(sortedWords));

    // Get a (ù{\codeincomments{SpellingSuggestion}}ù) for every word.
    auto&& suggestions = transform(
        std::forward<decltype(uniqueWords)>(uniqueWords),
        [](const std::string&x) {
            return std::tuple<std::string,SpellingSuggestion>(
                x,checkSpelling(x));
        });

    // Filter out correctly spelled words, keeping only elements where the
    // second element of the tuple, which is a (ù{\codeincomments{SpellingSuggestion}}ù), is not
    // (ù{\codeincomments{correct}}ù).
    auto&& corrections = filter(
        std::forward<decltype(suggestions)>(suggestions),
        [](auto&& suggestion){ return !std::get<1>(suggestion).isCorrect(); });

    // Return a map made from these 2-element tuples:
    return makeMap(std::forward<decltype(corrections)>(corrections));
}
\end{emcppslisting}

\noindent Now each step of this complex expression is documented, each temporary
has a name, but the net result of the lifetimes of each object is
functionally the same. No new conversions have been introduced, and
every object that was used as an \romeovalue{rvalue} in the original
expression will still be used as an \romeovalue{rvalue} in this much longer
and more descriptive implementation of the same functionality.

\subsection[Potential Pitfalls]{Potential Pitfalls}\label{potential-pitfalls}

\subsubsection[Surprising number of template instantiations with string literals]{Surprising number of template instantiations with string literals}\label{surprising-number-of-template-instantiations-with-string-literals}

When forwarding references are used as a means to avoid code repetition
between exactly two overloads of the same function (one accepting a
\lstinline!const!~\lstinline!T&! and the other a \lstinline!T&&!), it can be
surprising to see more than two template instantiations for that
particular template function, in particular when the function is invoked
using string literals.

Consider, as an example, a \lstinline!Dictionary! class containing two
overloads of an \lstinline!addWord! member function:
\begin{emcppshiddenlisting}[emcppsbatch={e15,e16}]
#include <string>  // (ù{\codeincomments{std::string}}ù)
\end{emcppshiddenlisting}
\begin{emcppslisting}[emcppsbatch=e15]
class Dictionary
{
    // ...

public:
    void addWord(const std::string& word);  // (0) copy (ù{\codeincomments{word}}ù) in the dictionary
    void addWord(std::string&& word);       // (1) move (ù{\codeincomments{word}}ù) in the dictionary
};

void f()
{
    Dictionary d;

    std::string s = "car";
    d.addWord(s);                    // invokes (0)

    const std::string cs = "toy";
    d.addWord(cs);                   // invokes (0)

    d.addWord("house");              // invokes (1)
    d.addWord("garage");             // invokes (1)
    d.addWord(std::string{"ball"});  // invokes (1)
}
\end{emcppslisting}

\noindent Now, imagine replacing the two overloads of \lstinline!addWord! with a
single \textit{perfectly forwarding} template member function, with the intention
of avoiding code repetition between the two overloads:

\begin{emcppslisting}[emcppsbatch=e16]
class Dictionary
{
    // ...

public:
    template <typename T>
    void addWord(T&& word);
};
\end{emcppslisting}

\noindent Perhaps surprisingly, the number of template instantiations skyrockets:

\begin{emcppslisting}[emcppsbatch=e16]
void f()
{
    Dictionary d;

    std::string s = "car";
    d.addWord(s);  // instantiates (ù{\codeincomments{addWord<std::string\&>}}ù)

    const std::string cs = "toy";
    d.addWord(cs);  // instantiates (ù{\codeincomments{addWord<const std::string\&>}}ù)

    d.addWord("house");              // instantiates (ù{\codeincomments{addWord<char const(\&)[6]>}}ù)
    d.addWord("garage");             // instantiates (ù{\codeincomments{addWord<char const(\&)[7]>}}ù)
    d.addWord(std::string{"ball"});  // instantiates (ù{\codeincomments{addWord<std::string\&\&>}}ù)
}
\end{emcppslisting}

\noindent Depending on the variety of argument types supplied to \lstinline!addWord!,
having many call sites could result in an undesirably large number of
distinct template instantiations, perhaps significantly increasing
object code size, compilation time, or both.

\subsubsection[\lstinline!std::forward<T>! can enable move operations]{{\SubsubsecCode std::forward<T>} can enable move operations}\label{std::forward<t>-can-enable-move-operations}

Invoking \lstinline!std::forward<T>(x)! is equivalent to conditionally
invoking \lstinline!std::move! (if \lstinline!T! is an
\romeovalue{lvalue} reference). Hence, any subsequent use of \lstinline!x! is
subject to the same caveats that would apply to an \romeovalue{lvalue} cast to
an unnamed \romeovalue{rvalue} reference; see \featureref{\locationc}{Rvalue-References}: 
%``\titleref{Rvalue-References}" on page~\pageref{Rvalue-References}:

\begin{emcppshiddenlisting}[emcppsbatch=e17]
#include <utility>      // (ù{\codeincomments{std::forward}}ù)
template <typename T>
void g(T&& x);
\end{emcppshiddenlisting}
\begin{emcppslisting}[emcppsbatch=e17]
template <typename T>
void f(T&& x)
{
    g(std::forward<T>(x));  // OK
    g(x);                   // Oops! (ù{\codeincomments{x}}ù) could have already been moved from.
}
\end{emcppslisting}

\noindent Once an object has been passed as an argument using
\lstinline!std::forward!, it should typically not be accessed again
without first assigning it a new value because it could now be in a
moved-from state.

\subsubsection[A perfect-forwarding constructor can hijack the copy constructor]{A perfect-forwarding constructor can hijack the copy constructor}\label{a-perfect-forwarding-constructor-can-hijack-the-copy-constructor}

A single-parameter constructor of a class \lstinline!S! accepting a
forwarding reference can unexpectedly be a better match during overload
resolution compared to \lstinline!S!'s copy constructor:

\begin{emcppslisting}
struct S
{
    S();                           // default constructor
    template <typename T> S(T&&);  // forwarding constructor
    S(const S&);                   // copy constructor
};

void f()
{
    S a;
    const S b;

    S x(a);  // invokes forwarding constructor
    S y(b);  // invokes copy constructor
}
\end{emcppslisting}

\noindent Despite the programmer's intention to copy from \lstinline!a! into
\lstinline!x!, the forwarding constructor of \lstinline!S! was invoked
instead, because \lstinline!a! is a non-\lstinline!const! \romeovalue{lvalue}
expression, and instantiating the forwarding constructor with
\lstinline!T!~\lstinline!=!~\lstinline!S&! results in a better match than even
the copy constructor.

This potential pitfall can arise in practice, for example, when writing
a value-semantic wrapper template (e.g., \lstinline!Wrapper!) that can be
initialized by \textit{perfectly forwarding} the object to be wrapped into it:

\begin{emcppshiddenlisting}[emcppsbatch=e18]
#include <string>   // (ù{\codeincomments{std::string}}ù)
#include <utility>  // (ù{\codeincomments{std::forward}}ù)
\end{emcppshiddenlisting}
\begin{emcppslisting}[emcppsbatch=e18]
template <typename T>
class Wrapper  // wrapper for an object of arbitrary type 'T'
{
private:
    T d_datum;

public:
    template <typename U>
    Wrapper(U&& datum) : d_datum(std::forward<U>(datum)) { }
        // perfect-forwarding constructor (to optimize runtime performance)

    // ...
};

void f()
{
    std::string s("hello world");
    Wrapper<std::string> w0(s);  // OK, (ù{\codeincomments{s}}ù) is copied into (ù{\codeincomments{d\_datum}}ù).

    Wrapper<std::string> w1(std::string("hello world"));
        // OK, the temporary string is moved into (ù{\codeincomments{d\_datum}}ù).
}
\end{emcppslisting}

\noindent Similarly to the example involving class \lstinline!S! in the example above, attempting
to copy-construct a non-\lstinline!const! instance of \lstinline!Wrapper!
(e.g., \lstinline!wr!, above) results in an error:

\begin{emcppslisting}[emcppsbatch=e18]
void g(Wrapper<int>& wr)  // The same would happen if (ù{\codeincomments{wr}}ù) were passed by value.
{
    Wrapper<int> w2(10);  // OK, invokes perfect-forwarding constructor
    Wrapper<int> w3(wr);  // Error, no conversion from (ù{\codeincomments{Wrapper<int>}}ù) to (ù{\codeincomments{int}}ù)
}
\end{emcppslisting}

\noindent The compilation failure above occurs because the perfect-forwarding
constructor template, instantiated with \lstinline!Wrapper<int>&!, is a
better match than the implicitly generated copy constructor, which
accepts a \lstinline!const!~\lstinline!Wrapper<int>&!. Constraining the
perfect forwarding constructor via \romeogloss{SFINAE} (e.g., with
\lstinline!std::enable_if!) to explicitly \emph{not} accept objects whose
type is \lstinline!Wrapper! fixes this problem:

\begin{emcppshiddenlisting}[emcppsbatch=e19]
#include <type_traits>  // (ù{\codeincomments{std::enable\_if}}ù), (ù{\codeincomments{std::decay}}ù)
#include <utility>  // (ù{\codeincomments{std::forward}}ù)
\end{emcppshiddenlisting}
\begin{emcppslisting}[emcppsbatch=e19]
template <typename T>
class Wrapper
{
private:
    T d_datum;

public:
    template <typename U,
        typename = typename std::enable_if<
            !std::is_same<typename std::decay<U>::type, Wrapper>::value
        >::type
    >
    Wrapper(U&& datum) : d_datum(std::forward<U>(datum)) { }
        // This constructor participates in overload resolution only if (ù{\codeincomments{U}}ù),
        // after being decayed, is not the same as (ù{\codeincomments{Wrapper}}ù).
};

void h(Wrapper<int>& wr)  // The same would happen if (ù{\codeincomments{wr}}ù) were passed by value.
{
    Wrapper<int> w4(10);  // OK, invokes the perfect-forwarding constructor
    Wrapper<int> w5(wr);  // OK, invokes the copy constructor
}
\end{emcppslisting}

\noindent Notice that the \lstinline!std::decay! \romeogloss{metafunction} was used as
part of the constraint; for more information on the using
\lstinline!std::decay!, see \intraref{annoyances-forwardingref}{metafunctions-are-required-in-constraints}. 
%\textit{\titleref{annoyances-forwardingref}: \titleref{metafunctions-are-required-in-constraints}} on page~\pageref{metafunctions-are-required-in-constraints}.

\subsection[Annoyances]{Annoyances}\label{annoyances-forwardingref}

\subsubsection[Forwarding references look just like rvalue references]{Forwarding references look just like rvalue references}\label{forwarding-references-look-just-like-rvalue-references}

Despite \emph{forwarding} references and \romeovalue{rvalue} 
references having significantly different semantics, as discussed in \intraref{description-forwardingref}{identifying-forwarding-references}, 
%\textit{\titleref{description-forwardingref}: \titleref{identifying-forwarding-references}} on page~\pageref{identifying-forwarding-references}, 
they share the same syntax. For any given type \lstinline!T!,
whether the \lstinline!T&&! syntax designates an \romeovalue{rvalue} 
reference or a \emph{forwarding} reference depends entirely on the
surrounding context.{\cprotect\footnote{In C++20, developers might be
subject to additional confusion due to the new terse concept notation
syntax, which allows function templates to be defined without any
explicit appearance of the \lstinline!template! keyword. As an example, a
constrained function parameter, like
\lstinline!Addable!~\lstinline!auto&&!~\lstinline!a! in the example below, is a forwarding
reference; looking for the presence of the mandatory \lstinline!auto!
keyword is helpful in identifying whether a type is a forwarding
reference or \romeovalue{rvalue} reference:

\begin{emcppslisting}[emcppsstandards=c++20,style=footcode]
template<typename T>
concept Addable = requires(T a, T b) { a + b; };

void f(Addable auto&& a);  // C++20 terse concept notation

void example()
{
    int i;

    f(i);  // OK, (ù{\fncodeincomments{decltype(a)}}ù) is (ù{\fncodeincomments{int\&}}ù) in (ù{\fncodeincomments{f}}ù).
    f(0);  // OK, (ù{\fncodeincomments{decltype(a)}}ù) is (ù{\fncodeincomments{int\&\&}}ù) in (ù{\fncodeincomments{f}}ù).
}
\end{emcppslisting}
      }}

\begin{emcppslisting}
template <typename T> struct S0 { void f(T&&); };  // rvalue reference
struct S1 { template <typename T> void f(T&&); };  // forwarding reference
\end{emcppslisting}

\noindent Furthermore, even if \lstinline!T! is subject to template argument
deduction, the presence of \emph{any} qualifier will suppress the
special \emph{forwarding}-reference deduction rules:

\begin{emcppslisting}
template <typename T> void f(T&&);           // forwarding reference
template <typename T> void g(const T&&);     // (ù{\codeincomments{const}}ù) rvalue reference
template <typename T> void h(volatile T&&);  // (ù{\codeincomments{volatile}}ù) rvalue reference
\end{emcppslisting}

\noindent It is truly remarkable that we still do not have some unique syntax
(e.g., \lstinline!&&&!) that we could use, at least optionally, to imply
unequivocally a \emph{forwarding} reference that is independent of its
context.

\subsubsection[Metafunctions are required in constraints]{Metafunctions are required in constraints}\label{metafunctions-are-required-in-constraints}

As we showed in \intrarefsimple{use-cases-forwardingref},
%\textit{\titleref{use-cases-forwardingref}} on page~\pageref{use-cases-forwardingref}, 
being able to
perfectly forward arguments of the same general type and effectively
leave only the value category of the argument up to type deduction is a frequent need. This
is necessary if you do not want to delay construction of the arguments
until they are forwarded, possibly because doing so would produce many
unnecessary temporaries.

The challenge to make this work correctly is significant. The template
must be constrained using \romeogloss{SFINAE} and the appropriate
\romeogloss{type traits} to disallow types that aren't some form of
cv-qualified or ref-qualified version of the type that you want to
accept. As an example, let's consider a function intended to \emph{copy}
or \emph{move} a \lstinline!Person! object into a data structure:

\begin{emcppshiddenlisting}[emcppsbatch=e20]
#include <type_traits>  // (ù{\codeincomments{std::enable\_if}}ù), (ù{\codeincomments{std::decay}}ù), (ù{\codeincomments{std::is\_same}}ù)
class Person {};
\end{emcppshiddenlisting}
\begin{emcppslisting}[emcppsbatch=e20]
class PersonManager {
// ...
template <typename T, typename = typename std::enable_if<
       std::is_same<typename std::decay<T>::type, Person>::value>::type>
void addPerson(T&& person) {}
    // This function participates in overload resolution only if (ù{\codeincomments{T}}ù) is
    // (possibly cv- or ref-qualified) (ù{\codeincomments{Person}}ù).
// ...
};
\end{emcppslisting}

\noindent This incantation to constrain \lstinline!T! has a number of layers to it,
so let's unpack them one at a time.
\begin{itemize}
\item{\lstinline!T! is the template argument we are trying to deduce. We’d like to limit it to being a \lstinline!Person! that is \lstinline!const!, \lstinline!volatile!, \lstinline!&!, \lstinline!&&!, or some (possibly empty) valid combination of those.}
\item{\lstinline!std::decay<T>::type! is then the application of the standard metafunction (defined in \lstinline!<type_traits>!) \lstinline!std::decay! to \lstinline!T!. This metafunction removes all cv-qualifiers and ref-qualifiers from \lstinline!T!, and so, for the types to which we want to limit \lstinline!T!, this will \emph{always} be \lstinline!Person!. Note that \lstinline!decay! will also allow some other implicitly convertible transformations, such as converting an array type to the corresponding pointer type. For types we are concerned with --- those that decay to a \lstinline!Person! --- this metafunction is equivalent to \lstinline!std::remove_cv<std::remove_reference<T>::type>::type!, or the equivalent and shorter \lstinline!std::remove_cvref<T>::type>! available in C++20. Due to historical availability and readability, we will continue with our use of \lstinline!decay! for this purpose.}
\item{\lstinline!std::is_same<std::decay<T>::type,!~\lstinline!Person>::value! is then the application of another metafunction, \lstinline!std::is_same!, to two arguments --- our decay expression and \lstinline!Person!, which results in a \lstinline!value! that is either \lstinline!std::true_type! or \lstinline!std::false_type! --- special types that can convert, in compile time, expressions to \lstinline!true! or \lstinline!false!. For the types \lstinline!T! that we care about, this expression will be \lstinline!true!, and for all other types this expression will be \lstinline!false!.}
\item{\lstinline!std::enable_if<X>::type! is yet another metafunction that evaluates to a valid type if and only if \lstinline!X! is true. Unlike the \lstinline!value! in \lstinline!std::is_same!, this expression is simply not valid if \lstinline!X! is false.}
\item{Finally, by using this \lstinline!enable_if! expression as a default-initialized template argument, the expression is going to be instantiated for any deduced \lstinline!T! considered during overload resolution for \lstinline!addPerson!. This instantiation will fail for any of the types we don’t want to allow (something that is not a cv). Because of this, for any \lstinline!T! that isn’t one of the types for which we want to allow \lstinline!addPerson! to be invoked, this substitution will fail. Rather than being an error, this just removes \lstinline!addPerson! from the overload set being considered, hence the term \romeogloss{SFINAE}. In this case, that would give us a different error indicating that we attempted to pass a non-\lstinline!Person! to \lstinline!addPerson!, which is exactly the result we want.}
\end{itemize}
Putting this all together means we get to call \lstinline!addPerson! with
\romeovalue{lvalues} and \romeovalue{rvalues} of type \lstinline!Person!, and the value
category will be appropriately usable within \lstinline!addPerson!
(generally with use of \lstinline!std::forward! within that function's
definition).

\subsection[See Also]{See Also}\label{see-also}

\begin{itemize}
\item{\seealsoref{Rvalue-References}{\seealsolocationc}Feature that can be confused with forwarding references due to similar syntax.}
\item{`\seealsoref{auto-feature}{\seealsolocationc}Feature that can introduce a forwarding reference with the \lstinline!auto&&! syntax.}
\item{\seealsoref{variadictemplate}{\seealsolocationc}Feature commonly used in conjunction with forwarding references to provide highly generic interfaces.}
\end{itemize}

\subsection[Further Reading]{Further Reading}\label{further-reading}

\begin{itemize}
\item{``Item 24: Distinguish universal references from rvalue references," \cite{meyers15}}
\item{\cite{sutter14}}
\item{\cite{niebler13}}
\end{itemize}


