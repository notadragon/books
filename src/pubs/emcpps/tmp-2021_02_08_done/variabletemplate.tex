% 8 Feb 2021 JMB, file comment and includes cleanup

\emcppsFeature{
    short={Variable Templates},
    long={Templated Variable Declarations/Definitions},
}{variable-templates}
\setcounter{table}{0}
\setcounter{footnote}{0}
\setcounter{lstlisting}{0}
%\section[Variable Templates]{Templated Variable Declarations/Definitions}\label{variable-templates}


Variable templates extend traditional template syntax to define, in namespace or
class (but not function) scope, a family of like-named variables that
can subsequently be instantiated explicitly.

\subsection[Description]{Description}\label{variabletemplate-description}

%By beginning a variable declaration with the familiar
%\textbf{template-head} syntax --- e.g.,
%\texttt{template}~\texttt{<typename}~\texttt{T>} --- we can create a
%\emph{variable template}, which defines a family of variables having the
%same name (e.g., \texttt{typeid}):
%
%\begin{lstlisting}[language=C++]
%template <typename> int typeId;  // template variable defined at file scope
%\end{lstlisting}
%
%\noindent Like any other kind of template, a variable template can be instantiated
%(explicitly) by providing an appropriate number (one or more) of type or
%non-type arguments:
%
%\begin{lstlisting}[language=C++]
%void f1()
%{
%    typeId<bool> = -1;    // (ù{\codeincomments{typeId<bool>}}ù) is an (ù{\codeincomments{int}}ù)
%    typeId<char> = 1000;  // (ù{\codeincomments{typeId<char>}}ù) is an (ù{\codeincomments{int}}ù)
%    typeId<void> = -666;  // (ù{\codeincomments{typeId<void>}}ù) is an (ù{\codeincomments{int}}ù)
%
%    assert(typeId<bool> ==   -1);
%    assert(typeId<char> == 1000);
%    assert(typeId<void> == -666);
%}
%\end{lstlisting}
%
%\noindent In the example above, the type of each instantiated variable --- i.e.,
%\texttt{typeId<bool>} and \texttt{typeId<char>} --- is \texttt{int}.
%Such need not be the case{\cprotect\footnote{Use of
%  \texttt{constexpr} variables would allow the instantiated
%  variables to be usable as a constant in a compile-time context (see
%  {\it\titleref{variabletemplate-use-cases}:} {\it\titleref{parametrized-constants}} on page~\pageref{parametrized-constants}).}}:
%
%\begin{lstlisting}[language=C++]
%template <typename T> const T pi(3.1415926535897932385);  // distinct types
%\end{lstlisting}
%
%\noindent In the example above, the type of the instantiated non-\texttt{const}
%variable is that of its (type) argument, and its (mutable) value is
%initialized to the best approximation of \(\pi\) offered by that type:
%
%\begin{lstlisting}[language=C++]
%void f2()
%{
%    bool        pi_as_bool        = 1;                      // ( 1 bit)
%    int         pi_as_int         = 3;                      // (32 bits)
%    float       pi_as_float       = 3.1415927;              // (32 bits)
%    double      pi_as_double      = 3.141592653589793;      // (64 bits)
%    long double pi_as_long_double = 3.1415926535897932385;  // (80 bits)
%
%    assert(pi<bool>        == pi_as_bool);
%    assert(pi<int>         == pi_as_int);
%    assert(pi<float>       == pi_as_float);
%    assert(pi<double>      == pi_as_double);
%    assert(pi<long double> == pi_as_long_double);
%}
%\end{lstlisting}
%
%\noindent For examples involving immutable variable templates, see {\it\titleref{variabletemplate-use-cases}:} {\it\titleref{parametrized-constants}} on page~\pageref{parametrized-constants}.

%%%%%%%%%%%%%%%% new material from Slava
By beginning a variable declaration with the familiar \romeogloss{template-head} syntax --- e.g., \lstinline!template! \lstinline!<typename! \lstinline!T>! --- we can create a \emph{variable template}, which defines a family of variables
having the same name (e.g., \lstinline!exampleOf!):

\begin{lstlisting}[language=C++]
template <typename T> T example;  // variable template defined at file scope
\end{lstlisting}

\noindent Like any other kind of template, a variable template can be instantiated explicitly by
providing an appropriate number of type or non-type arguments:

\begin{lstlisting}[language=C++]
#include <iostream>  // (ù{\codeincomments{std::cout}}ù)

void initializeExampleValues()
{
    exampleOf<int>   = -1;
    exampleOf<char>  = 'a';
    exampleOf<float> = 12.3f;
}

void printExampleValues()
{
    initializeExampleValues();
    std::cout << "int = "   << exampleOf<int>   << "; "
              << "char = "  << exampleOf<char>  << "; "
              << "float = " << exampleOf<float> << ';';

    // outputs "int = -1; char = a; float = 12.3;"
}
\end{lstlisting}

\noindent In the example above, the type of each instantiated variable is the same as its template parameter, but this matching is not required. For example, the type might be the same for all instantiated variables or derived from its parameters, such as by adding \lstinline!const! qualification:

\begin{lstlisting}[language=C++]
#include <type_traits>  // (ù{\codeincomments{std::is\_floating\_point}}ù)
#include <cassert>      // standard C (ù{\codeincomments{assert}}ù) macro

template <typename T>
const bool sane_for_pi = std::is_floating_point<T>::value;  // same type

template <typename T> const T pi(3.1415926535897932385);    // distinct types

void testPi()
{
    assert(!sane_for_pi<bool>);
    assert(!sane_for_pi<int>);

    assert( sane_for_pi<float>);
    assert( sane_for_pi<double>);
    assert( sane_for_pi<long double>);

    const float       pi_as_float       = 3.1415927;
    const double      pi_as_double      = 3.141592653589793;
    const long double pi_as_long_double = 3.1415926535897932385;

    assert(pi<float>       == pi_as_float);
    assert(pi<double>      == pi_as_double);
    assert(pi<long double> == pi_as_long_double);
}
\end{lstlisting}
%%%%%%%%%%%%% new material ends

\noindent Variable templates, like \textbf{C-style functions}, may be declared at
namespace-scope or as \texttt{static} members of a
\texttt{class}, \texttt{struct}, or \texttt{union} but are not
permitted as non\texttt{static} members nor at all in function scope:

\begin{lstlisting}[language=C++]
template <typename T> T vt1;             // OK (external linkage)
template <typename T> static T vt2;      // OK (internal linkage)

namespace N
{
    template <typename T> T vt3;           // OK (external linkage)
    template <typename T> static T vt4;    // OK (internal linkage)
}

struct S
{
    template <typename T> T vt5;         // Error, not (ù{\codeincomments{static}}ù)
    template <typename T> static T vt6;  // OK (external linkage)
};

void f3()  // Variable templates cannot be defined in functions.
{
    template <typename T> T vt7;         // compile-time error
    template <typename T> static T vt8;  // compile-time error

    vt1<bool> = true;                   // OK (to use them)
    N::vt3<bool> = true;
    N::vt4<bool> = true;
    S::vt6<bool> = true;
}
\end{lstlisting}

\noindent Like other templates, variable templates may be defined with multiple
parameters consisting of arbitrary combinations of type and non-type
parameters (including a \textbf{parameter pack}):

\begin{lstlisting}[language=C++]
namespace N
{
    template <typename V, int I, int J> V factor;  // namespace scope
}
\end{lstlisting}

\noindent Variable templates can even be defined recursively (but see
{\it\titleref{variabletemplate-potential-pitfalls}:} {\it\titleref{recursive-variable-template-initializations-require-const-or-constexpr}} on page~\pageref{recursive-variable-template-initializations-require-const-or-constexpr}):

\begin{lstlisting}[language=C++]
namespace {
template <int N>
const int sum = N + sum<N - 1>;    // recursive general template

template <> const int sum<0> = 0;  // base case specialization
}  // close unnamed namespace

void f()
{
    std::cout << sum<4> << '\n';  // prints 10
    std::cout << sum<5> << '\n';  // prints 15
    std::cout << sum<6> << '\n';  // prints 21
}
\end{lstlisting}

\noindent Note that while variable templates do not add new functionality, they significantly reduce the boilerplate associated with achieving the same goals without them.  For example, compare the definition of \lstinline!pi! above with the pre-C++14 code:

\begin{lstlisting}[language=C++]
// C++03 (obsolete)
template <typename T>
struct Pi {
    static const T value;
};

template <typename T>
const T Pi<T>::value(3.1415926535897932385);  // separate definition

void testCpp03Pi()
{
    const float       piAsFloat      = 3.1415927;
    const double      piAsDouble     = 3.141592653589793;
    const long double piAsLongDouble = 3.1415926535897932385;

    // additional boilerplate on use ((ù{\codeincomments{::value}ù))
    assert(Pi<float>::value       == piAsFloat);
    assert(Pi<double>::value      == piAsDouble);
    assert(Pi<long double>::value == piAsLongDouble);
}
\end{lstlisting}

\subsection[Use Cases]{Use Cases}\label{variabletemplate-use-cases}

\subsubsection[Parameterized constants]{Parameterized constants}\label{parametrized-constants}

A common effective use of variable templates is in the definition of
type-parameterized constants. As discussed in {\it\titleref{variabletemplate-description}} on page~\pageref{variabletemplate-description}, the mathematical
constant $\pi$ serves as our example. Here we want to
initialize the constant as part of the variable template (the literal
chosen is the shortest decimal string to do so accurately for an 80-bit
\texttt{long}~\texttt{double}){\cprotect\footnote{For
portability, a floating-point literal value of \(\pi\) that provides
sufficient precision for the longest \texttt{long}~\texttt{double} on
any relevant platform (e.g., 128 bits or 34 decimal digits:
\texttt{3.141'592'653'589'793'238'462'643'383'279'503}) should be
  used; see Section~\ref{digit-separators}, ``\titleref{digit-separators}."}}:

\begin{lstlisting}[language=C++]
template <typename T>
constexpr T pi(3.1415926535897932385);
    // smallest digit sequence to accurately represent pi as a (ù{\codeincomments{long double}}ù)
\end{lstlisting}

\noindent Notice that we have elected to use {\texttt{constexpr} variables in place of  \texttt{const} to guarantee that the floating-point \lstinline!pi! is a compile-time constant that will be usable as part of a constant expression.

With the definition above, we can provide a
\texttt{toRadians} function template that performs at maximum runtime
efficiency by avoiding needless type conversions during the computation:

\begin{lstlisting}[language=C++]
template <typename T>
constexpr T toRadians(T degrees)
{
    return degrees * (pi<T> / T(180));
}
\end{lstlisting}


\subsubsection[Reducing verbosity of type traits]{Reducing verbosity of type traits}\label{reducing-verbosity-of-type-traits}

A \textbf{type trait} is an empty type carrying compile-time information
about one or more aspects of another type. The way in which type traits
have been specified historically has been to define a class template
having the trait name and a public \texttt{static}
data member, that is conventionally called \texttt{value}, which is
initialized in the primary template to \texttt{false}. Then, for each
type that wants to advertise that it has this trait, the header defining
the trait is included and the trait is specialized for that type,
initializing \texttt{value} to \texttt{true}. We can achieve precisely
this same usage pattern replacing a trait \texttt{struct} with a
variable template whose name represents the type trait and whose type of
variable itself is always \texttt{bool}. Preferring variable templates
in this use case decreases the amount of \textbf{boilerplate code} ---
both at the point of definition and at the call
site.{\cprotect\footnote{As of C++17, the Standard Library provides a
more convenient way of inspecting the result of a type trait, by
introducing variable templates named the same way as the corresponding
traits but with an additional \texttt{\_v} suffix:

\begin{lstlisting}[language=C++, basicstyle={\ttfamily\footnotesize}]
// C++11/14
std::is_default_constructible<T>::value

// C++17
std::is_default_constructible_v<T>
\end{lstlisting}
This delay is a consequence of the train release model of the Standard: Thoughtful application of the new feature throughout the vast Standard Library required significant effort that could not be completed before the next release date for the Standard and thus was delayed until C++17.
      }}

Consider, for example, a boolean trait designating whether a particular
type \texttt{T} can be serialized to JSON:

\begin{lstlisting}[language=C++]
// isSerializableToJson.h:

template <typename T>
constexpr bool isSerializableToJson = false;
\end{lstlisting}

\noindent The header above contains the general variable template trait that, by
default, concludes that a given type is not serializable to JSON. Next
we consider the streaming utility itself:

\begin{lstlisting}[language=C++]
// serializeToJson.h:
#include <isSerializableToJson.h>  // general trait variable template

template <typename T>
JsonObject serializeToJson(const T& object)  // serialization function template
{
    static_assert(isSerializableToJson<T>,
                  "(ù{\codeincomments{T}}ù) must support serialization to JSON.");

    // ...
}
\end{lstlisting}

\noindent Notice that we have used the C++11
\texttt{static\_assert} feature to ensure that any type
used to instantiate this function will have specialized (see the next code snippet) the
general variable template associated with the specific type to be
\texttt{true}.

Now imagine that we have a type, \texttt{CompanyData}, that we would
like to advertise at compile time as being serializable to JSON. Like
other templates, variable templates can be specialized explicitly:

\begin{lstlisting}[language=C++]
// companyData.h:
#include <isSerializableToJson.h>  // general trait variable template

struct CompanyData { /* ... */ };  // type to be JSON serialized

template <>
constexpr bool isSerializableToJson<CompanyData> = true;
    // Let anyone who needs to know that this type is JSON serializable.
\end{lstlisting}

\noindent Finally, our \texttt{client} function incorporates all of the above and
attempts to serialize both a \texttt{CompanyData} object and an
\texttt{std::map<int,}~\texttt{char>}:

\begin{lstlisting}[language=C++]
// client.h:
#include <isSerializableToJson.h>  // general trait template
#include <companyData.h>           // JSON serializable type
#include <serializeToJson.h>       // serialization function
#include <map>                     // (ù{\codeincomments{std::map}}ù) (not JSON serializable)

void client()
{
    auto jsonObj0 = serializeToJson<CompanyData>();         // OK
    auto jsonObj1 = serializeToJson<std::map<int, char>>(); // compile-time error
}
\end{lstlisting}

\noindent In the \texttt{client()} function above, \texttt{CompanyData} works
fine, but, because the variable template \texttt{isSerializableToJson}
was never specialized to be \texttt{true} for type
\mbox{\texttt{std::map<int,} \texttt{char>}}, the client header will --- as
desired --- fail to compile.

\subsection[Potential Pitfalls]{Potential Pitfalls}\label{variabletemplate-potential-pitfalls}

\subsubsection[Recursive variable template initializations require {\tt const} or {\tt constexpr}]{Recursive variable template initializations require {\SubsubsecCode const} or {\SubsubsecCode constexpr}}\label{recursive-variable-template-initializations-require-const-or-constexpr}

Instantiating variable templates that are defined recursively might have a subtle issue that could produce different results{\cprotect\footnote{For
example, GCC version 4.7.0 (2017) produces the expected results whereas
  Clang version 10.x (2020) produces 1, 3, and 4, respectively.}} despite having no undefined behavior:

\begin{lstlisting}[language=C++]
#include <iostream>  // (ù{\codeincomments{std::cout}}ù)

template <int N>
int fib = fib<N - 1> + fib<N - 2>;

template <> int fib<2> = 1;
template <> int fib<1> = 1;

int main()
{
    std::cout << fib<4> << '\n';  // 3 expected
    std::cout << fib<5> << '\n';  // 5 expected
    std::cout << fib<6> << '\n';  // 8 expected

    return 0;
}
\end{lstlisting}

\noindent The root cause of this instability is
that the relative order of the initialization of the recursively generated variable template instantiations
is not guaranteed because they are not defined explicitly \emph{within the same translation
unit}.  Therefore, a similar issue might have occurred in C++03 using \lstinline!static! members of a \lstinline!struct!:
%The didactic value in answering this question dwarfs any potential
%practical value that recursive template variable instantiation can
%offer. First, consider that this same issue could, in theory, have
%occurred in C++03 using nested \texttt{static} members of a
%\texttt{struct}:

\begin{lstlisting}[language=C++]
#include <iostream>  // (ù{\codeincomments{std::cout}}ù)

template <int N> struct Fib
{
    static int value;                             // BAD IDEA: not (ù{\codeincomments{const}}ù)
};

template <> struct Fib<2> { static int value; };  // BAD IDEA: not (ù{\codeincomments{const}}ù)
template <> struct Fib<1> { static int value; };  // BAD IDEA: not (ù{\codeincomments{const}}ù)

template <int N> int Fib<N>::value = Fib<N - 1>::value + Fib<N - 2>::value;
int Fib<2>::value = 1;
int Fib<1>::value = 1;

int main()
{
    std::cout << Fib<4>::value << '\n';  // 3 expected
    std::cout << Fib<5>::value << '\n';  // 5 expected
    std::cout << Fib<6>::value << '\n';  // 8 expected

    return 0;
};
\end{lstlisting}

\noindent However, this is not an issue when using \lstinline!enum!s due to enumerators always being compile-time constants:
%The problem did not manifest, however, because the simpler solution of
%using \texttt{enum}s (below) obviated separate initialization of the
%local \texttt{static} and didn't admit the possibility of failing to
%make the initializer a compile-time constant:

\begin{lstlisting}[language=C++]
#include <iostream>  // (ù{\codeincomments{std::cout}}ù)

template <int N> struct Fib
{
    enum { value = Fib<N - 1>::value + Fib<N - 2>::value };  // OK - (ù{\codeincomments{const}}ù)
};

template <> struct Fib<2> { enum { value = 1 }; };           // OK - (ù{\codeincomments{const}}ù)
template <> struct Fib<1> { enum { value = 1 }; };           // OK - (ù{\codeincomments{const}}ù)

int main()
{
    std::cout << Fib<4>::value << '\n';  // 3 guaranteed
    std::cout << Fib<5>::value << '\n';  // 5 guaranteed
    std::cout << Fib<6>::value << '\n';  // 8 guaranteed

    return 0;
};
\end{lstlisting}

\noindent For integral variable templates, this issue can be resolved simply by adding a \lstinline!const! qualifier because the C++ Standard requires that any integral variable declared as \lstinline!const! and initialized with a compile-time constant is itself to be treated as a compile-time constant within the translation unit.
%It was not until C++14 that the variable templates feature
%readily exposed this latent pitfall involving recursive initialization
%of non-\texttt{const} variables. The root cause of the instability is
%that the relative order of the initialization of the (recursively
%generated) variable instantiations is not guaranteed because they are
%not defined explicitly \emph{within the same translation unit}. The
%magic sauce that makes everything work is the C++ language requirement
%that any variable that is declared \texttt{const} and initialized with a
%compile-time constant is itself to be treated as a compile-time constant
%within the translation unit. This compile-time-constant propagation
%requirement imposes the needed ordering to ensure that the expected
%results are portable to all conforming compilers:

\begin{lstlisting}[language=C++]
#include <iostream>  // (ù{\codeincomments{std::cout}}ù)

template <int N>
const int fib = fib<N - 1> + fib<N - 2>;  // OK - compile-time (ù{\codeincomments{const}}ù).

template <> const int fib<2> = 1;         // OK - compile-time (ù{\codeincomments{const}}ù).
template <> const int fib<1> = 1;         // OK - compile-time (ù{\codeincomments{const}}ù).

int main()
{
    std::cout << fib<4> << '\n';  // guaranteed to print out (ù{\codeincomments{3}}ù)
    std::cout << fib<5> << '\n';  // guaranteed to print out (ù{\codeincomments{5}}ù)
    std::cout << fib<6> << '\n';  // guaranteed to print out (ù{\codeincomments{8}}ù)

    return 0;
}
\end{lstlisting}

\noindent Note that replacing each of the three \texttt{const} keywords with
{\texttt{constexpr} in the example above also achieves the
desired goal, does not consume memory in the \textbf{static data
space}, and would also be applicable to non-integral constants.

\subsection[Annoyances]{Annoyances}\label{annoyances}

\subsubsection[Variable templates do not support template template parameters]{Variable templates do not support template template parameters}\label{variable-templates-do-not-support-template-template-parameters}

Although a class or function template can accept a
\textbf{template template class parameter}, no equivalent
construct is available for variable templates{\cprotect\footnote{Pusz has proposed for C++23 a way to increase consistency between
variable templates and class templates when used as template template
  parameters; see \textbf{{pusz20}}.}}:

\begin{lstlisting}[language=C++]
template <typename T> T vt(5);

template <template <typename> class>
struct S { };

S<vt> s1;  // compile-time error
\end{lstlisting}

\noindent Providing a wrapper \texttt{struct}
around a variable template might therefore be necessary in case the variable template needs to be passed to an interface
accepting a \textbf{template template parameter}:

\begin{lstlisting}[language=C++]
template <typename T>
struct Vt { static constexpr T value = vt<T>; }

S<Vt> s2;  // OK
\end{lstlisting}


\subsection[See Also]{See Also}\label{see-also}

\begin{itemize}
\item{Section~\ref{constexprvar}, ``\titleref{constexprvar}" — Conditionally safe C++11 feature providing an alternative to \texttt{const} template variables that can reduce unnecessary consumption of the \textbf{static data space}}
\end{itemize}

\subsection[Further Reading]{Further Reading}\label{further-reading}

None so far


