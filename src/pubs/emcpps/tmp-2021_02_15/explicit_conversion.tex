% 12 Feb 2021, revisions in; ready for Josh's code fixes




\emcppsFeature{
    short={\lstinline!explicit! Operators},
    tocshort={{\TOCCode explicit} Operators},
    long={Explicit Conversion Operators},
    toclong={Explicit Conversion Operators},
    rhshort={{\RHCode explicit} Operators},
}{explicit-conversion-operators}
\setcounter{table}{0}
\setcounter{footnote}{0}
\setcounter{lstlisting}{0}
%\section[{\tt explicit} Operators]{Explicit Conversion Operators}\label{explicit-conversion-operators}


Ensure that a user-defined type is convertible to another type only in
contexts where the conversion is made obvious in the code.

\subsection[Description]{Description}\label{description-explicitconv}

Though sometimes desirable, implicit conversions achieved via user-defined \emph{conversion
functions} --- either \romeogloss{converting constructors} accepting a
single argument or \romeogloss{conversion operators} --- can also be problematic, especially when the
conversion involves a commonly used type (e.g., \lstinline!int! or
\lstinline!double!){\cprotect\footnote{Use of a conversion operator to
calculate distance from the origin in this unrealistically simple \lstinline!Point!
example is for didactic purposes only. In practice, we would typically
use a named function for this purpose; see \intraref{potential-pitfalls-explicitconv}{sometimes-a-named-function-is-better}.}}:
%{\it\titleref{potential-pitfalls-explicitconv}: \titleref{sometimes-a-named-function-is-better}} on page~\pageref{sometimes-a-named-function-is-better}.}}:

\begin{emcppslisting}[language=C++]
class Point  // implicitly convertible from an (ù{\codeincomments{int}}ù) or to a (ù{\codeincomments{double}}ù)
{
    int d_x, d_y;

public:
    Point(int x = 0, int y = 0);  // default, conversion, & value constructor
    // ...
    operator double() const;  // Return distance from origin as a (ù{\codeincomments{double}}ù).
};
\end{emcppslisting}
    
\noindent As ever, calling a function that takes a \lstinline!Point! but
accidentally passing an \lstinline!int! can lead to surprises:

\begin{emcppslisting}[language=C++]
void g0(Point p);         // arbitrary function taking a (ù{\codeincomments{Point}}ù) object by value
void g1(const Point& p);  // arbitrary function taking a (ù{\codeincomments{Point}}ù) by (ù{\codeincomments{const}}ù) reference

void f1(int i)
{
    g0(i);  // oops, called (ù{\codeincomments{g0}}ù) with (ù{\codeincomments{Point(i, 0)}}ù) by mistake
    g1(i);  // oops, called (ù{\codeincomments{g1}}ù) with (ù{\codeincomments{Point(i, 0)}}ù) by mistake
}
\end{emcppslisting}
    
\noindent This problem could have been solved even in C++98 by declaring the
constructor to be \lstinline!explicit!:

\begin{emcppslisting}[language=C++]
explicit Point(int x = 0, int y = 0);  // explicit converting constructor
\end{emcppslisting}
    
\noindent If the conversion is desired, it must now be specified explicitly:

\begin{emcppslisting}[language=C++]
void f2(int i)
{
    g0(i);         // error: could not convert (ù{\codeincomments{i}}ù) from (ù{\codeincomments{int}}ù) to (ù{\codeincomments{Point}}ù)
    g1(i);         // error: invalid initialization of reference type
    g0(Point(i));  // OK
    g1(Point(i));  // OK
}
\end{emcppslisting}
    
\noindent The companion problem stemming from an \emph{implicit conversion
operator}, albeit less severe, remained:

\begin{emcppslisting}[language=C++]
void h(double d);

double f3(const Point& p)
{
    h(p);      // OK? Or maybe called (ù{\codeincomments{h}}ù) with a "hypotenuse" by mistake
    return p;  // OK? Or maybe this is a mistake too.
}
\end{emcppslisting}
    
\noindent As of C++11, we can now use the \romeogloss{\lstinline!explicit! specifier}
when declaring \romeogloss{conversion operators} (as well as
\romeogloss{converting constructors}), thereby forcing the client to request
conversion explicitly --- e.g., using \romeogloss{direct initialization} or
\lstinline!static_cast!:

\begin{emcppslisting}[language=C++]
struct S0 { explicit operator int(); };

void g()
{
    S0 s0;
    int i = s0;                    // error (copy initialization)
    int k(s0);                     // OK (direct initialization)
    double d = s0;                 // error (copy initialization)
    int j = static_cast<int>(s0);  // OK (static cast)
    if (s0) { }                    // error (contextual conversion to (ù{\codeincomments{bool}}ù))
    double e(s0);                  // error (direct initialization)
}
\end{emcppslisting}
    
\noindent In contrast, had the conversion operator above not been declared to be
\lstinline!explicit!, all conversions shown above would compile:

\begin{emcppslisting}[language=C++]
struct S1 { /* implicit */ operator int(); };

void f()
{
    S1 s1;
    int i = s1;                    // OK (copy initialization)
    double d = s1;                 // OK (copy initialization)
    int j = static_cast<int>(s1);  // OK (static cast)
    if (s1) { }                    // OK (contextual conversion to (ù{\codeincomments{bool}}ù))
    int k(s1);                     // OK (direct initialization)
    double e(s1);                  // OK (direct initialization)
}
\end{emcppslisting}
    
\noindent Additionally, the notion of \romeogloss{contextual convertibility to
\lstinline!bool!} 
%{\cprotect\footnote{Since the early days of C++, a common
%  idiom to test for validity of an object has been to use it in a
%  context where it can (implicitly) convert itself to a type whose value
%  can be interpreted (contextually) as a boolean, with \lstinline!true!
%  implying validity (and \lstinline!false! otherwise). Implicit conversion
%  to \lstinline!bool! (an integral type) was considered too dangerous, 
%  so the cumbersome \romeogloss{safe-\lstinline!bool! idiom} was used instead,
%  converting to a type that --- while contextually convertible to
%  \lstinline!bool! --- could not (by design) participate in any other
%  operations. While making the conversion to \lstinline!bool! (or
%  \lstinline!const!~\lstinline!bool!) \lstinline!explicit! solves the safety
%  issue, the benefit of the idiom would be entirely lost if an explicit
%  cast would have to be performed to test for validity. To address this,
%  C++11 extends contextual conversion to \lstinline!bool! for a given
%  expression \lstinline!E! to include an application of
%  \lstinline!static_cast<const!~\lstinline!volatile!~\lstinline!bool>! to
%  \lstinline!E!, thus enabling explicit conversion to \lstinline!bool! to be
%  used in lieu of the (now deprecated) \romeogloss{safe-\lstinline!bool! idiom}; see
%  \cite{sharpe13}.}} 
applicable to arguments of logical operations
(e.g., \lstinline!&&!, \lstinline!||!, and \lstinline|!|) and conditions of
most control-flow constructs (e.g., \lstinline!if!, \lstinline!while!) was
extended in C++11 to admit \emph{explicit} (user-defined) \lstinline!bool!
conversion operators (see \intraref{use-cases-explicitconv}{enabling-contextual-conversions-to-bool-as-a-test-for-validity}):
%{\it\titleref{use-cases-explicitconv}: \titleref{enabling-contextual-conversions-to-bool-as-a-test-for-validity}} on page~\pageref{enabling-contextual-conversions-to-bool-as-a-test-for-validity}):

\begin{emcppslisting}[language=C++]
struct S2 { explicit operator bool(); };

void h()
{
    S2 s2;
    int i = s2;                    // error (copy initialization)
    double d = s2;                 // error (copy initialization)
    int j = static_cast<int>(s2);  // error (static cast)
    if (s2) { }                    // OK (contextual conversion to (ù{\codeincomments{bool}}ù))
    int k(s2);                     // error (direct initialization)
    double fd(s2);                 // error (direct initialization)
    bool b0 = s2; // error (copy initialization)
    bool b1(s2);  // OK (direct initialization)                                     
    !s2;          // OK (contextual conversion to bool)                                                
    s2 && s2;     // OK (contextual conversion to bool)
}
\end{emcppslisting}
    
%\noindent Prior to C++11, essentially the same effect as having an \emph{explicit}
%\lstinline!operator!~\lstinline!bool()! member was achieved (albeit far less
%conveniently) via the \romeogloss{safe-\lstinline!bool! idiom}.

\subsection[Use Cases]{Use Cases}\label{use-cases-explicitconv}

\subsubsection[Enabling contextual conversions to {\tt bool} as a test for validity]{Enabling contextual conversions to {\SubsubsecCode bool} as a test for validity}\label{enabling-contextual-conversions-to-bool-as-a-test-for-validity}

Having a conventional test for validity that involves testing whether the object itself evaluates to \lstinline!true! or \lstinline!false! is an idiom that goes back to the
origins of C++. The Standard input/output library, for example, uses this
idiom to determine if a given stream is valid:

\begin{emcppslisting}[language=C++]
// C++03
#include <ostream>  // (ù{\codeincomments{std::ostream}}ù)

std::ostream& printTypeValue(std::ostream& stream, double value)
{
    if (stream)  // relies on an implicit conversion to (ù{\codeincomments{bool}}ù)
    {
        stream << "double(" << value << ')';
    }
    else
    {
        // ... (handle stream failure)
    }

    return stream;
}
\end{emcppslisting}
    
\noindent Implementing the implicit conversion to \lstinline!bool! was, however,
problematic as the straight\-forward approach of using a
\romeogloss{conversion operator} could easily allow accidental misuse to go
undetected:

\begin{emcppslisting}[language=C++]
class ostream
{
    // ...
  public:

    /* implicit */ operator bool();  // hypothetical (bad) idea
};

int client(std::ostream& out)
{
    // ...
    return out + 1;  // likely a latent runtime bug: always returns 1 or 2
}
\end{emcppslisting}
    
\noindent The classic workaround, the \romeogloss{safe-\lstinline!bool! idiom},\footnote{https://www.artima.com/cppsource/safebool.html} 
was to return some obscure pointer\linebreak[4] %%%%%
 type (e.g., \romeogloss{pointer to
member}) that could not possibly be useful in any context other\linebreak[4] %%%%%
 than one in which \lstinline!false! and a null pointer-to-member value are
treated equivalently. With explicit conversion operators, such workarounds are no longer required. As discussed in \intrarefsimple{description-explicitconv}, 
%\textit{\titleref{description-explicitconv}} on page~\pageref{description-explicitconv}, 
a conversion operator to type
\lstinline!bool! that is declared \lstinline!explicit! continues to act as if
it were \emph{implicit} only in those places where we might want it to
do so and nowhere else --- i.e., exactly those places that enable
\romeogloss{contextual conversion to \lstinline!bool!}.{\cprotect\footnote{Note that two
consecutive \lstinline|!| operators can be used to synthesize a
\romeogloss{contextual conversion to \lstinline!bool!} --- i.e., if \lstinline!X! is an
expression that is explicitly convertible to \lstinline!bool!, then
\lstinline|(!!(X))| will be \lstinline!(true)! or \lstinline!(false)!
  accordingly.}}

As a concrete example, consider a \lstinline!ConnectionHandle! class that
can be in either a \emph{valid} or \emph{invalid} state. For the user's
convenience and consistency with other proxy types (e.g., raw pointers)
that have a similar \emph{invalid} state, representing the invalid (or null) state via an explicit conversion to
\lstinline!bool! might be desirable:


\begin{emcppslisting}[language=C++]
#include <cstddef>  // (ù{\codeincomments{std::size\_t}ù)                                            
#include <iostream> // (ù{\codeincomments{std::cerr}ù)
struct ConnectionHandle
{
    std::size_t maxThroughput() const;
        // Return the maximum throughput (in bytes) of the connection.

    explicit operator bool() const;
        // Return (ù{\codeincomments{true}}ù) if the handle is valid and (ù{\codeincomments{false}}ù) otherwise.
};
\end{emcppslisting}
    
\noindent Instances of \lstinline!ConnectionHandle! will convert to \lstinline!bool!
only where one might reasonably want them to do so, say, as the
predicate of an \lstinline!if! statement:

\begin{emcppslisting}[language=C++]
int ping(const ConnectionHandle& handle)
{
    if (handle)  // OK (contextual conversion to bool)
    {
        // ...
        return 0;  // success
    }

    std::cerr << "Invalid connection handle.\n";
    return -1;  // failure
}
\end{emcppslisting}
    
\noindent Having an \lstinline!explicit! conversion operator prevents unwanted
conversions to \lstinline!bool! that might otherwise happen inadvertently:

\begin{emcppslisting}[language=C++]
bool hasEnoughThroughput(const ConnectionHandle& ingress,
                         const ConnectionHandle& egress)
{
    return ingress.throughput() <= egress;  // Compilation error, thankfully
//                                    ^~~~~~
}
\end{emcppslisting}
    
\noindent After the relational operator (\lstinline!<=!) in the example above, the
programmer mistakenly wrote \lstinline!egress! instead of
\mbox{\lstinline!egress.maxThroughput()!}. Fortunately the conversion operator of\linebreak[4] %%%%%
\mbox{\lstinline!ConnectionHandle!} was declared to be \lstinline!explicit! and a
compile-time error ensued; if the conversion had
been \emph{implicit}, the example code above would have compiled, and,
if executed, the above (faulty) implementation of the \mbox{\lstinline!hasEnoughThroughput!}
function would have silently exhibited well-defined but incorrect
behavior.

\subsection[Potential Pitfalls]{Potential Pitfalls}\label{potential-pitfalls-explicitconv}

\subsubsection[Sometimes implicit conversion \emph{is} indicated]{Sometimes implicit conversion {\sfbsubsubsecitalRomeo is} indicated}\label{sometimes-implicit-conversion-is-indicated}

Implicit conversions to and from common arithmetic types, especially
\lstinline!int!, are generally ill advised given the likelihood of
accidental misuse. 
%However, sometimes \emph{implicit}
%conversion is exactly what is needed. Such cases occur frequently with
%wrapper and proxy types that might need to interoperate with a large
%legacy codebase. Consider, for example, an initial implementation of
%memory allocators in which each constructor takes, as an optional
%trailing argument, a pointer to an abstract memory resource that itself
%provides pure virtual \lstinline!allocate! and \lstinline!deallocate! member
%functions. Later, we decide to move in the direction of the
%\lstinline!std::pmr! (C++17) standard and wrap those pointers in classes
%that support additional operations. Making such constructors on the
%wrapper explicit would force every client supplying an allocator to a
%container to rework their code (e.g., by using \lstinline!static_cast!).
%An implicit conversion in this case is further justified because the
%likelihood of accidental spontaneous conversion to an \lstinline!Allocator!
%is all but nonexistent.
%
%The same sort of stability argument favors implicit conversion for proxy
%types intended to be dropped in and used in existing codebases. If, for
%example, we wanted to provide a proxy for a writeable
%\lstinline!std::string! that, say, also logged, we might want an implicit
%conversion to a \lstinline!std::string&&! (perhaps using reference qualifiers; 
%see \featureref{\locatione}{refqualifiers}). In such cases, making the
%conversion explicit would entirely defeat the purpose of the proxy,
%which is to achieve new functionality with minimal effect on existing
%client code.
However, for proxy types that are intended to be drop-in replacements for the types they represent, implicit conversions are precisely what we want. Consider, for example, a \lstinline!NibbleConstReference!
 proxy type that represents the 4-bit integer elements of a \lstinline!PackedNibbleVector!:

%%%%%%%%%%%%% JOSH: first line is blank. Intentional or error? 
\begin{emcppslisting}

class NibbleConstReference
{
    // ...
public:
    operator int() const; // IMPLICIT

    // ...
};

class PackedNibbleVector 
{
    // ...
public:
    bool empty() const;
    NibbleConstReference operator[](int index) const;
    
    // ...
};
\end{emcppslisting}

\noindent The \lstinline!NibbleConstReference! proxy is intended to interoperate well with other integral types in various expressions and making its conversion operator \lstinline!explicit! hinders its intended use as a drop-in replacement by requiring an explicit conversion (a.k.a. cast):

\begin{emcppslisting}
int firstOrZero(const PackedNibbleVector& values)
{
    return values.empty() 
        ? 0 
        : values[0]; // Compiles only if conversion operator is implicit.
}
\end{emcppslisting}


\subsection[Sometimes a named function is better]{Sometimes a named function is better}\label{sometimes-a-named-function-is-better}

Other kinds of overuses of even \emph{explicit} conversion
operators exist. Like any user-defined operator, when the operation being
implemented is not somehow either canonical or ubiquitously idiomatic
for that operator, expressing that operation by a
named (i.e., non-operator) function is often better. Recall from \intrarefsimple{description-explicitconv} 
% {\it\titleref{description-explicitconv}} on page~\pageref{description-explicitconv} 
that we used a conversion operator of
class \lstinline!Point! to represent the distance from the origin. This
example serves both to illustrate how conversion operators \emph{can} be
used and also how they probably should \emph{not} be. Consider that (1)
many mathematical operations on a 2-D integral point might return a \lstinline!double! (e.g., \lstinline!magnitude!,
\lstinline!angle!) and (2) we might want to represent the same
information but in different units (e.g., \lstinline!angleInDegrees!,
\lstinline!angleInRadians!).{\cprotect\footnote{Another valid design
decision is returning an object of type \lstinline!Angle! that captures
the amplitude and provides named accessory to the different units
  (e.g., \lstinline!asDegrees!, \lstinline!asRadians!).}}

Rather than employing any conversion \emph{operator} (\lstinline!explicit!
or otherwise), consider instead providing a named function, which (1) is
automatically \lstinline!explicit! and (2) affords both flexibility (in
writing) and clarity (in reading) for a variety of domain-specific
functions --- now and in the future --- that might well have had
overlapping return types:

\begin{emcppslisting}[language=C++]
class Point  // only explicitly convertible (and from only an (ù{\codeincomments{int}}ù))
{
    int d_x, d_y;

public:
    explicit Point(int x = 0, int y = 0);  // explicit converting constructor
    // ...
    double magnitude() const;  // Return distance from origin as a (ù{\codeincomments{double}}ù).
};
\end{emcppslisting}
    
\noindent Note that defining \romeogloss{nonprimitive functionality}, like
\lstinline!magnitude!, in a separate \emph{utility} at a higher level in the
physical hierarchy (e.g., \lstinline!PointUtil::magnitude(const!~\lstinline!Point&!~\lstinline!p)!) might be better still.{\cprotect\footnote{For more on
separating out \romeogloss{nonprimitive functionality}, see
  \cite{lakos20}, sections 3.2.7--3.2.8, pp 529--552.}}

\subsection[Annoyances]{Annoyances}\label{annoyances}

None so far

\subsection[See Also]{See Also}\label{see-also}

None so far

\subsection[Further Reading]{Further Reading}\label{further-reading}

None so far

