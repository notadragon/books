\usepackage{glossaries-extra}

% This provides the \emcppsgloss command which specifies a glossary reference.  The
% non-optional argument is the text that will be shown.  The optional argument
% is the id of the glossary entry (if that differs from the text that will be
% shown).  Unlike the normal \gls commands, we want to put the full text of what
% will be shown at the location of the glossary entry, especially to facilitate
% putting additional markup like forced page and line breaks, or emphasizing
% part of the ter,. in the middle of the glossary term.  

% This toggle contro
\newtoggle{emcppsGlossDebug}
%\toggletrue{emcppsGlossDebug}
\togglefalse{emcppsGlossDebug}

% This single argument macro will be used to render a glossary term the first
% time it is used in a section or feature
\newcommand\emcppsfirstgloss[1]{{%
  \color{purple}\textbf{#1}}%
}

% This single argument macro will be used to render a glossary term whenever
% it is used again in the same feature, or used within the glossary itself.
\newcommand\emcppslatergloss[1]{{%
    \color{red}{#1}}%
}

% We pick up in tooling additional representations of a glossary term from the
% "alts" key on the glossary entries, which we add as a key here.   We don't
% currently use these values in the LaTeX itself.
\glsaddstoragekey{alts}%
                 {}%
                 {\glsentryalts}

% These force a reset of the tracking of "first use" for glossary terms at the
% start of chapters and features.
\appto\emcppschapterstart{\glsresetall}
\appto\emcppsfeaturestart{\glsresetall}

% This is the \emcppsgloss command.
%  Optional argument #1: Glossary Id
%  Argumenet #2: Text to show
\newcommand\emcppsgloss[2][]{%
  \ifthenelse{\equal{}{#1}}%
  {%
    \def\emcppsglossid{#2}%
  }%
  {%
    \def\emcppsglossid{#1}%
  }%
  \ifglsused{\emcppsglossid}%
  {%
    \iftoggle{emcppsGlossDebug}{<LATER <\emcppsglossid> > }{}%
    \emcppslatergloss{#2}%
  }%
  {%
    \iftoggle{emcppsGlossDebug}{<FIRST <\emcppsglossid> > }{}%
    \emcppsfirstgloss{#2}%
  }%
  \glsadd{\emcppsglossid}%
  \glsunset{\emcppsglossid}%
}

% This is the \emcppsglossgloss command, to be used within the glossary.  This
% takes the same arguments as \emcppsgloss, but essentially ignores the first
% argument
%  Optional argument #1: Glossary Id
%  Argumenet #2: Text to show
\newcommand\emcppsglossgloss[2][]{%
  \iftoggle{emcppsGlossDebug}{<GLOSS <\emcppsglossid> > }{}%
  \emcppslatergloss{#2}%
}

\makeglossaries
