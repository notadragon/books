\usepackage{listings}  % for setting attractive computer code
\usepackage{xespotcolor-Romeo}

\makeatletter
\@addtoreset{lstlisting}{section}  % to renumber listings per section
\makeatother

\NewSpotColorSpace{PANTONE}
\AddSpotColor     {PANTONE}  {PANTONE661PC} {PANTONE\SpotSpace 661\SpotSpace PC} {1 0 0.05 0.72}
\SetPageColorSpace{PANTONE}
\definecolor{PANTONE661PC}  {spotcolor} {PANTONE661PC,1.0}
\definecolor{PANTONE661PCA} {spotcolor} {PANTONE661PC,0.5}
\definecolor{grey}           {gray}      {0.5}
\definecolor{grey}           {gray}      {0.5}

\definecolor{maroon}{cmyk}{.26,1,.63,.17}
\definecolor{skyblue}{cmyk}{.62,.07,0,0}
\definecolor{forestgreen}{cmyk}{.9,0,.9,0}
\definecolor{mustard}{cmyk}{.22,.14,1,0}
\definecolor{purple}{cmyk}{.83,.81,0,0}
\definecolor{codegray}{cmyk}{0,0,0,.70}

%%%%% CHANGED FOR ROMEO
% changed font to Cousine, changed to 2ex indent

\makeatletter
\lstdefinestyle{Romeo}{
language=[11]C++,
 framerule=0.5pt,
    keywordstyle={\ttfamily\bfseries},
    morekeywords={char8\_t,co\_await,co\_return,co\_yield,
                  concept,consteval,constinit,requires},
    commentstyle={\color{PANTONE661PCA}\ttfamily},
    xleftmargin=2ex,
    escapeinside={(ù}{ù)},
    showstringspaces=false, 
  basicstyle=\ttfamily%
   \lst@ifdisplaystyle\small\fi
 }
\makeatother

\lstset{style=Romeo}

\lstdefinestyle{footcode}{
style=Romeo,
basicstyle=\ttfamily\footnotesize,framerule=0.5pt,
keywordstyle={\ttfamily\bfseries\footnotesize},
commentstyle={\color{grey}\ttfamily},
xleftmargin=2ex,
escapeinside={(ù}{ù)},
showstringspaces=false, 
}  


\lstdefinestyle{plain}{basicstyle=\ttfamily\small,framerule=0.5pt,
    keywordstyle={\ttfamily},
    commentstyle={\color{grey}\ttfamily},
    xleftmargin=2ex,
    escapeinside={(ù}{ù)},
    showstringspaces=false, 
}  

\lstdefinestyle{footcodeplain}{
basicstyle=\ttfamily\footnotesize,framerule=0.5pt,
keywordstyle={\ttfamily},
commentstyle={\color{grey}\ttfamily},
xleftmargin=2ex,
escapeinside={(ù}{ù)},
showstringspaces=false, 
}  


% setting the code in small sans serif font and
%% setting the margin for the code.  Nov 2015: setting the rules to be 0.5pt
%%%%%%%%%%%%%%
\newcommand{\codeincomments}{\color{PANTONE661PCA}\ttfamily}
\newcommand{\keyincomments}{\color{PANTONE661PCA}\ttfamily\bfseries}
\newcommand{\emphincomments}{\color{PANTONE661PCA}\ttfamily\itshape}
\newcommand{\commentcolor}{\color{PANTONE661PCA}}

% All code samples that should be tested are going to be written within the
% following environment
%
% Source for initial implementation for dumping to files:
% https://tex.stackexchange.com/questions/284189/how-to-dump-listings-to-a-file
%
% Change the following in existing features:
%    \begin{lstlisting}[language=C++]   \begin{emcppslisting}
%    \end{lstlisting}                   \end{emcppslisting}
%
\usepackage{currfile}
\makeatletter

% This counter is used to generate unique ids for the files that will be output.
% Ideally it would also be reset per feature, but that is not strictly needed.
\newcounter{emcppslistings}
\setcounter{emcppslistings}{0}
\@addtoreset{emcppslistings}{section}  % to renumber code snippets per section
\lst@RequireAspects{writefile}

% A batch id for code snippets that should be concatenated together to compile.
\lst@Key{emcppsbatch}{}{\def\emcppsBatch{#1}}

% A reason why this code snippet should not be compiled, such as code that is
% explained outside of the example as an error, or code that is more
% descriptive than actually valid code.  Useless and irrelevant on
% emcppshiddenslistings.
\lst@Key{emcppsignore}{}{\def\emcppsIgnore{#1}}

% Which standards the code example should be compiled for, which will be passed
% to the -std= argument of gcc or clang  This must be on the first snippet in a
% batch, otherwise a default oc "c++11,c++14" will be used.
% Example value [emcppsstandards={c++11,c++17}]
\lst@Key{emcppsstandards}{}{\def\emcppsStandards{#1}}

% Provides a list of line numbers within this snippet that are errors.
% Useless and irrelevant on emcppshiddenslistings.
\lst@Key{emcppserrorlines}{}{\def\emcppsErrorLines{#1}}

% Provides an alternate output file name for this snippet.  Multiple snippets
% in the same batch with the same output file name will be concatenated.
% This is equivalent to a comment "// <filename.(h|cpp)>" in the snippet, but
% is not visible in the text.
\lst@Key{emcppsoutputfile}{}{\def\emcppsOutputFile{#1}}

\lstnewenvironment{emcppslisting}[1][]
{%
  \lstset{style=Romeo,#1}%
  \stepcounter{emcppslistings}%
  \lst@BeginAlsoWriteFile{listings/\currfilename-\theemcppslistings.lst}%
  \immediate\write\lst@WF{// ----Listing start: \currfilename:\the\inputlineno-----}%
  \ifthenelse{\equal{}{\emcppsBatch}}{}{\immediate\write\lst@WF{// ----Batch: \emcppsBatch------}}%
  \ifthenelse{\equal{}{\emcppsIgnore}}{}{\immediate\write\lst@WF{// ----Ignore: \emcppsIgnore------}}%
  \ifthenelse{\equal{}{\emcppsStandards}}{}{\immediate\write\lst@WF{// ----Standards: \emcppsStandards------}}%
  \ifthenelse{\equal{}{\emcppsErrorLines}}{}{\immediate\write\lst@WF{// ----ErrorLines: \emcppsErrorLines------}}%
  \ifthenelse{\equal{}{\emcppsOutputFile}}{}{\immediate\write\lst@WF{// ----OutputFile: \emcppsOutputFile------}}%
}
{%
  \immediate\write\lst@WF{// ----Listing done: \currfilename:\the\inputlineno-----}%
  \lst@EndWriteFile%
}

\lstnewenvironment{emcppshiddenlisting}[1][]
{%
  \lstset{style=Romeo,aboveskip=0pt,belowskip=0pt,#1}%
  \stepcounter{emcppslistings}%
  \lst@BeginWriteFile{listings/\currfilename-\theemcppslistings.lst}%
  \immediate\write\lst@WF{// ----Hidden Listing start: \currfilename:\the\inputlineno-----}%
  \ifthenelse{\equal{}{\emcppsBatch}}{}{\immediate\write\lst@WF{// ----Batch: \emcppsBatch------}}%
  \ifthenelse{\equal{}{\emcppsStandards}}{}{\immediate\write\lst@WF{// ----Standards: \emcppsStandards------}}%
}
{%
  \immediate\write\lst@WF{// ----Listing done: \currfilename:\the\inputlineno-----}%
  \lst@EndWriteFile%
}

\makeatother
