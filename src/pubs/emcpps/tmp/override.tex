

The \texttt{override} keyword ensures that a member function overrides a corresponding \texttt{virtual} member function in a base class.

\subsection[Description]{Description}\label{description}

The \textbf{contextual keyword}\glossary{contextual keyword} \texttt{override} can be provided at the
end of a member-function declaration to ensure that the decorated
function is indeed \textbf{overriding}\glossary{overriding} a corresponding \texttt{virtual}
member function in a base class (i.e., not \textbf{hiding}\glossary{hiding} it or
otherwise inadvertently introducing a distinct function declaration):

\begin{lstlisting}[language=C++]
struct Base
{
    virtual void f(int);
    void g(int);
};

struct Derived : Base
{
    void f();              // hides (ù{\codeincomments{Base::f(int)}}ù) (likely mistake)
    void f() override;     // error: (ù{\codeincomments{Base::f()}}ù) not found

    void f(int);           // implicitly overrides (ù{\codeincomments{Base::f(int)}}ù)
    void f(int) override;  // explicitly overrides (ù{\codeincomments{Base::f(int)}}ù)

    void g();              // hides (ù{\codeincomments{Base::g(int)}}ù) (likely mistake)
    void g() override;     // error: (ù{\codeincomments{Base::g()}}ù) not found

    void g(int);           // hides (ù{\codeincomments{Base::g(int)}}ù) (likely mistake)
    void g(int) override;  // Error: (ù{\codeincomments{Base::g()}}ù) is not (ù{\codeincomments{virtual}}ù).
};
\end{lstlisting}
    
\noindent Use of this feature expresses design intent so that (1) human readers
are aware of it and (2) compilers can validate it.

\subsection[Use Cases]{Use Cases}\label{use-cases}

\subsubsection[Ensuring that a member function of a base class is being overridden]{Ensuring that a member function of a base class is being overridden}\label{ensuring-that-a-member-function-of-a-base-class-is-being-overridden}

Consider the following polymorphic hierarchy of error-category classes
(as we might have defined them using C++03):

\begin{lstlisting}[language=C++]
struct ErrorCategory
{
    virtual bool equivalent(const ErrorCode& code, int condition);
    virtual bool equivalent(int code, const ErrorCondition& condition);
};

struct AutomotiveErrorCategory : ErrorCategory
{
    virtual bool equivalent(const ErrorCode& code, int condition);
    virtual bool equivolent(int code, const ErrorCondition& condition);
};
\end{lstlisting}
    
\noindent Notice that there is a defect in the last line of the example above:
\texttt{equivalent} has been misspelled. Moreover, the compiler did not
catch that error. Clients calling \texttt{equivalent} on
\texttt{AutomotiveErrorCategory} will incorrectly invoke the base-class
function. If the function in the base class happens to be defined, the
code might compile and behave unexpectedly at runtime. Now, suppose
that over time the interface is changed by marking the
equivalence-checking function \texttt{const} to bring the interface
closer to that of \texttt{std::error\_category}:

\begin{lstlisting}[language=C++]
struct ErrorCategory
{
    virtual bool equivalent(const ErrorCode& code, int condition) const;
    virtual bool equivalent(int code, const ErrorCondition& condition) const;
};
\end{lstlisting}
    
Without applying the corresponding modification to all classes deriving
from\linebreak[4] \texttt{ErrorCategory}, the semantics of the program change due to
the derived classes now hiding (instead of overriding) the base class's
\texttt{virtual} member function. Both of the errors discussed above
would be detected automatically by decorating the \texttt{virtual}
functions in all derived classes with \texttt{override}:

\begin{lstlisting}[language=C++]
struct AutomotiveErrorCategory : ErrorCategory
{
    bool equivalent(const ErrorCode& code, int condition) override;
        // compile-time error when base class changed

    bool equivolent(int code, const ErrorCondition& code) override;
        // compile-time error when first written
};
\end{lstlisting}
    
\noindent What's more, \texttt{override} serves as a clear indication to the human
reader of the derived class's author's intent to customize the
behavior of \texttt{ErrorCategory}. For any given member function, use
of \texttt{override} necessarily renders any use of \texttt{virtual} for
that function syntactically and semantically redundant. The only
(cosmetic) reason for retaining \texttt{virtual} in the presence of
\texttt{override} would be that \texttt{virtual} appears to the left of
the function declaration (as it always has) instead of all the way to
the right (as \texttt{override} does now).

\subsection[Potential Pitfalls]{Potential Pitfalls}\label{potential-pitfalls}

\subsubsection[Lack of consistency across a code base]{Lack of consistency across a code base}\label{lack-of-consistency-across-a-codebase}

Relying on \texttt{override} as a means of ensuring that changes to
base-class interfaces are propagated across a codebase can prove
unreliable if this feature is used inconsistently --- i.e., statically
verified in every circumstance where its use would be appropriate. In
particular, altering the signature of a \texttt{virtual} member function
in a base class and then compiling ``the world'' will always flag (as an
error) any nonmatching derived-class function where \texttt{override}
was used but might fail (even to warn) where it is not.

\subsection[Annoyances]{Annoyances}\label{annoyances}

None so far

\subsection[See Also]{See Also}\label{see-also}

None so far

\subsection[Further Reading]{Further Reading}\label{further-reading}

None so far


