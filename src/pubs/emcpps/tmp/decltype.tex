

The keyword \texttt{decltype} enables the compile-time inspection of the \textbf{declared type}\glossary{declared type} of an \textbf{entity}\glossary{entity} \textit{or} the type and
\textbf{value category}\glossary{value category} of an expression.

\subsection[Description]{Description}\label{description}

What results from the use of \texttt{decltype} depends on the nature of
its operand.

\subsubsection[Use with (typically named) entities]{Use with (typically named) entities}\label{use-with-(typically-named)-entities}

If an unparenthesized operand is either an \textbf{id-expression} that names an entity (or a non-type template parameter) or a \textbf{class member access expression}\glossary{class member access expression} (that identifies a class member), \texttt{decltype} yields the \emph{declared type} (the type of the \emph{entity} indicated by the operand):

\begin{lstlisting}[language=C++]
int i;                // (ù{\codeincomments{decltype(i)}}ù)   -> (ù{\codeincomments{int}}ù)
std::string s;        // (ù{\codeincomments{decltype(s)}}ù)   -> (ù{\codeincomments{std::string}}ù)
int* p;               // (ù{\codeincomments{decltype(p)}}ù)   -> (ù{\codeincomments{int}}ù)*
const int& r = *p;    // (ù{\codeincomments{decltype(r)}}ù)   -> (ù{\codeincomments{const int\&}}ù)
struct { char c; } x; // (ù{\codeincomments{decltype(x.c)}}ù) -> (ù{\codeincomments{char}}ù)
double f();           // (ù{\codeincomments{decltype(f)}}ù)   -> (ù{\codeincomments{double()}}ù)
double g(int);        // (ù{\codeincomments{decltype(g)}}ù)   -> (ù{\codeincomments{double(int)}}ù)
\end{lstlisting}
    

\subsubsection[Use with (unnamed) expressions]{Use with (unnamed) expressions}\label{use-with-(unnamed)-expressions}

When \texttt{decltype} is used with any other expression \texttt{E} of
type \texttt{T}, the result incorporates both the expression's type and
its \textbf{value category}\glossary{value category}:
%\mclhead{Value category of \texttt{E}} & \mclhead{Result of
%\texttt{decltype(E)}} \\ 
%\emph{prvalue} & \texttt{T} \\ 
%\emph{lvalue} & \texttt{T\&} \\ 
%\emph{xvalue} & \texttt{T\&\&} \\
%%%%%%%%%%% changing this per Greg & design discussion 
\begin{center}
{\small \begin{tabular}{c|c}\thickhline 
\rowcolor[gray]{.9}   {\sffamily\bfseries Value category of {\ttfamily\bfseries E}} 
& {\sffamily\bfseries Result of {\ttfamily\bfseries decltype(E)}} \\ \hline 
{\it prvalue} &\texttt{T} \\ \hline
{\it lvalue} & \texttt{T\&} \\ \hline 
{\it xvalue} & \texttt{T\&\&} \\ \thickhline
\end{tabular}
}
\end{center}
\noindent The three integer expressions below illustrate the various value
categories:

\begin{lstlisting}[language=C++]
decltype(0)   // -> (ù{\codeincomments{int}}ù)   (*prvalue* category)

int i;
decltype((i)) // -> (ù{\codeincomments{int\&}}ù)  (*lvalue* category)

int&& g();
decltype(g()) // -> (ù{\codeincomments{int\&\&}}ù) (*xvalue* category)
\end{lstlisting}
    
\noindent Much like the \texttt{sizeof}
operator (which too is resolved at compile time), the expression operand to \texttt{decltype} is not evaluated:

\begin{lstlisting}[language=C++]
int i = 0;
decltype(i++) j;  // equivalent to (ù{\codeincomments{int j;}}ù)
assert(i == 0);   // The function (ù{\codeincomments{next}}ù) is never invoked.
\end{lstlisting}
    

\subsection[Use Cases]{Use Cases}\label{use-cases}

\subsubsection[Avoiding unnecessary use of explicit typenames]{Avoiding unnecessary use of explicit typenames}\label{avoiding-unnecessary-use-of-explicit-typenames}

Consider two logically equivalent ways of declaring a vector of
iterators into a list of \texttt{Widgets}:

\begin{lstlisting}[language=C++]
std::list<Widget> widgets;
std::vector<std::list<Widget>::iterator> widgetIterators;
    // (1) The full type of (ù{\codeincomments{widgets}}ù) needs to be restated, and (ù{\codeincomments{iterator}}ù)
    // needs to be explicitly named.

std::list<Widget> widgets;
std::vector<decltype(widgets.begin())> widgetIterators;
    // (2) Neither (ù{\codeincomments{std::list}}ù) nor (ù{\codeincomments{Widget}}ù) nor (ù{\codeincomments{iterator}}ù) need be named
    // explicitly.
\end{lstlisting}
    
\noindent Notice that, when using \texttt{decltype}, if the C++ type representing
the widget changes (e.g., from \texttt{Widget} to, say,
\texttt{ManagedWidget}) or the container used changes (e.g., from
\texttt{std::list} to \texttt{std::vector}), the declaration of
\texttt{widgetIterators} need not necessarily change.

\subsubsection[Expressing type-consistency explicitly]{Expressing type-consistency explicitly}\label{expressing-type-consistency-explicitly}

In some situations, repetition of explicit type names might
inadvertently result in latent defects caused by mismatched types during
maintenance. For example, consider a \texttt{Packet} class exposing a
\texttt{const} member function that returns a \texttt{std::uint8\_t}
representing the length of the packet's checksum:

\begin{lstlisting}[language=C++]
class Packet
{
    // ...
public:
    std::uint8_t checksumLength() const;
};
\end{lstlisting}
    
\noindent This (tiny) unsigned 8-bit type was selected to minimize bandwidth usage
as the checksum length is sent over the network. Next, picture a loop
that computes the checksum of a \texttt{Packet}, using the same (i.e.,
\texttt{std::uint8\_t}) type for its iteration variable (to match the
type returned by \texttt{Packet::checksumLength}):

\begin{lstlisting}[language=C++]
void f()
{
    Checksum sum;
    Packet data;

    for (std::uint8_t i = 0; i < data.checksumLength(); ++i)  // brittle
    {
        sum.appendByte(data.nthByte(i));
    }
}
\end{lstlisting}
    

Now suppose that, over time, the data transmitted by the \texttt{Packet}
type grows to the point where the range of a \texttt{std::uint8\_t}
value might not be enough to ensure a sufficiently reliable checksum. If
the type returned by \texttt{checksumLength()} is changed to, say,
\texttt{std::uint16\_t} without updating the type of the iteration
variable \texttt{i} in lockstep, the loop might
silently{\cprotect\footnote{As of this writing, neither
GCC~9.3 nor Clang~10.0.0 provide
a warning (using \texttt{-Wall}, \texttt{-Wextra}, and
\texttt{-Wpedantic}) for the comparison between \texttt{std::uint8\_t}
and \texttt{std::uint16\_t} --- even if both (1) the value returned by
\texttt{checksumLength} does not fit in a 8-bit integer and (2) the
body of the function is visible to the compiler. Decorating
\texttt{checksumLength} with \texttt{constexpr} causes
\texttt{clang++} to issue a warning, but this is clearly not a general
solution.}} become infinite.{\cprotect\footnote{The (tiny) loop
variable is promoted to an \texttt{unsigned} \texttt{int} for
comparison purposes but wraps (to 0) whenever its value prior to
  being incremented is 255.}}

Had \texttt{decltype(packet.checksumLength())} been used to express the
type of \texttt{i}, the types would have remained consistent and the
ensuing (``truncation'') defect would naturally have been avoided:

\begin{lstlisting}[language=C++]
// ...
for (decltype(data.checksumLength()) i = 0; i < data.checksumLength(); ++i)
// ...
\end{lstlisting}
    

\subsubsection[Creating an auxiliary variable of generic type]{Creating an auxiliary variable of generic type}\label{creating-an-auxiliary-variable-of-generic-type}

Consider the task of implementing a generic \texttt{loggedSum} function
(template) that returns the sum of two arbitrary objects \texttt{a} and
\texttt{b} after logging both the operands and the result value (e.g.,
for debugging or monitoring purposes). To avoid computing the (possibly
expensive) sum twice, we decide to create an auxiliary function-scope
variable, \texttt{result}. Since the type of the sum depends on both
\texttt{a} and \texttt{b}, we can use
\texttt{decltype(a}~\texttt{+}~\texttt{b)} to infer the type for both
(1) the (trailing) return type{\cprotect\footnote{Using
\texttt{decltype(a}~\texttt{+}~\texttt{b)} as a return type is
significantly different from relying on \emph{automatic return type
  deduction}. See Section~\ref{auto-feature}, ``\titleref{auto-feature}," for more information.}} of the
function (see Section~\ref{trailing-function-return-types}, ``\titleref{trailing-function-return-types}") and (2) the auxiliary variable:

\begin{lstlisting}[language=C++]
template <typename A, typename B>
auto loggedSum(const A& a, const B& b)
    -> decltype(a + b)                 // (1) exploiting trailing return types
{
    decltype(a + b) result = a + b;    // (2) auxiliary generic variable
    LOG_TRACE << a << " + " << b << " = " << result;
    return result;
}
\end{lstlisting}
    

\subsubsection[Determining the validity of a generic expression]{Determining the validity of a generic expression}\label{determining-the-validity-of-a-generic-expression}

In the context of generic-library development, \texttt{decltype} can be
used in conjunction with \textbf{SFINAE}\footnote{``Substitution Failure Is Not An Error"} to validate an expression involving a
template parameter.

For example, consider the task of writing a generic \texttt{sortRange}
function template that, given a \textbf{range}\glossary{range}, invokes either the
\texttt{sort} member function of the argument (the one specifically
optimized for that type) if available, or else falls back to the more
general \texttt{std::sort}:

\begin{lstlisting}[language=C++]
template <typename Range>
void sortRange(Range& range)
{
    sortRangeImpl(range, 0);
}
\end{lstlisting}
    
\noindent The client-facing \texttt{sortRange} function (above) delegates its
behavior to an (overloaded) \texttt{sortRangeImpl} function (below),
invoking the latter with the \texttt{range} and a \emph{disambiguator}
of type \texttt{int}. The type of this additional parameter (its value
is arbitrary) is used to give priority to the \texttt{sort} member
function (at compile time) by exploiting overload resolution rules in
the presence of an implicit (\emph{standard}) conversion (from
\texttt{int} to \texttt{long}):

\begin{lstlisting}[language=C++]
template <typename Range>
void sortRangeImpl(Range& range,
                   long)                  // low priority: standard conversion
{
    // fallback implementation
    std::sort(std::begin(range), std::end(range));
}
\end{lstlisting}
    
\noindent The fallback overload of \texttt{sortRangeImpl} (above) will accept a
\texttt{long} \emph{disambiguator} (requiring a standard conversion from
\texttt{int}) and will simply invoke \texttt{std::sort}. The more
specialized overload of \texttt{sortRangeImpl} (below) will accept an
\texttt{int} \emph{disambiguator} (requiring no conversions) and thus
will be a better match, provided a range-specific sort is available:

\begin{lstlisting}[language=C++]
template <typename Range>
void sortRangeImpl(Range& range,
                   int,                          // high priority: exact match
                   decltype(range.sort())* = 0)  // check expression validity
{
    // optimized implementation
    range.sort();
}
\end{lstlisting}
    
\noindent Note that, by exposing{\cprotect\footnote{The relative position of
\texttt{decltype(range.sort())} in the signature of
\texttt{sortRangeImpl} is not significant, as long as it is visible to
the compiler (as part of the function's \emph{logical interface})
during template substitution. This particular example (shown in the
main text) makes use of a function parameter that is defaulted to
\texttt{nullptr}. Alternatives involving a trailing return type or a
default template argument are also viable:

\begin{lstlisting}[language=C++, basicstyle={\ttfamily\footnotesize}]
template <typename Range>
auto sortRangeImpl(Range& range, int) -> decltype(range.sort(), void());
    // The comma operator is used to force the return type to (ù{\codeincomments{void}}ù),
    // regardless of the return type of (ù{\codeincomments{range.sort()}}ù).

template <typename Range, typename = decltype(std::declval<Range&>().sort()>
auto sortRangeImpl(Range& range, int);
    // (ù{\codeincomments{std::declval}}ù) is used to generate a reference to (ù{\codeincomments{Range}}ù) that can
    // be used in an unevaluated expression.
\end{lstlisting}
      }} \texttt{decltype(range.sort())} as part of
\texttt{sortRangeImpl}'s declaration, the more specialized overload will
be discarded during template substitution if \texttt{range.sort()} is
not a valid expression for the deduced \texttt{Range}
type.{\cprotect\footnote{The technique of exposing a (possibly unused)
unevaluated expression (e.g., using \texttt{decltype}) in a function's
declaration for the purpose of expression-validity detection prior to
template instantiation is commonly known as \textbf{expression SFINAE}\glossary{expression SFINAE}
and is a restricted form of the more general (classical) SFINAE that acts exclusively on
expressions visible in a function's signature rather than on (obscure)
  template-based type computations.}}

Putting it all together, we see that exactly two possible
outcomes exist for the original client-facing \texttt{sortRange} function
invoked with a range argument of type \texttt{R}:

\begin{itemize}
\item{If \texttt{R} does have a \texttt{sort} member function, the more specialized overload (of\linebreak[4] \texttt{sortRangeImpl}) will be viable (as \texttt{range.sort()} is a well-formed expression) and preferred because the \emph{disambiguator} \texttt{0} (of type \texttt{int}) requires no conversion.}
\item{Otherwise, the more specialized overload will be discarded during template substitution (as \texttt{range.sort()} is not a well-formed expression) and the only remaining (more general) \texttt{sortRangeImpl} overload will be chosen instead.}
\end{itemize}

\subsection[Potential pitfalls]{Potential pitfalls}\label{potential-pitfalls}

Perhaps surprisingly, \texttt{decltype(x)} and \texttt{decltype((x))}
will sometimes yield different results for the same expression
\texttt{x}:

\begin{lstlisting}[language=C++]
int i = 0; // (ù{\codeincomments{decltype(i)}}ù) yields (ù{\codeincomments{int}}ù).
           // (ù{\codeincomments{decltype((i))}}ù) yields (ù{\codeincomments{int\&}}ù).
\end{lstlisting}
    
\noindent In the case where the unparenthesized operand is an entity having a
declared type \texttt{T} and the parenthesized operand is an expression
whose value category is represented (by \texttt{decltype}) as the same
type \texttt{T}, the results will coincidentally be the same:

\begin{lstlisting}[language=C++]
int& ref = i;  // (ù{\codeincomments{decltype(ref)}}ù) yields (ù{\codeincomments{int\&}}ù).
               // (ù{\codeincomments{decltype((ref))}}ù) yields (ù{\codeincomments{int\&}}ù).
\end{lstlisting}
    
\noindent Wrapping its operand with parentheses ensures \texttt{decltype} yields
the \textbf{value category}\glossary{value category} of a given expression. This technique can be
useful in the context of metaprogramming --- particularly in the case of
\textbf{value category}\glossary{value category} propagation.

\subsection[Annoyances]{Annoyances}\label{annoyances}

None so far

\subsection[See Also]{See Also}\label{see-also}

None so far

\subsection[Further reading]{Further reading}\label{further-reading}

None so far


