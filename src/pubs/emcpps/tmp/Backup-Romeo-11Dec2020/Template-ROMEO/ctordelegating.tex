

Delegating constructors are constructors of a class that delegate initialization to another
constructor of the same class.

\subsection[Description]{Description}\label{description}

A \textbf{delegating constructor} is a constructor of a
\textbf{user-defined type} (i.e., \texttt{class}, \texttt{struct}, or
\texttt{union}) that invokes another constructor defined for the same
\textbf{UDT} as part of its initialization of an object of that type.
The syntax for invoking another constructor within a type is to specify
the name of the type as the only element in the \textbf{member
initializer list}:

\begin{lstlisting}[language=C++]
struct S0
{
    int d_i;

    S0(int i) : d_i(i)        { }  // non-delegating constructor
    S0()      : S0(0)         { }  // OK, delegates to (ù{\codeincomments{S0(int)}}ù)
    S0(bool)  : S0(0), d_i(0) { }  // error: delegation must be on its own
};
\end{lstlisting}
    
\noindent Multiple delegating constructors can be chained together (one calling
exactly one other) so long as cycles are avoided (see
{\it\titleref{ctordelegrating-potential-pitfalls}:} {\it\titleref{delegation-cycles}} on page~\pageref{delegation-cycles}). Once a \emph{target} (i.e., invoked via delegation) constructor returns,
the body of the delegator is invoked:

\begin{lstlisting}[language=C++]
#include <iostream>  // (ù{\codeincomments{std::cout}}ù)

struct S1
{
    S1(int, int)            { std::cout << 'a'; }
    S1(int)      : S1(0, 0) { std::cout << 'b'; }
    S1()         : S1(0)    { std::cout << 'c'; }
};

void f()
{
    S1 s;  // OK, prints (ù{\codeincomments{"abc"}}ù) to (ù{\codeincomments{stdout}}ù)
}
\end{lstlisting}
    
\noindent If an exception is thrown while executing a nondelegating constructor,
the object being initialized is considered only \textbf{partially
constructed} (i.e., the object is not yet known to be in a valid state)
and hence its destructor will \emph{not} be
invoked{\cprotect\footnote{The destructor of a \textbf{partially
constructed} object will not be invoked. However, the destructors of
each successfully constructed base and of data members will still be
invoked:

\begin{lstlisting}[language=C++, basicstyle={\ttfamily\footnotesize}]
#include <iostream>

using std::cout;
struct A { A() { cout << "A() "; } ~A() { cout << "~A() "; } };
struct B { B() { cout << "B() "; } ~B() { cout << "~B() "; } };

struct C : B
{
    A d;

    C()  { cout << "C() "; throw 0; }  // non-delegating constructor that throws
    ~C() { cout << "~C() ";         }  // destructor that never gets called
};

void f() try { C c; } catch(int) { }
    // prints (ù{\codeincomments{"B() A() C() \textasciitilde A() \textasciitilde B()"}}ù) to (ù{\codeincomments{stdout}}ù)
\end{lstlisting}
    
\noindent Notice that base-class \texttt{B} and member \texttt{d} of type
\texttt{a} were fully constructed, and so their respective destructors
are called, even though the destructor for class \texttt{C} itself is
  never executed.}}:

\begin{lstlisting}[language=C++]
#include <iostream>  // (ù{\codeincomments{std::cout}}ù)

struct S2
{
    S2()  { std::cout << "S2() ";  throw 0; }
    ~S2() { std::cout << "~S2() ";          }
};

void f() try { S2 s; } catch(int) { }
    // prints only "(ù{\codeincomments{S2() }}ù)" to (ù{\codeincomments{stdout}}ù) (i.e., the destructor of (ù{\codeincomments{S2}}ù) is never
    // invoked)
\end{lstlisting}
    
\noindent However, if an exception is thrown in the body of a delegating
constructor, the object being initialized is considered \textbf{fully
constructed}, as the target constructor must have returned control to
the delegator; hence the overall object's destructor \emph{will} be
invoked:

\begin{lstlisting}[language=C++]
#include <iostream>  // (ù{\codeincomments{std::cout}}ù)

struct S3
{
    S3()           { std::cout << "S3() "              }
    S3(int) : S3() { std::cout << "S3(int) "; throw 0; }
    ~S3()          { std::cout << "~S3() "             }
};

void f() try { S3 s(0); } catch(int) { }
    // prints "(ù{\codeincomments{S3() S3(int) \texttt{\~{}}S3() }}ù)" to (ù{\codeincomments{stdout}}ù)
\end{lstlisting}
    

\subsection[Use Cases]{Use Cases}\label{ctordelegating-use-cases}

\subsubsection[Avoiding code duplication among constructors]{Avoiding code duplication among constructors}\label{avoiding-code-duplication-among-constructors}

Avoiding gratuitous code duplication is considered by many to be a best
practice. Having one ordinary member function call another has always
been an option, but having one constructor invoke another constructor
directly has not. Classic workarounds included repeating the code or
else factoring the code into a private member function that would be
called from multiple constructors. The drawback with this workaround is
that the private method, not being a constructor, would be unable to
make use of \textbf{member initialization lists} to construct base-class
and member objects efficiently. As of C++11, \emph{delegating
constructors} can be used to minimize code duplication when some of
the same operations are performed across multiple constructors without
having to forgo efficient initialization.

As an example, consider an \texttt{IPV4Host} class representing a
network endpoint that can either be constructed by (1) a 32-bit address
and a 16-bit port or (2) an IPV4 string with
\texttt{XXX.XXX.XXX.XXX:XXXXX} format{\cprotect\footnote{Note that
this initial design might itself be suboptimal in that the
representation of the IPV4 address and port value might profitably be
factored out into a separate \textbf{value-semantic} class, say,
\texttt{IPV4Host}, that itself might be constructed in multiple ways;
  see {\it\titleref{ctordelegrating-potential-pitfalls}:} {\it\titleref{suboptimal-factoring}} on page~\pageref{suboptimal-factoring}.}}:

\begin{lstlisting}[language=C++]
#include <cstdint>  // (ù{\codeincomments{std::uint16\_t}}ù), (ù{\codeincomments{std::uint32\_t}}ù)

class IPV4Host
{
    // ...

public:
    IPV4Host(std::uint32_t address, std::uint16_t port)
    {
        if (!connect(address, port))  // code repetition: BAD IDEA
        {
            throw ConnectionException{address, port};
        }
    }

    IPV4Host(const std::string& ip)
    {
        std::uint32_t address = extractAddress(ip);
        std::uint16_t port = extractPort(ip);

        if (!connect(address, port))  // code repetition: BAD IDEA
        {
            throw ConnectionException{address, port};
        }
    }
};
\end{lstlisting}
    
\noindent Prior to C++11, working around such code duplication would require the
introduction of a separate, subordinate (private) helper function, that
would, in turn, be called by each of the constructors:

\begin{lstlisting}[language=C++]
#include <cstdint>  // (ù{\codeincomments{std::uint16\_t}}ù), (ù{\codeincomments{std::uint32\_t}}ù)

class IPV4Host
{
    // ...

private:
    void validate(std::uint32_t address, std::uint16_t port)  // helper function
    {
        if (!connect(address, port))  // factored implementation of needed logic
        {
            throw ConnectionException{address, port};
        }
    }

public:
    IPV4Host(std::uint32_t address, std::uint16_t port)
    {
        validate(address, port);  // Invoke factored private helper function.
    }

    IPV4Host(const std::string& ip)
    {
        std::uint32_t address = extractAddress(ip);
        std::uint16_t port = extractPort(ip);

        validate(address, port);  // Invoke factored private helper function.
    }
};
\end{lstlisting}
    
\noindent Alternatively, the constructor accepting a string can be rewritten to
delegate to the one accepting \texttt{address} and \texttt{port},
avoiding repetition without having to delegate to a private function:

\begin{lstlisting}[language=C++]
#include <cstdint>  // (ù{\codeincomments{std::uint16\_t}}ù), (ù{\codeincomments{std::uint32\_t}}ù)

class IPV4Host
{
    // ...

public:
    IPV4Host(std::uint32_t address, std::uint16_t port)
    {
        if(!connect(address, port))
        {
            throw ConnectionException{address, port};
        }
    }

    IPV4Host(const std::string& ip)
        : IPV4Host{extractAddress(ip), extractPort(ip)}
    {
    }
};
\end{lstlisting}
    
\noindent Compared to the pre-C++11 workaround of introducing a private
\texttt{init} function containing the duplicated logic, use of
delegating constructors results in less boilerplate and fewer run-time
operations, as data members (and base classes) can be initialized
directly through the \textbf{member initialization list}, rather than be
\emph{assigned-to} in the body of \texttt{init} (assuming copy
assignment is even supported on that type), but see
{\it\titleref{ctordelegrating-potential-pitfalls}:} {\it\titleref{suboptimal-factoring}} on page~\pageref{suboptimal-factoring}. 

\subsection[Potential Pitfalls]{Potential Pitfalls}\label{ctordelegrating-potential-pitfalls}

\subsubsection[Delegation cycles]{Delegation cycles}\label{delegation-cycles}

If a constructor delegates to itself either directly or indirectly, the
program is \textbf{ill-formed, no diagnostic required}. While some
compilers can detect delegation cycles at compile time, they are not
required (nor necessarily able) to do so. For example, consider a simple
delegation cycle comprising two constructors:

\begin{lstlisting}[language=C++]
struct S  // Object
{
    S(int)  : S(true) { }  // delegating constructor
    S(bool) : S(0)    { }  // delegating constructor
};
\end{lstlisting}
    
\noindent Not all popular compilers will warn you that the program above is
ill-formed.{\cprotect\footnote{GCC 10.x does not detect this delegation
cycle at compile time and produces a binary that, if run, will
necessarily exhibit \textbf{undefined behavior}. Clang 10.x, on the
other hand, halts compilation with a helpful error message:

\begin{lstlisting}[language=C++, basicstyle={\ttfamily\footnotesize}]
error: constructor for (ù{\codeincomments{S}}ù) creates a delegation cycle
\end{lstlisting}\vspace*{-1ex}
      }} Therefore the programmer is responsible for ensuring that
no delegation cycles are present.

\subsubsection[Suboptimal factoring]{Suboptimal factoring}\label{suboptimal-factoring}

The need for delegating constructors might result from initially
suboptimal factoring --- e.g., in the case where the same \textbf{value}
is being presented in different forms to a variety of different
\textbf{mechanisms}. For example, consider the \texttt{IPV4Host} class
in {\it\titleref{ctordelegating-use-cases}} (which starts on page~\pageref{ctordelegating-use-cases}). While having two constructors to
initialize the host might be appropriate, if either (1) the number of
ways of expressing the same value increases or (2) the number of
consumers of that value increases, we might be well advised to create a
separate \textbf{value semantic} type, e.g., \texttt{IPV4Address}, to
represent that value{\cprotect\footnote{The notion that each component
in a subsystem ideally performs one focused function well is sometimes
referred to as separation of (logical) concerns or
  fine-grained (physical) factoring; see \textbf{{lakos20}},
  sections 0.4, 3.2.7, and 3.5.9, pp.~20--28, 529--530, and 674--676,
  respectively.}}:

\begin{lstlisting}[language=C++]
#include <cstdint>  // (ù{\codeincomments{std::uint16\_t}}ù), (ù{\codeincomments{std::uint32\_t}}ù)

struct IPV4Address
{
    std::uint32_t d_address;
    std::uint16_t d_port;

    IPV4Address(std::uint32_t address, std::uint16_t port)
        : d_address{address}, d_port{port}
    {
    }

    IPV4Address(const std::string& ip)
        : IPV4Address{extractAddress(ip), extractPort(ip)}
    {
    }
};
\end{lstlisting}
    
\noindent Note that \texttt{IPV4Address} itself makes use of delegating
constructors but as a purely private, encapsulated implementation
detail. With the introduction of \texttt{IPV4Address} into the codebase,
\texttt{IPV4Host} (and similar components requiring an
\texttt{IPV4Address} value) can be redefined to have a single
constructor (or other previously overloaded member function) accepting
an \texttt{IPV4Address} object as an argument.

\subsection[Annoyances]{Annoyances}\label{annoyances}

None so far

\subsection[See Also]{See Also}\label{see-also}

None so far

\subsection[Further Reading]{Further Reading}\label{further-reading}

None so far



