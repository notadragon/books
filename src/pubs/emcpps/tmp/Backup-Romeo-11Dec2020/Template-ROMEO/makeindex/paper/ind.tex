% Document Type: LaTeX
% Master File: ind.tex

% Master File: ind.tex
% Document preamble for ind.tex

\documentstyle[11pt,doublespace,regmar]{article}

% Special fonts
\font\mf=manfnt
\font\b=cmb10
\font\Bigbf=cmbx10 scaled\magstep3
\font\bigbf=cmbx10 scaled\magstep2
\font\medbf=cmbx10 scaled\magstep1
\font\medsl=cmsl10 scaled\magstep1
\font\medit=cmmi10 scaled\magstep1
\font\medrm=cmr10 scaled\magstep1
\font\ssl=cmsl8
\font\sit=cmmi8
\font\sbf=cmbx8
\font\sc=cmcsc10


% Various TeX logos

\def\MF{\mf METAFONT}
\def\AmSTeX{$\cal A$\kern-.1667em\lower.5ex\hbox{$\cal M$}\kern-.125em
            $\cal S$-\TeX}
\def\BibTeX{{\rm B\kern-.05em{\sc i\kern-.025em b}\kern-.08em
            T\kern-.1667em\lower.7ex\hbox{E}\kern-.125emX}}
\def\LaTeX{{\rm L\kern-.36em\raise.3ex\hbox{\sc a}\kern-.15em
           T\kern-.1667em\lower.7ex\hbox{E}\kern-.125emX}}
\def\SliTeX{S\kern-.065em L\kern-.18em\raise.32ex\hbox{i}\kern-.03em\TeX}
\def\VorTeX{V\kern-2.7pt\lower.5ex\hbox{O\kern-1.4pt R}\kern-2.6pt
            T\kern-.1667em\lower.7ex\hbox{E}\kern-.125emX}
\def\TM{\sl {\TeX}-mode\/}
\def\BM{\sl B\kern-.05em{\ssl I\kern-.025em B}\kern-.08em
            T\kern-.1667em\lower.7ex\hbox{E}\kern-.125emX-mode}
\def\Lit{\it L\kern-.25em\raise.3ex\hbox{\sit a}\kern-.15em
           T\kern-.1667em\lower.7ex\hbox{E}\kern-.125emX}
\def\Lbf{\bf L\kern-.25em\raise.3ex\hbox{\sbf a}\kern-.15em
           T\kern-.1667em\lower.7ex\hbox{E}\kern-.125emX}
\def\Lsl{\sl L\kern-.25em\raise.3ex\hbox{\ssl a}\kern-.15em
           T\kern-.1667em\lower.7ex\hbox{E}\kern-.125emX}
\def\Bit{\it B\kern-.05em{\sit I\kern-.025em B}\kern-.08em
           T\kern-.1667em\lower.7ex\hbox{E}\kern-.125emX}
\def\Bbf{\Bigbf B\kern-.05em{\bigbf I\kern-.025em B}\kern-.08em{\TeX}}
\def\Mbf{\medbf B\kern-.05em{\sbf I\kern-.025em B}\kern-.08em{\TeX}}
\def\Bsl{\ssl B\kern-.05em{\sit I\kern-.025em B}\kern-.08em{\TeX}}

\def\AL{\sl Aloe\/}
\def\GAN{\sl Gandalf\/}
\def\ET{\sl Etude\/}
\def\PEN{\sl Pen\/}
\def\PAN{\sl Pan\/}
\def\DIT{\sl Ditroff\/}
\def\DVI{\sc DVI}
\def\DT{\sl DviTool\/}
\def\E{\sl Emacs\/}
\def\EASE{\sl Ease\/}
\def\GLN{\sl Gremlin\/}
\def\IL{\sl Illustrator\/}
\def\BR{\sl Bravo\/}
\def\ST{\sl Star\/}
\def\MW{\sl MacWrite\/}
\def\MD{\sl MacDraw\/}
\def\MP{\sl MacPaint\/}
\def\ID{\sl ideal\/}
\def\INI{\tt initex}
\def\PIC{\sl pic\/}
\def\TBL{\sl tbl\/}
\def\EQN{\sl eqn\/}
\def\PS{\sl PostScript\/}
\def\DDL{\sl DDL\/}
\def\IP{\sl Interpress\/}
\def\SB{\sl Scribe\/}
\def\TF{\sl troff\/}
\def\VI{\sl vi\/}
\def\VIR{\tt virtex}
\def\unix{\sc UNIX}
\def\TG{\sl Tioga\/}
\def\GG{\sl Gargoyle\/}
\def\ST{\sl Star\/}
\def\CD{\sl Cedar\/}
\def\AD{\sl Andrew\/}
\def\DI{\sl Diamond\/}
\def\IPS{\sl Interleaf Publishing System\/}
\def\FM{\sl FrameMaker\/}
\def\PM{\sl PageMaker\/}
\def\RSG{\sl Ready-Set-Go!\/}
\def\WD{\sl MicroSoft Word\/}
\def\CW{\sl CricketDraw\/}
\def\TW{\sl Tweedle\/}
\def\JN{\sl Juno\/}
\def\JAN{\sl Janus\/}
\def\SP{\sl Sketchpad\/}
\def\TL{\sl ThingLab\/}
\def\MT{\sl Mint\/}
\def\LR{\sl Lara\/}
\def\W{\sc WYSIWYG}
\def\NW{\sl NeWS\/}
\def\QL{\sl Quill\/}
\def\LL{\sl Lilac\/}
\def\TNT{\sl tnt\/}
\def\PI{\sl Pico\/}
\def\QD{\sl QuickDraw\/}
\def\VP{\sl Ventura Publisher\/}
\def\MI{\sl MakeIndex\/}
\def\SGML{\sl SGML\/}
\def\IS{\sl Interscript\/}
\def\ODA{\sl ODA\/}

\def\BF#1{{\bf #1}}
\def\IT#1{{\it #1\/}}
\def\SL#1{{\sl #1\/}}
\def\RM#1{{\rm #1}}
\def\TT#1{{\tt #1}}

\def\IF{{\bf if}}
\def\ELSE{{\bf else}}
\def\ELIF{{\bf else if}}
\def\WHILE{{\bf while}}
\def\FOR{{\bf for}}
\def\DO{{\bf do}}
\def\UNTIL{{\bf until}}
\def\LOOP{{\bf loop}}
\def\RETURN{{\bf return}}
\def\CONTINUE{{\bf continue}}
\def\TRUE{{\bf true}}
\def\FALSE{{\bf false}}
\def\NOT{{\bf not}}
\def\AND{{\bf and}}
\def\OR{{\bf or}}

%% psfig not needed after picture mode version of fig1.ps created 
%%	-- Nelson H. F. Beebe <beebe@math.utah.edu> [08-Jul-1991]
%% \input{psfig}
\nofiles
\begin{document}
\pagestyle{phcheadings}		% defined in undhead.tex
\def\s{\thinspace}
\newcommand{\sitem}{\hspace*{20pt}}
\newcommand{\ssitem}{\hspace*{30pt}}
\newenvironment{iexample}{\begin{center}
   \begin{tabular}[t]{@{}ll|@{\hspace{.2em}}}}{\end{tabbing}\end{minipage}
   \end{center}}
\newcommand{\sindex}{\end{tabular}\hspace{.3em}\begin{minipage}[t]{\textwidth}
   \begin{tabbing}}
\newcommand{\tindex}{\end{tabular}\hspace{1em}\begin{minipage}[t]{\textwidth}
   \begin{tabbing}}
\newcommand{\hd}[1]{\multicolumn{1}{c|}{\sl #1\/}}
\newcommand{\hdr}[1]{\multicolumn{1}{c}{\sl #1\/}}


\title{Index Preparation and Processing\thanks{Sponsored
in part by the National Science Foundation under Grant MCS-8311787,
and by the Defense Advanced Research Projects Agency (DoD),
ARPA Order No. 4871, monitored by Space and Naval Warfare Systems Command,
under Contract No. N00039-84-C-0089.}
}

\author{{\sl Pehong Chen}\thanks{Additional support has been provided
by an IBM Graduate Fellowship.}\\
	{\sl Michael A. Harrison}\\\\
	Computer Science Division \\
	University of California \\
	Berkeley, CA 94720
}

\def\today{}
\maketitle

\begin{abstract}
\singlespace
Index preparation is a tedious and time-consuming task.
This paper indicates how the indexing process can be automated
in a way that is largely independent of a specific typesetting system
and independent of the format being used.
Fundamental issues related to this process are identified and analyzed.
Specifically, we develop a framework for placing index
commands in the document.  In addition, the design of
a general purpose index processor that 
transforms a raw index into an alphabetized version is described.
The resulting system has proved very useful and effective in producing
indexes for several books, technical reports, and manuals.
A comparison of our system with indexing facilities
available from a variety of other document preparation environments is given.

{\it Keywords\/}: index placement, index processing, source-language model,
direct manipulation.
\end{abstract}

\section{Introduction}
Although there has been a great deal of activity in
electronic publishing~[1],
there are still aspects of document composition that have not been
fully automated.
One of the most time-consuming concerns is the preparation of an index.
In ordinary books, an index allows a reader to
access essential information easily.
A poor index with many omissions or poorly chosen concepts
detracts from other aspects of the book.
For highly complex technical material that may include computer programs,
different kinds of indices may
reference even the identifiers of a programming language.  A good example
of an elaborate indexing scheme can be found in Knuth's
{\it {\TeX}: The Program\/}~[2] and his WEB system~[3] in general.
For computer programs like these,
completeness is essential and the accuracy of
traditional hand methods will not suffice for software engineering
applications.

Standard authors' guides, such as Reference~[4],
recommend that index terms be marked on page proofs or on a separate set of
galley proofs.
The traditional method is to use $3 \times 5$ cards, appropriately called
index cards.
A page number is added to a card when a reference is encountered.
Sorting is done by hand and the process is tedious and error-prone.
Computers offer an opportunity to significantly reduce the amount of labor
invested in this process while noticeably improving the quality of the
resulting index.

This paper indicates how the indexing process can be automated
in a way that is largely independent of a specific typesetting system
and independent of the format being used.
Specifically, we develop a framework for placing index
commands in the document.  In addition, the design of
a general purpose index processor that 
transforms a raw index into an alphabetized version is described.
These concepts have been implemented as part of an extensive authoring
environment~[5].  This environment includes a suite of Lisp programs
for the index placing facility and a C program for the
index processor.  The resulting system has been successfully
used in producing indexes for some books~[6,7]
and a number of technical reports and manuals.

Indexing issues under both {\it source-language\/} and
{\it direct-manipulation\/}~[8,9] paradigms are considered.
In a source-language based system, the user specifies
the document with interspersed commands, which is then passed to a formatter,
and the output is obtained.  The source-language usually provides some
abstraction and control mechanisms such as procedures, macros,
conditionals, variables, etc. in much the same way as a high-level programming
language does.  In a direct-manipulation environment,
the user manipulates the document output appearance directly
by invoking built-in operators available through menus and buttons.
These systems are highly interactive; the result of invoking an operation
is observed instantaneously, thereby creating an illusion that the user
is ``directly'' manipulating the underlying object.

The document attributes in direct-manipulation systems
are usually specified by a declarative language encapsulated as
form-based property sheets.  These property sheets
correspond to textual markup tags that can be imported to
or exported from the direct-manipulation system for document interchange
purposes.  While a source representation is maintained explicitly in the
source-language model, the notion of a document semantics specification
language is somewhat implicit in direct-manipulation systems.

Some people have called direct-manipulation systems {\W}
({\it what-you-see-is-what-you-get\/}).
The two concepts are not equivalent, however.
{\W} refers to the correspondence between what is presented on a video display
and what can be generated on some other device.  In the context of
electronic publishing, {\W} means that there is a very close relationship
in terms of document appearance
between a screen representation and the final hardcopy.
Direct manipulation is a more general concept that models user interfaces.
A direct-manipulation document preparation system may not have
a {\W} relationship between its display representation and the final hardcopy.
Conversely, a batch-oriented, source-language based formatter 
may be coupled with a previewer that is {\W}.

This paper presents some general indexing problems and our solutions 
in a top-down fashion.
First we discuss the desirable features of an index processor
and then some design and implementation considerations for such a processor.
Our goal is to arrive at general purpose solutions.
Next, a framework is introduced under which an author enters index commands
or tags with much reduced overhead.
The examples shown are in {\LaTeX}~[10], a high-level
document preparation language based on {\TeX}~[11].
The model, however, is not restricted to any particular formatting language,
nor to the source-language paradigm.
We also examine some unique indexing issues in
electronic document development environments that do not seem to find
appropriate counterparts in traditional printed material.
Finally our indexing facilities are evaluated against those available in
other formatting systems such as {\SB}~[12],
{\TF}~[13], and some direct-manipulation environments.

%\newpage
\begin{figure}[p]
%% \centerline{
%% \psfig{figure=fig1.ps,height=4.5in}
%% }
%% Replace PostScript figure with LaTeX picture mode version
%% designed to closely match the original fig1.ps output
%% so that this document can be formatted and printed on any
%% system.  -- Nelson H. F. Beebe <beebe@math.utah.edu> [08-Jul-1991]
\setlength{\unitlength}{0.6pt}
\newcommand{\ctext}[1]{\makebox(0,0){#1}}

  \begin{picture}(600,500)(0,0)
  \small
  \put(90,330){\circle{90}}
  \put(90,335){\ctext{Interactive}}
  \put(90,315){\ctext{Editor}}

  \put(120,360){\vector(1,1){65}}
  \put(185,425){\vector(-1,-1){65}}
  \put(140,400){\ctext{I}}

  \put(185,425){\framebox(75,75){}}
  \put(225,475){\ctext{Document}}
  \put(225,445){\ctext{Source}}

  \put(260,425){\vector(1,-1){70}}
  \put(330,355){\vector(-1,1){70}}
  \put(300,400){\ctext{II}}

  \put(360,330){\circle{90}}
  \put(360,330){\ctext{Formatter}}

  \put(400,330){\vector(1,0){80}}
  \put(440,340){\ctext{IV}}

  \put(335,295){\vector(-1,-1){65}}
  \put(310,260){\ctext{II}}

  \put(480,295){\framebox(75,65){}}
  \put(520,346){\ctext{Final}}
  \put(520,321){\ctext{Index}}

  \put(130,160){\framebox(65,65){}}
  \put(160,205){\ctext{Input}}
  \put(160,190){\ctext{Output}}
  \put(160,175){\ctext{Style}}

  \put(160,160){\vector(1,-2){35}}
  \put(190,135){\ctext{III}}

  \put(255,160){\framebox(65,65){}}
  \put(290,205){\ctext{Raw}}
  \put(290,190){\ctext{Index}}
  \put(290,175){\ctext{File}}

  \put(290,160){\vector(-1,-2){35}}
  \put(290,130){\ctext{III}}

  \put(225,65){\circle{90}}
  \put(225,80){\ctext{Index}}
  \put(225,55){\ctext{Processor}}

  \put(260,65){\vector(1,0){40}}
  \put(285,55){\ctext{III}}

  \put(300,30){\framebox(110,70){}}
  \put(355,75){\ctext{Alphabetized}}
  \put(355,55){\ctext{Index}}
                      
  \put(360,100){\vector(0,1){195}}
  \put(375,190){\ctext{IV}}

  \end{picture}

\caption[{\it The sequential flow of index processing\/}.]{Circles
in the picture represent processors,
squares are documents or auxiliary files.  In Step~I, the author uses an
editor to place index commands in the document.  In Step~II, a raw index
is generated as a by-product of formatting.  In Step~III, this raw index
together with some optional style information are taken as input to
the index processor and an alphabetized version is created. 
Finally in Step~IV, the index is formatted to yield the ultimate result.}
\end{figure}
%\newpage

\section{Basic Tasks}
Index preparation is a process involving the following steps:
\renewcommand{\labelenumi}{\Roman{enumi}.}   % set the enumerated list in roman
\begin{enumerate}
  \item Placing index commands in the document source, which presumably
	comprises multiple files.  An index command takes a single
	argument: the key to be indexed.
  \item Creating a raw index file whose entries each consists
	of two arguments: the index key and the page on which the index
	command appears.
  \item Processing the raw index file.  Here, all index keys are
	sorted alphabetically.  Page numbers under the same key are merged
	and successive numbers may be collected into
	intervals (e.g., \verb|1, 2, 3, 4, 5| is replaced by \verb|1-5|).
	Subitems within an entry, if any, are properly handled.
  \item Formatting the processed index.  The result is the actual index.
\end{enumerate}
\renewcommand{\labelenumi}{\arabic{enumi}.}   % reset the list back to arabic
The idea is illustrated in Figure~1, where roman capitals I--IV marking
the edges correspond to the four steps here.
This procedure is a highly sequential,
for the input to one step depends upon the result from the previous one.

Figure~2 exemplifies a stepwise development of the process.
In {\LaTeX} and {\TeX}, all commands begin with a backslash
(\verb|\|).  Figure~2.a shows some occurrences of index
commands (\verb|\index|) in the document source, with corresponding pages
listed on the left.  The page number is not part of the source file since at
file-preparation time, it is unclear on which page a given textual material
will eventually appear.  Figure~2.a includes these numbers just to
indicate that ultimately these entries would appear on those pages.
Figure~2.b shows a raw index file generated by {\LaTeX}.
After running through the index processor, it becomes an alphabetized index
with commands specifying a particular output appearance (Figure~2.c).
The result after formatting is shown in Figure~2.d.

\begin{figure}
\begin{iexample}
&\\
{\sl Page iv\/}: & \verb|\index{alpha}| \\
{\sl Page 1\/}: & \verb|\index{alpha}| \\
{\sl Page 2\/}: & \verb|\index{alpha}| \\
{\sl Page 3\/}: & \verb|\index{alpha}| \\
{\sl Page 11\/}: & \verb#\index{alphabeta|see{beta}}# \\
{\sl Page 14\/}: & \verb|\index{alpha@{\it alpha\/}}| \\
         & \verb#\index{beta|bold}#  \\
{\sl Page 22\/}: & \verb|\index{alpha!beta!gamma}| \\
{\sl Page 38\/}: & \verb|\index{alpha!delta}| \\
 & \\
\sindex
\\
\verb|\indexentry{alpha}{iv}| \\
\verb|\indexentry{alpha}{1}| \\
\verb|\indexentry{alpha}{2}| \\
\verb|\indexentry{alpha}{3}| \\
\verb#\indexentry{alphabeta|see{beta}}{11}# \\
\verb|\indexentry{alpha@{\it alpha\/}}{14}| \\
\verb#\indexentry{beta|bold}{14}# \\
\verb|\indexentry{alpha!beta!gamma}{22}| \\
\verb|\indexentry{alpha!delta}{38}| \\
\end{iexample}
\hrule
\begin{iexample}
&\\
\verb|\begin{theindex}| &\\
&\\
\verb|\item alpha, iv, 1-3| &\\
\verb|  \subitem beta| &\\
\verb|    \subsubitem gamma, 22| &\\
\verb|  \subitem delta, 38| &\\
\verb|\item {\it alpha\/}, 14| &\\
\verb|\item alphabeta, \see{beta}{11}| &\\
&\\
\verb|\indexspace| &\\
&\\
\verb|\item beta, \bold{14}| &\\
&\\
\verb|\end{theindex}| &\\
\tindex
\\
alpha, iv, 1--3 \\
\sitem beta \\
\ssitem gamma, 22\\
\sitem delta, 38 \\
{\it alpha\/}, 14\\
alphabeta, {\it see\/} beta\\
\\
beta, {\bf 14}\\
\end{iexample}
\caption{{\it The stepwise development of index processing\/}.
This example is specified in {\LaTeX}.
(a) {\sl Top Left\/}: Occurrences of index commands in the document source.
Note that page numbers are unknown at the time of input when a source-based
formatter like {\LaTeX} is used.  Page numbers are included here simply to
illustrate where each instance will occur. 
        (b) {\sl Top Right\/}: raw index file generated by {\LaTeX}.
	(c) {\sl Bottom Left\/}: alphabetized index file.
        (d) {\sl Bottom Right\/}: formatted final index.}
\end{figure}

Based on the example given in Figure~2, these four steps are explained
below, where Steps~I and III are further expanded in subsequent sections.
Issues involved in Steps~II and IV are less complex
and are covered only in this section.

\subsection{Placing Index Commands}
Step~I deals with placing index commands in the document.
In a source-language based environment, the commands can simply be inserted
in the document source with a text editor.
They will be utilized by the formatter in generating
raw index entries (Step~II), but will contribute nothing to the output
appearance as far as the corresponding pages are concerned.

In a direct-manipulation system, index commands cannot be entered directly in
the document under manipulation.  A possible solution is to put them
in ``shadow pages'' instead of the document output representation.
A shadow document is the original document plus
special tags that, among other things,
mark logical objects like comments, bibliographical citations,
cross references, indexes, etc.
These tags are essential to document composition but do not correspond to
physical appearance in their original forms.  Upon request, the corresponding
markers of these tags can be displayed along with the original document
for editing purposes.  For the user's visual cue, each type of tags can be
represented by a different marker symbol.  
Normally for each tag entered in the
document, an embedded annotation can be specified.
An additional window can be created to show the associated annotation
if necessary.  This shadow document approach is widely adopted by
direct-manipulation document development or desktop publishing systems
such as Xerox {\ST}~[14], {\FM}~[15],
{\WD}~[16], and {\VP}~[17].

The primary issue in step I for both paradigms
is whether or not a systematic mechanism
can be derived for the entering of index commands or tags.
Details of a general model that accomplishes this task are given below.


\subsection{Generating the Raw Index}
Step~II concerns attaching the current page number to each index
command placed in the document.  The command used in the generated
raw index file may be renamed (in our example, it is changed
from \verb|\index| to \verb|\indexentry|).
The entries are in the exact order in which they appear in the document source.
Thus, as long as the current page number is accessible, be it
source-language or direct-manipulation based, generating raw index entries
is relatively straightforward.

There are minor differences between the two paradigms in this step,
however.  The generation of raw index entries in a source-language based
system, like formatting itself, is by and large a ``batch job''.
In a direct-manipulation editor, it is easier to maintain the list of raw index entries
incrementally because the document being manipulated is always formatted
so the page number is always current.

\subsection{Index Processing}
Processing raw index entries raises several issues.
Some high-level issues are described below with references to the example given
in Figure~2; how these tasks can be realized is detailed in the next section.

{\it Permutation\/}.  Index entries are sorted alphabetically (Figure~2.c).
The index processor must differentiate among different types of keys
such as strings, numbers, and special symbols.
Upper and lower case letters should be distinguished.
Furthermore, it may be necessary to handle roman, arabic, and
alphabetic page numbers.

{\it Merging\/}.  Different page numbers corresponding to the same
index key are merged into one list.  Also, three or more
successive page numbers are abbreviated as a range (as in the
case of \verb|alpha, iv, 1-3|, Figure~2.c).  If citations on successive
pages are logically distinct, good indexing practice suggests that they
should not be represented by a range.  Our system allows user control
of this practice.

{\it Subindexing\/}.  Multi-level indexing is supported.
Here, entries sharing a common prefix are grouped together
under the same prefix key.
The special symbol `\verb|!|' serves as the level operator
in the example (Figure~2.a and 2.b).  Primary indexes are 
converted to first level items (the \verb|\item| entries in Figure~2.c)
while subindexes are converted to lower level items
(e.g., \verb|\subitem| or \verb|\subsubitem| entries in Figure~2.c).

{\it Actual Field\/}.  The distinction between a {\it sort key\/} 
and its	{\it actual field\/} is made explicit.
Sort keys are used in comparison while their actual counterparts are 
what end up being placed in the printed index.
In the example, the `\verb|@|'
sign is used as the actual field operator, which means its
preceding string is the sort key and its succeeding string
is the actual key (e.g., the \verb|\index{alpha@{\it alpha\/}}|
in Figure~2.a).
The same sort key with and without an actual field
are treated as two separate entries (cf. \verb|alpha| and {\it alpha\/}
in the example).  If a key contains no actual operator,
it is used as both the sort field and the actual field.

The separation of a sort key from its actual field makes entry sorting
much easier.  If there were only one field,
the comparison routine would have to ignore syntactic sugar related to
output appearance and compare only the ``real'' keywords.
For instance, in \verb|{\it alpha\/}|, the program has to ignore the font
setting command \verb|\it|, the italic correction command \verb|\/|,
and the scope delimiters \verb|{}|.
In general, it is impossible to know
all the patterns that the index processor should ignore,
but with the separation of the fields, the sort key is used as a verbatim string
in comparison; any special effect can be achieved via the actual field.

{\it Page Encapsulation\/}.  Page numbers can be encapsulated
using the `\verb#|#' operator.  In the example,
page 14 on which \verb|\index{beta}| occurs is set in boldface, as
represented by the command \verb|\bold|.  The ability to set
page numbers in different fonts allows the index to convey
more information about whatever is being indexed.  For instance,
the place where a definition occurs can be set in one font, its
primary example in a second, and others in a third.

{\it Cross Referencing\/}.  Some index entries make references to
others.  In our example the \verb|alphabeta| entry is a reference
to \verb|beta|, as indicated by the {\it see\/} phrase.
The page number, however, disappears after formatting (Step~IV),
hence it is immaterial where index commands dealing with cross references
like {\it see\/} occur in the document.
This is a special case of page encapsulation (\verb|see{beta}| appears
after the `\verb#|#' operator).
Variations like {\it see also\/}, which gives page numbers as well
as references to other entries, work similarly.

{\it Input/Output Style\/}.  In order to be
formatter- and format-independent, the index processor must be able
to handle a variety of formats.  There are two reasons for considering
this independence issue in the input side:
Raw index files generated by systems
other than {\LaTeX} may not comply to the default format, and
the basic framework established for processing indexes can also be
used to process other objects of similar nature (e.g., glossaries).
But these other objects will certainly have a different keyword
(e.g., \verb|\glossaryentry| as opposed to \verb|\indexentry|)
in the very least.
Similarly in the output side the index style
may vary for different systems.  Even within the same formatting
system, the index may have to look differently under different
publishing requirements.   In other words, there must be a way to inform
the processor of the input format and the output style.


\subsection{Index Formatting}
Two key issues in this last step are
support for multiple styles and formatting independence.
First, the formatting style macros used in Step~III output
must be defined.  In our example,
the global environment (\verb|\begin{theindex}...\end{theindex}|) tells
{\LaTeX} to use a two-column page layout.  Each \verb|\item| is
left justified against the column margin and each \verb|\subitem| is indented
by 20 points, \verb|\subsubitem| by 30, etc.
There is a vertical space bound to
\verb|\indexspace| inserted before the beginning of a new letter
(e.g., before \verb|beta|).

The formatting independence problem refers to whether or not the final index
can be formatted independently of the entire document.
Indexing is typically the last step of document preparation, and
is attempted only when the entire document is finalized.
It is desirable to be able to generate the index without
reformatting the entire document.  To be able to format
the index separately, the global context must be known, which is made possible
by the extensible style facility in our design.
One can redefine \verb|preamble| and \verb|postamble| to
invoke a style consistent with the original document.

The other information needed to perform effective separate formatting
is the starting page number for the index.
Some styles require that the index
start on an even or odd page number.
In either case, there must be provisions for including the correct
starting page number in the pre-formatted version of index.


\section{Index Processing}
The index processor performs the tasks indicated above ---
permutation, page number merging, subindexing, style handling, and other
special effects.  In order to achieve format and formatter independence,
the index processor must be able to accept raw index
terms designated by different keywords and delimiters.
Likewise, it must be able to generate output in a specific style
so that the result can be processed by the corresponding formatter.
The intended tasks can be performed in multiple passes:
First the input format file and output style file are scanned and analyzed.
Entries in the input file are then processed.  Next, all legal entries are
sorted.  Finally, the output index is generated in the last pass.
The remainder of this section discusses the essential attributes
for input formats and output styles and points out relevant issues
for sorting and generating the entries.


\begin{table}
\begin{center}
{\small
\begin{tabular}{l|c|c|l}
\hline
\hd{specifier} & \hd{attribute} & \hd{default} & \hdr{meaning} \\
\hline\hline
\verb|keyword|  & {\it string\/} & \verb|"\\indexentry"| &
index command\\
\hline
\verb|arg_open|  & {\it char\/} & \verb|'{'| &
argument opening delimiter\\
\hline
\verb|arg_close|  & {\it char\/} & \verb|'}'| &
argument closing delimiter\\
\hline
\verb|range_open|  & {\it char\/} & \verb|'('| &
page range opening delimiter\\
\hline
\verb|range_close|  & {\it char\/} & \verb|')'| &
page range closing delimiter\\
\hline
\verb|level|  & {\it char\/} & \verb|'!'| &
index level delimiter\\
\hline
\verb|actual|  & {\it char\/} & \verb|'@'| &
actual key designator\\
\hline
\verb|encap|  & {\it char\/} & \verb#'|'# &
page number encapsulator\\
\hline
\verb|quote|  & {\it char\/} & \verb|'"'| &
quote symbol\\
\hline
\verb|escape|  & {\it char\/} & \verb|'\\'| &
symbol that escapes \verb|quote|\\
\hline
\verb|page_compositor| &  {\it string\/} & \verb|"-"| &
composite page delimiter\\
\hline
\end{tabular}
}
\end{center}
\caption{Input format parameters.}
\end{table}

\subsection{Input Format}
Table~1 is a summary of the input format that consists of a list
of \verb|<|{\it specifier\/}, {\it attribute\/}\verb|>| pairs.
These attributes
are the essential tokens and delimiters needed in scanning the
input index file.  Default string constants are enclosed in
double quotes (\verb|"..."|) and character constants are
in single quotes (\verb|'x'|).  The user can override
the default value by specifying a specifier and a new attribute
in the format file or property sheet.
The attribute of \verb|keyword| is self-explanatory;
\verb|arg_open| and \verb|arg_close| denote the argument opening and
closing delimiters, respectively.
The meanings of special operators such as \verb|level|,
\verb|actual|, and \verb|encap| are described above.

The two range delimiters \verb|range_open| and \verb|range_close| are
used with the \verb|encap| operator.
When \verb|range_open| immediately follows \verb|encap|
(i.e., \verb#\index{...|(...}#),
it tells the index processor that an explicit range is starting.
Conversely \verb|range_close| signals the closing of a range.
In our design, three or more successive page numbers are abbreviated as
a range implicitly.  This {\it implicit\/} range formation can be turned
off if an indexed term represents logically distinct concepts in different
pages.  When the implicit range is disabled, {\it explicit\/} page ranges
can be enforced by using the two range delimiters
\verb|range_open| and \verb|range_close|.
Therefore, it is possible to index an entire section or a large piece
of text related to a certain concept without having to insert an index
command in every single page.

The \verb|quote| operator is used to escape symbols.
Thus \verb|\index{foo"@goo}| means a sort key of \verb|foo@goo| 
rather than a sort key of \verb|foo"| and an actual key of \verb|goo|.
As an exception, \verb|quote|, when preceded by \verb|escape| (i.e.
\verb|\index{...\"...}|), does not escape its succeeding letter.
This special case is included because \verb|\"| is the umlaut
command in {\TeX}.  Requiring \verb|quote| itself to be quoted
in this case (i.e. \verb|\""|) is feasible but somewhat awkward;
\verb|quote| and \verb|escape| must be distinct.

A page number can be a composite of one or more fields
separated by the delimiter bound to \verb|page_compositor|
(e.g., II-12 for page 12 of Chapter II).
This attribute allows the lexical analyzer to separate 
these fields, simplifying the sorting of page numbers.

\subsection{Output Style}
Table~2 summarizes the output style parameters.  Again, it is a list
of \verb|<|{\it specifier\/}, {\it attribute\/}\verb|>| pairs.
In the default column, `\verb|\n|' and `\verb|\t|' denote a new line
and a tab, respectively.  These parameters can be further divided into
the following groups:

\begin{table}
\begin{center}
{\small
\begin{tabular}{l|c|l|l}
\hline
\hd{specifier} & \hd{attribute} & \hd{default} & \hdr{meaning} \\
\hline\hline
\verb|preamble| &  {\it string\/} & \verb|"\\begin{theindex}\n"| &
index preamble\\
\hline
\verb|postamble| &  {\it string\/} & \verb|"\n\n\\end{theindex}\n"| &
index postamble\\
\hline
\verb|setpage_prefix| &  {\it string\/} & \verb|"\n  \\setcounter{page}{"| &
page setting command prefix\\
\hline
\verb|setpage_suffix| &  {\it string\/} & \verb|"}\n"| &
page setting command suffix\\
\hline
\verb|group_skip| &  {\it string\/} & \verb|"\n\n  \\indexspace\n"| &
intergroup vertical space\\
\hline
\verb|lethead_prefix| &  {\it string\/} & \verb|""| &
new letter heading prefix\\
\hline
\verb|lethead_suffix| &  {\it string\/} & \verb|""| &
new letter heading suffix\\
\hline
\verb|lethead_flag| &  {\it number\/} & \verb|0| &
flag designating new letter\\
\hline
\verb|item_0| &  {\it string\/} & \verb|"\n  \\item "| &
level 0 item separator\\
\hline
\verb|item_1| &  {\it string\/} & \verb|"\n    \\subitem "| &
level 1 item separator\\
\hline
\verb|item_2| &  {\it string\/} & \verb|"\n      \\subsubitem "| &
level 2 item separator\\
\hline
\verb|item_01| &  {\it string\/} & \verb|"\n    \\subitem "| &
levels 0/1 separator\\
\hline
\verb|item_x1| &  {\it string\/} & \verb|"\n    \\subitem "| &
levels x/1 separator\\
\hline
\verb|item_12| &  {\it string\/} & \verb|"\n      \\subsubitem "| &
levels 1/2 separator\\
\hline
\verb|item_x2| &  {\it string\/} & \verb|"\n      \\subsubitem "| &
levels x/2 separator\\
\hline
\verb|delim_0| &  {\it string\/} & \verb|", "| &
level 0 key/page delimiter\\
\hline
\verb|delim_1| &  {\it string\/} & \verb|", "| &
level 1 key/page delimiter\\
\hline
\verb|delim_2| &  {\it string\/} & \verb|", "| &
level 2 key/page delimiter\\
\hline
\verb|delim_n| &  {\it string\/} & \verb|", "| &
inter page number delimiter\\
\hline
\verb|delim_r| &  {\it string\/} & \verb|"--"| &
page range designator\\
\hline
\verb|encap_prefix| &  {\it string\/} & \verb|"\\"| &
page encapsulator prefix\\
\hline
\verb|encap_infix| &  {\it string\/} & \verb|"{"| &
page encapsulator infix\\
\hline
\verb|encap_suffix| &  {\it string\/} & \verb|"}".| &
page encapsulator suffix\\
\hline
\verb|page_precedence| &  {\it string\/} & \verb|"rnaRA"| &
page type precedence\\
\hline
\verb|line_max| &  {\it number\/} & \verb|72| &
maximum line length\\
\hline
\verb|indent_space| &  {\it string\/} & \verb|"\t\t"| &
indentation for wrapped lines\\
\hline
\verb|indent_length| &  {\it number\/} & \verb|16| &
length of indentation\\
\hline
\end{tabular}
}
\end{center}
\caption{Output style parameters.}
\end{table}

{\it Context\/}.  Together, \verb|preamble| and \verb|postamble| 
	define the context in which the index is to be formatted.

{\it Starting Page\/}.  The starting page number can either be supplied
	by the user or retrieved automatically from the document transcript.
	In either case, this number can be enclosed with \verb|setpage_prefix|
	and \verb|setpage_suffix| to yield a page number initializing command.

{\it New Group/Letter\/}.  The string bound to \verb|group_skip|
denotes the extra vertical space needed when a group is started.
For a group beginning with a different letter,
the parameters \verb|lethead_prefix| and \verb|lethead_suffix| (both with
a default nil string) denote the group heading.
The flag \verb|lethead_flag| has a default value of 0, which
means other than \verb|group_skip| nothing else will be inserted before
the group.
On the other hand, if this flag is positive, the strings bound
to \verb|lethead_prefix| and \verb|lethead_suffix| will be inserted with
an instance of the new letter in uppercase in between.  Similarly,
a lowercase letter will be inserted if the flag is negative.

{\it Entry Separators\/}.  This group includes everything with the
\verb|item_| prefix.  First, \verb|item_|$i$
denotes the command and indentation to be inserted when
a key is started from a level greater than or equal to $i$.
Second, \verb|item_|$ij$ has a similar meaning, but with
$i = j-1$.  Finally, the two \verb|item_x|$j$'s
are included to handle the situation where the parent level has no
page numbers.  Some styles require cases like these to be
different from those with page numbers.

Table~2 depicts a system that supports three levels of
subindexing.  In general, suppose $n$ is the number of index
levels supported, there will be $n$ \verb|item_|$i$'s 
($0 \leq i \leq n-1$), $(n-1)$ \verb|item_|$ij$'s
($1 \leq j \leq n-1,  i = j-1$), and
$(n-1)$ \verb|item_x|$j$'s ($1 \leq j \leq n-1$).

{\it Page Delimiters\/}.  Each level has a key/page delimiter
that defines what is to be inserted between a key and its first
page number.  The inter-page delimiter is specified by \verb|delim_n|,
while the range designator is given by \verb|delim_r|.

{\it Page Encapsulator\/}.  The attributes of
\verb|encap_prefix|, \verb|encap_infix|, and \verb|encap_suffix|
form what is to be placed into the output
 when an encapsulator is specified
for a certain entry.  Suppose \verb|foo| is the specified encapsulator
and \verb|N| is the page number, the output sequence is
\begin{verbatim}
	     encap_prefix  foo  encap_infix  N  encap_suffix
\end{verbatim}

{\it Page Precedence\/}.  Five different types of numerals are
supported by most systems for page numbering.  These are
lowercase roman (\verb|r|), numeric or arabic (\verb|n|),
lowercase alphabetic (\verb|a|), uppercase roman (\verb|R|),
and uppercase alphabetic (\verb|A|).  The string bound to
\verb|page_precedence| (default \verb|"rnaRA"|)
specifies their order.

{\it Line Wrapping\/}.  In the output index file, 
the merged list of page numbers	can be wrapped in multiple lines,
if it is longer than \verb|line_max|.  The newly wrapped
line is indented by \verb|indent_space| whose length is
\verb|indent_length|.  This artificial line wrapping does not
make any difference in formatting, but does provide increased
readability for the pre-formatted final index.
This feature may seem somewhat trivial at first glance, but if no
formatters are involved whatsoever, the readability of the
verbatim output index is important.

\subsection{Sorting Entries}
Entries in the raw index file are sorted primarily on their index keys
and secondarily on their page numbers.
Index keys are sorted first; within the same index key,
page numbers are sorted numerically.
Sort keys and numeric page numbers are used in the comparison, while
actual keys and literal page fields are entered into the
resulting index.  In our design,
a complete index key is an aggregate of one or more sort keys plus the same
or a smaller number of actual keys.  The comparison is based on
sort keys, but if two aggregates have identical sort fields and
page numbers, the actual keys can be used to distinguish their order.

Index keys can be categorized into the following groups:
{\it strings\/}, {\it numbers\/}, and {\it symbols\/}.
A string is a pattern whose leading character is a letter in the alphabet.
A number is a pattern consisting of all digits.  A symbol is a pattern
beginning with a character not in the union of the English alphabet and
arabic digits or starting with a digit but mixed with non-digits.
Members of the same group should appear in sequence.
Hence there are two issues concerning ordering:
one deals with entries within a group; the other is
the global precedence among the three groups in question.
Details of sorting index keys can be found in Reference~[18].

There are three basic types of numerals for page numbers: {\it roman\/},
{\it alphabetic\/}, and {\it arabic\/}.
The sorting of arbitrary combinations of these three types of numerals
(e.g., 112, iv, II-12, A.1.3, etc.)
must be based on their numeric values and relative precedence.
The attribute of \verb|page_precedence| in Table~2, for instance, specifies
the precedence.  Again, details of sorting page numbers can be found in
Reference~[18].

\newpage
\subsection{Creating Output Index Entries}
Once all input entries are sorted, the output index file can be
created.  First the attribute bound to \verb|preamble| is placed into
the output file, followed by the string
\begin{quote}
  \verb|setpage_prefix |{\sl N\/}\verb| setpage_suffix|,
\end{quote}
provided {\sl N\/} is the starting page number and
such a setting is requested.  Next, each entry in the sorted list
is processed in order.  Finally, the attribute bound to \verb|postamble|
is appended at the end of the output file.
The algorithm for generating each entry with appropriate formatter-specific
commands or delimiters (i.e. \verb|item_|$i$'s, \verb|delim_|$i$'s, etc.)
is described in Reference~[18].

\section{Placing Index Commands}
This section discusses a simple framework for placing index commands
in a document.  It assumes an interactive editor is available with
{\it string search\/}, which positions the cursor to a specified pattern, and
{\it query-insert\/}, which displays a menu of
options and, upon the user's selection, inserts
a specified key, together with other constant strings (e.g.,
the index command and its argument delimiters).

We implemented this framework on top of
GNU {\E}~[19] as part of an interactive environment for
composing {\TeX}-based documents~[5].
The underlying model applies not just to conventional text
editors, but to direct-manipulation systems as well.

\subsection{Basic Framework}
The basic framework is very simple.  All the author needs is
to specify a {\it pattern\/} and a {\it key\/}.
The editor then finds the pattern,
issues a menu of options and inserts the index command along with the key
as its argument upon the user's request.
In our example, suppose both {\it pattern\/} and {\it key\/} are
\verb|alpha|, then the inserted string after an instance
of \verb|alpha| in the document is \verb|\index{alpha}|.
This insertion will be visible in a source-language based system and will be
invisible for a direct-manipulation system (or visible as hidden text
for its shadow pages).

Before the actual insertion is made, it is desirable to make a confirmation
request that presents a menu of options, of which
{\it confirm\/} and {\it ignore\/} are the most obvious ones.
Thus for each instance of the pattern found, the user can decide if
it is to be indexed.

Representing patterns as regular expressions gives significantly more power
to this query-insert operation.  The same key can represent a complicated
string of a basic pattern, its capitalized form, its acronym, and other
abbreviations.  For instance, the following patterns may all be indexed
by the key \verb|UCB|,
\begin{verbatim}
        University of California, Berkeley
        Berkeley
        berkeley
        UCB
\end{verbatim}

As a special case of this \verb|<|{\it key\/}, {\it pattern\/}\verb|>|
setup, one can use words in the neighborhood of current cursor position
as the implicit value for both the key and the pattern.
Some editors allow the use of special characters to delimit word boundaries.
These characters can be used in searching to reduce on the number of
``false drops''.  For example, one can position the cursor
after the desired pattern and with one editor command (typically
in two or three key strokes), an index entry will
be inserted with the preceding word (or words) as the implicit key.
The advantage of this facility is that there is no need to type
the key-pattern pair.
The same idea also applies to a region of text, which is a piece of
continuous text in the document.  In {\E}, a region is
everything between a marker and the current cursor position.
More generally, the implicit operand can be the {\it current selection\/},
in which case the bounding positions of the selected text
are not necessarily the insertion point.

In our system, there is also a special facility to index every author name
that appears in the bibliography or references section of a document.
This feature involves skipping citation entries without an author field and
for each author name found, issuing a query-insert prompt similar to
the normal case.  Instead of entering a name directly as the index term,
it is better to display it in the form of last name followed by first
and middle names for confirmation, as in
\begin{quote}
\verb|Confirm: Knuth, Donald E.|
\end{quote}
This reordering yields last names as primary sort keys.
Our name separation heuristic does not always
work for multi-word last names.  The confirmation prompt
allows the user to correct it before insertion.

\subsection{Key-Pattern List}
A collection of these \verb|<|{\it key\/}, {\it pattern\/}\verb|>| pairs
can be compiled in a list.  A global function can then be invoked to
process each pair for the entire document or parts of it.
This list can be created off-line by the user, or automatically
in an incremental fashion as the
user confirms new index insertions in an interactive session.  The pattern
matching mechanism must be able to recognize and skip instances already
indexed so that unnecessary repetitions are avoided.
In our system, this key-pattern list is a per document list.
If a document includes multiple files, the global function processes
each of them according to the preorder traversal of the file inclusion tree.


\subsection{Indexing Menu}
For each instance of the pattern found, a menu of options such as the
following may be presented.
%\newpage
\begin{itemize}
  \item {\it Confirm\/}.
  \item {\it Ignore\/}.
  \item {\it Key-Pattern List\/}.  Add the current
	\verb|<|{\it key\/}, {\it pattern\/}\verb|>| pair to the list
	associated with the current document.
  \item {\it Index Level\/}.  Prompt the user for an index prefix.
	The \verb|level| operator (`\verb|!|'), if not given at the end
	of the specified string, should be inserted automatically between
	the prefix and the current key.
  \item {\it Actual Field\/}.  Prompt the user for the actual field
	corresponding to the current (sort) key.  The \verb|actual|
	operator (`\verb|@|') should be automatically inserted in between.
  \item {\it Page Encapsulator\/}.  Prompt the user for the page
	number encapsulator.  The \verb|encap| operator (`\verb#|#'),
	if not given, should be attached in front of the specified string.
	Encapsulators corresponding to popular fonts such as bold, italic,
	and slanted, or to cross references like {\it see\/} and
	{\it see also\/} can be implemented as a submenu.
\end{itemize}

\subsection{Extended Framework}
A typical scenario for placing index commands under the extended framework
is as follows.  There are two query-insert modes: 
one based on a single key-pattern pair and the other
on multiple key-pattern pairs.
In the former mode, the user specifies a pattern and a key, and for
every instance of the pattern found, decides whether to insert the
index command with the specified key, or a variant of it (i.e., a combination
of \verb|level|, \verb|actual|, and \verb|encap|).
In the latter mode, there is a collection of entries that may be represented
as a long list for an entire document, or as a short list for a section.
Each key-pattern pair in the list is processed as in the former case.
Provisions are also available to help construct the list.

With such powerful mechanisms available, there is a tendency to ``over index''
because it is so easy to be comprehensive.
In some cases, ``less may be more''.  Therefore, one should be extremely
careful in confirming the insertion of {\it key\/}
to avoid false drops picked up by the search mechanism.


\section{Direct Manipulation}
Although our implementation of these facilities was for a source-based system,
a number of direct-manipulation systems do support a similar indexing facility.
Four systems are described briefly to indicate
the applicability of the principles depicted in the previous sections.
Also examined is a more elaborate on-line
indexing facility made possible using the electronic media.


\subsection{Indexing under Direct Manipulation}
In Xerox PARC's {\CD} environment~[20],
the {\TG} editor supports an
application called {\sl IndexTool\/}~[21]
that automatically prepares multiple
multi-level indexes (general index, author index, etc.) with cross
references ({\it see\/} and {\it see also\/})
and with substitution phrases (index the
phrase ``data structures'' under ``data structure'' to handle these
automatically).  {\sl IndexTool\/} takes a selection and creates an
index entry attached to the document over the selection range.  Index
entries can be edited in a separate tool to permit creating the cross
references and substitution text.  {\TG} also has a regular expression search
capability via {\sl EditTool\/}, which also
permits a wide range of search and replace operations.

In {\FM} 1.0~[15], index tags can be placed and edited
using a combination of the {\sl Markers\/} and the {\sl Search\/} tools.
{\sl Markers\/} allows one to specify invisible tags such as
subjects, comments, and of course, index entries.
For each marker, an associated annotation can be specified.
In the indexing case, the annotation is the key to appear in the final index.
Attributes of page encapsulation discussed previously can also be specified
in the {\sl Markers\/} window.
These attributes include explicit page range, fonts, and an option to disable
page numbers so that cross references like {\it see\/} can be realized.
{\FM}'s {\sl Search\/} can be used to locate the desired pattern in plain
or regular expressions.  An invisible character \verb|\m|
can be specified in {\sl Search\/} to identify each occurrence
of index markers in the document.
Whenever a marker is found, the corresponding
annotation is displayed in the {\sl Markers\/} window.
The annotated text can incorporate special symbols to yield
multi-level indexing and actual field substitution similar to the ones
described in above.  A processor called
{\sl fmBook\/} can be executed off-line to collect index markers,
sort them, and finally generate a formatted index whose style
is customizable via system supplied property sheets.

In the Macintosh version of {\WD} 3.0~[16], an index command is
designated by the ``index code'' \verb|.i.|  The text between \verb|.i.|
and an ``end-of-entry code'', such as a semicolon, is regarded
as the index key.  A colon (\verb|:|) in the entry text acts as the index
level operator.  The output appearance can be refined by using
variants of the index code.  For instance,
\verb|.iB.| and \verb|.iI.| set the page number in boldface and
italic fonts, respectively, \verb|.i(.| and \verb|.i).| create
an explicit page range, etc.  Index entries are ``compiled''
into the actual index that appears at the end of the document.
Before that takes place, the system must be notified that these index codes
are {\it hidden text\/}.  An option in the preference sheet
can be set to display hidden text embedded in the document.
Hidden text must be ``hidden'' when index entries are being compiled;
otherwise page number computation will be incorrect.
There are no special tools for entering index codes.
The {\sl find\/} and {\sl change\/} commands in the {\sl Search Menu\/}
do not support regular expressions.  There are no query-insert
modes in the search mechanism.  Although abbreviations can be registered
as glossary entries, a great many keystrokes are still required in placing
a single index entry.

In {\VP} 1.1~[17], a desktop publishing system
for documents prepared on any of several word processing
programs on the IBM PC, index entries can be placed in the document by
invoking a dialogue box at the desired position.
An alternative is to use a word processor to enter a markup tag such as
\begin{quote}
\verb|<$I<I>alpha<L>[alpha];beta>|
\end{quote}
where \verb|<$I|$\cdots$\verb|>| marks $\cdots$ as an index term,
\verb|<I>alpha<L>| is the actual key in italic,
\verb|[|$\cdots$\verb|]| designates $\cdots$ as the corresponding sort key,
and the semicolon is the index level operator.
It supports two levels of subindexing as well as {\it see\/} and
{\it see also\/}.  Index terms are hidden text; the word {\tt Index}
will be displayed by its {\it Current Selection Indicator\/} when
the cursor is placed at a location where an index term is anchored.
The output index style can be specified by changing attributes in
a property sheet called {\tt GENERATE INDEX}.  The collection of
attributes is a proper subset of Table~2.  Index processing is executed
as a batch job similar to {\sl fmBook\/} in {\FM}.  Searching is unavailable
in {\VP}, therefore an automated index placing subsystem is not possible.


\subsection{Dynamic Indexing and Hypertext}
Our proposed implementation of an indexing facility under direct manipulation
and the four real systems discussed previously all operate
in a multi-pass fashion, much the same as a source-language based system.
However, based on its interactive nature, a direct-manipulation document
editor can be constructed with an indexing subsystem that follows the same
central ideas, but with an user interface more closely related to the
direct-manipulation paradigm.

The ``directness'' may be observed in the following scenario.
The author specifies input/output styles by selecting options available
in system supplied property sheets.  For each key selected
in the document, its corresponding entry in the index is immediately
displayed, along with entries already entered.
Sorting is done as the entry is generated for the output.
Consequently, the internal structure for these entries can no longer be
an array that is suitable for fast sorting algorithms.
Instead, a balanced tree may be the preferred structure.
Insertion and deletion reorganize the tree automatically and
a traversal of the tree yields the correct ordering.
A link between a page number in the index entry and its corresponding
actual page is required so that the index needs not be regenerated
when the document is reformatted.  In other words, when a page
number changes due to reformatting, all instances of that page in the index
change automatically.  

One possible extension to this model is the ability to point at an entry
in the index and
have the corresponding page located automatically with a keyword highlighted.
Thus indexing becomes a dynamic activity from the reader's point of view.
This feature typifies the power of ``dynamics'' that an electronic document
environment is able to offer, which does not exist in traditional
static printed material.  In addition to dynamic indexing,
one can do such hypertext operations as navigation, filtering, summarizing,
etc.~[22] effectively based on markup tags, embedded annotations, and
links among compound objects in the document.


\section{Evaluation}
\subsection{Index Placing Subsystem}
An important aspect of our system is the framework for
placing index commands in the document.
This task has been performed traditionally in an ad hoc fashion.
It is unclear how placement is done in the 
UNIX {\TF} environment.  Reference~[23] does not describe
any automated assistance for the author.

Entering index codes in {\WD} is awkward because
its search mechanism lacks full regular expression support
and query-insert mode is unavailable.
To mark one piece of text as an index entry, an ordinary session
requires 8 mouse clicks and 4 keystrokes.  Even with accelerated keyboard
abbreviations, it still takes 2 mouse clicks and 8 keystrokes to mark a single
index entry.  The premise, however, is that the index pattern has been
located and that it is identical to the index key.
Locating the pattern may involve extra mouse clicks and keystrokes.
Moreover, if the index key and the search pattern are distinct, more
keystrokes would be necessary to enter the text for the index key.
This situation happens to the marking of each and every instance
of index entries.  No global scheme is available in {\WD}.

{\VP} does not support any searching mechanism; it assumes
searching is performed in a front-end word processor.
Therefore, systematic index placing in the stand-alone {\VP} is difficult.

The situation in {\FM} is somewhat better because of its more powerful
keyboard macro registration capability and the support for regular
expression search.  In {\FM}, a specific combination
of tools can be used to enter index tags at desired places.
Operations can be recorded as keyboard macros so that repetitions
can be done by a single keystroke (the invocation of the keyboard macro).
The problem is that a new keyboard macro has to be defined for each
key-pattern pair.  Furthermore, it
lacks the systematic global scheme available in our extended
framework to make the whole process efficient.

In our system, it takes only 1 to 3 keystrokes to mark an index entry.
Also available is a global scheme for marking index entries in the entire
document that may span over multiple files.  In the single key-pattern
case, the same key can be inserted at a variety of places
described by a single regular expression.  Patterns already indexed
are skipped automatically.  A number of options are
available upon each occurrence of the pattern.  Thus, marking each instance
takes just one keystroke to confirm and it works uniformly and continuously
throughout the entire document.  This can be expanded to a scheme of
multiple key-pattern pairs that works iteratively on a list of 
different index entries.
In our system,
a significant amount of time is saved not just in typing {\it per se\/}, but
in the user's mental reaction time due to the side effect of
unautomated discrete repetitions as in {\FM}, {\WD}, and {\VP}.

In the production of an index for a book~[6], our {\E} Lisp
implementation of the index placing subsystem has proved very useful
and effective.  The provision for multi-pair
\verb|<|{\it key\/}, {\it pattern\/}\verb|>| query-insert has been the most
time-saving operation in our experience.
With minor modifications, the same facility should work on
placing index commands for other formatting languages like
{\TF} and {\SB}.  Adapting it to direct-manipulation environments is more
involved, but given proper editor programming power,
the basic principles discussed above should apply.

\subsection{Index Processor}
The other major portion of our work is
a complete index processor implementation.
The resulting processor
is a single C program called {\MI}.  Actually, {\MI}
had several predecessors all written as UNIX shell scripts with
embedded \verb|sed|~[24] and \verb|awk|~[25] code.
We quickly became unsatisfied with them for various reasons.
One of the concerns was efficiency.  
Interpreted languages such as \verb|sed| and \verb|awk| are satisfactory for
rapid prototyping, but they are slow.
In our experience, {\MI} was able to process
a book index of 3,300 entries in less
than 50 seconds of user time plus an extra 5\% of system time on
a client node of SUN 3/50 (MC68020 at 12.5 MHz).
This is at least an order of magnitude faster than
using its most recent \verb|sed|/\verb|awk| predecessor, which has only
half the features of {\MI}.

%The efficiency issue is not that significant since normally
%an index processor is not needed until the document is at its
%final stage.  Most of the time is spent in placing index commands
%in the document.  There are relatively few invocations to
%process raw index entries.  A speedup of over an order of magnitude, however,
%can be significant.

The decision to do a C implementation was made because of C's dynamic
storage allocation and data structures.
Since we wanted as general a solution as possible, it would be difficult,
if not impossible, to implement all the desired features using \verb|sed/awk|.
For instance, in \verb|awk| a dynamically linked list of index entries is
impossible, and the style handling mechanism with a comprehensive error
processing facility is difficult to realize.  Originally developed
in UNIX C with portability in mind, {\MI} also runs on VAX/VMS,
TOPS-20, MS/DOS systems with a few trivial changes.
This portability would have
been very difficult for an implementation in UNIX shell scripts.
Furthermore, {\MI} has more features and is more extensible than tools like
{\sl Idx{\TeX}\/}~[26], a {\LaTeX}-specific index processor
that runs only on VAX/VMS systems.

Our approach is the direct opposite of that taken by \verb|make.index|,
a host of indexing tools~[23] built for {\TF}.
As part of the UNIX {\TF} document
preparation environment, these tools are canonical examples of
the pipelining model.  They are a collection of small \verb|awk|
programs working together in a very long pipeline.
The claim is that by breaking the system down into small pieces, it is
easier for the user to adapt the package to various situations
not possible to envision in advance.  Their approach, therefore, seems
to follow the ``do it yourself'' metaphor whereby you get a toolkit
and assemble the final product yourself.  The basic package covers
only those features used by its authors.  A third party user has to
modify their \verb|awk| code if something different is desired.

The design of {\MI} is quite different.
Our intention is to build a complete system by analyzing
the tasks involved in index processing.  Parameters are derived
to form a table-driven style handling facility.
Formatter and format independence is achieved by
simple style specification.  All the underlying
data structures and processing mechanisms are transparent.
Yet it is robust enough to handle different precedence schemes,
ordering rules, etc.
To use the system, the user needs not deal with the processing
details.  If the default is inappropriate,
the only adaptation required is a table specification for the style facility.

For instance, by assigning the output index header \verb|index.head|
defined in \verb|make.index| to our \verb|preamble| and
other related commands to \verb|item_|$i$'s, it is easy to produce
an alphabetized index in {\TF} format from a raw index generated by {\TF}.
The same technique applies to other formatting systems.
{\SB}, for example, has an indexing subsystem
that supports a subset of the functionality described above.
By adapting its raw index file format as input style and
its output appearance commands as output style, its indexing capability
can be readily expanded using {\MI}.  In both cases,
there is no need to modify the original code used to generate
the raw index (Step II), nor is it necessary to modify {\MI} itself.
Likewise, {\MI} can be integrated with direct-manipulation
document development or publishing systems by binding their descriptive
markup tags to the attributes of {\MI}'s input format and output style.

To summarize the differences in approach between {\MI} and \verb|make.index|,
let us examine the points of comparison: speed, size of the
programs, ease of use, portability, and extensibility.
{\MI} is an order of magnitude faster.  The \verb|awk| programs
of \verb|make.index| are less than 100 lines of code, while {\MI} is large:
6,000 lines of C code and a binary of 70K bytes.  {\MI} is a self-contained
C program, adapted from our original UNIX implementation, to run on
a variety of machines from micros to mainframes.  The \verb|make.index|
system will run on any UNIX system which has \verb|awk| so it too spans
many systems, although it is UNIX-specific while {\MI} is not.

{\MI} is easy to use.  It is monolithic and covers almost every conceivable
case likely to occur.  We actually examined many of the books in the
Addison-Wesley Computer Science Series and found that more than 95\% of the
cases could be handled by the program.  The \verb|make.index| suite, on the
other hand, handles the standard cases and all customizations must be done
by the user by modifying \verb|awk| code.  With {\MI}, user changes will 
very rarely occur.  If they should occur, and we have never had them happen,
then the user must change the C source code.  The question as to whether it is
easier to modify \verb|awk| code or C code is left to the reader.
Our answer is given in {\MI}.


\section{Acknowledgements}
We would like to thank Leslie Lamport for his collaboration in the design
of {\MI}.  He provided the initial specification and did some serious testing.
Charles Karney furnished the patches for {\MI} to run under VMS.
Nelson Beebe ported {\MI} to TOPS-20 and MS/DOS.
Some useful feedback came from Art Werschulz who tried out both {\MI}
and the index placing facility with {\AmSTeX}.
We are grateful to Richard Beach for his description of {\TG}'s
index processing facility, to Ravi Sethi for the information on {\TF}'s
index processing tools, and to Ventura Software, Inc. for giving us
documentation on their indexing mechanism.
Thanks also go to Ethan Munson and the anonymous referees for their helpful
comments on earlier drafts of this paper.

%\newpage
% Bibliography
% \bibliography{int,doc,env,hyper}
% \bibliographystyle{unsrt}

{\small
\singlespace
\input ind.bbl
}
\end{document}

