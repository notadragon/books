

A digit separator is a single-character token (\texttt{'}) that can appear as part of a
numeric literal without altering its value.

\subsection[Description]{Description}\label{description}

A digit separator --- i.e., an instance of the single-quote
character (\texttt{'}) --- may be placed anywhere within a numeric
literal to visually separate its digits without affecting its value:

%%%%%%%%%%%%%%%% ALL code snippets in this subsection
%%%%%%%%  setting this code coloring manually 
% Everything after // (including //) itself is a comment, and should be light blue (like -12345).
% `int`, `unsigned int`, `long`, `long long`, `float`, `double`, `long double`, `int`, `int`, `int` should be in the reddish color like the first occurrence of `int`.
% The variable names (e.g., i,u,j,k,f,d,e,hex,oct,bin) should be black.
% All equal signs should be black.
% All numeric literals (e.g., -12'345, 1'000'000u) should be black as well. 

\begin{lstlisting}[language=C++]
int          (ù{\color{black} i = -12'345;}ù)                    (ù{\itshape\color{skyblue}// same as}ù) (ù{\codeincomments{-12345}}ù)
(ù{\color{maroon}unsigned int}ù) (ù{\color{black}u = 1'000'000u;}ù)                 (ù{\itshape\color{skyblue}// same as}ù) (ù{\codeincomments{1000000u}}ù)
(ù{\color{maroon}long}ù)         (ù{\color{black}j = 5'0'0'0'0'0L;}ù)               // same as (ù{\codeincomments{500000L}}ù)
(ù{\color{maroon}long long}ù)    (ù{\color{black}k = 9'223'372'036'854'775'807;}ù)  // same as (ù{\codeincomments{9223372036854775807}}ù)
(ù{\color{maroon}float}ù)        (ù{\color{black}f = 10'00.42'45f;}ù)               // same as (ù{\codeincomments{1000.4245f}}ù)
(ù{\color{maroon}double}ù)       (ù{\color{black}d = 3.1415926'53589793;}ù)         (ù{\itshape\color{skyblue}// same as}ù) (ù{\codeincomments{3.141592653589793}}ù)
(ù{\color{maroon}long double}ù)  (ù{\color{black}e = 3.1415926'53589793'23846;}ù)   (ù{\itshape\color{skyblue}// same as}ù) (ù{\codeincomments{3.14159265358979323846}}ù)
(ù{\color{maroon}int}ù)        (ù{\color{black}hex = 0x8C25'00F9;}ù)                // same as (ù{\codeincomments{0x8C2500F9}}ù)
int        (ù{\color{black}oct = 044'73'26;}ù)                  // same as (ù{\codeincomments{0447326}}ù)
int        (ù{\color{black}bin = 0b1001'0110'1010'0111;}ù)      (ù{\itshape\color{skyblue}// same as}ù) (ù{\codeincomments{0b1001011000110001}}ù)
\end{lstlisting}
    
\noindent Multiple digit separators within a single literal are allowed, but they
cannot be contiguous, nor can they appear either before or after the
\emph{numeric} part (i.e., digit sequence) of the
literal{\cprotect\footnote{Although the leading \texttt{0x} and
\texttt{0b} prefixes for hexadecimal and binary literals,
respectively, are not considered part of the \emph{numeric} part of
  the lateral, a leading \texttt{0} in an octal literal is.}}:

\begin{lstlisting}[language=C++]
int (ù{\color{black}e0 = 10''00;}ù)  // error: consecutive digit separators
int (ù{\color{black}e1 = -'1000;}ù)  // error: before numeric part
(ù{\color{maroon}int}ù) (ù{\color{black}e2 = 1000'u;}ù)  // error: after numeric part
int (ù{\color{black}e3 = 0x'abc;}ù)  // error: before numeric part
(ù{\color{maroon}int}ù) (ù{\color{black}e4 = 0'xdef;}ù)  // error: way before numeric part
int (ù{\color{black}e5 = 0'89;}ù)    // error: non-octal digits
(ù{\color{maroon}int}ù) (ù{\color{black}e6 = 0'67;}ù)    // OK, valid octal literal
\end{lstlisting}
    
\noindent As a side note, remember that on some platforms an integer literal
that is too large to fit in a \texttt{long}~\texttt{long}~\texttt{int} but that does fit in an
\texttt{unsigned}~\texttt{long}~\texttt{long}~\texttt{int} might
generate a warning{\cprotect\footnote{Tested on GCC 7.4.0.}}:

\begin{lstlisting}[language=C++]
(ù{\color{maroon}unsigned long long}ù) (ù{\color{black}big1 = 9'223'372'036'854'775'808;}ù)  // (ù{\codeincomments{2\^{}63}}ù)
    // warning: integer constant is so large that it is an
    // unsigned long long big1 = 9'223'372'036'854'775'808;
    //                           ^~~~~~~~~~~~~~~~~~~~~~~~~~
\end{lstlisting}
    
\noindent Such warnings can typically be suppressed by adding a \texttt{ull}
suffix to the literal:

\begin{lstlisting}[language=C++]
(ù{\color{maroon}unsigned long long}ù) (ù{\color{black}big2 = 9'223'372'036'854'775'808ull;}ù)  // OK
\end{lstlisting}
    
\noindent Warnings like the one above, however, are not typical when the implied
precision of a floating-point literal exceeds what can be represented:

\begin{lstlisting}[language=C++]
(ù{\color{maroon}float}ù) (ù{\color{black}reallyPrecise = 3.141'592'653'589'793'238'462'643'383'279'502'884;}ù)  // OK
    // Everything after (ù{\codeincomments{3.141'592'6}}ù) is typically ignored silently.
\end{lstlisting}
    
\noindent For more information, see {\it\titleref{appendix:-silent-loss-of-precision-in-floating-point-literals}} on page~\pageref{appendix:-silent-loss-of-precision-in-floating-point-literals}.

\subsection[Use Cases]{Use Cases}\label{use-cases}

\subsubsection[Grouping digits together in large constants]{Grouping digits together in large constants}\label{grouping-digits-together-in-large-constants}

When embedding large constants in source code, consistently placing
digit separators (e.g., every thousand) might improve readability, as
illustrated in Table~\ref{digitseparator-table1}.

\begin{table}[h!]
\begin{center}
\begin{threeparttable}
\caption{Use of digit separators to improve readability}\label{digitseparator-table1}\vspace{1.5ex}
{\small \begin{tabular}{c|c}\thickhline
\rowcolor[gray]{.9}{\sffamily\bfseries Without Digit Separator} & {\sffamily\bfseries With Digit Separators} \\ \hline
\texttt{10000} & \texttt{10'000} \\ \hline 
\texttt{100000} & \texttt{100'000} \\ \hline 
\texttt{1000000} & \texttt{1'000'000} \\ \hline 
\texttt{1000000000} & \texttt{1'000'000'000} \\ \hline
\texttt{18446744073709551615ull} & \texttt{18'446'744'073'709'551'615ull} \\ \hline 
\texttt{1000000.123456} & \texttt{1'000'000.123'456} \\ \hline 
\texttt{3.141592653589793238462l} & \texttt{3.141'592'653'589'793'238'462l} \\ \thickhline
\end{tabular}
} % end small
\end{threeparttable}
\end{center} 
\end{table} 
\noindent Use of digit separators is especially useful with binary
literals, as shown in Table~\ref{digitseparator-table2}.\begin{table}[h!]
\begin{center}
\begin{threeparttable}
\caption{Use of digit separators in binary data}\label{digitseparator-table2}\vspace{1.5ex}
{\small \begin{tabular}{c|c}
\thickhline
\rowcolor[gray]{.9} {\sffamily\bfseries Without Digit Separator} & {\sffamily\bfseries With Digit Separators} \\\hline
\texttt{0b1100110011001100} & \texttt{0b1100'1100'1100'1100} \\ \hline
\texttt{0b0110011101011011} & \texttt{0b0110'0111'0101'1011} \\ \hline
\texttt{0b1100110010101010} & \texttt{0b11001100'10101010} \\
\thickhline
\end{tabular}
} % end small
\end{threeparttable}
\end{center}
\end{table}

\subsection[Potential Pitfalls]{Potential Pitfalls}\label{potential-pitfalls}

None so far

\subsection[See Also]{See Also}\label{see-also}

\begin{itemize}
\item{Section~\ref{binary-literals}, ``\titleref{binary-literals}" — Safe C++14 feature representing a binary constant for which digit separators are commonly used to group bits in octets (\textbf{bytes}) or quartets (\textbf{nibbles})}
\end{itemize}

\subsection[Further Reading]{Further Reading}\label{further-reading}

\begin{itemize}
\item{William Kahan. ``Lecture Notes on the Status of
IEEE Standard 754 for Binary Floating-Point Arithmetic,'' \textbf{kahan97}}
\item{{\textit{IEEE Standard for Floating-PointArithmetic}}, \textbf{ieee19}}
\end{itemize} 

\subsection[Appendix: Silent Loss of Precision in Floating-Point Literals]{Appendix: Silent Loss of Precision in Floating-Point Literals}\label{appendix:-silent-loss-of-precision-in-floating-point-literals}

Just because we can keep track of precision in floating-point literals
doesn't mean that the compiler can. As an aside, it is worth
pointing out that the binary representation of floating-point types is
not mandated by the Standard, nor are the precise minimums on the ranges
and precisions they must support. Although the C++ Standard says little
that is normative, the macros in \texttt{<cfloat>} are defined by
reference to the C Standard.{\cprotect\footnote{\textbf{PRODUCTION: WAITING FOR THESE REFERENCES TO DOUBLECHECK; CONSIDER THESE UNCONFIRMED.} \textbf{iso20}, sections [basic.fundamental] Fundamental types (6.8.1p12); [numeric.limits.members] \texttt{numeric\_limits} members 17.3.5.1; [cfloat.syn] Header \texttt{<cfloat>} synopsis (17.3.7p1); \textbf{iso18b}, section 5.2.4.2.2 Characteristics of floating types \texttt{<float.h>} }}

There are, however, normal and customary minimums that one can typically
rely upon in practice. On conforming compilers that employ the IEEE 754
floating-point standard representation\footnote{\textbf{ieee19}} (as most do),
a \texttt{float} can typically represent up to 7 significant decimal
digits accurately, while a \texttt{double} typically nearly 15 decimal
digits of precision. For any given program,
\texttt{long}~\texttt{double} is required to hold whatever a
\texttt{double} can hold, but is typically larger (e.g., 10, 12, or 16
bytes) and typically adds at least 5 decimal digits of precision (i.e.,
supports a total of at last 20 decimal digits). A table summarizing
typical precisions for various IEEE-conforming floating-point types is
presented for convenient reference in Table~\ref{digitseparator-table3}. The actual bounds on a
given platform can be found using the standard
\texttt{std::numeric\_limits} class template found in \texttt{<limits>}.\pagebreak[4]
\begin{table}[h!] 
\begin{center}
\begin{threeparttable}
\caption{Available precisions for various IEEE-754 floating-point types}\label{digitseparator-table3}\vspace{1.5ex} 
{\small \begin{tabular}{c|c|c|c|c|c}\thickhline 
\rowcolor[gray]{.9}   {\sffamily\bfseries Name} & {\sffamily\bfseries Common} &
{\sffamily\bfseries Significant } & {\sffamily\bfseries Decimal }& {\sffamily\bfseries Exponent } & {\sffamily\bfseries Dynamic}\\
\rowcolor[gray]{.9}    & {\sffamily\bfseries Name} &
{\sffamily\bfseries Bits\tnote{a}} & {\sffamily\bfseries Bits}& {\sffamily\bfseries Bits} & {\sffamily\bfseries Range}\\ \hline 
binary16 & Half precision & 11 & 3.31 & 5 & $\sim6.50\text{e}5$\\ \hline
binary32 & Single precision & 24 & 7.22 & 8 & $\sim3.4\text{e}38$\\ \hline
binary64 & Double precision & 53 & 15.95 & 11 & $\sim1.\text{e}308$\\ \hline
\rule{0pt}{3.5mm}binary80 & Extended precision & 69 & 20.77 & 11 & $\sim10^{308}$\\ \hline
\rule{0pt}{3.5mm}binary128 & Quadruple precision & 113 & 34.02 & 15 & $\sim10^{4932}$\\ \hline
\end{tabular}
} % end small
\begin{tablenotes}{\footnotesize  
\item[a]{Note that the most significant bit of the \textbf{mantissa}
is always a \texttt{1} and, hence, is not stored explicitly, leaving 1
additional bit to represent the sign of the overall floating-point value
(the sign of the exponent is encoded using \textbf{excess-$\mathbf{n}$}
notation).} 
} % end footnotesize 
\end{tablenotes} % we end the table footnotes
\end{threeparttable} 
\end{center}
\end{table}

Determining the minimum number of decimal digits needed to accurately
approximate a transcendental value, such as $\pi$, for a given type on
a given platform can be tricky (requiring some binary-search-like
detective work), which is likely why overshooting the precision without
warning is the default on most platforms. One way to establish that
\emph{all} of the decimal digits in a given floating-point literal are
relevant for a given floating-point type is to compare that literal and
a similar one with its least significant decimal digit
removed{\cprotect\footnote{Note that affixing the \texttt{f}
(\emph{literal suffix}) to a floating-point literal is
equivalent to applying a \texttt{static\_cast<float>} to the
(unsuffixed) literal:

\begin{lstlisting}[language=C++, basicstyle={\ttfamily\footnotesize}]
static_assert(3.14'159'265'358f == static_cast<float>(3.14'159'265'358));
\end{lstlisting}\vspace*{-1ex}
      }}:

\begin{lstlisting}[language=C++]
static_assert(3.1415926535f != 3.141592653f, "too precise for float");
    // This assert will fire on a typical platform.

static_assert(3.141592653f != 3.14159265f, "too precise for float");
    // This assert too will fire on a typical platform.

static_assert(3.14159265f != 3.1415926f, "too precise for float");
    // This assert will NOT fire on a typical platform.

static_assert(3.1415926f != 3.141592f, "too precise for float");
    // This assert too will NOT fire on a typical platform.
\end{lstlisting}
    
\noindent If the values are \emph{not} the same, then that floating-point type
\emph{can} make use of the precision suggested by the original literal; if
they \emph{are} the same, however, then it is likely that the available
precision has been exceeded. Iterative use of this technique by
developers can help them to empirically narrow down the maximal number
of decimal digits a particular platform will support for a particular
floating-point type and value.

One final useful tidbit pertains to the safe (lossless) conversion
between binary and decimal floating-point representations; note that
``Single'' (below) corresponds to a single-precision IEEE-754-conforming
(32-bit) \texttt{float}{\cprotect\footnote{\textbf{kahan97}, section ``Representable Numbers," p.~4}}:

\begin{quote}
If a decimal string with at most 6 sig. dec. is converted to Single and then converted back to the
 same number of sig. dec., then the final string should match the original. Also, ...

 If a Single Precision floating-point number is converted to a decimal string with at least 9 sig.
 dec. and then converted back to Single, then the final number must match the original.
\end{quote}
    
\noindent The ranges corresponding to 6--9 for a single-precision (32-bit)
\texttt{float} (described above), when applied to a double-precision
(64-bit) \texttt{double} and a quad-precision (128-bit)
\texttt{long}~\texttt{long}, are 15--17 and 33--36, respectively.



