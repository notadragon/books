% Jan 13, copyedits in and proofed


Using \lstinline!=!\,\lstinline!delete! in a function's first declaration
forces a compilation error upon any attempt to use or access it.

\subsection[Description]{Description}\label{description}

Declaring a particular function or function overload to result in a
fatal diagnostic upon invocation can be useful --- e.g., to suppress the
generation of a \romeogloss{special member function} or to limit the types of
arguments a particular overload set is able to accept. In such cases,
\lstinline!=!\,\lstinline!delete! followed by a semicolon (\lstinline!;!) can be used in place of the body of any
function on first declaration only to force a compile-time error if
any attempt is made to invoke it or take its address.

\begin{emcppslisting}
void g(double) { }
void g(int) = delete;

void f()
{
    g(3.14);  // OK, (ù{\codeincomments{f(double)}}ù) is invoked.
    g(0);     // Error, (ù{\codeincomments{f(int)}}ù) is deleted.
}
\end{emcppslisting}

\noindent Notice that deleted functions participate in \romeogloss{overload resolution}
and produce a compile-time error when selected as the best candidate.

\subsection[Use Cases]{Use Cases}\label{use-cases}

\subsubsection[Suppressing special member function generation]{Suppressing special member function generation}\label{suppressing-special-member-function-generation}

When instantiating an object of user-defined type, \romeogloss{special
member functions} that have not been declared explicitly are
often generated automatically by the compiler.
The generation of individual special member
functions can be affected by the existence of other user-defined special
member functions or by limitations imposed by the specific types of
any data members or base types; see
%Section~\ref{Defaulted-Special-Member-Functions}, ``\titleref{Defaulted-Special-Member-Functions} on page~\pageref{Defaulted-Special-Member-Functions}."
\featureref{\locationa}{Defaulted-Special-Member-Functions}. For certain kinds of types, the notion of \romeogloss{copy semantics}
including \romeogloss{move semantics} is not meaningful, and
hence permitting the generation of copy operations is contraindicated. The two special member functions controlling \romeogloss{\emph{move} operations}, introduced in C++11, are sometimes implemented as effective optimizations of \romeogloss{\emph{copy} operations} and much less frequently with \romeogloss{\emph{copy} operations} explicitly deleted; see \featureref{\locationc}{Rvalue-References}.

Consider a class, \lstinline!FileHandle!, that uses the \romeogloss{RAII} idiom
to safely acquire and release an I/O stream. As \romeogloss{copy semantics}
are typically not meaningful for such resources, we will want to
suppress generation of both the \romeogloss{copy constructor} and \romeogloss{copy
assignment operator}. Prior to C++11, there was no direct way to express
suppression of \romeogloss{special functions} in C++. The commonly
recommended workaround was to declare the two methods \lstinline!private!
and leave them unimplemented, typically resulting in a compile-time or
link-time error when accessed:

\begin{emcppslisting}
#include <cstdio>  // (ù{\codeincomments{FILE}}ù)
class FileHandle
{
private:
    // ...

    FileHandle(const FileHandle&);             // not implemented
    FileHandle& operator=(const FileHandle&);  // not implemented

public:
    explicit FileHandle(FILE* filePtr);
    ~FileHandle();

    // ...
};
\end{emcppslisting}

\noindent Not implementing a special member function that is declared to be private ensures that there will be at least a link-time error in case that function is inadvertently accessed from within the implementation of the class itself. With the \lstinline!=!\,\lstinline!delete! syntax, we are able to (1)
explicitly express our intention to make these special member
functions unavailable, (2) do so directly in the \lstinline!public! region
of the class, and (3) enable clearer compiler diagnostics:

%begin{emcppslisting}
%class FileHandle
%{
%private:
%    // ...
%
%public:
%    explicit FileHandle(FILE* filePtr);
%    ~FileHandle();
%
%    FileHandle(const FileHandle&) = delete;             // make unavailable
%    FileHandle& operator=(const FileHandle&) = delete;  // make unavailable
%
%    // ...
%};
%\end{emcppslisting}
\begin{emcppshiddenlisting}[emcppsbatch=e1]
#include <cstdio>  // (ù{\codeincomments{FILE}}ù)
\end{emcppshiddenlisting}
\begin{emcppslisting}[emcppsbatch=e1]
class FileHandle
{
private:
    // ...
    // Declarations of copy constructor and copy assignment are now public.

public:
    explicit FileHandle(FILE* filePtr);
    ~FileHandle();

    FileHandle(const FileHandle&) = delete;             // make unavailable
    FileHandle& operator=(const FileHandle&) = delete;  // make unavailable

    // ...
};
\end{emcppslisting}

\noindent Using the \lstinline!=!\,\lstinline!delete! syntax on declarations that are private results in error messages concerning privacy, not the use of deleted functions. Care must be exercised to make \emph{both} changes when converting code from the old style to the new syntax.

\subsubsection[Preventing a particular implicit conversion]{Preventing a particular implicit conversion}\label{preventing-a-particular-implicit-conversion}

Certain functions --- especially those that take a \lstinline!char! as an
argument --- are prone to inadvertent misuse. As a truly classic
example, consider the C library function \lstinline!memset!, which may be used
to write the character \lstinline!*! five times in a row, starting at a
specified memory address, \lstinline!buf!:

%begin{emcppslisting}
%#include <cstring>
%#include <cstdio>
%
%void f()
%{
%    char buf[] = "Hello World!";
%    memset(buf, 5, '*');  // undefined behavior
%    puts(buf);            // expected output: "***** World!"
%}
%\end{emcppslisting}
\begin{emcppslisting}
#include <cstdio>  // (ù{\codeincomments{puts}}ù)
#include <cstring> // (ù{\codeincomments{memset}}ù)

void f()
{
    char buf[] = "Hello World!";
    memset(buf, 5, '*');  // undefined behavior: buffer overflow
    puts(buf);            // expected output: "***** World!"
}
\end{emcppslisting}


\noindent Sadly, inadvertently reversing the order of the last two arguments is a commonly
recurring error, and the C language provides no help. As shown above, \lstinline!memset! writes the nonprinting character \lstinline!5! (e.g., the integer value of ASCII \lstinline!'*'!) 42 times --- way past the end of \lstinline!buf!. In C++, we
can target such observed misuse using an extra deleted overload:

%begin{emcppslisting}
%#include <cstring>  // memset()
%void* memset(void* str, int ch, size_t n);      // standard library function
%void* memset(void* str, int n, char) = delete;  // defensive against misuse
%\end{emcppslisting}
\begin{emcppslisting}
#include <cstring>  // (ù{\codeincomments{memset}}ù)
void* memset(void* str, int ch, size_t n);      // standard library function
void* memset(void* str, int n, char) = delete;  // defense against misuse
\end{emcppslisting}

\noindent Pernicious user errors can now be reported during compilation:

\begin{emcppslisting}
// ...
memset(buf, 5, '*');  // Error, (ù{\codeincomments{memset(void\*, int, char)}}ù) is deleted.
// ...
\end{emcppslisting}


\subsubsection[Preventing all implicit conversions]{Preventing all implicit conversions}\label{preventing-all-implicit-conversions}

The \lstinline!ByteStream::send! member function below is designed to work
with 8-bit unsigned integers only. Providing a deleted overload
accepting an \lstinline!int! forces a caller to ensure that the argument is
always of the appropriate type:

\begin{emcppslisting}
class ByteStream
{
public:
    void send(unsigned char byte) { /* ... */ }
    void send(int) = delete;

    // ...
};

void f()
{
    ByteStream stream;
    stream.send(0);   // Error, (ù{\codeincomments{send(int)}}ù) is deleted.     (1)
    stream.send('a'); // Error, (ù{\codeincomments{send(int)}}ù) is deleted.     (2)
    stream.send(0L);  // Error, ambiguous                 (3)
    stream.send(0U);  // Error, ambiguous                 (4)
    stream.send(0.0); // Error, ambiguous                 (5)
    stream.send(
        static_cast<unsigned char>(100));  // OK          (6)
}
\end{emcppslisting}

\noindent Invoking \lstinline!send! with an \lstinline!int! (noted with (1) in the code above) or any integral type
(other than \lstinline!unsigned!~\lstinline!char!) that promotes to \lstinline!int! (2)
will map exclusively to the deleted \lstinline!send(int)! overload; all
other integral (3 and 4) and floating-point types (5) are convertible to
both via a \romeogloss{standard conversion} and hence will be ambiguous. Note that
implicitly converting from \lstinline!unsigned!~\lstinline!char! to either a
\lstinline!long! or \lstinline!unsigned! integer involves a \romeogloss{standard
conversion} (not just an \romeogloss{integral promotion}), the same as
  converting to a \lstinline!double!.
An explicit cast to \lstinline!unsigned!~\lstinline!char! (6) can always be
pressed into service if needed.

\subsubsection[Hiding a structural (nonpolymorphic) base class's member function]{Hiding a structural (nonpolymorphic) base class’s member function}\label{hiding-a-structural-(nonpolymorphic)-base-class's-member-function}

It is commonly advised to avoid deriving publicly from concrete classes because by doing so, we do not hide the underlying capabilities, which can easily be accessed (potentially breaking any invariants the derived class may want to keep) via assignment to a pointer or reference to a base class, with no casting required.  Worse, inadvertently passing such a class to a function taking the base class by value will result in slicing, which can be especially problematic when the derived class holds data. A more robust approach would be to use layering or at least private inheritance.\footnote{For more on improving compositional designs at scale, see
  \cite{lakos20}, sections 3.5.10.5 and 3.7.3, pp.~687--703 and
  726--727, respectively.} Best practices notwithstanding,\footnote{See ``Inheritance and Object-Oriented Programming," Item 38, \cite{meyers92}, pp. 132--135.} it can be cost-effective in the short
term to provide an elided ``view'' on a concrete class for trusted
clients. Imagine a class \lstinline!AudioStream! designed to play sounds
and music that --- in addition to providing basic ``play'' and
``rewind'' operations --- sports a large, robust interface:

\begin{emcppshiddenlisting}[emcppsbatch=e2]
struct ForwardAudioStream;
struct FMRadio {
    static ForwardAudioStream getStream();
};
\end{emcppshiddenlisting}
\begin{emcppslisting}[emcppsbatch=e2]
struct AudioStream
{
    void play();
    void rewind();
    // ...
    // ... (large, robust interface)
    // ...
};
\end{emcppslisting}


Now suppose that, on short notice, we need to whip up a similar
class,\linebreak[4] \lstinline!ForwardAudioStream!, to use with audio samples that
cannot be rewound (e.g., coming directly from a live feed). Realizing
that we can readily reuse most of \lstinline!AudioStream!'s interface, we
pragmatically decide to prototype the new class simply by exploiting
public \romeogloss{structural inheritance} and then deleting just the lone
unwanted \lstinline!rewind! member function:

\begin{emcppslisting}[emcppsbatch=e2]
struct ForwardAudioStream : AudioStream
{
    void rewind() = delete; // Make just this one function unavailable.
};

void f()
{
    ForwardAudioStream stream = FMRadio::getStream();
    stream.play();   // fine
    stream.rewind(); // Error, (ù{\codeincomments{rewind()}}ù) is deleted.
}
\end{emcppslisting}

\noindent If the need for a \lstinline!ForwardAudioStream! type persists, we can always
consider reimplementing it more carefully later.{\cprotect\footnote{\cite[sections 3.5.10.5 and 3.7.3, pp.~687--703 and 726--727]{lakos20}}} As discussed at the beginning of this section, the protection provided by this example is easily circumvented:
\begin{emcppslisting}[emcppsbatch=e2]
void g(const ForwardAudioStream &stream)
{
    AudioStream fullStream = stream;
    fullStream.play();   // OK
    fullStream.rewind(); // compiles OK, but what happens at run time?
}
\end{emcppslisting}

\noindent Hiding non\lstinline!virtual! functions is something one undertakes only after attaining a complete understanding of what makes such an unorthodox endeavor \emph{safe}; see, in particular, the appendix of \featureref{\locatione}{final}.

\subsection[Potential Pitfalls]{Potential Pitfalls}\label{potential-pitfalls}

\hspace*{\fill}

\subsection[Annoyances]{Annoyances}\label{annoyances}

\hspace*{\fill}

\subsection[See Also]{See Also}\label{see-also}

\begin{itemize}
\item{%Section~\ref{Defaulted-Special-Member-Functions}, ``\titleref{Defaulted-Special-Member-Functions}" on page~\pageref{Defaulted-Special-Member-Functions} —
\seealsoref{Defaulted-Special-Member-Functions}{\locationa}Companion feature that enables defaulting, as opposed to deleting, special member functions.}
\item{%Section~\ref{Rvalue-References}, ``\titleref{Rvalue-References}" on page~\pageref{Rvalue-Referencess} —
\seealsoref{Rvalue-References}{\locationc}The two \emph{move} variants of special member functions, which use \romeovalue{rvalue} references in their signatures, may also be subject to deletion.}
\end{itemize}

\subsection[Further Reading]{Further Reading}\label{further-reading}

\begin{itemize}
\item{``Item 27" of \cite{meyers15b}}
\end{itemize}

