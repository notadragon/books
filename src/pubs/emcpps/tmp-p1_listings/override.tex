\newpage
\section[{\tt override}]{The {\SecCode override} Member-Function Specifier}\label{override}
% 13 Jan: copyedits in and proofed

The \lstinline!override! keyword ensures that a member function overrides a corresponding \lstinline!virtual! member function in a base class.

\subsection[Description]{Description}\label{description}

The \romeogloss{contextual keyword} \lstinline!override! can be provided at the
end of a member-function declaration to ensure that the decorated
function is indeed \emph{overriding} a corresponding \lstinline!virtual!
member function in a base class, as opposed to \emph{hiding} it or
otherwise inadvertently introducing a distinct function declaration:

% original
%begin{emcppslisting}
%struct Base
%{
%    virtual void f(int);
%    void g(int);
%};
%
%struct Derived : Base
%{
%    void f();              // hides (ù{\codeincomments{Base::f(int)}}ù) (likely mistake)
%    void f() override;     // error: (ù{\codeincomments{Base::f()}}ù) not found
%
%    void f(int);           // implicitly overrides (ù{\codeincomments{Base::f(int)}}ù)
%    void f(int) override;  // explicitly overrides (ù{\codeincomments{Base::f(int)}}ù)
%
%    void g();              // hides (ù{\codeincomments{Base::g(int)}}ù) (likely mistake)
%    void g() override;     // error: (ù{\codeincomments{Base::g()}}ù) not found
%
%    void g(int);           // hides (ù{\codeincomments{Base::g(int)}}ù) (likely mistake)
%    void g(int) override;  // Error: (ù{\codeincomments{Base::g()}}ù) is not (ù{\codeincomments{virtual}}ù).
%};
%\end{emcppslisting}
% Slava's replacement
\begin{emcppslisting}
struct Base
{
    virtual void f(int);
            void g(int);
    virtual void h(int) const;
    virtual void i(int) = 0;
};

struct DerivedWithoutOverride : Base
{
    void f();             // hides (ù{\codeincomments{Base::f(int)}}ù) (likely mistake)
    void f(int);          // OK, implicitly overrides (ù{\codeincomments{Base::f(int) }}ù)

    void g();             // hides (ù{\codeincomments{Base::g(int)}}ù) (likely mistake)
    void g(int);          // hides (ù{\codeincomments{Base::g(int)}}ù) (likely mistake)

    void h(int);          // hides (ù{\codeincomments{Base::h(int)}}ù) const (likely mistake)
    void h(int) const;    // OK, implicitly overrides (ù{\codeincomments{Base::h(int)}}ù) const

    void i(int);          // OK, implicitly overrides (ù{\codeincomments{Base::i(int)}}ù)
};

struct DerivedWithOverride : Base
{
    void f()          override;    // Error, (ù{\codeincomments{Base::f()}}ù) not found
    void f(int)       override;    // OK, explicitly overrides (ù{\codeincomments{Base::f(int)}}ù)

    void g()          override;    // Error, (ù{\codeincomments{Base::g()}}ù) not found
    void g(int)       override;    // Error, (ù{\codeincomments{Base::g()}}ù) is not virtual.

    void h(int)       override;    // Error, (ù{\codeincomments{Base::h(int)}}ù) not found
    void h(int) const override;    // OK, explicitly overrides (ù{\codeincomments{Base::h(int)}}ù)

    void i(int)       override;    // OK, explicitly overrides (ù{\codeincomments{Base::i(int)}}ù)
};
\end{emcppslisting}

\noindent Using this feature expresses design intent so that (1) human readers
are aware of it and (2) compilers can validate it.

As noted, \lstinline!override! is a contextual keyword. C++11 introduces keywords that have special meaning only in certain contexts.   In this case, \lstinline!override! is a keyword in the context of a declaration, but not otherwise using it as the identifier for a variable name, for example, is perfectly fine:

%%%%%%%% EXCEPTION! override here should NOT be bold
\begin{emcppslisting}
int override = 1;  // OK
\end{emcppslisting}

\subsection[Use Cases]{Use Cases}\label{use-cases}

\subsubsection[Ensuring that a member function of a base class is being overridden]{Ensuring that a member function of a base class is being overridden}\label{ensuring-that-a-member-function-of-a-base-class-is-being-overridden}

Consider the following polymorphic hierarchy of error-category classes, as we might have defined them using C++03:

\begin{emcppshiddenlisting}[emcppsbatch=e1]
struct ErrorCode;
struct ErrorCondition;
\end{emcppshiddenlisting}
\begin{emcppslisting}[emcppsbatch=e1]
struct ErrorCategory
{
    virtual bool equivalent(const ErrorCode& code, int condition);
    virtual bool equivalent(int code, const ErrorCondition& condition);
};

struct AutomotiveErrorCategory : ErrorCategory
{
    virtual bool equivalent(const ErrorCode& code, int condition);
    virtual bool equivolent(int code, const ErrorCondition& condition);
};
\end{emcppslisting}

\noindent Notice that there is a defect in the last line of the example above:
\lstinline!equivalent! has been misspelled. Moreover, the compiler did not
catch that error. Clients calling \lstinline!equivalent! on
\lstinline!AutomotiveErrorCategory! will incorrectly invoke the base-class
function. If the function in the base class happens to be defined, the
code might compile and behave unexpectedly at run time. Now, suppose
that over time the interface is changed by marking the
equivalence-checking function \lstinline!const! to bring the interface
closer to that of \lstinline!std::error_category!:

\begin{emcppshiddenlisting}[emcppsbatch=e2]
struct ErrorCode;
struct ErrorCondition;
\end{emcppshiddenlisting}
\begin{emcppslisting}[emcppsbatch=e2]
struct ErrorCategory
{
    virtual bool equivalent(const ErrorCode& code, int condition) const;
    virtual bool equivalent(int code, const ErrorCondition& condition) const;
};
\end{emcppslisting}

Without applying the corresponding modification to all classes deriving
from\linebreak[4] \lstinline!ErrorCategory!, the semantics of the program change due to
the derived classes now hiding
the base class's
\lstinline!virtual! member function instead of overriding it. Both errors discussed above
would be detected automatically if the \lstinline!virtual!
functions in all derived classes were decorated with \lstinline!override!:

\begin{emcppslisting}[emcppsbatch=e2]
struct AutomotiveErrorCategory : ErrorCategory
{
    bool equivalent(const ErrorCode& code, int condition) override;
        // Error, failed when base class changed

    bool equivolent(int code, const ErrorCondition& code) override;
        // Error, failed when first written
};
\end{emcppslisting}

\noindent What's more, \lstinline!override! serves as a clear indication of the derived-class author's intent to customize the
behavior of \lstinline!ErrorCategory!. For any given member function, using \lstinline!override! necessarily renders any use of \lstinline!virtual! for
that function syntactically and semantically redundant. The only
cosmetic reason for retaining \lstinline!virtual! in the presence of
\lstinline!override! would be that \lstinline!virtual! appears to the left of
the function declaration, as it always has, instead of all the way to
the right, as \lstinline!override! does now.

\subsection[Potential Pitfalls]{Potential Pitfalls}\label{potential-pitfalls}

\subsubsection[Lack of consistency across a codebase]{Lack of consistency across a codebase}\label{lack-of-consistency-across-a-codebase}

Relying on \lstinline!override! as a means of ensuring that changes to
base-class interfaces are propagated across a codebase can prove
unreliable if this feature is used inconsistently --- i.e., not applied in every circumstance where its use would be appropriate. In
particular, altering the signature of a \lstinline!virtual! member function
in a base class and then compiling ``the world'' will always flag as an
error any nonmatching derived-class function where \lstinline!override!
was used but might fail even to warn where it is not.

\subsection[Annoyances]{Annoyances}\label{annoyances}

\hspace*{\fill}

\subsection[See Also]{See Also}\label{see-also}

\hspace*{\fill}

\subsection[Further Reading]{Further Reading}\label{further-reading}

\begin{itemize}
\item{\cite{meyers15b}}
\item{\cite{fluentcpp}}
\end{itemize}


