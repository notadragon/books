% 8 March 2021, ready for Josh's code check
% 9 March 2021 JMB - code compiles, corrected u8 bugs

\emcppsFeature{
    short={User-Defined Literals},
    long={User-Defined Literal Operators},
}{userdeflit}
\setcounter{table}{0}
\setcounter{footnote}{0}
\setcounter{lstlisting}{0}
%\section[User-Defined Literals]{User-Defined Literal Operators}%\label{}


C++11 allows developers to define a new suffix for a \emph{numeric},
\emph{character}, or \emph{string} literal, enabling a convenient
lexical representation of the \emph{value} of a \emcppsgloss[user defined type (UDT)]{user-defined
type (UDT)} or even a novel notation for the value of a built-in type.

\subsection[Description]{Description}\label{description-userdeflit}

A \emcppsgloss{literal} is a single token in a program that represents a
value of an integer, floating-point, character, string, Boolean, or
pointer type.

Examples of familiar literal tokens are integer literals \lstinline!19! and
\lstinline!0x13!, each representing an \lstinline!int! having a value of 19;
floating-point literals \lstinline!0.19! and \lstinline!1.9e-1!, each
representing a \lstinline!double! having a value of 0.19; character
literals \lstinline!'a'! and \lstinline!\141!, 
each representing a
\lstinline!char! having the (ASCII) value for the letter ``a''; string
literal, \lstinline!"hello"!, representing a null-terminated array
containing the 6 characters \lstinline!'h'!, \lstinline!'e'!, \lstinline!'l'!,
\lstinline!'l'!, \lstinline!'o'!, and \lstinline!'\0'!; 
and Boolean
keyword literals \lstinline!true! and \lstinline!false!, representing the
corresponding Boolean values. C++11 added the keyword literal,
\lstinline!nullptr! (see \featureref{\locationa}{null-pointer-literal-(nullptr)}), representing the
null pointer value.

Both integer and floating-point literals have always had suffixes to
identify other numeric C++ types. For example, \lstinline!123L! and
\lstinline!123ULL! are literals of type \lstinline!signed!~\lstinline!long! and\linebreak[4]%%%%%   
\lstinline!unsigned!~\lstinline!long!~\lstinline!long!, respectively, both having
a decimal value 123, whereas \lstinline!123.f! is a literal of type
\lstinline!float! having precisely the decimal value 123. We can easily
distinguish programmatically between these different types of literals
using, e.g., overload resolution:

\begin{emcppslisting}
void f(const int&);            // (1) overload for type (ù{\codeincomments{int}}ù)
void f(const long&);           // (2)    "      "   "   (ù{\codeincomments{long}}ù)
void f(const double&);         // (3)    "      "   "   (ù{\codeincomments{double}}ù)
void f(const float&);          // (4)    "      "   "   (ù{\codeincomments{float}}ù)
void f(const unsigned int&);   // (5)    "      "   "   (ù{\codeincomments{unsigned int}}ù)
void f(const unsigned long&);  // (6)    "      "   "   (ù{\codeincomments{unsigned long}}ù)

void test0()
{
    f(123);    // OK, calls (1)
    f(123L);   // OK, calls (2)
    f(123.);   // OK, calls (3)
    f(123.f);  // OK, calls (4)
    f(123U);   // OK, calls (5)
    f(123Lu);  // OK, calls (6)
    f(123.L);  // Error, call to (ù{\codeincomments{f(long double)}}ù) is ambiguous
    f(123f);   // Error, invalid hex digit (ù{\codeincomments{f}}ù) in decimal constant
}
\end{emcppslisting}
    
\noindent Notice that applying an \lstinline!L! (or \lstinline!l!) suffix to a
floating-point literal (of default type \lstinline!double!) identifies it
as being of type \lstinline!long!~\lstinline!double!, which is a
\emcppsgloss{standard conversion} away from both \lstinline!float! and
\lstinline!double!, making the call ambiguous. Applying \lstinline!F! (or
\lstinline!f!) to an integer literal is, by default, not permitted unless a
user-defined literal (UDL) of a compatible type can be found.

Classic C++ allowed only values of built-in type to be represented as
compile-time literals. To express a hard-coded value of a UDT, a
developer would need to use a \emcppsgloss{value constructor} or
\emcppsgloss{factory function} and, unlike literals of built-in type, these
\emph{runtime} workarounds could never be used in a \emcppsgloss{constant
expression}. For example, we might want to create a user-defined type,
\lstinline!Name!, that can construct itself from a null-terminated string:

\begin{emcppslisting}[emcppsbatch=e1]
class Name  // user-defined type constructible from a literal string
{
    // ...

public:
    Name(const char*);  // value constructor taking a null-terminated string
    //...
};
\end{emcppslisting}
    
\noindent We can then initialize a variable of type \lstinline!Name! from a string
literal using the value constructor:

\begin{emcppslisting}[emcppsbatch=e1]
Name nameField("Maria");  // (ù{\codeincomments{Name}}ù) object having value (ù{\codeincomments{"Maria"}}ù)
\end{emcppslisting}
    
\noindent Alternatively, we could create one or more \emcppsgloss[factory function]{factory functions}
that return a constructed object appropriately configured with the
desired value. Multiple factory functions having different names can be
created without necessarily adding new constructors to the definition of
the returned type though, in some cases, a factory function might be a
\lstinline!friend! of the type it configures:

\begin{emcppshiddenlisting}[emcppsbatch=e2]
class Temperature;
bool operator==(const Temperature&,const Temperature&);
\end{emcppshiddenlisting}
\begin{emcppslisting}[emcppsbatch=e2]
#include <cassert>  // standard C (ù{\codeincomments{assert}}ù) macro

class Temperature { /*...*/ };

Temperature fahrenheit(double degrees);  // configured from degrees Fahrenheit
Temperature celsius(double degrees);     // configured from degrees Celsius

void test1()
{
    Temperature t1 = fahrenheit(32);  // water freezes at this temperature
    Temperature t2 = celsius(0.0);    //   "      "     "  "        "
    assert(t1 == t2);                 // expect same type and same value
}
\end{emcppslisting}
    
\noindent  Note that, as of C++11 and later, the two functional constructs
described above can be declared \lstinline!constexpr! and thus be eligible
to be evaluated as part of a \emcppsgloss{constant expression}; see
\featureref{\locationc}{constexprfunc}. 

Although usable, the aforementioned C++03 workarounds for representing
literal values of a UDT lack the compactness and expressiveness of those
for built-in types. A fundamental design goal of C++ has always been to
minimize such differences. To that end, C++11 extends the notion of
\emcppsgloss[type suffix]{type suffix} to include user-definable identifiers with
a leading underscore (e.g., \lstinline!_name!, \lstinline!_time!,
\lstinline!_temp!):

\begin{emcppslisting}[emcppsbatch=e2]
Temperature operator""_F(long double degrees) { /*...*/ }  // define suffix (ù{\codeincomments{\_F}}ù)
Temperature operator""_C(long double degrees) { /*...*/ }  // define suffix (ù{\codeincomments{\_C}}ù)

void test2()  // same as (ù{\codeincomments{test1}}ù) above, but this time with user-defined literals
{
    Temperature t1 = 32.0_F; // water freezes at 32 degrees Fahrenheit
    Temperature t2 = 0.0_C;  //   "      "     "  0 degrees Celsius
    assert(t1 == t2);        // expect same type and same value
}
\end{emcppslisting}
    
\noindent The example above demonstrates the basic idea of a \emcppsgloss[user defined
literal (UDL)]{UDL} implemented as a new kind of operator:
\lstinline!operator""! followed by a suffix name. A \emcppsgloss[user defined
literal (UDL)]{UDL} is a
literal token having a \emcppsgloss{UDL suffix} that names a \emcppsgloss{UDL
operator}. A \emcppsgloss[user defined
literal (UDL)]{UDL} \emph{uses} a suffix whereas a \emcppsgloss{UDL
operator} \emph{defines} a suffix. A \emcppsgloss{UDL suffix} must be a valid
identifier that begins with an underscore (\lstinline!_!). Note that the
leading underscore is \emph{not} required for \emcppsgloss[UDL suffix]{UDL suffixes}
defined by the C++ Standard Library. A \emcppsgloss[user defined
literal (UDL)]{UDL} is formed by
appending a \emcppsgloss{UDL suffix} to a native literal of one of four
\emcppsgloss[type category]{type categories}: \emcppsgloss[integer literal]{integer literals} (e.g.,
\lstinline!2020_year!), \emcppsgloss[floating point literal]{floating-point literals} (e.g.,
\lstinline!98.6_F!), \emcppsgloss[character literal]{character literals} (e.g.,
\lstinline!'x'_ebcdic!, and \emcppsgloss[string literal]{string literals} (e.g.,
\lstinline!"1!~\lstinline!Pennsylvania!~\lstinline!Ave"_validated!). Regardless
of the type category, a \emcppsgloss[user defined literal (UDL)]{UDL} can evaluate to any built-in or
user-defined type. What's more, the same \emcppsgloss{UDL suffix} can apply
to more than one of the four UDL type categories enumerated above,
potentially yielding a different type for each category.

Each UDL operator is effectively a \emcppsgloss{factory operator} that takes
a highly-constrained argument list (see \intrarefsimple{user-defined-literal-(udl)-operators}) 
and returns an
appropriately configured object. The defined \emcppsgloss{UDL suffix}, like
postfix operators \lstinline!++! and \lstinline!--!, names a function to be
called, but is parsed as part of the preceding literal (with no
intervening whitespace) and cannot be applied to an arbitrary run-time
expression. In fact, the compiler does not always produce an \emph{a
priori} interpretation of the literal before invoking the \emcppsgloss{UDL
operator}.

Let's now see how we might create a \emcppsgloss[user defined literal (UDL)]{UDL} for our original
user-defined type, \lstinline!Name!. In this example, the \emcppsgloss[user defined literal (UDL)]{UDL}
``function'' operates on a string literal, rather than a floating-point
literal:

\begin{emcppshiddenlisting}[emcppsbatch=e3]
#include <cstddef>  // (ù{\codeincomments{std::size\_t}}ù)
struct Name
{
public:
    Name(const char*);
};
\end{emcppshiddenlisting}
\begin{emcppslisting}[emcppsbatch=e3]
Name operator""_Nm(const char* n, std::size_t /* length */) { return Name(n); }
    // user-defined literal (UDL) operator for UDL suffix (ù{\codeincomments{\_Nm}}ù)

Name nameField = "Maria"_Nm;  // (ù{\codeincomments{Name}}ù) object having value "Maria"
\end{emcppslisting}
    
\noindent The UDL definition for string literals above, \lstinline!operator""_Nm! in
this case, takes two arguments: a \lstinline!const!~\lstinline!char*!
representing a null-terminated character string and a
\lstinline!std::size_t! representing the length of that string (excluding
the null terminator). In our example, we ignore the second parameter in
the body of the UDL operator, but it must nonetheless be present in the
parameter list.

\emcppsgloss[UDL operator]{UDL operators}, like \emcppsgloss[factory function]{factory functions}, can return a
value of any type, including built-in types. Unit-conversion functions,
for example, often return built-in types normalized to specific units:

\begin{emcppslisting}[emcppsbatch={e4,e5}]
#include <ctime>  // (ù{\codeincomments{std::time\_t}}ù)

constexpr std::time_t minutes(int m) { return m * 60; }    // minutes to seconds
constexpr std::time_t hours(int h)   { return h * 3600; }  // hours to seconds
\end{emcppslisting}
    
\noindent Each of the unit-conversion functions above returns an
\lstinline!std::time_t! (a standard type alias for a built-in integral
type) representing a duration in seconds. We can combine such uniform
quantities initialized from values in disparate units as needed:

\begin{emcppslisting}[emcppsbatch=e4]
std::time_t duration = hours(3) + minutes(15);   // 3.25 hours as seconds
\end{emcppslisting}
    
\noindent Replacing the unit-conversion functions with UDLs allows us to express
the desired value with a more natural-looking syntax. We now define two
new UDL operators defining suffixes \lstinline!_min! for minutes and
\lstinline!_hr! for hours, respectively:

\begin{emcppslisting}[emcppsbatch=e5]
std::time_t operator""_min(unsigned long long m)
{
    return static_cast<std::time_t>(m * 60);  // minutes-to-seconds conversion
}

std::time_t operator""_hr(unsigned long long h)
{
    return static_cast<std::time_t>(h * 3600);  // hours-to-seconds conversion
}

std::time_t duration = 3_hr + 15_min;  // 3.25 hours as seconds
\end{emcppslisting}
    
\noindent We are not done yet. The \emcppsgloss[user defined literal (UDL)]{UDL} language feature was designed with
the idea of providing a convenient syntax for arbitrary literal values
that can also be treated as compile-time constants usable in
\emcppsgloss[constant expression]{constant expressions} such as sizing an array or within a
\lstinline!static_assert!:

\begin{emcppslisting}
int a1[5_hr];                    // Error, (ù{\codeincomments{5\_hr}}ù) is not a compile-time constant.
static_assert(1_min == 60, "");  // Error, (ù{\codeincomments{1\_min}}ù) "  "  "    "      "     "
\end{emcppslisting}
    
\noindent Typical definitions of UDLs will, therefore, also involve another C++11
feature, \lstinline!constexpr! functions (see \featureref{\locationc}{constexprfunc}). By simply adding
\lstinline!constexpr! to the declaration of our UDL operators, we enable
them to be evaluated at compile-time and, hence, usable in
\emcppsgloss[constant expression]{constant expressions}:

\begin{emcppshiddenlisting}[emcppsbatch=e6]
#include <ctime>  // (ù{\codeincomments{std::time\_t}}ù)
\end{emcppshiddenlisting}
\begin{emcppslisting}[emcppsbatch=e6]
constexpr std::time_t operator""_Min(unsigned long long m)
{
    return static_cast<std::time_t>(m * 60);  // minutes-to-seconds conversion
}

constexpr std::time_t operator""_Hr(unsigned long long h)
{
    return static_cast<std::time_t>(h * 3600);  // hours-to-seconds conversion
}

int a2[5_Hr];                    // OK, (ù{\codeincomments{5\_Hr}}ù) is a compile-time constant.
static_assert(1_Min == 60, "");  // OK, (ù{\codeincomments{1\_Min}}ù) " "    "     "      "
\end{emcppslisting}
    
\noindent In short, a UDL operator is a new kind of \emcppsgloss{free operator} that
(1) may be defined with certain specific signatures limited to a subset
of the built-in types and (2) is invoked automatically when used as a
suffix of a built-in literal. There are, however, multiple ways to
define UDL operators --- \emph{precomputed-argument}, \emph{raw}, and
\emph{template} --- each more expressive and complex than the previous.
The many allowed variations on the definitions of \emcppsgloss[UDL operator]{UDL operators}
are elucidated below.

\subsubsection[Restrictions on UDLs]{Restrictions on UDLs}\label{restrictions-on-user-defined-literals-(udls)}

A UDL suffix can be defined for only integer, floating-point, character,
and string literals. Left deliberately unsupported are the two
\emph{Boolean} literals, \lstinline!true! and \lstinline!false!, and the
\emph{pointer} literal, \lstinline!nullptr! (see \featureref{\locationa}{null-pointer-literal-(nullptr)}). These omissions serve to sidestep
lexical ambiguities --- e.g., the Boolean literal, \lstinline!true!,
combined with the \emcppsgloss{UDL suffix} \lstinline!_fact! would be
indistinguishable from the identifier, \lstinline!true_fact!.

We'll refer to the sequence of characters that make up a literal
excluding any suffix as a \emcppsgloss{naked literal} --- e.g., for literal
\lstinline!"abc"_udl!, the \emcppsgloss{naked literal} is \lstinline!"abc"!. UDL
suffixes can be appended to otherwise valid lexical literal tokens only.
That is, appending a UDL suffix to a token that
wouldn't be considered a valid lexical literal without the suffix is not permitted.
Though creating a suffix, \lstinline!_ipv4!, to
represent an IPv4 Internet address consisting of four octets separated
by periods might be tempting, \lstinline!192.168.0.1! would not be a valid lexical token in a
program and, hence, neither would \lstinline!192.168.0.1_ipv4!.
Interestingly, creating a \emcppsgloss[user defined literal (UDL)]{UDL} that
would cause an overflow if interpreted without the suffix \emph{is} permitted. Thus, for
example, the \emcppsgloss[user defined literal (UDL)]{UDL} \lstinline!0x123456789abcdef012345678_verylong!,
which comprises 24 hex digits and the \emcppsgloss{UDL suffix}
\lstinline!_verylong!, would be valid even on an architecture whose native
integers cannot exceed 64 bits (16 hex digits); see \featureref{\locationa}{long-long}.

Note that there are no \emph{negative} numeric literals in C++. The
negative numeric value \lstinline!-123! is represented as two separate
tokens: the negation operator and a (positive) literal, \lstinline!123!.
Similarly, an expression like \lstinline!-3_t1! will attempt to apply the
negation operator to the object of, say, \lstinline!Type1! produced by the
\lstinline!3_t1! \emcppsgloss[user defined literal (UDL)]{UDL}, which in turn comes from passing an
\lstinline!unsigned!~\lstinline!long!~\lstinline!long! having the positive value
3 to the \emcppsgloss{UDL operator} \lstinline!operator""_t1!. Such an
expression will be \emcppsgloss{ill formed} unless there is a negation
operator that operates on \emph{rvalues} of type \lstinline!Type1!. None of
the three forms of \emcppsgloss[UDL operator]{UDL operators} (see \intrarefsimple{prepared-argument-udl-operators}, \intrarefsimple{raw-udl-operators}, and \intrarefsimple{templated-udl-operators}) needs to (or
is able to) handle a \emcppsgloss{naked literal} representing a negative
number.

When two or more string literals appear without intervening tokens, they
are concatenated to form a single string literal. If at least one of
them has a UDL suffix, then the suffix is applied to the concatenation
of the \emcppsgloss{naked literal} strings. If more than one of them has a
suffix, then all such suffixes must be the same \emcppsgloss{UDL suffix};
concatenating string literals having different suffixes is not
permitted, but strings having no suffix may be concatenated with strings
having a \emcppsgloss{UDL suffix}:

\begin{emcppslisting}
#include <cstddef>  // (ù{\codeincomments{std::size\_t}}ù)

struct XStr { /*...*/ };
XStr operator""_X(const char* n, std::size_t length);
XStr operator""_Y(const char* n, std::size_t length);

char a[] = "hello world";            // single native string literal
char b[] = "hello"      " world";    // native equivalent to "hello world"
XStr c   = "hello world"_X;          // user-defined string literal
XStr d   = "hello"_X    " world";    // UDL equivalent to "hello world"_X
XStr e   = "hello"      " world"_X;  //  "      "      "     "     "
XStr f   = "hello"_X    " world"_X;  //  "      "      "     "     "
XStr g   = "hel"_X "lo" " world"_X;  //  "      "      "     "     "
XStr h   = "hello"_X    " world"_Y;  // Error, mixing UDL suffixes (ù{\codeincomments{\_X}}ù) and (ù{\codeincomments{\_Y}}ù)
\end{emcppslisting}
    
\noindent Finally, combining a \emcppsgloss{UDL suffix} with a
second built-in or user-defined suffix on a single token is not possible. Writing
\lstinline!45L_Min!, for example, in an attempt to combine the \lstinline!L!
suffix (for \lstinline!long!) with the \lstinline!_Min! suffix (for the
user-defined minutes suffix described earlier) will
simply yield the undefined and invalid suffix, \lstinline!L_Min!.

\subsubsection[UDL operators]{UDL operators}\label{user-defined-literal-(udl)-operators}

A \emcppsgloss{UDL suffix} (e.g., \lstinline!_udl!) is created by defining a
\emcppsgloss{UDL operator} (e.g., \lstinline!operator""_udl!) that follows a
strict set of rules described in this section. In the declaration and
definition of a \emcppsgloss{UDL operator}, the name of the \emcppsgloss{UDL
suffix} \emph{may} be separated from the quotes by whitespace; in fact,
some older compilers might even \emph{require} such whitespace due to a
defect in the original C++11 specification that has since been
corrected. Thus, for all but the oldest C++11 compilers,
\lstinline!operator""_udl!, \lstinline!operator""!~\lstinline!_udl!, and
\lstinline!operator!~\lstinline!""!~\lstinline!_udl! are all valid spellings of
the same \emcppsgloss{UDL operator} name. Note that whitespace is \emph{not}
permitted between a literal and its suffix in the \emph{use} of a
\emcppsgloss[user defined literal (UDL)]{UDL}. For example, \lstinline!1.2!~\lstinline!_udl! is ill formed;
\lstinline!1.2_udl! must appear as a single token with no whitespace.

A \emcppsgloss[user defined literal (UDL)]{UDL} generally consists of two parts: (1) a valid lexical
literal token and (2) a user-defined suffix. The signature of each
\emcppsgloss{UDL operator} must conform to one of three patterns,
distinguished by the way the compiler supplies the \emcppsgloss{naked
literal} to the \emcppsgloss{UDL operator}:

\begin{enumerate}
\item{\emcppsgloss[prepared-argument UDL operator]{Prepared-argument UDL operator} — The \emcppsgloss{naked literal} is evaluated at compile-time and passed into the operator as a value:
\begin{emcppshiddenlisting}[emcppsbatch=e7]
struct Type1 {};
struct Type2 {};
struct Type3 {};
\end{emcppshiddenlisting}
\begin{emcppslisting}[emcppsbatch=e7]
Type1 operator"" _t1(unsigned long long n);
Type1 t1 = 780_t1;  // Calls (ù{\codeincomments{operator""\_t1(780ULL)}}ù)
\end{emcppslisting}
    }
\item{\emcppsgloss[raw UDL operator]{Raw UDL operator} — The characters that make up the \emcppsgloss{naked literal} are passed to the operator as a \emph{raw}, unevaluated string (for numeric literals only):
\begin{emcppslisting}[emcppsbatch=e7]
Type2 operator"" _t2(const char *token);
Type2 t2 = 780_t2;  // Calls (ù{\codeincomments{operator""\_t2("780")}}ù)
\end{emcppslisting}
    }
\item{\emcppsgloss[templated UDL operator]{Templated UDL operator} — The UDL operator is a template whose parameter list is a variadic sequence of \lstinline!char! values (see \featureref{\locationc}{variadictemplate}) that make up the \emcppsgloss{naked literal} (for numeric literals only):
\begin{emcppslisting}[emcppsbatch=e7]
template <char...> Type3 operator"" _t3();
Type3 t3 = 780_t3;  // Calls (ù{\codeincomments{operator""\_t3<'7', '8', '0'>()}}ù)
\end{emcppslisting}
    }
\end{enumerate}

Each of these three forms of \emcppsgloss[UDL operator]{UDL
operators} are expounded in more detail in its own separate subsection; see \intrarefsimple{prepared-argument-udl-operators}, \intrarefsimple{raw-udl-operators}, and \intrarefsimple{templated-udl-operators}.

When a \emcppsgloss[user defined literal (UDL)]{UDL} is encountered, the compiler prioritizes a
\emcppsgloss{prepared-argument UDL operator} over the other two. Given a
\emcppsgloss[user defined literal (UDL)]{UDL} having suffix \lstinline!_udl!, the compiler will look for
any \lstinline!operator""_udl! in the local scope (\emcppsgloss{unqualified
name lookup}). If, among the operators found, there is a
\emcppsgloss{prepared-argument UDL operator} that exactly matches the type of
the \emcppsgloss{naked literal}, then that \emcppsgloss{UDL operator} is called.
Otherwise, for numeric literals only, the \emcppsgloss[raw UDL operator]{raw} or
\emcppsgloss{templated UDL operator} (only one of which is permitted to exist
for a given suffix) is invoked. This set of lookup rules is deliberately
short and rigid. Importantly, this lookup sequence differs from other
operator invocations in that it does \emph{not} involve \emcppsgloss{overload
resolution} or argument conversions, nor does it employ
\emcppsgloss[argument-dependent lookup (ADL)]{argument-dependent lookup (ADL)} to find operators in
other namespaces.

Although \emcppsgloss[argument-dependent lookup (ADL)]{ADL} is never an issue for \emcppsgloss[user defined literal (UDL)]{UDLs}, common
practice is to gather related \emcppsgloss[UDL operator]{UDL operators} into a namespace
(whose name often contains the word ``literals''). This namespace is
then typically nested within the namespace containing the definitions of
the user-defined types that the \emcppsgloss[UDL operator]{UDL operators} return. These
literals-only nested namespaces enable a user to import, via a single
\lstinline!using! directive, just the literals into their scope, thereby
substantially decreasing the likelihood of name collisions:

\begin{emcppshiddenlisting}[emcppsbatch=e8]
#include <cstddef>  // (ù{\codeincomments{std::size\_t}}ù)
struct Type1 {};
\end{emcppshiddenlisting}
\begin{emcppslisting}[emcppsbatch=e8]
namespace ns1  // namespace containing types returned by UDL operators
{
    struct Type1 { };
    bool check(const Type1&);

    namespace literals  // nested namespace for UDL operators returning (ù{\codeincomments{ns1}}ù) types
    {
        Type1 operator"" _t1(const char*, std::size_t);
    }

    using namespace literals;  // Make literals available in namespace (ù{\codeincomments{ns1}}ù)
}

void test1()  // file scope: finds UDL operator via (ù{\codeincomments{using}}ù) directive
{
    using namespace ns1::literals;  // OK, imports only the *inner* UDL operators
    check("xyzzy"_t1);              // OK, finds (ù{\codeincomments{ns1::check}}ù) via ADL
}
\end{emcppslisting}
    
\noindent To use the \lstinline!_t1! \emcppsgloss{UDL suffix} above, \lstinline!test1! must
somehow be able to find the declaration of its corresponding \emcppsgloss{UDL
operator} locally, which is accomplished by placing the operator in a
nested namespace and importing the entire namespace via a \emcppsgloss[using
directive]{\lstinline!using! directive}. We could have, instead, avoided the nested namespace and
required each needed operator to be imported individually:

\begin{emcppshiddenlisting}[emcppsbatch=e9]
#include <cstddef>  // (ù{\codeincomments{std::size\_t}}ù)
struct Type2 {};
\end{emcppshiddenlisting}
\begin{emcppslisting}[emcppsbatch=e9]
namespace ns2  // namespace defining types returned by non-nested UDL operators
{
    struct Type2 { };
    bool check(const Type2&);

    Type2 operator"" _t2(const char*, std::size_t);  // BAD IDEA: not nested
}

void test2()  // file scope: finds UDL operator via (ù{\codeincomments{using}}ù) declaration
{
    using ns2::operator"" _t2;  // OK, imports just the needed UDL operator
    check("xyzzy"_t2);          // OK, finds (ù{\codeincomments{ns2::check}}ù) via ADL
}
\end{emcppslisting}
    
\noindent When multiple \emcppsgloss[UDL operator]{UDL operators} are provided for a collection of
types, however, the idiom of placing just the \emcppsgloss[UDL operator]{UDL operators} in
a nested namespace (typically incorporating the name ``literals'')
obviates most of the commonly cited ill effects (e.g., accidental
unwanted name collisions) attributed to more general use of
\lstinline!using! directives. In the interest of brevity, we will freely
omit the nested-literal namespaces in expository-only examples.

Finally, despite its use in the Standard for this specific purpose,
there is never a need for a namespace comprising only UDLs to be
declared \lstinline!inline! and doing so is contraindicated; see \featureref{\locatione}{inline-namespaces}.

\subsubsection[Prepared-argument UDL operators]{Prepared-argument UDL operators}\label{prepared-argument-udl-operators}

The \emph{prepared-argument} pattern for \emcppsgloss[UDL operator]{UDL operators} is
supported for all four of the \emcppsgloss[UDL type category]{UDL type categories}: integer,
floating-point, character, and string. If this form of \emcppsgloss{UDL
operator} is selected (see \intrarefsimple{user-defined-literal-(udl)-operators}), the compiler
first determines which of the four \emcppsgloss[UDL type category]{UDL type categories} applies
to the \emcppsgloss{naked literal}, evaluates it (without regard to the
\emcppsgloss{UDL suffix}), and then passes the prepared (i.e., precomputed)
value to the \emcppsgloss{UDL operator}, which further processes its argument
and returns the value of the UDL. Note that the \emcppsgloss{UDL type
category} of the \emcppsgloss{UDL operator} refers only to the \emcppsgloss{naked
literal}; its return value can be any arbitrary type, with or without an
obvious relationship to its type category:

\begin{emcppshiddenlisting}[emcppsbatch=e10]
#include <cstddef>  // (ù{\codeincomments{std::size\_t}}ù)
\end{emcppshiddenlisting}
\begin{emcppslisting}[emcppsbatch=e10]
struct Smile { /* ... */ };  // arbitrary user-defined type

Smile operator"" _fx(long double);  // floating-point literal returning (ù{\codeincomments{Smile}}ù)
float operator"" _ix(unsigned long long);  // integer literal returning (ù{\codeincomments{float}}ù)
int operator"" _sx(const char*, std::size_t);  // string literal returning (ù{\codeincomments{int}}ù)
\end{emcppslisting}
    
\noindent The \emcppsgloss{UDL type category} for a \emcppsgloss{prepared-argument UDL
operator} is determined by the operator's signature; integer and
floating-point \emcppsgloss[UDL operator]{UDL operators} each take a single argument of,
respectively, the largest \emph{unsigned} integer or floating-point type
defined by the language. Multiple \emcppsgloss[prepared-argument UDL
operator]{prepared-argument UDL
operators} may be declared in the same scope for the same \emcppsgloss{UDL
suffix} (e.g., \lstinline!_aa!) and each may return a distinct arbitrary
type (e.g., \lstinline!short!, \lstinline!Smile!):

\begin{emcppslisting}[emcppsbatch=e10]
short operator"" _aa(unsigned long long n);  // integer literal operator
Smile operator"" _aa(long double n);         // floating-point literal operator
bool  operator"" _bb(long double n);         // floating-point literal operator
\end{emcppslisting}
    
\noindent The \emcppsgloss{naked literal} is evaluated as an
\lstinline!unsigned!~\lstinline!long!~\lstinline!long! for integer literals (see \featureref{\locationa}{long-long}) or a
\lstinline!long!~\lstinline!double! for floating-point literals; the prepared
value is then passed via the \lstinline!n! parameter to the \emcppsgloss{UDL
operator}. The compiler does the work of parsing and evaluating the
sequence of digits, radix prefixes (e.g., \lstinline!0! for octal and
\lstinline!0x! for hexadecimal), decimal points, and exponents. C++14
provides additional features related to built-in
literals; see \featureref{\locationb}{binary-literals} and \featureref{\locationb}{digitseparator}:

\begin{emcppslisting}[emcppsbatch=e10]
short v1 = 123_aa;  // OK, invokes (ù{\codeincomments{operator"" \_aa(unsigned long long)}}ù)
Smile v2 = 1.3_aa;  // OK, invokes (ù{\codeincomments{operator"" \_aa(long double)}}ù)
\end{emcppslisting}
    
\noindent There is no \emcppsgloss{overload resolution},
\emcppsgloss{integer-to-floating-point conversion}, or
\emcppsgloss{floating-point-to-integer conversion}, nor are \emcppsgloss[UDL operator]{UDL
operators} having numerically lower precision permitted:

\begin{emcppslisting}[emcppsbatch=e10]
Smile operator"" _cc(double n);  // Error, invalid argument list
\end{emcppslisting}
    
\noindent If lookup does not find a \emcppsgloss{UDL operator} that matches the type
category of the \emcppsgloss{naked literal} exactly, the match fails:

\begin{emcppslisting}[emcppsbatch=e10]
bool v3 = 123_bb;  // Error, unable to find integer literal (ù{\codeincomments{\_bb}}ù)
bool v4 = 1.3_bb;  // OK, invokes (ù{\codeincomments{operator"" \_bb(long double)}}ù)
\end{emcppslisting}
    
\noindent Because the \emcppsgloss{naked literal} is fully evaluated by the compiler,
overflow and precision loss could become issues just as they are for
native literals. These limitations vary by platform, but typical
platforms are limited to 64-bit
\lstinline!unsigned!~\lstinline!long!~\lstinline!long! and 64-bit
\lstinline!long!~\lstinline!double! types in IEEE
754 format:

\begin{emcppslisting}[emcppsbatch=e10]
Smile v5 = 1.2e310_aa;              // Bug, argument evaluates to infinity
Smile v6 = 2.5e-310_aa;             // Bug, argument evaluates to denormalized
short v7 = 0x1234568790abcdef0_aa;  // Error, doesn't fit in any integer type
\end{emcppslisting}
    
\noindent Note that the over-sized integer initializer for \lstinline!v7! in the code snippet above
results in an error on some compilers but only a warning on others.

A rarely-used feature of C++03 is the \emcppsgloss{encoding prefix},
\lstinline!L!, on a character or string literal. The literals \lstinline!'x'!
and \lstinline!"hello"! have (built-in) types \lstinline!char! and
\lstinline!char[6]!, respectively, whereas \lstinline!L'x'! and
\lstinline!L"hello"! have the respective types \lstinline!wchar_t! and
\lstinline!wchar_t[6]!. C++11 added three more string 
\emcppsgloss[encoding prefix]{encoding prefixes}: \lstinline!u! to indicate \emcppsgloss{UTF-16}
with a type of \lstinline!const char16_t*!, \lstinline!U! to indicate \emcppsgloss{UTF-32} 
with a type \lstinline!const char32_t*!, and \lstinline!u8! to indicate \emcppsgloss{UTF-8}
with a type of \lstinline!const char*!.  The \lstinline!u! and \lstinline!U! prefixes can also
be used on character literals; see  \featureref{\locationa}{unicode-string-and-character-literals}.
{\cprotect\footnote{C++17 allows the \lstinline!u8! prefix on
  character literals as well as on string literals, while C++20 changes the type for
  \lstinline!u8! prefixed string and character literals from \lstinline!char! to \lstinline!char8_t!.}}

The four C++11 \emcppsgloss[encoding prefix]{encoding prefixes} for character literals can
each be supported by a distinct \emcppsgloss{UDL operator} signature (e.g.,
\lstinline!_dd! below), each of which might return a distinct, arbitrary
type:

\begin{emcppslisting}[emcppsbatch=e10]
int         operator"" _dd(char     ch);  //  'x'
double      operator"" _dd(char16_t ch);  // u'x'
const char* operator"" _dd(char32_t ch);  // U'x'
Smile       operator"" _dd(wchar_t  ch);  // L'x'
\end{emcppslisting}
    
\noindent Any or all of the above forms can co-exist. A character \emcppsgloss{naked
literal} (e.g., \lstinline!'Q'!) is translated into the appropriate
character type and value in the \emcppsgloss{execution character set} (i.e.,
the set of characters used at run time on the target operating system)
and passed via the \lstinline!ch! parameter to the body of the \emcppsgloss{UDL
operator}. As ever, there are no narrowing or widening conversions, so
the program is ill formed if the precise signature of the needed
\emcppsgloss{UDL operator} is not found:

\begin{emcppslisting}[emcppsbatch=e10]
int    operator"" _ee(char);      //  'x'
double operator"" _ee(char32_t);  // U'x'

int         c1 =  'Q'_ee;  // OK, matches (ù{\codeincomments{char}}ù) argument type
double      c2 = U'Q'_ee;  // OK, matches (ù{\codeincomments{char32\_t}}ù) argument type
const char* c3 = u'Q'_ee;  // Error, no match for (ù{\codeincomments{char16\_t}}ù) argument type
Smile       c4 = L'Q'_ee;  // Error, no match for (ù{\codeincomments{wchar\_t}}ù) argument type
\end{emcppslisting}
    
\noindent Similarly, there are four valid \emcppsgloss{UDL operator} signatures for
string literals in C++11 each of which again might
return a different type:{\cprotect\footnote{C++20 allows a fifth signature for
string literals with the \lstinline!u8! prefix that takes a \lstinline!const char8_t*!.}}

\begin{emcppslisting}[emcppsbatch=e10]
bool  operator"" _dd(const char*     str, std::size_t len);  //   "str"
int   operator"" _dd(const char16_t* str, std::size_t len);  //  u"str"
float operator"" _dd(const char32_t* str, std::size_t len);  //  U"str"
Smile operator"" _dd(const wchar_t*  str, std::size_t len);  //  L"str"
\end{emcppslisting}
    
\noindent The string \emcppsgloss{naked literal} evaluates to a null-terminated
character array. The address of the first element of that array is
passed to the \emcppsgloss{UDL operator} via the \lstinline!str! parameter and
its length (excluding the null terminator) is passed via \lstinline!len!.

To recap, multiple \emcppsgloss[prepared-argument UDL operator]{prepared-argument UDL operators} can co-exist
for a single \emcppsgloss{UDL suffix}, each having a different type category
and each potentially returning a different C++ type. To show another
example, the suffix \lstinline!_s! on a floating-point literal could
return a \lstinline!double! to mean seconds, whereas the same suffix on a
string literal might return a \lstinline!std::string!. Moreover, string
\emcppsgloss[UDL operator]{UDL operators} (and, similarly, character \emcppsgloss[UDL operator]{UDL
operators}) that differ by character type (\lstinline!char!,
\lstinline!char16_t!, and so on) often have different return types.
Similar string \emcppsgloss[UDL operator]{UDL operators} typically --- but not necessarily
--- return similar types, such as \lstinline!std::string! and
\lstinline!std::u16string!, that differ only in their underlying character
type:

\begin{emcppslisting}
#include <string>   // (ù{\codeincomments{std::string}}ù)
#include <utility>  // (ù{\codeincomments{std::pair}}ù)

double operator"" _s(unsigned long long);  // integer UDL operator
double operator"" _s(long double);         // floating-point UDL operator

std::string    operator"" _s(const char*,     std::size_t);  // string UDL
std::u16string operator"" _s(const char16_t*, std::size_t);  //   "     "

double         d = 12_s;       // yields (ù{\codeincomments{double}}ù) having value (ù{\codeincomments{12.0}}ù)
std::u16string w = u"Hola"_s;  // yields (ù{\codeincomments{std::u16string}}ù) having value (ù{\codeincomments{Hola}}ù)
std::string    s = "Hello"_s;  // yields (ù{\codeincomments{std::string}}ù) having value (ù{\codeincomments{Hello}}ù)
\end{emcppslisting}
    
\noindent Note again that these operators are invariably declared
\lstinline!constexpr! (see \featureref{\locationc}{constexprfunc})
because they can \emph{always} be evaluated at compile time because
their arguments are necessarily expressed \emph{literally} in the source
code.

\subsubsection[Raw UDL operators]{Raw UDL operators}\label{raw-udl-operators}

The \emph{raw} pattern for \emcppsgloss[UDL operator]{UDL operators} is supported for only
the integer and floating-point \emcppsgloss[UDL type category]{UDL type categories}. If this
form of \emcppsgloss{UDL operator} is selected (see \intrarefsimple{user-defined-literal-(udl)-operators}), the compiler packages up the \emcppsgloss{naked literal} as an
unprocessed string --- i.e., a sequence of raw characters transferred
from the source --- and passes it to the \emcppsgloss{UDL operator} as a
null-terminated character string. All \emcppsgloss[raw UDL operator]{raw UDL operators} (e.g.,
for suffix \lstinline!_rl! below) have the same signature:

\begin{emcppslisting}
struct Type { /*...*/ };

Type operator"" _rl(const char*);

Type t1 = 425_rl;  // invokes (ù{\codeincomments{operator ""\_rl("425")}}ù)
\end{emcppslisting}
    
\noindent This signature can be distinguished from a \emcppsgloss{prepared-argument UDL
operator} for string literals by the \emph{absence} of a
\lstinline!std::size_t! parameter representing its \emph{length}.

The raw string argument will be verified by the compiler to be a
well-formed integer or floating-point literal token but is otherwise
untouched by the compiler. For any given \emcppsgloss{UDL suffix}, at most
one matching \emcppsgloss{raw UDL operator} may be in scope at a time; hence,
the \emph{return} type cannot vary based on, e.g., whether the
\emcppsgloss{naked literal} contains a decimal point. Such a capability
\emph{is}, however, available; see \intrarefsimple{templated-udl-operators}.

In particular, it might be the case that not all valid tokens accepted
by the compiler will satisfy the \emcppsgloss{narrow contract} offered by a
specific \emcppsgloss{UDL operator}. For example, we can define a \emcppsgloss{UDL
suffix}, \lstinline!_3!, to express base-3 integers using a \emcppsgloss{raw
UDL operator}:

\begin{emcppslisting}[emcppsbatch=e11]
int operator"" _3(const char* digits)
{
    int result = 0;

    while (*digits)
    {
        result *= 3;
        result += *digits - '0';
        ++digits;
    }

    return result;
}
\end{emcppslisting}
    
\noindent We can now test this function at run time using the standard
\lstinline!assert! macro:

\begin{emcppslisting}[emcppsbatch=e11]
#include <cassert>  // standard C (ù{\codeincomments{assert}}ù) macro

void test()
{
    assert( 0 ==   0_3);
    assert( 1 ==   1_3);
    assert( 2 ==   2_3);
    assert( 3 ==  10_3);
    assert( 4 ==  11_3);
    // ...
    assert( 8 ==  22_3);
    assert( 9 == 100_3);
    assert(10 == 101_3);
}
\end{emcppslisting}
    
\noindent Note that in C++14, we could have declared our \lstinline!_3! \emcppsgloss{raw
UDL operator} \lstinline!constexpr! and replaced all of the (runtime)
\lstinline!assert! statements with (compile-time) \lstinline!static_assert!\linebreak[4]%%%%%% 
\mbox{declarations}.

Let's now consider valid lexical integer literals representing values
outside of what would be considered valid of base-3 integers:

\begin{emcppslisting}[emcppsbatch=e11]
int i1 = 22_3;                    // (1) OK, returns (ù{\codeincomments{(int) 8}}ù)
int i2 = 23_3;                    // (2) Bug, returns (ù{\codeincomments{(int) 9}}ù)
int i3 = 21.1_3;                  // (3) Bug, returns (ù{\codeincomments{(int) 58}}ù)
int i4 = 22211100022211100022_3;  // (4) Bug, too big for 32-bit (ù{\codeincomments{int}}ù)
\end{emcppslisting}
    
\noindent In the example code above, (1) is a valid base-3 integer; (2) is a valid
integer literal but contains the digit 3, which is \emph{not} a valid
base-3 digit; (3) is a valid floating-point literal, but the \emcppsgloss{UDL
operator} returns values of only type \lstinline!int!; and (4) is in
principle a valid integer literal but represents a value that is too
large to fit in a 32-bit \lstinline!int!. Because cases (2), (3), and (4)
are valid lexical literals, it is up to the implementation of the
\emcppsgloss{UDL operator} to reject invalid values.

Let's now consider a more robust implementation of a base-3 integer
\emcppsgloss[user defined literal (UDL)]{UDL}, \lstinline!_3b!, that throws an exception when the literal
fails to represent a valid base-3 integer:

\begin{emcppslisting}
#include <stdexcept>  // (ù{\codeincomments{std::out\_of\_range}}ù), (ù{\codeincomments{std::overflow\_error}}ù)
#include <limits>     // (ù{\codeincomments{std::numeric\_limits}}ù)

int operator"" _3b(const char *digits)
{
    int ret = 0;

    for (char c = *digits; c; c = *++digits)
    {
        if ('\'' == c)  // Ignore the C++14 digit separator.
        {
            continue;
        }

        if (c < '0' || '2' < c)  // Reject non-base-3 characters.
        {
            throw std::out_of_range("Invalid base-3 digit");
        }

        if (ret >= (std::numeric_limits<int>::max() - (c - '0')) / 3)
        {
            // Reject if (ù{\codeincomments{3 * ret + (c - '0')}}ù) would overflow.
            throw std::overflow_error("Integer too large");
        }

        ret = 3 * ret + (c - '0');  // Consume (ù{\codeincomments{c}}ù).
    }

    return ret;
}
\end{emcppslisting}
    
\noindent In this implementation of a \emcppsgloss{raw UDL operator} for a suffix
\lstinline!_3b!, the first \lstinline!if! statement looks for the C++14-only digit-separator and ignores it. The second \lstinline!if! statement
looks for characters not in the valid range for base-3 digits; it throws
an \lstinline!out_of_range! exception for cases (2) and (3). The third
\lstinline!if! statement determines whether the computation is about to
overflow; it throws an \lstinline!overflow_error! exception for case (4).
The absence of compiler interpretation makes \emcppsgloss[raw UDL operator]{raw UDL operators}
potentially difficult to write but also very powerful. Interpreting
existing literal characters in a new way and accepting literals that
would otherwise overflow or lose precision can make raw UDLs extremely
expressive; e.g., the base-3 \emcppsgloss{raw UDL operator} shown above could
not be expressed using a \emcppsgloss{prepared-argument UDL operator}. This
expressiveness comes at the cost of needing to parse the token in code,
including performing thorough error detection; even seemingly simple
code, like the base-3 example, can require significant effort to get
just right.

If the intent is to consume integer literals, a \emcppsgloss{raw UDL
operator} needs to handle (i.e., either process or reject) not only the
decimal digits \lstinline!'0'! to \lstinline!'9'!, but also the hex digits
\lstinline!'a'!--\lstinline!'f'! and \lstinline!'A'!--\lstinline!'F'! as well as
radix prefixes \lstinline!0!, \lstinline!0x!, and \lstinline!0X!. In C++14, it
must also handle the \lstinline!0b! (binary) radix prefix and the digit
separator \lstinline!'\''!.  

For floating-point literals, the \emcppsgloss{raw UDL operator} needs to
handle the decimal digits, decimal point, exponent prefix (\lstinline!'e'!
or \lstinline!'E'!), and optional exponent sign (\lstinline!'+'! or
\lstinline!'-'!). It is possible to handle both integer and floating-point
literals in a single \emcppsgloss{raw UDL operator}, provided the return type
is the same for both. In all cases, it is usually wise to reject
\emph{any} unexpected character, including characters that are not
currently legal within numeric literals, in case the set of legal
characters is enlarged in the future, e.g., by adding a new radix.

A \emcppsgloss{raw UDL operator} can be declared \lstinline!constexpr!, but be
aware that the rules for what is allowed in a \lstinline!constexpr!
function differ significantly between C++11 and C++14, as described in \featureref{\locationc}{constexprfunc}. In particular, the
loop-based implementation of \lstinline!operator""_3! (above) can be
declared \lstinline!constexpr! in C++14 but not in C++11, though it is
possible to define a \lstinline!constexpr! \emcppsgloss{UDL operator} in C++11
that has the same behavior by using a recursive implementation. If the
literal is evaluated as part of a constant expression and if the literal
contains errors that would result in exceptions being thrown (i.e., an
invalid character or overflow), the compiler will reject the invalid
literal at compile time. If, however, the literal is not part of a
constant expression, an exception will still be thrown at run time:

\begin{emcppslisting}[emcppsbatch=e11]
constexpr int i4 = 25_3;  // Error, "throw" not allowed in constant expression
          int i5 = 25_3;  // Bug, exception thrown at run time
\end{emcppslisting}
    
\noindent To ensure that every invalid literal is detected at compile time, use
templated UDL operators, as described in \intrarefnopage{templated-udl-operators}.

\subsubsection[Templated UDL operators]{Templated UDL operators}\label{templated-udl-operators}

A \emcppsgloss{templated UDL operator} (known as a \emph{literal operator
template} in the Standard) is a variadic template (see \featureref{\locationc}{variadictemplate}) having a template parameter list
consisting of a pack of an arbitrary number of \lstinline!char! parameters
and an empty runtime parameter list:

\begin{emcppslisting}[emcppsbatch=e12]
struct Type { /*...*/ };

template <char...> Type operator"" _udl();
\end{emcppslisting}
    
\noindent The \emcppsgloss{templated UDL operator} pattern supports only the integer
and floating-point type categories.{\cprotect\footnote{C++20 added
support for user-defined string literal templates, albeit with a
  different syntax.}} If this form of \emcppsgloss{UDL operator} is selected
(see \intrarefsimple{user-defined-literal-(udl)-operators}), the compiler breaks up the
\emcppsgloss{naked literal} into a sequence of raw characters and passes each
one as a separate template argument to the instantiation of the
\emcppsgloss{UDL operator}:

\begin{emcppslisting}[emcppsbatch=e12]
Type t1 = 42.5_udl;  // calls (ù{\codeincomments{operator""\_udl<'4', '2', '.', '5'>()}}ù)
\end{emcppslisting}
    
\noindent As in the case of \emcppsgloss[raw UDL operator]{raw UDL operators}, the raw sequence of
characters will be verified by the compiler to be a well-formed integer
or floating-point literal token, but the \emcppsgloss{UDL operator} must
deduce meaning from those characters. Unlike \emcppsgloss[raw UDL operator]{raw UDL operators},
a \emcppsgloss{templated UDL operator} can return different types based on
the content of the \emcppsgloss{naked literal}. For example, electrical
resistance might be expressed as \lstinline!Resistance<float>! or
\lstinline!Resistance<int>!, where the intention is to express resistance
less than 10 ohms using the \lstinline!float! specialization and larger
resistance using the \lstinline!int! specialization:

\begin{emcppslisting}[emcppsbatch=e12]
template <class T> class Resistance;
template <> class Resistance<int>   { /* ... (resistance >= 10 ohms) */ };
template <> class Resistance<float> { /* ... (resistance <  10 ohms) */ };
\end{emcppslisting}
    
\noindent The \lstinline!_ohms! \emcppsgloss{UDL operator} (in the code snippet below) determines the correct
return type and value based on the characters making up the naked
numeric literal. The compile-time template logic needed to make this
selection requires template metaprogramming, which in turn requires
partial template specialization. Function templates, including
\emcppsgloss[templated UDL operator]{templated UDL operators}, cannot have partial specializations,
so the selection logic is delegated to a helper class template,
\lstinline!MakeResistance!:

\begin{emcppslisting}[emcppsbatch=e12]
template <char C, char... Cs> struct MakeResistance;

template <char... Cs>
constexpr typename MakeResistance<Cs...>::ReturnType
operator"" _ohms() { return MakeResistance<Cs...>::factory(); }
\end{emcppslisting}
    
\noindent The return type is computed by the template metafunction\linebreak[4]%%%%%
\lstinline!MakeResistance<c...>::ReturnType!. In C++14, the return type can
be deduced directly from the \lstinline!return! statement; see \featureref{\locationf}{Function-Return-Type-Deduction}.
The implementation of \lstinline!MakeResistance! uses partial
specializations to detect specific numeric literal patterns:

\begin{emcppslisting}[emcppsbatch=e12]
template <char d0, char... d>
struct MakeResistance
{
    // primary template (for value >= 10)
    using ReturnType = Resistance<int>;
    static constexpr ReturnType factory();
};

template <char d0>
struct MakeResistance<d0>
{
    // specialize for single digit ('0' - '9')
    using ReturnType = Resistance<float>;
    static constexpr ReturnType factory(); // resistance 0 to 9 ohms
};

template <char d0, char... d>
struct MakeResistance<d0, '.', d...>
{
    // specialize for decimal point after first digit (e.g., (ù{\codeincomments{1.23}}ù))
    using ReturnType = Resistance<float>;
    static constexpr ReturnType factory(); // resistance 0.0 to < 10.0
};
\end{emcppslisting}
    
\noindent The primary \lstinline!MakeResistance! template will return a
\lstinline!Resistance<int>! for any sequence of characters that does not
match one of the partial specializations. The first partial
specialization matches a single-digit integer literal (i.e., an integer
value from 0 through 9). The second partial specialization matches a
sequence of two or more characters where the second character is a
decimal point. Thus, any character sequence representing a value less
than 10 will return \lstinline!Resistance<float>!:

\begin{emcppslisting}[emcppsbatch=e12]
Resistance<int>   r1 = 200_ohms;
Resistance<float> r2 = 5_ohms;
Resistance<float> r3 = 2.5_ohms;
\end{emcppslisting}
    
\noindent Note that this simplified example does not recognize floating-point
literals such as \lstinline!12.5! or \lstinline!.04! that don't have their
decimal point as the second character. The description of fixed-point
literals in \intrarefsimple{use-cases-userdeflit} provides a more complete exposition of this sort
of return-type selection, including details of the recursive template
metaprogramming used to determine the return type and its value.

Being template arguments, the characters that make up the \emcppsgloss{naked
literal} are constant expressions and can be used with
\lstinline!static_assert! to force error detection at compile time. Unlike
\emcppsgloss[raw UDL operator]{raw UDL operators}, there is no risk of throwing an exception at
run time, even when initializing a value in a non-\lstinline!constexpr!
context:

%%% JOSH: the comment style here is questionable. Should "Error (compile time)" (x2) be simply "Error,"? 
\begin{emcppslisting}
constexpr auto r4 = 12.5_ohms;  // Error (compile time), (ù{\codeincomments{constexpr}}ù) context
          auto r5 = 12.5_ohms;  // Error (   "      " ), non-(ù{\codeincomments{constexpr}}ù) context
\end{emcppslisting}
    
\noindent Being able to select a context-specific return type and to force
compile-time error checking makes \emcppsgloss[templated UDL operator]{templated UDL operators} the
most expressive pattern for defining \emcppsgloss[UDL operator]{UDL operators}. These
capabilities come at the cost, however, of having to develop them using
the less-than-readable template sublanguage in C++.

\subsubsection[UDLs in the C++14 Standard Library]{UDLs in the C++14 Standard Library}\label{udls-in-the-c++14-standard-library}

This book is primarily about modern C++ \emph{language} features, but a
short description of \emcppsgloss[UDL suffix]{UDL suffixes} in the Standard
\emph{Library} provides context for better understanding and
appreciating the \emcppsgloss[user defined literal (UDL)]{UDL} language feature. These new suffixes
(starting with C++14) make it easier to write software using standard
strings, units of time, and complex numbers. Note that, because these
are \emph{standard} \emcppsgloss[UDL suffix]{UDL suffixes}, their names do not have a
leading underscore.

A native string literal, without a suffix, describes an C-style array of
characters, which decays to a pointer-to-character when passed as a
function argument. The C++ Standard Library has had, from the start,
string classes (\lstinline!std::basic_string!, \lstinline!std::string!, and
\lstinline!std::wstring!) that improve on C-style character arrays by
providing proper copy semantics, equality comparison, variable sizing,
and so on. With the advent of \emcppsgloss[user defined literal (UDL)]{UDLs}, we can finally create
literals of these library string types by utilizing the standard
\lstinline!s! suffix. The \emcppsgloss[UDL operator]{UDL operators} for string literals are in
header file \lstinline!<string>! in namespace\linebreak%%%%%
\lstinline!std::literals::string_literals!:{\cprotect\footnote{C++20 adds \lstinline!char8_t!
and changes the type of \lstinline!s3! in the above example from \lstinline!std::string! to
\lstinline!std::u8string!.}}

\begin{emcppslisting}[emcppsstandards=c++14]
#include <string>  // (ù{\codeincomments{std::basic\_string}}ù), related types, and UDL operators

using namespace std::literals::string_literals;  // (ù{\codeincomments{std::basic\_string}}ù) UDLs
const char*    s1 =   "hello";    // Value decays to (ù{\codeincomments{(const char *) "hello"}}ù).
std::string    s2 =   "hello"s;   // value (ù{\codeincomments{std::string("hello")}}ù)
std::string    s3 = u8"hello"s;   // value (ù{\codeincomments{std::string(u8"hello")}}ù)
std::u16string s4 =  u"hello"s;   // value (ù{\codeincomments{std::u16string(u"hello")}}ù)
std::u32string s5 =  U"hello"s;   // value (ù{\codeincomments{std::u32string(U"hello")}}ù)
std::wstring   s6 =  L"hello"s;   // value (ù{\codeincomments{std::wstring(L"hello")}}ù)
\end{emcppslisting}
    
\noindent Complex numbers can also be expressed using a more natural style,
mimicking the notation used in mathematics. Within namespace
\lstinline!std::literals::complex_literals!, the suffixes \lstinline!i!,
\lstinline!il!, and \lstinline!if! are used to name \lstinline!double!,
\lstinline!long!~\lstinline!double!, and \lstinline!float! imaginary numbers,
respectively. Note that all three suffixes work for both integer and
floating-point literals:

\begin{emcppslisting}[emcppsstandards=c++14]
#include <complex>  // (ù{\codeincomments{std::complex}}ù) and UDL operators

using namespace std::literals::complex_literals; // (ù{\codeincomments{std::complex}ù) UDLs
std::complex<double>      c1 = 2.4 + 3i;    // value (ù{\codeincomments{{2.4, 3.0}}}ù)
std::complex<long double> c2 = 1.2 + 5.1l;  // value (ù{\codeincomments{{1.2L, 5.1L}}}ù)
std::complex<float>       c3 = 0.1f + 2.if;  // value (ù{\codeincomments{{0.1F, 2.0F}}}ù)
\end{emcppslisting}
    
\noindent The time utilities in the standard header, \lstinline!<chrono>!, contain an
elaborate and flexible system of units of duration. Each unit is a
specialization of the class template\linebreak%%%%%
 \lstinline!std::chrono::duration!,
which is instantiated with a representation (either integral or
floating-point) and a ratio relative to seconds. Thus,
\lstinline!duration<long,!~\lstinline!ratio<3600,!~\lstinline!1>>! can represent
an integral number of hours.

The \lstinline!<chrono>! header also defines literal suffixes (in namespace\linebreak%%%%% 
\lstinline!std::literals::chrono_literals!) having familiar names for time
units such as \lstinline!s! for seconds, \lstinline!min! for minutes, and so
on. Integer literals will yield a \lstinline!duration! having an integral
internal representation, and floating-point literals will yield one
having a floating-point internal representation:

\begin{emcppslisting}[emcppsbatch=e13,emcppsstandards=c++14]
#include <chrono>  // (ù{\codeincomments{std::literals::chrono\_literals}}ù)

using namespace std::literals::chrono_literals;  // (ù{\codeincomments{std::chrono::duration}}ù) UDLs
auto d1 = 2h;      // 2 hours   (integral internal representation)
auto d2 = 1.3h;    // 1.3 hours (floating-point internal representation)
auto d3 = 10min;   // 10 minutes (integral)
auto d4 = 30s;     // 30 seconds (integral)
auto d5 = 250ms;   // 250 milliseconds (integral)
auto d6 = 90us;    // 90 microseconds (integral)
auto d7 = 104.ns;  // 104.0 nanoseconds (floating-point)
\end{emcppslisting}
    
\noindent In the example above, the \lstinline!auto! keyword (see \featureref{\locationc}{auto}) is used to allow the compiler to deduce
the correct type from the literal expression. Although simple integer
duration types have convenient aliases such as
\lstinline!std::chrono::hours!, some types do not have a standard name for
the corresponding \lstinline!duration! specialization; e.g., \lstinline!1.2hr!
returns a value of type
\lstinline!std::chrono::duration<T, std::ratio<3600>>! where \lstinline!T! is a signed integer of at least 23 bits and 
where the actual integer representation is implementation-defined.
Naming a duration is even more complex when adding \lstinline!duration!s
together; the resulting \lstinline!duration! type is selected by the
library to minimize loss of precision:

\begin{emcppslisting}[emcppsbatch=e13]
auto d8 = 2h + 35min + 20s;  // integral, 9320 seconds (2:35:20 in seconds)
auto d9 = 2.4s + 100ms;       // floating-point, 2500.0 milliseconds
\end{emcppslisting}
    
\noindent We are certain to see more \emcppsgloss[UDL suffix]{UDL suffixes} defined in future
Standards.

\subsection[Use Cases]{Use Cases}\label{use-cases-userdeflit}

\subsubsection[Wrapper classes]{Wrapper classes}\label{wrapper-classes}

Wrappers can be used to add or remove capabilities for their (often
built-in) underlying type. They assign \emph{meaning} to a type and are
thus useful in preventing programmer confusion or ambiguity in overload
resolution. For example, an inventory-control system might track items
by both part number and model number. Both numbers could be simple
integers, but they have very different meanings. To prevent programming
errors, we create wrapper classes, \lstinline!PartNumber! and
\lstinline!ModelNumber!, each holding an \lstinline!int! value:

\begin{emcppslisting}[emcppsbatch=e14]
class PartNumber
{
    int d_value;

public:
    constexpr PartNumber(int v) : d_value(v) { }
    // ...
};

class ModelNumber
{
    int d_value;

public:
    constexpr ModelNumber(int v) : d_value(v) { }
    // ...
};
\end{emcppslisting}
    
\noindent Neither \lstinline!PartNumber! nor \lstinline!ModelNumber! defines integer
operations such as addition or multiplication, so any attempt to modify
one (other than by assignment) or add two such values would result in a
compile-time error. Moreover, having wrapper classes allows us to
overload on the different types, preventing overload resolution
ambiguities. Without \emcppsgloss[user defined literal (UDL)]{UDLs}, however, we must represent
\lstinline!PartNumber! or \lstinline!ModelNumber! literals by explicitly
casting \lstinline!int! literals to the correct type:

\begin{emcppslisting}[emcppsbatch=e14]
// operations on model and part numbers:
int inventory(ModelNumber n) { int count = 0; /*...*/ return count; }
int inventory(PartNumber n)  { int count = 0; /*...*/ return count; }
void registerPart(const char* shortName, ModelNumber mn, PartNumber pn) { }

PartNumber pn1 = PartNumber(77) + 90;  // Error, no (ù{\codeincomments{operator+(PartNumber, int)}}ù)

int c1 = inventory(77);               // Error, ambiguous overload
int c2 = inventory(ModelNumber(77));  // OK, call (ù{\codeincomments{inventory(ModelNumber)}}ù)
int c3 = inventory(PartNumber(77));   // OK, call (ù{\codeincomments{inventory(PartNumber)}}ù)

void registerParts1()
{
    registerPart("Bolt", PartNumber(77), ModelNumber(77));  // Error, reversed
    registerPart("Bolt", ModelNumber(77), PartNumber(77));  // OK, correct args
}
\end{emcppslisting}
    
\noindent The code above allows the compiler to detect a number of errors that
would be easy to make had part and model numbers been represented as raw
\lstinline!int! values. The attempt at adding to a part number is rejected,
as is the attempt to call \lstinline!inventory! without specifying whether
part-number inventory or model-number inventory is desired. Finally, the
\lstinline!registerPart! function cannot be called with its arguments
accidentally reversed.

We can now create \emcppsgloss[UDL suffix]{UDL suffixes}, \lstinline!_part! and
\lstinline!_model!, to simplify our use of hard-coded part and model
numbers, making the code more readable:

\begin{emcppslisting}[emcppsbatch=e14]
namespace inventory_literals
{
    constexpr ModelNumber operator"" _model(unsigned long long v) { return v; }
    constexpr PartNumber  operator"" _part (unsigned long long v) { return v; }
}

using namespace inventory_literals;  // Make literals available.
int c5 = inventory(77);              // Error, ambiguous overload
int c6 = inventory(77_model);        // OK, call (ù{\codeincomments{inventory(ModelNumber)}}ù).
int c7 = inventory(77_part);         // OK, call (ù{\codeincomments{inventory(PartNumber)}}ù).

void registerParts2()
{
    registerPart("Bolt", 77_part, 77_model); // Error, reversed model & part
    registerPart("Bolt", 77_model, 77_part); // OK, arguments in correct order
}
\end{emcppslisting}
    
\noindent A wrapper class can also be useful for tracking certain compile-time
attributes of a general-purpose type such as \lstinline!std::string!. For
example, a system that reads input from a user must sanitize each input
string before passing it to, for example, a database. Raw input and
sanitized input are both strings, but an unsanitized string must never
be confused for a sanitized one. Thus, we create a wrapper class,
\lstinline!SanitizedString!, that can be constructed only by a member
factory function, to which we give the easily-searchable name
\lstinline!fromRawString!:

\begin{emcppslisting}[emcppsbatch=e15]
#include <string>  // (ù{\codeincomments{std::string}}ù)

class SanitizedString
{
    std::string d_value;

    // private construction from general-purpose string:
    explicit SanitizedString(const std::string& value) : d_value(value) { }

public:
    // factory function:
    static SanitizedString fromRawString(const std::string& rawStr)
    {
        return SanitizedString(rawStr);
    }

    // ...

    // friend operator for concatenating sanitized strings:
    friend SanitizedString operator+(const SanitizedString& s1,
                                     const SanitizedString& s2)
    {
        return SanitizedString(s1.d_value + s2.d_value);
    }

    // ...
};
\end{emcppslisting}
    
\noindent Calling \lstinline!SanitizedString::fromRawString! is deliberately
cumbersome in an attempt to make developers think carefully before using
it. It is, however, \emph{too} cumbersome in situations where the safety
of a literal string is not in question:

\begin{emcppslisting}[emcppsbatch=e15]
std::string getInput();  // Read (unsanitized) string from input.
bool isSafeString(const std::string& s);  // Determine whether (ù{\codeincomments{s}}ù) is safe.

void process(const SanitizedString& instructions);
    // Run the specified (ù{\codeincomments{instructions}}ù).

void processInstructions1()
    // Read instructions from input and process them.
{
    // Read instructions from input.
    std::string instructions = getInput();

    if (isSafeString(instructions))
    {
       // String is considered safe; sanitize it.
       SanitizedString sanInstr = SanitizedString::fromRawString(instructions);
       // ...

       // Prepend a "begin" instruction, then process the instructions.

       process("Instructions = begin\n" + sanInstr);
           // Error, no (ù{\codeincomments{operator+(const char*, SanitizedString)}}ù)

       process(SanitizedString::fromRawString("Instructions = begin\n") +
               sanInstr);
           // OK, but cumbersome
    }
    else
    {
        // ...                  (error handling)
    }
}
\end{emcppslisting}
    
\noindent The first call to \lstinline!process! does not compile because, by design,
we cannot concatenate a raw string and a sanitized string. The second
call works but is unnecessarily cumbersome. Literal strings are
\emph{always} assumed to be safe (if there is proper code review)
because they cannot originate from outside the program; there should be
no need to call \lstinline!SanitizedString::fromRawString!. Again, a
\emcppsgloss[user defined literal (UDL)]{UDL} can make the code more compact and readable:

\begin{emcppslisting}[emcppsbatch=e15]
namespace sanitized_string_literals
{
    SanitizedString operator""_san(const char *str, std::size_t len)
    {
        return SanitizedString::fromRawString(std::string(str, len));
    }
}

void processInstructions2()
    // Read instructions from input and process them.
{
    using namespace sanitized_string_literals;

    // Read instructions from input.
    std::string instructions = getInput();

    if (isSafeString(instructions))
    {
       // String is considered safe; sanitize it.
       SanitizedString sanInstr = SanitizedString::fromRawString(instructions);
       // ...

       // Prepend a "begin" instruction, then process the instructions.

       process("Instructions = begin\n"_san + sanInstr);
           // OK, concatenate two sanitized strings.
    }
    // ...
}
\end{emcppslisting}
    
\noindent This usage shows a case where the \emcppsgloss[user defined literal (UDL)]{UDL} is more than just
convenient; because a \emcppsgloss[user defined literal (UDL)]{UDL} applies \emph{only} to literals, it
is largely immune to accidental misuse.

\subsubsection[User-defined numeric types]{User-defined numeric types}\label{user-defined-numeric-types}

There is sometimes a need to represent an integer of indefinite
magnitude, i.e., where computations are immune from overflow (within the
bounds of available memory). A \lstinline!BigNum! class, along with
associated arithmetic operators, can represent such indefinite-magnitude
integers:

\begin{emcppslisting}[emcppsbatch=e16]
namespace bignum
{
class BigNum
{
    // ...
};

BigNum operator+(const BigNum&);
BigNum operator-(const BigNum&);
BigNum operator+(const BigNum&, const BigNum&);
BigNum operator-(const BigNum&, const BigNum&);
BigNum operator*(const BigNum&, const BigNum&);
BigNum operator/(const BigNum&, const BigNum&);
BigNum abs(const BigNum&);
// ...
\end{emcppslisting}
    
\noindent A \lstinline!BigNum! literal must be able to represent a value larger than
would fit in the largest built-in integral type, so we define the suffix
using a \emcppsgloss{raw UDL operator}:

\begin{emcppslisting}[emcppsbatch=e16]
namespace literals
{
BigNum operator"" _bignum(const char *digits)  // raw literal
{
    BigNum value;
    // ...        (Compute (ù{\codeincomments{BigNum}}ù) from (ù{\codeincomments{digits}}ù).)
    return value;
}
}  // close namespace (ù{\codeincomments{literals}}ù) 

using namespace literals;
}  // close namespace (ù{\codeincomments{bignum}}ù) 

using namespace bignum::literals;  // Make (ù{\codeincomments{\_bignum}}ù) literal available.
bignum::BigNum bnval  = 587135094024263344739630005208021231626182814_bignum;
bignum::BigNum bigone = 1_bignum;  // small value, but still has type (ù{\codeincomments{BigNum}}ù)
\end{emcppslisting}
    
\noindent The \lstinline!BigNum! class is great for large integers, but numbers with
fractional parts have a different problem: The IEEE standard
\lstinline!double! floating-point type cannot exactly represent certain
values, e.g., \lstinline!24692134.03!. When \lstinline!double!s are used to
represent values, long summations may eventually produce an error in the
hundredths place, i.e., the result will be off by \lstinline!.01! or more.
In financial calculations, where values represent money, errors of just
one or two pennies might be unacceptable. For this problem, we turn to
decimal fixed-point (rather than binary floating-point) arithmetic.

A decimal fixed-point representation for a number is one where the
number of decimal places of precision is chosen by the programmer and
fixed at compile time. Within the specified size and precision, every
decimal value can be represented exactly --- e.g., a fixed-point number
with two decimal digits of precision can represent the value
\lstinline!24692134.03! exactly but cannot represent the value
\lstinline!24692134.035!. We'll define our \lstinline!FixedPoint! class as a
template, where the \lstinline!Precision! parameter specifies the number of
decimal places{\cprotect\footnote{A more complete and powerful
fixed-point class template was proposed for standardization in \cite{mcfarlane19}.}}:

\begin{emcppslisting}[emcppsbatch=e17]
#include <limits>  // (ù{\codeincomments{std::numeric\_limits}}ù)
#include <string>  // (ù{\codeincomments{std::string}}ù), (ù{\codeincomments{std::to\_string}}ù)

namespace fixedpoint
{

template <unsigned Precision>
class FixedPoint
{
    long long d_data;  // integral data = value * pow(10, Precision)

public:
    constexpr FixedPoint() : d_data(0) { } // zero value
    constexpr FixedPoint(long long);       // Convert from (ù{\codeincomments{long long}}ù).
    constexpr FixedPoint(double);          // Convert from (ù{\codeincomments{double}}ù).

    // "Raw" constructor: Create a (ù{\codeincomments{FixedPoint}}ù) object with the specified data.
    // No precision adjustment is made to the data.
    constexpr FixedPoint(long long data, std::true_type /*isRaw*/)
        : d_data(data) { }

    friend std::ostream& operator<<(std::ostream& os, const FixedPoint& v)
    {
        std::string str = std::to_string(v.d_data);
        // Insert leading '0's, if needed.
        if (str.length() < Precision)
            str.insert(0, (Precision - str.length()), '0');
        str.insert(str.length() - Precision, 1, '.');
        return os << str;
    }
};
\end{emcppslisting}
    
\noindent Our data representation is a \lstinline!long!~\lstinline!long!, making the
largest value that can be represented
\lstinline!std::numeric_limits<long!~\lstinline!long>::max()!~\lstinline!/!~\lstinline!pow(10,!~\lstinline!Precision)!, assuming an integer \lstinline!pow! function. The output function,
\lstinline!operator<<!, simply converts \lstinline!d_data! to a string and
then inserts the decimal point into the correct location. The special
``raw'' constructor exists so that our \emcppsgloss{UDL operator} can easily
construct a value without losing precision, as we'll see later; the
unused second parameter is a dummy to distinguish it from other
constructors.

We want to define a \emcppsgloss{templated UDL operator} to return a
\lstinline!FixedPoint! type such that, for example, \lstinline!12.34! would
return a value of type \lstinline!FixedPoint<2>! (for the two decimal
places) whereas \lstinline!12.340! would return a value of type
\lstinline!FixedPoint<3>!. We must first define a helper template to
compute both the type and raw value of the fixed point number, given a
sequence of digits:

\begin{emcppslisting}[emcppsbatch=e17]
namespace literals  // fixed-point literals defined in this namespace
{

template <long long rawVal, int precision, char... c>
struct MakeFixedPoint;
\end{emcppslisting}
    
\noindent This helper template will be recursively instantiated; at each level of
recursion \lstinline!rawVal! is the value computed so far,
\lstinline!precision! is the number of decimal places seen so far, and
\lstinline!c...! is the list of literal characters to be consumed. A
special value of \lstinline!-1! for \lstinline!precision! indicates that the decimal
point has not yet been consumed.

The base case of our recursive template is when the parameter pack,
\lstinline!c...!, is empty --- i.e., there are no more characters to
consume. In this case, the computed type is simply
\lstinline!FixedPoint<precision>! and the value of the \emcppsgloss{UDL
operator} is computed from \lstinline!rawVal!. We define the base case as a
partial specialization of \lstinline!MakeFixedPoint! where there is no
character parameter pack:

\begin{emcppslisting}[emcppsbatch=e17]
template <long long rawVal, int precision>
struct MakeFixedPoint<rawVal, precision> {
    // base case when there are no more characters
    using type = FixedPoint<(precision < 0) ? 0 : precision>;
    static constexpr type makeValue() { return { rawVal, std::true_type{} }; }
};
\end{emcppslisting}
    
\noindent The other case is when there are one or more characters yet to be
consumed. The helper must perform error checking for bad input
characters and overflow before consuming the character and instantiating
itself recursively:

\begin{emcppslisting}[emcppsbatch=e17]
template <long long rawVal, int precision, char c0, char... c>
struct MakeFixedPoint<rawVal, precision, c0, c...>
{
private:
    static constexpr long long maxData = std::numeric_limits<long long>::max();
    static constexpr bool      c0isdig = ('0' <= c0 && c0 <= '9');

    // Check for out-of-range characters and overflow.
    static_assert(c0isdig || '\'' == c0 || '.' == c0,
                  "Invalid fixed-point digit");
    static_assert(!c0isdig || (maxData - (c0 - '0')) / 10 >= rawVal,
                  "Fixed-point overflow");

    // (ù{\codeincomments{precision}}ù) is
    // (1) < 0 if a decimal point was not seen,
    // (2) 0   if a decimal point was seen but no digits after the decimal point,
    // (3) > 0 otherwise, incremented once for each digit after the decimal point.
    static constexpr int nextPrecision = ('.' == c0     ?  0 :
                                          precision < 0 ? -1 :
                                          precision + 1);

    // Instantiate this template recursively to consume remaining characters in
    // (ù{\codeincomments{c...}}ù).
    using RecurseType = MakeFixedPoint<(c0isdig ? 10 * rawVal + c0 - '0' :
                                        rawVal), nextPrecision, c...>;

public:
    using                 type = typename RecurseType::type;
    static constexpr type makeValue() { return RecurseType::makeValue(); }
};
\end{emcppslisting}
    
\noindent This specialization consumes one character, \lstinline!c0!, from the
parameter pack. The first\linebreak[4]%%%%%%  
\lstinline!static_assert! checks that
\lstinline!c0! is either a digit, digit separator
(\lstinline!'\''!), 
or decimal point (\lstinline!'.'!). The second
\lstinline!static_assert! checks that the computation is not in danger of
overflowing the maximum value of a \lstinline!long!~\lstinline!long!. The
constant, \lstinline!nextPrecision!, which will be passed to the recursive
instantiation of this template, keeps track of how many digits have been
consumed after the decimal point (or \lstinline!-1! if the decimal point has not yet
been consumed). The \lstinline!RecurseType! alias is the recursive
instantiation of this template with the updated raw value (after
consuming \lstinline!c0!), the updated precision, and the input character
sequence after having dropped \lstinline!c0!. Thus, each recursion gets a
potentially larger \lstinline!rawVal!, a potentially larger
\lstinline!precision!, and a shorter list of unconsumed characters. The
definitions of \lstinline!type! and \lstinline!makeValue! simply defer to the
definitions in the recursive instantiation.

Finally, we define the \lstinline!_fixed! \emcppsgloss{templated UDL operator},
instantiating \lstinline!MakeFixedPoint! with an initial \lstinline!rawVal! of
\lstinline!0!, an initial \lstinline!precision! of \lstinline!-1!, and with a \lstinline!c...!
parameter pack consisting of all of the characters in the \emcppsgloss{naked
literal}:

\begin{emcppslisting}[emcppsbatch=e17]
template <char... c>
constexpr typename MakeFixedPoint<0, -1, c...>::type operator"" _fixed()
{
    return MakeFixedPoint<0, -1, c...>::makeValue();
}

}  // close namespace (ù{\codeincomments{literals}}ù) 
}  // close namespace (ù{\codeincomments{fixedpoint}}ù) 
\end{emcppslisting}
    
\noindent Now the \lstinline!_fixed! suffix can be used for fixed-point
\emcppsgloss[user defined literal (UDL)]{UDLs} where the precision of the returned type is automatically
deduced based on the number of decimal places in the literal. Note that
the literal can be used in a \lstinline!constexpr! context:

\begin{emcppshiddenlisting}[emcppsbatch=e17]
#include <iostream>  // (ù{\codeincomments{std::cout}}ù)
template <unsigned Precision>
std::ostream& operator<<(std::ostream&,const fixedpoint::FixedPoint<Precision>&);
\end{emcppshiddenlisting}
\begin{emcppslisting}[emcppsbatch=e17]
int fixedTest()
{
    using namespace fixedpoint::literals;

    constexpr auto fx1 = 123.45_fixed;   // return type (ù{\codeincomments{FixedPoint<2>}}ù)
    constexpr auto fx2 = 123.450_fixed;  // return type (ù{\codeincomments{FixedPoint<3>}}ù)
    std::cout << fx1 << '\n';            // prints "123.45"
    std::cout << fx2 << '\n';            // prints "123.450"
}
\end{emcppslisting}
    
\noindent An effort is underway to define a standard \emcppsgloss[decimal floating point]{decimal
floating-point} type, which, like our decimal fixed-point type, retains
the benefit of precisely representing decimal fractions but where the
precision is variable at run time. If implemented as a library type, a
\emcppsgloss{UDL suffix} would allow such type to have a natural
representation in code.\footnote{\cite{kuhl12}}

\subsubsection[User-defined string types]{User-defined string types}\label{user-defined-string-types}

A Universally Unique Identifier (UUID) is a 128-bit number that
identifies specific pieces of data in computer systems. It can be
readily expressed as an array of two 64-bit integers:

\begin{emcppshiddenlisting}[emcppsbatch=e18]
#include <cstddef>  // (ù{\codeincomments{std::size\_t}}ù)
\end{emcppshiddenlisting}
\begin{emcppslisting}[emcppsbatch=e18]
class UUIDv4
{
    unsigned long long d_value[2];
    // ...
};
\end{emcppslisting}
    
\noindent A version-4 UUID has a canonical human-readable format consisting of
five groups of hex digits separated by hyphens, e.g.,
``ed66b67a-f593-4def-9a9b-e69d1d6295ef''. Although storing this
representation as a string would be easy, converting it using the
packed, 128-bit integer format of the above \lstinline!UUIDv4! class is
more efficient and convenient. Moreover, UUIDs that are hard-coded into
software are often generated by external tools --- i.e., to identify the
exact product build --- and should therefore be compile-time constants.
Using \emcppsgloss[user defined literal (UDL)]{UDLs}, we can readily express a UUID literal using the
human-readable string format, converting it to a compile-time constant
in the packed format:

\begin{emcppslisting}[emcppsbatch=e18]
namespace uuid_literals
{
    constexpr UUIDv4 operator""_uuid(const char* s, std::size_t len)
    {
        return { /* ... (decode UUID expressed in canonical format) */ };
    }
}

using namespace uuid_literals;
constexpr UUIDv4 buildId = "eeec1114-8078-49c5-93ca-fea6fbd6a280"_uuid;
\end{emcppslisting}
    

\subsubsection[Unit conversions and dimensional units]{Unit conversions and dimensional units}\label{unit-conversions-and-dimensional-units}

\emcppsgloss[user defined literal (UDL)]{UDLs} can be convenient for specifying a unit name on a numeric
literal, providing a concise way both to convert the number to a
normalized unit and to annotate a value's unit within the code. For
example, the standard trigonometric functions all operate on
\lstinline!double! values where angles are expressed in radians. However,
many people are more comfortable with degrees than radians, especially
when expressing the value directly as a hand-written number:

\begin{emcppslisting}[emcppsbatch=e19]
#include <cmath>  // (ù{\codeincomments{std::sin}}ù), (ù{\codeincomments{std::cos}}ù)

constexpr double pi = 3.14159265358979311599796346854;

double s1 = std::sin(30.0);    // Bug, intended (ù{\codeincomments{sin(30 deg)}}ù) but got (ù{\codeincomments{sin(30 rad)}}ù)
double s2 = std::sin(pi / 6);  // OK, returns (ù{\codeincomments{sin(30 deg)}}ù)
\end{emcppslisting}
    
\noindent The normalized unit in this case is a radian, expressed as a
\lstinline!double!, but radians are generally fractions of pi and are thus
inconvenient to write. \emcppsgloss[user defined literal (UDL)]{UDLs} can provide convenient
normalization from degrees or gradians to radians:

\begin{emcppslisting}[emcppsbatch=e19]
namespace trig_literals {

constexpr double operator"" _rad(long double r)  { return r; }
constexpr double operator"" _deg(long double d)  { return pi * d / 180.0; }
constexpr double operator"" _grad(long double d) { return pi * d / 200.0; }

}

using namespace trig_literals;
double s3 = std::sin(30.0_deg);     // OK, returns (ù{\codeincomments{sin(30 deg)}}ù)
double s4 = std::sin(4.7124_rad);   // OK, returns approx (ù{\codeincomments{-1.0}}ù)
double s5 = std::cos(50.0_grad);    // OK, returns (ù{\codeincomments{sin(50 grad) == sin(45 deg)}}ù)
\end{emcppslisting}
    
\noindent Unfortunately, the applicability of this approach to unit normalization
is very limited. First, it is a one-way conversion --- e.g., the
expression \lstinline!std::cout!~\lstinline!<<!~\lstinline!30.0_deg! will print
out \lstinline!0.524!, not \lstinline!30.0!, necessitating a call to a
radians-to-degrees conversion function when a human-readable value is
desired. Second, a \lstinline!double! does not encode any information about
the units that it holds, so \lstinline!double!~\lstinline!inputAngle! doesn't
tell the reader (or program) whether the angle is expected to be input
in degrees or radians.

A much more robust way to use \emcppsgloss[user defined literal (UDL)]{UDLs} to express units is to
define them as part of a comprehensive library of unit classes.
Dimensional quantities (length, temperature, currency, and so on) often
benefit from being represented by \emcppsgloss[dimensional unit type]{dimensional unit types} that
prevent confusion as to both the units and dimension of numeric values.
For example, creating a function to compute kinetic energy from speed
and mass seems simple:

\begin{emcppslisting}[emcppsbatch=e20]
// Return kinetic energy in joules given (ù{\codeincomments{speed}}ù) in mps and (ù{\codeincomments{mass}}ù) in kg.
double kineticE(double speed, double mass)
{
    return speed * speed * mass;
}
\end{emcppslisting}
    
\noindent Yet this simple function can be called incorrectly in numerous ways,
with no compiler diagnostics to help prevent the errors:

\begin{emcppslisting}[emcppsbatch=e20]
double d1   = 15;             // distance in meters
double t1   = 4;              // time in seconds
double s1   = d1 / t1;        // speed in m/s (meters/second)
double m1   = 2045;           // mass in g
double m1Kg = 2045.0 / 1000;  // mass in kg

double x1 = kineticE(d1, m1Kg);  // Bug, distance instead of speed
double x2 = kineticE(m1Kg, s1);  // Bug, arguments reversed
double x3 = kineticE(s1, m1);    // Bug, mass should be in kg, not g
double x4 = kineticE(s1, m1Kg);  // OK, meters per sec and kg units
\end{emcppslisting}
    
\noindent One way to detect some of these errors at compile time is to use a
wrapper for each dimension:

\begin{emcppslisting}[emcppsbatch=e20]
struct Time     { constexpr Time(double sec);        /* ... */ };
struct Distance { constexpr Distance(double meters); /* ... */ };
struct Speed    { constexpr Speed(double mps);       /* ... */ };
struct Mass     { constexpr Mass(double kg);         /* ... */ };
struct Energy   { constexpr Energy(double joules);   /* ... */ };

// Compute speed from distance and time:
Speed operator/(Distance, Time);

Distance d2(15.0);             // distance in meters
Time     t2(4.0);              // time in seconds
Speed    s2(d2 / t2);          // speed in m/s (meters/second)
Mass     m2(2045.0);           // Bug, trying to get g, got kg instead
Mass     m2Kg(2045.0 / 1000);  // OK, mass in kg

Energy kineticE(Speed s, Mass m);
Energy x5 = kineticE(d2, m2Kg); // Error, 1st argument has an incompatible type.
Energy x6 = kineticE(m2Kg, s2); // Error, reversed arguments, incompatible types
Energy x7 = kineticE(s2, m2);   // Bug, mass should be in kg, not g.
Energy x8 = kineticE(s2, m2Kg); // OK, m/s and Kg units
\end{emcppslisting}
    
\noindent Note that the compiler correctly diagnoses an error in the
initializations of \lstinline!x5! and \lstinline!x6! but still fails to
diagnose the unit error in the initialization of \lstinline!x7!.
User-defined literals can amplify the benefits of dimensional unit
classes by adding unit suffixes to numeric literals, eliminating
implicit unit assumptions:

\begin{emcppslisting}[emcppsbatch=e20]
namespace si_literals
{
    constexpr Distance operator"" _m  (long double meters);
    constexpr Distance operator"" _cm (long double centimeters);
    constexpr Time     operator"" _s  (long double seconds);
    constexpr Speed    operator"" _mps(long double mps);
    constexpr Mass     operator"" _g  (long double grams);
    constexpr Mass     operator"" _kg (long double kg);
    constexpr Energy   operator"" _j  (long double joules);
}

using namespace si_literals;
auto d3   = 15.0_m;    // distance in meters
auto t3   = 4.0_s;     // time in seconds
auto s3   = d3 / t3;   // speed in m/s (meters/second)
auto m3   = 2045.0_g;  // mass expressed as g but stored as Kg
auto m3Kg = 2.045_kg;  // mass expressed as kg

Energy x9  = kineticE(s3, m3);    // OK, (ù{\codeincomments{m3}}ù) has been normalized to Kg units.
Energy x10 = kineticE(s3, m3Kg);  // OK, m/s and Kg units
\end{emcppslisting}
    
\noindent Note that there are two \emcppsgloss[user defined literal (UDL)]{UDLs} that yield \lstinline!Distance! and
two \emcppsgloss[user defined literal (UDL)]{UDLs} that yield \lstinline!Mass!. Typically, the internal
representation of each of these dimensional types has a normalized
representation, e.g., \lstinline!Distance! might be represented in meters
internally, so \lstinline!25_cm! would be represented by a \lstinline!double!
data member with value \lstinline!0.25!. It is also possible, however, for
the unit to be stored alongside the value, thus avoiding rounding errors
in certain cases. Better yet, the unit can be encoded as a template
parameter at compile time:

\begin{emcppslisting}
#include <ratio>  // (ù{\codeincomments{std::ratio}}ù)
template <class Ratio> class MassUnit;
using Grams     = MassUnit<std::ratio<1, 1000>>;
using Kilograms = MassUnit<std::ratio<1>>;

namespace unit_literals
{
    constexpr Grams     operator"" _g  (long double grams);
    constexpr Kilograms operator"" _kg (long double kg);
}
\end{emcppslisting}
    
\noindent We now get \emph{different} types for \lstinline!100_g! and
\lstinline!0.1_kg!, but we can define \lstinline!MassUnit! in such a way that
they interoperate. The time-interval units that we saw previously in
{\emcppsgloss[user defined literal (UDL)]{UDLs} in the C++14 Standard Library} provide a taste of what is
possible with this approach.{\cprotect\footnote{Mateusz Pusz explores
the topic of a comprehensive physical units library in \cite{pusz20b}.}}

\subsubsection[Test drivers]{Test drivers}\label{test-drivers}

Because code is easier to maintain when ``magic'' values are expressed
as named constants rather than literal values, a typical program does
not contain many literals (see \intraref{potential-pitfalls-defmemberinit}{overuse-defmemberinit}). The
exception to this general rule is in unit tests where many values are
successively passed to a subsystem to test its behavior with a number of
different input values. For example, we define a \lstinline!Date! class
that supplies a subtraction operator returning the number of days
between two \lstinline!Date!s:

\begin{emcppslisting}[emcppsbatch=e21]
// date.h: (component header file)

class Date
{
    // ...
public:
    constexpr Date(int year, int month, int day);
    // ...
};

int operator-(const Date& lhs, const Date& rhs);
    // Return the number of days from (ù{\codeincomments{rhs}}ù) to (ù{\codeincomments{lhs}}ù).
\end{emcppslisting}
    
\noindent To test the subtraction operator, we need to feed it combinations of
dates and compare the result with the expected result. We do this by
creating an array where each row holds a pair of dates and the expected
result from subtracting them. Due to the large number of hard-coded
values in the array, having a literal representation for the
\lstinline!Date! class would be convenient, even if the author of
\lstinline!Date! did not see fit to provide one:

\begin{emcppslisting}[emcppsbatch=e21]
// date.t.cpp: (component test driver)

#include <date.h>   // (ù{\codeincomments{Date}}ù)
#include <cstdlib>  // (ù{\codeincomments{std::size\_t}}ù)
#include <cassert>  // standard C (ù{\codeincomments{assert}}ù) macro

namespace test_literals
{
    constexpr Date operator"" _date(const char*, std::size_t);
        // UDL to convert date in "yyyy-mm-dd" format to a (ù{\codeincomments{Date}}ù) object
}

void testSubtraction()
{
    using namespace test_literals;  // Import (ù{\codeincomments{\_date}}ù) UDL suffix.

    struct TestRow
    {
        Date lhs;  // left operand
        Date rhs;  // right operand
        int  exp;  // expected result
    };

    const TestRow testData[] =
    {
        { "2021-01-01"_date, "2021-01-01"_date,  0 },
        { "2021-01-01"_date, "2020-12-31"_date,  1 },
        { "2021-01-01"_date, "2021-01-02"_date, -1 },
        // ...
    };

    const std::size_t testDataSize = sizeof(testData) / sizeof(TestRow);

    for (std::size_t i = 0; i < testDataSize; ++i)
    {
        assert(testData[i].lhs - testData[i].rhs == testData[i].exp);
    }
}
\end{emcppslisting}
    

\subsection[Potential Pitfalls]{Potential Pitfalls}\label{potential-pitfalls-defmemberinit}

\subsubsection[Unexpected characters can yield bad values]{Unexpected characters can yield bad values}\label{unexpected-characters-can-yield-bad-values}

\emcppsgloss[raw UDL operator]{Raw UDL operators} and \emcppsgloss[templated UDL operator]{templated UDL operators} must
parse and handle every character from the \emph{union} of the set of
legal characters in integer and floating-point literals, even if the
\emcppsgloss{UDL operator} is expecting only one of the two numeric type
categories. Failure to generate an error for an invalid character is
likely to produce an incorrect value, rather than a program crash or
compilation error:

\begin{emcppslisting}
short operator"" _short(const char *digits)
{
    short result = 0;
    for (; *digits; ++digits)
    {
        result = result * 10 + *digits - '0';
    }

    return result;
}

short s1 = 123_short;   // OK, value 123
short s2 = 123._short;  // Bug, (ù{\codeincomments{.}}ù) treated as digit value (ù{\codeincomments{-2}}ù)
\end{emcppslisting}
    
\noindent Testing only for the \emph{expected} characters and rejecting any others
is better than checking for \emph{invalid} characters and accepting the
rest:

\begin{emcppslisting}
#include <stdexcept>  // (ù{\codeincomments{std::out\_of\_range}}ù)

short operator"" _shrt2(const char *digits)
{
    short result = 0;
    for (; *digits; ++digits)
    {
        if (*digits == '.')
        {
            throw std::out_of_range("Bad digit");  // BAD IDEA
        }

        if (!std::isdigit(*digits))
        {
            throw std::out_of_range("Bad digit");  // BETTER
        }

        result = result * 10 + *digits - '0';
    }

    return result;
}

short s3 = 123_shrt2;    // OK, value 123
short s4 = 123._shrt2;   // Error (detected), throws (ù{\codeincomments{out\_of\_range("Bad digit")}}ù)
short s5 = 0x123_shrt2;  // Error (detected), throws (ù{\codeincomments{out\_of\_range("Bad digit")}}ù)
\end{emcppslisting}
    
\noindent The first \lstinline!if! will catch an unexpected decimal point but not an
unexpected \lstinline!'e'!, \lstinline!'x'!, \lstinline!'\'!, 
and so
on. The second \lstinline!if! will catch all unexpected characters. If a
new radix or other currently illegal character is introduced in a future
Standard, the second \lstinline!if! will avoid processing it incorrectly.
Note that, for example, after \emcppsgloss[user defined literal (UDL)]{UDLs} were added in C++11, both
the \lstinline!0b! radix and the digit separator (\lstinline!'!) were
introduced in C++14, potentially breaking any C++11-compliant
\emcppsgloss{UDL operator} that didn't properly handle those characters.


%%%%%%%% PRODUCTION: May want to soften/neutralize the language on FPPs. 
%\emph{TODO (VR): this annoyance seems quite subjective.}
\subsubsection[Overuse]{Overuse}\label{overuse-defmemberinit}

While UDLs offer conciseness, they aren't always the best way to create
a literal value in a program. Sometimes a regular constructor or
function call is almost as concise and is simpler and more flexible. For
example, \lstinline!deg(90)! is as readable as \lstinline!90_deg! and can be
applied to runtime values as well as to literals. If the constructor for
a type naturally takes two or more arguments, a string \emcppsgloss{UDL
operator} could theoretically parse a comma-separated list of arguments
--- e.g., \lstinline!"(2.0,!~\lstinline!6.0)"_point! to represent a 2-D
coordinate --- but is the literal really easier to read than
\lstinline!Point(2.0,!~\lstinline!6.0)!?

Finally, even for simple cases, consider how \emph{often} a literal is
likely to be used. The use of ``magic numbers'' in code is widely
discouraged. Numeric literals other than \lstinline!-1!, \lstinline!0!,
\lstinline!1!, \lstinline!2!, and \lstinline!10! or string literals other than
\lstinline!""! are usually used only to initialize named constants. The
speed of sound would probably be written as a constant,
\lstinline!speedOfSound!, rather than a literal, \lstinline!343_mps!. The use
of literals to supply the special values used to initialize named
constants provides little benefit by way of overall program readability:

\begin{emcppshiddenlisting}[emcppsbatch=e22]
struct Speed    { constexpr Speed(long double mps) {} };
constexpr Speed operator"" _mps(unsigned long long speed) { return Speed(speed); }
constexpr Speed operator*(long d, const Speed&rhs) { return Speed(d); }
\end{emcppshiddenlisting}
\begin{emcppslisting}[emcppsbatch=e22]
constexpr Speed operator"" _mps(unsigned long long speed);  // meters per second

constexpr Speed speedOfSound1 = 343_mps;   // OK, very clear
constexpr Speed speedOfSound2(343);        // OK, almost as clear

constexpr Speed mach2 = 2 * speedOfSound1; // Literal is irrelevant.
constexpr Speed mach3 = 3 * speedOfSound2; // Literal is irrelevant.
constexpr Speed mach4 = 4 * 343_mps;       // Bad style: "magic" number
\end{emcppslisting}
    
\noindent Often, the most common literal is the one that expresses the notion of
an \emph{empty} or \emph{zero} value. Consider creating constants for
such values, such as
\lstinline!constexpr!~\lstinline!Thing!~\lstinline!EmptyThing{}!, instead of
defining a \emcppsgloss{UDL operator} just to be able to write
\lstinline!""_thing! or \lstinline!0_thing!.


%%%%%%%%%% REMOVE for this edition but keep for perhaps using for later editions. 
%\subsubsection[Obfuscation]{Obfuscation}\label{obfuscation}
%
%\emcppsgloss[user defined literal (UDL)]{UDLs} are mostly \emcppsgloss{syntactic sugar}, although they can
%sometimes help with correctness (e.g., for expressions that should never
%be invoked on user input). As in the case of operator overloading,
%overuse of \emcppsgloss[user defined literal (UDL)]{UDLs} to create ``cute'' syntax can result in
%hard-to-understand code. For example, although
%\lstinline!192.168.0.1_ipv4! is not a valid token (as we mentioned in \ref{}
%{Restrictions on \emcppsgloss[user defined literal (UDL)]{UDLs}), the same token \emph{would be} valid
%(in C++14 and later) if the dots are replaced by digit separators,
%yielding \lstinline!192'168'0'1_ipv4!. Whether such a potentially
%appealing notation is of benefit to users is a matter of debate. If a
%\emcppsgloss[user defined literal (UDL)]{UDL} returns a new interpretation of an existing type, the
%meaning might be muddled rather than clarified.
%
%\emph{TODO (VR): this annoyance seems quite subjective.}
%%%%%%%%%%%%%%%%%%%%%%%%%%%%%%%%%

\subsubsection[Preprocessor surprises]{Preprocessor surprises}\label{preprocessor-surprises}

A string literal with a suffix, e.g., \lstinline!"hello"_wrld!, is a
single token in C++11 but was two tokens, \lstinline!"hello"! and
\lstinline!_wrld!, in previous versions of the language. This change could
result in a subtle difference in meaning, usually resulting in a
compilation error, if \lstinline!_wrld! is a macro:

\begin{emcppslisting}[emcppserrorlines=2]
#define _wrld " world"
const char* s = "hello"_wrld;  // "hello world" in C++03, UDL in C++11
\end{emcppslisting}

\subsubsection[Over-verbose usage]{Over-verbose usage}

One of the main selling points of user-defined literals is enabling developers to write concise and expressive code. However, since literal operators are commonly bundled together in a separate namespace, the required \emcppsgloss[using directive]{\lstinline!using! directive} on the caller side can become a burden if only one or few uses of a literal operator appear in a scope:

\begin{emcppshiddenlisting}[emcppsbatch={e23,e24}]
#include <cstddef>  // (ù{\codeincomments{std::size\_t}}ù)
struct Date {
constexpr Date(const char *) {};
};
namespace test_literals {
constexpr Date operator"" _date(const char *c, std::size_t s) { return Date(c); }
}
bool operator >(const Date&,const Date&);
\end{emcppshiddenlisting}
\begin{emcppslisting}[emcppsbatch=e23]
bool isAfterLunarLanding(Date d)
{
    using namespace test_literals;  // Import (ù{\codeincomments{\_date}}ù) UDL suffix
    return d > "1969-07-20"_date;
}
\end{emcppslisting}

\noindent The authors of the \lstinline!_date\lstinline! UDL should provide its same functionality as a function or constructor to ensure that users do not always require a possibly over-verbose \emcppsgloss[using directive]{\lstinline!using! directive} to achieve their goals:

\begin{emcppslisting}[emcppsbatch=e24]
bool isAfterLunarLanding(Date d)
{
    return d > Date("1969-07-20");
}
\end{emcppslisting}

    

\subsection[Annoyances]{Annoyances}\label{annoyances}

\subsubsection[No conversion from floating-point to integer \lstinline!UDL!]{No conversion from floating-point to integer {\SubsubsecCode UDL}}\label{no-conversion-from-floating-point-to-integer-udl}

Defining a prepared-argument floating-point \emcppsgloss{UDL operator} does
not make the corresponding suffix available to numeric literals that
look like integers or vice versa:

\begin{emcppslisting}
double operator"" _mpg(long double v);

double v1 = 12_mpg;   // Error, no integer UDL operator for (ù{\codeincomments{\_mpg}}ù)
double v2 = 12._mpg;  // OK, floating-point UDL operator for (ù{\codeincomments{\_mpg}}ù) found
\end{emcppslisting}
    
\noindent While this behavior is deliberate, it is important when authoring
\emcppsgloss[UDL operator]{UDL operators} to consider whether integer literals should match
the suffix (even if they are treated as floating-point types) and, if
so, to add appropriate unit tests.

\subsubsection[Potential suffix-name collisions]{Potential suffix-name collisions}\label{potential-suffix-name-collisions}

Using a \emcppsgloss{UDL suffix} requires bringing the corresponding
\emcppsgloss{UDL operator} into the current scope, e.g., by means of a
\lstinline!using! directive. If the scope is large enough and if more than
one imported namespace contains a \emcppsgloss{UDL operator} with the same
name, a name collision can result:

\begin{emcppshiddenlisting}[emcppsbatch=e25]
namespace trig_literals {
bool operator"" _deg(long double d);
}
namespace temperature_literals {
bool operator"" _deg(long double d);
}
\end{emcppshiddenlisting}
\begin{emcppslisting}[emcppsbatch=e25]
using namespace trig_literals;        // (ù{\codeincomments{\_deg}}ù), (ù{\codeincomments{\_rad}}ù), and (ù{\codeincomments{\_grad}}ù) suffixes
using namespace temperature_literals; // colliding (ù{\codeincomments{\_deg}}ù) suffix

auto d = 12.0_deg;  // Error, ambiguous use of suffix, (ù{\codeincomments{\_deg}}ù)
\end{emcppslisting}
    
%\noindent Furthermore, there is no straightforward way to disambiguate this case
%because qualified name lookup does not work for \emcppsgloss[UDL suffix]{UDL suffixes}
%(e.g., \lstinline!12.0trig_literals::_deg! does not compile).

\noindent While it is possible to disambiguate the colliding suffixes via qualified name lookup, the required verbosity defeats the entire purpose of using a UDL in the first place:

\begin{emcppslisting}[emcppsbatch=e25]
auto a = trig_literals::operator""_deg(12.0);
auto b = temperature_literals::operator""_deg(12.0);
\end{emcppslisting}
   

\subsubsection[Easy to confuse raw with string UDL operators]{Easy to confuse raw with string UDL operators}\label{easy-to-confuse-raw-with-string-udl-operators}

A \emcppsgloss{UDL operator} that takes a single
\lstinline!const!~\lstinline!char*! argument is a \emcppsgloss{raw UDL operator}
for numeric literals but is easily confused for a
\emcppsgloss{prepared-argument UDL operator} for string literals:

\begin{emcppslisting}
int operator"" _udl(const char *);

int s = "hello"_udl;  // Error, no match for (ù{\codeincomments{operator""(const char*, size\_t)}}ù)
\end{emcppslisting}
    
\noindent Fortunately, such a problem will typically result in a compile-time
error that is easily \mbox{diagnosed}.

\subsubsection[No templated UDL operators for string literals]{No templated UDL operators for string literals}\label{no-templated-udl-operators-for-string-literals}

\emcppsgloss[templated UDL operator]{Templated UDL operators} are called only for numeric literals;
string literals are limited to the prepared-argument pattern. It is thus
not possible to choose, at compile-time, different return types based on
the contents of a string literal. This limitation was removed in
C++20 with a new syntax.

\subsubsection[No way to parse a leading \lstinline!-! or \lstinline!+!]{No way to parse a leading {\SubsubsecCode -} or {\SubsubsecCode +}}\label{no-way-to-parse-a-leading---or-+}

As described in \intraref{description-userdeflit}{restrictions-on-user-defined-literals-(udls)}, a
\lstinline!-!~or~\lstinline!+! before a numeric literal is a separate negation
operator and not part of the literal. There are occasions, however,
where it would be convenient to know whether the literal value is being
negated. For example, if temperatures are being stored as
\lstinline!double! values in Kelvin and if the \emcppsgloss{UDL suffix}
\lstinline!_C! converts a floating-point literal from Celsius to Kelvin by
calling a function, \lstinline!cToK(double)!, then the expression
\lstinline!-10.0_C! produces the nonsensical value \lstinline!-283.15!
(\lstinline!-cToK(10.0)!) rather than the intuitive value of \lstinline!+263.15!
(\lstinline!cToK(-10.0)!). Parsing the
\lstinline!-!~sign as part of the literal would be nice but is simply not possible.

\subsubsection[Parsing numbers is hard]{Parsing numbers is hard}\label{parsing-numbers-is-hard}

Many of the benefits of \emcppsgloss[raw UDL operator]{raw UDL operators} and \emcppsgloss[templated
UDL operator]{templated
UDL operators} require parsing integer and/or floating-point values
manually, in code, often using recursion. Getting this right is tedious
at best. The Standard Library does not provide much support, especially
for \lstinline!constexpr! parsing.



\subsection[See Also]{See Also}\label{see-also}

\begin{itemize}
\item{\seealsoref{decltype}{\seealsolocationa}is often helpful for deducing the return type of a \emcppsgloss{templated UDL operator}.}
\item{\seealsoref{null-pointer-literal-(nullptr)}{\seealsolocationa}is a keyword that unambiguously denotes the null pointer literal.}
\item{\seealsoref{auto}{\seealsolocationc} can be used to declare a variable to hold the value of a \emcppsgloss[user defined literal (UDL)]{UDL} when the type of the \emcppsgloss[user defined literal (UDL)]{UDL} varies based on its contents.}
\item{\seealsoref{constexprfunc}{\seealsolocationc}allow most \emcppsgloss[user defined literal (UDL)]{UDLs} to be used as part of a constant expression.}
\item{\seealsoref{variadictemplate}{\seealsolocationc}are required for implementing \emcppsgloss[templated UDL operator]{templated UDL operators}.}
\item{\seealsoref{inline-namespaces}{\seealsolocatione}are not recommended for \emcppsgloss[UDL operator]{UDL operators}. However, the C++14 Standard Library puts \emcppsgloss[UDL operator]{UDL operators} into \lstinline!inline! namespaces.}
\end{itemize}

\subsection[Further Reading]{Further Reading}\label{further-reading}

TO DO
