% 12 Feb 2021, revisions in; ready for Josh's code fixes
% 16 Feb 2021 JMB; code compiles

\emcppsFeature{
    short={Raw String Literals},
    long={Syntax for Unprocessed String Contents},
}{raw-string-literals}
\setcounter{table}{0}
\setcounter{footnote}{0}
\setcounter{lstlisting}{0}
%\section[Raw String Literals]{Syntax for Unprocessed String Contents}

Raw string literals obviate the need to escape each contained special
character individually.

\subsection[Description]{Description}\label{description}

A \emph{raw} string literal is a new form of syntax for string literals
that allows developers to embed arbitrary character sequences in a
program's source code, without having to modify them by escaping
individual special characters. As an introductory example, suppose that
we want to write a small program that outputs the following text into the standard output stream:

% this is intentionally left plain - this is not code, this is what we are saying
% we  want to output
\begin{lstlisting}[style=plain]
printf("Hello, %s%c\n", "World", '!');
\end{lstlisting}
    
\noindent In C++03, capturing the line of C code above in a string literal would
require five escape (\lstinline!\!) characters distributed
throughout the string:

\begin{emcppslisting}
#include <iostream>  // (ù{\codeincomments{std::cout}}ù), (ù{\codeincomments{std::endl}}ù)

int main()
{
    std::cout << "printf(\"Hello, %s%c\\n\", \"World\", '!');" << std::endl;
    return 0;  //        ^            ^  ^   ^      ^
               //             escape characters
}
\end{emcppslisting}
    
\noindent If we use C++11's \emph{raw} string-literal syntax, no escaping is
required:

\begin{emcppslisting}
#include <iostream>  // (ù{\codeincomments{std::cout}}ù), (ù{\codeincomments{std::endl}}ù)

int main()
{
    std::cout << R"(printf("Hello, %s%c\n", "World", '!');)" << std::endl;
    return 0;  //^ ^                                      ^
               //   additional raw string-literal syntax (C++11)
}
\end{emcppslisting}
    
\noindent To represent the original character
data as a raw string literal, we typically need only to add a capital \lstinline!R! immediately
(adjacently) before the starting quote (\lstinline!"!) and nest the
character data within parentheses, \lstinline!(!~\lstinline!)! (with some exceptions; see \intrarefsimple{collisions}).  
%\textit{\titleref{collisions}} on page~\pageref{collisions}). 
Sequences of
characters that would be escaped in a regular string literal are instead
interpreted verbatim:

\begin{emcppslisting}
const char s0[] = R"({ "key": "value" })";
    // OK, equivalent to (ù{\codeincomments{"\{ $\backslash$"key$\backslash$": $\backslash$"value$\backslash$" \}"}}ù)
\end{emcppslisting}
    
\noindent In contrast to conventional string literals, \emph{raw} string literals
(1) treat unescaped embedded double quotes (\lstinline!"!) as literal data,
(2) do not interpret special-character escape sequences (e.g.,
\lstinline!\n!, \lstinline!\t!), and (3) interpret both vertical{\cprotect\footnote{To incorporate a newline character into a conventional string literal one must represent that newline using the escape sequence \lstinline!\n!.  Attempting to do so by actually entering a newline into the source (i.e., making the string literal span lines of source code) is an error.}} and horizontal 
whitespace characters present in
the source file as part of the string contents{\cprotect\footnote{In
this example, we assume that all trailing whitespace has been stripped
since even trailing whitespace in a raw literal would be captured.}}:


\begin{emcppslisting}
const char s1[] = R"(line one
line two
    line three)";
    // OK
\end{emcppslisting}
    
\noindent Note that any literal tab characters are treated the same as a
\lstinline!\t! and hence can be problematic, especially when
developers have inconsistent tab settings; see \intraref{potential-pitfalls-rawstringliteral}{unexpected-indentation}.  
%\textit{\titleref{potential-pitfalls-rawstringliteral}: \titleref{unexpected-indentation}} on page~\pageref{unexpected-indentation}. 
Finally, all string literals are concatenated with
adjacent ones in the same way the conventional ones are in C++03:

\begin{emcppslisting}
const char s2[] = R"(line one)"         "\n"
                    "line two"          "\n"
                  R"(    line three)";
    // OK, equivalent to (ù{\codeincomments{"line one$\backslash$nline two$\backslash$n~~~~line three"}}ù)    
\end{emcppslisting}

\noindent These same rules apply to both raw \emph{wide} string literals and raw \emph{Unicode} ones (see \featureref{\locationa}{unicode-string-and-character-literals}) that are introduced by placing their corresponding prefix before the \lstinline!R! character:

\begin{emcppslisting}
const wchar_t  ws [] =  LR"(Raw\tWide\tLiteral)";  
    // Represents "Raw\tWide\tLiteral", not "Raw    Wide    Literal".

const char     u8s[] = u8R"(\U0001F378)";  // Represents "\U0001F378", *not* "(ù{\martini}ù)"
const char16_t us [] =  uR"(\U0001F378)";  //       "         "          "     " 
const char32_t Us [] =  UR"(\U0001F378)";  //       "         "          "     " 
\end{emcppslisting}

\subsubsection[Collisions]{Collisions}\label{collisions}

Although unlikely, the data to be expressed within a
string literal might itself contain the character sequence \lstinline!)"!
embedded within it:

\begin{emcppslisting}
#include <cstdio>   // (ù{\codeincomments{printf}}ù)

void emitHelloWorld()
{
    printf("printf(\"Hello, World!\")");
  //                                ^^
  // The (ù{\codeincomments{)"}}ù) character sequence terminates a typical raw string literal.
}
\end{emcppslisting}
    
\noindent If we use the basic syntax for a \emph{raw} string literal we will get a
syntax error:


\begin{emcppslisting}[emcppsignore={syntax error}]
const char s3[] = R"(printf("printf(\"Hello, World!\")"))";  (ù{\codeincomments{// collision}}ù)
(ù{\codeincomments{//}}ù)                                                   (ù{\codeincomments{\textasciicircum\textasciicircum}}ù)
(ù{\codeincomments{//}}ù)                      (ù{\codeincomments{Syntax error after literal ends}}ù)
\end{emcppslisting}
    
\noindent To circumvent this problem, we could escape every special character in
the string separately, as in C++03, but the result is difficult to read
and error prone:

\begin{emcppslisting}
const char s4[] = "printf(\"printf(\\\"Hello, World!\\\")\")";  // error prone
\end{emcppslisting}
    
\noindent Instead, we can use the extended disambiguation syntax of \emph{raw}
string literals to resolve the issue:

% LORI: The comment is colored properly.  The code is not colored properly.  If we choose to keep the code colored then we need to adjust this example significantly.
%\begin{emcppslisting}
%const char s5[] = R"###(printf("printf(\"Hello, World!\")"))###";  (ù{\codeincomments{\color{skyblue}\ttfamily\itshape // cleaner}}ù)
%\end{emcppslisting}
\begin{emcppslisting}
const char s5[] = R"###(printf("printf(\"Hello, World!\")"))###";  (ù{\codeincomments{// cleaner}}ù)
\end{emcppslisting}

    
\noindent This disambiguation syntax allows us to insert an essentially
arbitrary{\cprotect\footnote{The delimiter of a raw string literal can
comprise any member of the \romeogloss{basic source character set} except
space, backslash, parentheses, and the control characters representing
  horizontal tab, vertical tab, form feed, and new line.}} sequence of
characters between the outermost quote/parenthesis pairs that avoids the
collision with the literal data when taken as a combined sequence (e.g.,
\lstinline!)###"!):

\begin{emcppslisting}
//                       delimiter and parenthesis
//                  v~~~                           ~~~v
const char s6[] = R"xyz(<-- Literal String Data -->)xyz";
//                ^     ^~~~~~~~~~~~~~~~~~~~~~~~~~^
//                |          string contents
//                |
//                | uppercase (ù{\codeincomments{R}}ù)
\end{emcppslisting}
    
\noindent The value of \lstinline!s6! above is equivalent to
\lstinline!"<--!~\lstinline!Literal!~\lstinline!String!~\lstinline!Data!~\lstinline!-->"!.
Every raw string literal comprises these syntactical elements, in order:
\begin{itemize}
\item{an uppercase \lstinline!R!}
\item{the opening double quotes, \lstinline!"!} 
\item{an optional arbitrary sequence of characters called the \emph{delimiter} (e.g., \lstinline!xyz!)}
\item{an opening parenthesis, \lstinline!(!}
\item{the contents of the string}
\item{a closing parenthesis, \lstinline!)!}
\item{the same delimiter (if any) specified previously (i.e., \lstinline!xyz!, not reversed)} 
\item{the closing double quotes, \lstinline!"!}
\end{itemize}

The delimiter can be --- and, in practice, very often is --- an empty character
sequence:

\begin{emcppslisting}
const char s7[] = R"("Hello, World!")";
    // OK, equivalent to (ù{\codeincomments{$\backslash$"Hello, World!$\backslash$"}}ù)
\end{emcppslisting}
    
\noindent A nonempty delimiter (e.g.,~\lstinline|!|) can be used to disambiguate any
appearance of the \lstinline!)"! character sequence within the literal data

\begin{emcppslisting}
const char s8[] = R"!("--- R"(Raw literals are not recursive!)" ---")!";
    // OK, equivalent to (ù{\codeincomments{$\backslash$"--- R$\backslash$"(Raw literals are not recursive!)$\backslash$" ---$\backslash$"}}ù)
\end{emcppslisting}
    
\noindent Had an empty delimiter been used to initialize \lstinline!s8! (above), the
compiler would have produced a (perhaps obscure) compile-time error:

%\begin{emcppslisting}
% // error: decrement of read-only location
%
% const char s8[] = R"("--- R"(Raw literals are not recursive!)" ---")";
% //                                                              ^~
%\end{emcppslisting}
\begin{emcppslisting}[emcppserrorlines={1}]
const char s8a[] = R"("---R"( Raw literals are not recursive!)" ---")";         
    //                                                            ^~            
    // Error, decrement of read-only location
\end{emcppslisting}
    
\noindent In fact, it could turn out that a program with an unexpectedly
terminated \emph{raw} string literal could still be valid and compile
quietly:

\begin{emcppshiddenlisting}[emcppsbatch=e1]
#include <cstdio>   // (ù{\codeincomments{printf}}ù)
\end{emcppshiddenlisting}
\begin{emcppslisting}[emcppsbatch=e1]
void emitPith()
{
    printf(R"("Live-Free, don't (ever)","Die!");
        (ù{\codeincomments{// Prints: "Live-Free, don't (ever}}ù)

    printf((R"("Live-Free, don't (ever)","Die!"));
        (ù{\codeincomments{// Prints: Die!}}ù)
}
\end{emcppslisting}
    
\noindent Fortunately, examples like the one above are invariably contrived, not
accidental.

\subsection[Use Cases]{Use Cases}\label{use-cases}

\subsubsection[Embedding code in a C++ program]{Embedding code in a C++ program}\label{embedding-code-in-a-c++-program}

When a source code snippet needs to be embedded as part of the source
code of a C++ program, use of a \emph{raw} string literal can
significantly reduce the syntactic noise that would otherwise be caused
by repeated escape sequences. As an example, consider a regular
expression for an online shopping product ID represented as a
conventional string literal:

\begin{emcppslisting}
const char* productIdRegex = "[0-9]{5}\\(\".*\"\\)";
    // This regular expression matches strings like (ù{\codeincomments{12345("Product")}}ù).
\end{emcppslisting}
    
\noindent Not only do the backslashes obscure the meaning to human readers, a
mechanical translation is often needed{\cprotect\footnote{Such as when
you want to copy the contents of the string literal into an online
  regular-expression validation tool.}} when transforming between source
and data, introducing significant opportunities for human error. Using a
raw string literal solves these problems:

\begin{emcppslisting}
const char* productIdRegex = R"([0-9]{5}\(".*"\))";
\end{emcppslisting}
    
\noindent Another format that benefits from raw string literals is JSON, due to
its frequent use of double quotes:

\begin{emcppslisting}
const char* testProductResponse = R"!(
{
    "productId": "58215(\"Camera\")",
    "availableUnits": 5,
    "relatedProducts": ["59214(\"CameraBag\")", "42931(\"SdStorageCard\")"]
})!";
\end{emcppslisting}
    
\noindent With a conventional string literal, the JSON string above would require
every occurrence of \lstinline!"! and \lstinline!\! to be escaped
and every new line to be represented as \lstinline!\n!, resulting
in visual noise, less interoperability with other tools accepting or
producing JSON, and heightened risk during manual maintenance.

Finally, raw string literals can also be helpful for
whitespace-sensitive languages, such as Python (but see \intraref{potential-pitfalls-rawstringliteral}{encoding-of-newlines-and-whitespace}): 
%\textit{\titleref{potential-pitfalls-rawstringliteral}: \titleref{encoding-of-newlines-and-whitespace}} on page~\pageref{encoding-of-newlines-and-whitespace}):

\begin{emcppslisting}
const char* testPythonInterpreterPrint = R"(def test():
    print("test printing from Python")
)";
\end{emcppslisting}
    

\subsection[Potential Pitfalls]{Potential Pitfalls}\label{potential-pitfalls-rawstringliteral}

\subsubsection[Unexpected indentation]{Unexpected indentation}\label{unexpected-indentation}

Consistent indentation and formatting of source code facilitates human
comprehension of program structure. Space and tabulation
(\lstinline!\t!) characters{\cprotect\footnote{Always
representing indentation as the precise number of spaces (instead of
tab characters) --- especially when committed to source-code control
  systems --- goes a long way to avoiding this issue.}} used for the
purpose of source code formatting are, however, always interpreted as
part of a raw string literal's contents:

\begin{emcppshiddenlisting}[emcppsbatch={e2,e3,e4}]
#include <iostream>  // (ù{\codeincomments{std::cout}}ù)
\end{emcppshiddenlisting}
\begin{emcppslisting}[emcppsbatch=e2]
void emitPythonEvaluator0(const char* expression)
{
    std::cout << R"(
        def evaluate():
            print("Evaluating...")
            return )" << expression << '\n';
}
\end{emcppslisting}
    
\noindent Despite the intention of the programmer to aid readability by indenting
the above raw string literal consistently with the rest of the code, the
streamed data will contain a large number of spaces (or tabulation
characters), resulting in an invalid Python program:

\begin{lstlisting}[language=python]
        def evaluate():
            print("Evaluating...")
            return (ù{\emph{someExpression}}ù)
# ^~~~~~
# Error: excessive indentation
\end{lstlisting}
    
\noindent Correct Python code would start unindented and then be indented the same number
of spaces (e.g.,~exactly four):

\begin{lstlisting}[language=python]
def evaluate():
    print("Evaluating...")
    return (ù{\emph{someExpression}}ù)
\end{lstlisting}
    
\noindent Correct --- albeit visually jarring --- Python code can be expressed with a
single \emph{raw} string literal, but visualizing the final output requires some effort:

\begin{emcppslisting}[emcppsbatch=e3]
void emitPythonEvaluator1(const char* expression)
{
    std::cout << R"(def evaluate():
    print("Evaluating...")
    return )" << expression << '\n';
}
\end{emcppslisting}
    
\noindent When more explicit control is desired, we can use a mixture of
\romeogloss{raw string literals} and explicit new lines represented as
\romeogloss{conventional string literals}:

\begin{emcppslisting}[emcppsbatch=e4]
void emitPythonEvaluator2(const char *expression)
{
    std::cout <<
        R"(def evaluate():)"                 "\n"
        R"(    print("Evaluating..."))"      "\n"
        R"(    return )" << expression << '\n';
}
\end{emcppslisting}
    

\subsubsection[Encoding of new lines and whitespace]{Encoding of new lines and whitespace}\label{encoding-of-newlines-and-whitespace}

The intent of the feature is that new lines should map to a single
\lstinline!\n! character regardless of how new lines are encoded
in the platform-specific encoding of the source file (e.g., \lstinline!\r\n!). The wording of the C++ Standard, however, is not
entirely clear.\footnote{\cite{miller13}} While all major compiler
implementations act in accordance with the original intent of the
feature, relying on a specific new line encoding may lead to nonportable
code until clarity is achieved.

In a similar fashion, the type of whitespace characters (e.g., tabs
versus spaces) used as part of a raw string literal can be significant.
As an example, consider a unit test verifying that a string representing
the status of the system is as expected:

\begin{emcppshiddenlisting}[emcppsbatch=e5]
#include <string>   // (ù{\codeincomments{std::string}}ù)
#include <cassert>  // standard C (ù{\codeincomments{assert}}ù) macro
struct System
{
    static std::string outputStatus();
};
\end{emcppshiddenlisting}
\begin{emcppslisting}[emcppsbatch=e5]
void verifyDefaultOutput()
{
    const std::string output = System::outputStatus();
    const std::string expected = R"(Current status:
    - No violations detected.)";

    assert(output == expected);
}
\end{emcppslisting}
    
\noindent The unit test might pass for years, until, for instance, the
company's indentation style changes from tabulation characters to
spaces, leading the \lstinline!expected! string to contain spaces instead of tabs, and thus test failures.{\cprotect\footnote{A
well-designed unit test will typically be imbued with \emph{expected
values}, rather than values that were produced by the previous run.
The latter is sometimes referred to as a \romeogloss{benchmark test}, and such tests are often implemented as \emph{diffs} against
a file containing output from a previous run. This file has
presumably been reviewed and is known (believed) to be correct and is sometimes called the \romeogloss{golden file}. Though ill advised, when trying to get a new version of the software to pass the benchmark test and when the precise format of the output of a system
changes subtly, the \romeogloss{golden file} may be summarily jettisoned --- and the new
output installed in its stead --- with little if any detailed review.
Hence, well-designed unit tests will often hard code exactly what is
to be expected (nothing more or less) directly in the
  \romeogloss{test-driver} source code.}}

\subsection[Annoyances]{Annoyances}\label{annoyances}

None so far

\subsection[See Also]{See Also}\label{see-also}

None so far

\subsection[Further Reading]{Further Reading}\label{further-reading}

None so far


