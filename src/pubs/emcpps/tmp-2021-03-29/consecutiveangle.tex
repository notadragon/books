% 13 Jan 2021 copyedits in and proofed

In the context of template argument lists, \lstinline!>>! is parsed as two
separate closing angle brackets.

\subsection[Description]{Description}\label{description}

Prior to C++11, a pair of consecutive right-pointing angle brackets anywhere in
the source code was always interpreted as a bitwise right-shift
operator, making an intervening space mandatory for them to be treated
as separate closing-angle-bracket tokens:

\begin{lstlisting}[language=C++]
// C++03
std::vector<std::vector<int>> v0;   // annoying compile-time error in C++03
std::vector<std::vector<int> > v1;  // OK
\end{lstlisting}

\noindent To facilitate the common use case above, a special rule was added
whereby, when parsing a template-argument expression, \emph{non-nested}
(i.e., within parentheses) appearances of \lstinline!>!, \lstinline!>>!,
\lstinline!>>>!, and so on are to be treated as separate closing angle brackets:

\begin{lstlisting}[language=C++]
// C++11
std::vector<std::vector<int>> v0;               // OK
std::vector<std::vector<std::vector<int>>> v1;  // OK
\end{lstlisting}
    
\subsubsection[Using the greater-than or right-shift operators within template-argument expressions]{Using the greater-than or right-shift operators\\[0.5ex] within template-argument expressions}\label{using-the-greater-than-or-right-shift-operators-within-template-argument-expressions}

For templates that take only type parameters, there's no issue. When the
template parameter is a non-type, however, the greater-than or right-shift operators might be
useful. In the unlikely event that we need either the greater-than
operator (\lstinline!>!) or the right-shift operator (\lstinline!>>!) within a
non-type template-argument expression, we can achieve our goal by
nesting that expression within parentheses:

\begin{lstlisting}[language=C++]
const int i = 1, j = 2;  // arbitrary integer values (used below)

template <int I> class C { /*...*/ };
    // class (ù{\codeincomments{C}}ù) taking non-type template parameter (ù{\codeincomments{I}}ù) of type (ù{\codeincomments{int}}ù)

C<i > j>    a1;  // compile-time error (always has been)
C<i >> j>   b1;  // compile-time error in C++11 (OK in C++03)
C<(i > j)>  a2;  // OK
C<(i >> j)> b2;  // OK
\end{lstlisting}
    

\noindent In the definition of \lstinline!a1! above, the first \lstinline!>! is
interpreted as a closing angle bracket, and the subsequent \lstinline!j! is
(and always has been) a syntax error. In the case of \lstinline!b1!, the
\lstinline!>>! is, as of C++11, parsed as a pair of separate tokens in this
context, so the second \lstinline!>! is now considered an error. For
both \lstinline!a2! and \lstinline!b2!, however, the would-be operators appear
nested within parentheses and thus are blocked from matching any
active open angle bracket to the left of the parenthesized expression.

\subsection[Use Cases]{Use Cases}\label{use-cases}

\subsubsection[Avoiding annoying whitespace when composing template types]{Avoiding annoying whitespace when composing template types}\label{avoiding-annoying-whitespace-when-composing-template-types}

When using nested templated types (e.g.,~nested containers) in C++03,
having to remember to insert an intervening space between trailing angle
brackets added no value. What made it even more galling was that every
popular compiler was able to tell you confidently that you had forgotten
to leave the space. With this new
feature (rather, this repaired defect), we can now render closing angle
brackets contiguously, just like parentheses and square brackets:

\begin{lstlisting}[language=C++]
// OK in both C++03 and C++11
std::list<std::map<int, std::vector<std::string> > > idToNameMappingList;

// OK in C++11, compile-time error in C++03
std::list<std::map<int, std::vector<std::string>>> idToNameMappingList;
\end{lstlisting}
    

\subsection[Potential Pitfalls]{Potential Pitfalls}\label{potential-pitfalls}

\subsubsection[Some C++03 programs may stop compiling in C++11]{Some C++03 programs may stop compiling in C++11}\label{some-c++03-programs-may-stop-working-in-c++11}

If a right-shift operator is used in a template expression, the newer
parsing rules may result in a compile-time error where before there was
none:

\begin{lstlisting}[language=C++]
T<1 >> 5>;  // worked in C++03, compile-time error in C++11
\end{lstlisting}
    
\noindent The easy fix is simply to parenthesize the expression:

\begin{lstlisting}[language=C++]
T<(1 >> 5)>;  // OK
\end{lstlisting}
    
\noindent This rare syntax error is invariably caught at compile time, avoiding undetected surprises at run time.

\subsubsection[The meaning of a C++03 program can, in theory, silently change in C++11]{The meaning of a C++03 program can, in theory, silently change in C++11}\label{the-meaning-of-a-c++03-program-can-(in-theory)-silently-change-in-c++11}

Though pathologically rare, the same valid expression can, in theory, have a different interpretation in C++11 than it had when compiled for C++03.
Consider the case{\cprotect\footnote{Example adapted from \cite{gustedt13}}} where the
\lstinline!>>! token is embedded as part of an expression involving
templates:

\begin{lstlisting}[language=C++]
S<G< 0 >>::c>::b>::a
//   ^~~~~~~
\end{lstlisting}
    
\noindent In the expression above, \lstinline!0!~\lstinline!>>::c! will be interpreted
as a \emph{bitwise right-shift operator} in C++03 but not in C++11. Writing a program that (1) compiles under both C++03 and
C++11 and (2) exposes the difference in parsing rules is possible:

\begin{lstlisting}[language=C++]
enum Outer { a = 1, b = 2, c = 3 };

template <typename> struct S
{
    enum Inner { a = 100, c = 102 };
};

template <int> struct G
{
    typedef int b;
};

int main()
{
    std::cout << (S<G< 0 >>::c>::b>::a) << '\n';
}
\end{lstlisting}
    
\noindent The program above will print \lstinline!100! when compiled for C++03 and
\lstinline!0! for C++11:

\begin{lstlisting}[language=C++]
// C++03

//     (2) instantiation of (ù{\codeincomments{G<0>}}ù)
//    (ù{\codeincomments{$\|$}}ù)~~~~~~~~~~~~
//    (ù{\codeincomments{$\|\:\,|\:\,\|$}}ù)   (4) instantiation of (ù{\codeincomments{S<int>}}ù)
//  ~~(ù{\codeincomments{$\|\downarrow\|$}}ù)~~~~~~~~~~~~~~(ù{\codeincomments{$\downarrow$}}ù)
    S< G< 0 >>::c > ::b >::a
//    ~~(ù{\codeincomments{$\|\,\,\uparrow\,\,\|$}}ù)~~~~~~~~~(ù{\codeincomments{$\uparrow$}}ù)
//      (ù{\codeincomments{$\|\:\,\,\,|\:\,\,\,\|$}}ù) (3) type alias for (ù{\codeincomments{int}}ù)
//      (ù{\codeincomments{$\|$}}ù)~~~~~~~
// (1) bitwise right-shift ((ù{\codeincomments{0 >> 3}}ù))
\end{lstlisting}
    

\begin{lstlisting}[language=C++]
// C++11

//
//
//  (2) compare ((ù{\codeincomments{>}}ù)) (ù{\codeincomments{Inner::c}}ù) and (ù{\codeincomments{Outer::b}}ù)
//  (ù{\codeincomments{$\downarrow$}}ù) ~~~~~~~~~~~~~~~~~~
    S< G< 0 >>::c > ::b >::a
//  (ù{\codeincomments{$\uparrow$}}ù) ~~~~~~~~~
//  (1) instantiation of (ù{\codeincomments{S<G<0>>}}ù)
//
//
\end{lstlisting}
    
\noindent Though theoretically possible, programs that (1) are syntactically valid
in both C++03 and C++11 and (2) have distinct semantics have not
emerged in practice anywhere that we are aware of.

\subsection[Annoyances]{Annoyances}\label{annoyances}

\hspace{\fill}

\subsection[See Also]{See Also}\label{see-also}

\hspace{\fill}

\subsection[Further Reading]{Further Reading}\label{further-reading}

\begin{itemize}
\item{Daveed Vandevoorde, \textit{Right Angle Brackets,} \cite{vandevoorde05}}
\end{itemize}


