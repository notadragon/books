% All code samples that should be tested are going to be written within the
% following environment
%
% Source for initial implementation for dumping to files:
% https://tex.stackexchange.com/questions/284189/how-to-dump-listings-to-a-file
%
% Change the following in existing features:
%    \begin{lstlisting}[language=C++]   \begin{emcppslisting}
%    \end{lstlisting}                   \end{emcppslisting}
%
%\usepackage{currfile}
\makeatletter

% This counter is used to generate unique ids for the files that will be output.
% Ideally it would also be reset per feature, but that is not strictly needed.
\newcounter{emcppslistings}
\setcounter{emcppslistings}{0}

% A batch id for code snippets that should be concatenated together to compile.
\lst@Key{emcppsbatch}{}{\def\emcppsBatch{#1}}

% A reason why this code snippet should not be compiled, such as code that is
% explained outside of the example as an error, or code that is more
% descriptive than actually valid code.  Useless and irrelevant on
% emcppshiddenslistings.
\lst@Key{emcppsignore}{}{\def\emcppsIgnore{#1}}

% Which standards the code example should be compiled for, which will be passed
% to the -std= argument of gcc or clang  This must be on the first snippet in a
% batch, otherwise a default oc "c++11,c++14" will be used.
% Example value [emcppsstandards={c++11,c++17}]
\lst@Key{emcppsstandards}{}{\def\emcppsStandards{#1}}

% Provides a list of line numbers within this snippet that are errors.
% Useless and irrelevant on emcppshiddenslistings.
\lst@Key{emcppserrorlines}{}{\def\emcppsErrorLines{#1}}

% Provides an alternate output file name for this snippet.  Multiple snippets
% in the same batch with the same output file name will be concatenated.
% This is equivalent to a comment "// <filename.(h|cpp)>" in the snippet, but
% is not visible in the text.
\lst@Key{emcppsoutputfile}{}{\def\emcppsOutputFile{#1}}

% This environment should be used for all C++ listings/code examples that
% readers should see.  Note that in other builds this will write snippets to
% files in order to validate compilation.
\lstnewenvironment{emcppslisting}[1][]
{%
%  \lstset{language=C++,#1}%
  \lstset{style=Romeo,#1}%  %% change from josh 5 march 2021
  \stepcounter{emcppslistings}%
} 
{}

% This environment should be used for adding "hidden" parts of a listing batch 
% are needed to make it compile (use [emcppsbatch=...] on all related snippets}
% This should take up no space and nothing should be visible in the text.  Note
% that in other builds this will write snippets to files in order to validate
% compilation.
\lstnewenvironment{emcppshiddenlisting}[1][]
{%
%  \lstset{language=C++,aboveskip=0pt,belowskip=0pt,gobble=1000,#1}%  
    \lstset{style=Romeo,aboveskip=0pt,belowskip=0pt,gobble=1000,#1}%  %% change from josh 5 march 2021
  \stepcounter{emcppslistings}%
}
{}

\makeatother
