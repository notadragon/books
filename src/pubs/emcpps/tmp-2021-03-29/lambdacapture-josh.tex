% 24 March 2021, no longer holding for revisions. Sending to Josh for code review. 
% 24 March 2021 JMB - compile check done
% 25 March 2021, commands updated



\emcppsFeature{
    short={Lambda Captures},
    long={Lambda-Capture Expressions},
}{lambda-capture-expressions}
\setcounter{table}{0}
\setcounter{footnote}{0}
\setcounter{lstlisting}{0}
%\section[Lambda Captures]{Lambda-Capture Expressions}\label{lambda-capture-expressions}

Lambda-capture expressions enable \emcppsgloss{synthetization} (spontaneous implicit creation) of
arbitrary data members within \emcppsgloss[closure]{closures} generated by
lambda expressions (see \featureref{\locationc}{lambda}).

\subsection[Description]{Description}\label{description}

In C++11, lambda expressions can capture variables in the
surrounding scope either \emph{by value} or \emph{by reference}{\cprotect\footnote{We use the familiar (C++11) feature
  \lstinline!auto! (see \featureref{\locationc}{auto-feature}) to deduce a closure's type since there is no
  way to name such a type explicitly.}}:

\begin{emcppslisting}
void test()
{
    int i = 0;
    auto f0 = [i]{ };   // Create a copy of (ù{\codeincomments{i}}ù) in the closure named (ù{\codeincomments{f0}}ù).
    auto f1 = [&i]{ };  // Store a reference to (ù{\codeincomments{i}}ù) in the closure named (ù{\codeincomments{f1}}ù).
}
\end{emcppslisting}
    
\noindent Although one could specify \emph{which} and \emph{how} existing
variables were captured, the programmer had no control over the creation
of new variables within a \emcppsgloss{closure}. C++14 extends the
\emcppsgloss[lambda introducer]{lambda-introducer} syntax to support implicit creation of
arbitrary data members inside a \emcppsgloss{closure} via either \emcppsgloss{copy
initialization} or \emcppsgloss{list initialization}:

\begin{emcppslisting}[emcppsstandards={c++14}]
auto f2 = [i = 10]{ /* body of closure */ };
    // Synthesize an (ù{\codeincomments{int}}ù) data member, (ù{\codeincomments{i}}ù), initialized with (ù{\codeincomments{10}}ù) in the closure.

auto f3 = [c{'a'}]{ /* body of closure */ };
    // Synthesize a (ù{\codeincomments{char}}ù) data member, (ù{\codeincomments{c}}ù), initialized with (ù{\codeincomments{'a'}}ù) in the closure.
\end{emcppslisting}
    
\noindent Note that the identifiers \lstinline!i! and \lstinline!c! above do not refer
to any existing variable; they are specified by the programmer creating
the closure. For example, the \emcppsgloss{closure} type assigned (i.e.,
bound) to \lstinline!f2! (above) is similar in functionality to an
\emcppsgloss{invocable} \lstinline!struct! containing an \lstinline!int! data
member:

\begin{emcppslisting}[emcppsstandards={c++14}]
// pseudocode
struct f2LikeInvocableStruct
{
    int i = 10;  // The type (ù{\codeincomments{int}}ù) is deduced from the initialization expression.
    auto operator()() const { /* closure body */ }  // The (ù{\codeincomments{struct}}ù) is invocable.
};
\end{emcppslisting}
    
\noindent The type of the data member is deduced from the initialization
expression provided as part of the capture in the same vein as
\lstinline!auto! (see \featureref{\locationc}{auto-feature}) type deduction; hence, it's not possible to
synthesize an uninitialized \emcppsgloss{closure} data member:

\begin{emcppslisting}
auto f4 = [u]{ };    // Error, (ù{\codeincomments{u}}ù) initializer is missing for lambda capture.
auto f5 = [v{}]{ };  // Error, (ù{\codeincomments{v}}ù)'s type cannot be deduced.
\end{emcppslisting}
    
\noindent It is possible, however, to use variables outside the scope of the
lambda as part of a lambda-capture expression (even capturing them \emph{by
reference} by prepending the \lstinline!&! token to the name of the
synthesized data member):

\begin{emcppslisting}[emcppsstandards={c++14}]
int i = 0;  // zero-initialized (ù{\codeincomments{int}}ù) variable defined in the enclosing scope

auto f6 = [j   = i]{ };  // OK, the local (ù{\codeincomments{j}}ù) data member is a copy of (ù{\codeincomments{i}}ù).
auto f7 = [&ir = i]{ };  // OK, the local (ù{\codeincomments{ir}}ù) data member is an alias to (ù{\codeincomments{i}}ù).
\end{emcppslisting}
    
\noindent Though capturing \emph{by reference} is possible, enforcing \lstinline!const! on a lambda-capture expression is not:

\begin{emcppslisting}[emcppsstandards={c++14}]
auto f8 = [const i = 10]{ };                    // Error, invalid syntax
auto f9 = [const auto i = 10]{ };               // Error, invalid syntax
auto fA = [i = static_cast<const int>(10)]{ };  // OK, (ù{\codeincomments{const}}ù) is ignored.
\end{emcppslisting}
    
\noindent The initialization expression is evaluated during the \emph{creation} of
the closure, not its \emph{invocation}:

\begin{emcppslisting}[emcppsstandards={c++14}]
#include <cassert>  // standard C (ù{\codeincomments{assert}}ù) macro

void g()
{
    int i = 0;

    auto fB = [k = ++i]{ };  // (ù{\codeincomments{++i}}ù) is evaluated at creation only.
    assert(i == 1);  // OK

    fB();  // Invoke (ù{\codeincomments{fB}}ù) (no change to (ù{\codeincomments{i}}ù)).
    assert(i == 1);  // OK
}
\end{emcppslisting}
    
\noindent Finally, using the same identifier as an existing
variable is possible for a synthesized capture, resulting in the original variable
being \emcppsgloss{shadowed} (essentially hidden) in the lambda expression's
body but not in its \emcppsgloss{declared interface}. In the example below,
we use the (C++11) compile-time operator
\lstinline!decltype! (see \featureref{\locationa}{decltype}) to infer the C++ type from the
initializer in the capture to create a parameter of that same type as
that part of its \emcppsgloss{declared interface}{\cprotect\footnote{Note
that, in the shadowing example defining \lstinline!fC!, GCC version 10.x
incorrectly evaluates \lstinline!decltype(i)! inside the body of the
lambda expression as \lstinline!const!~\lstinline!char!, rather than
  \lstinline!char!; see \intraref{potential-pitfalls-lambdacapture}{forwarding-an-existing-variable-into-a-closure-always-results-in-an-object-(never-a-reference)}.}}{\cprotect\footnote{Here we are using the (C++14) variable
  template (see \featureref{\locationb}{variable-templates}) version of the standard \lstinline!is_same! metafunction where \lstinline!std::is_same<A,!~\lstinline!B>::value! is replaced with
  \lstinline!std::is_same_v<A,!~\lstinline!B>!.}}:

\begin{emcppslisting}[emcppserrorlines={10},emcppsstandards={c++14}]
#include <type_traits>  // (ù{\codeincomments{std::is\_same}}ù)

int i = 0;

auto fC = [i = 'a'](decltype(i) arg)
{
    static_assert(std::is_same<decltype(arg), int>::value, "");
         // (ù{\codeincomments{i}}ù) in the interface (same as (ù{\codeincomments{arg}}ù)) refers to the (ù{\codeincomments{int}}ù) parameter.

    static_assert(std::is_same<decltype(i), char>::value, "");
        // (ù{\codeincomments{i}}ù) in the body refers to the (ù{\codeincomments{char}}ù) data member deduced at capture.
};
\end{emcppslisting}
    
\noindent Notice that we have again used \lstinline!decltype!, in conjunction with
the standard \lstinline!is_same! metafunction (which is \lstinline!true! if
and only if its two arguments are the same C++ type). This time, we're using \lstinline!decltype! to
demonstrate that the type (\lstinline!int!), extracted from the local
variable \lstinline!i! within the declared-interface portion of
\lstinline!fC!, is distinct from the type (\lstinline!char!) extracted from
the \lstinline!i! within \lstinline!fC!'s body. In other words, the effect
of initializing a variable in the capture portion of the lambda is to
hide the name of an existing variable that would otherwise be accessible
in the lambda's body.{\cprotect\footnote{Also note that, since the
deduced \lstinline!char! member variable, \lstinline!i!, is not materially
used (\emcppsgloss{ODR-used}) in the body of the lambda expression assigned
(bound) to \lstinline!fC!, some compilers, e.g., Clang, may warn:

\begin{lstlisting}[language=bash,style=footcodeplain]
warning: lambda capture 'i' is not required to be captured for this use
\end{lstlisting}
      }}

\pagebreak%%%%%%%
\subsection[Use Cases]{Use Cases}\label{use-cases-lambdacapture}

\subsubsection[Moving (as opposed to copying) objects into a closure]{Moving (as opposed to copying) objects into a closure}\label{moving-(as-opposed-to-copying)-objects-into-a-closure}

Lambda-capture expressions can be used to \emph{move} (see
\featureref{\locationc}{Rvalue-References}) an existing variable into a
closure{\cprotect\footnote{Though possible, it is surprisingly difficult
in C++11 to \emph{move} from an existing variable into a closure.
Programmers are either forced to pay the price of an unnecessary copy or to employ esoteric and fragile techniques, such as writing a wrapper
that hijacks the behavior of its copy constructor to do a \emph{move}
instead:

\begin{emcppslisting}[style=footcode]
#include <utility>  // (ù{\codeincomments{std::move}}ù)
#include <memory>   // (ù{\codeincomments{std::unique\_ptr}}ù)

template <typename T>
struct MoveOnCopy  // wrapper template used to hijack copy ctor to do move
{
    T d_obj;

    MoveOnCopy(T&& object) : d_obj{std::move(object)} { }
    MoveOnCopy(MoveOnCopy& rhs) : d_obj{std::move(rhs.d_obj)} { }
};

void f()
{
    std::unique_ptr<int> handle{new int(100)};  // move-only
        // Create an example of a handle type with a large body.

    MoveOnCopy<decltype(handle)> wrapper(std::move(handle));
        // Create an instance of a wrapper that moves on copy.

    auto &&lambda = [wrapper](){ /* use (ù{\codeincomments{wrapper.d\_obj}}ù) */ };
        // Create a "copy" from a wrapper that is captured by value.
}
\end{emcppslisting}
    
\noindent In the example above, we make use of the bespoke (``hacked'')
\lstinline!MoveOnCopy! class template to wrap a movable object;
when the lambda-capture expression tries to \emph{copy} the wrapper (\emph{by value}),
the wrapper in turn \emph{moves} the wrapped
  \lstinline!handle! into the body of the closure.}} (as opposed to
capturing it \emph{by copy} or \emph{by reference}). As an example of
\emph{needing} to move from an existing object into a closure, consider
the problem of accessing the data managed by
\emcppsgloss[std::uniqueptr]{\lstinline!std::unique_ptr!} (movable but not copyable) from a
separate thread --- for example, by enqueuing a task in a \emcppsgloss{thread
pool}:

\begin{emcppshiddenlisting}[emcppsbatch=e1,emcppsstandards={c++14}]
#include <memory>   // (ù{\codeincomments{std::unique\_ptr}}ù)
struct ThreadPool {
  using Handle = int;
  template <typename T>
  Handle enqueueTask(T&& t) { return 0; }
};
ThreadPool& getThreadPool();
struct Dataset {};
int processDataset(const std::unique_ptr<Dataset> &);
\end{emcppshiddenlisting}
\begin{emcppslisting}[emcppsbatch=e1]
ThreadPool::Handle processDatasetAsync(std::unique_ptr<Dataset> dataset)
{
    return getThreadPool().enqueueTask([data = std::move(dataset)]
    {
        return processDataset(data);
    });
}
\end{emcppslisting}
    
\noindent As illustrated above, the \lstinline!dataset! smart pointer is moved into
the closure passed to\linebreak[4] \mbox{\lstinline!enqueueTask!} by leveraging lambda-capture
expressions --- the \emcppsgloss[std::uniqueptr]{\lstinline!std::unique_ptr!} is \emph{moved}
to a different thread because a copy would have not been possible.

\subsubsection[Providing mutable state for a closure]{Providing mutable state for a closure}\label{providing-mutable-state-for-a-closure}

Lambda-capture expressions can be useful in conjunction with
\lstinline!mutable! lambda expressions to provide an initial state that
will change across invocations of the closure. Consider, for instance,
the task of logging how many TCP packets have been received on a socket
(e.g., for debugging or monitoring purposes){\cprotect\footnote{In
this example, we are making use of the (C++11) \lstinline!mutable! feature
  of lambdas to enable the counter to be modified on each invocation.}}:

\begin{emcppshiddenlisting}[emcppsbatch=e2]
#include <iostream>  // (ù{\codeincomments{std::cout}}ù)
struct TcpSocket {
    TcpSocket(int s);
    template <typename T>
    void onPacketReceived(T&& t) {}
};

\end{emcppshiddenlisting}
\begin{emcppslisting}[emcppsbatch=e2,emcppsstandards={c++14}]
void listen()
{
    TcpSocket tcpSocket(27015);  // some well-known port number
    tcpSocket.onPacketReceived([counter = 0]() mutable
    {
        std::cout << "Received " << ++counter << " packet(s)\n";
        // ...
    });
}
\end{emcppslisting}
    
\noindent Use of \lstinline!counter!~\lstinline!=!~\lstinline!0! as part of the
\emcppsgloss{lambda introducer} tersely produces a \emcppsgloss{function object}
that has an internal counter (initialized with zero), which is
incremented on every received packet. Compared to, say, capturing a
\lstinline!counter! variable \emph{by reference} in the closure, the solution
above limits the scope of \lstinline!counter! to the body of the lambda
expression and ties its lifetime to the closure itself, thereby
preventing any risk of dangling references.

\subsubsection[Capturing a modifiable copy of an existing \lstinline!const! variable]{Capturing a modifiable copy of an existing {\SubsubsecCode const} variable}\label{capturing-a-modifiable-copy-of-an-existing-const-variable}

Capturing a variable \emph{by value} in C++11 does allow the
programmer to control its \lstinline!const! qualification; the generated
closure data member will have the same \lstinline!const! qualification as
the captured variable, irrespective of whether the lambda is decorated
with \lstinline!mutable!:

\begin{emcppshiddenlisting}[emcppsbatch=e3,emcppsstandards={c++14}]
#include <type_traits>   // (ù{\codeincomments{std::is\_same}}ù)
\end{emcppshiddenlisting}
\begin{emcppslisting}[emcppsbatch=e3]
void f()
{
    int i = 0;
    const int ci = 0;

    auto lc = [i, ci]             // This lambda is not decorated with (ù{\codeincomments{mutable}}ù).
    {
        static_assert(std::is_same<decltype(i), int>::value, "");
        static_assert(std::is_same<decltype(ci), const int>::value, "");
    };

    auto lm = [i, ci]() mutable       // Decorating with (ù{\codeincomments{mutable}}ù) has no effect.
    {
        static_assert(std::is_same<decltype(i), int>::value, "");
        static_assert(std::is_same<decltype(ci), const int>::value, "");
    };
}
\end{emcppslisting}
    
\noindent In some cases, however, a lambda capturing a \lstinline!const! variable
\emph{by value} might need to modify that value when invoked. As an
example, consider the task of comparing the output of two Sudoku-solving
algorithms, executed in parallel:

\begin{emcppshiddenlisting}[emcppsbatch=e4,emcppsstandards={c++14}]
struct Puzzle {};
Puzzle generateRandomSudokuPuzzle();
struct ThreadPool {
  using Handle = int;
  template <typename T>
  Handle enqueueTask(T&& t) { return 0; }
};
ThreadPool& getThreadPool();
struct NaiveAlgorithm {};
struct FastAlgorithm {};
template <typename U, typename V>
void waitForCompletion(U&&,V&&);
\end{emcppshiddenlisting}
\begin{emcppslisting}[emcppsbatch=e4]
template <typename Algorithm> void solve(Puzzle&);
    // This (ù{\codeincomments{solve}}ù) function template mutates a Sudoku grid in place to solution.

void performAlgorithmComparison()
{
    const Puzzle puzzle = generateRandomSudokuPuzzle();
        // (ù{\codeincomments{const}}ù)-correct: (ù{\codeincomments{puzzle}}ù) is not going to be mutated after being
        // randomly generated.

    auto task0 = getThreadPool().enqueueTask([puzzle]() mutable
    {
        solve<NaiveAlgorithm>(puzzle);  // Error, (ù{\codeincomments{puzzle}}ù) is (ù{\codeincomments{const}}ù)-qualified.
        return puzzle;
    });

    auto task1 = getThreadPool().enqueueTask([puzzle]() mutable
    {
        solve<FastAlgorithm>(puzzle);  // Error, (ù{\codeincomments{puzzle}}ù) is (ù{\codeincomments{const}}ù)-qualified.
        return puzzle;
    });

    waitForCompletion(task0, task1);
    // ...
}
\end{emcppslisting}
    
\noindent The code above will fail to compile as capturing \lstinline!puzzle! will
result in a \lstinline!const!-qualified closure data member, despite the
presence of \lstinline!mutable!. A convenient workaround is to use a
(C++14) lambda-capture expression in which a local modifiable copy is deduced:

\begin{emcppslisting}[emcppsbatch=e4]
void performAlgorithmComparison2()
{
    // ...

    const Puzzle puzzle = generateRandomSudokuPuzzle();

    auto task0 = getThreadPool().enqueueTask([p = puzzle]() mutable
    {
        solve<NaiveAlgorithm>(p);  // OK, (ù{\codeincomments{p}}ù) is now modifiable.
        return p;
    });

    // ...
}
\end{emcppslisting}
    
\noindent Note that use of \lstinline!p!~\lstinline!=!~\lstinline!puzzle! (above) is
roughly equivalent to the creation of a new variable using
\lstinline!auto! (i.e.,
\lstinline!auto!~\lstinline!p!~\lstinline!=!~\lstinline!puzzle;!), which guarantees
that the type of \lstinline!p! will be deduced as a non-\lstinline!const!
\lstinline!Puzzle!. Capturing an existing \lstinline!const! variable as a mutable copy is
possible, but doing the opposite is not easy; see \intraref{annoyances-lambdacapture}{there’s-no-easy-way-to-synthesize-a-const-data-member}.

\subsection[Potential Pitfalls]{Potential Pitfalls}\label{potential-pitfalls-lambdacapture}

\subsubsection[Forwarding an existing variable into a closure always results in an object (never a reference)]{Forwarding an existing variable into a closure always results in\\ an object (never a reference)}\label{forwarding-an-existing-variable-into-a-closure-always-results-in-an-object-(never-a-reference)}

Lambda-capture expressions allow existing variables to be
\emcppsgloss{perfectly forwarded} (see \featureref{\locationc}{forwardingref})
into a closure:

\begin{emcppslisting}[emcppsstandards={c++14}]
#include <utility>  // (ù{\codeincomments{std::forward}}ù)

template <typename T>
void f(T&& x)  // (ù{\codeincomments{x}}ù) is of type forwarding reference to (ù{\codeincomments{T}}ù).
{
    auto lambda = [y = std::forward<T>(x)]  
        // Perfectly forward (ù{\codeincomments{x}}ù) into the closure.
    {
        // ... (use (ù{\codeincomments{y}}ù) directly in this lambda body)
    };
}
\end{emcppslisting}
    
\noindent Because \lstinline!std::forward<T>! can evaluate to a reference (depending
on the nature of \lstinline!T!), programmers might incorrectly assume that
a capture such as \lstinline!y!~\lstinline!=!~\lstinline!std::forward<T>(x)!
(above) is somehow either a capture \emph{by value} or a capture
\emph{by reference}, depending on the original \emcppsgloss{value category}
of \lstinline!x!.

Remembering that lambda-capture expressions work similarly to
\lstinline!auto! type deduction for variables, however,
reveals that such captures will \emph{always} result in an object, \emph{never} a reference:

\begin{emcppslisting}[emcppsignore={pseudocode}]
// pseudocode ((ù{\codeincomments{auto}}ù) is not allowed in a lambda introducer.)
auto lambda = [auto y = std::forward<T>(x)] { };
    // The capture expression above is semantically similar to an (ù{\codeincomments{auto}}ù)
    // (deduced-type) variable.
\end{emcppslisting}
    
\noindent If \lstinline!x! was originally an \romeovalue{lvalue}, then \lstinline!y! will be
equivalent to a \emph{by-copy} capture of \lstinline!x!. Otherwise,
\lstinline!y! will be equivalent to a \emph{by-move} capture of
\lstinline!x!.{\cprotect\footnote{Note that both \emph{by-copy} and
\emph{by-move} capture communicate \emcppsgloss{value} for
  \emcppsgloss[value semantic type (VST)]{value-semantic types}.}}

If the desired semantics are to capture \lstinline!x! \emph{by move} if it
originated from \emcppsgloss{rvalue} and \emph{by reference} otherwise, then
the use of an extra layer of indirection (using, e.g.,
\lstinline!std::tuple!) is required:

\begin{emcppslisting}[emcppsstandards={c++14}]
#include <tuple>  // (ù{\codeincomments{std::tuple}}ù)

template <typename T>
void f(T&& x)
{
    auto lambda = [y = std::tuple<T>(std::forward<T>(x))]
    {
        // ... (Use (ù{\codeincomments{std::get<0>(y)}}ù) instead of (ù{\codeincomments{y}}ù) in this lambda body.)
    };
}
\end{emcppslisting}
    
\noindent In the revised code example above, \lstinline!T! will be an \emcppsgloss[lvalue reference]{\romeovalue{lvalue} reference} if \lstinline!x! was originally an \romeovalue{lvalue}, resulting in
the \emcppsgloss{synthetization} of a \lstinline!std::tuple! containing an
\emcppsgloss[lvalue reference]{\romeovalue{lvalue} reference}, which --- in turn --- has semantics
equivalent to \lstinline!x!'s being captured \emph{by reference}.
Otherwise, \lstinline!T! will not be a reference type, and \lstinline!x! will
be \emph{moved} into the closure.

\subsection[Annoyances]{Annoyances}\label{annoyances-lambdacapture}

\subsubsection[There’s no easy way to synthesize a \lstinline!const! data member]{There’s no easy way to synthesize a {\SubsubsecCode const} data member}\label{there’s-no-easy-way-to-synthesize-a-const-data-member}

Consider the (hypothetical) case where the programmer desires to capture
a copy of a non-\lstinline!const! integer \lstinline!k! as a \lstinline!const!
closure data member:

\begin{emcppslisting}[emcppsstandards={c++14}]
void test1()
{
    int k;
    [k = static_cast<const int>(k)]() mutable  // (ù{\codeincomments{const}}ù) is ignored
    {
        ++k;  // "OK" -- i.e., compiles anyway even though we don't want it to
    };
}
\end{emcppslisting}

\begin{emcppslisting}[emcppsstandards={c++14}]
void test2()
{
    int k;
    [const k = k]() mutable  // Error, invalid syntax
    {
        ++k;  // no easy way to force this variable to be (ù{\codeincomments{const}}ù)
    };
}
\end{emcppslisting}
    
\noindent The language simply does not provide a convenient mechanism for
synthesizing, from a modifiable variable, a \lstinline!const! data member.
If such a \lstinline!const! data member somehow proves to be necessary, we
can either create a \lstinline!ConstWrapper! \lstinline!struct! (that adds
\lstinline!const! to the captured object) or write a full-fledged
\emcppsgloss{function object} in lieu of the leaner \emcppsgloss[lambda expressions]{lambda
expression}. Alternatively, a \lstinline!const! copy of the object can be
captured with traditional (C++11) lambda-capture expressions:

\begin{emcppslisting}
int test3()
{
    int k;
    const int kc = k;

    auto l = [kc]() mutable
    {
        ++kc;  // Error, increment of read-only variable (ù{\codeincomments{kc}}ù)
    };
}
\end{emcppslisting}
    

\subsubsection[\lstinline!std::function! supports only copyable callable objects]{{\SubsubsecCode std::function} supports only copyable callable objects}\label{std::function-supports-only-copyable-callable-objects}

Any lambda expression capturing a move-only object produces a closure
type that is itself movable but \emph{not} copyable:

\begin{emcppshiddenlisting}[emcppsbatch=e4,emcppsstandards={c++14}]
#include <memory>   // (ù{\codeincomments{std::unique\_ptr}}ù)
#include <utility>  // (ù{\codeincomments{std::move}}ù)
\end{emcppshiddenlisting}
\begin{emcppslisting}[emcppsbatch=e4]
void f()
{
    std::unique_ptr<int> moo(new int);    // some move-only object
    auto la = [moo = std::move(moo)]{ };  // lambda that does move capture

    static_assert(false == std::is_copy_constructible<decltype(la)>::value, "");
    static_assert( true == std::is_move_constructible<decltype(la)>::value, "");
}
\end{emcppslisting}
    
\noindent Lambdas are sometimes used to initialize instances of
\lstinline!std::function!, which requires the stored \emcppsgloss{callable
object} to be copyable:

\begin{emcppslisting}
std::function<void()> f = la;  // Error, (ù{\codeincomments{la}}ù) must be copyable.
\end{emcppslisting}
    
\noindent Such a limitation --- which is more likely to be encountered when using
lambda-capture expressions --- can make \lstinline!std::function!
unsuitable for use cases where move-only closures might conceivably be
reasonable. Possible workarounds include (1) using a different
type-erased, \emcppsgloss{callable object} wrapper type that supports
move-only callable objects,{\cprotect\footnote{The
\lstinline!any_invocable! library type, proposed for C++23, is an
example of a type-erased wrapper for move-only callable objects; see
  \cite{calabrese20}.}} (2) taking a performance hit by wrapping the
desired \emcppsgloss{callable object} into a copyable wrapper (such as
\lstinline!std::shared_ptr!), or (3) designing software such that
noncopyable objects, once constructed, never need to
move.\footnote{For an in-depth discussion of how large systems can
benefit from a design that embraces local arena memory allocators and,
thus, minimizes the use of moves across natural memory boundaries
identified throughout the system, see \cite{lakos22}.}

\subsection[See Also]{See Also}\label{see-also}

\begin{itemize}

\item{\seealsoref{auto-feature}{\seealsolocationc}offers a model with the same type deduction rules.}
\item{\seealsoref{bracedinit}{\seealsolocationc}illustrates one possible way of initializing the captures.}
\item{\seealsoref{forwardingref}{\seealsolocationc}describes a feature that contributes to a source of misunderstanding of this feature.}
\item{\seealsoref{lambda}{\seealsolocationc}provides the needed background for understanding the feature in general.}
\item{\seealsoref{Rvalue-References}{\seealsolocationc}gives a full description of an important feature used in conjunction with movable types.}
\end{itemize}

\subsection[Further Reading]{Further Reading}\label{further-reading}

None so far


