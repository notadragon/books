% 13 Jan, copyedits in and proofed
% 27 Jan, code compiled by Josh
% 27 Jan, new section added 
% 27 Jan, new section copyedited and copyedits entered

\emcppsFeature{
    toclong={Using \lstinline!=!\,\lstinline!delete! for Arbitrary Functions},
    short={Deleted Functions},
    long={Using {\SecCode =}\,{\SecCode delete} for Arbitrary Functions},
}{deleted-functions}
%\section[Deleted Functions]{Using {\SecCode =delete} for Arbitrary Functions}\label{deleted-functions}
\setcounter{table}{0}
\setcounter{footnote}{0}
\setcounter{lstlisting}{0}




Using \lstinline!=!\,\lstinline!delete! in a function's first declaration
forces a compilation error upon any attempt to use or access it.

\subsection[Description]{Description}\label{description}

Declaring a particular function or function overload to result in a
fatal diagnostic upon invocation can be useful --- e.g., to suppress the
generation of a \romeogloss{special member function} or to limit the types of
arguments a particular overload set is able to accept. In such cases,
\lstinline!=!\,\lstinline!delete! followed by a semicolon (\lstinline!;!) can be used in place of the body of any
function on first declaration only to force a compile-time error if
any attempt is made to invoke it or take its address.

\begin{emcppslisting}
void g(double) { }
void g(int) = delete;

void f()
{
    g(3.14);  // OK, (ù{\codeincomments{f(double)}}ù) is invoked.
    g(0);     // Error, (ù{\codeincomments{f(int)}}ù) is deleted.
}
\end{emcppslisting}

\noindent Notice that deleted functions participate in \romeogloss{overload resolution}
and produce a compile-time error when selected as the best candidate.

\subsection[Use Cases]{Use Cases}\label{use-cases}

\subsubsection[Suppressing special member function generation]{Suppressing special member function generation}\label{suppressing-special-member-function-generation}

When instantiating an object of user-defined type, \romeogloss{special
member functions} that have not been declared explicitly are
often generated automatically by the compiler.
The generation of individual special member
functions can be affected by the existence of other user-defined special
member functions or by limitations imposed by the specific types of
any data members or base types; see 
%Section~\ref{Defaulted-Special-Member-Functions}, ``\titleref{Defaulted-Special-Member-Functions} on page~\pageref{Defaulted-Special-Member-Functions}."
\featureref{\locationa}{Defaulted-Special-Member-Functions}. 
%For certain kinds of types, the notion of \romeogloss{copy semantics}
%including \romeogloss{move semantics} is not meaningful, and
%hence permitting the generation of copy operations is contraindicated. The two special member functions controlling \romeogloss{\emph{move} operations}, introduced in C++11, are sometimes implemented as effective optimizations of \romeogloss{\emph{copy} operations} and much less frequently with \romeogloss{\emph{copy} operations} explicitly deleted; see \featureref{\locationc}{Rvalue-References}.
For certain kinds of types, the notion of \emph{copying} is not meaningful, and hence permitting the compiler to generate \emph{copy} operations would be inappropriate. The two special member functions controlling \romeogloss{move operations}, introduced in C++11, are typically implemented as effective optimizations of \romeogloss{\emph{copy} operations} and thus would be similarly contraindicated. Much less frequently, a useful notion of moving exists where copying does not, and so we may choose to have move operations generated, while \romeogloss{\emph{copy} operations} are explicitly deleted; see \featureref{\locationc}{Rvalue-References}.

Consider a class, \lstinline!FileHandle!, that uses the \romeogloss{RAII} idiom
to safely acquire and release an I/O stream. As \romeogloss{copy semantics}
are typically not meaningful for such resources, we will want to
suppress generation of both the \romeogloss{copy constructor} and \romeogloss{copy
assignment operator}. Prior to C++11, there was no direct way to express
suppression of \romeogloss{special functions} in C++. The commonly
recommended workaround was to declare the two methods \lstinline!private!
and leave them unimplemented, typically resulting in a compile-time or
link-time error when accessed:

\begin{emcppslisting}
#include <cstdio>  // (ù{\codeincomments{FILE}}ù)
class FileHandle
{
private:
    // ...

    FileHandle(const FileHandle&);             // not implemented
    FileHandle& operator=(const FileHandle&);  // not implemented

public:
    explicit FileHandle(FILE* filePtr);
    ~FileHandle();

    // ...
};
\end{emcppslisting}

\noindent Not implementing a special member function that is declared to be private ensures that there will be at least a link-time error in case that function is inadvertently accessed from within the implementation of the class itself. With the \lstinline!=!\,\lstinline!delete! syntax, we are able to (1)
explicitly express our intention to make these special member
functions unavailable, (2) do so directly in the \lstinline!public! region
of the class, and (3) enable clearer compiler diagnostics:

%begin{emcppslisting}
%class FileHandle
%{
%private:
%    // ...
%
%public:
%    explicit FileHandle(FILE* filePtr);
%    ~FileHandle();
%
%    FileHandle(const FileHandle&) = delete;             // make unavailable
%    FileHandle& operator=(const FileHandle&) = delete;  // make unavailable
%
%    // ...
%};
%\end{emcppslisting}
\begin{emcppshiddenlisting}[emcppsbatch=e1]
#include <cstdio>  // (ù{\codeincomments{FILE}}ù)
\end{emcppshiddenlisting}
\begin{emcppslisting}[emcppsbatch=e1]
class FileHandle
{
private:
    // ...
    // Declarations of copy constructor and copy assignment are now public.

public:
    explicit FileHandle(FILE* filePtr);
    ~FileHandle();

    FileHandle(const FileHandle&) = delete;             // make unavailable
    FileHandle& operator=(const FileHandle&) = delete;  // make unavailable

    // ...
};
\end{emcppslisting}

\noindent Using the \lstinline!=!\,\lstinline!delete! syntax on declarations that are private results in error messages concerning privacy, not the use of deleted functions. Care must be exercised to make \emph{both} changes when converting code from the old style to the new syntax.

\subsubsection[Preventing a particular implicit conversion]{Preventing a particular implicit conversion}\label{preventing-a-particular-implicit-conversion}

Certain functions --- especially those that take a \lstinline!char! as an
argument --- are prone to inadvertent misuse. As a truly classic
example, consider the C library function \lstinline!memset!, which may be used
to write the character \lstinline!*! five times in a row, starting at a
specified memory address, \lstinline!buf!:

%begin{emcppslisting}
%#include <cstring>
%#include <cstdio>
%
%void f()
%{
%    char buf[] = "Hello World!";
%    memset(buf, 5, '*');  // undefined behavior
%    puts(buf);            // expected output: "***** World!"
%}
%\end{emcppslisting}
\begin{emcppslisting}
#include <cstdio>  // (ù{\codeincomments{puts}}ù)
#include <cstring> // (ù{\codeincomments{memset}}ù)

void f()
{
    char buf[] = "Hello World!";
    memset(buf, 5, '*');  // undefined behavior: buffer overflow
    puts(buf);            // expected output: "***** World!"
}
\end{emcppslisting}


\noindent Sadly, inadvertently reversing the order of the last two arguments is a commonly
recurring error, and the C language provides no help. As shown above, \lstinline!memset! writes the nonprinting character \lstinline!5! (e.g., the integer value of ASCII \lstinline!'*'!) 42 times --- way past the end of \lstinline!buf!. In C++, we
can target such observed misuse using an extra deleted overload:

%begin{emcppslisting}
%#include <cstring>  // memset()
%void* memset(void* str, int ch, size_t n);      // standard library function
%void* memset(void* str, int n, char) = delete;  // defensive against misuse
%\end{emcppslisting}
\begin{emcppslisting}
#include <cstring>  // (ù{\codeincomments{memset}}ù)
void* memset(void* str, int ch, size_t n);      // standard library function
void* memset(void* str, int n, char) = delete;  // defense against misuse
\end{emcppslisting}

\noindent Pernicious user errors can now be reported during compilation:

\begin{emcppslisting}
// ...
memset(buf, 5, '*');  // Error, (ù{\codeincomments{memset(void\*, int, char)}}ù) is deleted.
// ...
\end{emcppslisting}


\subsubsection[Preventing all implicit conversions]{Preventing all implicit conversions}\label{preventing-all-implicit-conversions}

The \lstinline!ByteStream::send! member function below is designed to work
with 8-bit unsigned integers only. Providing a deleted overload
accepting an \lstinline!int! forces a caller to ensure that the argument is
always of the appropriate type:

\begin{emcppslisting}
class ByteStream
{
public:
    void send(unsigned char byte) { /* ... */ }
    void send(int) = delete;

    // ...
};

void f()
{
    ByteStream stream;
    stream.send(0);   // Error, (ù{\codeincomments{send(int)}}ù) is deleted.     (1)
    stream.send('a'); // Error, (ù{\codeincomments{send(int)}}ù) is deleted.     (2)
    stream.send(0L);  // Error, ambiguous                 (3)
    stream.send(0U);  // Error, ambiguous                 (4)
    stream.send(0.0); // Error, ambiguous                 (5)
    stream.send(
        static_cast<unsigned char>(100));  // OK          (6)
}
\end{emcppslisting}

\noindent Invoking \lstinline!send! with an \lstinline!int! (noted with (1) in the code above) or any integral type
(other than \lstinline!unsigned!~\lstinline!char!) that promotes to \lstinline!int! (2)
will map exclusively to the deleted \lstinline!send(int)! overload; all
other integral (3 and 4) and floating-point types (5) are convertible to
both via a \romeogloss{standard conversion} and hence will be ambiguous. Note that
implicitly converting from \lstinline!unsigned!~\lstinline!char! to either a
\lstinline!long! or \lstinline!unsigned! integer involves a \romeogloss{standard
conversion} (not just an \romeogloss{integral promotion}), the same as
  converting to a \lstinline!double!.
An explicit cast to \lstinline!unsigned!~\lstinline!char! (6) can always be
pressed into service if needed.

\subsubsection[Hiding a structural (nonpolymorphic) base class's member function]{Hiding a structural (nonpolymorphic) base class’s member function}\label{hiding-a-structural-(nonpolymorphic)-base-class's-member-function}

It is commonly advised to avoid deriving publicly from concrete classes because by doing so, we do not hide the underlying capabilities, which can easily be accessed (potentially breaking any invariants the derived class may want to keep) via assignment to a pointer or reference to a base class, with no casting required.  Worse, inadvertently passing such a class to a function taking the base class by value will result in slicing, which can be especially problematic when the derived class holds data. A more robust approach would be to use layering or at least private inheritance.\footnote{For more on improving compositional designs at scale, see
  \cite{lakos20}, sections 3.5.10.5 and 3.7.3, pp.~687--703 and
  726--727, respectively.} Best practices notwithstanding,\footnote{See ``Inheritance and Object-Oriented Programming," Item 38, \cite{meyers92}, pp. 132--135.} it can be cost-effective in the short
term to provide an elided ``view'' on a concrete class for trusted
clients. Imagine a class \lstinline!AudioStream! designed to play sounds
and music that --- in addition to providing basic ``play'' and
``rewind'' operations --- sports a large, robust interface:

\begin{emcppshiddenlisting}[emcppsbatch=e2]
struct ForwardAudioStream;
struct FMRadio {
    static ForwardAudioStream getStream();
};
\end{emcppshiddenlisting}
\begin{emcppslisting}[emcppsbatch=e2]
struct AudioStream
{
    void play();
    void rewind();
    // ...
    // ... (large, robust interface)
    // ...
};
\end{emcppslisting}


Now suppose that, on short notice, we need to whip up a similar
class,\linebreak[4] \lstinline!ForwardAudioStream!, to use with audio samples that
cannot be rewound (e.g., coming directly from a live feed). Realizing
that we can readily reuse most of \lstinline!AudioStream!'s interface, we
pragmatically decide to prototype the new class simply by exploiting
public \romeogloss{structural inheritance} and then deleting just the lone
unwanted \lstinline!rewind! member function:

\begin{emcppslisting}[emcppsbatch=e2]
struct ForwardAudioStream : AudioStream
{
    void rewind() = delete; // Make just this one function unavailable.
};

void f()
{
    ForwardAudioStream stream = FMRadio::getStream();
    stream.play();   // fine
    stream.rewind(); // Error, (ù{\codeincomments{rewind()}}ù) is deleted.
}
\end{emcppslisting}

\noindent If the need for a \lstinline!ForwardAudioStream! type persists, we can always
consider reimplementing it more carefully later.{\cprotect\footnote{\cite[sections 3.5.10.5 and 3.7.3, pp.~687--703 and 726--727]{lakos20}}} As discussed at the beginning of this section, the protection provided by this example is easily circumvented:
\begin{emcppslisting}[emcppsbatch=e2]
void g(const ForwardAudioStream &stream)
{
    AudioStream fullStream = stream;
    fullStream.play();   // OK
    fullStream.rewind(); // compiles OK, but what happens at run time?
}
\end{emcppslisting}

\noindent Hiding non\lstinline!virtual! functions is something one undertakes only after attaining a complete understanding of what makes such an unorthodox endeavor \emph{safe}; see, in particular, the appendix of \featureref{\locatione}{final}.

\subsection[Potential Pitfalls]{Potential Pitfalls}\label{potential-pitfalls}

\hspace*{\fill}

\subsection[Annoyances]{Annoyances}\label{annoyances}

%%%%%%%%%%%%%%% new material added 27 Jan 2021
\subsubsection[Deleting a function declares it]{Deleting a function declares it}

It should come as no surprise that when we ``declare" a \romeogloss{free function}
followed by \lstinline!=delete!, we \emph{are} in fact \emph{declaring} it.  For example, consider
the pair of overloads of functions \lstinline!f! declared taking a \lstinline!char! and \lstinline!int!,
respectively:

\begin{emcppslisting}
int f(char);          // (1) accessible declaration of (ù{\codeincomments{f}ù) taking a (ù{\codeincomments{char}ù)
int f(int) = delete;  // (2) inaccessible declaration of (ù{\codeincomments{f}ù) taking an (ù{\codeincomments{int}ù)

int x = f('a');       // OK, exact match for (1) (ù{\codeincomments{f(char)}ù), which is accessible
int y = f(123);       // Error, exact match for (2) (ù{\codeincomments{f(int)}ù), which is deleted
\end{emcppslisting}

\noindent It is necessary that both functions above are \emph{declared} so that both of them
can participate in overload resolution; it is only after the inaccessible 
overload is selected tat it will be reported as a compile-time error.

When it comes to deleting certain \romeogloss{special member functions} of a class (or
class  template), however, what might seem like a tiny bit of extra,
self-documenting code can have subtle, unintended consequences as evidenced 
below.

Let's begin by considering an empty \lstinline!struct!, \lstinline!S0!:

\begin{emcppslisting}[emcppsbatch=e3]
struct S0 { };  // The default constructor is declared implicitly.

S0 x0;  // OK, invokes the implicitly generated default constructor
\end{emcppslisting}

\noindent As \lstinline!S0! defines not constructors, destructors, or assignment operators, 
the compiler will generate (\romeogloss{declare} and \romeogloss{define}), for \lstinline!S0!, all 
\emph{six} of  the \romeogloss{special member functions} available as of C++11; see 
\featureref{\locationa}{defaulted-special-member-functions}.

Next, suppose we create a second \lstinline!struct!, \lstinline!S1!, that differs from \lstinline!S0!
only in that \lstinline!S1! declares a \emph{value} constructor taking an \lstinline!int!:

\begin{emcppslisting}[emcppsbatch=e3]
struct S1  // Implicit declaration of the default constructor is suppressed.
{
    S1(int);  // explicit declaration of (ù{\emphincomments{value}ù) constructor
};

S1 y1(5);  // OK, invokes the explicitly declared (ù{\emphincomments{value}ù) constructor
S1 x1;     // Error, no declaration for default constructor (ù{\codeincomments{S1::S1()}ù)
\end{emcppslisting}

\noindent By explicitly declaring a \emph{value} constructor (or any other constructor
for that matter), we automatically suppress the implicit declaration of
the default constructor for \lstinline!S1!. If suppressing the default destructor is
\emph{not} our intention, we can always reinstate it via an explicit declaration
followed by \lstinline!=default;! --- see \featureref{\locationa}{defaulted-special-member-functions}.

Let's now suppose it \emph{is} our intention to suppress generation of the default
constructor and, to make our intention clear, we elect to explicitly \romeogloss{declare} 
and \romeogloss{delete} it:

\begin{emcppslisting}[emcppsbatch=e3]
struct S2  // Implicit declaration of the default constructor is suppressed.
{
    S2() = delete;  // explicit declaration of inaccessible default constructor
    S2(int);        // explicit declaration of (ù{\emphincomments{value}ù) constructor
};

S2 y2(5);  // OK, invokes the explicitly declared (ù{\emphincomments{value}ù) constructor
S2 x2;     // Error, use of deleted function, (ù{\codeincomments{S2::S2()}ù)
\end{emcppslisting}

\noindent By declaring and then deleting the default constructor we have, it would
appear that we (1) made our intentions clear and (2) improved diagnostics for our
clients at the cost of a single extra line of self-documenting code.  Ah, if
C++ were only that straightforward.

Deleting certain \romeogloss{special member functions} --- i.e., \emph{default} constructor,
\emph{move} constructor, or \emph{move}-assignment operator --- that are not necessarily
implicitly declared can have nonobvious consequence that adversely affect
subtle compile-time properties of a class. One such subtle property is whether 
the compiler considers it to be a \romeogloss{literal type} --- i.e., a type whose 
\emph{value} is eligible for use as part of a \romeogloss{constant expression}.  This same 
property of being a \romeogloss{literal type} is what determines whether an arbitrary 
type may be passed by value in the interface of a \lstinline!constexpr! function; see
\featureref{\locationc}{constexprfunc}.

As a simple illustration of a subtle compile-time difference between \lstinline!S1! and
\lstinline!S2!, consider this practically useful \emph{pattern} for a developer's ``test"
function that will compile if and only if its by-value parameter, \lstinline!x!, is of a
literal type:

\begin{emcppslisting}[emcppsbatch=e3]
constexpr int test(S0 x) { return 0; }  // OK,    (ù{\codeincomments{S0}ù) (ù{\emphincomments{is}ù)     a literal type.
constexpr int test(S1 x) { return 0; }  // Error, (ù{\codeincomments{S1}ù) is (ù{\emphincomments{not}ù) a literal type.
constexpr int test(S2 x) { return 0; }  // OK,    (ù{\codeincomments{S2}ù) (ù{\emphincomments{is}ù)     a literal type.
\end{emcppslisting}

\noindent For the compiler to treat a given class type as a \romeogloss{literal type}, it must,
among other things, have at least one constructor (other than the \emph{copy} or 
\emph{move} constructor) declared as \lstinline!constexpr!.

In the case of the empty \lstinline!S0! class, the implicitly generated default constructor 
is  \romeogloss{trivial} and so it is implicitly \emph{declared} \lstinline!constexpr! too.  Class \lstinline!S1!'s 
explicitly declared \emph{non}-\lstinline!constexpr! value constructor suppresses the declaration 
of its only \lstinline!constexpr! constructor, the  default constructor; hence, \lstinline!S1! does not 
qualify as a \emph{literal type}.

Finally, by conspicuously declaring and deleting \lstinline!S2!'s default constructor, we
\emph{declare} it nonetheless.  What's more, the declaration brought about by deleting it 
is the same as if it had been generated implicitly (or declared explicitly and then 
defaulted); hence, \lstinline!S2!, unlike \lstinline!S1!, \emph{is} a \romeogloss{literal type}.  Go figure!
%%%%%%%%%%%%%%%%%%%%%% new material ends

\subsection[See Also]{See Also}\label{see-also}

\begin{itemize}
\item{%Section~\ref{Defaulted-Special-Member-Functions}, ``\titleref{Defaulted-Special-Member-Functions}" on page~\pageref{Defaulted-Special-Member-Functions} —
\seealsoref{Defaulted-Special-Member-Functions}{\locationa}Companion feature that enables defaulting, as opposed to deleting, special member functions.}
\item{%Section~\ref{Rvalue-References}, ``\titleref{Rvalue-References}" on page~\pageref{Rvalue-Referencess} —
\seealsoref{Rvalue-References}{\locationc}The two \emph{move} variants of special member functions, which use \romeovalue{rvalue} references in their signatures, may also be subject to deletion.}
\end{itemize}

\subsection[Further Reading]{Further Reading}\label{further-reading}

\begin{itemize}
\item{``Item 27" of \cite{meyers15b}}
\end{itemize}

