
\addtocontents{toc}{\protect\newpage}
\chapter[Unsafe Features]{Unsafe Features}\label{ch-unsafe}
\emcppschapterstart{}

PRODUCTION: Material copied from Ch 0; AUs will edit/replace on FPPs. 

AUTHORS: Remember to add material about whether these features are easy to teach. 

When an expert programmer uses any C++ feature appropriately, the feature typically does no direct harm. Yet other developers --- seeing the feature’s use in the code base but failing to appreciate the highly specialized or nuanced reasoning justifying it --- might attempt to use it in what they perceive to be a similar way, yet with profoundly less desirable results. Similarly, maintainers may change the use of a fragile feature altering its semantics in subtle but damaging ways.

Features that are classified as unsafe are those that might have valid, and even very important, use cases, yet our experience indicates that routine or widespread use thereof would be counterproductive in a typical large-scale software-development enterprise.

For example, we deem the final contextual keyword an unsafe feature because the situations in which it would be misused overwhelmingly outnumber those vanishingly few isolated cases in which it is appropriate, let alone valuable. Furthermore, its widespread use would inhibit fine-grained (e.g., hierarchical) reuse, which is critically important to the success of a large organization.

%%%%%%%%%%%%%%% C++11
\stepcounter{cppxx}
\addtocontents{toc}{\protect\renewcommand*\protect\cftsecindent{1.25em}}
\cftaddnumtitleline{toc}{section}{\thechapter.\thecppxx}{C++11}{\thepage}
% to add to the TOC at the section level
\renewcommand{\cppxx}{C++11}

\newpage
\addtocontents{toc}{\protect\renewcommand*\protect\cftsecindent{4.5em}}
% \section[{\tt carries\_dependency}]{The {\SecCode [[carries\_dependency]]} Attribute}\label{carriesdependency}
% 19 Feb 2021, ready for Josh's code fixes
% 20 Feb 2021 JMB - code cleaned up, compiles.
% 2 March 2021, copyedits entered and proofed. 
% 18 March 2021, author revisions made after copyediting; entered and proofed.
% 26 March 2021, in FPPs 


\emcppsFeature{
    short={\lstinline!carries_dependency!},
    tocshort={\TOCCode carries\_dependency},
    long={The {\SecCode [[carries\_dependency]]} Attribute},
    toclong={The \lstinline![[carries\_dependency]]! Attribute},
    rhshort={\RHCode carries\_dependency},
}{carriesdependency}
\label{the-carries_dependency-attribute}
\setcounter{table}{0}
\setcounter{footnote}{0}
\setcounter{lstlisting}{0}
 %\section[{\tt carries\_dependency}]{The {\SecCode [[carries\_dependency]]} Attribute}\label{carriesdependency}
%\subsection[The \lstinline![[carries_dependency]]! Attribute]{The {\SecCode [[carries\_dependency]]} Attribute}\label{the-[[carries_dependency]]-attribute}

The\lstinline![[carries_dependency]]! attribute provides a means to
manually identify function parameters and \lstinline!return! values as
components of \emcppsgloss[data dependency chain]{data dependency chains} to enable (primarily 
across translation units) use of the lighter-weight
\emcppsgloss[release consume]{release-consume} \emcppsgloss{synchronization paradigm} as an
optimization over the more conservative \emcppsgloss[release acquire]{release-acquire}
paradigm.{\cprotect\footnote{The authors would like to thank Michael
Wong, Paul McKenney, and Maged Michael for reviewing and contributing to this feature section.}}

\subsection[Description]{Description}\label{description}

C++11 ushered in support for multithreading by introducing a rigorously
specified memory model. The Standard Library provides support for
managing threads, including their execution, synchronization, and
intercommunication. As a part of the new memory model, the Standard
defines various \emcppsgloss[synchronization operation]{synchronization operations} that are classified as either \emph{sequentially
consistent}, \emph{release}, \emph{acquire}, 
\emph{release-and-acquire}, or \emph{consume} operations. These
operations play a key role in making changes in data in one thread
visible in another.

The modern C++ memory model describes two \emcppsgloss[synchronization
paradigm]{synchronization paradigms} that are used
to coordinate data flow among concurrent threads of execution. The current suite of supported
\emcppsgloss[synchronization paradigm]{synchronization paradigms} comprises \emcppsgloss[release acquire]{release-acquire} and \emcppsgloss[release consume]{release-consume}, although in practice
\emcppsgloss[release consume]{release-consume} is implemented in terms of \emcppsgloss[release acquire]{release-acquire} in all known implementations. In
particular, the \emcppsgloss[release consume]{release-consume} paradigm requires that the
compiler be given fine-grained understanding of the \emcppsgloss[intra thread dependency]{intra-thread dependencies} among the reads and writes within a program
and relates those to atomic \emph{release stores} and \emph{consume
loads} that happen concurrently across multiple threads of execution.
Dependency chains in the \emcppsgloss[release consume]{release-consume} synchronization
paradigm specify which evaluations following the \emph{consume load} are
\emcppsgloss{ordered after} a corresponding \emph{release store}.

\subsubsection[The release-acquire paradigm]{The release-acquire paradigm}\label{the-release-acquire-paradigm}

A \emph{release} operation writes a value to a memory location, and an
\emph{acquire} operation reads a value from a memory location. Although many have
referred to the release-acquire paradigm as \emph{acquire-release}, the proper, standard,
  time-ordered nomenclature is \emph{release-acquire}. In a \emcppsgloss[release acquire]{release-acquire} pair, the acquire operation reads the value written by
the release operation, which means that all of the reads and writes to
\emph{any} memory location \emph{before the release operation} happen
before \emph{all} of the reads and writes \emph{after the acquire operation}. Note that this paradigm does \emph{not} use 
dependency chains or the \lstinline![[carries_dependency]]! attribute. See \intraref{use-cases-carriesdependency}{producer-consumer-programming-pattern} 
%{\intraref{{Use Cases: A producer-consumer implementation}}} below 
for a
complete example that implements this paradigm.

\subsubsection[Data dependency]{Data dependency}\label{data-dependency}

In the current revisions of C++, \emcppsgloss{data dependency} is defined as
existing whenever the output of one evaluation is used as the input of
another. 
 When one evaluation has a data dependency on another
evaluation, the second evaluation is said to \emcppsgloss{carry dependency}
to the other. The Standard Library function
\lstinline!std::kill_dependency! is also related and can be used to
  \emph{break} a data dependency chain. Naturally the compiler must
ensure that any evaluation that depends on another must not be started
until the first evaluation is complete. A \emcppsgloss{data dependency chain}
is formed when multiple evaluations carry dependency transitively; the
output of one evaluation is used as the input of the next evaluation in
the chain.

\subsubsection[The release-consume paradigm]{The release-consume paradigm}\label{the-release-consume-paradigm}

Some systems use the read-copy-update (RCU) synchronization mechanism.
This approach preserves the order of \emph{loads} and \emph{stores} that
form in a \emcppsgloss{data dependency chain}, which is a sequence of
\emph{loads} and \emph{stores} in which the input to one operation is an
output of another. A compiler can use guaranteed order of loads
and stores provided by the RCU synchronization mechanism for performance purposes by omitting certain
\emcppsgloss[memory fence instruction]{memory-fence instructions} that would otherwise be required to
enforce the correct ordering. In such cases, however, ordering is
guaranteed only between those operations making up the relevant
\emcppsgloss{data dependency chain}. The C++ definition
of data dependency is intended to mimic the data dependency on RCU
systems. Note, however, that C++ currently defines data dependency in
terms of evaluations, while RCU data dependency is defined in terms of
  loads and stores.

This optimization was intended to be available in C++ through use of a
\emph{release-consume} pair, which, as its name
suggests, consists of a \emph{release-store} operation and a
\emph{consume-load} operation. A \emph{consume} operation is much like
an \emph{acquire} operation, except that it guarantees only the ordering
of those evaluations in a \emcppsgloss{data dependency chain}, starting with
the consume-load operation.

Note, however, that currently no known implementation is able to take
advantage of the current C++ \emph{consume} semantics; hence, all
current compilers promote \emph{consume} loads to \emph{acquire} loads,
effectively making the \lstinline![[carries_dependency]]! attribute
redundant. Revisions to render this feature implementable and therefore
usable are currently under consideration by the C++ Standards Committee.
Prototypes for various approaches have been produced. When a usable
feature with real implementations is delivered, it quite possibly will
not work exactly as described in the examples here; see \intrarefsimple{use-cases-carriesdependency}. 
% {\intraref{{Use Cases}}}, below.

\subsubsection[Using the {\protect\lstinline![[carries_dependency]]!} attribute]{Using the {\SubsubsecCode [[carries\_dependency]]} attribute}\label{using-the-[[carries_dependency]]-attribute}

\emcppsgloss[data dependency chain]{Data dependency chains} can and do propagate into and out of
called functions. If one of these interoperating functions is in a
separate translation unit, the compiler will have no way of seeing the
dependency chain. In such cases, the user can apply the
\lstinline![[carries_dependency]]! attribute to imbue the necessary
information for the compiler to track the propagation of dependency
chains into and out of functions across translation units, thus possibly
avoiding unnecessary memory-fence instructions; see \intrarefsimple{use-cases-carriesdependency}. Note that the Standard Library function
\lstinline!std::kill_dependency! is also related and can be used to
  \emph{break} a data dependency chain.
% {\intrarefsub{{Using the \lstinline![[carries_dependency]]! attribute}}}, below.

The \lstinline![[carries_dependency]]! attribute can be applied to a
function declaration as a whole by placing it in front of the function declaration,
in which case the attribute applies to the \lstinline!return! value:

\begin{emcppslisting}
[[carries_dependency]] int* f();  // attribute applied to entire function (ù{\codeincomments{f}}ù)
\end{emcppslisting}
    
\noindent In the example above, the \lstinline![[carries_dependency]]! attribute was
applied to the declaration of function \lstinline!f! to indicate that the
\lstinline!return! value carries a dependency out of the function. The
compiler may now be able to avoid emitting a memory-fence instruction
for the return value of \lstinline!f!.

The \lstinline![[carries_dependency]]! attribute can also be applied
to one or more of the function's parameter declarations by placing it
immediately after the parameter name:

\begin{emcppslisting}
void g(int* input [[carries_dependency]]); // attribute applied to (ù{\codeincomments{input}}ù)
\end{emcppslisting}
    
\noindent In the declaration of function \lstinline!g! in the example above, the
\lstinline![[carries_dependency]]! attribute is applied to the
\lstinline!input! parameter to indicate that a dependency is carried
through that parameter into the function, which may obviate the
compiler's having to emit an unnecessary memory-fence instruction for
the \lstinline!input! parameter; see \featureref{\locationa}{attributes}.
%{\intraref{{Attribute Syntax}}}.

In both cases, if a function or a parameter declaration specifies the
\lstinline![[carries_dependency]]! attribute, the first declaration of
that function shall specify that \lstinline![[carries_dependency]]!
attribute. Similarly, if the first declaration of a function or one of
its parameters specifies the \lstinline![[carries_dependency]]! attribute
in one translation unit and the first declaration of the same function
in another translation unit doesn't, the program is \emcppsgloss{ill formed, no diagnostic required (IFNDR)}.

%The user is responsible for ensuring that an existing dependency chain
%is available if needed for synchronization purposes. The
%\lstinline![[carries_dependency]]! attribute will not create a \mbox{dependency}.

It is important to note that while the \lstinline![[carries_dependency]]! attribute informs the compiler about the presence of a dependency chain, it does not itself create one.  The dependency chain must be present in the implementation to have any effect on synchronization.

\subsection[Use Cases]{Use Cases}\label{use-cases-carriesdependency}

\subsubsection[Producer-consumer programming pattern]{Producer-consumer programming pattern}\label{producer-consumer-programming-pattern}

The popular producer-consumer programming pattern uses
\emph{release-acquire} pairs to synchronize between threads:

%\begin{emcppslisting}
%#include <cassert>  // standard C (ù{\codeincomments{assert}}ù) macro
%#include <atomic>   // (ù{\codeincomments{std::atomic}}ù), (ù{\codeincomments{std::memory\_order\_release}}ù)
%                    // (ù{\codeincomments{std::memory\_order\_acquire}}ù)
%
%struct S
%{
%    int i;
%    char c;
%    double d;
%};
%
%S data;
%std::atomic<int> guard;
%
%void producerThread()
%{
%    data.i = 42;
%    data.c = 'c';
%    data.d = 5.0;
%    guard.store(1, std::memory_order_release);
%}
%
%void consumerThread()
%{
%    if (guard.load(std::memory_order_acquire) == 1)
%    {
%        // By dint of the release-acquire guarantee, we know that all the
%        // data changes are visible if the guard change is visible.
%        assert(data.i == 42);
%        assert(data.c == 'c');
%        assert(data.d == 5.0);
%    }
%}
%\end{emcppslisting}
\newpage%%%%%%%
\begin{emcppslisting}
// my_shareddata.h                                                              
void initSharedData();                                                          
    // Initialize the shared data of (ù{\codeincomments{my\_shareddata.o}}ù) to a well-known          
    // aggregation of values.                                                   
                                                                                
void accessSharedData();                                                        
    // Confirm that the shared data of (ù{\codeincomments{my\_shareddata.o}}ù) have been initialized  
    // and have their expected values.
\end{emcppslisting}
\vspace*{2ex}
\begin{emcppslisting}
// my_shareddata.cpp
#include <my_shareddata.h>

#include <atomic>  // (ù{\codeincomments{std::atomic}}ù), (ù{\codeincomments{std::memory\_order\_release}}ù), and
                   // (ù{\codeincomments{std::memory\_order\_acquire}}ù)
#include <cassert> // standard C (ù{\codeincomments{assert}}ù) macro

struct S
{
    int    i;
    char   c;
    double d;
};

static S                data;     // (ù{\codeincomments{static}}ù) for insulation
static std::atomic<int> guard(0); // (ù{\codeincomments{static}}ù) for insulation

void initSharedData()
{
    data.i = 42;
    data.c = 'c';
    data.d = 5.0;

    guard.store(1, std::memory_order_release);
}

void accessSharedData()
{
    while(0 == guard.load(std::memory_order_acquire))
        /* empty */ ;

    assert(42  == data.i);
    assert('c' == data.c);
    assert(5.0 == data.d);
}
\end{emcppslisting}
\newpage%%%%%%%
\begin{emcppslisting}
// my_app.cpp
#include <my_shareddata.h>
#include <thread>  // (ù{\codeincomments{std::thread}}ù)

int main()
{
    std::thread t2(accessSharedData);
    std::thread t1(  initSharedData);

    t1.join();
    t2.join();
}
\end{emcppslisting}
    
\noindent When this \emph{release-acquire} \emph{synchronization paradigm} is
used, the compiler must maintain the ordering of the statements to avoid
breaking the \emph{release-acquire} guarantee; the compiler will also
need to insert memory-fence instructions to prevent the hardware from
breaking this guarantee.

If we wanted to modify the example above to use \emph{release-consume}
semantics, we would somehow need to make the \lstinline!assert! statements
a part of the dependency chain on the \lstinline!load! from the
\lstinline!guard! object. We can accomplish this because reading data
through a pointer establishes a dependency chain between the reading of
that pointer value and the reading of the referenced data.  Since \emph{release-consume} allows the developer to specify that data of concern, using that policy instead of \emph{release-acquire} policy (in the code example above) allows the compiler to be more selective in its use of memory fences::

%\begin{emcppslisting}[emcppsbatch=e1]
%#include <cassert>  // standard C (ù{\codeincomments{assert}}ù) macro
%#include <atomic>   // (ù{\codeincomments{std::atomic}}ù), (ù{\codeincomments{std::memory\_order\_release}}ù)
%                    // (ù{\codeincomments{std::memory\_order\_consume}}ù)
%
%struct S
%{
%    int i;
%    char c;
%    double d;
%};
%
%S data;
%std::atomic<S*> guard(nullptr);
%
%void producerThread()
%{
%    data.i = 42;
%    data.c = 'c';
%    data.d = 5.0;
%    guard.store(&data, std::memory_order_release);
%}
%
%void consumerThread()
%{
%    S* setData = guard.load(std::memory_order_consume);
%    if (setData)
%    {
%        assert(setData->i == 42);
%        assert(setData->c == 'c');
%        assert(setData->d == 5.0);
%    }
%}
%\end{emcppslisting}
\enlargethispage*{2ex}
\begin{emcppslisting}
// my_shareddata.cpp (use (ù{\codeincomments{\_*consume}}ù), not (ù{\codeincomments{*\_acquire}}ù))                                                          
#include <my_shareddata.h>                                                      
                                                                                
#include <atomic>  // (ù{\codeincomments{std::atomic}}ù), (ù{\codeincomments{std::memory\_order\_release}}ù), and               
                   // (ù{\codeincomments{std::memory\_order\_consume}}ù) (not (ù{\codeincomments{*\_acquire}}ù))                 
#include <cassert> // standard C (ù{\codeincomments{assert}}ù) macro                                   
                                                                                
struct S                                                                        
{                                                                               
    /* definition not changed */                                                
};                                                                              
                                                                                
static S               data;           // (ù{\codeincomments{static}}ù) for insulation (as before)   
static std::atomic<S*> guard(nullptr); // guards just one (ù{\codeincomments{struct S}}ù).           
                                                                                
void initSharedData()                                                           
{                                                                               
    data.i = 42;   // as before                                                 
    data.c = 'c';  // as before                                                 
    data.d = 5.0;  // as before                                                 
                                                                                
    guard.store(&data, std::memory_order_release);  // Set (ù{\codeincomments{\&data}}ù), not 1.      
}                                                                               
                                                                                
void accessSharedData()                                                         
{                                                                               
    S *sharedDataPtr = nullptr;                                                 
                                                                                
    // Load using (ù{\codeincomments{*\_consume}}ù), not (ù{\codeincomments{*\_acquire}}ù).                                 
    while (nullptr == (sharedDataPtr = guard.load(std::memory_order_consume)))  
        /* empty */ ;                                                           
                                                                                
    assert(&data == sharedDataPtr);                                             
                                                                                
    assert(42  == sharedDataPtr->i);                                            
    assert('c' == sharedDataPtr->c);                                            
    assert(5.0 == sharedDataPtr->d);                                            
}
\end{emcppslisting}
    
\noindent Finally, if we want to start to refactor the work of
the \lstinline!my_sharedata! component into multiple 
 functions across different
translation units, we would want to carefully apply the 
\lstinline![[carries_dependency]]! attribute to the newly refactored
functions, so calling into these functions might conceivably be better
optimized:

%\begin{emcppslisting}[emcppsbatch=e1]
%[[carries_dependency]] S* loadData()
%{
%    return guard.load(std::memory_order_consume);
%}
%
%void checkData(S* s [[carries_dependency]])
%{
%    assert(s->i == 42);
%    assert(s->c == 'c');
%    assert(s->d == 5.0);
%}
%
%void betterThreadB()
%{
%    S* setData = loadData();
%    if (setData)
%    {
%        checkData(setData);
%    }
%}
%\end{emcppslisting}
\begin{emcppslisting}
// (ù{\codeincomments{my\_shareddataimpl.h}}ù)                                                          
                                                                                
struct S                                                                        
{                                                                               
    int    i;                                                                   
    char   c;                                                                   
    double d;                                                                   
};                                                                              
                                                                                
[[carries_dependency]] S* getSharedDataPtr();                                   
    // Return the address of the shared data in this translation unit.          
                                                                                
void releaseSharedData(S *sharedDataPtr [[carries_dependency]]);                
    // Release the shared data in this translation unit.  The behavior is       
    // undefined unless (ù{\codeincomments{getSharedDataPtr() == sharedDataPtr}}ù).                  
                                                                                
[[carries_dependency]] S* accessInitializedSharedData();                        
    // Return the address of the initialized shared data in this translation    
    // unit.                                                                    
                                                                                
void checkSharedDataValue(S* s [[carries_dependency]],                          
                          int    i,                                             
                          char   c,                                             
                          double d);                                            
    // Confirm that data at the specified (ù{\codeincomments{s}}ù) has the specified (ù{\codeincomments{i}}ù), (ù{\codeincomments{c}}ù), and   
    // (ù{\codeincomments{d}}ù) as constituent values.
\end{emcppslisting}
\newpage%%%%%%
\begin{emcppslisting}
// (ù{\codeincomments{my\_shareddataimpl.cpp}}ù)                                                        
                                                                                
#include <my_shareddataimpl.h>                                                  
                                                                                
#include <cassert>                                                              
#include <atomic>                                                               
                                                                                
static S               data;           // (ù{\codeincomments{static}}ù) for insulation               
static std::atomic<S*> guard(nullptr); // guards one (ù{\codeincomments{struct S}}ù).                
                                                                                
[[carries_dependency]] S* getSharedDataPtr()                                    
{                                                                               
    return &data;                                                               
}                                                                               
                                                                                
void releaseSharedData(S *sharedDataPtr [[carries_dependency]])                 
{                                                                               
    assert(&data == sharedDataPtr);                                             
                                                                                
    guard.store(sharedDataPtr, std::memory_order_release);                      
}                                                                               
                                                                                
[[carries_dependency]] S* accessInitializedSharedData()                         
{                                                                               
    S *sharedDataPtr = nullptr;                                                 
                                                                                
    while (nullptr == (sharedDataPtr = guard.load(std::memory_order_consume)))  
        /* empty */ ;                                                           
                                                                                
    assert(&data == sharedDataPtr);                                             
                                                                                
    return sharedDataPtr;                                                       
}  

void checkSharedDataValue(S*     s [[carries_dependency]],                      
                          int    i,                                             
                          char   c,                                             
                          double d)                                             
{                                                                               
    assert(i == s->i);                                                          
    assert(c == s->c);                                                          
    assert(d == s->d);                                                          
}    
\end{emcppslisting}
\newpage%%%%%
\begin{emcppslisting}
// (ù{\codeincomments{my\_shareddata.cpp}}ù) (re-factored to use (ù{\codeincomments{*impl}}ù))                              
#include <my_shareddata.h>                                                      
#include <my_shareddataimpl.h>                                                  
                                                                                
void initSharedData()                                                           
{                                                                               
    S *sharedDataPtr = getSharedDataPtr();                                      
                                                                                
    sharedDataPtr->i = 42;                                                      
    sharedDataPtr->c = 'c';                                                     
    sharedDataPtr->d = 5.0;                                                     
                                                                                
    releaseSharedData(sharedDataPtr);                                           
}                                                                               
                                                                                
void accessSharedData()                                                         
{                                                                               
    S *sharedDataPtr = accessInitializedSharedData();                           
    checkSharedDataValue(sharedDataPtr, 42, 'c', 5.0);                          
}       
\end{emcppslisting}

    
%\noindent Again, as of this writing, all known compilers implement \emph{consume}
%loads as \emph{acquire} loads and thus fail to provide the desired
%optimization.

\subsection[Potential Pitfalls]{Potential Pitfalls}\label{potential-pitfalls}

\subsubsection[No practical use on current platforms]{No practical use on current platforms}\label{no-practical-use-on-current-platforms}

All known compilers promote \emph{consume} loads to \emph{acquire}
loads, thus failing to omit superfluous memory-fence instructions.
Developers writing code with the expectation that it will be run under
the more efficient \emcppsgloss[release consume]{release-consume} \emcppsgloss{synchronization paradigm} will find that their code will continue to work --- as
expected --- under the more conservative \emcppsgloss[release acquire]{release-acquire}
guarantees until such time as a theoretical, not-yet-existent compiler
that properly supports the \emcppsgloss[release consume]{release-consume}
\emcppsgloss{synchronization paradigm} becomes widely available. In the
meantime, applications that require the potential performance benefits
of \emph{consume} semantics typically make careful use of platform-specific functionality instead.{\cprotect\footnote{Since C++17, the
use of \lstinline!memory_order_consume! has been explicitly
  discouraged after the acceptance of \cite{boehm16}. The specific
  note in the standard now says, ``Prefer
  \lstinline!memory_order_acquire!, which provides stronger guarantees
  than \lstinline!memory_order_consume!. Implementations have found it infeasible to provide performance better
  than that of \lstinline!memory_order_acquire!. Specification revisions
  are under consideration'' (\cite{iso17}, section~32.4 ``Order and Consistency," paragraph~1.3, Note~1, p.~1346). }}

\subsection[Annoyances]{Annoyances}\label{annoyances}

\hspace*{\fill}\\


%\subsubsection[Ill formed when inconsistently applied]{Ill formed when inconsistently applied}\label{ill-formed-when-inconsistently-applied}
%
%Like many aspects of a declaration, such as the
%\lstinline![[noreturn]]! attribute (see\linebreak%%%%%
%\featureref{\locationa}{the-noreturn-attribute}),
%\lstinline!alignas! specifier (see \featureref{\locationa}{alignas}), language linkage, and so on,
%the \lstinline![[carries_dependency]]! attribute must be applied to a
%function's declaration consistently across all translation units.
%Failing to apply it on the first declaration in a translation unit and
%then later to a (re)declaration is ill formed. Such uniqueness issues
%are readily dispatched when (1) each function's owner supplies a
%corresponding header having the canonical declarations for that function
%and (2) every client includes that corresponding header rather than
%attempting to redefine the function locally.

\newpage%%%%%%
\subsection[See Also]{See Also}\label{see-also}

\begin{itemize}
\item{%\intraref{Attribute Syntax}
\seealsoref{attributes}{\seealsolocationa}provides an in-depth discussion of how attributes pertain to C++ language entities.}
\item{%\intraref{\texttt{[[noreturn]]} Attribute}
\seealsoref{the-noreturn-attribute}{\seealsolocationa}offers an example of another \emph{attribute} that \emph{is} implemented ubiquitously.}
\end{itemize}

\subsection[Further Reading]{Further Reading}\label{further-reading}

\begin{itemize}
\item{\cite{marton17}}
\item{\cite{marton18}}
\end{itemize}




\newpage
%\section[{\tt final}]{Preventing Overriding and Derivation}\label{final}



\emcppsFeature{
    short={\lstinline!final!},
    tocshort={\TOCCode final},
    long={Preventing Overriding and Derivation},
    rhshort={\RHCode final},
}{final}
\setcounter{table}{0}
\setcounter{footnote}{0}
\setcounter{lstlisting}{0}
%\section[{\tt final}]{Preventing Overriding and Derivation}\label{final}


placeholder


\newpage
%\section[{\tt friend} '11]{Extended {\SecCode friend} Declarations}\label{extended-friend-declarations}
% 12 Feb 2021, revisions in; ready for Josh's code fixes
% 16 Feb 2021 JMB; code compiles
% 24 Feb 2021, copyedits in and proofed

\emcppsFeature{
    short={\lstinline!friend! '11},
    tocshort={{\TOCCode friend} '11},
    long={Extended {\SecCode friend} Declarations},
    toclong={Extended \lstinline!friend! Declarations},
    rhshort={{\RHCode friend} '11},
}{extended-friend-declarations}
\setcounter{table}{0}
\setcounter{footnote}{0}
\setcounter{lstlisting}{0}
%\section[{\tt friend} '11]{Extended {\SecCode friend} Declarations}\label{extended-friend-declarations}


Extended \lstinline!friend! declarations enable a class's author to
designate a type alias, a template parameter, or any other previously
declared type as a \lstinline!friend! of that class.

\subsection[Description]{Description}\label{description-extendedfriend}

A \lstinline!friend! declaration located within a given
\emcppsgloss[user defined type (UDT)]{user-defined type (UDT)} grants a designated type (or
\emph{free} function) access to private and protected members of that
class. Because the extended \lstinline!friend! syntax does not affect
\emph{function} friendships, this feature section addresses extended
friendship only between \emph{types}.

Prior to C++11, the Standard required an \emph{elaborated type
specifier} to be provided after the \lstinline!friend! keyword to designate
some other \emph{class} as being a \lstinline!friend! of a given type. An
elaborated type specifier for a class is a syntactical element having the form\linebreak[4]
\mbox{\lstinline!<class|struct|union>!~\lstinline!<identifier>!}. Elaborated
type specifiers can be used to refer to a previously declared entity or
to declare a new one, with the restriction that such an entity is
one of \lstinline!class!, \lstinline!struct!, or \lstinline!union!:

\begin{emcppslisting}
// C++03

struct S;
class C;
enum E { };

struct X0
{
    friend S;         // Error, not legal C++98/03
    friend struct S;  // OK, refers to (ù{\codeincomments{S}}ù) above
    friend class S;   // OK, refers to (ù{\codeincomments{S}}ù) above (might warn)
    friend class C;   // OK, refers to (ù{\codeincomments{C}}ù) above
    friend class C0;  // OK, declares (ù{\codeincomments{C0}}ù) in (ù{\codeincomments{X0}}ù)'s namespace
    friend union U0;  // OK, declares (ù{\codeincomments{U0}}ù) in (ù{\codeincomments{X0}}ù)'s namespace
    friend enum E;    // Error, (ù{\codeincomments{enum}}ù) cannot be a friend.
    friend enum E2;   // Error, (ù{\codeincomments{enum}}ù) cannot be forward-declared.
};
\end{emcppslisting}
    
\noindent This restriction prevents other potentially useful entities, e.g., type
aliases and template parameters, from being designated as friends:

\begin{emcppslisting}
// C++03

struct S;
typedef S SAlias;

struct X1
{
    friend struct SAlias;  // Error, using typedef-name after (ù{\codeincomments{struct}}ù)
};

template <typename T>
struct X2
{
    friend class T;        // Error, using template type parameter after (ù{\codeincomments{class}}ù)
};
\end{emcppslisting}
    
\noindent Furthermore, even though an entity belonging to a namespace other than
the class containing a \lstinline!friend! declaration might be visible,
explicit qualification is required to avoid unintentionally declaring a new type:

\begin{emcppslisting}
// C++03

struct S;  // This (ù{\codeincomments{S}}ù) resides in the global namespace.

namespace ns
{
    class X3
    {
        friend struct S;
            // OK, but declares a new (ù{\codeincomments{ns::S}}ù) instead of referring to (ù{\codeincomments{::S}}ù)
    };
}
\end{emcppslisting}
    
\noindent C++11 relaxes the aforementioned \emph{elaborated type specifier}
requirement and extends the classic \lstinline!friend! syntax by instead
allowing either a \emph{simple type specifier}, which is any unqualified
type or type alias, or a \emph{typename specifier}, e.g., the name of a
template \emph{type} parameter or dependent type thereof:

\begin{emcppslisting}
struct S;
typedef S SAlias;

namespace ns
{
    template <typename T>
    struct X4
    {
        friend T;           // OK
        friend S;           // OK, refers to (ù{\codeincomments{::S}}ù)
        friend SAlias;      // OK, refers to (ù{\codeincomments{::S}}ù)
        friend decltype(0); // OK, equivalent to (ù{\codeincomments{friend int;}}ù)
        friend C;           // Error, (ù{\codeincomments{C}}ù) does not name a type.
    };
}
\end{emcppslisting}
    
\noindent Notice that now it is again possible to declare as a \lstinline!friend! a
type that is expected to have already been declared, e.g., \lstinline!S!,
without having to worry that a typo in the spelling of the type would
silently introduce a new type declaration, e.g., \lstinline!C!, in the
enclosing scope.

Finally, consider the hypothetical case in which a class template,
\lstinline!C!, befriends a \emph{dependent} (e.g., nested) type, \lstinline!N!,
of its type parameter, \lstinline!T!:

\begin{emcppslisting}[emcppsbatch=e6]
template <typename T>
class C
{
    friend typename T::N;       // (ù{\codeincomments{N}}ù) is a *dependent* *type* of parameter (ù{\codeincomments{T}}ù).
    enum { e_SECRET = 10022 };  // This information is (ù{\codeincomments{private}}ù) to class (ù{\codeincomments{C}}ù).
};

struct S
{
    struct N
    {
        static constexpr int f()  // (ù{\codeincomments{f}}ù) is eligible for compile-time computation.
        {
            return C<S>::e_SECRET;  // Type (ù{\codeincomments{S::N}}ù) is a (ù{\codeincomments{friend}}ù) of (ù{\codeincomments{C<S>}}ù).
        }
    };
};

static_assert(S::N::f() == 10022, "");  // (ù{\codeincomments{N}}ù) has (ù{\codeincomments{private}}ù) access to (ù{\codeincomments{C<S>}}ù).
\end{emcppslisting}
    
\noindent In the example above, the nested type \lstinline!S::N! --- but not
\lstinline!S! itself --- has private access to \lstinline!C<S>::e_SECRET!.{\cprotect\footnote{Note that the need for \lstinline!typename! in the \lstinline!friend!
declaration in the example above to introduce the dependent type \lstinline!N! is relaxed
in C++20. For information on other contexts in which
\lstinline!typename! will eventually no longer be required, see~\cite{meredith20}.}}

\subsection[Use Cases]{Use Cases}\label{use-cases}

\subsubsection[Safely declaring a previously declared type to be a friend]{Safely declaring a previously declared type to be a friend}\label{safely-declaring-a-previously-declared-type-to-be-a-friend}

In C++98/03, to befriend a type that was already declared required
\emph{redeclaring} it. If the type were misspelled in the friend
declaration, a new type would be declared:

\begin{emcppslisting}
class Container { /* ... */ };

class ContainerIterator
{
    friend class Contianer;  // Compiles but wrong: (ù{\codeincomments{ia}}ù) should have been (ù{\codeincomments{ai}}ù).
    // ...
};
\end{emcppslisting}
    
\noindent The code above will compile and appear to be correct until
\lstinline!ContainerIterator! attempts to access a \lstinline!private! or
\lstinline!protected! member of \lstinline!Container!. At that point, the
compiler will surprisingly produce an error. As of C++11, we have the
option of preventing this mistake by using extended \lstinline!friend!
declarations:

\begin{emcppslisting}
class Container { /* ... */ };

class ContainerIterator
{
    friend Contianer;  // Error, (ù{\codeincomments{Contianer}}ù) not found
    // ...
};
\end{emcppslisting}
    

\subsubsection[Befriending a type alias used as a customization point]{Befriending a type alias used as a customization point}\label{befriending-a-type-alias-used-as-a-customization-point}

In C++03, the only option for friendship was to specify a particular
\lstinline!class! or \lstinline!struct! when granting private access. Let's
begin by considering a scenario in which we have an
\emcppsgloss{in-process} \emcppsgloss[value semantic type (VST)]{value-semantic type (VST)}
that serves as a \emph{handle} to a platform-specific object, such as a
\lstinline!Window! in a graphical application. (When used to qualify a VST, the
term \emcppsgloss{in-process}, also called \emph{in-core}, refers to a type
that has typical value-type--like operations but does not refer to a
value that is meaningful outside of the current process.\footnote{See
  \cite{lakos2a}, section~4.2.}) Large parts of a codebase
may seek to interact with \lstinline!Window! objects without needing or
obtaining access to the internal representation.

A very small part of the codebase that handles platform-specific window
management, however, needs privileged access to the internal
representation of \lstinline!Window!. One way to achieve this goal is to
make the platform-specific \lstinline!WindowManager! a \lstinline!friend! of
the \lstinline!Window! class; however, see \intraref{potential-pitfalls-extendedfriend}{long-distance-friendship}. 
%\textit{\titleref{potential-pitfalls-extendedfriend}: \titleref{long-distance-friendship}} on page~\pageref{long-distance-friendship}.

\begin{emcppslisting}
class WindowManager;  // forward declaration enabling extended (ù{\codeincomments{friend}}ù) syntax

class Window
{
private:
    friend class WindowManager;  // could instead use (ù{\codeincomments{friend WindowManager;}}ù)
    int d_nativeHandle;          // in-process (only) value of this object

public:
    // ... all the typical (e.g., special) functions we expect of a value type
};
\end{emcppslisting}
    
\noindent In the example above, class \lstinline!Window! befriends class
\lstinline!WindowManager!, granting it private access. Provided that the
implementation of \lstinline!WindowManager! resides in the same physical
\emcppsgloss[components]{component} as that of class \lstinline!Window!, no
\emcppsgloss{long-distance friendship} results. The consequence of such a
monolithic design would be that every client that makes use of the
otherwise lightweight \lstinline!Window! class would necessarily depend
physically on the presumably heavier-weight \lstinline!WindowManager! class.

Now consider that the \lstinline!WindowManager! implementations on
different platforms might begin to diverge significantly. To keep the
respective implementations maintainable, one might choose to factor them
into distinct C++ types, perhaps even defined in separate files, and to
use a \emph{type alias} determined using platform-detection preprocessor
macros to configure that alias:

%%%%%%%%%%%%%%%%%%%%% all separate. intended? 
%%% yes, and VR requests some separation

\begin{emcppslisting}[emcppsbatch=e1]
// windowmanager_win32.h:

#ifdef WIN32
class Win32WindowManager { /* ... */ };
#endif
\end{emcppslisting}

\vspace*{2ex}    

\begin{emcppshiddenlisting}[emcppsbatch=e1]
// windowmanager_unix.h:
#define UNIX
\end{emcppshiddenlisting}
\begin{emcppslisting}[emcppsbatch=e1]
// windowmanager_unix.h:

#ifdef UNIX
class UnixWindowManager { /* ... */ };
#endif
\end{emcppslisting}
    
\vspace*{2ex}

\begin{emcppslisting}[emcppsbatch=e1]
// windowmanager.h:

#ifdef WIN32
#include <windowmanager_win32.h>
typedef Win32WindowManager WindowManager;
#else
#include <windowmanager_unix.h>
typedef UnixWindowManager WindowManager;
#endif
\end{emcppslisting}
    
\vspace*{2ex}

\begin{emcppslisting}[emcppsbatch=e1]
// window.h:
#include <windowmanager.h>

class Window
{
private:
    friend WindowManager;  // C++11 extended friend declaration
    int d_nativeHandle;

public:
    // ...
};
\end{emcppslisting}
   
%%%%%%%%%%%%%%%%%%%%%%%      

In this example, class \lstinline!Window! no longer befriends a specific
class named \lstinline!WindowManager!; instead, it befriends the
\lstinline!WindowManager! type alias, which in turn has been set to the
correct platform-specific window manager implementation. Such extended
use of \lstinline!friend! syntax was not available in C++03.

Note that this use case involves \emcppsgloss{long-distance friendship}
inducing an implicit cyclic dependency between the \emcppsgloss[components]{component}
implementing \lstinline!Window! and those implementing
\lstinline!WindowManager!; see \intraref{potential-pitfalls-extendedfriend}{long-distance-friendship}. 
%\textit{\titleref{potential-pitfalls-extendedfriend}: \titleref{long-distance-friendship}} on page~\pageref{long-distance-friendship}. 
Such designs, though
undesirable, can result from an emergent need to add new platforms while
keeping tightly related code sequestered within smaller, more manageable
physical units. An alternative design would be to obviate the
\emcppsgloss{long-distance friendship} by widening the API for the
\lstinline!Window! class, the natural consequence of which would be to
invite public client abuse vis-a-vis \emcppsgloss{Hyrum's law}.

\subsubsection[Using the \lstinline!PassKey! idiom to enforce initialization]{Using the {\SubsubsecCode PassKey} idiom to enforce initialization}\label{using-the-passkey-idiom-to-enforce-initialization}

Prior to C++11,
efforts to grant private access to a class defined in a separate
physical unit required declaring the higher-level type itself to be a
\lstinline!friend!, resulting in this highly undesirable form of
friendship; see \intraref{potential-pitfalls-extendedfriend}{long-distance-friendship}.  
%\textit{\titleref{potential-pitfalls-extendedfriend}: \titleref{long-distance-friendship}} on page~\pageref{long-distance-friendship}. 
The ability in C++11 to
declare a template \emph{type} parameter or any other type specifier to
be a friend affords new opportunities to enforce selective
private access (e.g., to one or more individual functions) without
explicitly declaring another type to be a \lstinline!friend!; see also \intrarefsimple{granting-a-specific-type-access-to-a-single-private-function}. 
%\textit{\titleref{description-extendedfriend}: \titleref{granting-a-specific-type-access-to-a-single-private-function}} on page~\pageref{granting-a-specific-type-access-to-a-single-private-function}. 
In this use case, however, our use of extended
\lstinline!friend! syntax to befriend a template parameter is unlikely to run
afoul of sound physical design.

Let's say we have a commercial library, and we want it to verify a
software-license key in the form of a C-style string, prior to allowing use of 
other parts of the API:


\begin{emcppshiddenlisting}[emcppsbatch=e2]
class LibPassKey;
\end{emcppshiddenlisting}
\begin{emcppslisting}[emcppsbatch=e2]
// simplified pseudocode
LibPassKey initializeLibrary(const char* licenseKey);
int utilityFunction1(LibPassKey object /*, ... (other parameters) */);
int utilityFunction2(LibPassKey object /*, ... (other parameters) */);
\end{emcppslisting}
    
\noindent Knowing full well that this is not a \emph{secure} approach and that
innumerable deliberate, malicious ways exist to get around the C++ type
system, we nonetheless want to create a plausible regime where no
\emph{well-formed} code can \emph{accidentally} gain access
to library functionality other than by legitimately initializing the
system using a valid license key. We could easily cause a
function to \lstinline!throw!, \lstinline!abort!, and so on, at run time when
the function is called prior to the client's license key being
authenticated. However, part of our goal, as a friendly library vendor, is to
ensure that clients do not \emph{inadvertently} call other library
functions prior to initialization. To that end, we propose the following protocol:
\begin{enumerate}
\item{use an instantiation of the \lstinline!PassKey! class template\cprotect\footnote{\cite{mayrand15}} that only our API \emph{utility}\linebreak[4] \emph{\lstinline!struct!}\cprotect\footnote{\cite{lakos20}, section~2.4.9, pp.~312--321, specifically Figure~2-23, p.~316} can create}
\item{return a constructed object of this type only upon successful validation of the license key}
\item{require that clients present this (constructed) passkey \emph{object} every time they invoke any other function in the API}
\end{enumerate}
Here's an example that encompasses all three aforementioned points:

\begin{emcppslisting}
template <typename T>
class PassKey  // reusable standard utility type
{
    PassKey() { }  // private default constructor (no aggregate initialization)
    friend T;      // Only (ù{\codeincomments{T}}ù) is allowed to create this object.
};

struct BestExpensiveLibraryUtil
{
    class LicenseError { /*...*/ };  // thrown if license string is invalid

    using LibPassKey = PassKey<BestExpensiveLibraryUtil>;
        // This is the type of the (ù{\codeincomments{PassKey}}ù) that will be returned when this
        // utility is initialized successfully, but only this utility is able 
        // to construct an object of this type. Without a valid license string,
        // the client will have no way to create such an object and thus no way
        // to call functions within this library.

    static LibPassKey initializeLibrary(const char* licenseKey)
        // This function must be called with a valid (ù{\codeincomments{licenseKey}}ù) string prior
        // to using this library; if the supplied license is valid, a
        // (ù{\codeincomments{LibPassKey}}ù) *object* will be returned for mandatory use in *all*
        // subsequent calls to useful functions of this library. This function
        // throws (ù{\codeincomments{LicenseError}}ù) if the supplied (ù{\codeincomments{licenseKey}}ù) string is invalid.
    {
        if (isValid(licenseKey))
        {
            // Initialize library properly.

            return LibPassKey();
                // Return a default-constructed (ù{\codeincomments{LibPassKey}}ù). Note that only
                // this utility is able to construct such a key.
        }

        throw LicenseError();  // supplied license string was invalid
    }

    static int doUsefulStuff(LibPassKey key /*,...*/);
        // The function requires a (ù{\codeincomments{LibPassKey}}ù) object, which can be constructed
        // only by invoking the (ù{\codeincomments{static initializeLibrary}}ù) function, to be
        // supplied as its first argument. ...

private:
    static bool isValid(const char* key);
        // externally defined function that returns true if (ù{\codeincomments{key}}ù) is valid
};
\end{emcppslisting}
    
\noindent Other than going outside the language with invalid constructs or circumventing the
type system with esoteric tricks, this approach, among other things,
prevents invoking the\linebreak[4]%%%%%%%%  
\lstinline!doUsefulStuff! function without a proper
license. What's more, the C++ type system \emph{at compile time} forces
a prospective client to have initialized the library before any attempt
is made to use any of its other functionality.

\subsubsection[Granting a specific type access to a single \lstinline!private! function]{Granting a specific type access to a single {\SubsubsecCode private} function}\label{granting-a-specific-type-access-to-a-single-private-function}

When designing in purely logical terms, wanting to grant some other
logical entity special access to a type that no other entity enjoys is a
common situation. Doing so does not necessarily become problematic until
that friendship spans physical boundaries; see \intraref{potential-pitfalls-extendedfriend}{long-distance-friendship}. 
%\textit{\titleref{potential-pitfalls-extendedfriend}: \titleref{long-distance-friendship}} on page~\pageref{long-distance-friendship}.

As a simple approximation to a real-world use
case,\footnote{For an example of a real-world database
implementation that requires managed objects to befriend that database
manager, see \cite{csb15}, section~2.1.} suppose we have a lightweight object-database class, \lstinline!Odb!, that is designed to operate
collaboratively with objects, such as \lstinline!MyWidget!, that are
themselves designed to work collaboratively with \lstinline!Odb!. Every compliant UDT
suitable for management by \lstinline!Odb! will need to maintain an integer
object ID that is read/write accessible by an \lstinline!Odb! object. Under
no circumstances is any other object permitted to access, let alone
modify, that ID independently of the \lstinline!Odb! API.

Prior to C++11, the design of such a feature might require every
participating class to define a data member named \lstinline!d_objectId!
and to declare the \lstinline!Odb! class a \lstinline!friend! (using old-style
\lstinline!friend! syntax):

\begin{emcppslisting}
class MyWidget  // grants just (ù{\codeincomments{Odb}}ù) access to *all* of its private data
{
    int d_objectId;    // required by our collaborative-design strategy
    friend class Odb;  //    "     "   "        "        "       "
    // ...

public:
    // ...
};

class Odb
{
    // ...

public:
    template <typename T>
    void processObject(T& object)
        // This function template is generally callable by clients.
    {
        int& objId = object.d_objectId;
        // ... (process as needed)
    }

    // ...
};
\end{emcppslisting}
    
\noindent In this example, the \lstinline!Odb! class implements the public member
function template,\linebreak[4] \lstinline!processObject!, which then extracts the
\lstinline!objectId! field for access. The collateral damage is that we
have exposed all of our private details to \lstinline!Odb!, which is at
best a gratuitous widening of our sphere of encapsulation.

Using the \lstinline!PassKey! pattern allows us to be more selective with
what we share:

\begin{emcppslisting}[emcppsbatch=e3]
template <typename T>
class Passkey
    // Implement this eminently reusable (ù{\codeincomments{Passkey}}ù) class template again here.
{
    Passkey() { }  // prevent aggregate initialization
    friend T;      // Only the (ù{\codeincomments{T}}ù) in (ù{\codeincomments{PassKey<T>}}ù) can create a (ù{\codeincomments{PassKey}}ù) object.
    Passkey(const Passkey&) = delete;             // no copy/move construction
    Passkey& operator=(const Passkey&) = delete;  // no copy/move assignment
};
\end{emcppslisting}
    
\noindent We are now able to adjust the design of our systems such that only the
minimum private functionality is exposed to \lstinline!Odb!:

\begin{emcppslisting}[emcppsbatch=e3]
class Odb;      // Objects of this class have special access to other objects.

class MyWidget  // grants just (ù{\codeincomments{Odb}}ù) access to only its (ù{\codeincomments{objectId}}ù) member function
{
    int d_objectId;  // must have an (ù{\codeincomments{int}}ù) data member of any name we choose
    // ...

public:
    int& objectId(const Passkey<Odb>&) { return d_objectId; }
        // Return a non-(ù{\codeincomments{const}}ù) reference to the mandated (ù{\codeincomments{int}}ù) data member.
        // (ù{\codeincomments{objectId}}ù) is callable only within the scope of (ù{\codeincomments{Odb}}ù).

    // ...
};

class Odb
{
    // ...

public:
    template <typename T>
    void processObject(T& object)
        // This function template is generally callable by clients.
    {
        int& objId = object.objectId(Passkey<Odb>());
        // ...
    }

    // ...
};
\end{emcppslisting}
    
\noindent Instead of granting \lstinline!Odb! private access to \emph{all}
encapsulated implementation details of\linebreak[4] \mbox{\lstinline!MyWidget!}, this example
uses the \lstinline!PassKey! idiom to enable just \lstinline!Odb! to call the
(syntactically \lstinline!public!) \lstinline!objectId! member function of
\lstinline!MyWidget! with no private access whatsoever. As a further
demonstration of the efficacy of this approach, consider that we are
able to create and invoke the \lstinline!processObject! method of an
\lstinline!Odb! object from a function, \lstinline!f!, but we are blocked from
calling the \lstinline!objectId! method of a \lstinline!MyWidget! object
directly:

\begin{emcppslisting}[emcppsbatch=e3]
void f()
{
    Odb mgr;          // object receiving fine-grained privileged access
    MyWidget widget;  // object granting selective private access to just (ù{\codeincomments{Odb}}ù)
    mgr.processObject(widget);

    int& objId = widget.objectId(PassKey<Odb>());  // cannot call out of (ù{\codeincomments{Odb}}ù)
       // Error, (ù{\codeincomments{Passkey<T>::Passkey()}}ù) [with(ù{\codeincomments{T}}ù) = (ù{\codeincomments{Odb}}ù)] is private within
       // this context.
}
\end{emcppslisting}
    
\noindent Notice that use of the extended \lstinline!friend! syntax to befriend a
template parameter and thereby enable the \lstinline!PassKey! idiom here
improved the granularity with which we effectively grant privileged
access to an individually named type but didn't fundamentally alter the
testability issues that result when private access to specific C++
types is allowed to extend across physical boundaries; again, see \intraref{potential-pitfalls-extendedfriend}{long-distance-friendship}. 
%\textit{\titleref{potential-pitfalls-extendedfriend}: \titleref{long-distance-friendship}} on page~\pageref{long-distance-friendship}.

\subsubsection[Curiously recurring template pattern]{Curiously recurring template pattern}\label{curiously-recurring-template-pattern}

Befriending a template parameter via extended \lstinline!friend!
declarations can be helpful when implementing the \emcppsgloss[curiously recurring template pattern (CRTP)]{curiously
recurring template pattern (CRTP)}. For use-case examples and more
information on the pattern itself, see \intrarefsimple{appendix:-crtp-use-cases}. 
%\textit{\titleref{appendix:-crtp-use-cases}} on page~\pageref{appendix:-crtp-use-cases}.

\subsection[Potential Pitfalls]{Potential Pitfalls}\label{potential-pitfalls-extendedfriend}

\subsubsection[Long-distance friendship]{Long-distance friendship}\label{long-distance-friendship}

Since before C++ was standardized, granting private access via a
\lstinline!friend! declaration across physical boundaries, known as
\emcppsgloss{long-distance friendship}, was
observed\footnote{\cite{lakos96}, section~3.6.1, pp.~141--144}\footnote{\cite{lakos20}, section~2.6, pp.~342--370, specifically p.~367 and p.~362} to potentially lead
to designs that are qualitatively more difficult to understand, test,
and maintain. When a user-defined type, \lstinline!X!, befriends some other
specific type, \lstinline!Y!, in a separate, higher-level translation unit,
testing \lstinline!X! thoroughly without also testing \lstinline!Y! is no
longer possible. The effect is a test-induced cyclic dependency between
\lstinline!X! and \lstinline!Y!. Now imagine that \lstinline!Y! depends on a
sequence of other types, \lstinline!C1!, \lstinline!C2!, \ldots,
\lstinline!CN-2!, each defined in its own physical \emcppsgloss[components]{component},
\lstinline!CI!, where \lstinline!CN-2! depends on \lstinline!X!. The result is a
physical design cycle of size \emph{N}. As \emph{N} increases, the
ability to manage complexity quickly becomes intractable. Accordingly,
the two design imperatives that were most instrumental in shaping the
C++20 \emcppsgloss{modules} feature were (1) to have no cyclic module
dependencies and (2) to avoid intermodule friendships.

\subsection[See Also]{See Also}\label{see-also}

\begin{itemize}
\item{\seealsoref{alias-declarations-and-alias-templates}{\seealsolocationa}are among the beneficiaries of extended friend declarations.}
\end{itemize}

\subsection[Further Reading]{Further Reading}\label{further-reading}

\begin{itemize}
\item{For yet more potential uses of the extended friend pattern in
metaprogramming contexts, such as using CRTP, see
\cite{alexandrescu01}.}
\item{\cite{lakos96}, section~3.6, pp.~136--146, is dedicated to the
classic use (and misuse) of friendship.}
\item{\cite{miller05}}
\item{\cite{{lakos20}} provides extensive advice on \emph{sound}
\emcppsgloss{physical design}, which generally precludes \emcppsgloss{long-distance
friendship}.}
\end{itemize}

\subsection[Appendix: Curiously Recurring Template Pattern Use Cases]{Appendix: Curiously Recurring Template Pattern Use Cases}\label{appendix:-crtp-use-cases}

\subsubsection[Refactoring using the curiously recurring template pattern]{Refactoring using the curiously recurring template pattern}\label{refactoring-using-the-curiously-recurring-template-pattern}

Avoiding code duplication across disparate classes can sometimes be
achieved using a strange template pattern first recognized in the
mid-90s, which has since become known as the \emcppsgloss{curiously recurring
template pattern (CRTP)}. The pattern is \emph{curious} because it
involves the surprising step of declaring as a base class, such as
\lstinline!B!, a template that \emph{expects} the derived class, such as
\lstinline!C!, as a template argument, such as \lstinline!T!:

\begin{emcppslisting}
template <typename T>
class B
{
    // ...
};

class C : public B<C>
{
   // ...
};
\end{emcppslisting}
    
\noindent As a trivial illustration of how the CRTP can be used as a refactoring tool,
suppose that we have several classes for which we would like to track,
say, just the number of active instances:

\begin{emcppslisting}
class A
{
    static int s_count;  // declaration
    // ...

public:
    static int count() { return s_count; }

    A()          { ++s_count; }
    A(const A&)  { ++s_count; }
    A(const A&&) { ++s_count; }
    ~A()         { --s_count; }

    A& operator=(A&)  = default;  // see special members
    A& operator=(A&&) = default;  //  "     "       "
    // ...
};

int A::s_count;  // definition (in (ù{\codeincomments{.cpp}}ù) file)

class B { /* similar to A (above) */ };
// ...

void test()
{          // (ù{\codeincomments{A::s\_count}}ù) = (ù{\codeincomments{0}}ù), (ù{\codeincomments{B::s\_count}}ù) = (ù{\codeincomments{0}}ù)
    A a1;  // (ù{\codeincomments{A::s\_count}}ù) = (ù{\codeincomments{1}}ù), (ù{\codeincomments{B::s\_count}}ù) = (ù{\codeincomments{0}}ù)
    B b1;  // (ù{\codeincomments{A::s\_count}}ù) = (ù{\codeincomments{1}}ù), (ù{\codeincomments{B::s\_count}}ù) = (ù{\codeincomments{1}}ù)
    A a2;  // (ù{\codeincomments{A::s\_count}}ù) = (ù{\codeincomments{2}}ù), (ù{\codeincomments{B::s\_count}}ù) = (ù{\codeincomments{1}}ù)
}          // (ù{\codeincomments{A::s\_count}}ù) = (ù{\codeincomments{0}}ù), (ù{\codeincomments{B::s\_count}}ù) = (ù{\codeincomments{0}}ù)
\end{emcppslisting}
    
\noindent In this example, we have multiple classes, each repeating the same
common machinery. Let's now explore how we might refactor this example
using the CRTP:

\begin{emcppslisting}[emcppsbatch=e4]
template <typename T>
class InstanceCounter
{
protected:
    static int s_count;  // declaration

public:
    static int count() { return s_count; }
};

template <typename T>
int InstanceCounter<T>::s_count;  // definition (in same file as declaration)

struct A : InstanceCounter<A>
{
    A()          { ++s_count; }
    A(const A&)  { ++s_count; }
    A(const A&&) { ++s_count; }
    ~A()         { --s_count; }

    A& operator=(const A&)  = default;
    A& operator=(A&&)       = default;
    // ...
};
\end{emcppslisting}
    
\noindent Notice that we have factored out a common counting mechanism into an
\lstinline!InstanceCounter! class template and then derived our
representative class \lstinline!A! from \lstinline!InstanceCounter<A>!, and we
would do similarly for classes \lstinline!B!, \lstinline!C!, and so on. This
approach works because the compiler does not need to see the derived
type until the point at which the template is instantiated, which will
be \emph{after} it has seen the derived type.

Prior to C++11, however, there was plenty of room for user error.
Consider, for example, forgetting to change the base-type parameter when
copying and pasting a new type:

\begin{emcppslisting}[emcppsbatch=e4]
struct B : InstanceCounter<A>  // Oops! We forgot to change (ù{\codeincomments{A}}ù) to (ù{\codeincomments{B}}ù) in
                               // (ù{\codeincomments{InstanceCounter}}ù): The wrong count will be
                               // updated!
{
    B() { ++s_count; }
};
\end{emcppslisting}
    
\noindent Another problem is that a client deriving from our class can mess with
our protected \lstinline!s_count!:

\begin{emcppslisting}[emcppsbatch=e4]
struct AA : A
{
    AA() { s_count = -1; }  // Oops! *Hyrum's Law* is at work again!
};
\end{emcppslisting}
    
\noindent We could inherit from the \lstinline!InstanceCounter! class privately, but
then \lstinline!InstanceCounter! would have no way to add to the derived
class's public interface, for example, the public \lstinline!count! static
member function.

As it turns out, however, both of these missteps can be erased simply by
making the internal mechanism of the \lstinline!InstanceCounter! template
private and then having \mbox{\lstinline!InstanceCounter!} befriend its template
parameter, \lstinline!T!:

\begin{emcppslisting}
template <typename T>
class InstanceCounter
{
    static int s_count;  // Make this static data member private.
    friend T;            // Allow access only from the derived (ù{\codeincomments{T}}ù).

public:
    static int count() { return s_count; }
};

template <typename T>
int InstanceCounter<T>::s_count;
\end{emcppslisting}
    
\noindent Now if some other class does try to derive from this type, it cannot
access this type's counting mechanism. If we want to suppress even that
possibility, we can declare and default (see \featureref{\locationa}{defaulted-special-member-functions}) 
% ``\titleref{defaulted-special-member-functions}" on page~\pageref{defaulted-special-member-functions}) 
the \lstinline!InstanceCounter!
class constructors to be private as well.

\subsubsection[Synthesizing equality using the curiously recurring template pattern]{Synthesizing equality using the curiously recurring template pattern}\label{synthesizing-equality-using-crtp}

As a second example of code factoring using the CRTP, suppose that we want
to create a factored way of synthesizing \lstinline!operator==! for types
that implement just an \lstinline!operator<!.{\cprotect\footnote{This
example is based on a similar one found on stackoverflow.com:\linebreak[3]
  https://\linebreak[3]stackoverflow.\linebreak[3]com/\linebreak[3]questions/\linebreak[3]4173254/what-is-the-curiously-recurring-template-pattern-crtp}}
In this example, the CRTP base-class template, \lstinline!E!, will
synthesize the homogeneous \lstinline!operator==! for its parameter type,
\lstinline!D!, by returning \lstinline!false! if either argument is \emph{less
than} the other:

\begin{emcppslisting}[emcppsbatch=e5]
template <typename D>
class E { }; // CRTP base class used to synthesize (ù{\codeincomments{operator==}}ù) for (ù{\codeincomments{D}}ù)

template <typename D>
bool operator==(const E<D>& lhs, const E<D>& rhs)
{
    const D& d1 = static_cast<const D&>(lhs);  // derived type better be (ù{\codeincomments{D}}ù)
    const D& d2 = static_cast<const D&>(rhs);  //    "     "     "    "  "
    return !(d1 < d2) && !(d2 < d1);           // assuming (ù{\codeincomments{D}}ù) has an (ù{\codeincomments{operator<}}ù)
}
\end{emcppslisting}
    
\noindent A client that implements an \lstinline!operator<! can now reuse this CRTP
base case to synthesize an \lstinline!operator==!:

\begin{emcppshiddenlisting}[emcppsbatch={e5,e6}]
#include <cassert>  // standard C (ù{\codeincomments{assert}}ù) macro
\end{emcppshiddenlisting}

\begin{emcppslisting}[emcppsbatch=e5]
struct S : E<S>
{
    int d_size;
};

bool operator<(const S& lhs, const S& rhs)
{
    return lhs.d_size < rhs.d_size;
}

void test1()
{
    S s1; s1.d_size = 10;
    S s2; s2.d_size = 10;

    assert(s1 == s2);  // compiles and passes
}
\end{emcppslisting}
    
\noindent As this code snippet suggests, the base-class template, \lstinline!E!, is
able to use the template parameter, \lstinline!D! (representing the derived
class, \lstinline!S!), to synthesize the homogeneous free
\lstinline!operator==! function for \lstinline!S!.

Prior to C++11, no means existed to guard against accidents, such as
inheriting from the wrong base and then perhaps even forgetting to
define the \lstinline!operator<!:

\begin{emcppslisting}[emcppsbatch=e5]
struct P : E<S>  // Oops! should have been (ù{\codeincomments{E(P)}}ù) -- a serious latent defect
{
    int d_x;
    int d_y;
};

void test2()
{
    P p1; p1.d_x = 10; p1.d_y = 15;
    P p2; p2.d_x = 10; p2.d_y = 20;

    assert( !(p1 == p2) );  // Oops! This fails because of (ù{\codeincomments{E(S)}}ù) above.
}
\end{emcppslisting}
    
\noindent Again, thanks to C++11's extended \lstinline!friend! syntax, we can defend
against these defects at compile time simply by making the CRTP base
class's default constructor \emph{private} and befriending its template
parameter:

\begin{emcppslisting}
template <typename D>
class E
{
     E() = default;
     friend D;
};
\end{emcppslisting}
    
\noindent Note that the goal here is not security but simply guarding against
accidental typos, copy-paste errors, and other occasional  human errors. By making this change, we will soon realize that there
is no \lstinline!operator<! defined for \lstinline!P!.

\subsubsection[Compile-time polymorphism using the curiously recurring template pattern]{Compile-time polymorphism using the curiously recurring template pattern}\label{compile-time-polymorphism-using-crtp}

Object-oriented programming provides certain flexibility that at times
might be supererogatory. Here we will exploit the familiar domain of
abstract/concrete shapes to demonstrate a mapping between runtime
polymorphism using virtual functions and compile-time polymorphism using
the CRTP. We begin with a simple abstract \lstinline!Shape! class that
implements a single, pure, virtual \lstinline!draw! function:

\begin{emcppslisting}[emcppsbatch=e6]
class Shape
{
public:
    virtual void draw() const = 0;  // abstract (ù{\codeincomments{draw}}ù) function (interface)
};
\end{emcppslisting}
    
\noindent From this abstract \lstinline!Shape! class, we now derive two concrete
shape types, \lstinline!Circle! and \mbox{\lstinline!Rectangle!}, each implementing
the \emph{abstract} \lstinline!draw! function:

\begin{emcppslisting}[emcppsbatch=e6]
#include <iostream>  // (ù{\codeincomments{std::cout}}ù)

class Circle : public Shape
{
    int d_radius;

public:
    Circle(int radius) : d_radius(radius) { }

    void draw() const  // concrete implementation of abstract (ù{\codeincomments{draw}}ù) function
    {
        std::cout << "Circle(radius = " << d_radius << ")\n";
    }
};

class Rectangle : public Shape
{
    int d_length;
    int d_width;

public:
    Rectangle(int length, int width) : d_length(length), d_width(width) { }

    void draw() const  // concrete implementation of abstract (ù{\codeincomments{draw}}ù) function
    {
        std::cout << "Rectangle(length = " << d_length << ", "
                                "width = " << d_width  << ")\n";
    }
};
\end{emcppslisting}
    
\noindent Notice that a \lstinline!Circle! is constructed with a single integer
argument, i.e., \lstinline!radius!, and a \lstinline!Rectangle! is constructed
with two integers, i.e., \lstinline!length! and \lstinline!width!.

We now implement a function that takes an arbitrary shape, via a
\lstinline!const! \romeovalue{lvalue} reference to its abstract base class, and
prints it:

\begin{emcppslisting}[emcppsbatch=e6]
void print(const Shape& shape)
{
    shape.draw();
}

void testShape()
{
    print(Circle(1));        // OK, prints: (ù{\codeincomments{Circle(radius = 1)}}ù)
    print(Rectangle(2, 3));  // OK, prints: (ù{\codeincomments{Rectangle(length = 2, width = 3)}}ù)
    print(Shape());          // Error, (ù{\codeincomments{Shape}}ù) is an abstract class.
}
\end{emcppslisting}
    
\noindent Now suppose that we didn't need all the runtime flexibility offered by
this system and wanted to map just what we have in the previous code
snippet onto templates that avoid the spatial and runtime overhead of
virtual-function tables and dynamic dispatch. Such transformation again
involves creating a CRTP base class, this time in lieu of our abstract
interface:

\begin{emcppslisting}[emcppsbatch=e7]
template <typename T>
struct Shape
{
    void draw() const
    {
        static_cast<const T*>(this)->draw();  // assumes (ù{\codeincomments{T}}ù) derives from (ù{\codeincomments{Shape}}ù)
    }
};
\end{emcppslisting}
    
\noindent Notice that we are using a \lstinline!static_cast! to the address of an
object of the \lstinline!const! template parameter type, \lstinline!T!,
assuming that the template argument is of the same type as some derived
class of this object's type. We now define our types as before, the only
difference being the form of the base type:

\begin{emcppshiddenlisting}[emcppsbatch=e7]
#include <iostream>  // (ù{\codeincomments{std::cout}}ù)

class Circle : public Shape<Circle>
{
    int d_radius;

public:
    Circle(int radius) : d_radius(radius) { }

    void draw() const  // concrete implementation of abstract (ù{\codeincomments{draw}}ù) function
    {
        std::cout << "Circle(radius = " << d_radius << ")\n";
    }
};

class Rectangle : public Shape<Rectangle>
{
    int d_length;
    int d_width;

public:
    Rectangle(int length, int width) : d_length(length), d_width(width) { }

    void draw() const  // concrete implementation of abstract (ù{\codeincomments{draw}}ù) function
    {
        std::cout << "Rectangle(length = " << d_length << ", "
                                "width = " << d_width  << ")\n";
    }
};
\end{emcppshiddenlisting}
\begin{emcppslisting}[emcppsbatch=e7,emcppsignore={expanded as hidden listing}]
class Circle : public Shape<Circle>
{
    // same as above
};

class Rectangle : public Shape<Rectangle>
{
    // same as above
};
\end{emcppslisting}
    
\noindent We now define our \lstinline!print! function, this time as a function
template taking a \lstinline!Shape! of arbitrary type \lstinline!T!:

\begin{emcppslisting}[emcppsbatch=e7]
template <typename T>
void print(const Shape<T>& shape)
{
    shape.draw();
}
\end{emcppslisting}
    
\noindent The result of compiling and running \lstinline!testShape! above is the
same, including that \lstinline!Shape()! doesn't compile.

However, opportunities for undetected failure remain. Suppose we decide
to add a third shape, \lstinline!Triangle!, constructed with three sides:

\begin{emcppslisting}[emcppsbatch=e7]
class Triangle : public Shape<Rectangle>  // Oops!
{
    int d_side1;
    int d_side2;
    int d_side3;

public:
    Triangle(int side1, int side2, int side3)
        : d_side1(side1), d_side2(side2), d_side3(side3) { }

    void draw() const
    {
        std::cout << "Triangle(side1 = " << d_side1 << ", "
                              "side2 = " << d_side2 << ", "
                              "side3 = " << d_side3 << ")\n";
    }
};
\end{emcppslisting}
    
\noindent Unfortunately, we forgot to change the base-class type parameter when we
copy-pasted from \lstinline!Rectangle!.

Let's now create a new test that exercises all three and see what
happens on our platform:

\begin{emcppslisting}[emcppsbatch=e7]
void test2()
{
    print(Circle(1));          // prints: (ù{\codeincomments{Circle(radius = 1)}}ù)
    print(Rectangle(2, 3));    // prints: (ù{\codeincomments{Rectangle(length = 2, width = 3)}}ù)
    print(Triangle(4, 5, 6));  // prints: (ù{\codeincomments{Rectangle(length = 4, width = 5)}}ù) ?!
    Shape<int> bug;            // Compiles?!
}
\end{emcppslisting}
    
\noindent As should by now be clear, a defect in our \lstinline!Triangle!
implementation results in \emph{hard} \emcppsgloss{undefined behavior} that
could have been prevented at compile time by using the extended
\lstinline!friend! syntax. Had we defined the CRTP base-class template's
default constructor to be \emph{private} and made its type parameter a
\lstinline!friend!, we could have prevented the copy-paste error with
\lstinline!Triangle! and suppressed the ability to create a \lstinline!Shape!
object without deriving from it (e.g., see \lstinline!bug! in the previous
code snippet):

\begin{emcppslisting}[emcppsbatch=e8]
template <typename T>
class Shape
{
    Shape() = default;  // Default the (ù{\codeincomments{default}}ù) constructor to be (ù{\codeincomments{private}}ù).
    friend T;           // Ensure only a type derived from (ù{\codeincomments{T}}ù) has access.
};
\end{emcppslisting}
    
\noindent Generally, whenever we are using the CRTP, making just the
default constructor of the base-class template \lstinline!private! and
having it befriend its type parameter is typically a trivial local
change, is helpful in avoiding various forms of accidental misuse and
is unlikely to induce long-distance friendships where none previously
existed: Applying extended \lstinline!friend! syntax to an existing CRTP
is typically \emph{safe}.

\subsubsection[Compile-time visitor using the curiously recurring template pattern]{Compile-time visitor using the curiously recurring template pattern}\label{compile-time-visitor-using-crtp}

As more real-world applications of compile-time polymorphism using the CRTP,
consider implementing traversal and visitation of complex data
structures. In particular, we want to facilitate employing
\emph{default-action} functions, which allow for simpler code from the
point of view of the programmer who needs the results of the traversal.
We illustrate our compile-time visitation approach using binary trees as
our data structure.

We begin with the traditional node structure of a binary tree, where
each node has a left and right subtree plus a label:

\begin{emcppslisting}[emcppsbatch=e9]
struct Node
{
    Node* d_left;
    Node* d_right;
    char  d_label;  // (ù{\codeincomments{label}}ù) will be used in the pre-order example.

    Node() : d_left(0), d_right(0), d_label(0) { }
};
\end{emcppslisting}
    
\noindent Now we wish to have code that traverses the tree in one of the three
traditional ways: \emph{pre-order}, \emph{in-order}, \emph{post-order}.
Such traversal code is often intertwined with the actions to be taken.
In our implementation, however, we will write a CRTP-like base-class
template, \lstinline!Traverser!, that implements empty stub functions for
each of the three traversal types, relying on the CRTP-derived
type to supply the desired functionality:

\begin{emcppslisting}[emcppsbatch=e9]
template <typename T>
class Traverser
{
private:
    Traverser() = default;  // Make the default constructor (ù{\codeincomments{private}}ù).
    friend T;               // Grant access only to the derived class.

public:
    void visitPreOrder(Node*)  { }  // stub-functions & placeholders
    void visitInOrder(Node*)   { }  // (Each of these three functions
    void visitPostOrder(Node*) { }  // defaults to an inline "no-op.")

    void traverse(Node* n)  // factored subfunctionality
    {
        T *t = static_cast<T*>(this);  // Cast (ù{\codeincomments{this}}ù) to the derived type.

        if (n) { t->visitPreOrder(n);     }  // optionally defined in derived
        if (n) { t->traverse(n->d_left);  }  //     "         "    "     "
        if (n) { t->visitInOrder(n);      }  //     "         "    "     "
        if (n) { t->traverse(n->d_right); }  //     "         "    "     "
        if (n) { t->visitPostOrder(n);    }  //     "         "    "     "
    }
};
\end{emcppslisting}
    
\noindent The factored traversal mechanism is implemented in the
\lstinline!Traverser! base-class template. A proper subset of the four
customization points, that is, the four member functions invoked from
the \emph{public} \lstinline!traverse! function of the \lstinline!Traverser!
base class, is implemented as appropriate in the derived class,
identified by \lstinline!T!. Each of these customization functions is
invoked in order. Notice that the \lstinline!traverse! function is safe to call on a
\lstinline!nullptr! as each individual customization-function invocation
will be independently bypassed if its supplied \lstinline!Node! pointer is
null. If a customization function is defined in the derived class, that
version of it is invoked; otherwise, the corresponding empty
\lstinline!inline! base-class version of that function is invoked instead.
This approach allows for any of the three traversal orders to be
implemented simply by supplying an appropriately configured derived type
where clients are obliged to implement only the portions they need. Even
the traversal itself can be modified, as we will soon see, where we
create the very data structure we're traversing.

Let's now look at how derived-class authors might use this pattern.
First, we'll write a traversal class that fully populates a tree to a
specified depth:

\begin{emcppslisting}[emcppsbatch=e9]
struct FillToDepth : Traverser<FillToDepth>
{
    using Base = Traverser<FillToDepth>;  // similar to a local (ù{\codeincomments{typedef}}ù)

    int d_depth;            //  final "height" of the tree
    int d_currentDepth;     //  current distance from the root

    FillToDepth(int depth) : d_depth(depth), d_currentDepth(0) { }

    void traverse(Node*& n)
    {
        if (d_currentDepth++ < d_depth && !n)  // descend; if not balanced...
        {
            n = new Node;     // Add (ù{\codeincomments{node}}ù) since it's not already there.
        }

        Base::traverse(n);    // Recurse by invoking the *base* version.

        --d_currentDepth;     // Ascend.
    }
};
\end{emcppslisting}
    
\noindent The derived class's version of the \lstinline!traverse! member function
acts as if it overrides the \lstinline!traverse! function in the base-class
template and then, as part of its re-implementation, defers to the base-class version to perform the actual traversal.

Importantly, note that we have re-implemented \lstinline!traverse! in the
derived class with a function by the same name but having a
\emph{different} \emcppsgloss{signature} that has more capability (i.e., it's able
to modify its immediate argument) than the one in the base-class
template. In practice, this signature modification is something we would
do rarely, but part of the flexibility of this design pattern, as with
templates in general, is that we can take advantage of \emcppsgloss{duck
typing} to achieve useful functionality in somewhat unusual ways. For
this pattern, the designers of the base-class template and the designers
of the derived classes are, at least initially, likely to be the same
people, and they will arrange for these sorts of signature variants to
work correctly if they need such functionality. Or they may decide that
overridden methods should follow a proper contract and signature that
they determine is appropriate, and they may declare improper overrides
to be undefined behavior. In this example, we aim for illustrative
flexibility over rigor.

\begin{emcppslisting}[emcppsbatch=e9]
void traverse(Node* n);   // as declared in the (ù{\codeincomments{Traverser}}ù) base-class template
void traverse(Node*& n);  // as declared in the (ù{\codeincomments{FillToDepth}}ù) derived class
\end{emcppslisting}
    
\noindent Unlike virtual functions, the signatures of corresponding functions in
the base and derived classes need not match exactly \emph{provided} the
derived-class function can be called in the same way as the
corresponding one in the base class. In this case, the compiler has all
the information it needs to make the call properly:

\begin{emcppslisting}[emcppsignore={illustrative only}]
static_cast<FillToDepth *>(this)->traverse(n);  // what the compiler sees
\end{emcppslisting}
    
\noindent Suppose that we now want to create a type that labels a \emph{small}
tree, balanced or not, according to its pre-order traversal:

\begin{emcppslisting}[emcppsbatch=e9]
struct PreOrderLabel : Traverser<PreOrderLabel>
{
    char d_label;

    PreOrderLabel() : d_label('a') { }

    void visitPreOrder(Node* n)  // This choice controls traversal order.
    {
        n->d_label = d_label++;   
            // Each successive label is sequential alphabetically.
    }
};
\end{emcppslisting}
    
\noindent The simple pre-order traversal class, \lstinline!PreOrderLabel!, labels the
nodes such that it visits each parent \emph{before} it visits either of
its two children.

Alternatively, we might want to create a read-only derived class,
\lstinline!InOrderPrint!, that simply prints out the sequence of labels
resulting from an \emph{in-order} traversal of the, e.g., previously
pre-ordered, labels:

\begin{emcppslisting}[emcppsbatch=e9]
#include <cstdio>  // (ù{\codeincomments{std::putchar}}ù)

struct InOrderPrint : Traverser<InOrderPrint>
{
    ~InOrderPrint()
    {
        std::putchar('\n');  // Print single newline at the end of the string.
    }

    void visitInOrder(const Node* n) const
    {
        std::putchar(n->d_label);  // Print the label character exactly as is.
    }
};
\end{emcppslisting}
    
\noindent The simple \lstinline!InOrderPrint!-derived class, shown in the example above, prints out the
labels of a tree \emph{in order}: left subtree, then node, then right
subtree. Notice that since we are only examining the tree here --- not
modifying it --- we can declare the overriding method to take a
\lstinline!const!~\lstinline!Node*! rather than a \lstinline!Node*! and make the
method itself \lstinline!const!. Once again, compatibility of signatures,
not identity, is the key.

Finally, we might want to clean up the tree. We do so in
\emph{post-order} since we do not want to delete a node before we have
cleaned up its children!

\begin{emcppslisting}[emcppsbatch=e9]
struct CleanUp : Traverser<CleanUp>
{
    void visitPostOrder(Node*& n)
    {
        delete n;  // always necessary
        n = 0;     // might be omitted in a "raw" version of the type
    }
};
\end{emcppslisting}
    
\noindent Putting it all together, we can create a \lstinline!main! program that
creates a balanced tree to a depth of four and then labels it in
\emph{pre-order}, prints those labels in \emph{in-order}, and destroys
it in \emph{post-order}:

\begin{emcppslisting}[emcppsbatch=e9]
int main()
{
    Node* n = 0;                  // tree handle

    FillToDepth(4).traverse(n);   // (1) Create balanced tree.
    PreOrderLabel().traverse(n);  // (2) Label tree in pre-order.
    InOrderPrint().traverse(n);   // (3) Print labels in order.
    CleanUp().traverse(n);        // (4) Destroy tree in post-order.
    return 0;
}
\end{emcppslisting}
    
\noindent Running this program results in a binary tree of height 4, as
illustrated in the code snippet below, and has reliably consistent output:

\begin{lstlisting}[style=plain]
dcebgfhakjlinmo
\end{lstlisting}
    
\begin{lstlisting}[style=plain]
Level 0:                                a
                              .    '         '    .
Level 1:                b '                           ' i
                   .  '    '  .                   .  '    '  .
Level 2:        c                f              j              m
              /   \            /   \          /   \          /   \
Level 3:    d       e        g       h      k       l      n       o
\end{lstlisting}
    
\noindent This use of the CRTP for traversal truly shines when the data structure to
be traversed is especially complex, such as an abstract-syntax-tree
(AST) representation of a computer program, where tree nodes have many
different types, with each type having custom ways of representing the
subtrees it contains. For example, a translation unit is a sequence of
declarations; a declaration can be a type, a variable, or a function;
functions have return types, parameters, and a compound statement; the
statement has substatements, expressions, and so on. We would not want
to rewrite the traversal code for each new application. Given a reusable
CRTP-based traverser for our AST, we don't have to.

For example, consider writing a type that visits each integer literal
node in a given AST:

\begin{emcppshiddenlisting}[emcppsbatch=e10]
template <typename T>
class AstTraverser {};
class IntegerLiteral {};
\end{emcppshiddenlisting}
\begin{emcppslisting}[emcppsbatch=e10]
struct IntegerLiteralHandler : AstTraverser<IntegerLiteralHandler>
{
    void visit(IntegerLiteral* iLit)
    {
        // ... (do something with this integer literal)
    }
};
\end{emcppslisting}
    
\noindent The AST traverser, which would implement a separate empty \lstinline!visit!
overload for each syntactic node type in the grammar, would invoke our
derived \lstinline!visit! member function with every integer literal in the
program, regardless of where it appeared. This CRTP-based traverser
would also call many other \lstinline!visit! methods, but each of those would perform no action at all by default and would likely be elided at even modest compiler-optimization levels. Be aware, however, that although we ourselves are
not rewriting the traversal code each time, the compiler is still doing
it because every CRTP instantiation produces a new copy of the traversal
code. If the traversal code is large and complex, the consequence might
be increased program size, that is, \emcppsgloss{code bloat}.

Finally, the CRTP can be used in a variety of situations for many
purposes,\footnote{\cite{fluentcpp17}} which explains both its \emph{curiously recurring} nature and nomenclature. Those uses invariably
benefit from (1) declaring the base-class template's default constructor
\emph{private} and (2) having that template befriend its type parameter,
which is possible only by means of the extended \lstinline!friend! syntax.
Thus, the CRTP base-class template can ensure, at compile time, that its
type argument is actually derived from the base class as required by the
pattern.




\newpage
%\section[{\tt inline} {\tt namespace}]{Transparently Nested Namespaces}\label{inline-namespaces}%\sectionmark{{\RHCode inline} namespaces}
% 14 Feb 2021, revisions in; ready for Josh's code fixes
% 16 Feb 2021 JMB; code compiles

\emcppsFeature{
    short={\lstinline!inline!~\lstinline!namespace!},
    tocshort={{\TOCCode inline}~{\TOCCode namespace}},
    long={Transparently Nested Namespaces},
    rhshort={{\RHCode inline}~{\RHCode namespace}},
}{inline-namespaces}
\setcounter{table}{0}
\setcounter{footnote}{0}
\setcounter{lstlisting}{0}
%\section[{\tt inline} {\tt namespace}]{Transparently Nested Namespaces}\label{inline-namespaces}%


An \lstinline!inline! namespace is a nested namespace whose member entities
closely behave as if they were declared directly within the enclosing
namespace.

\subsection[Description]{Description}\label{description-inlinenamespace}

To a first approximation, an \emcppsgloss[inline namespace]{\lstinline!inline!~\lstinline!namespace!}
(e.g., \lstinline!v2! in the code snippet below) acts a lot like a
conventional nested namespace (e.g., \lstinline!v1!) followed by a
\lstinline!using! directive for that namespace in its enclosing namespace{\cprotect\footnote{C++17 allows developers to concisely declare nested
namespaces with shorthand notation:
\begin{emcppslisting}[style=footcode]
namespace a::b { /* ... */ }
// is the same as
namespace a { namespace b { /* ... */ } }
\end{emcppslisting}
C++20 expands on the above syntax by allowing the insertion of the
\lstinline!inline! keyword in front of any of the namespaces, except the
first one:
\begin{emcppslisting}[style=footcode]
namespace a::inline b::inline c { /* ... */ }
// is the same as
namespace a { inline namespace b { inline namespace c { /* ... */ } } }

inline namespace a::b { }  // Error, cannot start with (ù{\codeincomments{inline}}ù) for compound namespace names
namespace inline a::b { }  // Error, (ù{\codeincomments{inline}}ù) at front of sequence explicitly disallowed
\end{emcppslisting}
      }}:

\begin{emcppslisting}
// example.cpp:
namespace n
{
    namespace v1  // conventional nested namespace followed by (ù{\codeincomments{using}}ù) directive
    {
        struct T { };     // nested type declaration (identified as (ù{\codeincomments{::n::v1::T}}ù))
        int d;            // (ù{\codeincomments{::n::v1::d}}ù) at, e.g., (ù{\codeincomments{0x01a64e90}}ù)
    }

    using namespace v1;   // import names (ù{\codeincomments{T}}ù) and (ù{\codeincomments{d}}ù) into (ù{\codeincomments{namespace n}}ù)
}

namespace n
{
    inline namespace v2   // similar to being followed by (ù{\codeincomments{using namespace v2}}ù)
    {
        struct T { };     // nested type declaration (identified as (ù{\codeincomments{::n::v2::T}}ù))
        int d;            // (ù{\codeincomments{::n::v2::d}}ù) at, e.g., (ù{\codeincomments{0x01a64e94}}ù)
    }

    // using namespace v2;  // redundant when used with an (ù{\codeincomments{inline namespace}}ù)
}
\end{emcppslisting}
    
 
%%%%% because the long list of code terms in the FN below cause a persistent end-of-line error, those long lists are being set as a command:
%\newcommand{inlinelonglistone}{\lstinline!namespace!~\lstinline!n!~\lstinline!{!~\lstinline!inline!~\lstinline!namespace~\lstinline!v!~\lstinline!{!~\lstinline!int!~\lstinline!d;!~\lstinline!}!~\lstinline!}!}
%\newcommand{inlinelonglisttwo}{\lstinline!namespace!~\lstinline!n!~\lstinline!{!~\lstinline!namespace}!~\lstinline!v!~\lstinline!{!~\lstinline!int!~\lstinline!d;!~\lstinline!}!~\lstinline!using!~\lstinline!namespace!~\lstinline!v;!~\lstinline!}! }
\noindent Four subtle details distinguish these approaches:
\begin{enumerate}
\item{Name collisions with existing names behave differently due to differing name-lookup rules.}
\item{\emcppsgloss[argument-dependent lookup (ADL)]{Argument-dependent lookup (ADL)} gives special treatment to \lstinline!inline! namespaces.}
\item{Template specializations can refer to the primary template in an \lstinline!inline! namespace even if written in the enclosing namespace.}
\item{Reopening namespaces might reopen an \lstinline!inline! namespace.}
\end{enumerate}

One important aspect that all forms of namespaces share, however, is
that (1) nested symbolic names (e.g., \lstinline!n::v1::T!) at the
\emcppsgloss{API} level, (2) \emcppsgloss{mangled names} (e.g.,
\lstinline!_ZN1n2v11dE!, \lstinline!_ZN1n2v21dE!), and (3) assigned
relocatable addresses (e.g., \lstinline!0x01a64e90!, \lstinline!0x01a64e94!)
at the \emcppsgloss{ABI} level remain unaffected by the use of either
\lstinline!inline! or \lstinline!using! or both. To be precise, source files containing, alternately, \lstinline!namespace!~\lstinline!n!~\lstinline!{!~\lstinline!inline!~\lstinline!namespace!~\lstinline!v!~\lstinline!{!~\lstinline!int!~\lstinline!d;!~\lstinline!}!~\lstinline!}! 
and 
\lstinline!namespace!~\lstinline!n!~\lstinline!{!~\lstinline!namespace!~\lstinline!v!~\lstinline!{!~\lstinline!int!~\lstinline!d;!~\lstinline!}!~\lstinline!using!~\lstinline!namespace!~\lstinline!v;!~\lstinline!}!, will produce identical assembly.{\cprotect\footnote{These mangled names can be seen with GCC by running \lstinline!g++!~\lstinline!-S!~\lstinline!<file>.cpp! and viewing the contents of the generated \lstinline!<file>.s!. Note that Compiler Explorer is another valuable tool for learning about what comes out the other end of a C++ compiler: see https://godbolt.org/.}} Note that a \lstinline!using! directive immediately following an \lstinline!inline!
namespace is superfluous; name lookup will always consider names in
\lstinline!inline! namespaces before those imported by a \lstinline!using!
directive. Such a directive can, however, be used to import the contents
of an \lstinline!inline! namespace to some other namespace, albeit only in
the conventional, \emcppsgloss[using directive]{\lstinline!using! directive} sense; see \intraref{annoyances-inlinenamespace}{only-one-namespace-can-contain-any-given-inline-namespace}. 

More generally, each namespace has what is called its
\emph{\lstinline!inline! namespace set}, which is the transitive closure of
all \lstinline!inline! namespaces within the namespace. All names in the
\lstinline!inline! namespace set are roughly intended to behave as if they
are defined in the enclosing namespace. Conversely, each \lstinline!inline!
namespace has an \emph{enclosing namespace set} that comprises all
enclosing namespaces up to and including the first non-\lstinline!inline!
namespace.

\subsubsection[Loss of access to duplicate names in enclosing namespace]{Loss of access to duplicate names in enclosing namespace}\label{loss-of-access-to-duplicate-names-in-enclosing-namespace}

When both a type and a variable are declared with the same name in the
same scope, the variable name hides the type name --- such behavior can
be demonstrated by using the form of \lstinline!sizeof! that accepts a
nonparenthesized \emph{expression} (recall that the form of \lstinline!sizeof! that accepts a \emph{type} as its argument requires parentheses):

\begin{emcppslisting}
struct A { double d; };  static_assert(sizeof(  A) == 8, "");  // type
                      // static_assert(sizeof   A  == 8, "");  // Error

int A;                   static_assert(sizeof(  A) == 4, "");  // data
                         static_assert(sizeof   A  == 4, "");  // OK
\end{emcppslisting}
    
\noindent Unless both type and variable entities are declared within the same
scope, no preference is given to variable names; the name of an entity
in an inner scope hides a like-named entity in an enclosing scope:

\begin{emcppslisting}
void f()
{
    double B;                 static_assert(sizeof(B) == 8, "");  // variable
    {                         static_assert(sizeof(B) == 8, "");  // variable
        struct B { int d; };  static_assert(sizeof(B) == 4, "");  // type
    }                         static_assert(sizeof(B) == 8, "");  // variable
}
\end{emcppslisting}
    
\noindent When an entity is declared in an enclosing \lstinline!namespace! and
another entity having the same name hides it in a \emph{lexically}
nested scope, then (apart from \lstinline!inline! namespaces) access to a
hidden element can generally be recovered by using scope resolution:

\begin{emcppslisting}
struct C { double d; };  static_assert(sizeof(  C) == 8, "");

void g()
{                        static_assert(sizeof(  C) == 8, "");  // type
    int C;               static_assert(sizeof(  C) == 4, "");  // variable
                         static_assert(sizeof(::C) == 8, "");  // type
}                        static_assert(sizeof(  C) == 8, "");  // type
\end{emcppslisting}
    
\noindent A conventional nested namespace behaves as one might expect:

\begin{emcppslisting}
namespace outer
{
    struct D { double d; }; static_assert(sizeof(       D) == 8, ""); // type

    namespace inner
    {                       static_assert(sizeof(       D) == 8, ""); // type
        int D;              static_assert(sizeof(       D) == 4, ""); // var
    }                       static_assert(sizeof(       D) == 8, ""); // type
                            static_assert(sizeof(inner::D) == 4, ""); // var
                            static_assert(sizeof(outer::D) == 8, ""); // type
    using namespace inner;//static_assert(sizeof(       D) == 0, ""); // Error
                            static_assert(sizeof(inner::D) == 4, ""); // var
                            static_assert(sizeof(outer::D) == 8, ""); // type
}                           static_assert(sizeof(outer::D) == 8, ""); // type
\end{emcppslisting}
    
\noindent In the example above, the inner variable name, \lstinline!D!, hides the
outer type with the same name, starting from the point of \lstinline!D!'s
declaration in \lstinline!inner! until \lstinline!inner! is closed, after
which the unqualified name \lstinline!D! reverts to the type in the
\lstinline!outer! namespace. Then, right after the subsequent
\lstinline!using!~\lstinline!namespace!~\lstinline!inner;! directive, the meaning
of the unqualified name \lstinline!D! in \lstinline!outer! becomes ambiguous,
shown here with a \lstinline!static_assert! that is commented out; any
attempt to refer to an unqualified \lstinline!D! from here to the end of
the scope of \lstinline!outer! will fail to compile. The type entity
declared as \lstinline!D! in the \lstinline!outer! namespace can, however,
still be accessed --- from inside or outside of the \lstinline!outer!
namespace, as shown in the example --- via its qualified name,
\lstinline!outer::D!.

If an \lstinline!inline! namespace were used instead of a nested namespace
followed by a \mbox{\lstinline!using!} directive, however, the ability to recover
by name the hidden entity in the enclosing namespace is lost.
Unqualified name lookup considers the inline namespace set and the used
namespace set simultaneously. Qualified name lookup first considers the
\lstinline!inline! namespace set and \emph{then} goes on to look into used
namespaces. This means we can still refer to \lstinline!outer::D! in the
example above, but doing so would still be ambiguous if \lstinline!inner!
were an inline namespace. This subtle difference in behavior is a
byproduct of the highly specific use case that motivated this feature
and for which it was explicitly designed; see \intraref{use-cases-inlinenamespace}{link-safe-abi-versioning}. 
%\textit{\titleref{use-cases-inlinenamespace}: \titleref{link-safe-abi-versioning}} on page~\pageref{link-safe-abi-versioning}.

\subsubsection[Argument-dependent–lookup interoperability across \lstinline!inline! \lstinline!namespace! boundaries]{Argument-dependent–lookup interoperability across\\[0.5ex] {\SubsubsecCode inline} {\SubsubsecCode namespace} boundaries}\label{argument-dependent–lookup-interoperability-across-inline-namespace-boundaries}

Another important aspect of \lstinline!inline! namespaces is that they
allow \emcppsgloss[argument-dependent lookup (ADL)]{ADL} to work seamlessly across \lstinline!inline! namespace
boundaries. Whenever unqualified function names are being resolved, a
list of \emph{associated namespaces} is built for each argument of the
function. This list of associated namespaces comprises the namespace of the
argument, its enclosing namespace set, plus the \lstinline!inline!
namespace set.

Consider the case of a type, \lstinline!U!, defined in an \lstinline!outer!
namespace, and a function, \lstinline!f(U)!, declared in an \lstinline!inner!
namespace nested within \lstinline!outer!. A second type, \lstinline!V!, is
defined in the \lstinline!inner! namespace, and a function, \lstinline!g!, is
declared, after the close of \lstinline!inner!, in the \lstinline!outer!
namespace:

\begin{emcppslisting}
namespace outer
{
    struct U { };

    // inline               // Uncommenting this line fixes the problem.
    namespace inner
    {
         void f(U) { }
         struct V { };
    }

    using namespace inner;  // If we inline (ù{\codeincomments{inner}}ù), we don't need this line.

    void g(V) { }
}

void client()
{
    f(outer::U());         // Error, (ù{\codeincomments{f}}ù) is not declared in this scope.
    g(outer::inner::V());  // Error, (ù{\codeincomments{g}}ù) is not declared in this scope.
}
\end{emcppslisting}
    
\noindent In the example above, a \lstinline!client! invoking \lstinline!f! with an
object of type \lstinline!outer::U! fails to compile because
\lstinline!f(outer::U)! is declared in the nested \lstinline!inner! namespace,
which is not the same as declaring it in \lstinline!outer!. Because
\emcppsgloss[argument-dependent lookup (ADL)]{ADL} does not look into namespaces added with the \lstinline!using!
directive, \emcppsgloss[argument-dependent lookup (ADL)]{ADL} does not find the needed
\lstinline!outer::inner::f! function. Similarly, the type \lstinline!V!,
defined in namespace \lstinline!outer::inner!, is not declared in the same
namespace as the function \lstinline!g! that operates on it. Hence, when
\lstinline!g! is invoked from within \lstinline!client! on an object of type
\lstinline!outer::inner::V!, \emcppsgloss[argument-dependent lookup (ADL)]{ADL} again does not find the needed
function \lstinline!outer::g(outer::V)!.

Simply making the \lstinline!inner! namespace \lstinline!inline! solves both
of these \emcppsgloss[argument-dependent lookup (ADL)]{ADL}-related problems. All transitively nested
\lstinline!inline! namespaces --- up to and including the most proximate
non-\lstinline!inline! enclosing namespace --- are treated as one with
respect to \emcppsgloss[argument-dependent lookup (ADL)]{ADL}.

\subsubsection[The ability to specialize templates declared in a nested \lstinline!inline! namespace]{The ability to specialize templates declared in a nested {\SubsubsecCode inline} namespace}\label{the-ability-to-specialize-templates-declared-in-a-nested-inline-namespace}

The third property that distinguishes \lstinline!inline! namespaces from
conventional ones, even when followed by a \lstinline!using! directive, is
the ability to specialize a class template defined within an
\lstinline!inline! namespace from within an enclosing one; this ability
holds transitively up to and including the most proximate
non-\lstinline!inline! namespace:

\begin{emcppslisting}[emcppserrorlines={24}]
namespace out                      // proximate non-(ù{\codeincomments{inline}}ù) outer namespace
{
    inline namespace in1           // first-level nested (ù{\codeincomments{inline}}ù) namespace
    {
        inline namespace in2       // second-level nested (ù{\codeincomments{inline}}ù) namespace
        {
            template <typename T>  // primary class template general definition
            struct S { };

            template <>            // class template *full* specialization
            struct S<char> { };
        }

        template <>                // class template *full* specialization
        struct S<short> { };
    }

    template <>                    // class template *full* specialization
    struct S<int> { };
}

using namespace out;               // conventional using directive

template <>
struct S<int> { };                 // Error, cannot specialize from this scope
\end{emcppslisting}
    
\noindent Note that the conventional nested namespace \lstinline!out! followed by a
\lstinline!using! directive in the enclosing namespace does not admit
specialization from that outermost namespace, whereas all of the
\lstinline!inline! namespaces do. Function templates behave similarly
except that --- unlike class templates, whose definitions must reside
entirely within the namespace in which they are declared --- a function
template can be \emph{declared} within a nested namespace and then be
\emph{defined} from anywhere via a \emcppsgloss{qualified name}:

\begin{emcppslisting}
namespace out                      // proximate non-(ù{\codeincomments{inline}}ù) outer namespace
{
    inline namespace in1           // first-level nested (ù{\codeincomments{inline}}ù) namespace
    {
        template <typename T>      // function template declaration
        void f();

        template <>                // function template (full) specialization
        void f<short>() { }
    }

    template <>                    // function template (full) specialization
    void f<int>() { }
}

template <typename T>              // function template general definition
void out::in1::f() { }
\end{emcppslisting}
    
\noindent An important takeaway from the examples above is that every template
entity --- be it class or function --- \emph{must} be declared
in \emph{exactly} one place within the collection of namespaces that
comprise the \lstinline!inline! namespace set. In particular, declaring a
class template in a nested \lstinline!inline! namespace and then
subsequently defining it in a containing namespace is not possible
because, unlike a function definition, a type definition cannot be
placed into a namespace via name qualification alone:

\begin{emcppslisting}[emcppserrorlines={17}]
namespace outer
{
    inline namespace inner
    {
        template <typename T>      // class template declaration
        struct Z;                  // (if defined, must be within same namespace)

        template <>                // class template full specialization
        struct Z<float> { };
    }

    template <typename T>          // inconsistent declaration (and definition)
    struct Z { };                  // (ù{\codeincomments{Z}}ù) is now ambiguous in namespace (ù{\codeincomments{outer}}ù).

    const int i = sizeof(Z<int>);  // Error, Reference to (ù{\codeincomments{Z}}ù) is ambiguous.

    template <>                    // attempted class template full specialization
    struct Z<double> { };          // Error, (ù{\codeincomments{outer::Z}}ù) or (ù{\codeincomments{outer::inner::Z}}ù)?
}
\end{emcppslisting}
    

\subsubsection[Reopening namespaces can reopen nested \lstinline!inline! ones]{Reopening namespaces can reopen nested {\SubsubsecCode inline} ones}\label{reopening-namespaces-can-reopen-nested-inline-ones}

Another subtlety specific to \lstinline!inline! namespaces is related to
reopening namespaces. Consider a namespace \lstinline!outer! that declares
a nested namespace \lstinline!outer::m! and an \lstinline!inline!
namespace \lstinline!inner! that, in turn, declares a nested namespace
\lstinline!outer:inner::m!. In this case, subsequent attempts to reopen
namespace \lstinline!m! cause an ambiguity error:

\begin{emcppshiddenlisting}[emcppsbatch=e1]
#include <type_traits>  // (ù{\codeincomments{std::is\_same}}ù)
\end{emcppshiddenlisting}
\begin{emcppslisting}[emcppsbatch=e1,emcppserrorlines={17,18,19}]
namespace outer
{
    namespace m { }      // opens and closes (ù{\codeincomments{::outer::m}}ù)

    inline namespace inner
    {
        namespace n { }  // opens and closes (ù{\codeincomments{::outer::inner::n}}ù)
        namespace m { }  // opens and closes (ù{\codeincomments{::outer::inner::m}}ù)
    }

    namespace n          // OK, reopens (ù{\codeincomments{::outer::inner::n}}ù)
    {
        struct S { };    // defines (ù{\codeincomments{::outer::inner::n::S}}ù)
    }

    namespace m          // Error, (ù{\codeincomments{namespace m}}ù) is ambiguous.
    {
        struct T { };    // with clang defines (ù{\codeincomments{::outer::m::T}}ù)
    }
}

static_assert(std::is_same<outer::n::S, outer::inner::n::S>::value, "");
\end{emcppslisting}
    
\noindent In the code snippet above, no issue occurs with reopening
\lstinline!outer::inner::n! and no issue would have occurred with reopening
\lstinline!outer::m! but for the \lstinline!inner! namespaces having been
declared \lstinline!inline!. When a new namespace declaration is
encountered, a lookup determines if a matching namespace having that
name appears anywhere in the \emph{\lstinline!inline! namespace set} of the
current namespace. If the namespace is ambiguous, as is the case with
\lstinline!m! in the example above, one can get the surprising error shown.{\cprotect\footnote{Note that reopening already declared
namespaces, such as \lstinline!m! and \lstinline!n! in the \lstinline!inner!
and \lstinline!outer! example, is handled incorrectly on several
popular platforms. Clang, for example, will perform a name lookup when
encountering a new namespace declaration and give preference to the
outermost namespace found, causing the last declaration of \lstinline!m!
to reopen \lstinline!::outer::m! instead of being ambiguous. GCC, prior
to version~8.1, will not perform name lookup and will place \emph{any}
nested namespace declarations directly within their enclosing
namespace. This compiler defect causes the last declaration of
\lstinline!m! to reopen \lstinline!::outer::m! instead of
\lstinline!::outer::inner::m! and the last declaration of \lstinline!n! to
open a new namespace, \lstinline!::outer::n!, instead of reopening
  \lstinline!::outer::inner::n!.}} If a matching namespace is found
unambiguously inside an \lstinline!inline! namespace, \lstinline!n! in this
case, then it is that nested namespace that is reopened --- here,
\lstinline!::outer::inner::n!. The inner namespace is reopened even though
the last declaration of \lstinline!n! is not lexically scoped within
\lstinline!inner!. Notice that the definition of \lstinline!S! is perhaps
surprisingly defining \lstinline!::outer::inner::n::S!, not
\lstinline!::outer::n::S!. For more on what is \emph{not} supported by this
feature, see \intraref{annoyances-inlinenamespace}{inability-to-redeclare-across-namespaces-impedes-code-factoring}. 
%\textit{\titleref{annoyances-inlinenamespace}: \titleref{inability-to-redeclare-across-namespaces-impedes-code-factoring}} on page~\pageref{inability-to-redeclare-across-namespaces-impedes-code-factoring}.

\subsection[Use Cases]{Use Cases}\label{use-cases-inlinenamespace}

\subsubsection[Facilitating API migration]{Facilitating API migration}\label{facilitating-api-migration}

Getting a large codebase to \emph{promptly} upgrade to a new version of
a library in any sort of timely fashion can be challenging. As a
simplistic illustration, imagine that we have just developed a new
library, \lstinline!parselib!, comprising a class template, \lstinline!Parser!,
and a function template, \lstinline!analyze!, that takes a \lstinline!Parser!
object as its only argument:

\begin{emcppslisting}[emcppsbatch=e2]
namespace parselib
{
    template <typename T>
    class Parser
    {
        // ...

    public:
        Parser();
        int parse(T* result, const char* input);
            // Load (ù{\codeincomments{result}}ù) from null-terminated (ù{\codeincomments{input}}ù); return (ù{\codeincomments{0}}ù) (on
            // success) or nonzero (with no effect on (ù{\codeincomments{result}}ù)).
    };

    template <typename T>
    double analyze(const Parser<T>& parser);
}
\end{emcppslisting}
    
\noindent To use our library, clients will need to specialize our \lstinline!Parser!
class directly within the \lstinline!parselib! namespace:

\begin{emcppslisting}[emcppsbatch=e2]
struct MyClass { /*...*/ };  // end-user-defined type

namespace parselib  // necessary to specialize (ù{\codeincomments{Parser}}ù)
{
    template <>            // Create *full* specialization of class
    class Parser<MyClass>  // (ù{\codeincomments{Parser}}ù) for user-type (ù{\codeincomments{MyClass}}ù).
    {
        // ...

    public:
        Parser();
        int parse(MyClass* result, const char* input);
            // The *contract* for a specialization typically remains the same.
    };

    double analyze(const Parser<MyClass>& parser);
};
\end{emcppslisting}
    
\noindent Typical \lstinline!client! code will also look for the \lstinline!Parser!
class directly within the \lstinline!parselib! namespace:

\begin{emcppslisting}[emcppsbatch=e2]
void client()
{
    MyClass result;
    parselib::Parser<MyClass> parser;

    int status = parser.parse(&result, "...( MyClass value )...");
    if (status != 0)
    {
        return;
    }

    double value = analyze(parser);
    // ...
}
\end{emcppslisting}
    
\noindent Note that invoking \lstinline!analyze! on objects of some instantiated type
of the \lstinline!Parser! class template will rely on \emcppsgloss[argument-dependent lookup (ADL)]{ADL} to find
the corresponding overload.

We anticipate that our library's API will evolve over time, so we want to
enhance the design of \lstinline!parselib! accordingly. One of our goals is
to somehow encourage clients to move essentially all at once, yet also to
accommodate both the early adopters and the inevitable stragglers that
make up a typical adoption curve. Our approach will be to create, within
our outer \lstinline!parselib! namespace, a nested \lstinline!inline!
namespace, \lstinline!v1!, which will hold the current implementation of
our library software:

\begin{emcppslisting}[emcppsbatch=e3]
namespace parselib
{
    inline namespace v1             // Note our use of (ù{\codeincomments{inline}}ù) namespace here.
    {
        template <typename T>
        class Parser
        {
            // ...

        public:
            Parser();
            int parse(T* result, const char* input);
                // Load (ù{\codeincomments{result}}ù) from null-terminated (ù{\codeincomments{input}}ù); return 0 (on
                // success) or nonzero (with no effect on (ù{\codeincomments{result}}ù)).
        };

        template <typename T>
        double analyze(const Parser<T>& parser);
    }
}
\end{emcppslisting}
    
\noindent As suggested by the name \lstinline!v1!, this namespace serves primarily as
a mechanism to support library evolution through \emcppsgloss{API} and
\emcppsgloss{ABI} versioning (see \intraref{use-cases-inlinenamespace}{link-safe-abi-versioning} 
%\textit{\titleref{use-cases-inlinenamespace}: \titleref{link-safe-abi-versioning}} on page~\pageref{link-safe-abi-versioning} 
and \intraref{use-cases-inlinenamespace}{build-modes-and-abi-link-safety} ). 
%\textit{\titleref{use-cases-inlinenamespace}: \titleref{build-modes-and-abi-link-safety}} on page~\pageref{build-modes-and-abi-link-safety}). 
The need to specialize \lstinline!class!~\lstinline!Parser! and,
independently, the reliance on ADL to find the free function template
\lstinline!analyze! require the use of \lstinline!inline! namespaces, as
opposed to a conventional namespace followed by a \lstinline!using!
directive.

Note that, whenever a subsystem starts out directly in a first-level
namespace and is subsequently moved to a second-level nested namespace
for the purpose of versioning, declaring the inner namespace
\lstinline!inline! is the most reliable way to avoid inadvertently
destabilizing existing clients; see also \intraref{potential-pitfalls-inlinenamespace}{enabling-selective-using-directives-for-short-named-entities}.  
%\textit{\titleref{potential-pitfalls-inlinenamespace}: \titleref{enabling-selective-using-directives-for-short-named-entities}} on page~\pageref{enabling-selective-using-directives-for-short-named-entities}.

Now suppose we decide to enhance \lstinline!parselib! in a
non--backwards-compatible manner, such that the signature of
\lstinline!parse! takes a second argument \lstinline!size! of type
\lstinline!std::size_t! to allow parsing of non--null-terminated strings
and to reduce the risk of buffer overruns. Instead of unilaterally
removing all support for the previous version in the new release, we can
create a second namespace, \lstinline!v2!, containing the new
implementation and then, at some point, make \lstinline!v2! the
\lstinline!inline! namespace instead of \lstinline!v1!:

\begin{emcppslisting}
#include <cstddef>  // (ù{\codeincomments{std::size\_t}}ù)

namespace parselib
{
    namespace v1  // Notice that (ù{\codeincomments{v1}}ù) is now just a nested namespace.
    {
        template <typename T>
        class Parser
        {
            // ...

        public:
            Parser();
            int parse(T* result, const char* input);
                // Load (ù{\codeincomments{result}}ù) from null-terminated (ù{\codeincomments{input}}ù); return 0 (on
                // success) or nonzero (with no effect on (ù{\codeincomments{result}}ù)).
        };

        template <typename T>
        double analyze(const Parser<T>& parser);
    }

    inline namespace v2    // Notice that use of (ù{\codeincomments{inline}}ù) keyword has moved here.
    {
        template <typename T>
        class Parser
        {
            // ...

        public:  // note incompatible change to (ù{\codeincomments{Parser}}ù)'s essential API
            Parser();
            int parse(T* result, const char* input, std::size_t size);
                // Load (ù{\codeincomments{result}}ù) from (ù{\codeincomments{input}}ù) of specified (ù{\codeincomments{size}}ù); return 0
                // on success) or nonzero (with no effect on (ù{\codeincomments{result}}ù)).
        };

        template <typename T>
        double analyze(const Parser<T>& parser);
    }
}
\end{emcppslisting}
    
\noindent When we release this new version with \lstinline!v2! made \lstinline!inline!,
all existing clients that rely on the version supported directly in
\lstinline!parselib! will, by design, break when they recompile. At
that point, each client will have two options. The first one is
to upgrade the code immediately by passing in the size of the input
string (e.g., \lstinline!23!) along with the address of its first
character:

\begin{emcppslisting}[emcppserrorlines={4}]
void client()
{
    // ...
    int status = parser.parse(&result, "...( MyClass value )...", 23);
    // ...                                                      ^^^^ Look here!
}
\end{emcppslisting}
    
\noindent The second option is to change all references to
\lstinline!parselib! to refer to the original version in \lstinline!v1!
explicitly:

\begin{emcppshiddenlisting}[emcppsbatch=e3]
struct MyClass { /*...*/ };  // end-user-defined type
\end{emcppshiddenlisting}
\begin{emcppslisting}[emcppsbatch=e3]
namespace parselib
{
    namespace v1  // specializations moved to nested namespace
    {
        template <>
        class Parser<MyClass>
        {
            // ...

        public:
            Parser();
            int parse(MyClass* result, const char* input);
        };

        double analyze(const Parser<MyClass>& parser);
    }
};

void client1()
{
    MyClass result;
    parselib::v1::Parser<MyClass> parser;  // reference nested namespace (ù{\codeincomments{v1}}ù)

    int status = parser.parse(&result, "...( MyClass value )...");
    if (status != 0)
    {
        return;
    }

    double value = analyze(parser);
    // ...
}
\end{emcppslisting}
    
\noindent Providing the updated version in a new \lstinline!inline! namespace
\lstinline!v2! provides a more flexible migration path --- especially for a
large population of independent client programs --- compared to manual
targeted changes in client code.

Although new users would pick up the latest version automatically either
way, existing users of \lstinline!parselib! will have the option of
converting immediately by making a few small syntactic changes or
opting to remain with the original version for a while longer by making
all references to the library namespace refer explicitly to the desired
version. If the library is released before the \lstinline!inline! keyword
is moved, early adopters will have the option of opting in by referring
to \lstinline!v2! explicitly until it becomes the default. Those who have
no need for enhancements can achieve stability by referring to a
particular version in perpetuity or until it is physically removed from
the library source.

Although this same functionality can sometimes be realized without using \lstinline!inline! namespaces (i.e., by adding a
\lstinline!using!~\lstinline!namespace! directive at the end of the
\lstinline!parselib! namespace), any benefit of ADL and the ability to
specialize templates from within the enclosing \lstinline!parselib!
namespace itself would be lost. Note that, because
specialization doesn't kick in until overload resolution is completed,
specializing overloaded functions is dubious at best; see \intraref{potential-pitfalls-inlinenamespace}{specializing-templates-in-std-can-be-problematic}.
%\textit{\titleref{potential-pitfalls-inlinenamespace}: \titleref{specializing-templates-in-std-can-be-problematic}} on page~\pageref{specializing-templates-in-std-can-be-problematic}.}}

Providing separate namespaces for each successive version has an
additional advantage in an entirely separate dimension: avoiding inadvertent, difficult-to-diagnose, latent linkage defects. Though not
demonstrated by this specific example,
cases do arise where simply changing which of the version namespaces is
declared \lstinline!inline! might lead to an \emcppsgloss[ill formed, no diagnostic required (IFNDR)]{ill formed, no-diagnostic
required (IFNDR)} program. This might happen when one or more of
its translation units that use the library are not recompiled before the
program is relinked to the new static or dynamic library containing the
updated version of the library software; see \intraref{use-cases-inlinenamespace}{link-safe-abi-versioning}. 
%\textit{\titleref{use-cases-inlinenamespace}: \titleref{link-safe-abi-versioning}} on page~\pageref{link-safe-abi-versioning}.

For distinct
nested namespaces to guard effectively against accidental link-time
errors, the symbols involved have to (1) reside in object code (e.g.,
a \emcppsgloss{header-only library} would fail this requirement) and (2)
have the same \emcppsgloss{name mangling} (i.e., linker symbol) in both
versions. In this particular instance, however, the signature of the
\lstinline!parse! member function of \lstinline!parser! did change, and its
mangled name will consequently change as well; hence the same
  \lstinline!undefined!~\lstinline!symbol! link error would result either way.

\subsubsection[Link-safe ABI versioning]{Link-safe ABI versioning}\label{link-safe-abi-versioning}

\lstinline!inline! namespaces are not intended as a mechanism for
source-code versioning; instead, they prevent programs from being
\emcppsgloss[ill formed]{ill formed} due to linking some version of a library with client
code compiled using some other, typically older version of the same
library. Below, we present two examples: a simple pedagogical example
to illustrate the principle followed by a more real-world example.
Suppose we have a library component \lstinline!my_thing! that implements
an example type, \lstinline!Thing!, which wraps an \lstinline!int! and
initializes it with some value in its default constructor defined
out-of-line in the \lstinline!cpp! file:

\begin{emcppslisting}
struct Thing  // version 1 of class (ù{\codeincomments{Thing}}ù)
{
    int i;    // integer data member (size is 4)
    Thing();  // original non-(ù{\codeincomments{inline}}ù) constructor (defined in (ù{\codeincomments{.cpp}}ù) file)
};
\end{emcppslisting}
    
\noindent Compiling a source file with this version of the header included might produce an object file that can be incompatible yet linkable with an object file resulting from compiling a different source file with a different version of this header included:

\begin{emcppslisting}
struct Thing   // version 2 of class (ù{\codeincomments{Thing}}ù)
{
    double d;  // double-precision floating-point data member (size is 8)
    Thing();   // updated non-(ù{\codeincomments{inline}}ù) constructor (defined in (ù{\codeincomments{.cpp}}ù) file)
};
\end{emcppslisting}
    
\noindent To make the problem that we are illustrating concrete, let's represent
the client as a \lstinline!main! program that does nothing but create a
\lstinline!Thing! and print the value of its only data member, \lstinline!i!.

\begin{emcppslisting}[emcppsbatch=e4]
// main.cpp:
#include <my_thing.h>  // (ù{\codeincomments{my::Thing}}ù) (version 1)
#include <iostream>    // (ù{\codeincomments{std::cout}}ù)

int main()
{
     my::Thing t;
     std::cout << t.i << '\n';
}
\end{emcppslisting}
    
\noindent If we compile this program, a reference to a locally undefined linker
symbol, such as
\lstinline!_ZN2my7impl_v15ThingC1Ev!,{\cprotect\footnote{On a Unix
machine, typing \lstinline!nm!~\lstinline!main.o! reveals the symbols used
in the specified object file. A symbol prefaced with a capital
\lstinline!U! represents an undefined symbol that must be resolved by the
linker. Note that the linker symbol shown here incorporates an
intervening \lstinline!inline! namespace, \lstinline!impl_v1!, as will be
  explained shortly.}} which represents the \lstinline!my::Thing::Thing!
constructor, will be generated in the \lstinline!main.o! file:

\begin{lstlisting}[language=bash,style=plain]
$ g++ -c main.cpp
\end{lstlisting}
    
\noindent Without explicit intervention, the spelling of this linker symbol would
be unaffected by any subsequent changes made to the implementation of
\lstinline!my::Thing!, such as its data members or implementation of its
default constructor, even after recompiling. The same, of course,
applies to its definition in a separate translation unit.

We now turn to the translation unit implementing type
\lstinline!my::Thing!. The \lstinline!my_thing! \emcppsgloss{component} consists
of a \lstinline!.h!/\lstinline!.cpp! pair: \lstinline!my_thing.h! and
\lstinline!my_thing.cpp!. The header file \lstinline!my_thing.h! provides
the physical interface, such as the definition of the principal type,
\lstinline!Thing!, its member and associated free function declarations,
plus definitions for inline functions and function templates, if any:

\begin{emcppslisting}[emcppsbatch=e4]
// my_thing.h:
#ifndef INCLUDED_MY_THING
#define INCLUDED_MY_THING

namespace my                  // outer namespace (used directly by clients)
{
    inline namespace impl_v1  // inner namespace (for implementer use only)
    {
        struct Thing
        {
            int i;    // original data member, size = 4
            Thing();  // default constructor (defined in (ù{\codeincomments{my\_thing.cpp}}ù))
        };
    };
}

#endif
\end{emcppslisting}
    
\noindent The implementation file \lstinline!my_thing.cpp! contains all of the
non-\lstinline!inline! function bodies that will be translated separately
into the \lstinline!my_thing.o! file:

\begin{emcppslisting}[emcppsbatch=e4]
// my_thing.cpp:
#include <my_thing.h>

namespace my                   // outer namespace (used directly by clients)
{
    inline namespace impl_v1   // inner namespace (for implementer use only)
    {
        Thing::Thing() : i(0)  // load a 4-byte value into (ù{\codeincomments{Thing}}ù)'s data member
        {
        }
    }
}
\end{emcppslisting}
    
\noindent Observing common good practice, we include the header file of the component
as the first substantive line of code to ensure that --- irrespective of
anything else --- the header always compiles in isolation, thereby
avoiding insidious include-order dependencies.{\cprotect\footnote{See
  \cite{lakos20}, section~1.6.1, ``Component Property 1," pp.~210--212.}} When we compile the source file \lstinline!my_thing.cpp!,
we produce an object file \lstinline!my_thing.o! containing the definition
of the very same linker symbol, such as
\lstinline!_ZN2my7impl_v15ThingC1Ev!, for the default constructor of
\lstinline!my::Thing! needed by the client:

\begin{lstlisting}[language=bash,style=plain]
$ g++ -c my_thing.cpp
\end{lstlisting}
    
\noindent We can then link \lstinline!main.o! and \lstinline!my_thing.o! into an
executable and run it:

\begin{lstlisting}[language=bash,style=plain]
$ g++ -o prog main.o my_thing.o
$ ./prog

0
\end{lstlisting}
    
\noindent Now, suppose we were to change the definition of \lstinline!my::Thing! to hold a \lstinline!double! instead of an \lstinline!int!, recompile
\lstinline!my_thing.cpp!, and then relink with the original
\lstinline!main.o! without recompiling \lstinline!main.cpp! first. None of the
relevant linker symbols would change, and the code would recompile and link
just fine, but the resulting binary \lstinline!prog! would be
\emcppsgloss[ill formed, no diagnostic required (IFNDR)]{IFNDR}: the client would be trying to print a 4-byte,
\lstinline!int! data member, \lstinline!i!, in \lstinline!main.o! that was loaded
by the library component as an 8-byte, \lstinline!double! into \lstinline!d!
in \lstinline!my_thing.o!. We can resolve this problem by changing --- or,
if we didn't think of it in advance, by adding --- a new \lstinline!inline!
namespace and making that change there:

\begin{emcppshiddenlisting}[emcppsbatch=e5]
// my_thing.h:
#ifndef INCLUDED_MY_THING
#define INCLUDED_MY_THING

namespace my                  // outer namespace (used directly by clients)
{
    inline namespace impl_v2  // inner namespace (for implementer use only)
    {
        struct Thing
        {
            double d; // new data member, size = 8
            Thing();  // default constructor (defined in (ù{\codeincomments{my\_thing.cpp}}ù))
        };
    };
}

#endif
\end{emcppshiddenlisting}
\begin{emcppslisting}[emcppsbatch=e5]
// my_thing.cpp:
#include <my_thing.h>

namespace my                     // outer namespace (used directly by clients)
{
    inline namespace impl_v2     // inner namespace (for implementer use only)
    {
        Thing::Thing() : d(0.0)  // load 8-byte value into (ù{\codeincomments{Thing}}ù)'s data member
        {
        }
    }
}
\end{emcppslisting}
    
\noindent Now clients that attempt to link against the new library will not find
the linker symbol, such as \lstinline!_Z...impl_v1...v!, and the link
stage will fail. Once clients recompile, however, the undefined linker
symbol will match the one available in the new \lstinline!my_thing.o!,
such as \lstinline!_Z...impl_v2...v!, the link stage will succeed, and
the program will again work as expected. What's more, we have the option
of keeping the original implementation. In that case, existing clients
that have not as yet recompiled will continue to link against the old
version until it is eventually removed after some suitable deprecation
period.

As a more realistic second example of using \lstinline!inline! namespaces
to guard against linking incompatible versions, suppose we have two
versions of a \lstinline!Key! class in a security library in the enclosing
namespace, \lstinline!auth! --- the original version in a regular nested
namespace \lstinline!v1!, and the new current version in an \lstinline!inline!
nested namespace \lstinline!v2!:

\begin{emcppslisting}
#include <cstdint>  // (ù{\codeincomments{std::uint32\_t}}ù), (ù{\codeincomments{std::unit64\_t}}ù)

namespace auth      // outer namespace (used directly by clients)
{
    namespace v1    // inner namespace (optionally used by clients)
    {
        class Key
        {
        private:
            std::uint32_t d_key;
                // (ù{\codeincomments{sizeof(Key)}}ù) is 4 bytes

        public:
            std::uint32_t key() const;  // stable interface function

            // ...
        };
    }

    inline namespace v2    // inner namespace (default current version)
    {
        class Key
        {
        private:
            std::uint64_t d_securityHash;
            std::uint32_t d_key;
                // (ù{\codeincomments{sizeof(Key)}}ù) is 16 bytes

        public:
            std::uint32_t key() const;  // stable interface function

            // ...
        };
    }
}
\end{emcppslisting}
    
\noindent Attempting to link together older binary artifacts built against version
1 with binary artifacts built against version 2 will result in a
link-time error rather than allowing an \emcppsgloss[ill formed]{ill formed} program to
be created. Note, however, that this approach works only if functionality essential
to typical use is defined out of line in a \lstinline!.cpp! file. For example, it would add absolutely no value for libraries that are
shipped entirely as header files, since the versioning offered here
occurs strictly at the binary level (i.e., between object files)
during the link stage.

\subsubsection[Build modes and ABI link safety]{Build modes and ABI link safety}\label{build-modes-and-abi-link-safety}

In certain scenarios, a class might have two different memory layouts
depending on compilation flags. For instance, consider a low-level
\lstinline!ManualBuffer! class template in which an additional data member
is added for debugging purposes:

\begin{emcppslisting}
template <typename T>
struct ManualBuffer
{
private:
    alignas(T) char d_data[sizeof(T)];  // aligned and big enough to hold a (ù{\codeincomments{T}}ù)

#ifndef NDEBUG
    bool d_engaged;  // tracks whether buffer is full (debug builds only)
#endif

public:
    void construct(const T& obj);
        // Emplace (ù{\codeincomments{obj}}ù). (Engage the buffer.) The behavior is undefined unless
        // the buffer was not previously engaged.

    void destroy();
        // Destroy the current (ù{\codeincomments{obj}}ù). (Disengage the buffer.) The behavior is
        // undefined unless the buffer was previously engaged.

    // ...
};
\end{emcppslisting}
    
\noindent Note that we have
  employed the C++11 \lstinline!alignas! attribute (see \featureref{\locationc}{alignas})   
 % ``\titleref{alignas}" on page~\pageref{alignas}) 
  here
  because it is exactly what's needed for this usage example.

The \lstinline!d_engaged! flag in the example above serves as a way to detect misuse of
the \mbox{\lstinline!ManualBuffer!} class but only in debug builds. The extra
space and run time required to maintain this Boolean flag is undesirable
in a release build because \mbox{\lstinline!ManualBuffer!} is intended to be an
efficient, lightweight abstraction over the direct use of
\emcppsgloss[placement new]{placement \lstinline!new!} and explicit destruction.

The linker symbol names generated for the methods of
\lstinline!ManualBuffer! are the same irrespective of the chosen build
mode. If the same program links together two object files where
\lstinline!ManualBuffer! is used --- one built in debug mode and one built
in release mode --- the \emcppsgloss[one-definition rule (ODR)]{one-definition rule (ODR)} will be violated,
and the program will again be \emcppsgloss[ill formed, no diagnostic required (IFNDR)]{IFNDR}.

Prior
to \lstinline!inline! namespaces, it was possible to control the
ABI-level name of linked symbols by creating separate template
instantiations on a per-build-mode basis:

\begin{emcppslisting}[]
#ifndef NDEBUG
enum { is_debug_build = 1 };
#else
enum { is_debug_build = 0 };
#endif

template <typename T, bool Debug = is_debug_build>
struct ManualBuffer { /* ... */ };
\end{emcppslisting}
    
\noindent While the code above changes the interface of \lstinline!ManualBuffer! to
accept an additional template parameter, it also allows debug and
release versions of the same class to coexist in the same program,
  which might prove useful, e.g., for testing.
  
Another way of avoiding incompatibilities at link time is to
introduce two \lstinline!inline! namespaces, the entire purpose of which is
to change the ABI-level names of the linker symbols associated with
\lstinline!ManualBuffer! depending on the build mode:

\begin{emcppslisting}
#ifndef NDEBUG            // perhaps a BAD IDEA
inline namespace release
#else
inline namespace debug
#endif
{
    template <typename T>
    struct ManualBuffer
    {
        // ... (same as above)
    };
}
\end{emcppslisting}
    
\noindent The approach demonstrated in this example tries to ensure that a linker
error will occur if any attempt is made to link objects built with a
build mode different from that of \mbox{\lstinline!manualbuffer.o!}. Tying it to
the \lstinline!NDEBUG! flag, however, might have unintended consequences;
we might introduce unwanted restrictions in what we call
\emcppsgloss{mixed-mode builds}. Most modern platforms support the notion of
linking a collection of object files irrespective of their
optimization levels. The same is certainly true for whether or not
C-style \lstinline!assert! is enabled. In other words, we may want to have
a mixed-mode build where we link object files that differ in their optimization and assertion options, as long as they are binary compatible --- i.e., in this case, they all must be uniform with respect to the implementation of \lstinline!ManualBuffer!.  Hence, a more general, albeit more complicated and
manual, approach would be to tie the non-interoperable behavior
associated with this ``safe'' or ``defensive'' build mode to a different
switch entirely. Another consideration would be to avoid ever inlining a
namespace into the global namespace since no method is available to
recover a symbol when there is a collision:

\begin{emcppslisting}
namespace buflib  // GOOD IDEA: enclosing namespace for nested (ù{\codeincomments{inline}}ù) namespace
{
#ifdef SAFE_MODE  // GOOD IDEA: separate control of non-interoperable versions
    inline namespace safe_build_mode
#else
    inline namespace normal_build_mode
#endif
    {
        template <typename T>
        struct ManualBuffer
        {
        private:
            alignas(T) char d_data[sizeof(T)];  // aligned/sized to hold a (ù{\codeincomments{T}}ù)

#ifdef SAFE_MODE
            bool d_engaged;  // tracks whether buffer is full (safe mode only)
#endif

        public:
            void construct(const T& obj);  // sets (ù{\codeincomments{d\_engaged}}ù) (safe mode only)
            void destroy();                // sets (ù{\codeincomments{d\_engaged}}ù) (safe mode only)
            // ...
        };
    }
}
\end{emcppslisting}
    
\noindent And, of course, the appropriate conditional compilation within the
function bodies would need to be in the corresponding \lstinline!.cpp!
file.

Finally, if we have two implementations of a particular entity that are
sufficiently distinct, we might choose to represent them in their
entirety, controlled by their own bespoke conditional-compilation
switches, as illustrated here using the \lstinline!my::VersionedThing! type
(see \intraref{use-cases-inlinenamespace}{link-safe-abi-versioning}): 
% \textit{\titleref{use-cases-inlinenamespace}: \titleref{link-safe-abi-versioning}} on page~\pageref{link-safe-abi-versioning}:

\begin{emcppshiddenlisting}[emcppsbatch=e7]
// my_versionedthing.h:
#define MY_THING_VERSION_1
\end{emcppshiddenlisting}
\begin{emcppslisting}[emcppsbatch=e7]
// my_versionedthing.h:
#ifndef INCLUDED_MY_VERSIONEDTHING
#define INCLUDED_MY_VERSIONEDTHING

namespace my
{
#ifdef MY_THING_VERSION_1  // bespoke switch for this component version
    inline
#endif
    namespace v1
    {
        struct VersionedThing
        {
            int d_i;
            VersionedThing();
        };
    }

#ifdef MY_THING_VERSION_2  // bespoke switch for this component version
    inline
#endif
    namespace v2
    {
        struct VersionedThing
        {
            double d_i;
            VersionedThing();
        };
    }
}
#endif
\end{emcppslisting}
    
\noindent However, see \intraref{potential-pitfalls-inlinenamespace}{inline-namespace-based-versioning-doesn’t-scale}. 
%\textit{\titleref{potential-pitfalls-inlinenamespace}: \titleref{inline-namespace-based-versioning-doesn’t-scale}} on page~\pageref{inline-namespace-based-versioning-doesn’t-scale}.

\subsubsection[Enabling selective \lstinline!using! directives for short-named entities]{Enabling selective {\SubsubsecCode using} directives for short-named entities}\label{enabling-selective-using-directives-for-short-named-entities}

Introducing a large number of small names into client code that doesn't
follow rigorous nomenclature can be problematic. Hoisting these names
into one or more nested namespaces so that they are easier to identify
as a unit and can be used more selectively by clients, such as through
explicit qualification or using directives, can sometimes be an
effective way of organizing shared codebases. For example,
\lstinline!std::literals! and its nested namespaces, such as
\lstinline!chrono_literals!, were introduced as \lstinline!inline! namespaces
in C++14. As it turns out, clients of these nested namespaces have no
need to specialize any templates defined in these namespaces nor do they
define types that must be found through \emcppsgloss[argument-dependent lookup (ADL)]{ADL}, but one can at
least imagine special circumstances in which such tiny-named entities
are either templates that require specialization or operator-like
functions, such as \lstinline!swap!, defined for local types within those
nested namespaces. In those cases, \lstinline!inline! namespaces would be
required to preserve the desired ``as if'' properties.

Even without either of these two needs, another property of an
\lstinline!inline! namespace differentiates it from a non-\lstinline!inline!
one followed by a \lstinline!using! directive. Recall from\intraref{description-inlinenamespace}{loss-of-access-to-duplicate-names-in-enclosing-namespace}  
%\textit{\titleref{description-inlinenamespace}: \titleref{loss-of-access-to-duplicate-names-in-enclosing-namespace}} on page~\pageref{loss-of-access-to-duplicate-names-in-enclosing-namespace} 
that a name in an outer namespace will
hide a duplicate name imported via a \lstinline!using! directive, whereas
any access to that duplicate name within the enclosing namespace would
be ambiguous when that symbol is installed by way of an \lstinline!inline!
namespace. To see why this more forceful clobbering behavior might be
preferred over hiding, suppose we have a communal namespace \lstinline!abc!
that is shared across multiple disparate headers. The first header,
\lstinline!abc_header1.h!, represents a collection of logically related
small functions declared directly in \lstinline!abc!:

\begin{emcppslisting}[emcppsbatch=e6]
// abc_header1.h:
namespace abc
{
    int i();
    int am();
    int smart();
}
\end{emcppslisting}
    
\noindent A second header, \lstinline!abc_header2.h!, creates a suite of many
functions having tiny function names. In a perhaps misguided effort to
avoid clobbering other symbols within the \lstinline!abc! namespace having
the same name, all of these tiny functions are sequestered within a
\lstinline!nested! namespace:

\begin{emcppslisting}[emcppsbatch=e6]
// abc_header2.h:
namespace abc
{
    namespace nested  // Should this instead have been an (ù{\codeincomments{inline}}ù) namespace?
    {
        int a();  // lots of functions with tiny names
        int b();
        int c();
        // ...
        int h();
        int i();  // might collide with another name declared in (ù{\codeincomments{abc}}ù)
        // ...
        int z();
    }

    using namespace nested;  // becomes superfluous if (ù{\codeincomments{nested}}ù) is made (ù{\codeincomments{inline}}ù)
}
\end{emcppslisting}
    
\noindent Now suppose that a client application includes
both of these headers to accomplish some task:

\begin{emcppslisting}[emcppsbatch=e6]
// client.cpp:
#include <abc_header1.h>
#include <abc_header2.h>

int function()
{
    if (abc::smart() < 0) { return -1; }  // uses (ù{\codeincomments{smart()}}ù) from (ù{\codeincomments{abc\_header1.h}}ù)
    return abc::z() + abc::i() + abc::a() + abc::h() + abc::c();  // Oops!
        // Bug, silently uses the (ù{\codeincomments{abc::i()}}ù) defined in (ù{\codeincomments{abc\_header1.h}}ù)
}
\end{emcppslisting}
    
\noindent In trying to cede control to the client as to whether the declared or
imported \lstinline!abc::i()! function is to be used, we have, in effect,
invited the defect illustrated in the above example whereby the client
was expecting the \lstinline!abc::i()! from \lstinline!abc_header2.h! and yet
picked up the one from \lstinline!abc_header1.h! by default. Had the
\lstinline!nested! namespace in \lstinline!abc_header2.h! been declared
\lstinline!inline!, the qualified name \lstinline!abc::i()! would have
automatically been rendered \emph{ambiguous} in namespace \lstinline!abc!,
the translation would have failed \emph{safely}, and the defect would
have been exposed at compile time. The downside, however, is that no
method would be available to recover nominal access to the
\lstinline!abc::i()! defined in \lstinline!abc_header1.h! once
\lstinline!abc_header2.h! is included, even though the two functions
(e.g., including their \emcppsgloss{mangled names} at the ABI level) remain
distinct.

\subsection[Potential Pitfalls]{Potential Pitfalls}\label{potential-pitfalls-inlinenamespace}

\subsubsection[\lstinline!nline!-namespace-based versioning doesn’t scale]{{\SubsubsecCode inline}-namespace-based versioning doesn’t scale}\label{inline-namespace-based-versioning-doesn’t-scale}

The problem with using \lstinline!inline! namespaces for ABI link safety is
that the protection they offer is only partial; in a few major places,
critical problems can linger until run time instead of being caught at
compile time.

Controlling which namespace is \lstinline!inline! using macros, such as was
done in the\linebreak[4] \lstinline!my::VersionedThing! example in \intraref{use-cases-inlinenamespace}{link-safe-abi-versioning},  
%\textit{\titleref{use-cases-inlinenamespace}: \titleref{link-safe-abi-versioning}} on page~\pageref{link-safe-abi-versioning}, 
will result in code that
directly uses the unversioned name, \lstinline!my::VersionedThing! being
bound directly to the versioned name \lstinline!my::v1::VersionedThing! or
\lstinline!my::v2::VersionedThing!, along with the class layout of that
particular entity. Sometimes details of using the \lstinline!inline!
namespace member are not resolved by the linker, such as the object
layout when we use types from that namespace as member variables in
other objects:

\begin{emcppslisting}[emcppsbatch=e7]
// my_thingaggregate.h:

// ...
#include <my_versionedthing.h>
// ...

namespace my
{
    struct ThingAggregate
    {
        // ...
        VersionedThing d_thing;
        // ...
    };
}
\end{emcppslisting}
    
\noindent This new \lstinline!ThingAggregate! type does not have the versioned
\lstinline!inline! namespace as part of its mangled name; it does, however,
have a completely different layout if built with
\lstinline!MY_THING_VERSION_1! defined
versus \lstinline!MY_THING_VERSION_2! defined. Linking a program with mixed
versions of these flags will result in runtime failures that are
decidedly difficult to diagnose.

This same sort of problem will arise for functions taking arguments of
such types; calling a function from code that is wrong about the layout
of a particular type will result in stack corruption and other undefined
and unpredictable behavior. This macro-induced problem will also arise in cases where an old object
file is linked against new code that changes which namespace is
\lstinline!inline!d but still provides the definitions for the old version
namespace. The old object file for the client can still link, but new
object files using the headers for the old objects might attempt to
manipulate those objects using the new namespace.

The only viable workaround for this approach is to propagate the
\lstinline!inline! namespace hierarchy through the entire software stack.
Every object or function that uses\linebreak[4] \lstinline!my::VersionedThing! needs to
also be in a namespace that differs based on the same control macro. In
the case of \lstinline!ThingAggregate!, one could just use the same
\lstinline!my::v1! and \lstinline!my::v2! namespaces, but higher-level
libraries would need their own \lstinline!my!-specific nested namespaces.
Even worse, for higher-level libraries, every lower-level library having
a versioning scheme of this nature would need to be considered,
resulting in having to provide the full cross-product of nested
namespaces to get link-time protection against mixed-mode builds.

This need for layers above a library to be aware of and to integrate
into their own structure the same namespaces the library has removes all
or most of the benefits of using \lstinline!inline! namespaces for
versioning. For an authentic real-world case study of heroic industrial use --- and
eventual disuse --- of \lstinline!inline!-namespaces for versioning, see \intrarefsimple{appendix:-case-study-of-using-inline-namespaces-for-versioning}. 
% \textit{\titleref{appendix:-case-study-of-using-inline-namespaces-for-versioning}} on page~\pageref{appendix:-case-study-of-using-inline-namespaces-for-versioning}.

\subsubsection[Relying on \lstinline!inline! namespaces to solve library evolution]{Relying on {\SubsubsecCode inline} namespaces to solve library evolution}\label{specializing-templates-in-std-can-be-problematic}

%Inline namespaces might be perceived as a complete solution for the owner of a library to evolve its API. Such, however, is not the case. As an example, consider the case of the C++ Standard Library, which does not use \lstinline!inline! namespaces for versioning. Instead, to allow for its necessary evolution, the Standard
%Library --- the \lstinline!std! namespace, in particular --- carries
%certain special restrictions on what one can do with it that are
%enforced by deeming certain constructs \romeogloss{ill formed} or
%engendering \romeogloss{undefined behavior}.

Inline namespaces might be misperceived as a complete solution for the owner of a library to evolve its API. As an especially relevant example, consider the C++ Standard Library, which itself does not use inline namespaces for versioning. Instead, to allow for its anticipated essential evolution, the Standard Library imposes certain special restrictions on what is permitted to occur within its own \lstinline!std! namespace by dint of deeming certain problematic uses as either \emcppsgloss{ill formed} or otherwise engendering \emcppsgloss{undefined behavior}.

Since C++11, several restrictions related to the Standard Library were
put in place:
\begin{itemize}
\item{Users may not add any new declarations within namespace \lstinline!std!. This means that users cannot add new \emph{functions}, \emph{overloads}, \emph{types}, or \emph{templates} to \lstinline!std!. This restriction gives the Standard Library freedom to add new \emph{names} in future versions of the Standard.}
\item{Users may not specialize member functions, member function templates, or member class templates. Specializing any of those entities might significantly inhibit a Standard Library vendor’s ability to maintain its otherwise encapsulated implementation details.}
\item{Users may add specializations of top-level Standard Library templates only if the declaration depends on the name of a nonstandard user-defined type and only if that user-defined type meets all requirements of the original template. Specialization of function templates is allowed but generally discouraged because this practice doesn’t scale since function templates cannot be partially specialized. Specializing of standard class templates when the specialization names a nonstandard user-defined type, such as \lstinline!std::vector<MyType*>!, is allowed but also problematic when not explicitly supported. While certain specific types, such as \lstinline!std::hash!, are designed for user specialization, steering clear of the practice for any other type helps to avoid surprises.}
\end{itemize}

Several other good practices facilitate smooth evolution for the
Standard Library{\cprotect\footnote{These restrictions are normative in
C++20, having finally formalized what were long identified as best
practices. Though these restrictions might not be codified in the
Standard for pre-C++20 software, they have been recognized best
practices for as long as the Standard Library has existed and
adherence to them will materially improve the ability of software to
migrate to future language standards irrespective of what version of
  the language standard is being targeted.}}:
\begin{itemize}
\item{Avoid specializing variable templates, even if dependent on user-defined types, except for those variable templates where specialization is explicitly allowed.\cprotect\footnote{C++20 limits the specialization of variable templates to only those instances where specialization is explicitly allowed and does so only for the mathematical constants in \lstinline!<numbers>!.}}
\item{Other than a few very specific exceptions, avoiding the forming of pointers to Standard Library functions — either explicitly or implicitly — allows the library to add overloads, either as part of the Standard or as an implementation detail for a particular Standard Library, without breaking user code.\cprotect\footnote{C++20 identifies these functions as \lstinline!addressable! and gives that property to only \lstinline!iostream! manipulators since those are the only functions in the Standard Library for which taking their address is part of normal usage.}}
\item{Overloads of Standard Library functions that depend on user-defined types are permitted, but, as with specializing Standard Library templates, users must still meet the requirements of the Standard Library function. Some functions, such as \lstinline!std::swap!, are designed to be customization points via overloading, but leaving functions not specifically designed for this purpose to vendor implementations only helps to avoid surprises.}
\end{itemize}

Finally, upon reading about this \lstinline!inline! namespace feature, one
might think that all names in namespace \lstinline!std! could be made
available at a global scope simply by inserting an\linebreak[4]%%%%%%
\lstinline!inline!~\lstinline!namespace!~\lstinline!std!~\lstinline!{}! before
including any standard headers. This practice is, however, explicitly
called out as ill-formed within the C++11 Standard. Although not uniformly diagnosed as an error by
all compilers, attempting this forbidden practice is apt to lead to
surprising problems even if not diagnosed as an error immediately.

\subsubsection[Inconsistent use of \lstinline!inline! keyword is ill formed, no diagnostic required]{Inconsistent use of {\SubsubsecCode inline} keyword is ill formed, no diagnostic required}\label{inconsistent-use-of-inline-keyword-is-ifndr}

It is an \emcppsgloss[one-definition rule (ODR)]{ODR} violation, \emcppsgloss[ill formed, no diagnostic required (IFNDR)]{IFNDR}, for a nested namespace
to be \lstinline!inline! in one translation unit and non-\lstinline!inline! in
another. And yet, the motivating use case of this feature relies on the
linker to actively complain whenever different, incompatible versions
--- nested within different, possibly \lstinline!inline!-inconsistent,
namespaces of an ABI --- are used within a single executable. Because
declaring a nested namespace \lstinline!inline! does not, by design, affect
linker-level symbols, developers must take appropriate care, such as
effective use of header files, to defend against such preventable
inconsistencies.

\subsection[Annoyances]{Annoyances}\label{annoyances-inlinenamespace}

\subsubsection[Inability to redeclare across namespaces impedes code factoring]{Inability to redeclare across namespaces impedes code factoring}\label{inability-to-redeclare-across-namespaces-impedes-code-factoring}

An essential feature of an \lstinline!inline! namespace is the ability to
declare a template within a nested \lstinline!inline! namespace and then
specialize it within its enclosing namespace. For example, we can
declare
\begin{itemize}
\item{a type template, \lstinline!S0!}
\item{a couple of function templates, \lstinline!f0! and \lstinline!g0!}
\item{and a member function template \lstinline!h0!, which is similar to \lstinline!f0!}
\end{itemize}
in an \lstinline!inline! namespace, \lstinline!inner!, and specialize each of
them, such as for \lstinline!int!, in the enclosing namespace,
\lstinline!outer!:

\begin{emcppslisting}
namespace outer                                          // enclosing namespace
{
    inline namespace inner                               // nested namespace
    {
        template<typename T> struct S0;                  // declarations of
        template<typename T> void f0();                  // various class
        template<typename T> void g0(T v);               // and function
        struct A0 { template <typename T> void h0(); };  // templates
    }

    template<> struct S0<int> { };                       // specializations
    template<> void f0<int>() { }                        // of the various
    void g0(int) { }  /* overload not specialization */  // class and function
    template<> void A0::h0<int>() { }                    // declarations above
}                                                        // in (ù{\codeincomments{outer}}ù) namespace
\end{emcppslisting}
    
\noindent Note that, in the case of \lstinline!g0! in this example, the
``specialization'' \lstinline!void!~\lstinline!g0(int)! is a non-template
\emph{overload} of the function template \lstinline!g0! rather than a
specialization of it. We \emph{cannot}, however,
portably{\cprotect\footnote{GCC provides the \lstinline!-fpermissive! flag,
which allows the example containing specializations within the inner
namespace to compile with warnings. Note again that \lstinline!g1(int)!,
being an \emph{overload} and not a \emph{specialization}, wasn't an
  error and, therefore, isn't a warning either.}} declare these
templates within the \lstinline!outer! namespace and then specialize them
within the \lstinline!inner! one, even though the \lstinline!inner! namespace
is \lstinline!inline!:

\begin{emcppslisting}
namespace outer                                     // enclosing namespace
{
    template<typename T> struct S1;                 // class template
    template<typename T> void f1();                 // function template
    template<typename T> void g1(T v);              // function template

    struct A1 { template <typename T> void h1(); }; // member function template

    inline namespace inner                          // nested namespace
    {                                               // BAD IDEA
        template<> struct S1<int> { };              // Error, (ù{\codeincomments{S1}}ù) not a template
        template<> void f1<int>() { }               // Error, (ù{\codeincomments{f1}}ù) not a template
        void g1(int) { }                            // OK, overloaded function
        template<> void A1::h1<int>() { }           // Error, (ù{\codeincomments{h1}}ù) not a template
    }
}
\end{emcppslisting}
    
\noindent Attempting to declare a template in the \lstinline!outer! namespace and
then define, effectively redeclaring, it in an \lstinline!inline! inner one
causes the name to be inaccessible within the \lstinline!outer! namespace:

\begin{emcppslisting}[emcppserrorlines=13]
namespace outer                                          // enclosing namespace
{                                                        // BAD IDEA
    template<typename T> struct S2;                      // declarations of
    template<typename T> void f2();                      // various class and
    template<typename T> void g2(T v);                   // function templates
    
    inline namespace inner                               // nested namespace
    {
        template<typename T> struct S2 { };              // definitions of
        template<typename T> void f2() { }               // unrelated class and
        template<typename T> void g2(T v) { }            // function templates
    }

    template<> struct S2<int> { };     // Error, (ù{\codeincomments{S2}}ù) is ambiguous in (ù{\codeincomments{outer}}ù).
    template<> void f2<int>() { }      // Error, (ù{\codeincomments{f2}}ù) is ambiguous in (ù{\codeincomments{outer}}ù).
    void g2(int) { }                   // OK, (ù{\codeincomments{g2}}ù) is an overload definition.
}
\end{emcppslisting}
    
\noindent Finally, declaring a template in the nested \lstinline!inline! namespace
\lstinline!inner! in the example above and then subsequently defining it in
the enclosing \lstinline!outer! namespace has the same effect of making
declared symbols ambiguous in the \lstinline!outer! namespace:

\begin{emcppslisting}[emcppsbatch=e8]
namespace outer                                          // enclosing namespace
{                                                        // BAD IDEA
    inline namespace inner                               // nested namespace
    {
        template<typename T> struct S3;                  // declarations of
        template<typename T> void f3();                  // various class
        template<typename T> void g3(T v);               // and function
        struct A3 { template <typename T> void h3(); };  // templates
    }

    template<typename T> struct S3 { };                  // definitions of
    template<typename T> void f3() { }                   // unrelated class
    template<typename T> void g3(T v) { }                // and function
    template<typename T> void A3::h3() { };              // templates

    template<> struct S3<int> { };     // Error, (ù{\codeincomments{S3}}ù) is ambiguous in (ù{\codeincomments{outer}}ù).
    template<> void f3<int>() { }      // Error, (ù{\codeincomments{f3}}ù) is ambiguous in (ù{\codeincomments{outer}}ù).
    void g3(int) { }                   // OK, (ù{\codeincomments{g3}}ù) is an *overload* definition.
    template<> void A3::h3<int>() { }  // Error, (ù{\codeincomments{h2}}ù) is ambiguous in (ù{\codeincomments{outer}}ù).
}
\end{emcppslisting}
    
\noindent Note that, although the definition for a member function template must
be located directly within the namespace in which it is declared, a
class or function template, once declared, may instead be defined in a different
scope by using an appropriate name qualification:

\begin{emcppshiddenlisting}[emcppsbatch=e9]
namespace outer
{
    inline namespace inner
    {
        template<typename T> struct S3;
        template<typename T> void f3();
        template<typename T> void g3(T v);
        struct A3 { template <typename T> void h3(); };
    }
}
\end{emcppshiddenlisting}
\begin{emcppslisting}[emcppsbatch=e9]
template <typename T> struct outer::S3 { };        // OK, enclosing namespace
template <typename T> void outer::inner::f3() { }  // OK, nested namespace
template <typename T> void outer::g3(T v) { }      // OK, enclosing namespace
template <typename T> void outer::A3::h3<T>() { }  // Error, ill-formed

namespace outer
{
    inline namespace inner
    {
        template <typename T> void A3::h3() { }    // OK, within same namespace
    }
}
\end{emcppslisting}
    
\noindent Also note that, as ever, the corresponding definition of the declared
template must have been seen before it can be used in a context
requiring a complete type. The importance of ensuring that all
specializations of a template have been seen before it is used
substantively (i.e., \emcppsgloss[one-definition rule (ODR)]{ODR}-used) cannot be overstated, giving
rise to the only limerick, which is actually part of the normative text,
in the C++ Language Standard{\cprotect\footnote{See \cite{iso11},
  section~14.7.3.7, pp.~375--375, specifically p.~376.}}:
\begin{quote}
When writing a specialization,\\
be careful about its location;\\
or to make it compile\\
will be such a trial\\
as to kindle its self-immolation.
\end{quote}

\subsubsection[Only one namespace can contain any given \lstinline!inline! namespace]{Only one namespace can contain any given {\SubsubsecCode inline} namespace}\label{only-one-namespace-can-contain-any-given-inline-namespace}

Unlike conventional \lstinline!using! directives, which can be used to
generate arbitrary many-to-many relationships between different
namespaces, \lstinline!inline! namespaces can be used only to contribute
names to the sequence of enclosing namespaces up to the first
non-\lstinline!inline! one. In cases in which the names from a namespace
are desired in multiple other namespaces, the classical \lstinline!using!
directive must be used, with the subtle differences between the two
modes properly addressed.

As an example, the C++14 Standard Library provides a hierarchy of nested
\lstinline!inline! namespaces for literals of different sorts within
namespace\linebreak[4] \lstinline!std!: \lstinline!std::literals::complex_literals!,
\lstinline!std::literals::chrono_literals!,\linebreak[4]
\lstinline!std::literals::string_literals!, and
\lstinline!std::literals::string_view_literals!.\linebreak[4] 
These namespaces can
be imported to a local scope in one shot via a
\lstinline!using!~\lstinline!std::literals! or instead, more selectively, by
\lstinline!using! the nested namespaces directly. This separation of the
types used with user-defined literals, which are all in namespace
\lstinline!std!, from the user-defined literals that can be used to create
those types led to some frustration; those who had a
\lstinline!using!~\lstinline!namespace!~\lstinline!std;! could reasonably have
expected to get the user-defined literals associated with their
\lstinline!std! types. However, the types in the nested namespace
\lstinline!std::chrono! did \emph{not} meet this
expectation.{\cprotect\footnote{\cite{hinnant17}}}

Eventually \emph{both} solutions for incorporating literal namespaces,
\lstinline!inline! from\linebreak[4] \lstinline!std::literals! and non-\lstinline!inline! from
\lstinline!std::chrono!, were pressed into service when, in C++17, a
\mbox{\lstinline!using!~\lstinline!namespace!~\lstinline!literals::chrono_literals;!}
was added to the\linebreak[4]
 \lstinline!std::chrono! namespace. The Standard does not, however, benefit in any objective way from any of
these namespaces being \lstinline!inline! since the artifacts in the
\lstinline!literals! namespace neither depend on ADL nor are templates in
need of user-defined specializations; hence, having all
non-\lstinline!inline! namespaces with appropriate \lstinline!using!
declarations would have been functionally indistinguishable from the
bifurcated approach taken.

\subsection[See Also]{See Also}\label{see-also}

\begin{itemize}
\item{%``\titleref{alignas}" on page~\pageref{alignas} — 
\seealsoref{alignas}{\seealsolocationc}provides properly aligned storage for an object of arbitrary type \lstinline!T! in the example in  \intraref{use-cases-inlinenamespace}{build-modes-and-abi-link-safety}.}
\end{itemize}

\subsection[Further Reading]{Further Reading}\label{further-reading}

\begin{itemize}
\item{\cite{sutter14b} uses inline namespaces as part of a proposal for a portable ABI across compilers.}
\item{\cite{lopez-gomez20} uses inline namespaces as part of a solution to avoid ODR violation in an interpreter.}
\end{itemize}

\subsection[Appendix: Case study of using \lstinline!inline! namespaces for versioning]{Appendix: Case study of using {\SubsecCode inline} namespaces for versioning}\label{appendix:-case-study-of-using-inline-namespaces-for-versioning}

\noindent\textbf{By Niall Douglas}\\[.5ex]

\noindent Let me tell you what I (don't) use them for. It is not a conventional
opinion.

At a previous well-regarded company, they were shipping no less than
forty-three copies of Boost in their application. Boost was not on the
approved libraries list, but the great thing about header-only libraries
is that they don't obviously appear in final binaries, unless you look
for them. So each individual team was including bits of Boost quietly
and without telling their legal department. Why? Because it saved time.
(This was C++98, and \lstinline!boost::shared_ptr! and
\lstinline!boost::function! are both extremely attractive facilities.)

Here's the really interesting part: Most of these copies of Boost were
not the same version. They were varying over a five-year release period.
And, unfortunately, Boost makes no API or ABI guarantees. So,
theoretically, you could get two different incompatible versions of
Boost appearing in the same program binary, and BOOM! there goes memory
corruption.

I advocated to Boost that a simple solution would be for Boost to wrap
up their implementation into an internal inline namespace. That inline
namespace ought to mean something:
\begin{itemize}
\item{\lstinline!lib::v1! is the \emph{stable}, version-1 ABI, which is guaranteed to be compatible with all past and future \lstinline!lib::v1! ABIs, forever, as determined by the ABI-compliance-check tool that runs on \emcppsgloss{CI}. The same goes for \lstinline!v2!, \lstinline!v3!, and so on.}
\item{\lstinline!lib::v2_a7fe42d! is the \emph{unstable}, version-2 ABI, which may be incompatible with any other \lstinline!lib::*!~\lstinline!ABI!; hence, the seven hex chars after the underscore are the git short \emcppsgloss{SHA}, permuted by every commit to the git repository but, in practice, per CMake configure, because nobody wants to rebuild everything per commit. This ensures that no symbols from any revision of \lstinline!lib! will \emph{ever} silently collide or otherwise interfere with any other revision of \lstinline!lib!, when combined into a single binary by a dumb linker.}
\end{itemize}

I have been steadily making progress on getting Boost to avoid putting
anything in the global namespace, so a straightforward find-and-replace
can let you ``fix'' on a particular version of Boost.

That's all the same as the pitch for \lstinline!inline! namespaces. You'll
see the same technique used in \lstinline!libstdc++! and many other major modern C++
codebases.

But I'll tell you now, I don't use \lstinline!inline! namespaces anymore.
Now what I do is use a macro defined to a uniquely named namespace. My
build system uses the git SHA to synthesize namespace macros for my
namespace name, beginning the namespace and ending the namespace.
Finally, in the documentation, I teach people to always use a namespace
alias to a macro to denote the namespace:

\begin{emcppshiddenlisting}[emcppsbatch=e10]
#define OUTCOME_V2_NAMESPACE ov2ns
namespace OUTCOME_V2_NAMESPACE {
}
\end{emcppshiddenlisting}
\begin{emcppslisting}[emcppsbatch=e10]
namespace output = OUTCOME_V2_NAMESPACE;
\end{emcppslisting} 

\noindent That macro expands to something like \lstinline!::outcome_v2_ee9abc2!;
that is, I don't use \lstinline!inline! namespaces anymore.

Why?

Well, for \emph{existing} libraries that don't want to break backward
source compatibility, I think \lstinline!inline! namespaces serve a need.
For \emph{new} libraries, I think a macro-defined namespace is clearer.
\begin{itemize}
\item{It causes users to publicly commit to ``I know what you’re doing here, what it means, and what its consequences are.''}
\item{It declares to \emph{other} users that something unusual (i.e., go read the documentation) is happening here, instead of silent magic behind the scenes.}
\item{It prevents accidents that interfere with ADL and other customization points, which induce surprise, such as accidentally injecting a customization point into \lstinline!lib!, not into \lstinline!lib::v2!.}
\item{Using macros to denote namespace lets us reuse the preprocessor machinery to generate C++ modules using the exact same codebase; C++ modules are used if the compiler supports them, else we fall back to inclusion.}
\end{itemize}

Finally, and here's the real rub, because we now have namespace aliases,
if I were tempted to use an \lstinline!inline! namespace, nowadays I probably would
instead use a uniquely named namespace instead, and, in the \lstinline!include! file,
I'd alias a user-friendly name to that uniquely named namespace. I think
that approach is less likely to induce surprise in the typical
developer's likely use cases than \lstinline!inline! namespaces, such as
injecting customization points into the wrong namespace.

So now I hope you've got a good handle on \lstinline!inline! namespaces: I
was once keen on them, but after some years of experience, I've gone off
them in favor of better-in-my-opinion alternatives. Unfortunately, if your type \lstinline!x::S! has members of type
\lstinline!a::T! and macros decide if that is \lstinline!a::v1::T! or
\lstinline!a::v2::T!, then no linker protects the higher-level types from
ODR bugs, unless you also version \lstinline!x!.




\newpage
%\section[{\tt noexcept} Specifier]{The {\SecCode noexcept} Function Specification}\label{noexcept-specifier}


\emcppsFeature{
    short={\lstinline!noexcept! Specifier},
    tocshort={{\TOCCode noexcept} Specifier},
    long={The {\SecCode noexcept} Function Specification},
    toclong={The \lstinline!noexcept! Function Specification},
    rhshort={{\RHCode noexcept} Specifier},
}{noexcept-specifier}
\setcounter{table}{0}
\setcounter{footnote}{0}
\setcounter{lstlisting}{0}
%\section[{\tt noexcept} Specifier]{The {\SecCode noexcept} Function Specification}\label{noexcept-specifier}

placeholder

%%%%%%% NOTE: Labels will need to be changed since this is now two different sections, not one. 

\newpage
%\section[Ref-Qualifiers]{Reference Qualified Member Functions}\label{refqualifiers}



\emcppsFeature{
    short={Ref-Qualifiers},
    long={Reference-Qualified Member Functions},
}{refqualifiers}
\setcounter{table}{0}
\setcounter{footnote}{0}
\setcounter{lstlisting}{0}
%\section[Ref-Qualifiers]{Reference Qualified Member Functions}\label{refqualifiers}


placeholder


\newpage
%\section[{\tt union} '11]{Unions Having Non-Trivial Members\sectionmark{{\RHCode union}~'11}}\label{unrestricted-unions}\sectionmark{{\RHCode union}~'11}
%%%%% copyedits in and proofed
%% 27 Jan, code checked and replaced as needed by Josh



\emcppsFeature{
    short={\lstinline!union!~'11},
    tocshort={{\TOCCode union}~'11},
    rhshort={{\RHCode union}~'11},
    long={Unions Having Non-Trivial Members},
}{unrestricted-unions}
%\section[{\tt union} '11]{Unions Having Non-Trivial Members\sectionmark{{\RHCode union}~'11}}\label{unrestricted-unions}\sectionmark{{\RHCode union}~'11}


Any nonreference type is permitted to be a member of a \lstinline!union!.

\subsection[Description]{Description}\label{unrestrictedunion-description}

Prior to C++11, only \romeogloss{trivial types} --- e.g.,
\romeogloss{fundamental types}, such as \lstinline!int! and \lstinline!double!,
enumerated or pointer types, or a C-style array or \lstinline!struct!
(a.k.a. a \romeogloss{POD}) --- were allowed to be members of a
\lstinline!union!. This limitation prevented any user-defined type having
a \romeogloss{non-trivial special member function} from being a member of a
\lstinline!union!:

\begin{emcppshiddenlisting}[emcppsbatch=e1]
#include <string>  // (ù{\codeincomments{std::string}}ù)
\end{emcppshiddenlisting}
\begin{emcppslisting}[emcppsbatch=e1]
union U0
{
    int         d_i;  // OK
    std::string d_s;  // compile-time error in C++03 (OK as of C++11)
};
\end{emcppslisting}

\noindent C++11 relaxes such restrictions on \lstinline!union! members, such as
\lstinline!d_s! above, allowing any type other than a \romeogloss{reference
type} to be a member of a \lstinline!union!.

A \lstinline!union! type is permitted to have user-defined special member
functions but --- by design --- does not initialize any of its members
automatically. Any member of a \lstinline!union! having a
\romeogloss{non-trivial constructor}, such as \lstinline!struct!~\lstinline!Nt!
below, must be constructed manually (e.g., via \romeogloss{placement
\lstinline!new!}) before it can be used:

\begin{emcppslisting}[emcppsbatch=e2]
struct Nt  // used as part of a (ù{\codeincomments{union}}ù) (below)
{
    Nt();   // non-trivial default constructor
    ~Nt();  // non-trivial destructor

    // Copy construction and assignment are implicitly defaulted.
    // Move construction and assignment are implicitly deleted.
};
\end{emcppslisting}

\noindent As an added safety measure, any non-trivial \romeogloss{special member
function} defined --- either implicitly or explicitly --- for any
\romeogloss{member} of a \lstinline!union! results in the compiler implicitly
deleting (see %Section~\ref{deleted-functions},
``\titleref{deleted-functions}" on page~\pageref{deleted-functions}) the corresponding \romeogloss{special
member function} of the \lstinline!union! itself:

\begin{emcppslisting}[emcppsbatch=e2]
union U1
{
    int d_i;   // fundamental type having all trivial special member functions
    Nt  d_nt;  // user-defined type having non-trivial special member functions

    // Implicitly deleted special member functions of (ù{\codeincomments{U1}}ù):
    /*
        U1()                     = delete; // due to explicit (ù{\codeincomments{Nt::Nt()}}ù)
        U1(const U1&)            = delete; // due to implicit (ù{\codeincomments{Nt::Nt(const Nt\&)}}ù)
        ~U1()                    = delete; // due to explicit (ù{\codeincomments{Nt::}}ù)~(ù{\codeincomments{Nt()}}ù)
        U1& operator=(const U1&) = delete; // due to implicit
                                           // (ù{\codeincomments{Nt::operator=(const Nt\&)}}ù)
    */
};
\end{emcppslisting}

%\noindent This same sort of precautionary deletion also occurs for any class
%containing such a union as a data member (see {\it\titleref{unrestrictedunion-use-cases}: \titleref{implementing-a-sum-type-as-a-discriminating-(or-tagged)-union}} on page~\pageref{implementing-a-sum-type-as-a-discriminating-(or-tagged)-union}).

A special member function of a \lstinline!union! that is implicitly deleted
can be restored via explicit declaration, thereby forcing a programmer
to consider how non-trivial members should be managed. For example,
we can start providing a \emph{value constructor} and corresponding
\emph{destructor}:

%begin{emcppslisting}
%struct U2
%{
%    union
%    {
%        int  d_i;   // fundamental type (trivial)
%        Nt   d_nt;  // non-trivial user-defined type
%    };
%
%    bool d_useInt;  // discriminator
%
%    U2(bool useInt) : d_useInt(useInt)       // value constructor
%    {
%        if (d_useInt) { new (&d_i) int(); }  // value initialized (to (ù{\codeincomments{0}}ù))
%        else          { new (&d_nt) Nt(); }  // default constructed in place
%    }
%
%    ~U2()  // destructor
%    {
%        if (!d_useInt) { d_nt.~Nt(); }
%    }
%};
%\end{emcppslisting}
\begin{emcppslisting}[emcppsbatch=e2]
#include <new>  // placement (ù{\codeincomments{new}}ù)

struct U2
{
    union
    {
        int  d_i;   // fundamental type (trivial)
        Nt   d_nt;  // non-trivial user-defined type
    };

    bool d_useInt;  // discriminator

    U2(bool useInt) : d_useInt(useInt)
    {
        if (d_useInt) { new (&d_i) int(); }  // value initialized (to (ù{\codeincomments{0}}ù))
        else          { new (&d_nt) Nt(); }  // default constructed in place
    }

    ~U2()  // destructor
    {
        if (!d_useInt) { d_nt.~Nt(); }
    }
};
\end{emcppslisting}


\noindent Notice that we have employed \romeogloss{placement \lstinline!new!} syntax to
control the lifetime of both member objects. Although assignment would
be permitted for the trivial \lstinline!int! type, it would be
\romeogloss{undefined behavior} for the non-trivial \lstinline!Nt! type:

%begin{emcppslisting}
%U2(bool useInt) : d_useInt(useInt)  // value constructor
%{
%    if (d_useInt) { d_i = int(); }  // value initialized (to (ù{\codeincomments{0}}ù))
%    else          { d_nt = Nt(); }  // undefined behavior
%}
%\end{emcppslisting}
\begin{emcppshiddenlisting}[emcppsbatch=e3]
// duplicate the requirements for U2 so that we can show an alternate version
// of the constructor
struct Nt  // used as part of a (ù{\codeincomments{union}}ù) (below)
{
    Nt();   // non-trivial default constructor
    ~Nt();  // non-trivial destructor
};
struct U2
{
    union
    {
        int  d_i;   // fundamental type (trivial)
        Nt   d_nt;  // non-trivial user-defined type
    };

    bool d_useInt;  // discriminator
\end{emcppshiddenlisting}
\begin{emcppslisting}[emcppsbatch=e3]
    U2(bool useInt) : d_useInt(useInt)
    {
        if (d_useInt) { d_i = int(); }  // value initialized (to (ù{\codeincomments{0}}ù))
        else          { d_nt = Nt(); }  // BAD IDEA: undefined behavior (no
                                        // lhs object)
    }
\end{emcppslisting}
\begin{emcppshiddenlisting}[emcppsbatch=e3]
};
\end{emcppshiddenlisting}


\noindent Now if we were to try to copy-construct or assign one object of type
\lstinline!U2! to another, the operation would fail because we have not
yet specifically addressed those \romeogloss{special member functions}:

\begin{emcppslisting}[emcppsbatch=e2]
void f()
{
    U2 a(false), b(true);  // OK (construct both instances of (ù{\codeincomments{U2}}ù))
    U2 c(a);               // Error, no (ù{\codeincomments{U2(const U2\&)}}ù)
    a = b;                 // Error, no (ù{\codeincomments{U2\& operator=(const U2\&)}}ù)
}
\end{emcppslisting}

\noindent We can restore these implicitly deleted special member functions too,
simply by adding appropriate copy-constructor and assignment-operator
definitions for \lstinline!U2! explicitly:

%begin{emcppslisting}
%union U2
%{
%    // ... (everything in (ù{\codeincomments{U2}}ù) above)
%
%    U2(const U2& original) : d_useInt(original.d_useInt)
%    {
%        if (d_useInt) { new (&d_i) int(original.d_i);  }
%        else          { new (&d_nt) Nt(original.d_nt); }
%    }
%
%    U2& operator=(const U2& rhs)
%    {
%        if (this == &rhs) // Prevent self-assignment.
%        {
%            return *this;
%        }
%
%        // Resolve all possible combinations of active types between the
%        // left-hand side and right-hand side of the assignment:
%
%        if (d_useInt)
%        {
%            if (rhs.d_useInt) { d_i = rhs.d_i; }
%            else              { new (&d_nt) Nt(rhs.d_nt); }
%        }
%        else
%        {
%            if (rhs.d_useInt) { d_nt.~Nt(); new (&d_i) int(rhs.d_i); }
%            else              { d_nt = rhs.d_nt; }
%        }
%
%        return *this;
%    }
%};
%\end{emcppslisting}
\begin{emcppshiddenlisting}[emcppsbatch=e4]
#include <new>  // placement (ù{\codeincomments{new}}ù)
// duplicate the requirements for U2 so that we can show an alternate version
// of the constructor
struct Nt  // used as part of a (ù{\codeincomments{union}}ù) (below)
{
    Nt();   // non-trivial default constructor
    ~Nt();  // non-trivial destructor
};
// --- Replace
// ... (everything in (ù{\codeincomments{U2}}ù) above)
    union
    {
        int  d_i;   // fundamental type (trivial)
        Nt   d_nt;  // non-trivial user-defined type
    };

    bool d_useInt;  // discriminator    
// --- End
\end{emcppshiddenlisting}
\begin{emcppslisting}[emcppsbatch=e4]
class U2
{
    // ... (everything in (ù{\codeincomments{U2}}ù) above)

    U2(const U2& original) : d_useInt(original.d_useInt)
    {
        if (d_useInt) { new (&d_i) int(original.d_i);  }
        else          { new (&d_nt) Nt(original.d_nt); }
    }

    U2& operator=(const U2& rhs)
    {
        if (this == &rhs) // Prevent self-assignment.
        {
            return *this;
        }

        // Resolve all possible combinations of active types between the
        // left-hand side and right-hand side of the assignment:

        if (d_useInt)
        {
            if (rhs.d_useInt) { d_i = rhs.d_i; }
            else              { new (&d_nt) Nt(rhs.d_nt); }  // (ù{\codeincomments{int}}ù) DTOR trivial
        }
        else
        {
            if (rhs.d_useInt) { d_nt.~Nt(); new (&d_i) int(rhs.d_i); }
            else              { d_nt = rhs.d_nt; }
        } d_useInt = rhs.d_useInt;

        // Resolve all possible combinations of active types between the
        // left-hand side and right-hand side of the assignment.  Use the
        // corresponding assignment operator when they match; otherwise,
        // if the old member is (ù{\codeincomments{d\_nt}}ù), run its non-trivial destructor, and
        // then copy-construct the new member in place:

        return *this;
    }
};
\end{emcppslisting}
Note that in the code example above, we ignore exceptions for exposition simplicity. Note also that attempting to
restore a \lstinline!union!'s implicitly deleted special member
functions by using the \lstinline!=!~\lstinline!default! syntax (see
%``\titleref{Defaulted-Special-Member-Functions}" on page~\pageref{Defaulted-Special-Member-Functions}
\featureref{\locationa}{Defaulted-Special-Member-Functions}) will still result in their being deleted because
the compiler cannot know which member of the union is active.

\subsection[Use Cases]{Use Cases}\label{unrestrictedunion-use-cases}

\subsubsection[Implementing a \romeogloss{sum type} as a discriminated {\tt union}]{Implementing a sum type as a discriminated {\SubsubsecCode union}}\label{implementing-a-sum-type-as-a-discriminating-(or-tagged)-union}

A \romeogloss{sum type} is an algebraic data type that provides a choice
among a fixed set of specific types. A C++11 unrestricted union can serve as a convenient and efficient way to define storage for a sum type (also called a \emph{tagged} or \emph{discriminated} union) because the alignment and size calculations are performed automatically by the compiler.

As an example, consider writing a parsing function \lstinline!parseInteger!
that, given a\linebreak[4] \lstinline!std::string! \lstinline!input!, will return, as a
\romeogloss{sum type} \lstinline!ParseResult! (see below), containing either an
\lstinline!int! result (on success) or an informative error message
on failure:

\begin{emcppshiddenlisting}[emcppsbatch=e5]
#include <sstream>  // (ù{\codeincomments{std::ostringstream}}ù)
#include <string>   // (ù{\codeincomments{std::string}}ù)
struct ParseResult {
    explicit ParseResult(int values);
    explicit ParseResult(const std::string& error);
};

// --- Replace
    if (/* Failure case (1). */)
    if (true)
// --- End

// --- Replace
    if (/* Failure case (2). */)
    if (true)
// --- End

\end{emcppshiddenlisting}
\begin{emcppslisting}[emcppsbatch=e5]
ParseResult parseInteger(const std::string& input)  // Return a sum type.
{
    int result;     // accumulate (ù{\codeincomments{result}}ù) as we go
    std::size_t i;  // current character index

    // ...

    if (/* Failure case (1). */)
    {
        std::ostringstream oss;
        oss << "Found non-numerical character '" << input[i]
            << "' at index '" << i << "'.";

        return ParseResult(oss.str());
    }

    if (/* Failure case (2). */)
    {
        std::ostringstream oss;
        oss << "Accumulating '" << input[i]
            << "' at index '" << i
            << "' into the current running total '" << result
            << "' would result in integer overflow.";

        return ParseResult(oss.str());
    }

    // ...

    return ParseResult(result);  // Success!
}
\end{emcppslisting}

\noindent The implementation above relies on \lstinline!ParseResult! being able to
hold a value of type either \lstinline!int! or \lstinline!std::string!. By
encapsulating a C++ \lstinline!union! and a \emph{discriminator} as part
of the \lstinline!ParseResult! \romeogloss{sum type}, we can achieve the
desired semantics:

%begin{emcppslisting}
%class ParseResult
%{
%    union  // storage for either the result or the error
%    {
%        int         d_value;  // trivial result type
%        std::string d_error;  // non-trivial error type
%    };
%
%    bool d_isError;  // discriminator
%
%public:
%    explicit ParseResult(int value);                 // value constructor (1)
%    explicit ParseResult(const std::string& error);  // value constructor (2)
%
%    ParseResult(const ParseResult& rhs);             // copy constructor
%    ParseResult& operator=(const ParseResult& rhs);  // copy assignment
%
%    ~ParseResult();                                  // destructor
%};
%\end{emcppslisting}
\begin{emcppshiddenlisting}[emcppsbatch=e6]
#include <string>   // (ù{\codeincomments{std::string}}ù)
\end{emcppshiddenlisting}
\begin{emcppslisting}[emcppsbatch=e6]
class ParseResult
{
    union  // storage for either the result or the error
    {
        int         d_value; // result type (trivial)
        std::string d_error; // error  type (non-trivial)
    };

    bool d_isError;  // discriminator

public:
    explicit ParseResult(int value);                 // value constructor (1)
    explicit ParseResult(const std::string& error);  // value constructor (2)

    ParseResult(const ParseResult& rhs);             // copy constructor
    ParseResult& operator=(const ParseResult& rhs);  // copy assignment

    ~ParseResult();                                  // destructor
};
\end{emcppslisting}


\noindent If a \romeogloss{sum type} comprised more than two types, the discriminator would be an appropriately-sized integral or enumerated type instead of a Boolean.

As discussed in
%{\it\titleref{unrestrictedunion-description}} on page~\pageref{unrestrictedunion-description}
\intrarefsimple{unrestrictedunion-description}, having a non-trivial
type within a \lstinline!union! forces the programmer to provide each
desired special member function and define it manually; note
that the use of placement \lstinline!new! is not required for either of the
two \emph{value constructors} (above) because the initializer syntax
(below) is sufficient to begin the lifetime of even a non-trivial
object:

%begin{emcppslisting}
%ParseResult::ParseResult(double value) : d_value(value), d_isError(false)
%{
%}
%
%ParseResult::ParseResult(const std::string& error)
%    : d_error(error), d_isError(true)
%    // Note that placement (ù{\codeincomments{new}}ù) was not necessary here because a new
%    // (ù{\codeincomments{std::string}}ù) object will be created as part of the initialization of
%    // (ù{\codeincomments{d\_error}}ù).
%{
%}
%\end{emcppslisting}
\begin{emcppslisting}[emcppsbatch=e6]
ParseResult::ParseResult(int value) : d_value(value), d_isError(false)
{
}

ParseResult::ParseResult(const std::string& error)
    : d_error(error), d_isError(true)
    // Note that placement (ù{\codeincomments{new}}ù) was not necessary here because a new
    // (ù{\codeincomments{std::string}}ù) object will be created as part of the initialization of
    // (ù{\codeincomments{d\_error}}ù).
{
}
\end{emcppslisting}


\noindent Placement \lstinline!new! and explicit destructor calls are still, however,
required for destruction and both copy operations{\cprotect\footnote{For
more information on initiating the lifetime of an object, see \cite{iso14}, section 3.8, ``Object Lifetime," pp. 66--69.}}:

%begin{emcppslisting}
%ParseResult::~ParseResult()
%{
%    if(d_isError)
%    {
%        d_error.std::string::~string();
%            // An explicit destructor call is required for (ù{\codeincomments{d\_error}}ù) because its
%            // destructor is non-trivial.
%    }
%}
%
%ParseResult::ParseResult(const ParseResult& rhs) : d_isError(rhs.d_isError)
%{
%    if (d_isError)
%    {
%        new (&d_error) std::string(rhs.d_error);
%            // Placement (ù{\codeincomments{new}}ù) is necessary here to begin the lifetime of a
%            // (ù{\codeincomments{std::string}}ù) object at the address of (ù{\codeincomments{d\_error}}ù).
%    }
%    else
%    {
%        d_value = rhs.d_value;
%            // Placement (ù{\codeincomments{new}}ù) is not necessary here as (ù{\codeincomments{int}}ù) is a trivial type.
%    }
%}
%
%ParseResult& ParseResult::operator=(const ParseResult& rhs)
%{
%    // Destroy (ù{\codeincomments{lhs}}ù)'s error string if existent:
%    if (d_isError) { d_error.std::string::~string(); }
%
%    // Copy (ù{\codeincomments{rhs}}ù)'s object:
%    if (rhs.d_isError) { new (&d_error) std::string(rhs.d_error); }
%    else               { d_value = rhs.d_value; }
%
%    d_isError = rhs.d_isError;
%    return *this;
%}
%\end{emcppslisting}
\begin{emcppshiddenlisting}[emcppsbatch=e6]
// The calls to d_error.std::string::~string() below do not compile on clang
// before clang 11.  Invoking it as ~basic_string instead works, but that
// is a confusing visible change.  Adding this using fixes the problem invisibly.
using std::string;
\end{emcppshiddenlisting}
\begin{emcppslisting}[emcppsbatch=e6]
ParseResult::~ParseResult()
{
    if (d_isError)
    {
        d_error.std::string::~string();
            // An explicit destructor call is required for (ù{\codeincomments{d\_error}}ù) because its
            // destructor is non-trivial.
    }
}

ParseResult::ParseResult(const ParseResult& rhs) : d_isError(rhs.d_isError)
{
    if (d_isError)
    {
        new (&d_error) std::string(rhs.d_error);
            // Placement (ù{\codeincomments{new}}ù) is necessary here to begin the lifetime of a
            // (ù{\codeincomments{std::string}}ù) object at the address of (ù{\codeincomments{d\_error}}ù).
    }
    else
    {
        d_value = rhs.d_value;
            // Placement (ù{\codeincomments{new}}ù) is not necessary here as (ù{\codeincomments{int}}ù) is a trivial type.
    }
}

ParseResult& ParseResult::operator=(const ParseResult& rhs)
{
    if (this == &rhs) // Prevent self-assignment.
    {
        return *this;
    }
    // Destroy (ù{\codeincomments{lhs}}ù)'s error string if existent:
    if (d_isError) { d_error.std::string::~string(); }

    // Copy (ù{\codeincomments{rhs}}ù)'s object:
    if (rhs.d_isError) { new (&d_error) std::string(rhs.d_error); }
    else               { d_value = rhs.d_value; }

    d_isError = rhs.d_isError;
    return *this;
}
\end{emcppslisting}
In practice, \lstinline!ParseResult! would typically use a more general \romeogloss{sum type}{\cprotect\footnote{\lstinline!std::variant!, introduced
in C++17, is the standard construct used to represent a \romeogloss{sum
type} as a \emph{discriminated union}. Prior to C++17,
\lstinline!boost::variant! was the most widely used \emph{tagged} union
  implementation of a \romeogloss{sum type}.}} abstraction to support arbitrary value types and provide proper exception safety.

%\noindent In practice, \texttt{ParseResult} would typically be defined as a
%template and renamed to allow any arbitrary result type \texttt{T} to be
%returned or else implemented in terms of a more general \textbf{sum
%type} abstraction.{\cprotect\footnote{\texttt{std::variant}, introduced
%in C++17, is the standard construct used to represent a \textbf{sum
%type} as a \emph{discriminating union}. Prior to C++17,
%\texttt{boost::variant} was the most widely used \emph{tagged} union
%  implementation of a \textbf{sum type}.}}

\subsection[Potential Pitfalls]{Potential Pitfalls}\label{potential-pitfalls}

\subsubsection[Inadvertent misuse can lead to latent \romeogloss{undefined behavior} at runtime]{Inadvertent misuse can lead to latent \romeogloss{undefined behavior} at runtime}\label{inadvertent-misuse-can-lead-to-latent-undefined-behavior-at-runtime}

When implementing a type that makes use of an unrestricted union,
forgetting to initialize a non-trivial object (using either a
\emph{member initialization list} or \romeogloss{placement \lstinline!new!}) or
accessing a different object than the one that was actually initialized
can result in tacit \romeogloss{undefined behavior}. Although forgetting to
destroy an object does not necessarily result in \romeogloss{undefined
behavior}, failing to do so for any object that manages a resource such
as dynamic memory will result in a \emph{resource leak} and/or lead to
unintended behavior. Note that destroying an
object having a trivial destructor is never necessary; there are, however, rare cases where
we may choose not to destroy an object having a non-trivial
one.
%{\cprotect\footnote{A specific example of where one might
%deliberately choose \emph{not} to destroy an object occurs when a
%collection of related objects are allocated from the same local memory
%resource and then deallocated unilaterally by releasing the memory
%back to the resource. No issue arises if the only resource that is ``leaked''
%by not invoking each individual destructor is the memory allocated
%from that memory resource, and that memory can be
%reused without resulting in \textbf{undefined behavior} if it
%is not subsequently referenced in the context of the deallocated
%  objects.}}

\subsection[Annoyances]{Annoyances}\label{annoyances}

\hspace*{\fill}

\subsection[See Also]{See Also}\label{see-also}

\begin{itemize}
\item{%Section~\ref{deleted-functions}, ``\titleref{deleted-functions}" on page~\pageref{deleted-functions} —
\seealsoref{deleted-functions}{\locationa}Any special member function of a \lstinline!union! that corresponds to a non-trivial one in any of its member elements will be implicitly deleted.} % "or when" changed to "and when" per JD Garcia and Slava
\end{itemize}

\subsection[Further Reading]{Further Reading}\label{further-reading}

\begin{itemize}
\item{\cite{goldthwaite07}}
\item{\cite{ouellet16}}
\end{itemize}





 %%%%%%%%%%%%%%% C++14 starts
\stepcounter{cppxx}
\addtocontents{toc}{\protect\renewcommand*\protect\cftsecindent{1.25em}}
\cftaddnumtitleline{toc}{section}{\thechapter.\thecppxx}{C++14}{\thepage}
% to add to the TOC at the section level
\renewcommand{\cppxx}{C++14}
%%%%%%%%%%%%%%

\newpage
\addtocontents{toc}{\protect\renewcommand*\protect\cftsecindent{4.5em}}
\renewcommand{\cppxx}{C++14}
%\section[{\tt decltypeauto}]{Deducing Types Using {\SecCode decltype} Sematics}\label{decltypeauto}



\emcppsFeature{
    short={\lstinline!decltypeauto!},
    tocshort={\TOCCode decltypeauto},
    long={Deducing Types Using {\SecCode decltype} Semantics},
    toclong={Deducing Types Using \lstinline!decltype! Semantics},
    rhshort={\RHCode decltypeauto},
}{decltypeauto}
\setcounter{table}{0}
\setcounter{footnote}{0}
\setcounter{lstlisting}{0}
%\section[{\tt decltypeauto}]{Deducing Types Using {\SecCode decltype} Semantics}\label{decltypeauto}


placeholder




\newpage
% \section[Deduced Return Type]{Function ({\SecCode auto}) {\SecCode return}-Type Deduction}\label{Function-Return-Type-Deduction}



\emcppsFeature{
    short={Deduced Return Type},
    long={Function ({\SecCode auto}) {\SecCode return}-Type Deduction},
    toclong={Function (\lstinline!auto!) \lstinline!return!-Type Deduction},
}{Function-Return-Type-Deduction}
\setcounter{table}{0}
\setcounter{footnote}{0}
\setcounter{lstlisting}{0}
%\section[Deduced Return Type]{Function ({\SecCode auto}) {\SecCode return}-Type Deduction}\label{Function-Return-Type-Deduction}



placeholder text........
 
 



